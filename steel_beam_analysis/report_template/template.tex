\documentclass[12pt, fleqn]{article}
\usepackage{pgfplots}
\usepackage{bm}
\usepackage{marginnote}
\usepackage{wallpaper}
\usepackage{lastpage}
\usepackage[left=1.3cm,right=2.0cm,top=1.8cm,bottom=5.0cm,marginparwidth=3.4cm]{geometry}
\usepackage{amsmath}
\usepackage{amssymb}
\usepackage{xcolor}
\usepackage{enumitem}
\usepackage{float}
\usepackage{textgreek}
\usepackage{textcomp}
\usepackage{fancyhdr}
\usepackage{graphicx}
\usepackage{pstricks}
\usepackage{subfigure}
\usepackage{caption}
\captionsetup{justification=centering,labelfont=bf, belowskip=12pt,aboveskip=12pt}
\usepackage{textcomp}
\setlength{\headheight}{70pt}
\setlength{\textfloatsep}{12pt}
\setlength{\intextsep}{0pt}
\pagestyle{fancy}\fancyhf{}
\renewcommand{\headrulewidth}{0pt}
\definecolor{darkBlue}{cmyk}{.80, .32, 0, 0}
\setlength{\parindent}{0cm}
\newcommand{\tab}{\hspace*{2em}}
\newcommand\BackgroundStructure{
\setlength{\unitlength}{1mm}
\setlength\fboxsep{0mm}
\setlength\fboxrule{0.5mm}
\put(10, 20pr){\fcolorbox{black}{gray!5}{\framebox(155,247){}}}
\put(165, 20){\fcolorbox{black}{gray!10}{\framebox(37,247){}}}
\put(10, 262){\fcolorbox{black}{white!10}{\framebox(192, 25){}}}
\put(175, 263){\includegraphics{}}}
\setlength{\abovedisplayskip}{0pt}
\setlength{\belowdisplayskip}{0pt}
%	----------------------------------- HEADER -----------------------------------
\fancyhead[L]{\begin{tabular}{l l | l l}
\textbf{Member:} & {\VAR{beam.projectInfo_memberName}} & \textbf{Firm:} & {\VAR{beam.projectInfo_firm}} \\
\textbf{Project:} & {\VAR{beam.projectInfo_project}} & \textbf{Engineer:} & {\VAR{beam.projectInfo_engineer}} \\
\textbf{Level:} & {\VAR{beam.projectInfo_level}} & \textbf{Checker:} & {\VAR{beam.projectInfo_checker}}  \\
\textbf{Date:} & {\VAR{date}} & \textbf{Page:} & \thepage\\
\end{tabular}}
%	---------------------------- APPLIED LOADS SECTION ---------------------------
\begin{document}
\begin{center}
\textbf{\LARGE \VAR{beam.shape} Design Report}
\end{center}
\section{Applied Loading}
\vspace{-30pt}
\begin{figure}[H]
\begin{center}
\input{\VAR{loadDiagramPlotFile}}
\end{center}
\vspace{-18pt}
\caption{Applied Loads}
\end{figure}
The following distributed loads are applied to the beam. The program can handle all possible mass and force units in both metric and imperial systems simultaneously. Loads are plotted to scale according to their relative magnitudes. A "positive" load is defined as a load acting in the direction of gravity.
\begin{table}[ht]
\caption{Applied Distributed Loads}
\centering
\begin{tabular}{l l l l l l l}
\hline
Load & Start Loc. & Start Mag. & End Loc. & End Mag. & Type & Description\\
\hline
\BLOCK{for x in range(0, len(beam.rawDistLoads))}
w\textsubscript{\VAR{x+1}} & \VAR{distLoads[x]['startLoc']} & \VAR{distLoads[x]['startMag']} & \VAR{distLoads[x]['endLoc']} & \VAR{distLoads[x]['endMag']} & \VAR{distLoads[x]['type']} & \VAR{distLoads[x]['desc']}\\
\BLOCK{endfor}
\hline
\end{tabular}
\end{table}
\begin{table}[ht]
\caption{Applied Point Loads}
\centering
\begin{tabular}{l l l l l l}
\hline
Load & Loc. & Shear & Type & Description \\
\hline
\BLOCK{for x in range(0, len(beam.pointLoads))}
\VAR{beam.pointLoads[x]['id']} & \VAR{beam.pointLoads[x]['loc']} & \VAR{beam.pointLoads[x]['shear']} & \VAR{beam.pointLoads[x]['type']} & \VAR{beam.pointLoads[x]['desc']}\\
\BLOCK{endfor}
\hline
\end{tabular}
\end{table}
%	-------------------------------- LOAD COMBOS	--------------------------------
\section{Load Combinations}
The following load combinations are used for the design. Duplicate load combinations are not listed and only loads that are used on the beam are included in the load combinations (i.e. If soil load is not included as a load type in any of the applied loads, then "H" loads will not be included in the listed load combinations). S\textsubscript{DS} is input as \VAR{beam.SDS} and \textOmega\textsubscript{0} is input as \VAR{beam.omega0} for use in seismic load combinations. Any load designated as a pattern load is applied to spans in all possible permutations to create the most extreme loading condition. Numbers after a load indicate the span over which the pattern load is applied (i.e. L0 indicates that live load is applied only on the first span).
\begin{table}[H]
\caption{Strength (LRFD) Load Combinations}
\centering
\begin{tabular}{l l l}
\hline
Load Combo & Loads and Factors & Reference\\
\hline
\BLOCK{for x in range(0, len(beam.strengthCombos))}
LC \VAR{x+1} & \VAR{strengthCombos[x]['lc']} & \VAR{strengthCombos[x]['ref']}\\
\BLOCK{endfor}
\hline
\end{tabular}
\end{table}
\begin{table}[H]
\caption{Deflection (ASD) Load Combinations}
\centering
\begin{tabular}{l l l}
\hline
Load Combo & Loads and Factors & Reference\\
\hline
\BLOCK{for x in range(0, len(beam.deflCombos))}
LC \VAR{x+1} & \VAR{deflCombos[x]['lc']} & \VAR{deflCombos[x]['ref']}\\
\BLOCK{endfor}
\hline
\end{tabular}
\end{table}
%	---------------------- SECTIONAL & MATERIAL PROPERTIES -----------------------
\section{Sectional and Material Properties}
The following are sectional and material properties used for analysis \textbf{(\VAR{beam.shape}, Grade \VAR{beam.grade})}:
\begin{table}[ht]
\caption{Sectional and Material Properties}
\vspace{-10pt}
\centering
\begin{tabular}{lll}
\centering
\begin{tabular}[t]{ll}
\cline{1-2}
Property & Value \\
\cline{1-2}
\BLOCK{for x in range(0, refDesignValsTableBreak1 )}
\VAR{refDesignValsTable[x]} \\
\BLOCK{endfor}
\cline{1-2}
\end{tabular}
&
\begin{tabular}[t]{ll}
\cline{1-2}
Property & Value \\
\cline{1-2}
\BLOCK{for x in range(refDesignValsTableBreak1, refDesignValsTableBreak2 )}
\VAR{refDesignValsTable[x]} \\
\BLOCK{endfor}
\cline{1-2}
\end{tabular}
&
\begin{tabular}[t]{ll}
\cline{1-2}
Property & Value \\
\cline{1-2}
\BLOCK{for x in range(refDesignValsTableBreak2, len(refDesignValsTable))}
\VAR{refDesignValsTable[x]} \\
\BLOCK{endfor}
\cline{1-2}
\end{tabular}
\end{tabular}
\end{table}
%	-------------------------------- BENDING CHECK -------------------------------
\section{Bending Check}
\begin{figure}[H]
\begin{center}
\input{\VAR{momentsPlotFile}}
\end{center}
\caption{Moment Demand Envelope}
\end{figure}
L\textsubscript{p}, the limiting laterally unbraced length for the limit state of yielding, is calculated per \VAR{beam.Lp.ref} as follows:
\begin{flalign*}
\VAR{Lp}
\end{flalign*}
r\textsubscript{ts}, a coefficient used in the calculation of L\textsubscript{r} and C\textsubscript{b}, is calculated per \VAR{beam.rts.ref} as follows:
\begin{flalign*}
\VAR{rts}
\end{flalign*}
L\textsubscript{r}, the limiting unbraced length for the limit state of inelastic lateral-torsional buckling, is calculated per \VAR{beam.Lr.ref} as follows:
\begin{flalign*}
\VAR{Lr}
\end{flalign*}
\textlambda\textsubscript{web}, the web width-to-thickness ratio, is calculated per {\VAR{beam.web.lam.ref}} as follows:
\begin{flalign*}
\VAR{lamWeb}
\end{flalign*}
\textlambda\textsubscript{P-web}, the limiting width-to-thickness ratio for compact/noncompact web, is calculated per {\VAR{beam.web.lamP.ref}} as follows:
\begin{flalign*}
\VAR{lamPweb}
\end{flalign*}
\textlambda\textsubscript{R-web}, the limiting width-to-thickness ratio for noncompact/slender web, is calculated per {\VAR{beam.web.lamR.ref}} as follows:
\begin{flalign*}
\VAR{lamRweb}
\end{flalign*}
\VAR{web}
\\\\
\textlambda\textsubscript{flange}, the flange width-to-thickness ratio, is calculated per {\VAR{beam.flange.lam.ref}} as follows:
\begin{flalign*}
\VAR{lamFlange}
\end{flalign*}
\textlambda\textsubscript{P-flange}, the limiting width-to-thickness ratio for compact/noncompact flange, is calculated per {\VAR{beam.flange.lamP.ref}} as follows:
\begin{flalign*}
\VAR{lamPflange}
\end{flalign*}
\textlambda\textsubscript{R-flange}, the limiting width-to-thickness ratio for noncompact/slender flange, is calculated per {\VAR{beam.flange.lamR.ref}} as follows:
\begin{flalign*}
\VAR{lamRflange}
\end{flalign*}
\VAR{flange}
\\\\
Since \VAR{beam.maxMomentNode.unbracedSpan.phiMn.condition} and the beam's flanges are \textbf{\VAR{flangeCompactness}}, controlling limit state for flexure is \textbf{\VAR{consideredLimitStates}}.
\\\\
M\textsubscript{p}, the plastic bending moment, is calculated per \VAR{beam.Mp.ref} as follows:
\begin{flalign*}
\VAR{Mp}
\end{flalign*}
C\textsubscript{b}, the lateral-torsional buckling modification factor in the critical unbraced span for the critical load combination, is calculated per \VAR{beam.maxMomentNode.unbracedSpan.Cb.ref} as follows:
\\
\begin{flalign*}
\VAR{str(beam.maxMomentNode.unbracedSpan.Cb)}
\end{flalign*}
\\
For brevity, the C\textsubscript{b} calculation is not shown for each span. The following figure illustrates the value of C\textsubscript{b} for each span.
\begin{figure}[H]
\begin{center}
\input{\VAR{CbPlotFile}}
\end{center}
\caption{C\textsubscript{b} Along Member}
\end{figure}
F\textsubscript{cr}, the buckling stress for the critical section in the critical unbraced span, is calculated per \VAR{beam.maxMomentNode.unbracedSpan.Fcr.ref} as follows:
\begin{flalign*}
\VAR{str(beam.maxMomentNode.unbracedSpan.Fcr)}
\end{flalign*}
\\
\textphi\textsubscript{b}, the resistance factor for bending, is determined per \VAR{beam.maxMomentNode.unbracedSpan.phiMn.phiB.ref} as \textbf{\VAR{beam.maxMomentNode.unbracedSpan.phiMn.phiB.val}}.
\\\\
\textphi\textsubscript{b}M\textsubscript{n}, the design flexural strength, is calculated per \VAR{beam.maxMomentNode.unbracedSpan.phiMn.ref} as follows:
\begin{flalign*}
\VAR{str(beam.maxMomentNode.unbracedSpan.phiMn)}
\end{flalign*}
\vspace{-20pt}
{\setlength{\mathindent}{0cm}
\begin{flalign*}
\VAR{bendingCheck}
\end{flalign*}
\textbf{(\VAR{beam.maxMomentNode.unbracedSpan.phiMn.limitState} controls)}
%	-------------------------------- SHEAR CHECK ---------------------------------
\section{Shear Check}
\begin{figure}[H]
\begin{center}
\input{\VAR{shearsPlotFile}}
\end{center}
\caption{Shear Demand Envelope}
\end{figure}
C\textsubscript{v1}, the web shear strength coefficient, is calculated per \VAR{beam.phiVn.Cv1.ref} as follows, based on the ratio of the clear distance between flanges to web thickness:
\begin{flalign*}
\VAR{Cv1}
\end{flalign*}
\textphi\textsubscript{v}, the resistance factor for shear, is calculated per \VAR{beam.phiVn.phiV.ref} as follows:
\begin{flalign*}
\VAR{phiV}
\end{flalign*}
\textphi\textsubscript{v}V\textsubscript{n}, the design shear strength, is calculated per \VAR{beam.phiVn.ref} as follows:
\begin{flalign*}
\VAR{phiVn}
\end{flalign*}
\vspace{-26pt}
{\setlength{\mathindent}{0cm}
\begin{flalign*}
\VAR{shearCheck}
\end{flalign*}
%	----------------------------- DEFLECTION CHECK -------------------------------
\section{Deflection Check}
\begin{figure}[H]
\begin{center}
\input{\VAR{deflsPlotFile}}
\end{center}
\caption{Deflection Envelope}
\end{figure}
\BLOCK{for x in range(0, len(deflChecks))}
\VAR{deflChecks[x]}\\
\bigbreak
\BLOCK{endfor}
\vspace{-30pt}
%	---------------------------------- REACTIONS ---------------------------------
\section{Reactions}
The following is a summary of service-level reactions at each support:
\begin{table}[ht]
\caption{Reactions at Supports}
\centering
\begin{tabular}{\VAR{reactionCols}}
\hline
\VAR{reactionHeaders}
\hline
\VAR{reactions}
\hline
\end{tabular}
\end{table}
\end{document}
