\documentclass[12pt, fleqn]{article}
\usepackage{pgfplots}
\usepackage{bm}
\usepackage{marginnote}
\usepackage{wallpaper}
\usepackage{lastpage}
\usepackage[left=1.3cm,right=2.0cm,top=1.8cm,bottom=5.0cm,marginparwidth=3.4cm]{geometry}
\usepackage{amsmath}
\usepackage{amssymb}
\usepackage{xcolor}
\usepackage{enumitem}
\usepackage{float}
\usepackage{textgreek}
\usepackage{textcomp}
\usepackage{fancyhdr}
\usepackage{graphicx}
\usepackage{pstricks}
\usepackage{subfigure}
\usepackage{caption}
\captionsetup{justification=centering,labelfont=bf, belowskip=12pt,aboveskip=12pt}
\usepackage{textcomp}
\setlength{\headheight}{70pt}
\setlength{\textfloatsep}{12pt}
\setlength{\intextsep}{0pt}
\pagestyle{fancy}\fancyhf{}
\renewcommand{\headrulewidth}{0pt}
\definecolor{darkBlue}{cmyk}{.80, .32, 0, 0}
\setlength{\parindent}{0cm}
\newcommand{\tab}{\hspace*{2em}}
\newcommand\BackgroundStructure{
\setlength{\unitlength}{1mm}
\setlength\fboxsep{0mm}
\setlength\fboxrule{0.5mm}
\put(10, 20pr){\fcolorbox{black}{gray!5}{\framebox(155,247){}}}
\put(165, 20){\fcolorbox{black}{gray!10}{\framebox(37,247){}}}
\put(10, 262){\fcolorbox{black}{white!10}{\framebox(192, 25){}}}
\put(175, 263){\includegraphics{}}}
\setlength{\abovedisplayskip}{0pt}
\setlength{\belowdisplayskip}{0pt}
%	----------------------------------- HEADER -----------------------------------
\fancyhead[L]{\begin{tabular}{l l | l l}
\textbf{Member:} & {Typical Roof Beam} & \textbf{Firm:} & {xxx} \\
\textbf{Project:} & {Ski House} & \textbf{Engineer:} & {xxx} \\
\textbf{Level:} & {Roof} & \textbf{Checker:} & {None}  \\
\textbf{Date:} & {2021-05-23} & \textbf{Page:} & \thepage\\
\end{tabular}}
%	---------------------------- APPLIED LOADS SECTION ---------------------------
\begin{document}
\begin{center}
\textbf{\LARGE W10x112 Design Report}
\end{center}
\section{Applied Loading}
\vspace{-30pt}
\begin{figure}[H]
\begin{center}
%% Creator: Matplotlib, PGF backend
%%
%% To include the figure in your LaTeX document, write
%%   \input{<filename>.pgf}
%%
%% Make sure the required packages are loaded in your preamble
%%   \usepackage{pgf}
%%
%% Figures using additional raster images can only be included by \input if
%% they are in the same directory as the main LaTeX file. For loading figures
%% from other directories you can use the `import` package
%%   \usepackage{import}
%%
%% and then include the figures with
%%   \import{<path to file>}{<filename>.pgf}
%%
%% Matplotlib used the following preamble
%%
\begingroup%
\makeatletter%
\begin{pgfpicture}%
\pgfpathrectangle{\pgfpointorigin}{\pgfqpoint{8.000000in}{3.000000in}}%
\pgfusepath{use as bounding box, clip}%
\begin{pgfscope}%
\pgfpathrectangle{\pgfqpoint{1.000000in}{0.825864in}}{\pgfqpoint{6.200000in}{1.814136in}}%
\pgfusepath{clip}%
\pgfsetbuttcap%
\pgfsetmiterjoin%
\definecolor{currentfill}{rgb}{0.121569,0.466667,0.705882}%
\pgfsetfillcolor{currentfill}%
\pgfsetfillopacity{0.400000}%
\pgfsetlinewidth{1.003750pt}%
\definecolor{currentstroke}{rgb}{0.121569,0.466667,0.705882}%
\pgfsetstrokecolor{currentstroke}%
\pgfsetstrokeopacity{0.400000}%
\pgfsetdash{}{0pt}%
\pgfpathmoveto{\pgfqpoint{1.281818in}{0.908325in}}%
\pgfpathlineto{\pgfqpoint{1.281818in}{0.996240in}}%
\pgfpathlineto{\pgfqpoint{6.918182in}{0.996240in}}%
\pgfpathlineto{\pgfqpoint{6.918182in}{0.908325in}}%
\pgfpathclose%
\pgfusepath{stroke,fill}%
\end{pgfscope}%
\begin{pgfscope}%
\pgfpathrectangle{\pgfqpoint{1.000000in}{0.825864in}}{\pgfqpoint{6.200000in}{1.814136in}}%
\pgfusepath{clip}%
\pgfsetbuttcap%
\pgfsetmiterjoin%
\definecolor{currentfill}{rgb}{1.000000,0.498039,0.054902}%
\pgfsetfillcolor{currentfill}%
\pgfsetfillopacity{0.400000}%
\pgfsetlinewidth{1.003750pt}%
\definecolor{currentstroke}{rgb}{1.000000,0.498039,0.054902}%
\pgfsetstrokecolor{currentstroke}%
\pgfsetstrokeopacity{0.400000}%
\pgfsetdash{}{0pt}%
\pgfpathmoveto{\pgfqpoint{1.281818in}{0.996240in}}%
\pgfpathlineto{\pgfqpoint{1.281818in}{1.084156in}}%
\pgfpathlineto{\pgfqpoint{6.918182in}{1.084156in}}%
\pgfpathlineto{\pgfqpoint{6.918182in}{0.996240in}}%
\pgfpathclose%
\pgfusepath{stroke,fill}%
\end{pgfscope}%
\begin{pgfscope}%
\pgfpathrectangle{\pgfqpoint{1.000000in}{0.825864in}}{\pgfqpoint{6.200000in}{1.814136in}}%
\pgfusepath{clip}%
\pgfsetbuttcap%
\pgfsetmiterjoin%
\definecolor{currentfill}{rgb}{0.172549,0.627451,0.172549}%
\pgfsetfillcolor{currentfill}%
\pgfsetfillopacity{0.400000}%
\pgfsetlinewidth{1.003750pt}%
\definecolor{currentstroke}{rgb}{0.172549,0.627451,0.172549}%
\pgfsetstrokecolor{currentstroke}%
\pgfsetstrokeopacity{0.400000}%
\pgfsetdash{}{0pt}%
\pgfpathmoveto{\pgfqpoint{1.281818in}{1.084156in}}%
\pgfpathlineto{\pgfqpoint{1.281818in}{2.512782in}}%
\pgfpathlineto{\pgfqpoint{6.918182in}{2.512782in}}%
\pgfpathlineto{\pgfqpoint{6.918182in}{1.084156in}}%
\pgfpathclose%
\pgfusepath{stroke,fill}%
\end{pgfscope}%
\begin{pgfscope}%
\pgfpathrectangle{\pgfqpoint{1.000000in}{0.825864in}}{\pgfqpoint{6.200000in}{1.814136in}}%
\pgfusepath{clip}%
\pgfsetbuttcap%
\pgfsetmiterjoin%
\definecolor{currentfill}{rgb}{0.839216,0.152941,0.156863}%
\pgfsetfillcolor{currentfill}%
\pgfsetfillopacity{0.400000}%
\pgfsetlinewidth{1.003750pt}%
\definecolor{currentstroke}{rgb}{0.839216,0.152941,0.156863}%
\pgfsetstrokecolor{currentstroke}%
\pgfsetstrokeopacity{0.400000}%
\pgfsetdash{}{0pt}%
\pgfpathmoveto{\pgfqpoint{1.281818in}{2.512782in}}%
\pgfpathlineto{\pgfqpoint{1.281818in}{2.557539in}}%
\pgfpathlineto{\pgfqpoint{6.918182in}{2.557539in}}%
\pgfpathlineto{\pgfqpoint{6.918182in}{2.512782in}}%
\pgfpathclose%
\pgfusepath{stroke,fill}%
\end{pgfscope}%
\begin{pgfscope}%
\pgfsetbuttcap%
\pgfsetmiterjoin%
\definecolor{currentfill}{rgb}{0.800000,0.800000,0.800000}%
\pgfsetfillcolor{currentfill}%
\pgfsetlinewidth{1.003750pt}%
\definecolor{currentstroke}{rgb}{0.000000,0.000000,0.000000}%
\pgfsetstrokecolor{currentstroke}%
\pgfsetdash{}{0pt}%
\pgfpathmoveto{\pgfqpoint{3.983970in}{0.855832in}}%
\pgfpathcurveto{\pgfqpoint{4.018692in}{0.821110in}}{\pgfqpoint{4.181308in}{0.821110in}}{\pgfqpoint{4.216030in}{0.855832in}}%
\pgfpathcurveto{\pgfqpoint{4.250752in}{0.890554in}}{\pgfqpoint{4.250752in}{1.014011in}}{\pgfqpoint{4.216030in}{1.048733in}}%
\pgfpathcurveto{\pgfqpoint{4.181308in}{1.083455in}}{\pgfqpoint{4.018692in}{1.083455in}}{\pgfqpoint{3.983970in}{1.048733in}}%
\pgfpathcurveto{\pgfqpoint{3.949248in}{1.014011in}}{\pgfqpoint{3.949248in}{0.890554in}}{\pgfqpoint{3.983970in}{0.855832in}}%
\pgfpathclose%
\pgfusepath{stroke,fill}%
\end{pgfscope}%
\begin{pgfscope}%
\definecolor{textcolor}{rgb}{0.000000,0.000000,0.000000}%
\pgfsetstrokecolor{textcolor}%
\pgfsetfillcolor{textcolor}%
\pgftext[x=4.100000in,y=0.952282in,,]{\color{textcolor}\rmfamily\fontsize{10.000000}{12.000000}\selectfont w\textsubscript{1}}%
\end{pgfscope}%
\begin{pgfscope}%
\pgfsetbuttcap%
\pgfsetmiterjoin%
\definecolor{currentfill}{rgb}{0.800000,0.800000,0.800000}%
\pgfsetfillcolor{currentfill}%
\pgfsetlinewidth{1.003750pt}%
\definecolor{currentstroke}{rgb}{0.000000,0.000000,0.000000}%
\pgfsetstrokecolor{currentstroke}%
\pgfsetdash{}{0pt}%
\pgfpathmoveto{\pgfqpoint{3.983970in}{0.943747in}}%
\pgfpathcurveto{\pgfqpoint{4.018692in}{0.909025in}}{\pgfqpoint{4.181308in}{0.909025in}}{\pgfqpoint{4.216030in}{0.943747in}}%
\pgfpathcurveto{\pgfqpoint{4.250752in}{0.978470in}}{\pgfqpoint{4.250752in}{1.101926in}}{\pgfqpoint{4.216030in}{1.136648in}}%
\pgfpathcurveto{\pgfqpoint{4.181308in}{1.171371in}}{\pgfqpoint{4.018692in}{1.171371in}}{\pgfqpoint{3.983970in}{1.136648in}}%
\pgfpathcurveto{\pgfqpoint{3.949248in}{1.101926in}}{\pgfqpoint{3.949248in}{0.978470in}}{\pgfqpoint{3.983970in}{0.943747in}}%
\pgfpathclose%
\pgfusepath{stroke,fill}%
\end{pgfscope}%
\begin{pgfscope}%
\definecolor{textcolor}{rgb}{0.000000,0.000000,0.000000}%
\pgfsetstrokecolor{textcolor}%
\pgfsetfillcolor{textcolor}%
\pgftext[x=4.100000in,y=1.040198in,,]{\color{textcolor}\rmfamily\fontsize{10.000000}{12.000000}\selectfont w\textsubscript{2}}%
\end{pgfscope}%
\begin{pgfscope}%
\pgfsetbuttcap%
\pgfsetmiterjoin%
\definecolor{currentfill}{rgb}{0.800000,0.800000,0.800000}%
\pgfsetfillcolor{currentfill}%
\pgfsetlinewidth{1.003750pt}%
\definecolor{currentstroke}{rgb}{0.000000,0.000000,0.000000}%
\pgfsetstrokecolor{currentstroke}%
\pgfsetdash{}{0pt}%
\pgfpathmoveto{\pgfqpoint{3.983970in}{1.702018in}}%
\pgfpathcurveto{\pgfqpoint{4.018692in}{1.667296in}}{\pgfqpoint{4.181308in}{1.667296in}}{\pgfqpoint{4.216030in}{1.702018in}}%
\pgfpathcurveto{\pgfqpoint{4.250752in}{1.736741in}}{\pgfqpoint{4.250752in}{1.860197in}}{\pgfqpoint{4.216030in}{1.894919in}}%
\pgfpathcurveto{\pgfqpoint{4.181308in}{1.929642in}}{\pgfqpoint{4.018692in}{1.929642in}}{\pgfqpoint{3.983970in}{1.894919in}}%
\pgfpathcurveto{\pgfqpoint{3.949248in}{1.860197in}}{\pgfqpoint{3.949248in}{1.736741in}}{\pgfqpoint{3.983970in}{1.702018in}}%
\pgfpathclose%
\pgfusepath{stroke,fill}%
\end{pgfscope}%
\begin{pgfscope}%
\definecolor{textcolor}{rgb}{0.000000,0.000000,0.000000}%
\pgfsetstrokecolor{textcolor}%
\pgfsetfillcolor{textcolor}%
\pgftext[x=4.100000in,y=1.798469in,,]{\color{textcolor}\rmfamily\fontsize{10.000000}{12.000000}\selectfont w\textsubscript{3}}%
\end{pgfscope}%
\begin{pgfscope}%
\pgfsetbuttcap%
\pgfsetmiterjoin%
\definecolor{currentfill}{rgb}{0.800000,0.800000,0.800000}%
\pgfsetfillcolor{currentfill}%
\pgfsetlinewidth{1.003750pt}%
\definecolor{currentstroke}{rgb}{0.000000,0.000000,0.000000}%
\pgfsetstrokecolor{currentstroke}%
\pgfsetdash{}{0pt}%
\pgfpathmoveto{\pgfqpoint{3.983970in}{2.438710in}}%
\pgfpathcurveto{\pgfqpoint{4.018692in}{2.403988in}}{\pgfqpoint{4.181308in}{2.403988in}}{\pgfqpoint{4.216030in}{2.438710in}}%
\pgfpathcurveto{\pgfqpoint{4.250752in}{2.473432in}}{\pgfqpoint{4.250752in}{2.596889in}}{\pgfqpoint{4.216030in}{2.631611in}}%
\pgfpathcurveto{\pgfqpoint{4.181308in}{2.666334in}}{\pgfqpoint{4.018692in}{2.666334in}}{\pgfqpoint{3.983970in}{2.631611in}}%
\pgfpathcurveto{\pgfqpoint{3.949248in}{2.596889in}}{\pgfqpoint{3.949248in}{2.473432in}}{\pgfqpoint{3.983970in}{2.438710in}}%
\pgfpathclose%
\pgfusepath{stroke,fill}%
\end{pgfscope}%
\begin{pgfscope}%
\definecolor{textcolor}{rgb}{0.000000,0.000000,0.000000}%
\pgfsetstrokecolor{textcolor}%
\pgfsetfillcolor{textcolor}%
\pgftext[x=4.100000in,y=2.535161in,,]{\color{textcolor}\rmfamily\fontsize{10.000000}{12.000000}\selectfont w\textsubscript{4}}%
\end{pgfscope}%
\begin{pgfscope}%
\pgfpathrectangle{\pgfqpoint{1.000000in}{0.330000in}}{\pgfqpoint{6.200000in}{0.604712in}}%
\pgfusepath{clip}%
\pgfsetbuttcap%
\pgfsetroundjoin%
\definecolor{currentfill}{rgb}{1.000000,0.000000,0.000000}%
\pgfsetfillcolor{currentfill}%
\pgfsetlinewidth{1.003750pt}%
\definecolor{currentstroke}{rgb}{1.000000,0.000000,0.000000}%
\pgfsetstrokecolor{currentstroke}%
\pgfsetdash{}{0pt}%
\pgfsys@defobject{currentmarker}{\pgfqpoint{-0.098209in}{-0.098209in}}{\pgfqpoint{0.098209in}{0.098209in}}{%
\pgfpathmoveto{\pgfqpoint{0.000000in}{0.098209in}}%
\pgfpathlineto{\pgfqpoint{-0.098209in}{-0.098209in}}%
\pgfpathlineto{\pgfqpoint{0.098209in}{-0.098209in}}%
\pgfpathclose%
\pgfusepath{stroke,fill}%
}%
\begin{pgfscope}%
\pgfsys@transformshift{2.016996in}{0.357487in}%
\pgfsys@useobject{currentmarker}{}%
\end{pgfscope}%
\begin{pgfscope}%
\pgfsys@transformshift{6.264690in}{0.357487in}%
\pgfsys@useobject{currentmarker}{}%
\end{pgfscope}%
\end{pgfscope}%
\begin{pgfscope}%
\pgfpathrectangle{\pgfqpoint{1.000000in}{0.330000in}}{\pgfqpoint{6.200000in}{0.604712in}}%
\pgfusepath{clip}%
\pgfsetbuttcap%
\pgfsetroundjoin%
\definecolor{currentfill}{rgb}{0.000000,0.000000,1.000000}%
\pgfsetfillcolor{currentfill}%
\pgfsetlinewidth{1.003750pt}%
\definecolor{currentstroke}{rgb}{0.000000,0.000000,1.000000}%
\pgfsetstrokecolor{currentstroke}%
\pgfsetdash{}{0pt}%
\pgfsys@defobject{currentmarker}{\pgfqpoint{-0.098209in}{-0.098209in}}{\pgfqpoint{0.098209in}{0.098209in}}{%
\pgfpathmoveto{\pgfqpoint{-0.098209in}{-0.098209in}}%
\pgfpathlineto{\pgfqpoint{0.098209in}{-0.098209in}}%
\pgfpathlineto{\pgfqpoint{0.098209in}{0.098209in}}%
\pgfpathlineto{\pgfqpoint{-0.098209in}{0.098209in}}%
\pgfpathclose%
\pgfusepath{stroke,fill}%
}%
\end{pgfscope}%
\begin{pgfscope}%
\pgfpathrectangle{\pgfqpoint{1.000000in}{0.330000in}}{\pgfqpoint{6.200000in}{0.604712in}}%
\pgfusepath{clip}%
\pgfsetrectcap%
\pgfsetroundjoin%
\pgfsetlinewidth{1.003750pt}%
\definecolor{currentstroke}{rgb}{0.000000,0.000000,0.000000}%
\pgfsetstrokecolor{currentstroke}%
\pgfsetdash{}{0pt}%
\pgfpathmoveto{\pgfqpoint{1.281818in}{0.907225in}}%
\pgfpathlineto{\pgfqpoint{6.918182in}{0.907225in}}%
\pgfusepath{stroke}%
\end{pgfscope}%
\begin{pgfscope}%
\pgfpathrectangle{\pgfqpoint{1.000000in}{0.330000in}}{\pgfqpoint{6.200000in}{0.604712in}}%
\pgfusepath{clip}%
\pgfsetrectcap%
\pgfsetroundjoin%
\pgfsetlinewidth{1.003750pt}%
\definecolor{currentstroke}{rgb}{0.000000,0.000000,0.000000}%
\pgfsetstrokecolor{currentstroke}%
\pgfsetdash{}{0pt}%
\pgfpathmoveto{\pgfqpoint{1.281818in}{0.870958in}}%
\pgfpathlineto{\pgfqpoint{6.918182in}{0.870958in}}%
\pgfusepath{stroke}%
\end{pgfscope}%
\begin{pgfscope}%
\pgfpathrectangle{\pgfqpoint{1.000000in}{0.330000in}}{\pgfqpoint{6.200000in}{0.604712in}}%
\pgfusepath{clip}%
\pgfsetrectcap%
\pgfsetroundjoin%
\pgfsetlinewidth{1.003750pt}%
\definecolor{currentstroke}{rgb}{0.000000,0.000000,0.000000}%
\pgfsetstrokecolor{currentstroke}%
\pgfsetdash{}{0pt}%
\pgfpathmoveto{\pgfqpoint{1.281818in}{0.472016in}}%
\pgfpathlineto{\pgfqpoint{6.918182in}{0.472016in}}%
\pgfusepath{stroke}%
\end{pgfscope}%
\begin{pgfscope}%
\pgfpathrectangle{\pgfqpoint{1.000000in}{0.330000in}}{\pgfqpoint{6.200000in}{0.604712in}}%
\pgfusepath{clip}%
\pgfsetrectcap%
\pgfsetroundjoin%
\pgfsetlinewidth{1.003750pt}%
\definecolor{currentstroke}{rgb}{0.000000,0.000000,0.000000}%
\pgfsetstrokecolor{currentstroke}%
\pgfsetdash{}{0pt}%
\pgfpathmoveto{\pgfqpoint{1.281818in}{0.508283in}}%
\pgfpathlineto{\pgfqpoint{6.918182in}{0.508283in}}%
\pgfusepath{stroke}%
\end{pgfscope}%
\begin{pgfscope}%
\pgfpathrectangle{\pgfqpoint{1.000000in}{0.330000in}}{\pgfqpoint{6.200000in}{0.604712in}}%
\pgfusepath{clip}%
\pgfsetrectcap%
\pgfsetroundjoin%
\pgfsetlinewidth{1.003750pt}%
\definecolor{currentstroke}{rgb}{0.000000,0.000000,0.000000}%
\pgfsetstrokecolor{currentstroke}%
\pgfsetdash{}{0pt}%
\pgfpathmoveto{\pgfqpoint{1.281818in}{0.472016in}}%
\pgfpathlineto{\pgfqpoint{1.281818in}{0.907225in}}%
\pgfusepath{stroke}%
\end{pgfscope}%
\begin{pgfscope}%
\pgfpathrectangle{\pgfqpoint{1.000000in}{0.330000in}}{\pgfqpoint{6.200000in}{0.604712in}}%
\pgfusepath{clip}%
\pgfsetrectcap%
\pgfsetroundjoin%
\pgfsetlinewidth{1.003750pt}%
\definecolor{currentstroke}{rgb}{0.000000,0.000000,0.000000}%
\pgfsetstrokecolor{currentstroke}%
\pgfsetdash{}{0pt}%
\pgfpathmoveto{\pgfqpoint{6.918182in}{0.472016in}}%
\pgfpathlineto{\pgfqpoint{6.918182in}{0.907225in}}%
\pgfusepath{stroke}%
\end{pgfscope}%
\begin{pgfscope}%
\pgfsetbuttcap%
\pgfsetmiterjoin%
\definecolor{currentfill}{rgb}{0.800000,0.800000,0.800000}%
\pgfsetfillcolor{currentfill}%
\pgfsetlinewidth{1.003750pt}%
\definecolor{currentstroke}{rgb}{0.000000,0.000000,0.000000}%
\pgfsetstrokecolor{currentstroke}%
\pgfsetdash{}{0pt}%
\pgfpathmoveto{\pgfqpoint{3.783641in}{0.593170in}}%
\pgfpathcurveto{\pgfqpoint{3.818363in}{0.558448in}}{\pgfqpoint{4.381637in}{0.558448in}}{\pgfqpoint{4.416359in}{0.593170in}}%
\pgfpathcurveto{\pgfqpoint{4.451081in}{0.627892in}}{\pgfqpoint{4.451081in}{0.751349in}}{\pgfqpoint{4.416359in}{0.786071in}}%
\pgfpathcurveto{\pgfqpoint{4.381637in}{0.820793in}}{\pgfqpoint{3.818363in}{0.820793in}}{\pgfqpoint{3.783641in}{0.786071in}}%
\pgfpathcurveto{\pgfqpoint{3.748919in}{0.751349in}}{\pgfqpoint{3.748919in}{0.627892in}}{\pgfqpoint{3.783641in}{0.593170in}}%
\pgfpathclose%
\pgfusepath{stroke,fill}%
\end{pgfscope}%
\begin{pgfscope}%
\definecolor{textcolor}{rgb}{0.000000,0.000000,0.000000}%
\pgfsetstrokecolor{textcolor}%
\pgfsetfillcolor{textcolor}%
\pgftext[x=4.100000in,y=0.689620in,,]{\color{textcolor}\rmfamily\fontsize{10.000000}{12.000000}\selectfont W10x112}%
\end{pgfscope}%
\begin{pgfscope}%
\pgfsetbuttcap%
\pgfsetmiterjoin%
\definecolor{currentfill}{rgb}{0.800000,0.800000,0.800000}%
\pgfsetfillcolor{currentfill}%
\pgfsetlinewidth{1.003750pt}%
\definecolor{currentstroke}{rgb}{0.000000,0.000000,0.000000}%
\pgfsetstrokecolor{currentstroke}%
\pgfsetdash{}{0pt}%
\pgfpathmoveto{\pgfqpoint{1.145934in}{0.184684in}}%
\pgfpathcurveto{\pgfqpoint{1.180656in}{0.149962in}}{\pgfqpoint{2.118158in}{0.149962in}}{\pgfqpoint{2.152880in}{0.184684in}}%
\pgfpathcurveto{\pgfqpoint{2.187603in}{0.219406in}}{\pgfqpoint{2.187603in}{0.342863in}}{\pgfqpoint{2.152880in}{0.377585in}}%
\pgfpathcurveto{\pgfqpoint{2.118158in}{0.412307in}}{\pgfqpoint{1.180656in}{0.412307in}}{\pgfqpoint{1.145934in}{0.377585in}}%
\pgfpathcurveto{\pgfqpoint{1.111212in}{0.342863in}}{\pgfqpoint{1.111212in}{0.219406in}}{\pgfqpoint{1.145934in}{0.184684in}}%
\pgfpathclose%
\pgfusepath{stroke,fill}%
\end{pgfscope}%
\begin{pgfscope}%
\definecolor{textcolor}{rgb}{0.000000,0.000000,0.000000}%
\pgfsetstrokecolor{textcolor}%
\pgfsetfillcolor{textcolor}%
\pgftext[x=1.649407in,y=0.281134in,,]{\color{textcolor}\rmfamily\fontsize{10.000000}{12.000000}\selectfont Span 0 = 4.5 ft}%
\end{pgfscope}%
\begin{pgfscope}%
\pgfsetbuttcap%
\pgfsetmiterjoin%
\definecolor{currentfill}{rgb}{0.800000,0.800000,0.800000}%
\pgfsetfillcolor{currentfill}%
\pgfsetlinewidth{1.003750pt}%
\definecolor{currentstroke}{rgb}{0.000000,0.000000,0.000000}%
\pgfsetstrokecolor{currentstroke}%
\pgfsetdash{}{0pt}%
\pgfpathmoveto{\pgfqpoint{3.602648in}{0.184684in}}%
\pgfpathcurveto{\pgfqpoint{3.637370in}{0.149962in}}{\pgfqpoint{4.644317in}{0.149962in}}{\pgfqpoint{4.679039in}{0.184684in}}%
\pgfpathcurveto{\pgfqpoint{4.713761in}{0.219406in}}{\pgfqpoint{4.713761in}{0.342863in}}{\pgfqpoint{4.679039in}{0.377585in}}%
\pgfpathcurveto{\pgfqpoint{4.644317in}{0.412307in}}{\pgfqpoint{3.637370in}{0.412307in}}{\pgfqpoint{3.602648in}{0.377585in}}%
\pgfpathcurveto{\pgfqpoint{3.567925in}{0.342863in}}{\pgfqpoint{3.567925in}{0.219406in}}{\pgfqpoint{3.602648in}{0.184684in}}%
\pgfpathclose%
\pgfusepath{stroke,fill}%
\end{pgfscope}%
\begin{pgfscope}%
\definecolor{textcolor}{rgb}{0.000000,0.000000,0.000000}%
\pgfsetstrokecolor{textcolor}%
\pgfsetfillcolor{textcolor}%
\pgftext[x=4.140843in,y=0.281134in,,]{\color{textcolor}\rmfamily\fontsize{10.000000}{12.000000}\selectfont Span 1 = 26.0 ft}%
\end{pgfscope}%
\begin{pgfscope}%
\pgfsetbuttcap%
\pgfsetmiterjoin%
\definecolor{currentfill}{rgb}{0.800000,0.800000,0.800000}%
\pgfsetfillcolor{currentfill}%
\pgfsetlinewidth{1.003750pt}%
\definecolor{currentstroke}{rgb}{0.000000,0.000000,0.000000}%
\pgfsetstrokecolor{currentstroke}%
\pgfsetdash{}{0pt}%
\pgfpathmoveto{\pgfqpoint{6.087963in}{0.184684in}}%
\pgfpathcurveto{\pgfqpoint{6.122685in}{0.149962in}}{\pgfqpoint{7.060187in}{0.149962in}}{\pgfqpoint{7.094909in}{0.184684in}}%
\pgfpathcurveto{\pgfqpoint{7.129632in}{0.219406in}}{\pgfqpoint{7.129632in}{0.342863in}}{\pgfqpoint{7.094909in}{0.377585in}}%
\pgfpathcurveto{\pgfqpoint{7.060187in}{0.412307in}}{\pgfqpoint{6.122685in}{0.412307in}}{\pgfqpoint{6.087963in}{0.377585in}}%
\pgfpathcurveto{\pgfqpoint{6.053241in}{0.342863in}}{\pgfqpoint{6.053241in}{0.219406in}}{\pgfqpoint{6.087963in}{0.184684in}}%
\pgfpathclose%
\pgfusepath{stroke,fill}%
\end{pgfscope}%
\begin{pgfscope}%
\definecolor{textcolor}{rgb}{0.000000,0.000000,0.000000}%
\pgfsetstrokecolor{textcolor}%
\pgfsetfillcolor{textcolor}%
\pgftext[x=6.591436in,y=0.281134in,,]{\color{textcolor}\rmfamily\fontsize{10.000000}{12.000000}\selectfont Span 2 = 4.0 ft}%
\end{pgfscope}%
\end{pgfpicture}%
\makeatother%
\endgroup%

\end{center}
\vspace{-18pt}
\caption{Applied Loads}
\end{figure}
The following distributed loads are applied to the beam. The program can handle all possible mass and force units in both metric and imperial systems simultaneously. Loads are plotted to scale according to their relative magnitudes. A "positive" load is defined as a load acting in the direction of gravity.
\begin{table}[ht]
\caption{Applied Distributed Loads}
\centering
\begin{tabular}{l l l l l l l}
\hline
Load & Start Loc. & Start Mag. & End Loc. & End Mag. & Type & Description\\
\hline
w\textsubscript{1} & 0 {\color{darkBlue}{\textbf{ft}}} & 220 {\color{darkBlue}{\textbf{ft}}} * psf & 34.5 {\color{darkBlue}{\textbf{ft}}} & 220 {\color{darkBlue}{\textbf{ft}}} * psf & D & roof dead\\
w\textsubscript{2} & 0 {\color{darkBlue}{\textbf{ft}}} & 220 {\color{darkBlue}{\textbf{ft}}} * psf & 34.5 {\color{darkBlue}{\textbf{ft}}} & 220 {\color{darkBlue}{\textbf{ft}}} * psf & Lr & roof live\\
w\textsubscript{3} & 0 {\color{darkBlue}{\textbf{ft}}} & 3575 {\color{darkBlue}{\textbf{ft}}} * psf & 34.5 {\color{darkBlue}{\textbf{ft}}} & 3575 {\color{darkBlue}{\textbf{ft}}} * psf & S & roof snow\\
w\textsubscript{4} & 0 {\color{darkBlue}{\textbf{ft}}} & 112.0 {\color{darkBlue}{\textbf{plf}}} & 34.5 {\color{darkBlue}{\textbf{ft}}} & 112.0 {\color{darkBlue}{\textbf{plf}}} & D & Self weight\\
\hline
\end{tabular}
\end{table}
\begin{table}[ht]
\caption{Applied Point Loads}
\centering
\begin{tabular}{l l l l l l}
\hline
Load & Loc. & Shear & Type & Description \\
\hline
\hline
\end{tabular}
\end{table}
%	-------------------------------- LOAD COMBOS	--------------------------------
\section{Load Combinations}
The following load combinations are used for the design. Duplicate load combinations are not listed and only loads that are used on the beam are included in the load combinations (i.e. If soil load is not included as a load type in any of the applied loads, then "H" loads will not be included in the listed load combinations). S\textsubscript{DS} is input as 1.0 and \textOmega\textsubscript{0} is input as 2.5 for use in seismic load combinations. Any load designated as a pattern load is applied to spans in all possible permutations to create the most extreme loading condition. Numbers after a load indicate the span over which the pattern load is applied (i.e. L0 indicates that live load is applied only on the first span).
\begin{table}[H]
\caption{Strength (LRFD) Load Combinations}
\centering
\begin{tabular}{l l l}
\hline
Load Combo & Loads and Factors & Reference\\
\hline
LC 1 & 1.2D + 0.5L\textsubscriptr1 & ASCE7-16 \S2.3.1 (LC 2)\\
LC 2 & 1.2D + 1.6L\textsubscriptr0 + 1.6L\textsubscriptr2 & ASCE7-16 \S2.3.1 (LC 3)\\
LC 3 & 1.2D + 0.5L\textsubscriptr0 + 0.5L\textsubscriptr1 + 0.5L\textsubscriptr2 & ASCE7-16 \S2.3.1 (LC 2)\\
LC 4 & 1.2D + 0.5L\textsubscriptr0 + 0.5L\textsubscriptr1 & ASCE7-16 \S2.3.1 (LC 2)\\
LC 5 & 1.2D + 1.6L\textsubscriptr2 & ASCE7-16 \S2.3.1 (LC 3)\\
LC 6 & 1.4D + 0.2S & ASCE7-16 \S2.3.6 (LC 6)\\
LC 7 & 1.2D + 0.5L\textsubscriptr1 + 0.5L\textsubscriptr2 & ASCE7-16 \S2.3.1 (LC 2)\\
LC 8 & 1.2D + 1.6S & ASCE7-16 \S2.3.1 (LC 3)\\
LC 9 & 1.2D + 0.5L\textsubscriptr2 & ASCE7-16 \S2.3.1 (LC 2)\\
LC 10 & 1.2D + 1.6L\textsubscriptr0 + 1.6L\textsubscriptr1 + 1.6L\textsubscriptr2 & ASCE7-16 \S2.3.1 (LC 3)\\
LC 11 & 1.4D & ASCE7-16 \S2.3.1 (LC 1)\\
LC 12 & 1.2D + 0.5L\textsubscriptr0 & ASCE7-16 \S2.3.1 (LC 2)\\
LC 13 & 0.7D & ASCE7-16 \S2.3.6 (LC 7)\\
LC 14 & 1.2D + 1.6L\textsubscriptr1 & ASCE7-16 \S2.3.1 (LC 3)\\
LC 15 & 1.2D + 1.6L\textsubscriptr0 & ASCE7-16 \S2.3.1 (LC 3)\\
LC 16 & 1.2D + 1.6L\textsubscriptr1 + 1.6L\textsubscriptr2 & ASCE7-16 \S2.3.1 (LC 3)\\
LC 17 & 1.2D + 1.6L\textsubscriptr0 + 1.6L\textsubscriptr1 & ASCE7-16 \S2.3.1 (LC 3)\\
LC 18 & 1.2D + 0.5L\textsubscriptr0 + 0.5L\textsubscriptr2 & ASCE7-16 \S2.3.1 (LC 2)\\
LC 19 & 1.2D & ASCE7-16 \S2.3.1 (LC 2)\\
LC 20 & 0.9D & ASCE7-16 \S2.3.1 (LC 5)\\
LC 21 & 1.2D + 0.5S & ASCE7-16 \S2.3.1 (LC 2)\\
\hline
\end{tabular}
\end{table}
\begin{table}[H]
\caption{Deflection (ASD) Load Combinations}
\centering
\begin{tabular}{l l l}
\hline
Load Combo & Loads and Factors & Reference\\
\hline
LC 1 & 1.14D & ASCE7-16 \S2.4.5 (LC 8)\\
LC 2 & 0.6D & ASCE7-16 \S2.4.1 (LC 7)\\
LC 3 & 1.0D + 0.75L\textsubscriptr0 + 0.75L\textsubscriptr2 & ASCE7-16 \S2.4.1 (LC 4)\\
LC 4 & 1.0D + 1.0S & ASCE7-16 \S2.4.1 (LC 3)\\
LC 5 & 1.1D + 0.75S & ASCE7-16 \S2.4.5 (LC 8)\\
LC 6 & 1.0D + 0.75L\textsubscriptr2 & ASCE7-16 \S2.4.1 (LC 4)\\
LC 7 & 1.0D + 0.75S & ASCE7-16 \S2.4.1 (LC 4)\\
LC 8 & 1.0D + 1.0L\textsubscriptr1 + 1.0L\textsubscriptr2 & ASCE7-16 \S2.4.1 (LC 3)\\
LC 9 & 1.0D & ASCE7-16 \S2.4.1 (LC 1)\\
LC 10 & 1.0D + 1.0L\textsubscriptr2 & ASCE7-16 \S2.4.1 (LC 3)\\
LC 11 & 1.0D + 1.0L\textsubscriptr0 + 1.0L\textsubscriptr1 & ASCE7-16 \S2.4.1 (LC 3)\\
LC 12 & 1.0D + 0.75L\textsubscriptr0 + 0.75L\textsubscriptr1 + 0.75L\textsubscriptr2 & ASCE7-16 \S2.4.1 (LC 4)\\
LC 13 & 1.0D + 1.0L\textsubscriptr0 + 1.0L\textsubscriptr1 + 1.0L\textsubscriptr2 & ASCE7-16 \S2.4.1 (LC 3)\\
LC 14 & 0.4D & ASCE7-16 \S2.4.5 (LC 9)\\
LC 15 & 1.0D + 0.75L\textsubscriptr0 & ASCE7-16 \S2.4.1 (LC 4)\\
LC 16 & 1.0D + 0.75L\textsubscriptr1 & ASCE7-16 \S2.4.1 (LC 4)\\
LC 17 & 1.0D + 0.75L\textsubscriptr1 + 0.75L\textsubscriptr2 & ASCE7-16 \S2.4.1 (LC 4)\\
LC 18 & 1.0D + 1.0L\textsubscriptr0 + 1.0L\textsubscriptr2 & ASCE7-16 \S2.4.1 (LC 3)\\
LC 19 & 1.0D + 0.75L\textsubscriptr0 + 0.75L\textsubscriptr1 & ASCE7-16 \S2.4.1 (LC 4)\\
LC 20 & 1.0D + 1.0L\textsubscriptr0 & ASCE7-16 \S2.4.1 (LC 3)\\
LC 21 & 1.0D + 1.0L\textsubscriptr1 & ASCE7-16 \S2.4.1 (LC 3)\\
\hline
\end{tabular}
\end{table}
%	---------------------- SECTIONAL & MATERIAL PROPERTIES -----------------------
\section{Sectional and Material Properties}
The following are sectional and material properties used for analysis \textbf{(W10x112, Grade A992)}:
\begin{table}[ht]
\caption{Sectional and Material Properties}
\vspace{-10pt}
\centering
\begin{tabular}{lll}
\centering
\begin{tabular}[t]{ll}
\cline{1-2}
Property & Value \\
\cline{1-2}
A\textsubscript{w} & 8.6 {\color{darkBlue}{\textbf{{\color{darkBlue}{\textbf{in}}}\textsuperscript{2}}}} \\
C\textsubscript{w} & 6020 {\color{darkBlue}{\textbf{{\color{darkBlue}{\textbf{in}}}\textsuperscript{6}}}} \\
F\textsubscript{u} & 65 {\color{darkBlue}{\textbf{ksi}}} \\
F\textsubscript{y} & 50 {\color{darkBlue}{\textbf{ksi}}} \\
I\textsubscript{x} & 716 {\color{darkBlue}{\textbf{{\color{darkBlue}{\textbf{in}}}\textsuperscript{4}}}} \\
I\textsubscript{y} & 236 {\color{darkBlue}{\textbf{{\color{darkBlue}{\textbf{in}}}\textsuperscript{4}}}} \\
\cline{1-2}
\end{tabular}
&
\begin{tabular}[t]{ll}
\cline{1-2}
Property & Value \\
\cline{1-2}
S\textsubscript{x} & 126 {\color{darkBlue}{\textbf{{\color{darkBlue}{\textbf{in}}}\textsuperscript{3}}}} \\
S\textsubscript{y} & 45.3 {\color{darkBlue}{\textbf{{\color{darkBlue}{\textbf{in}}}\textsuperscript{3}}}} \\
Z\textsubscript{x} & 147 {\color{darkBlue}{\textbf{{\color{darkBlue}{\textbf{in}}}\textsuperscript{3}}}} \\
Z\textsubscript{y} & 69.2 {\color{darkBlue}{\textbf{{\color{darkBlue}{\textbf{in}}}\textsuperscript{3}}}} \\
r\textsubscript{x} & 4.7 {\color{darkBlue}{\textbf{in}}} \\
r\textsubscript{y} & 2.7 {\color{darkBlue}{\textbf{in}}} \\
\cline{1-2}
\end{tabular}
&
\begin{tabular}[t]{ll}
\cline{1-2}
Property & Value \\
\cline{1-2}
b\textsubscript{f} & 10.4 {\color{darkBlue}{\textbf{in}}} \\
t\textsubscript{f} & 1.2 {\color{darkBlue}{\textbf{in}}} \\
t\textsubscript{w} & 0.8 {\color{darkBlue}{\textbf{in}}} \\
h\textsubscript{0} & 10.2 {\color{darkBlue}{\textbf{in}}} \\
U.W. & 490 {\color{darkBlue}{\textbf{pcf}}} \\
\cline{1-2}
\end{tabular}
\end{tabular}
\end{table}
%	-------------------------------- BENDING CHECK -------------------------------
\section{Bending Check}
\begin{figure}[H]
\begin{center}
%% Creator: Matplotlib, PGF backend
%%
%% To include the figure in your LaTeX document, write
%%   \input{<filename>.pgf}
%%
%% Make sure the required packages are loaded in your preamble
%%   \usepackage{pgf}
%%
%% Figures using additional raster images can only be included by \input if
%% they are in the same directory as the main LaTeX file. For loading figures
%% from other directories you can use the `import` package
%%   \usepackage{import}
%%
%% and then include the figures with
%%   \import{<path to file>}{<filename>.pgf}
%%
%% Matplotlib used the following preamble
%%
\begingroup%
\makeatletter%
\begin{pgfpicture}%
\pgfpathrectangle{\pgfpointorigin}{\pgfqpoint{8.000000in}{3.000000in}}%
\pgfusepath{use as bounding box, clip}%
\begin{pgfscope}%
\pgfsetbuttcap%
\pgfsetmiterjoin%
\definecolor{currentfill}{rgb}{1.000000,1.000000,1.000000}%
\pgfsetfillcolor{currentfill}%
\pgfsetlinewidth{0.000000pt}%
\definecolor{currentstroke}{rgb}{1.000000,1.000000,1.000000}%
\pgfsetstrokecolor{currentstroke}%
\pgfsetdash{}{0pt}%
\pgfpathmoveto{\pgfqpoint{0.000000in}{0.000000in}}%
\pgfpathlineto{\pgfqpoint{8.000000in}{0.000000in}}%
\pgfpathlineto{\pgfqpoint{8.000000in}{3.000000in}}%
\pgfpathlineto{\pgfqpoint{0.000000in}{3.000000in}}%
\pgfpathclose%
\pgfusepath{fill}%
\end{pgfscope}%
\begin{pgfscope}%
\pgfsetbuttcap%
\pgfsetmiterjoin%
\definecolor{currentfill}{rgb}{1.000000,1.000000,1.000000}%
\pgfsetfillcolor{currentfill}%
\pgfsetlinewidth{0.000000pt}%
\definecolor{currentstroke}{rgb}{0.000000,0.000000,0.000000}%
\pgfsetstrokecolor{currentstroke}%
\pgfsetstrokeopacity{0.000000}%
\pgfsetdash{}{0pt}%
\pgfpathmoveto{\pgfqpoint{1.000000in}{0.330000in}}%
\pgfpathlineto{\pgfqpoint{7.200000in}{0.330000in}}%
\pgfpathlineto{\pgfqpoint{7.200000in}{2.640000in}}%
\pgfpathlineto{\pgfqpoint{1.000000in}{2.640000in}}%
\pgfpathclose%
\pgfusepath{fill}%
\end{pgfscope}%
\begin{pgfscope}%
\pgfpathrectangle{\pgfqpoint{1.000000in}{0.330000in}}{\pgfqpoint{6.200000in}{2.310000in}}%
\pgfusepath{clip}%
\pgfsetbuttcap%
\pgfsetroundjoin%
\pgfsetlinewidth{0.803000pt}%
\definecolor{currentstroke}{rgb}{0.000000,0.000000,0.000000}%
\pgfsetstrokecolor{currentstroke}%
\pgfsetdash{{0.800000pt}{1.320000pt}}{0.000000pt}%
\pgfpathmoveto{\pgfqpoint{1.281818in}{0.330000in}}%
\pgfpathlineto{\pgfqpoint{1.281818in}{2.640000in}}%
\pgfusepath{stroke}%
\end{pgfscope}%
\begin{pgfscope}%
\pgfsetbuttcap%
\pgfsetroundjoin%
\definecolor{currentfill}{rgb}{0.000000,0.000000,0.000000}%
\pgfsetfillcolor{currentfill}%
\pgfsetlinewidth{0.803000pt}%
\definecolor{currentstroke}{rgb}{0.000000,0.000000,0.000000}%
\pgfsetstrokecolor{currentstroke}%
\pgfsetdash{}{0pt}%
\pgfsys@defobject{currentmarker}{\pgfqpoint{0.000000in}{-0.048611in}}{\pgfqpoint{0.000000in}{0.000000in}}{%
\pgfpathmoveto{\pgfqpoint{0.000000in}{0.000000in}}%
\pgfpathlineto{\pgfqpoint{0.000000in}{-0.048611in}}%
\pgfusepath{stroke,fill}%
}%
\begin{pgfscope}%
\pgfsys@transformshift{1.281818in}{0.330000in}%
\pgfsys@useobject{currentmarker}{}%
\end{pgfscope}%
\end{pgfscope}%
\begin{pgfscope}%
\pgfsetbuttcap%
\pgfsetroundjoin%
\definecolor{currentfill}{rgb}{0.000000,0.000000,0.000000}%
\pgfsetfillcolor{currentfill}%
\pgfsetlinewidth{0.803000pt}%
\definecolor{currentstroke}{rgb}{0.000000,0.000000,0.000000}%
\pgfsetstrokecolor{currentstroke}%
\pgfsetdash{}{0pt}%
\pgfsys@defobject{currentmarker}{\pgfqpoint{0.000000in}{0.000000in}}{\pgfqpoint{0.000000in}{0.048611in}}{%
\pgfpathmoveto{\pgfqpoint{0.000000in}{0.000000in}}%
\pgfpathlineto{\pgfqpoint{0.000000in}{0.048611in}}%
\pgfusepath{stroke,fill}%
}%
\begin{pgfscope}%
\pgfsys@transformshift{1.281818in}{2.640000in}%
\pgfsys@useobject{currentmarker}{}%
\end{pgfscope}%
\end{pgfscope}%
\begin{pgfscope}%
\definecolor{textcolor}{rgb}{0.000000,0.000000,0.000000}%
\pgfsetstrokecolor{textcolor}%
\pgfsetfillcolor{textcolor}%
\pgftext[x=1.281818in,y=0.232778in,,top]{\color{textcolor}\rmfamily\fontsize{10.000000}{12.000000}\selectfont \(\displaystyle {0}\)}%
\end{pgfscope}%
\begin{pgfscope}%
\pgfpathrectangle{\pgfqpoint{1.000000in}{0.330000in}}{\pgfqpoint{6.200000in}{2.310000in}}%
\pgfusepath{clip}%
\pgfsetbuttcap%
\pgfsetroundjoin%
\pgfsetlinewidth{0.803000pt}%
\definecolor{currentstroke}{rgb}{0.000000,0.000000,0.000000}%
\pgfsetstrokecolor{currentstroke}%
\pgfsetdash{{0.800000pt}{1.320000pt}}{0.000000pt}%
\pgfpathmoveto{\pgfqpoint{2.098682in}{0.330000in}}%
\pgfpathlineto{\pgfqpoint{2.098682in}{2.640000in}}%
\pgfusepath{stroke}%
\end{pgfscope}%
\begin{pgfscope}%
\pgfsetbuttcap%
\pgfsetroundjoin%
\definecolor{currentfill}{rgb}{0.000000,0.000000,0.000000}%
\pgfsetfillcolor{currentfill}%
\pgfsetlinewidth{0.803000pt}%
\definecolor{currentstroke}{rgb}{0.000000,0.000000,0.000000}%
\pgfsetstrokecolor{currentstroke}%
\pgfsetdash{}{0pt}%
\pgfsys@defobject{currentmarker}{\pgfqpoint{0.000000in}{-0.048611in}}{\pgfqpoint{0.000000in}{0.000000in}}{%
\pgfpathmoveto{\pgfqpoint{0.000000in}{0.000000in}}%
\pgfpathlineto{\pgfqpoint{0.000000in}{-0.048611in}}%
\pgfusepath{stroke,fill}%
}%
\begin{pgfscope}%
\pgfsys@transformshift{2.098682in}{0.330000in}%
\pgfsys@useobject{currentmarker}{}%
\end{pgfscope}%
\end{pgfscope}%
\begin{pgfscope}%
\pgfsetbuttcap%
\pgfsetroundjoin%
\definecolor{currentfill}{rgb}{0.000000,0.000000,0.000000}%
\pgfsetfillcolor{currentfill}%
\pgfsetlinewidth{0.803000pt}%
\definecolor{currentstroke}{rgb}{0.000000,0.000000,0.000000}%
\pgfsetstrokecolor{currentstroke}%
\pgfsetdash{}{0pt}%
\pgfsys@defobject{currentmarker}{\pgfqpoint{0.000000in}{0.000000in}}{\pgfqpoint{0.000000in}{0.048611in}}{%
\pgfpathmoveto{\pgfqpoint{0.000000in}{0.000000in}}%
\pgfpathlineto{\pgfqpoint{0.000000in}{0.048611in}}%
\pgfusepath{stroke,fill}%
}%
\begin{pgfscope}%
\pgfsys@transformshift{2.098682in}{2.640000in}%
\pgfsys@useobject{currentmarker}{}%
\end{pgfscope}%
\end{pgfscope}%
\begin{pgfscope}%
\definecolor{textcolor}{rgb}{0.000000,0.000000,0.000000}%
\pgfsetstrokecolor{textcolor}%
\pgfsetfillcolor{textcolor}%
\pgftext[x=2.098682in,y=0.232778in,,top]{\color{textcolor}\rmfamily\fontsize{10.000000}{12.000000}\selectfont \(\displaystyle {5}\)}%
\end{pgfscope}%
\begin{pgfscope}%
\pgfpathrectangle{\pgfqpoint{1.000000in}{0.330000in}}{\pgfqpoint{6.200000in}{2.310000in}}%
\pgfusepath{clip}%
\pgfsetbuttcap%
\pgfsetroundjoin%
\pgfsetlinewidth{0.803000pt}%
\definecolor{currentstroke}{rgb}{0.000000,0.000000,0.000000}%
\pgfsetstrokecolor{currentstroke}%
\pgfsetdash{{0.800000pt}{1.320000pt}}{0.000000pt}%
\pgfpathmoveto{\pgfqpoint{2.915547in}{0.330000in}}%
\pgfpathlineto{\pgfqpoint{2.915547in}{2.640000in}}%
\pgfusepath{stroke}%
\end{pgfscope}%
\begin{pgfscope}%
\pgfsetbuttcap%
\pgfsetroundjoin%
\definecolor{currentfill}{rgb}{0.000000,0.000000,0.000000}%
\pgfsetfillcolor{currentfill}%
\pgfsetlinewidth{0.803000pt}%
\definecolor{currentstroke}{rgb}{0.000000,0.000000,0.000000}%
\pgfsetstrokecolor{currentstroke}%
\pgfsetdash{}{0pt}%
\pgfsys@defobject{currentmarker}{\pgfqpoint{0.000000in}{-0.048611in}}{\pgfqpoint{0.000000in}{0.000000in}}{%
\pgfpathmoveto{\pgfqpoint{0.000000in}{0.000000in}}%
\pgfpathlineto{\pgfqpoint{0.000000in}{-0.048611in}}%
\pgfusepath{stroke,fill}%
}%
\begin{pgfscope}%
\pgfsys@transformshift{2.915547in}{0.330000in}%
\pgfsys@useobject{currentmarker}{}%
\end{pgfscope}%
\end{pgfscope}%
\begin{pgfscope}%
\pgfsetbuttcap%
\pgfsetroundjoin%
\definecolor{currentfill}{rgb}{0.000000,0.000000,0.000000}%
\pgfsetfillcolor{currentfill}%
\pgfsetlinewidth{0.803000pt}%
\definecolor{currentstroke}{rgb}{0.000000,0.000000,0.000000}%
\pgfsetstrokecolor{currentstroke}%
\pgfsetdash{}{0pt}%
\pgfsys@defobject{currentmarker}{\pgfqpoint{0.000000in}{0.000000in}}{\pgfqpoint{0.000000in}{0.048611in}}{%
\pgfpathmoveto{\pgfqpoint{0.000000in}{0.000000in}}%
\pgfpathlineto{\pgfqpoint{0.000000in}{0.048611in}}%
\pgfusepath{stroke,fill}%
}%
\begin{pgfscope}%
\pgfsys@transformshift{2.915547in}{2.640000in}%
\pgfsys@useobject{currentmarker}{}%
\end{pgfscope}%
\end{pgfscope}%
\begin{pgfscope}%
\definecolor{textcolor}{rgb}{0.000000,0.000000,0.000000}%
\pgfsetstrokecolor{textcolor}%
\pgfsetfillcolor{textcolor}%
\pgftext[x=2.915547in,y=0.232778in,,top]{\color{textcolor}\rmfamily\fontsize{10.000000}{12.000000}\selectfont \(\displaystyle {10}\)}%
\end{pgfscope}%
\begin{pgfscope}%
\pgfpathrectangle{\pgfqpoint{1.000000in}{0.330000in}}{\pgfqpoint{6.200000in}{2.310000in}}%
\pgfusepath{clip}%
\pgfsetbuttcap%
\pgfsetroundjoin%
\pgfsetlinewidth{0.803000pt}%
\definecolor{currentstroke}{rgb}{0.000000,0.000000,0.000000}%
\pgfsetstrokecolor{currentstroke}%
\pgfsetdash{{0.800000pt}{1.320000pt}}{0.000000pt}%
\pgfpathmoveto{\pgfqpoint{3.732411in}{0.330000in}}%
\pgfpathlineto{\pgfqpoint{3.732411in}{2.640000in}}%
\pgfusepath{stroke}%
\end{pgfscope}%
\begin{pgfscope}%
\pgfsetbuttcap%
\pgfsetroundjoin%
\definecolor{currentfill}{rgb}{0.000000,0.000000,0.000000}%
\pgfsetfillcolor{currentfill}%
\pgfsetlinewidth{0.803000pt}%
\definecolor{currentstroke}{rgb}{0.000000,0.000000,0.000000}%
\pgfsetstrokecolor{currentstroke}%
\pgfsetdash{}{0pt}%
\pgfsys@defobject{currentmarker}{\pgfqpoint{0.000000in}{-0.048611in}}{\pgfqpoint{0.000000in}{0.000000in}}{%
\pgfpathmoveto{\pgfqpoint{0.000000in}{0.000000in}}%
\pgfpathlineto{\pgfqpoint{0.000000in}{-0.048611in}}%
\pgfusepath{stroke,fill}%
}%
\begin{pgfscope}%
\pgfsys@transformshift{3.732411in}{0.330000in}%
\pgfsys@useobject{currentmarker}{}%
\end{pgfscope}%
\end{pgfscope}%
\begin{pgfscope}%
\pgfsetbuttcap%
\pgfsetroundjoin%
\definecolor{currentfill}{rgb}{0.000000,0.000000,0.000000}%
\pgfsetfillcolor{currentfill}%
\pgfsetlinewidth{0.803000pt}%
\definecolor{currentstroke}{rgb}{0.000000,0.000000,0.000000}%
\pgfsetstrokecolor{currentstroke}%
\pgfsetdash{}{0pt}%
\pgfsys@defobject{currentmarker}{\pgfqpoint{0.000000in}{0.000000in}}{\pgfqpoint{0.000000in}{0.048611in}}{%
\pgfpathmoveto{\pgfqpoint{0.000000in}{0.000000in}}%
\pgfpathlineto{\pgfqpoint{0.000000in}{0.048611in}}%
\pgfusepath{stroke,fill}%
}%
\begin{pgfscope}%
\pgfsys@transformshift{3.732411in}{2.640000in}%
\pgfsys@useobject{currentmarker}{}%
\end{pgfscope}%
\end{pgfscope}%
\begin{pgfscope}%
\definecolor{textcolor}{rgb}{0.000000,0.000000,0.000000}%
\pgfsetstrokecolor{textcolor}%
\pgfsetfillcolor{textcolor}%
\pgftext[x=3.732411in,y=0.232778in,,top]{\color{textcolor}\rmfamily\fontsize{10.000000}{12.000000}\selectfont \(\displaystyle {15}\)}%
\end{pgfscope}%
\begin{pgfscope}%
\pgfpathrectangle{\pgfqpoint{1.000000in}{0.330000in}}{\pgfqpoint{6.200000in}{2.310000in}}%
\pgfusepath{clip}%
\pgfsetbuttcap%
\pgfsetroundjoin%
\pgfsetlinewidth{0.803000pt}%
\definecolor{currentstroke}{rgb}{0.000000,0.000000,0.000000}%
\pgfsetstrokecolor{currentstroke}%
\pgfsetdash{{0.800000pt}{1.320000pt}}{0.000000pt}%
\pgfpathmoveto{\pgfqpoint{4.549275in}{0.330000in}}%
\pgfpathlineto{\pgfqpoint{4.549275in}{2.640000in}}%
\pgfusepath{stroke}%
\end{pgfscope}%
\begin{pgfscope}%
\pgfsetbuttcap%
\pgfsetroundjoin%
\definecolor{currentfill}{rgb}{0.000000,0.000000,0.000000}%
\pgfsetfillcolor{currentfill}%
\pgfsetlinewidth{0.803000pt}%
\definecolor{currentstroke}{rgb}{0.000000,0.000000,0.000000}%
\pgfsetstrokecolor{currentstroke}%
\pgfsetdash{}{0pt}%
\pgfsys@defobject{currentmarker}{\pgfqpoint{0.000000in}{-0.048611in}}{\pgfqpoint{0.000000in}{0.000000in}}{%
\pgfpathmoveto{\pgfqpoint{0.000000in}{0.000000in}}%
\pgfpathlineto{\pgfqpoint{0.000000in}{-0.048611in}}%
\pgfusepath{stroke,fill}%
}%
\begin{pgfscope}%
\pgfsys@transformshift{4.549275in}{0.330000in}%
\pgfsys@useobject{currentmarker}{}%
\end{pgfscope}%
\end{pgfscope}%
\begin{pgfscope}%
\pgfsetbuttcap%
\pgfsetroundjoin%
\definecolor{currentfill}{rgb}{0.000000,0.000000,0.000000}%
\pgfsetfillcolor{currentfill}%
\pgfsetlinewidth{0.803000pt}%
\definecolor{currentstroke}{rgb}{0.000000,0.000000,0.000000}%
\pgfsetstrokecolor{currentstroke}%
\pgfsetdash{}{0pt}%
\pgfsys@defobject{currentmarker}{\pgfqpoint{0.000000in}{0.000000in}}{\pgfqpoint{0.000000in}{0.048611in}}{%
\pgfpathmoveto{\pgfqpoint{0.000000in}{0.000000in}}%
\pgfpathlineto{\pgfqpoint{0.000000in}{0.048611in}}%
\pgfusepath{stroke,fill}%
}%
\begin{pgfscope}%
\pgfsys@transformshift{4.549275in}{2.640000in}%
\pgfsys@useobject{currentmarker}{}%
\end{pgfscope}%
\end{pgfscope}%
\begin{pgfscope}%
\definecolor{textcolor}{rgb}{0.000000,0.000000,0.000000}%
\pgfsetstrokecolor{textcolor}%
\pgfsetfillcolor{textcolor}%
\pgftext[x=4.549275in,y=0.232778in,,top]{\color{textcolor}\rmfamily\fontsize{10.000000}{12.000000}\selectfont \(\displaystyle {20}\)}%
\end{pgfscope}%
\begin{pgfscope}%
\pgfpathrectangle{\pgfqpoint{1.000000in}{0.330000in}}{\pgfqpoint{6.200000in}{2.310000in}}%
\pgfusepath{clip}%
\pgfsetbuttcap%
\pgfsetroundjoin%
\pgfsetlinewidth{0.803000pt}%
\definecolor{currentstroke}{rgb}{0.000000,0.000000,0.000000}%
\pgfsetstrokecolor{currentstroke}%
\pgfsetdash{{0.800000pt}{1.320000pt}}{0.000000pt}%
\pgfpathmoveto{\pgfqpoint{5.366140in}{0.330000in}}%
\pgfpathlineto{\pgfqpoint{5.366140in}{2.640000in}}%
\pgfusepath{stroke}%
\end{pgfscope}%
\begin{pgfscope}%
\pgfsetbuttcap%
\pgfsetroundjoin%
\definecolor{currentfill}{rgb}{0.000000,0.000000,0.000000}%
\pgfsetfillcolor{currentfill}%
\pgfsetlinewidth{0.803000pt}%
\definecolor{currentstroke}{rgb}{0.000000,0.000000,0.000000}%
\pgfsetstrokecolor{currentstroke}%
\pgfsetdash{}{0pt}%
\pgfsys@defobject{currentmarker}{\pgfqpoint{0.000000in}{-0.048611in}}{\pgfqpoint{0.000000in}{0.000000in}}{%
\pgfpathmoveto{\pgfqpoint{0.000000in}{0.000000in}}%
\pgfpathlineto{\pgfqpoint{0.000000in}{-0.048611in}}%
\pgfusepath{stroke,fill}%
}%
\begin{pgfscope}%
\pgfsys@transformshift{5.366140in}{0.330000in}%
\pgfsys@useobject{currentmarker}{}%
\end{pgfscope}%
\end{pgfscope}%
\begin{pgfscope}%
\pgfsetbuttcap%
\pgfsetroundjoin%
\definecolor{currentfill}{rgb}{0.000000,0.000000,0.000000}%
\pgfsetfillcolor{currentfill}%
\pgfsetlinewidth{0.803000pt}%
\definecolor{currentstroke}{rgb}{0.000000,0.000000,0.000000}%
\pgfsetstrokecolor{currentstroke}%
\pgfsetdash{}{0pt}%
\pgfsys@defobject{currentmarker}{\pgfqpoint{0.000000in}{0.000000in}}{\pgfqpoint{0.000000in}{0.048611in}}{%
\pgfpathmoveto{\pgfqpoint{0.000000in}{0.000000in}}%
\pgfpathlineto{\pgfqpoint{0.000000in}{0.048611in}}%
\pgfusepath{stroke,fill}%
}%
\begin{pgfscope}%
\pgfsys@transformshift{5.366140in}{2.640000in}%
\pgfsys@useobject{currentmarker}{}%
\end{pgfscope}%
\end{pgfscope}%
\begin{pgfscope}%
\definecolor{textcolor}{rgb}{0.000000,0.000000,0.000000}%
\pgfsetstrokecolor{textcolor}%
\pgfsetfillcolor{textcolor}%
\pgftext[x=5.366140in,y=0.232778in,,top]{\color{textcolor}\rmfamily\fontsize{10.000000}{12.000000}\selectfont \(\displaystyle {25}\)}%
\end{pgfscope}%
\begin{pgfscope}%
\pgfpathrectangle{\pgfqpoint{1.000000in}{0.330000in}}{\pgfqpoint{6.200000in}{2.310000in}}%
\pgfusepath{clip}%
\pgfsetbuttcap%
\pgfsetroundjoin%
\pgfsetlinewidth{0.803000pt}%
\definecolor{currentstroke}{rgb}{0.000000,0.000000,0.000000}%
\pgfsetstrokecolor{currentstroke}%
\pgfsetdash{{0.800000pt}{1.320000pt}}{0.000000pt}%
\pgfpathmoveto{\pgfqpoint{6.183004in}{0.330000in}}%
\pgfpathlineto{\pgfqpoint{6.183004in}{2.640000in}}%
\pgfusepath{stroke}%
\end{pgfscope}%
\begin{pgfscope}%
\pgfsetbuttcap%
\pgfsetroundjoin%
\definecolor{currentfill}{rgb}{0.000000,0.000000,0.000000}%
\pgfsetfillcolor{currentfill}%
\pgfsetlinewidth{0.803000pt}%
\definecolor{currentstroke}{rgb}{0.000000,0.000000,0.000000}%
\pgfsetstrokecolor{currentstroke}%
\pgfsetdash{}{0pt}%
\pgfsys@defobject{currentmarker}{\pgfqpoint{0.000000in}{-0.048611in}}{\pgfqpoint{0.000000in}{0.000000in}}{%
\pgfpathmoveto{\pgfqpoint{0.000000in}{0.000000in}}%
\pgfpathlineto{\pgfqpoint{0.000000in}{-0.048611in}}%
\pgfusepath{stroke,fill}%
}%
\begin{pgfscope}%
\pgfsys@transformshift{6.183004in}{0.330000in}%
\pgfsys@useobject{currentmarker}{}%
\end{pgfscope}%
\end{pgfscope}%
\begin{pgfscope}%
\pgfsetbuttcap%
\pgfsetroundjoin%
\definecolor{currentfill}{rgb}{0.000000,0.000000,0.000000}%
\pgfsetfillcolor{currentfill}%
\pgfsetlinewidth{0.803000pt}%
\definecolor{currentstroke}{rgb}{0.000000,0.000000,0.000000}%
\pgfsetstrokecolor{currentstroke}%
\pgfsetdash{}{0pt}%
\pgfsys@defobject{currentmarker}{\pgfqpoint{0.000000in}{0.000000in}}{\pgfqpoint{0.000000in}{0.048611in}}{%
\pgfpathmoveto{\pgfqpoint{0.000000in}{0.000000in}}%
\pgfpathlineto{\pgfqpoint{0.000000in}{0.048611in}}%
\pgfusepath{stroke,fill}%
}%
\begin{pgfscope}%
\pgfsys@transformshift{6.183004in}{2.640000in}%
\pgfsys@useobject{currentmarker}{}%
\end{pgfscope}%
\end{pgfscope}%
\begin{pgfscope}%
\definecolor{textcolor}{rgb}{0.000000,0.000000,0.000000}%
\pgfsetstrokecolor{textcolor}%
\pgfsetfillcolor{textcolor}%
\pgftext[x=6.183004in,y=0.232778in,,top]{\color{textcolor}\rmfamily\fontsize{10.000000}{12.000000}\selectfont \(\displaystyle {30}\)}%
\end{pgfscope}%
\begin{pgfscope}%
\pgfpathrectangle{\pgfqpoint{1.000000in}{0.330000in}}{\pgfqpoint{6.200000in}{2.310000in}}%
\pgfusepath{clip}%
\pgfsetbuttcap%
\pgfsetroundjoin%
\pgfsetlinewidth{0.803000pt}%
\definecolor{currentstroke}{rgb}{0.000000,0.000000,0.000000}%
\pgfsetstrokecolor{currentstroke}%
\pgfsetdash{{0.800000pt}{1.320000pt}}{0.000000pt}%
\pgfpathmoveto{\pgfqpoint{6.999868in}{0.330000in}}%
\pgfpathlineto{\pgfqpoint{6.999868in}{2.640000in}}%
\pgfusepath{stroke}%
\end{pgfscope}%
\begin{pgfscope}%
\pgfsetbuttcap%
\pgfsetroundjoin%
\definecolor{currentfill}{rgb}{0.000000,0.000000,0.000000}%
\pgfsetfillcolor{currentfill}%
\pgfsetlinewidth{0.803000pt}%
\definecolor{currentstroke}{rgb}{0.000000,0.000000,0.000000}%
\pgfsetstrokecolor{currentstroke}%
\pgfsetdash{}{0pt}%
\pgfsys@defobject{currentmarker}{\pgfqpoint{0.000000in}{-0.048611in}}{\pgfqpoint{0.000000in}{0.000000in}}{%
\pgfpathmoveto{\pgfqpoint{0.000000in}{0.000000in}}%
\pgfpathlineto{\pgfqpoint{0.000000in}{-0.048611in}}%
\pgfusepath{stroke,fill}%
}%
\begin{pgfscope}%
\pgfsys@transformshift{6.999868in}{0.330000in}%
\pgfsys@useobject{currentmarker}{}%
\end{pgfscope}%
\end{pgfscope}%
\begin{pgfscope}%
\pgfsetbuttcap%
\pgfsetroundjoin%
\definecolor{currentfill}{rgb}{0.000000,0.000000,0.000000}%
\pgfsetfillcolor{currentfill}%
\pgfsetlinewidth{0.803000pt}%
\definecolor{currentstroke}{rgb}{0.000000,0.000000,0.000000}%
\pgfsetstrokecolor{currentstroke}%
\pgfsetdash{}{0pt}%
\pgfsys@defobject{currentmarker}{\pgfqpoint{0.000000in}{0.000000in}}{\pgfqpoint{0.000000in}{0.048611in}}{%
\pgfpathmoveto{\pgfqpoint{0.000000in}{0.000000in}}%
\pgfpathlineto{\pgfqpoint{0.000000in}{0.048611in}}%
\pgfusepath{stroke,fill}%
}%
\begin{pgfscope}%
\pgfsys@transformshift{6.999868in}{2.640000in}%
\pgfsys@useobject{currentmarker}{}%
\end{pgfscope}%
\end{pgfscope}%
\begin{pgfscope}%
\definecolor{textcolor}{rgb}{0.000000,0.000000,0.000000}%
\pgfsetstrokecolor{textcolor}%
\pgfsetfillcolor{textcolor}%
\pgftext[x=6.999868in,y=0.232778in,,top]{\color{textcolor}\rmfamily\fontsize{10.000000}{12.000000}\selectfont \(\displaystyle {35}\)}%
\end{pgfscope}%
\begin{pgfscope}%
\pgfpathrectangle{\pgfqpoint{1.000000in}{0.330000in}}{\pgfqpoint{6.200000in}{2.310000in}}%
\pgfusepath{clip}%
\pgfsetbuttcap%
\pgfsetroundjoin%
\pgfsetlinewidth{0.803000pt}%
\definecolor{currentstroke}{rgb}{0.000000,0.000000,0.000000}%
\pgfsetstrokecolor{currentstroke}%
\pgfsetdash{{0.800000pt}{1.320000pt}}{0.000000pt}%
\pgfpathmoveto{\pgfqpoint{1.000000in}{0.682137in}}%
\pgfpathlineto{\pgfqpoint{7.200000in}{0.682137in}}%
\pgfusepath{stroke}%
\end{pgfscope}%
\begin{pgfscope}%
\pgfsetbuttcap%
\pgfsetroundjoin%
\definecolor{currentfill}{rgb}{0.000000,0.000000,0.000000}%
\pgfsetfillcolor{currentfill}%
\pgfsetlinewidth{0.803000pt}%
\definecolor{currentstroke}{rgb}{0.000000,0.000000,0.000000}%
\pgfsetstrokecolor{currentstroke}%
\pgfsetdash{}{0pt}%
\pgfsys@defobject{currentmarker}{\pgfqpoint{-0.048611in}{0.000000in}}{\pgfqpoint{-0.000000in}{0.000000in}}{%
\pgfpathmoveto{\pgfqpoint{-0.000000in}{0.000000in}}%
\pgfpathlineto{\pgfqpoint{-0.048611in}{0.000000in}}%
\pgfusepath{stroke,fill}%
}%
\begin{pgfscope}%
\pgfsys@transformshift{1.000000in}{0.682137in}%
\pgfsys@useobject{currentmarker}{}%
\end{pgfscope}%
\end{pgfscope}%
\begin{pgfscope}%
\pgfsetbuttcap%
\pgfsetroundjoin%
\definecolor{currentfill}{rgb}{0.000000,0.000000,0.000000}%
\pgfsetfillcolor{currentfill}%
\pgfsetlinewidth{0.803000pt}%
\definecolor{currentstroke}{rgb}{0.000000,0.000000,0.000000}%
\pgfsetstrokecolor{currentstroke}%
\pgfsetdash{}{0pt}%
\pgfsys@defobject{currentmarker}{\pgfqpoint{0.000000in}{0.000000in}}{\pgfqpoint{0.048611in}{0.000000in}}{%
\pgfpathmoveto{\pgfqpoint{0.000000in}{0.000000in}}%
\pgfpathlineto{\pgfqpoint{0.048611in}{0.000000in}}%
\pgfusepath{stroke,fill}%
}%
\begin{pgfscope}%
\pgfsys@transformshift{7.200000in}{0.682137in}%
\pgfsys@useobject{currentmarker}{}%
\end{pgfscope}%
\end{pgfscope}%
\begin{pgfscope}%
\definecolor{textcolor}{rgb}{0.000000,0.000000,0.000000}%
\pgfsetstrokecolor{textcolor}%
\pgfsetfillcolor{textcolor}%
\pgftext[x=0.833333in, y=0.633912in, left, base]{\color{textcolor}\rmfamily\fontsize{10.000000}{12.000000}\selectfont \(\displaystyle {0}\)}%
\end{pgfscope}%
\begin{pgfscope}%
\pgfpathrectangle{\pgfqpoint{1.000000in}{0.330000in}}{\pgfqpoint{6.200000in}{2.310000in}}%
\pgfusepath{clip}%
\pgfsetbuttcap%
\pgfsetroundjoin%
\pgfsetlinewidth{0.803000pt}%
\definecolor{currentstroke}{rgb}{0.000000,0.000000,0.000000}%
\pgfsetstrokecolor{currentstroke}%
\pgfsetdash{{0.800000pt}{1.320000pt}}{0.000000pt}%
\pgfpathmoveto{\pgfqpoint{1.000000in}{1.083553in}}%
\pgfpathlineto{\pgfqpoint{7.200000in}{1.083553in}}%
\pgfusepath{stroke}%
\end{pgfscope}%
\begin{pgfscope}%
\pgfsetbuttcap%
\pgfsetroundjoin%
\definecolor{currentfill}{rgb}{0.000000,0.000000,0.000000}%
\pgfsetfillcolor{currentfill}%
\pgfsetlinewidth{0.803000pt}%
\definecolor{currentstroke}{rgb}{0.000000,0.000000,0.000000}%
\pgfsetstrokecolor{currentstroke}%
\pgfsetdash{}{0pt}%
\pgfsys@defobject{currentmarker}{\pgfqpoint{-0.048611in}{0.000000in}}{\pgfqpoint{-0.000000in}{0.000000in}}{%
\pgfpathmoveto{\pgfqpoint{-0.000000in}{0.000000in}}%
\pgfpathlineto{\pgfqpoint{-0.048611in}{0.000000in}}%
\pgfusepath{stroke,fill}%
}%
\begin{pgfscope}%
\pgfsys@transformshift{1.000000in}{1.083553in}%
\pgfsys@useobject{currentmarker}{}%
\end{pgfscope}%
\end{pgfscope}%
\begin{pgfscope}%
\pgfsetbuttcap%
\pgfsetroundjoin%
\definecolor{currentfill}{rgb}{0.000000,0.000000,0.000000}%
\pgfsetfillcolor{currentfill}%
\pgfsetlinewidth{0.803000pt}%
\definecolor{currentstroke}{rgb}{0.000000,0.000000,0.000000}%
\pgfsetstrokecolor{currentstroke}%
\pgfsetdash{}{0pt}%
\pgfsys@defobject{currentmarker}{\pgfqpoint{0.000000in}{0.000000in}}{\pgfqpoint{0.048611in}{0.000000in}}{%
\pgfpathmoveto{\pgfqpoint{0.000000in}{0.000000in}}%
\pgfpathlineto{\pgfqpoint{0.048611in}{0.000000in}}%
\pgfusepath{stroke,fill}%
}%
\begin{pgfscope}%
\pgfsys@transformshift{7.200000in}{1.083553in}%
\pgfsys@useobject{currentmarker}{}%
\end{pgfscope}%
\end{pgfscope}%
\begin{pgfscope}%
\definecolor{textcolor}{rgb}{0.000000,0.000000,0.000000}%
\pgfsetstrokecolor{textcolor}%
\pgfsetfillcolor{textcolor}%
\pgftext[x=0.694444in, y=1.035328in, left, base]{\color{textcolor}\rmfamily\fontsize{10.000000}{12.000000}\selectfont \(\displaystyle {100}\)}%
\end{pgfscope}%
\begin{pgfscope}%
\pgfpathrectangle{\pgfqpoint{1.000000in}{0.330000in}}{\pgfqpoint{6.200000in}{2.310000in}}%
\pgfusepath{clip}%
\pgfsetbuttcap%
\pgfsetroundjoin%
\pgfsetlinewidth{0.803000pt}%
\definecolor{currentstroke}{rgb}{0.000000,0.000000,0.000000}%
\pgfsetstrokecolor{currentstroke}%
\pgfsetdash{{0.800000pt}{1.320000pt}}{0.000000pt}%
\pgfpathmoveto{\pgfqpoint{1.000000in}{1.484969in}}%
\pgfpathlineto{\pgfqpoint{7.200000in}{1.484969in}}%
\pgfusepath{stroke}%
\end{pgfscope}%
\begin{pgfscope}%
\pgfsetbuttcap%
\pgfsetroundjoin%
\definecolor{currentfill}{rgb}{0.000000,0.000000,0.000000}%
\pgfsetfillcolor{currentfill}%
\pgfsetlinewidth{0.803000pt}%
\definecolor{currentstroke}{rgb}{0.000000,0.000000,0.000000}%
\pgfsetstrokecolor{currentstroke}%
\pgfsetdash{}{0pt}%
\pgfsys@defobject{currentmarker}{\pgfqpoint{-0.048611in}{0.000000in}}{\pgfqpoint{-0.000000in}{0.000000in}}{%
\pgfpathmoveto{\pgfqpoint{-0.000000in}{0.000000in}}%
\pgfpathlineto{\pgfqpoint{-0.048611in}{0.000000in}}%
\pgfusepath{stroke,fill}%
}%
\begin{pgfscope}%
\pgfsys@transformshift{1.000000in}{1.484969in}%
\pgfsys@useobject{currentmarker}{}%
\end{pgfscope}%
\end{pgfscope}%
\begin{pgfscope}%
\pgfsetbuttcap%
\pgfsetroundjoin%
\definecolor{currentfill}{rgb}{0.000000,0.000000,0.000000}%
\pgfsetfillcolor{currentfill}%
\pgfsetlinewidth{0.803000pt}%
\definecolor{currentstroke}{rgb}{0.000000,0.000000,0.000000}%
\pgfsetstrokecolor{currentstroke}%
\pgfsetdash{}{0pt}%
\pgfsys@defobject{currentmarker}{\pgfqpoint{0.000000in}{0.000000in}}{\pgfqpoint{0.048611in}{0.000000in}}{%
\pgfpathmoveto{\pgfqpoint{0.000000in}{0.000000in}}%
\pgfpathlineto{\pgfqpoint{0.048611in}{0.000000in}}%
\pgfusepath{stroke,fill}%
}%
\begin{pgfscope}%
\pgfsys@transformshift{7.200000in}{1.484969in}%
\pgfsys@useobject{currentmarker}{}%
\end{pgfscope}%
\end{pgfscope}%
\begin{pgfscope}%
\definecolor{textcolor}{rgb}{0.000000,0.000000,0.000000}%
\pgfsetstrokecolor{textcolor}%
\pgfsetfillcolor{textcolor}%
\pgftext[x=0.694444in, y=1.436744in, left, base]{\color{textcolor}\rmfamily\fontsize{10.000000}{12.000000}\selectfont \(\displaystyle {200}\)}%
\end{pgfscope}%
\begin{pgfscope}%
\pgfpathrectangle{\pgfqpoint{1.000000in}{0.330000in}}{\pgfqpoint{6.200000in}{2.310000in}}%
\pgfusepath{clip}%
\pgfsetbuttcap%
\pgfsetroundjoin%
\pgfsetlinewidth{0.803000pt}%
\definecolor{currentstroke}{rgb}{0.000000,0.000000,0.000000}%
\pgfsetstrokecolor{currentstroke}%
\pgfsetdash{{0.800000pt}{1.320000pt}}{0.000000pt}%
\pgfpathmoveto{\pgfqpoint{1.000000in}{1.886385in}}%
\pgfpathlineto{\pgfqpoint{7.200000in}{1.886385in}}%
\pgfusepath{stroke}%
\end{pgfscope}%
\begin{pgfscope}%
\pgfsetbuttcap%
\pgfsetroundjoin%
\definecolor{currentfill}{rgb}{0.000000,0.000000,0.000000}%
\pgfsetfillcolor{currentfill}%
\pgfsetlinewidth{0.803000pt}%
\definecolor{currentstroke}{rgb}{0.000000,0.000000,0.000000}%
\pgfsetstrokecolor{currentstroke}%
\pgfsetdash{}{0pt}%
\pgfsys@defobject{currentmarker}{\pgfqpoint{-0.048611in}{0.000000in}}{\pgfqpoint{-0.000000in}{0.000000in}}{%
\pgfpathmoveto{\pgfqpoint{-0.000000in}{0.000000in}}%
\pgfpathlineto{\pgfqpoint{-0.048611in}{0.000000in}}%
\pgfusepath{stroke,fill}%
}%
\begin{pgfscope}%
\pgfsys@transformshift{1.000000in}{1.886385in}%
\pgfsys@useobject{currentmarker}{}%
\end{pgfscope}%
\end{pgfscope}%
\begin{pgfscope}%
\pgfsetbuttcap%
\pgfsetroundjoin%
\definecolor{currentfill}{rgb}{0.000000,0.000000,0.000000}%
\pgfsetfillcolor{currentfill}%
\pgfsetlinewidth{0.803000pt}%
\definecolor{currentstroke}{rgb}{0.000000,0.000000,0.000000}%
\pgfsetstrokecolor{currentstroke}%
\pgfsetdash{}{0pt}%
\pgfsys@defobject{currentmarker}{\pgfqpoint{0.000000in}{0.000000in}}{\pgfqpoint{0.048611in}{0.000000in}}{%
\pgfpathmoveto{\pgfqpoint{0.000000in}{0.000000in}}%
\pgfpathlineto{\pgfqpoint{0.048611in}{0.000000in}}%
\pgfusepath{stroke,fill}%
}%
\begin{pgfscope}%
\pgfsys@transformshift{7.200000in}{1.886385in}%
\pgfsys@useobject{currentmarker}{}%
\end{pgfscope}%
\end{pgfscope}%
\begin{pgfscope}%
\definecolor{textcolor}{rgb}{0.000000,0.000000,0.000000}%
\pgfsetstrokecolor{textcolor}%
\pgfsetfillcolor{textcolor}%
\pgftext[x=0.694444in, y=1.838159in, left, base]{\color{textcolor}\rmfamily\fontsize{10.000000}{12.000000}\selectfont \(\displaystyle {300}\)}%
\end{pgfscope}%
\begin{pgfscope}%
\pgfpathrectangle{\pgfqpoint{1.000000in}{0.330000in}}{\pgfqpoint{6.200000in}{2.310000in}}%
\pgfusepath{clip}%
\pgfsetbuttcap%
\pgfsetroundjoin%
\pgfsetlinewidth{0.803000pt}%
\definecolor{currentstroke}{rgb}{0.000000,0.000000,0.000000}%
\pgfsetstrokecolor{currentstroke}%
\pgfsetdash{{0.800000pt}{1.320000pt}}{0.000000pt}%
\pgfpathmoveto{\pgfqpoint{1.000000in}{2.287801in}}%
\pgfpathlineto{\pgfqpoint{7.200000in}{2.287801in}}%
\pgfusepath{stroke}%
\end{pgfscope}%
\begin{pgfscope}%
\pgfsetbuttcap%
\pgfsetroundjoin%
\definecolor{currentfill}{rgb}{0.000000,0.000000,0.000000}%
\pgfsetfillcolor{currentfill}%
\pgfsetlinewidth{0.803000pt}%
\definecolor{currentstroke}{rgb}{0.000000,0.000000,0.000000}%
\pgfsetstrokecolor{currentstroke}%
\pgfsetdash{}{0pt}%
\pgfsys@defobject{currentmarker}{\pgfqpoint{-0.048611in}{0.000000in}}{\pgfqpoint{-0.000000in}{0.000000in}}{%
\pgfpathmoveto{\pgfqpoint{-0.000000in}{0.000000in}}%
\pgfpathlineto{\pgfqpoint{-0.048611in}{0.000000in}}%
\pgfusepath{stroke,fill}%
}%
\begin{pgfscope}%
\pgfsys@transformshift{1.000000in}{2.287801in}%
\pgfsys@useobject{currentmarker}{}%
\end{pgfscope}%
\end{pgfscope}%
\begin{pgfscope}%
\pgfsetbuttcap%
\pgfsetroundjoin%
\definecolor{currentfill}{rgb}{0.000000,0.000000,0.000000}%
\pgfsetfillcolor{currentfill}%
\pgfsetlinewidth{0.803000pt}%
\definecolor{currentstroke}{rgb}{0.000000,0.000000,0.000000}%
\pgfsetstrokecolor{currentstroke}%
\pgfsetdash{}{0pt}%
\pgfsys@defobject{currentmarker}{\pgfqpoint{0.000000in}{0.000000in}}{\pgfqpoint{0.048611in}{0.000000in}}{%
\pgfpathmoveto{\pgfqpoint{0.000000in}{0.000000in}}%
\pgfpathlineto{\pgfqpoint{0.048611in}{0.000000in}}%
\pgfusepath{stroke,fill}%
}%
\begin{pgfscope}%
\pgfsys@transformshift{7.200000in}{2.287801in}%
\pgfsys@useobject{currentmarker}{}%
\end{pgfscope}%
\end{pgfscope}%
\begin{pgfscope}%
\definecolor{textcolor}{rgb}{0.000000,0.000000,0.000000}%
\pgfsetstrokecolor{textcolor}%
\pgfsetfillcolor{textcolor}%
\pgftext[x=0.694444in, y=2.239575in, left, base]{\color{textcolor}\rmfamily\fontsize{10.000000}{12.000000}\selectfont \(\displaystyle {400}\)}%
\end{pgfscope}%
\begin{pgfscope}%
\pgfpathrectangle{\pgfqpoint{1.000000in}{0.330000in}}{\pgfqpoint{6.200000in}{2.310000in}}%
\pgfusepath{clip}%
\pgfsetrectcap%
\pgfsetroundjoin%
\pgfsetlinewidth{1.505625pt}%
\definecolor{currentstroke}{rgb}{0.121569,0.466667,0.705882}%
\pgfsetstrokecolor{currentstroke}%
\pgfsetdash{}{0pt}%
\pgfpathmoveto{\pgfqpoint{1.281818in}{0.682137in}}%
\pgfpathlineto{\pgfqpoint{1.281818in}{0.682137in}}%
\pgfpathlineto{\pgfqpoint{1.472420in}{0.681075in}}%
\pgfpathlineto{\pgfqpoint{1.663022in}{0.677836in}}%
\pgfpathlineto{\pgfqpoint{1.853623in}{0.672420in}}%
\pgfpathlineto{\pgfqpoint{2.016996in}{0.666045in}}%
\pgfpathlineto{\pgfqpoint{2.193983in}{0.693722in}}%
\pgfpathlineto{\pgfqpoint{2.370971in}{0.719004in}}%
\pgfpathlineto{\pgfqpoint{2.547958in}{0.741891in}}%
\pgfpathlineto{\pgfqpoint{2.724945in}{0.762383in}}%
\pgfpathlineto{\pgfqpoint{2.901932in}{0.780479in}}%
\pgfpathlineto{\pgfqpoint{3.078920in}{0.796181in}}%
\pgfpathlineto{\pgfqpoint{3.255907in}{0.809487in}}%
\pgfpathlineto{\pgfqpoint{3.432894in}{0.820399in}}%
\pgfpathlineto{\pgfqpoint{3.609881in}{0.828915in}}%
\pgfpathlineto{\pgfqpoint{3.786869in}{0.835036in}}%
\pgfpathlineto{\pgfqpoint{3.963856in}{0.838762in}}%
\pgfpathlineto{\pgfqpoint{4.140843in}{0.840093in}}%
\pgfpathlineto{\pgfqpoint{4.317830in}{0.839029in}}%
\pgfpathlineto{\pgfqpoint{4.494818in}{0.835569in}}%
\pgfpathlineto{\pgfqpoint{4.671805in}{0.829715in}}%
\pgfpathlineto{\pgfqpoint{4.848792in}{0.821466in}}%
\pgfpathlineto{\pgfqpoint{5.025780in}{0.810821in}}%
\pgfpathlineto{\pgfqpoint{5.202767in}{0.797781in}}%
\pgfpathlineto{\pgfqpoint{5.379754in}{0.782347in}}%
\pgfpathlineto{\pgfqpoint{5.556741in}{0.764517in}}%
\pgfpathlineto{\pgfqpoint{5.733729in}{0.744292in}}%
\pgfpathlineto{\pgfqpoint{5.910716in}{0.721672in}}%
\pgfpathlineto{\pgfqpoint{6.087703in}{0.696656in}}%
\pgfpathlineto{\pgfqpoint{6.264690in}{0.669251in}}%
\pgfpathlineto{\pgfqpoint{6.455292in}{0.675651in}}%
\pgfpathlineto{\pgfqpoint{6.645894in}{0.679875in}}%
\pgfpathlineto{\pgfqpoint{6.836495in}{0.681923in}}%
\pgfpathlineto{\pgfqpoint{6.918182in}{0.682137in}}%
\pgfpathlineto{\pgfqpoint{6.918182in}{0.682137in}}%
\pgfusepath{stroke}%
\end{pgfscope}%
\begin{pgfscope}%
\pgfpathrectangle{\pgfqpoint{1.000000in}{0.330000in}}{\pgfqpoint{6.200000in}{2.310000in}}%
\pgfusepath{clip}%
\pgfsetrectcap%
\pgfsetroundjoin%
\pgfsetlinewidth{1.505625pt}%
\definecolor{currentstroke}{rgb}{1.000000,0.498039,0.054902}%
\pgfsetstrokecolor{currentstroke}%
\pgfsetdash{}{0pt}%
\pgfpathmoveto{\pgfqpoint{1.281818in}{0.682137in}}%
\pgfpathlineto{\pgfqpoint{1.281818in}{0.682137in}}%
\pgfpathlineto{\pgfqpoint{1.417962in}{0.681126in}}%
\pgfpathlineto{\pgfqpoint{1.554106in}{0.678023in}}%
\pgfpathlineto{\pgfqpoint{1.690250in}{0.672829in}}%
\pgfpathlineto{\pgfqpoint{1.826394in}{0.665542in}}%
\pgfpathlineto{\pgfqpoint{1.962538in}{0.656164in}}%
\pgfpathlineto{\pgfqpoint{2.016996in}{0.651827in}}%
\pgfpathlineto{\pgfqpoint{2.207598in}{0.675264in}}%
\pgfpathlineto{\pgfqpoint{2.398199in}{0.696525in}}%
\pgfpathlineto{\pgfqpoint{2.588801in}{0.715609in}}%
\pgfpathlineto{\pgfqpoint{2.779403in}{0.732517in}}%
\pgfpathlineto{\pgfqpoint{2.970004in}{0.747247in}}%
\pgfpathlineto{\pgfqpoint{3.160606in}{0.759801in}}%
\pgfpathlineto{\pgfqpoint{3.351208in}{0.770178in}}%
\pgfpathlineto{\pgfqpoint{3.541809in}{0.778378in}}%
\pgfpathlineto{\pgfqpoint{3.732411in}{0.784402in}}%
\pgfpathlineto{\pgfqpoint{3.923013in}{0.788249in}}%
\pgfpathlineto{\pgfqpoint{4.113614in}{0.789919in}}%
\pgfpathlineto{\pgfqpoint{4.304216in}{0.789412in}}%
\pgfpathlineto{\pgfqpoint{4.494818in}{0.786729in}}%
\pgfpathlineto{\pgfqpoint{4.685419in}{0.781869in}}%
\pgfpathlineto{\pgfqpoint{4.876021in}{0.774832in}}%
\pgfpathlineto{\pgfqpoint{5.066623in}{0.765618in}}%
\pgfpathlineto{\pgfqpoint{5.257224in}{0.754228in}}%
\pgfpathlineto{\pgfqpoint{5.447826in}{0.740661in}}%
\pgfpathlineto{\pgfqpoint{5.638428in}{0.724917in}}%
\pgfpathlineto{\pgfqpoint{5.829029in}{0.706997in}}%
\pgfpathlineto{\pgfqpoint{6.019631in}{0.686899in}}%
\pgfpathlineto{\pgfqpoint{6.210233in}{0.664625in}}%
\pgfpathlineto{\pgfqpoint{6.264690in}{0.657865in}}%
\pgfpathlineto{\pgfqpoint{6.400834in}{0.666895in}}%
\pgfpathlineto{\pgfqpoint{6.536978in}{0.673833in}}%
\pgfpathlineto{\pgfqpoint{6.673123in}{0.678679in}}%
\pgfpathlineto{\pgfqpoint{6.809267in}{0.681433in}}%
\pgfpathlineto{\pgfqpoint{6.918182in}{0.682137in}}%
\pgfpathlineto{\pgfqpoint{6.918182in}{0.682137in}}%
\pgfusepath{stroke}%
\end{pgfscope}%
\begin{pgfscope}%
\pgfpathrectangle{\pgfqpoint{1.000000in}{0.330000in}}{\pgfqpoint{6.200000in}{2.310000in}}%
\pgfusepath{clip}%
\pgfsetrectcap%
\pgfsetroundjoin%
\pgfsetlinewidth{1.505625pt}%
\definecolor{currentstroke}{rgb}{0.172549,0.627451,0.172549}%
\pgfsetstrokecolor{currentstroke}%
\pgfsetdash{}{0pt}%
\pgfpathmoveto{\pgfqpoint{1.281818in}{0.682137in}}%
\pgfpathlineto{\pgfqpoint{1.281818in}{0.682137in}}%
\pgfpathlineto{\pgfqpoint{1.458805in}{0.680970in}}%
\pgfpathlineto{\pgfqpoint{1.635793in}{0.677408in}}%
\pgfpathlineto{\pgfqpoint{1.812780in}{0.671451in}}%
\pgfpathlineto{\pgfqpoint{1.989767in}{0.663099in}}%
\pgfpathlineto{\pgfqpoint{2.016996in}{0.661602in}}%
\pgfpathlineto{\pgfqpoint{2.193983in}{0.689316in}}%
\pgfpathlineto{\pgfqpoint{2.370971in}{0.714635in}}%
\pgfpathlineto{\pgfqpoint{2.547958in}{0.737558in}}%
\pgfpathlineto{\pgfqpoint{2.724945in}{0.758087in}}%
\pgfpathlineto{\pgfqpoint{2.901932in}{0.776220in}}%
\pgfpathlineto{\pgfqpoint{3.078920in}{0.791959in}}%
\pgfpathlineto{\pgfqpoint{3.255907in}{0.805302in}}%
\pgfpathlineto{\pgfqpoint{3.432894in}{0.816250in}}%
\pgfpathlineto{\pgfqpoint{3.609881in}{0.824804in}}%
\pgfpathlineto{\pgfqpoint{3.786869in}{0.830962in}}%
\pgfpathlineto{\pgfqpoint{3.963856in}{0.834725in}}%
\pgfpathlineto{\pgfqpoint{4.140843in}{0.836092in}}%
\pgfpathlineto{\pgfqpoint{4.317830in}{0.835065in}}%
\pgfpathlineto{\pgfqpoint{4.494818in}{0.831643in}}%
\pgfpathlineto{\pgfqpoint{4.671805in}{0.825825in}}%
\pgfpathlineto{\pgfqpoint{4.848792in}{0.817613in}}%
\pgfpathlineto{\pgfqpoint{5.025780in}{0.807005in}}%
\pgfpathlineto{\pgfqpoint{5.202767in}{0.794002in}}%
\pgfpathlineto{\pgfqpoint{5.379754in}{0.778604in}}%
\pgfpathlineto{\pgfqpoint{5.556741in}{0.760811in}}%
\pgfpathlineto{\pgfqpoint{5.733729in}{0.740623in}}%
\pgfpathlineto{\pgfqpoint{5.910716in}{0.718040in}}%
\pgfpathlineto{\pgfqpoint{6.087703in}{0.693061in}}%
\pgfpathlineto{\pgfqpoint{6.264690in}{0.665693in}}%
\pgfpathlineto{\pgfqpoint{6.441678in}{0.673369in}}%
\pgfpathlineto{\pgfqpoint{6.618665in}{0.678651in}}%
\pgfpathlineto{\pgfqpoint{6.795652in}{0.681537in}}%
\pgfpathlineto{\pgfqpoint{6.918182in}{0.682137in}}%
\pgfpathlineto{\pgfqpoint{6.918182in}{0.682137in}}%
\pgfusepath{stroke}%
\end{pgfscope}%
\begin{pgfscope}%
\pgfpathrectangle{\pgfqpoint{1.000000in}{0.330000in}}{\pgfqpoint{6.200000in}{2.310000in}}%
\pgfusepath{clip}%
\pgfsetrectcap%
\pgfsetroundjoin%
\pgfsetlinewidth{1.505625pt}%
\definecolor{currentstroke}{rgb}{0.839216,0.152941,0.156863}%
\pgfsetstrokecolor{currentstroke}%
\pgfsetdash{}{0pt}%
\pgfpathmoveto{\pgfqpoint{1.281818in}{0.682137in}}%
\pgfpathlineto{\pgfqpoint{1.281818in}{0.682137in}}%
\pgfpathlineto{\pgfqpoint{1.458805in}{0.680970in}}%
\pgfpathlineto{\pgfqpoint{1.635793in}{0.677408in}}%
\pgfpathlineto{\pgfqpoint{1.812780in}{0.671451in}}%
\pgfpathlineto{\pgfqpoint{1.989767in}{0.663099in}}%
\pgfpathlineto{\pgfqpoint{2.016996in}{0.661602in}}%
\pgfpathlineto{\pgfqpoint{2.193983in}{0.689464in}}%
\pgfpathlineto{\pgfqpoint{2.370971in}{0.714931in}}%
\pgfpathlineto{\pgfqpoint{2.547958in}{0.738003in}}%
\pgfpathlineto{\pgfqpoint{2.724945in}{0.758680in}}%
\pgfpathlineto{\pgfqpoint{2.901932in}{0.776962in}}%
\pgfpathlineto{\pgfqpoint{3.078920in}{0.792848in}}%
\pgfpathlineto{\pgfqpoint{3.255907in}{0.806340in}}%
\pgfpathlineto{\pgfqpoint{3.432894in}{0.817436in}}%
\pgfpathlineto{\pgfqpoint{3.609881in}{0.826138in}}%
\pgfpathlineto{\pgfqpoint{3.786869in}{0.832444in}}%
\pgfpathlineto{\pgfqpoint{3.963856in}{0.836355in}}%
\pgfpathlineto{\pgfqpoint{4.140843in}{0.837871in}}%
\pgfpathlineto{\pgfqpoint{4.317830in}{0.836992in}}%
\pgfpathlineto{\pgfqpoint{4.494818in}{0.833718in}}%
\pgfpathlineto{\pgfqpoint{4.671805in}{0.828049in}}%
\pgfpathlineto{\pgfqpoint{4.848792in}{0.819985in}}%
\pgfpathlineto{\pgfqpoint{5.025780in}{0.809525in}}%
\pgfpathlineto{\pgfqpoint{5.202767in}{0.796671in}}%
\pgfpathlineto{\pgfqpoint{5.379754in}{0.781421in}}%
\pgfpathlineto{\pgfqpoint{5.556741in}{0.763776in}}%
\pgfpathlineto{\pgfqpoint{5.733729in}{0.743736in}}%
\pgfpathlineto{\pgfqpoint{5.910716in}{0.721301in}}%
\pgfpathlineto{\pgfqpoint{6.087703in}{0.696471in}}%
\pgfpathlineto{\pgfqpoint{6.264690in}{0.669251in}}%
\pgfpathlineto{\pgfqpoint{6.455292in}{0.675651in}}%
\pgfpathlineto{\pgfqpoint{6.645894in}{0.679875in}}%
\pgfpathlineto{\pgfqpoint{6.836495in}{0.681923in}}%
\pgfpathlineto{\pgfqpoint{6.918182in}{0.682137in}}%
\pgfpathlineto{\pgfqpoint{6.918182in}{0.682137in}}%
\pgfusepath{stroke}%
\end{pgfscope}%
\begin{pgfscope}%
\pgfpathrectangle{\pgfqpoint{1.000000in}{0.330000in}}{\pgfqpoint{6.200000in}{2.310000in}}%
\pgfusepath{clip}%
\pgfsetrectcap%
\pgfsetroundjoin%
\pgfsetlinewidth{1.505625pt}%
\definecolor{currentstroke}{rgb}{0.580392,0.403922,0.741176}%
\pgfsetstrokecolor{currentstroke}%
\pgfsetdash{}{0pt}%
\pgfpathmoveto{\pgfqpoint{1.281818in}{0.682137in}}%
\pgfpathlineto{\pgfqpoint{1.281818in}{0.682137in}}%
\pgfpathlineto{\pgfqpoint{1.472420in}{0.681075in}}%
\pgfpathlineto{\pgfqpoint{1.663022in}{0.677836in}}%
\pgfpathlineto{\pgfqpoint{1.853623in}{0.672420in}}%
\pgfpathlineto{\pgfqpoint{2.016996in}{0.666045in}}%
\pgfpathlineto{\pgfqpoint{2.207598in}{0.688844in}}%
\pgfpathlineto{\pgfqpoint{2.398199in}{0.709467in}}%
\pgfpathlineto{\pgfqpoint{2.588801in}{0.727913in}}%
\pgfpathlineto{\pgfqpoint{2.779403in}{0.744183in}}%
\pgfpathlineto{\pgfqpoint{2.970004in}{0.758275in}}%
\pgfpathlineto{\pgfqpoint{3.160606in}{0.770191in}}%
\pgfpathlineto{\pgfqpoint{3.351208in}{0.779930in}}%
\pgfpathlineto{\pgfqpoint{3.541809in}{0.787493in}}%
\pgfpathlineto{\pgfqpoint{3.732411in}{0.792878in}}%
\pgfpathlineto{\pgfqpoint{3.923013in}{0.796087in}}%
\pgfpathlineto{\pgfqpoint{4.113614in}{0.797119in}}%
\pgfpathlineto{\pgfqpoint{4.304216in}{0.795975in}}%
\pgfpathlineto{\pgfqpoint{4.494818in}{0.792653in}}%
\pgfpathlineto{\pgfqpoint{4.685419in}{0.787155in}}%
\pgfpathlineto{\pgfqpoint{4.876021in}{0.779480in}}%
\pgfpathlineto{\pgfqpoint{5.066623in}{0.769629in}}%
\pgfpathlineto{\pgfqpoint{5.257224in}{0.757600in}}%
\pgfpathlineto{\pgfqpoint{5.447826in}{0.743395in}}%
\pgfpathlineto{\pgfqpoint{5.638428in}{0.727013in}}%
\pgfpathlineto{\pgfqpoint{5.829029in}{0.708455in}}%
\pgfpathlineto{\pgfqpoint{6.019631in}{0.687720in}}%
\pgfpathlineto{\pgfqpoint{6.210233in}{0.664808in}}%
\pgfpathlineto{\pgfqpoint{6.264690in}{0.657865in}}%
\pgfpathlineto{\pgfqpoint{6.400834in}{0.666895in}}%
\pgfpathlineto{\pgfqpoint{6.536978in}{0.673833in}}%
\pgfpathlineto{\pgfqpoint{6.673123in}{0.678679in}}%
\pgfpathlineto{\pgfqpoint{6.809267in}{0.681433in}}%
\pgfpathlineto{\pgfqpoint{6.918182in}{0.682137in}}%
\pgfpathlineto{\pgfqpoint{6.918182in}{0.682137in}}%
\pgfusepath{stroke}%
\end{pgfscope}%
\begin{pgfscope}%
\pgfpathrectangle{\pgfqpoint{1.000000in}{0.330000in}}{\pgfqpoint{6.200000in}{2.310000in}}%
\pgfusepath{clip}%
\pgfsetrectcap%
\pgfsetroundjoin%
\pgfsetlinewidth{1.505625pt}%
\definecolor{currentstroke}{rgb}{0.549020,0.337255,0.294118}%
\pgfsetstrokecolor{currentstroke}%
\pgfsetdash{}{0pt}%
\pgfpathmoveto{\pgfqpoint{1.281818in}{0.682137in}}%
\pgfpathlineto{\pgfqpoint{1.281818in}{0.682137in}}%
\pgfpathlineto{\pgfqpoint{1.390733in}{0.681129in}}%
\pgfpathlineto{\pgfqpoint{1.499649in}{0.678015in}}%
\pgfpathlineto{\pgfqpoint{1.608564in}{0.672797in}}%
\pgfpathlineto{\pgfqpoint{1.717479in}{0.665474in}}%
\pgfpathlineto{\pgfqpoint{1.826394in}{0.656046in}}%
\pgfpathlineto{\pgfqpoint{1.935310in}{0.644513in}}%
\pgfpathlineto{\pgfqpoint{2.016996in}{0.634482in}}%
\pgfpathlineto{\pgfqpoint{2.139526in}{0.679599in}}%
\pgfpathlineto{\pgfqpoint{2.248441in}{0.717466in}}%
\pgfpathlineto{\pgfqpoint{2.357356in}{0.753229in}}%
\pgfpathlineto{\pgfqpoint{2.466271in}{0.786886in}}%
\pgfpathlineto{\pgfqpoint{2.575187in}{0.818439in}}%
\pgfpathlineto{\pgfqpoint{2.684102in}{0.847887in}}%
\pgfpathlineto{\pgfqpoint{2.793017in}{0.875230in}}%
\pgfpathlineto{\pgfqpoint{2.901932in}{0.900468in}}%
\pgfpathlineto{\pgfqpoint{3.010848in}{0.923601in}}%
\pgfpathlineto{\pgfqpoint{3.119763in}{0.944630in}}%
\pgfpathlineto{\pgfqpoint{3.228678in}{0.963554in}}%
\pgfpathlineto{\pgfqpoint{3.337593in}{0.980372in}}%
\pgfpathlineto{\pgfqpoint{3.446509in}{0.995086in}}%
\pgfpathlineto{\pgfqpoint{3.555424in}{1.007695in}}%
\pgfpathlineto{\pgfqpoint{3.664339in}{1.018200in}}%
\pgfpathlineto{\pgfqpoint{3.773254in}{1.026599in}}%
\pgfpathlineto{\pgfqpoint{3.882170in}{1.032894in}}%
\pgfpathlineto{\pgfqpoint{3.991085in}{1.037083in}}%
\pgfpathlineto{\pgfqpoint{4.100000in}{1.039168in}}%
\pgfpathlineto{\pgfqpoint{4.208915in}{1.039148in}}%
\pgfpathlineto{\pgfqpoint{4.317830in}{1.037024in}}%
\pgfpathlineto{\pgfqpoint{4.426746in}{1.032794in}}%
\pgfpathlineto{\pgfqpoint{4.535661in}{1.026459in}}%
\pgfpathlineto{\pgfqpoint{4.644576in}{1.018020in}}%
\pgfpathlineto{\pgfqpoint{4.753491in}{1.007476in}}%
\pgfpathlineto{\pgfqpoint{4.862407in}{0.994827in}}%
\pgfpathlineto{\pgfqpoint{4.971322in}{0.980073in}}%
\pgfpathlineto{\pgfqpoint{5.080237in}{0.963214in}}%
\pgfpathlineto{\pgfqpoint{5.189152in}{0.944251in}}%
\pgfpathlineto{\pgfqpoint{5.298068in}{0.923182in}}%
\pgfpathlineto{\pgfqpoint{5.406983in}{0.900009in}}%
\pgfpathlineto{\pgfqpoint{5.515898in}{0.874731in}}%
\pgfpathlineto{\pgfqpoint{5.624813in}{0.847348in}}%
\pgfpathlineto{\pgfqpoint{5.733729in}{0.817860in}}%
\pgfpathlineto{\pgfqpoint{5.842644in}{0.786267in}}%
\pgfpathlineto{\pgfqpoint{5.951559in}{0.752570in}}%
\pgfpathlineto{\pgfqpoint{6.060474in}{0.716768in}}%
\pgfpathlineto{\pgfqpoint{6.169390in}{0.678860in}}%
\pgfpathlineto{\pgfqpoint{6.264690in}{0.643976in}}%
\pgfpathlineto{\pgfqpoint{6.373606in}{0.655596in}}%
\pgfpathlineto{\pgfqpoint{6.482521in}{0.665112in}}%
\pgfpathlineto{\pgfqpoint{6.591436in}{0.672523in}}%
\pgfpathlineto{\pgfqpoint{6.700351in}{0.677829in}}%
\pgfpathlineto{\pgfqpoint{6.809267in}{0.681030in}}%
\pgfpathlineto{\pgfqpoint{6.918182in}{0.682137in}}%
\pgfpathlineto{\pgfqpoint{6.918182in}{0.682137in}}%
\pgfusepath{stroke}%
\end{pgfscope}%
\begin{pgfscope}%
\pgfpathrectangle{\pgfqpoint{1.000000in}{0.330000in}}{\pgfqpoint{6.200000in}{2.310000in}}%
\pgfusepath{clip}%
\pgfsetrectcap%
\pgfsetroundjoin%
\pgfsetlinewidth{1.505625pt}%
\definecolor{currentstroke}{rgb}{0.890196,0.466667,0.760784}%
\pgfsetstrokecolor{currentstroke}%
\pgfsetdash{}{0pt}%
\pgfpathmoveto{\pgfqpoint{1.281818in}{0.682137in}}%
\pgfpathlineto{\pgfqpoint{1.281818in}{0.682137in}}%
\pgfpathlineto{\pgfqpoint{1.472420in}{0.681075in}}%
\pgfpathlineto{\pgfqpoint{1.663022in}{0.677836in}}%
\pgfpathlineto{\pgfqpoint{1.853623in}{0.672420in}}%
\pgfpathlineto{\pgfqpoint{2.016996in}{0.666045in}}%
\pgfpathlineto{\pgfqpoint{2.193983in}{0.693574in}}%
\pgfpathlineto{\pgfqpoint{2.370971in}{0.718707in}}%
\pgfpathlineto{\pgfqpoint{2.547958in}{0.741446in}}%
\pgfpathlineto{\pgfqpoint{2.724945in}{0.761790in}}%
\pgfpathlineto{\pgfqpoint{2.901932in}{0.779738in}}%
\pgfpathlineto{\pgfqpoint{3.078920in}{0.795291in}}%
\pgfpathlineto{\pgfqpoint{3.255907in}{0.808449in}}%
\pgfpathlineto{\pgfqpoint{3.432894in}{0.819213in}}%
\pgfpathlineto{\pgfqpoint{3.609881in}{0.827581in}}%
\pgfpathlineto{\pgfqpoint{3.786869in}{0.833553in}}%
\pgfpathlineto{\pgfqpoint{3.963856in}{0.837131in}}%
\pgfpathlineto{\pgfqpoint{4.140843in}{0.838314in}}%
\pgfpathlineto{\pgfqpoint{4.317830in}{0.837101in}}%
\pgfpathlineto{\pgfqpoint{4.494818in}{0.833494in}}%
\pgfpathlineto{\pgfqpoint{4.671805in}{0.827491in}}%
\pgfpathlineto{\pgfqpoint{4.848792in}{0.819094in}}%
\pgfpathlineto{\pgfqpoint{5.025780in}{0.808301in}}%
\pgfpathlineto{\pgfqpoint{5.202767in}{0.795113in}}%
\pgfpathlineto{\pgfqpoint{5.379754in}{0.779530in}}%
\pgfpathlineto{\pgfqpoint{5.556741in}{0.761552in}}%
\pgfpathlineto{\pgfqpoint{5.733729in}{0.741178in}}%
\pgfpathlineto{\pgfqpoint{5.910716in}{0.718410in}}%
\pgfpathlineto{\pgfqpoint{6.087703in}{0.693247in}}%
\pgfpathlineto{\pgfqpoint{6.264690in}{0.665693in}}%
\pgfpathlineto{\pgfqpoint{6.441678in}{0.673369in}}%
\pgfpathlineto{\pgfqpoint{6.618665in}{0.678651in}}%
\pgfpathlineto{\pgfqpoint{6.795652in}{0.681537in}}%
\pgfpathlineto{\pgfqpoint{6.918182in}{0.682137in}}%
\pgfpathlineto{\pgfqpoint{6.918182in}{0.682137in}}%
\pgfusepath{stroke}%
\end{pgfscope}%
\begin{pgfscope}%
\pgfpathrectangle{\pgfqpoint{1.000000in}{0.330000in}}{\pgfqpoint{6.200000in}{2.310000in}}%
\pgfusepath{clip}%
\pgfsetrectcap%
\pgfsetroundjoin%
\pgfsetlinewidth{1.505625pt}%
\definecolor{currentstroke}{rgb}{0.498039,0.498039,0.498039}%
\pgfsetstrokecolor{currentstroke}%
\pgfsetdash{}{0pt}%
\pgfpathmoveto{\pgfqpoint{1.281818in}{0.682137in}}%
\pgfpathlineto{\pgfqpoint{1.281818in}{0.682137in}}%
\pgfpathlineto{\pgfqpoint{1.336276in}{0.680887in}}%
\pgfpathlineto{\pgfqpoint{1.390733in}{0.676907in}}%
\pgfpathlineto{\pgfqpoint{1.445191in}{0.670198in}}%
\pgfpathlineto{\pgfqpoint{1.499649in}{0.660761in}}%
\pgfpathlineto{\pgfqpoint{1.554106in}{0.648594in}}%
\pgfpathlineto{\pgfqpoint{1.608564in}{0.633699in}}%
\pgfpathlineto{\pgfqpoint{1.663022in}{0.616075in}}%
\pgfpathlineto{\pgfqpoint{1.717479in}{0.595722in}}%
\pgfpathlineto{\pgfqpoint{1.771937in}{0.572640in}}%
\pgfpathlineto{\pgfqpoint{1.826394in}{0.546829in}}%
\pgfpathlineto{\pgfqpoint{1.880852in}{0.518289in}}%
\pgfpathlineto{\pgfqpoint{1.935310in}{0.487020in}}%
\pgfpathlineto{\pgfqpoint{1.989767in}{0.453022in}}%
\pgfpathlineto{\pgfqpoint{2.016996in}{0.435000in}}%
\pgfpathlineto{\pgfqpoint{2.085068in}{0.566691in}}%
\pgfpathlineto{\pgfqpoint{2.153140in}{0.694117in}}%
\pgfpathlineto{\pgfqpoint{2.221212in}{0.817280in}}%
\pgfpathlineto{\pgfqpoint{2.289284in}{0.936179in}}%
\pgfpathlineto{\pgfqpoint{2.357356in}{1.050814in}}%
\pgfpathlineto{\pgfqpoint{2.425428in}{1.161185in}}%
\pgfpathlineto{\pgfqpoint{2.493500in}{1.267292in}}%
\pgfpathlineto{\pgfqpoint{2.561572in}{1.369136in}}%
\pgfpathlineto{\pgfqpoint{2.629644in}{1.466715in}}%
\pgfpathlineto{\pgfqpoint{2.697716in}{1.560030in}}%
\pgfpathlineto{\pgfqpoint{2.765788in}{1.649082in}}%
\pgfpathlineto{\pgfqpoint{2.833860in}{1.733869in}}%
\pgfpathlineto{\pgfqpoint{2.901932in}{1.814393in}}%
\pgfpathlineto{\pgfqpoint{2.956390in}{1.875741in}}%
\pgfpathlineto{\pgfqpoint{3.010848in}{1.934361in}}%
\pgfpathlineto{\pgfqpoint{3.065305in}{1.990252in}}%
\pgfpathlineto{\pgfqpoint{3.119763in}{2.043414in}}%
\pgfpathlineto{\pgfqpoint{3.174220in}{2.093847in}}%
\pgfpathlineto{\pgfqpoint{3.228678in}{2.141552in}}%
\pgfpathlineto{\pgfqpoint{3.283136in}{2.186527in}}%
\pgfpathlineto{\pgfqpoint{3.337593in}{2.228773in}}%
\pgfpathlineto{\pgfqpoint{3.392051in}{2.268291in}}%
\pgfpathlineto{\pgfqpoint{3.446509in}{2.305080in}}%
\pgfpathlineto{\pgfqpoint{3.500966in}{2.339139in}}%
\pgfpathlineto{\pgfqpoint{3.555424in}{2.370470in}}%
\pgfpathlineto{\pgfqpoint{3.609881in}{2.399072in}}%
\pgfpathlineto{\pgfqpoint{3.664339in}{2.424945in}}%
\pgfpathlineto{\pgfqpoint{3.718797in}{2.448089in}}%
\pgfpathlineto{\pgfqpoint{3.773254in}{2.468504in}}%
\pgfpathlineto{\pgfqpoint{3.827712in}{2.486190in}}%
\pgfpathlineto{\pgfqpoint{3.882170in}{2.501147in}}%
\pgfpathlineto{\pgfqpoint{3.936627in}{2.513376in}}%
\pgfpathlineto{\pgfqpoint{3.991085in}{2.522875in}}%
\pgfpathlineto{\pgfqpoint{4.045542in}{2.529646in}}%
\pgfpathlineto{\pgfqpoint{4.100000in}{2.533687in}}%
\pgfpathlineto{\pgfqpoint{4.154458in}{2.535000in}}%
\pgfpathlineto{\pgfqpoint{4.208915in}{2.533584in}}%
\pgfpathlineto{\pgfqpoint{4.263373in}{2.529439in}}%
\pgfpathlineto{\pgfqpoint{4.317830in}{2.522565in}}%
\pgfpathlineto{\pgfqpoint{4.372288in}{2.512962in}}%
\pgfpathlineto{\pgfqpoint{4.426746in}{2.500630in}}%
\pgfpathlineto{\pgfqpoint{4.481203in}{2.485569in}}%
\pgfpathlineto{\pgfqpoint{4.535661in}{2.467779in}}%
\pgfpathlineto{\pgfqpoint{4.590119in}{2.447261in}}%
\pgfpathlineto{\pgfqpoint{4.644576in}{2.424013in}}%
\pgfpathlineto{\pgfqpoint{4.699034in}{2.398037in}}%
\pgfpathlineto{\pgfqpoint{4.753491in}{2.369331in}}%
\pgfpathlineto{\pgfqpoint{4.807949in}{2.337897in}}%
\pgfpathlineto{\pgfqpoint{4.862407in}{2.303734in}}%
\pgfpathlineto{\pgfqpoint{4.916864in}{2.266842in}}%
\pgfpathlineto{\pgfqpoint{4.971322in}{2.227221in}}%
\pgfpathlineto{\pgfqpoint{5.025780in}{2.184871in}}%
\pgfpathlineto{\pgfqpoint{5.080237in}{2.139792in}}%
\pgfpathlineto{\pgfqpoint{5.134695in}{2.091984in}}%
\pgfpathlineto{\pgfqpoint{5.189152in}{2.041448in}}%
\pgfpathlineto{\pgfqpoint{5.243610in}{1.988182in}}%
\pgfpathlineto{\pgfqpoint{5.298068in}{1.932188in}}%
\pgfpathlineto{\pgfqpoint{5.352525in}{1.873464in}}%
\pgfpathlineto{\pgfqpoint{5.406983in}{1.812012in}}%
\pgfpathlineto{\pgfqpoint{5.461440in}{1.747831in}}%
\pgfpathlineto{\pgfqpoint{5.529513in}{1.663767in}}%
\pgfpathlineto{\pgfqpoint{5.597585in}{1.575439in}}%
\pgfpathlineto{\pgfqpoint{5.665657in}{1.482847in}}%
\pgfpathlineto{\pgfqpoint{5.733729in}{1.385991in}}%
\pgfpathlineto{\pgfqpoint{5.801801in}{1.284871in}}%
\pgfpathlineto{\pgfqpoint{5.869873in}{1.179487in}}%
\pgfpathlineto{\pgfqpoint{5.937945in}{1.069840in}}%
\pgfpathlineto{\pgfqpoint{6.006017in}{0.955928in}}%
\pgfpathlineto{\pgfqpoint{6.074089in}{0.837753in}}%
\pgfpathlineto{\pgfqpoint{6.142161in}{0.715313in}}%
\pgfpathlineto{\pgfqpoint{6.210233in}{0.588610in}}%
\pgfpathlineto{\pgfqpoint{6.264690in}{0.484234in}}%
\pgfpathlineto{\pgfqpoint{6.319148in}{0.515730in}}%
\pgfpathlineto{\pgfqpoint{6.373606in}{0.544498in}}%
\pgfpathlineto{\pgfqpoint{6.428063in}{0.570536in}}%
\pgfpathlineto{\pgfqpoint{6.482521in}{0.593846in}}%
\pgfpathlineto{\pgfqpoint{6.536978in}{0.614426in}}%
\pgfpathlineto{\pgfqpoint{6.591436in}{0.632278in}}%
\pgfpathlineto{\pgfqpoint{6.645894in}{0.647400in}}%
\pgfpathlineto{\pgfqpoint{6.700351in}{0.659794in}}%
\pgfpathlineto{\pgfqpoint{6.754809in}{0.669459in}}%
\pgfpathlineto{\pgfqpoint{6.809267in}{0.676395in}}%
\pgfpathlineto{\pgfqpoint{6.863724in}{0.680602in}}%
\pgfpathlineto{\pgfqpoint{6.918182in}{0.682137in}}%
\pgfpathlineto{\pgfqpoint{6.918182in}{0.682137in}}%
\pgfusepath{stroke}%
\end{pgfscope}%
\begin{pgfscope}%
\pgfpathrectangle{\pgfqpoint{1.000000in}{0.330000in}}{\pgfqpoint{6.200000in}{2.310000in}}%
\pgfusepath{clip}%
\pgfsetrectcap%
\pgfsetroundjoin%
\pgfsetlinewidth{1.505625pt}%
\definecolor{currentstroke}{rgb}{0.737255,0.741176,0.133333}%
\pgfsetstrokecolor{currentstroke}%
\pgfsetdash{}{0pt}%
\pgfpathmoveto{\pgfqpoint{1.281818in}{0.682137in}}%
\pgfpathlineto{\pgfqpoint{1.281818in}{0.682137in}}%
\pgfpathlineto{\pgfqpoint{1.472420in}{0.681075in}}%
\pgfpathlineto{\pgfqpoint{1.663022in}{0.677836in}}%
\pgfpathlineto{\pgfqpoint{1.853623in}{0.672420in}}%
\pgfpathlineto{\pgfqpoint{2.016996in}{0.666045in}}%
\pgfpathlineto{\pgfqpoint{2.207598in}{0.689196in}}%
\pgfpathlineto{\pgfqpoint{2.398199in}{0.710170in}}%
\pgfpathlineto{\pgfqpoint{2.588801in}{0.728967in}}%
\pgfpathlineto{\pgfqpoint{2.779403in}{0.745588in}}%
\pgfpathlineto{\pgfqpoint{2.970004in}{0.760032in}}%
\pgfpathlineto{\pgfqpoint{3.160606in}{0.772299in}}%
\pgfpathlineto{\pgfqpoint{3.351208in}{0.782389in}}%
\pgfpathlineto{\pgfqpoint{3.541809in}{0.790303in}}%
\pgfpathlineto{\pgfqpoint{3.732411in}{0.796039in}}%
\pgfpathlineto{\pgfqpoint{3.923013in}{0.799600in}}%
\pgfpathlineto{\pgfqpoint{4.113614in}{0.800983in}}%
\pgfpathlineto{\pgfqpoint{4.304216in}{0.800190in}}%
\pgfpathlineto{\pgfqpoint{4.494818in}{0.797219in}}%
\pgfpathlineto{\pgfqpoint{4.685419in}{0.792073in}}%
\pgfpathlineto{\pgfqpoint{4.876021in}{0.784749in}}%
\pgfpathlineto{\pgfqpoint{5.066623in}{0.775249in}}%
\pgfpathlineto{\pgfqpoint{5.257224in}{0.763571in}}%
\pgfpathlineto{\pgfqpoint{5.447826in}{0.749718in}}%
\pgfpathlineto{\pgfqpoint{5.638428in}{0.733687in}}%
\pgfpathlineto{\pgfqpoint{5.829029in}{0.715480in}}%
\pgfpathlineto{\pgfqpoint{6.019631in}{0.695096in}}%
\pgfpathlineto{\pgfqpoint{6.210233in}{0.672535in}}%
\pgfpathlineto{\pgfqpoint{6.264690in}{0.665693in}}%
\pgfpathlineto{\pgfqpoint{6.441678in}{0.673369in}}%
\pgfpathlineto{\pgfqpoint{6.618665in}{0.678651in}}%
\pgfpathlineto{\pgfqpoint{6.795652in}{0.681537in}}%
\pgfpathlineto{\pgfqpoint{6.918182in}{0.682137in}}%
\pgfpathlineto{\pgfqpoint{6.918182in}{0.682137in}}%
\pgfusepath{stroke}%
\end{pgfscope}%
\begin{pgfscope}%
\pgfpathrectangle{\pgfqpoint{1.000000in}{0.330000in}}{\pgfqpoint{6.200000in}{2.310000in}}%
\pgfusepath{clip}%
\pgfsetrectcap%
\pgfsetroundjoin%
\pgfsetlinewidth{1.505625pt}%
\definecolor{currentstroke}{rgb}{0.090196,0.745098,0.811765}%
\pgfsetstrokecolor{currentstroke}%
\pgfsetdash{}{0pt}%
\pgfpathmoveto{\pgfqpoint{1.281818in}{0.682137in}}%
\pgfpathlineto{\pgfqpoint{1.281818in}{0.682137in}}%
\pgfpathlineto{\pgfqpoint{1.417962in}{0.681126in}}%
\pgfpathlineto{\pgfqpoint{1.554106in}{0.678023in}}%
\pgfpathlineto{\pgfqpoint{1.690250in}{0.672829in}}%
\pgfpathlineto{\pgfqpoint{1.826394in}{0.665542in}}%
\pgfpathlineto{\pgfqpoint{1.962538in}{0.656164in}}%
\pgfpathlineto{\pgfqpoint{2.016996in}{0.651827in}}%
\pgfpathlineto{\pgfqpoint{2.153140in}{0.683607in}}%
\pgfpathlineto{\pgfqpoint{2.289284in}{0.713295in}}%
\pgfpathlineto{\pgfqpoint{2.425428in}{0.740891in}}%
\pgfpathlineto{\pgfqpoint{2.561572in}{0.766395in}}%
\pgfpathlineto{\pgfqpoint{2.697716in}{0.789808in}}%
\pgfpathlineto{\pgfqpoint{2.833860in}{0.811128in}}%
\pgfpathlineto{\pgfqpoint{2.970004in}{0.830357in}}%
\pgfpathlineto{\pgfqpoint{3.106148in}{0.847494in}}%
\pgfpathlineto{\pgfqpoint{3.242292in}{0.862540in}}%
\pgfpathlineto{\pgfqpoint{3.378437in}{0.875493in}}%
\pgfpathlineto{\pgfqpoint{3.514581in}{0.886355in}}%
\pgfpathlineto{\pgfqpoint{3.650725in}{0.895124in}}%
\pgfpathlineto{\pgfqpoint{3.786869in}{0.901802in}}%
\pgfpathlineto{\pgfqpoint{3.923013in}{0.906389in}}%
\pgfpathlineto{\pgfqpoint{4.059157in}{0.908883in}}%
\pgfpathlineto{\pgfqpoint{4.195301in}{0.909285in}}%
\pgfpathlineto{\pgfqpoint{4.331445in}{0.907596in}}%
\pgfpathlineto{\pgfqpoint{4.467589in}{0.903815in}}%
\pgfpathlineto{\pgfqpoint{4.603733in}{0.897942in}}%
\pgfpathlineto{\pgfqpoint{4.739877in}{0.889977in}}%
\pgfpathlineto{\pgfqpoint{4.876021in}{0.879921in}}%
\pgfpathlineto{\pgfqpoint{5.012165in}{0.867772in}}%
\pgfpathlineto{\pgfqpoint{5.148309in}{0.853532in}}%
\pgfpathlineto{\pgfqpoint{5.284453in}{0.837200in}}%
\pgfpathlineto{\pgfqpoint{5.420597in}{0.818776in}}%
\pgfpathlineto{\pgfqpoint{5.556741in}{0.798260in}}%
\pgfpathlineto{\pgfqpoint{5.692885in}{0.775653in}}%
\pgfpathlineto{\pgfqpoint{5.829029in}{0.750953in}}%
\pgfpathlineto{\pgfqpoint{5.965173in}{0.724162in}}%
\pgfpathlineto{\pgfqpoint{6.101318in}{0.695279in}}%
\pgfpathlineto{\pgfqpoint{6.237462in}{0.664304in}}%
\pgfpathlineto{\pgfqpoint{6.264690in}{0.657865in}}%
\pgfpathlineto{\pgfqpoint{6.400834in}{0.666895in}}%
\pgfpathlineto{\pgfqpoint{6.536978in}{0.673833in}}%
\pgfpathlineto{\pgfqpoint{6.673123in}{0.678679in}}%
\pgfpathlineto{\pgfqpoint{6.809267in}{0.681433in}}%
\pgfpathlineto{\pgfqpoint{6.918182in}{0.682137in}}%
\pgfpathlineto{\pgfqpoint{6.918182in}{0.682137in}}%
\pgfusepath{stroke}%
\end{pgfscope}%
\begin{pgfscope}%
\pgfpathrectangle{\pgfqpoint{1.000000in}{0.330000in}}{\pgfqpoint{6.200000in}{2.310000in}}%
\pgfusepath{clip}%
\pgfsetrectcap%
\pgfsetroundjoin%
\pgfsetlinewidth{1.505625pt}%
\definecolor{currentstroke}{rgb}{0.121569,0.466667,0.705882}%
\pgfsetstrokecolor{currentstroke}%
\pgfsetdash{}{0pt}%
\pgfpathmoveto{\pgfqpoint{1.281818in}{0.682137in}}%
\pgfpathlineto{\pgfqpoint{1.281818in}{0.682137in}}%
\pgfpathlineto{\pgfqpoint{1.458805in}{0.681070in}}%
\pgfpathlineto{\pgfqpoint{1.635793in}{0.677814in}}%
\pgfpathlineto{\pgfqpoint{1.812780in}{0.672368in}}%
\pgfpathlineto{\pgfqpoint{1.989767in}{0.664732in}}%
\pgfpathlineto{\pgfqpoint{2.016996in}{0.663363in}}%
\pgfpathlineto{\pgfqpoint{2.193983in}{0.688700in}}%
\pgfpathlineto{\pgfqpoint{2.370971in}{0.711848in}}%
\pgfpathlineto{\pgfqpoint{2.547958in}{0.732805in}}%
\pgfpathlineto{\pgfqpoint{2.724945in}{0.751574in}}%
\pgfpathlineto{\pgfqpoint{2.901932in}{0.768152in}}%
\pgfpathlineto{\pgfqpoint{3.078920in}{0.782541in}}%
\pgfpathlineto{\pgfqpoint{3.255907in}{0.794740in}}%
\pgfpathlineto{\pgfqpoint{3.432894in}{0.804749in}}%
\pgfpathlineto{\pgfqpoint{3.609881in}{0.812569in}}%
\pgfpathlineto{\pgfqpoint{3.786869in}{0.818199in}}%
\pgfpathlineto{\pgfqpoint{3.963856in}{0.821639in}}%
\pgfpathlineto{\pgfqpoint{4.140843in}{0.822889in}}%
\pgfpathlineto{\pgfqpoint{4.317830in}{0.821950in}}%
\pgfpathlineto{\pgfqpoint{4.494818in}{0.818821in}}%
\pgfpathlineto{\pgfqpoint{4.671805in}{0.813503in}}%
\pgfpathlineto{\pgfqpoint{4.848792in}{0.805994in}}%
\pgfpathlineto{\pgfqpoint{5.025780in}{0.796296in}}%
\pgfpathlineto{\pgfqpoint{5.202767in}{0.784409in}}%
\pgfpathlineto{\pgfqpoint{5.379754in}{0.770331in}}%
\pgfpathlineto{\pgfqpoint{5.556741in}{0.754064in}}%
\pgfpathlineto{\pgfqpoint{5.733729in}{0.735607in}}%
\pgfpathlineto{\pgfqpoint{5.910716in}{0.714961in}}%
\pgfpathlineto{\pgfqpoint{6.087703in}{0.692125in}}%
\pgfpathlineto{\pgfqpoint{6.264690in}{0.667103in}}%
\pgfpathlineto{\pgfqpoint{6.441678in}{0.674121in}}%
\pgfpathlineto{\pgfqpoint{6.618665in}{0.678950in}}%
\pgfpathlineto{\pgfqpoint{6.795652in}{0.681589in}}%
\pgfpathlineto{\pgfqpoint{6.918182in}{0.682137in}}%
\pgfpathlineto{\pgfqpoint{6.918182in}{0.682137in}}%
\pgfusepath{stroke}%
\end{pgfscope}%
\begin{pgfscope}%
\pgfpathrectangle{\pgfqpoint{1.000000in}{0.330000in}}{\pgfqpoint{6.200000in}{2.310000in}}%
\pgfusepath{clip}%
\pgfsetrectcap%
\pgfsetroundjoin%
\pgfsetlinewidth{1.505625pt}%
\definecolor{currentstroke}{rgb}{1.000000,0.498039,0.054902}%
\pgfsetstrokecolor{currentstroke}%
\pgfsetdash{}{0pt}%
\pgfpathmoveto{\pgfqpoint{1.281818in}{0.682137in}}%
\pgfpathlineto{\pgfqpoint{1.281818in}{0.682137in}}%
\pgfpathlineto{\pgfqpoint{1.458805in}{0.680970in}}%
\pgfpathlineto{\pgfqpoint{1.635793in}{0.677408in}}%
\pgfpathlineto{\pgfqpoint{1.812780in}{0.671451in}}%
\pgfpathlineto{\pgfqpoint{1.989767in}{0.663099in}}%
\pgfpathlineto{\pgfqpoint{2.016996in}{0.661602in}}%
\pgfpathlineto{\pgfqpoint{2.207598in}{0.685112in}}%
\pgfpathlineto{\pgfqpoint{2.398199in}{0.706445in}}%
\pgfpathlineto{\pgfqpoint{2.588801in}{0.725601in}}%
\pgfpathlineto{\pgfqpoint{2.779403in}{0.742581in}}%
\pgfpathlineto{\pgfqpoint{2.970004in}{0.757383in}}%
\pgfpathlineto{\pgfqpoint{3.160606in}{0.770010in}}%
\pgfpathlineto{\pgfqpoint{3.351208in}{0.780459in}}%
\pgfpathlineto{\pgfqpoint{3.541809in}{0.788732in}}%
\pgfpathlineto{\pgfqpoint{3.732411in}{0.794827in}}%
\pgfpathlineto{\pgfqpoint{3.923013in}{0.798747in}}%
\pgfpathlineto{\pgfqpoint{4.113614in}{0.800489in}}%
\pgfpathlineto{\pgfqpoint{4.304216in}{0.800055in}}%
\pgfpathlineto{\pgfqpoint{4.494818in}{0.797444in}}%
\pgfpathlineto{\pgfqpoint{4.685419in}{0.792656in}}%
\pgfpathlineto{\pgfqpoint{4.876021in}{0.785691in}}%
\pgfpathlineto{\pgfqpoint{5.066623in}{0.776550in}}%
\pgfpathlineto{\pgfqpoint{5.257224in}{0.765232in}}%
\pgfpathlineto{\pgfqpoint{5.447826in}{0.751737in}}%
\pgfpathlineto{\pgfqpoint{5.638428in}{0.736065in}}%
\pgfpathlineto{\pgfqpoint{5.829029in}{0.718217in}}%
\pgfpathlineto{\pgfqpoint{6.019631in}{0.698192in}}%
\pgfpathlineto{\pgfqpoint{6.210233in}{0.675990in}}%
\pgfpathlineto{\pgfqpoint{6.264690in}{0.669251in}}%
\pgfpathlineto{\pgfqpoint{6.455292in}{0.675651in}}%
\pgfpathlineto{\pgfqpoint{6.645894in}{0.679875in}}%
\pgfpathlineto{\pgfqpoint{6.836495in}{0.681923in}}%
\pgfpathlineto{\pgfqpoint{6.918182in}{0.682137in}}%
\pgfpathlineto{\pgfqpoint{6.918182in}{0.682137in}}%
\pgfusepath{stroke}%
\end{pgfscope}%
\begin{pgfscope}%
\pgfpathrectangle{\pgfqpoint{1.000000in}{0.330000in}}{\pgfqpoint{6.200000in}{2.310000in}}%
\pgfusepath{clip}%
\pgfsetrectcap%
\pgfsetroundjoin%
\pgfsetlinewidth{1.505625pt}%
\definecolor{currentstroke}{rgb}{0.172549,0.627451,0.172549}%
\pgfsetstrokecolor{currentstroke}%
\pgfsetdash{}{0pt}%
\pgfpathmoveto{\pgfqpoint{1.281818in}{0.682137in}}%
\pgfpathlineto{\pgfqpoint{1.281818in}{0.682137in}}%
\pgfpathlineto{\pgfqpoint{1.540492in}{0.680988in}}%
\pgfpathlineto{\pgfqpoint{1.799166in}{0.677501in}}%
\pgfpathlineto{\pgfqpoint{2.030610in}{0.673763in}}%
\pgfpathlineto{\pgfqpoint{2.289284in}{0.691787in}}%
\pgfpathlineto{\pgfqpoint{2.547958in}{0.707471in}}%
\pgfpathlineto{\pgfqpoint{2.806632in}{0.720817in}}%
\pgfpathlineto{\pgfqpoint{3.065305in}{0.731824in}}%
\pgfpathlineto{\pgfqpoint{3.323979in}{0.740493in}}%
\pgfpathlineto{\pgfqpoint{3.582653in}{0.746823in}}%
\pgfpathlineto{\pgfqpoint{3.841326in}{0.750814in}}%
\pgfpathlineto{\pgfqpoint{4.100000in}{0.752466in}}%
\pgfpathlineto{\pgfqpoint{4.358674in}{0.751780in}}%
\pgfpathlineto{\pgfqpoint{4.617347in}{0.748755in}}%
\pgfpathlineto{\pgfqpoint{4.876021in}{0.743391in}}%
\pgfpathlineto{\pgfqpoint{5.134695in}{0.735689in}}%
\pgfpathlineto{\pgfqpoint{5.393368in}{0.725647in}}%
\pgfpathlineto{\pgfqpoint{5.652042in}{0.713268in}}%
\pgfpathlineto{\pgfqpoint{5.910716in}{0.698549in}}%
\pgfpathlineto{\pgfqpoint{6.169390in}{0.681492in}}%
\pgfpathlineto{\pgfqpoint{6.264690in}{0.674620in}}%
\pgfpathlineto{\pgfqpoint{6.523364in}{0.679380in}}%
\pgfpathlineto{\pgfqpoint{6.782038in}{0.681800in}}%
\pgfpathlineto{\pgfqpoint{6.918182in}{0.682137in}}%
\pgfpathlineto{\pgfqpoint{6.918182in}{0.682137in}}%
\pgfusepath{stroke}%
\end{pgfscope}%
\begin{pgfscope}%
\pgfpathrectangle{\pgfqpoint{1.000000in}{0.330000in}}{\pgfqpoint{6.200000in}{2.310000in}}%
\pgfusepath{clip}%
\pgfsetrectcap%
\pgfsetroundjoin%
\pgfsetlinewidth{1.505625pt}%
\definecolor{currentstroke}{rgb}{0.839216,0.152941,0.156863}%
\pgfsetstrokecolor{currentstroke}%
\pgfsetdash{}{0pt}%
\pgfpathmoveto{\pgfqpoint{1.281818in}{0.682137in}}%
\pgfpathlineto{\pgfqpoint{1.281818in}{0.682137in}}%
\pgfpathlineto{\pgfqpoint{1.472420in}{0.681075in}}%
\pgfpathlineto{\pgfqpoint{1.663022in}{0.677836in}}%
\pgfpathlineto{\pgfqpoint{1.853623in}{0.672420in}}%
\pgfpathlineto{\pgfqpoint{2.016996in}{0.666045in}}%
\pgfpathlineto{\pgfqpoint{2.153140in}{0.697734in}}%
\pgfpathlineto{\pgfqpoint{2.289284in}{0.727331in}}%
\pgfpathlineto{\pgfqpoint{2.425428in}{0.754837in}}%
\pgfpathlineto{\pgfqpoint{2.561572in}{0.780250in}}%
\pgfpathlineto{\pgfqpoint{2.697716in}{0.803572in}}%
\pgfpathlineto{\pgfqpoint{2.833860in}{0.824802in}}%
\pgfpathlineto{\pgfqpoint{2.970004in}{0.843940in}}%
\pgfpathlineto{\pgfqpoint{3.106148in}{0.860986in}}%
\pgfpathlineto{\pgfqpoint{3.242292in}{0.875941in}}%
\pgfpathlineto{\pgfqpoint{3.378437in}{0.888803in}}%
\pgfpathlineto{\pgfqpoint{3.514581in}{0.899574in}}%
\pgfpathlineto{\pgfqpoint{3.650725in}{0.908253in}}%
\pgfpathlineto{\pgfqpoint{3.786869in}{0.914840in}}%
\pgfpathlineto{\pgfqpoint{3.923013in}{0.919336in}}%
\pgfpathlineto{\pgfqpoint{4.059157in}{0.921739in}}%
\pgfpathlineto{\pgfqpoint{4.195301in}{0.922051in}}%
\pgfpathlineto{\pgfqpoint{4.331445in}{0.920271in}}%
\pgfpathlineto{\pgfqpoint{4.467589in}{0.916399in}}%
\pgfpathlineto{\pgfqpoint{4.603733in}{0.910435in}}%
\pgfpathlineto{\pgfqpoint{4.739877in}{0.902380in}}%
\pgfpathlineto{\pgfqpoint{4.876021in}{0.892232in}}%
\pgfpathlineto{\pgfqpoint{5.012165in}{0.879993in}}%
\pgfpathlineto{\pgfqpoint{5.148309in}{0.865662in}}%
\pgfpathlineto{\pgfqpoint{5.284453in}{0.849239in}}%
\pgfpathlineto{\pgfqpoint{5.420597in}{0.830724in}}%
\pgfpathlineto{\pgfqpoint{5.556741in}{0.810118in}}%
\pgfpathlineto{\pgfqpoint{5.692885in}{0.787420in}}%
\pgfpathlineto{\pgfqpoint{5.829029in}{0.762629in}}%
\pgfpathlineto{\pgfqpoint{5.965173in}{0.735747in}}%
\pgfpathlineto{\pgfqpoint{6.101318in}{0.706774in}}%
\pgfpathlineto{\pgfqpoint{6.237462in}{0.675708in}}%
\pgfpathlineto{\pgfqpoint{6.264690in}{0.669251in}}%
\pgfpathlineto{\pgfqpoint{6.455292in}{0.675651in}}%
\pgfpathlineto{\pgfqpoint{6.645894in}{0.679875in}}%
\pgfpathlineto{\pgfqpoint{6.836495in}{0.681923in}}%
\pgfpathlineto{\pgfqpoint{6.918182in}{0.682137in}}%
\pgfpathlineto{\pgfqpoint{6.918182in}{0.682137in}}%
\pgfusepath{stroke}%
\end{pgfscope}%
\begin{pgfscope}%
\pgfpathrectangle{\pgfqpoint{1.000000in}{0.330000in}}{\pgfqpoint{6.200000in}{2.310000in}}%
\pgfusepath{clip}%
\pgfsetrectcap%
\pgfsetroundjoin%
\pgfsetlinewidth{1.505625pt}%
\definecolor{currentstroke}{rgb}{0.580392,0.403922,0.741176}%
\pgfsetstrokecolor{currentstroke}%
\pgfsetdash{}{0pt}%
\pgfpathmoveto{\pgfqpoint{1.281818in}{0.682137in}}%
\pgfpathlineto{\pgfqpoint{1.281818in}{0.682137in}}%
\pgfpathlineto{\pgfqpoint{1.417962in}{0.681126in}}%
\pgfpathlineto{\pgfqpoint{1.554106in}{0.678023in}}%
\pgfpathlineto{\pgfqpoint{1.690250in}{0.672829in}}%
\pgfpathlineto{\pgfqpoint{1.826394in}{0.665542in}}%
\pgfpathlineto{\pgfqpoint{1.962538in}{0.656164in}}%
\pgfpathlineto{\pgfqpoint{2.016996in}{0.651827in}}%
\pgfpathlineto{\pgfqpoint{2.207598in}{0.675775in}}%
\pgfpathlineto{\pgfqpoint{2.398199in}{0.697547in}}%
\pgfpathlineto{\pgfqpoint{2.588801in}{0.717142in}}%
\pgfpathlineto{\pgfqpoint{2.779403in}{0.734560in}}%
\pgfpathlineto{\pgfqpoint{2.970004in}{0.749802in}}%
\pgfpathlineto{\pgfqpoint{3.160606in}{0.762866in}}%
\pgfpathlineto{\pgfqpoint{3.351208in}{0.773754in}}%
\pgfpathlineto{\pgfqpoint{3.541809in}{0.782466in}}%
\pgfpathlineto{\pgfqpoint{3.732411in}{0.789000in}}%
\pgfpathlineto{\pgfqpoint{3.923013in}{0.793358in}}%
\pgfpathlineto{\pgfqpoint{4.113614in}{0.795539in}}%
\pgfpathlineto{\pgfqpoint{4.304216in}{0.795543in}}%
\pgfpathlineto{\pgfqpoint{4.494818in}{0.793371in}}%
\pgfpathlineto{\pgfqpoint{4.685419in}{0.789021in}}%
\pgfpathlineto{\pgfqpoint{4.876021in}{0.782495in}}%
\pgfpathlineto{\pgfqpoint{5.066623in}{0.773793in}}%
\pgfpathlineto{\pgfqpoint{5.257224in}{0.762913in}}%
\pgfpathlineto{\pgfqpoint{5.447826in}{0.749857in}}%
\pgfpathlineto{\pgfqpoint{5.638428in}{0.734624in}}%
\pgfpathlineto{\pgfqpoint{5.829029in}{0.717215in}}%
\pgfpathlineto{\pgfqpoint{6.019631in}{0.697628in}}%
\pgfpathlineto{\pgfqpoint{6.210233in}{0.675865in}}%
\pgfpathlineto{\pgfqpoint{6.264690in}{0.669251in}}%
\pgfpathlineto{\pgfqpoint{6.455292in}{0.675651in}}%
\pgfpathlineto{\pgfqpoint{6.645894in}{0.679875in}}%
\pgfpathlineto{\pgfqpoint{6.836495in}{0.681923in}}%
\pgfpathlineto{\pgfqpoint{6.918182in}{0.682137in}}%
\pgfpathlineto{\pgfqpoint{6.918182in}{0.682137in}}%
\pgfusepath{stroke}%
\end{pgfscope}%
\begin{pgfscope}%
\pgfpathrectangle{\pgfqpoint{1.000000in}{0.330000in}}{\pgfqpoint{6.200000in}{2.310000in}}%
\pgfusepath{clip}%
\pgfsetrectcap%
\pgfsetroundjoin%
\pgfsetlinewidth{1.505625pt}%
\definecolor{currentstroke}{rgb}{0.549020,0.337255,0.294118}%
\pgfsetstrokecolor{currentstroke}%
\pgfsetdash{}{0pt}%
\pgfpathmoveto{\pgfqpoint{1.281818in}{0.682137in}}%
\pgfpathlineto{\pgfqpoint{1.281818in}{0.682137in}}%
\pgfpathlineto{\pgfqpoint{1.472420in}{0.681075in}}%
\pgfpathlineto{\pgfqpoint{1.663022in}{0.677836in}}%
\pgfpathlineto{\pgfqpoint{1.853623in}{0.672420in}}%
\pgfpathlineto{\pgfqpoint{2.016996in}{0.666045in}}%
\pgfpathlineto{\pgfqpoint{2.153140in}{0.697369in}}%
\pgfpathlineto{\pgfqpoint{2.289284in}{0.726601in}}%
\pgfpathlineto{\pgfqpoint{2.425428in}{0.753742in}}%
\pgfpathlineto{\pgfqpoint{2.561572in}{0.778790in}}%
\pgfpathlineto{\pgfqpoint{2.697716in}{0.801747in}}%
\pgfpathlineto{\pgfqpoint{2.833860in}{0.822612in}}%
\pgfpathlineto{\pgfqpoint{2.970004in}{0.841385in}}%
\pgfpathlineto{\pgfqpoint{3.106148in}{0.858067in}}%
\pgfpathlineto{\pgfqpoint{3.242292in}{0.872656in}}%
\pgfpathlineto{\pgfqpoint{3.378437in}{0.885154in}}%
\pgfpathlineto{\pgfqpoint{3.514581in}{0.895560in}}%
\pgfpathlineto{\pgfqpoint{3.650725in}{0.903874in}}%
\pgfpathlineto{\pgfqpoint{3.786869in}{0.910096in}}%
\pgfpathlineto{\pgfqpoint{3.923013in}{0.914227in}}%
\pgfpathlineto{\pgfqpoint{4.059157in}{0.916265in}}%
\pgfpathlineto{\pgfqpoint{4.195301in}{0.916212in}}%
\pgfpathlineto{\pgfqpoint{4.331445in}{0.914067in}}%
\pgfpathlineto{\pgfqpoint{4.467589in}{0.909830in}}%
\pgfpathlineto{\pgfqpoint{4.603733in}{0.903502in}}%
\pgfpathlineto{\pgfqpoint{4.739877in}{0.895081in}}%
\pgfpathlineto{\pgfqpoint{4.876021in}{0.884569in}}%
\pgfpathlineto{\pgfqpoint{5.012165in}{0.871965in}}%
\pgfpathlineto{\pgfqpoint{5.148309in}{0.857269in}}%
\pgfpathlineto{\pgfqpoint{5.284453in}{0.840481in}}%
\pgfpathlineto{\pgfqpoint{5.420597in}{0.821601in}}%
\pgfpathlineto{\pgfqpoint{5.556741in}{0.800630in}}%
\pgfpathlineto{\pgfqpoint{5.692885in}{0.777567in}}%
\pgfpathlineto{\pgfqpoint{5.829029in}{0.752411in}}%
\pgfpathlineto{\pgfqpoint{5.965173in}{0.725165in}}%
\pgfpathlineto{\pgfqpoint{6.101318in}{0.695826in}}%
\pgfpathlineto{\pgfqpoint{6.237462in}{0.664395in}}%
\pgfpathlineto{\pgfqpoint{6.264690in}{0.657865in}}%
\pgfpathlineto{\pgfqpoint{6.400834in}{0.666895in}}%
\pgfpathlineto{\pgfqpoint{6.536978in}{0.673833in}}%
\pgfpathlineto{\pgfqpoint{6.673123in}{0.678679in}}%
\pgfpathlineto{\pgfqpoint{6.809267in}{0.681433in}}%
\pgfpathlineto{\pgfqpoint{6.918182in}{0.682137in}}%
\pgfpathlineto{\pgfqpoint{6.918182in}{0.682137in}}%
\pgfusepath{stroke}%
\end{pgfscope}%
\begin{pgfscope}%
\pgfpathrectangle{\pgfqpoint{1.000000in}{0.330000in}}{\pgfqpoint{6.200000in}{2.310000in}}%
\pgfusepath{clip}%
\pgfsetrectcap%
\pgfsetroundjoin%
\pgfsetlinewidth{1.505625pt}%
\definecolor{currentstroke}{rgb}{0.890196,0.466667,0.760784}%
\pgfsetstrokecolor{currentstroke}%
\pgfsetdash{}{0pt}%
\pgfpathmoveto{\pgfqpoint{1.281818in}{0.682137in}}%
\pgfpathlineto{\pgfqpoint{1.281818in}{0.682137in}}%
\pgfpathlineto{\pgfqpoint{1.417962in}{0.681126in}}%
\pgfpathlineto{\pgfqpoint{1.554106in}{0.678023in}}%
\pgfpathlineto{\pgfqpoint{1.690250in}{0.672829in}}%
\pgfpathlineto{\pgfqpoint{1.826394in}{0.665542in}}%
\pgfpathlineto{\pgfqpoint{1.962538in}{0.656164in}}%
\pgfpathlineto{\pgfqpoint{2.016996in}{0.651827in}}%
\pgfpathlineto{\pgfqpoint{2.153140in}{0.683972in}}%
\pgfpathlineto{\pgfqpoint{2.289284in}{0.714024in}}%
\pgfpathlineto{\pgfqpoint{2.425428in}{0.741986in}}%
\pgfpathlineto{\pgfqpoint{2.561572in}{0.767855in}}%
\pgfpathlineto{\pgfqpoint{2.697716in}{0.791632in}}%
\pgfpathlineto{\pgfqpoint{2.833860in}{0.813318in}}%
\pgfpathlineto{\pgfqpoint{2.970004in}{0.832912in}}%
\pgfpathlineto{\pgfqpoint{3.106148in}{0.850414in}}%
\pgfpathlineto{\pgfqpoint{3.242292in}{0.865824in}}%
\pgfpathlineto{\pgfqpoint{3.378437in}{0.879142in}}%
\pgfpathlineto{\pgfqpoint{3.514581in}{0.890369in}}%
\pgfpathlineto{\pgfqpoint{3.650725in}{0.899504in}}%
\pgfpathlineto{\pgfqpoint{3.786869in}{0.906546in}}%
\pgfpathlineto{\pgfqpoint{3.923013in}{0.911498in}}%
\pgfpathlineto{\pgfqpoint{4.059157in}{0.914357in}}%
\pgfpathlineto{\pgfqpoint{4.195301in}{0.915124in}}%
\pgfpathlineto{\pgfqpoint{4.331445in}{0.913800in}}%
\pgfpathlineto{\pgfqpoint{4.467589in}{0.910384in}}%
\pgfpathlineto{\pgfqpoint{4.603733in}{0.904876in}}%
\pgfpathlineto{\pgfqpoint{4.739877in}{0.897276in}}%
\pgfpathlineto{\pgfqpoint{4.876021in}{0.887584in}}%
\pgfpathlineto{\pgfqpoint{5.012165in}{0.875800in}}%
\pgfpathlineto{\pgfqpoint{5.148309in}{0.861925in}}%
\pgfpathlineto{\pgfqpoint{5.284453in}{0.845958in}}%
\pgfpathlineto{\pgfqpoint{5.420597in}{0.827899in}}%
\pgfpathlineto{\pgfqpoint{5.556741in}{0.807748in}}%
\pgfpathlineto{\pgfqpoint{5.692885in}{0.785506in}}%
\pgfpathlineto{\pgfqpoint{5.829029in}{0.761171in}}%
\pgfpathlineto{\pgfqpoint{5.965173in}{0.734745in}}%
\pgfpathlineto{\pgfqpoint{6.101318in}{0.706227in}}%
\pgfpathlineto{\pgfqpoint{6.237462in}{0.675617in}}%
\pgfpathlineto{\pgfqpoint{6.264690in}{0.669251in}}%
\pgfpathlineto{\pgfqpoint{6.455292in}{0.675651in}}%
\pgfpathlineto{\pgfqpoint{6.645894in}{0.679875in}}%
\pgfpathlineto{\pgfqpoint{6.836495in}{0.681923in}}%
\pgfpathlineto{\pgfqpoint{6.918182in}{0.682137in}}%
\pgfpathlineto{\pgfqpoint{6.918182in}{0.682137in}}%
\pgfusepath{stroke}%
\end{pgfscope}%
\begin{pgfscope}%
\pgfpathrectangle{\pgfqpoint{1.000000in}{0.330000in}}{\pgfqpoint{6.200000in}{2.310000in}}%
\pgfusepath{clip}%
\pgfsetrectcap%
\pgfsetroundjoin%
\pgfsetlinewidth{1.505625pt}%
\definecolor{currentstroke}{rgb}{0.498039,0.498039,0.498039}%
\pgfsetstrokecolor{currentstroke}%
\pgfsetdash{}{0pt}%
\pgfpathmoveto{\pgfqpoint{1.281818in}{0.682137in}}%
\pgfpathlineto{\pgfqpoint{1.281818in}{0.682137in}}%
\pgfpathlineto{\pgfqpoint{1.458805in}{0.680970in}}%
\pgfpathlineto{\pgfqpoint{1.635793in}{0.677408in}}%
\pgfpathlineto{\pgfqpoint{1.812780in}{0.671451in}}%
\pgfpathlineto{\pgfqpoint{1.989767in}{0.663099in}}%
\pgfpathlineto{\pgfqpoint{2.016996in}{0.661602in}}%
\pgfpathlineto{\pgfqpoint{2.207598in}{0.684952in}}%
\pgfpathlineto{\pgfqpoint{2.398199in}{0.706125in}}%
\pgfpathlineto{\pgfqpoint{2.588801in}{0.725122in}}%
\pgfpathlineto{\pgfqpoint{2.779403in}{0.741942in}}%
\pgfpathlineto{\pgfqpoint{2.970004in}{0.756585in}}%
\pgfpathlineto{\pgfqpoint{3.160606in}{0.769052in}}%
\pgfpathlineto{\pgfqpoint{3.351208in}{0.779341in}}%
\pgfpathlineto{\pgfqpoint{3.541809in}{0.787454in}}%
\pgfpathlineto{\pgfqpoint{3.732411in}{0.793391in}}%
\pgfpathlineto{\pgfqpoint{3.923013in}{0.797150in}}%
\pgfpathlineto{\pgfqpoint{4.113614in}{0.798733in}}%
\pgfpathlineto{\pgfqpoint{4.304216in}{0.798139in}}%
\pgfpathlineto{\pgfqpoint{4.494818in}{0.795368in}}%
\pgfpathlineto{\pgfqpoint{4.685419in}{0.790421in}}%
\pgfpathlineto{\pgfqpoint{4.876021in}{0.783296in}}%
\pgfpathlineto{\pgfqpoint{5.066623in}{0.773995in}}%
\pgfpathlineto{\pgfqpoint{5.257224in}{0.762518in}}%
\pgfpathlineto{\pgfqpoint{5.447826in}{0.748863in}}%
\pgfpathlineto{\pgfqpoint{5.638428in}{0.733032in}}%
\pgfpathlineto{\pgfqpoint{5.829029in}{0.715024in}}%
\pgfpathlineto{\pgfqpoint{6.019631in}{0.694839in}}%
\pgfpathlineto{\pgfqpoint{6.210233in}{0.672478in}}%
\pgfpathlineto{\pgfqpoint{6.264690in}{0.665693in}}%
\pgfpathlineto{\pgfqpoint{6.441678in}{0.673369in}}%
\pgfpathlineto{\pgfqpoint{6.618665in}{0.678651in}}%
\pgfpathlineto{\pgfqpoint{6.795652in}{0.681537in}}%
\pgfpathlineto{\pgfqpoint{6.918182in}{0.682137in}}%
\pgfpathlineto{\pgfqpoint{6.918182in}{0.682137in}}%
\pgfusepath{stroke}%
\end{pgfscope}%
\begin{pgfscope}%
\pgfpathrectangle{\pgfqpoint{1.000000in}{0.330000in}}{\pgfqpoint{6.200000in}{2.310000in}}%
\pgfusepath{clip}%
\pgfsetrectcap%
\pgfsetroundjoin%
\pgfsetlinewidth{1.505625pt}%
\definecolor{currentstroke}{rgb}{0.737255,0.741176,0.133333}%
\pgfsetstrokecolor{currentstroke}%
\pgfsetdash{}{0pt}%
\pgfpathmoveto{\pgfqpoint{1.281818in}{0.682137in}}%
\pgfpathlineto{\pgfqpoint{1.281818in}{0.682137in}}%
\pgfpathlineto{\pgfqpoint{1.472420in}{0.681075in}}%
\pgfpathlineto{\pgfqpoint{1.663022in}{0.677836in}}%
\pgfpathlineto{\pgfqpoint{1.853623in}{0.672420in}}%
\pgfpathlineto{\pgfqpoint{2.016996in}{0.666045in}}%
\pgfpathlineto{\pgfqpoint{2.207598in}{0.689355in}}%
\pgfpathlineto{\pgfqpoint{2.398199in}{0.710489in}}%
\pgfpathlineto{\pgfqpoint{2.588801in}{0.729446in}}%
\pgfpathlineto{\pgfqpoint{2.779403in}{0.746226in}}%
\pgfpathlineto{\pgfqpoint{2.970004in}{0.760830in}}%
\pgfpathlineto{\pgfqpoint{3.160606in}{0.773257in}}%
\pgfpathlineto{\pgfqpoint{3.351208in}{0.783507in}}%
\pgfpathlineto{\pgfqpoint{3.541809in}{0.791580in}}%
\pgfpathlineto{\pgfqpoint{3.732411in}{0.797476in}}%
\pgfpathlineto{\pgfqpoint{3.923013in}{0.801196in}}%
\pgfpathlineto{\pgfqpoint{4.113614in}{0.802739in}}%
\pgfpathlineto{\pgfqpoint{4.304216in}{0.802105in}}%
\pgfpathlineto{\pgfqpoint{4.494818in}{0.799295in}}%
\pgfpathlineto{\pgfqpoint{4.685419in}{0.794308in}}%
\pgfpathlineto{\pgfqpoint{4.876021in}{0.787144in}}%
\pgfpathlineto{\pgfqpoint{5.066623in}{0.777803in}}%
\pgfpathlineto{\pgfqpoint{5.257224in}{0.766286in}}%
\pgfpathlineto{\pgfqpoint{5.447826in}{0.752591in}}%
\pgfpathlineto{\pgfqpoint{5.638428in}{0.736720in}}%
\pgfpathlineto{\pgfqpoint{5.829029in}{0.718673in}}%
\pgfpathlineto{\pgfqpoint{6.019631in}{0.698448in}}%
\pgfpathlineto{\pgfqpoint{6.210233in}{0.676047in}}%
\pgfpathlineto{\pgfqpoint{6.264690in}{0.669251in}}%
\pgfpathlineto{\pgfqpoint{6.455292in}{0.675651in}}%
\pgfpathlineto{\pgfqpoint{6.645894in}{0.679875in}}%
\pgfpathlineto{\pgfqpoint{6.836495in}{0.681923in}}%
\pgfpathlineto{\pgfqpoint{6.918182in}{0.682137in}}%
\pgfpathlineto{\pgfqpoint{6.918182in}{0.682137in}}%
\pgfusepath{stroke}%
\end{pgfscope}%
\begin{pgfscope}%
\pgfpathrectangle{\pgfqpoint{1.000000in}{0.330000in}}{\pgfqpoint{6.200000in}{2.310000in}}%
\pgfusepath{clip}%
\pgfsetrectcap%
\pgfsetroundjoin%
\pgfsetlinewidth{1.505625pt}%
\definecolor{currentstroke}{rgb}{0.090196,0.745098,0.811765}%
\pgfsetstrokecolor{currentstroke}%
\pgfsetdash{}{0pt}%
\pgfpathmoveto{\pgfqpoint{1.281818in}{0.682137in}}%
\pgfpathlineto{\pgfqpoint{1.281818in}{0.682137in}}%
\pgfpathlineto{\pgfqpoint{1.499649in}{0.681093in}}%
\pgfpathlineto{\pgfqpoint{1.717479in}{0.677917in}}%
\pgfpathlineto{\pgfqpoint{1.935310in}{0.672608in}}%
\pgfpathlineto{\pgfqpoint{2.016996in}{0.670068in}}%
\pgfpathlineto{\pgfqpoint{2.234827in}{0.689915in}}%
\pgfpathlineto{\pgfqpoint{2.452657in}{0.707630in}}%
\pgfpathlineto{\pgfqpoint{2.670487in}{0.723212in}}%
\pgfpathlineto{\pgfqpoint{2.888318in}{0.736663in}}%
\pgfpathlineto{\pgfqpoint{3.106148in}{0.747980in}}%
\pgfpathlineto{\pgfqpoint{3.323979in}{0.757166in}}%
\pgfpathlineto{\pgfqpoint{3.541809in}{0.764219in}}%
\pgfpathlineto{\pgfqpoint{3.759640in}{0.769140in}}%
\pgfpathlineto{\pgfqpoint{3.977470in}{0.771929in}}%
\pgfpathlineto{\pgfqpoint{4.195301in}{0.772585in}}%
\pgfpathlineto{\pgfqpoint{4.413131in}{0.771109in}}%
\pgfpathlineto{\pgfqpoint{4.630962in}{0.767500in}}%
\pgfpathlineto{\pgfqpoint{4.848792in}{0.761760in}}%
\pgfpathlineto{\pgfqpoint{5.066623in}{0.753887in}}%
\pgfpathlineto{\pgfqpoint{5.284453in}{0.743881in}}%
\pgfpathlineto{\pgfqpoint{5.502284in}{0.731744in}}%
\pgfpathlineto{\pgfqpoint{5.720114in}{0.717474in}}%
\pgfpathlineto{\pgfqpoint{5.937945in}{0.701071in}}%
\pgfpathlineto{\pgfqpoint{6.155775in}{0.682537in}}%
\pgfpathlineto{\pgfqpoint{6.264690in}{0.672472in}}%
\pgfpathlineto{\pgfqpoint{6.482521in}{0.677825in}}%
\pgfpathlineto{\pgfqpoint{6.700351in}{0.681046in}}%
\pgfpathlineto{\pgfqpoint{6.918182in}{0.682137in}}%
\pgfpathlineto{\pgfqpoint{6.918182in}{0.682137in}}%
\pgfusepath{stroke}%
\end{pgfscope}%
\begin{pgfscope}%
\pgfpathrectangle{\pgfqpoint{1.000000in}{0.330000in}}{\pgfqpoint{6.200000in}{2.310000in}}%
\pgfusepath{clip}%
\pgfsetrectcap%
\pgfsetroundjoin%
\pgfsetlinewidth{1.505625pt}%
\definecolor{currentstroke}{rgb}{0.121569,0.466667,0.705882}%
\pgfsetstrokecolor{currentstroke}%
\pgfsetdash{}{0pt}%
\pgfpathmoveto{\pgfqpoint{1.281818in}{0.682137in}}%
\pgfpathlineto{\pgfqpoint{1.281818in}{0.682137in}}%
\pgfpathlineto{\pgfqpoint{1.363505in}{0.681101in}}%
\pgfpathlineto{\pgfqpoint{1.445191in}{0.677872in}}%
\pgfpathlineto{\pgfqpoint{1.526877in}{0.672449in}}%
\pgfpathlineto{\pgfqpoint{1.608564in}{0.664832in}}%
\pgfpathlineto{\pgfqpoint{1.690250in}{0.655021in}}%
\pgfpathlineto{\pgfqpoint{1.771937in}{0.643017in}}%
\pgfpathlineto{\pgfqpoint{1.853623in}{0.628820in}}%
\pgfpathlineto{\pgfqpoint{1.935310in}{0.612428in}}%
\pgfpathlineto{\pgfqpoint{2.016996in}{0.593843in}}%
\pgfpathlineto{\pgfqpoint{2.112297in}{0.659285in}}%
\pgfpathlineto{\pgfqpoint{2.207598in}{0.721741in}}%
\pgfpathlineto{\pgfqpoint{2.302899in}{0.781211in}}%
\pgfpathlineto{\pgfqpoint{2.398199in}{0.837695in}}%
\pgfpathlineto{\pgfqpoint{2.493500in}{0.891194in}}%
\pgfpathlineto{\pgfqpoint{2.575187in}{0.934673in}}%
\pgfpathlineto{\pgfqpoint{2.656873in}{0.975959in}}%
\pgfpathlineto{\pgfqpoint{2.738560in}{1.015051in}}%
\pgfpathlineto{\pgfqpoint{2.820246in}{1.051949in}}%
\pgfpathlineto{\pgfqpoint{2.901932in}{1.086654in}}%
\pgfpathlineto{\pgfqpoint{2.983619in}{1.119165in}}%
\pgfpathlineto{\pgfqpoint{3.065305in}{1.149483in}}%
\pgfpathlineto{\pgfqpoint{3.146992in}{1.177607in}}%
\pgfpathlineto{\pgfqpoint{3.228678in}{1.203537in}}%
\pgfpathlineto{\pgfqpoint{3.310365in}{1.227274in}}%
\pgfpathlineto{\pgfqpoint{3.392051in}{1.248817in}}%
\pgfpathlineto{\pgfqpoint{3.473737in}{1.268166in}}%
\pgfpathlineto{\pgfqpoint{3.555424in}{1.285322in}}%
\pgfpathlineto{\pgfqpoint{3.637110in}{1.300284in}}%
\pgfpathlineto{\pgfqpoint{3.718797in}{1.313053in}}%
\pgfpathlineto{\pgfqpoint{3.800483in}{1.323628in}}%
\pgfpathlineto{\pgfqpoint{3.882170in}{1.332009in}}%
\pgfpathlineto{\pgfqpoint{3.963856in}{1.338196in}}%
\pgfpathlineto{\pgfqpoint{4.045542in}{1.342190in}}%
\pgfpathlineto{\pgfqpoint{4.127229in}{1.343991in}}%
\pgfpathlineto{\pgfqpoint{4.208915in}{1.343597in}}%
\pgfpathlineto{\pgfqpoint{4.290602in}{1.341010in}}%
\pgfpathlineto{\pgfqpoint{4.372288in}{1.336230in}}%
\pgfpathlineto{\pgfqpoint{4.453975in}{1.329255in}}%
\pgfpathlineto{\pgfqpoint{4.535661in}{1.320088in}}%
\pgfpathlineto{\pgfqpoint{4.617347in}{1.308726in}}%
\pgfpathlineto{\pgfqpoint{4.699034in}{1.295171in}}%
\pgfpathlineto{\pgfqpoint{4.780720in}{1.279422in}}%
\pgfpathlineto{\pgfqpoint{4.862407in}{1.261480in}}%
\pgfpathlineto{\pgfqpoint{4.944093in}{1.241344in}}%
\pgfpathlineto{\pgfqpoint{5.025780in}{1.219014in}}%
\pgfpathlineto{\pgfqpoint{5.107466in}{1.194491in}}%
\pgfpathlineto{\pgfqpoint{5.189152in}{1.167774in}}%
\pgfpathlineto{\pgfqpoint{5.270839in}{1.138863in}}%
\pgfpathlineto{\pgfqpoint{5.352525in}{1.107759in}}%
\pgfpathlineto{\pgfqpoint{5.434212in}{1.074461in}}%
\pgfpathlineto{\pgfqpoint{5.515898in}{1.038969in}}%
\pgfpathlineto{\pgfqpoint{5.597585in}{1.001284in}}%
\pgfpathlineto{\pgfqpoint{5.679271in}{0.961405in}}%
\pgfpathlineto{\pgfqpoint{5.760957in}{0.919333in}}%
\pgfpathlineto{\pgfqpoint{5.842644in}{0.875067in}}%
\pgfpathlineto{\pgfqpoint{5.924330in}{0.828607in}}%
\pgfpathlineto{\pgfqpoint{6.019631in}{0.771631in}}%
\pgfpathlineto{\pgfqpoint{6.114932in}{0.711670in}}%
\pgfpathlineto{\pgfqpoint{6.210233in}{0.648723in}}%
\pgfpathlineto{\pgfqpoint{6.264690in}{0.611433in}}%
\pgfpathlineto{\pgfqpoint{6.346377in}{0.627946in}}%
\pgfpathlineto{\pgfqpoint{6.428063in}{0.642266in}}%
\pgfpathlineto{\pgfqpoint{6.509750in}{0.654392in}}%
\pgfpathlineto{\pgfqpoint{6.591436in}{0.664324in}}%
\pgfpathlineto{\pgfqpoint{6.673123in}{0.672063in}}%
\pgfpathlineto{\pgfqpoint{6.754809in}{0.677608in}}%
\pgfpathlineto{\pgfqpoint{6.836495in}{0.680959in}}%
\pgfpathlineto{\pgfqpoint{6.918182in}{0.682137in}}%
\pgfpathlineto{\pgfqpoint{6.918182in}{0.682137in}}%
\pgfusepath{stroke}%
\end{pgfscope}%
\begin{pgfscope}%
\pgfsetroundcap%
\pgfsetroundjoin%
\pgfsetlinewidth{1.003750pt}%
\definecolor{currentstroke}{rgb}{0.000000,0.000000,0.000000}%
\pgfsetstrokecolor{currentstroke}%
\pgfsetdash{}{0pt}%
\pgfpathmoveto{\pgfqpoint{3.790686in}{2.535000in}}%
\pgfpathquadraticcurveto{\pgfqpoint{3.958695in}{2.535000in}}{\pgfqpoint{4.126704in}{2.535000in}}%
\pgfusepath{stroke}%
\end{pgfscope}%
\begin{pgfscope}%
\pgfsetbuttcap%
\pgfsetmiterjoin%
\definecolor{currentfill}{rgb}{0.800000,0.800000,0.800000}%
\pgfsetfillcolor{currentfill}%
\pgfsetlinewidth{1.003750pt}%
\definecolor{currentstroke}{rgb}{0.000000,0.000000,0.000000}%
\pgfsetstrokecolor{currentstroke}%
\pgfsetdash{}{0pt}%
\pgfpathmoveto{\pgfqpoint{2.730847in}{2.438549in}}%
\pgfpathcurveto{\pgfqpoint{2.765569in}{2.403827in}}{\pgfqpoint{3.698232in}{2.403827in}}{\pgfqpoint{3.732954in}{2.438549in}}%
\pgfpathcurveto{\pgfqpoint{3.767676in}{2.473272in}}{\pgfqpoint{3.767676in}{2.596728in}}{\pgfqpoint{3.732954in}{2.631451in}}%
\pgfpathcurveto{\pgfqpoint{3.698232in}{2.666173in}}{\pgfqpoint{2.765569in}{2.666173in}}{\pgfqpoint{2.730847in}{2.631451in}}%
\pgfpathcurveto{\pgfqpoint{2.696124in}{2.596728in}}{\pgfqpoint{2.696124in}{2.473272in}}{\pgfqpoint{2.730847in}{2.438549in}}%
\pgfpathclose%
\pgfusepath{stroke,fill}%
\end{pgfscope}%
\begin{pgfscope}%
\definecolor{textcolor}{rgb}{0.000000,0.000000,0.000000}%
\pgfsetstrokecolor{textcolor}%
\pgfsetfillcolor{textcolor}%
\pgftext[x=2.765569in,y=2.535000in,left,]{\color{textcolor}\rmfamily\fontsize{10.000000}{12.000000}\selectfont \(\displaystyle M_u =\) 461.6 kft}%
\end{pgfscope}%
\begin{pgfscope}%
\pgfsetbuttcap%
\pgfsetmiterjoin%
\definecolor{currentfill}{rgb}{0.800000,0.800000,0.800000}%
\pgfsetfillcolor{currentfill}%
\pgfsetlinewidth{1.003750pt}%
\definecolor{currentstroke}{rgb}{0.000000,0.000000,0.000000}%
\pgfsetstrokecolor{currentstroke}%
\pgfsetdash{}{0pt}%
\pgfpathmoveto{\pgfqpoint{0.965278in}{0.358599in}}%
\pgfpathcurveto{\pgfqpoint{1.000000in}{0.323877in}}{\pgfqpoint{2.720682in}{0.323877in}}{\pgfqpoint{2.755404in}{0.358599in}}%
\pgfpathcurveto{\pgfqpoint{2.790127in}{0.393321in}}{\pgfqpoint{2.790127in}{0.668784in}}{\pgfqpoint{2.755404in}{0.703506in}}%
\pgfpathcurveto{\pgfqpoint{2.720682in}{0.738228in}}{\pgfqpoint{1.000000in}{0.738228in}}{\pgfqpoint{0.965278in}{0.703506in}}%
\pgfpathcurveto{\pgfqpoint{0.930556in}{0.668784in}}{\pgfqpoint{0.930556in}{0.393321in}}{\pgfqpoint{0.965278in}{0.358599in}}%
\pgfpathclose%
\pgfusepath{stroke,fill}%
\end{pgfscope}%
\begin{pgfscope}%
\definecolor{textcolor}{rgb}{0.000000,0.000000,0.000000}%
\pgfsetstrokecolor{textcolor}%
\pgfsetfillcolor{textcolor}%
\pgftext[x=1.000000in, y=0.580049in, left, base]{\color{textcolor}\rmfamily\fontsize{10.000000}{12.000000}\selectfont Max combo: 1.2D + 1.6S}%
\end{pgfscope}%
\begin{pgfscope}%
\definecolor{textcolor}{rgb}{0.000000,0.000000,0.000000}%
\pgfsetstrokecolor{textcolor}%
\pgfsetfillcolor{textcolor}%
\pgftext[x=1.000000in, y=0.428043in, left, base]{\color{textcolor}\rmfamily\fontsize{10.000000}{12.000000}\selectfont ASCE7-16 Sec. 2.3.1 (LC 3)}%
\end{pgfscope}%
\end{pgfpicture}%
\makeatother%
\endgroup%

\end{center}
\caption{Moment Demand Envelope}
\end{figure}
L\textsubscript{p}, the limiting laterally unbraced length for the limit state of yielding, is calculated per AISC/ANSI 360-16 Eq. F2-5 as follows:
\begin{flalign*}
L_p = 1.76\cdot r_y \cdot \sqrt{\frac{E}{F_y}}  = 1.76\cdot 2.68 {\color{darkBlue}{\mathbf{ \; in}}} \cdot \sqrt{\frac{29000 {\color{darkBlue}{\mathbf{ \; ksi}}}}{50 {\color{darkBlue}{\mathbf{ \; ksi}}}}}  = \mathbf{9.5 {\color{darkBlue}{\mathbf{ \; ft}}}}
\end{flalign*}
r\textsubscript{ts}, a coefficient used in the calculation of L\textsubscript{r} and C\textsubscript{b}, is calculated per AISC/ANSI 360-16 Eq. F2-7 as follows:
\begin{flalign*}
r_{{ts}} = \sqrt{\frac{\sqrt{I_y \cdot C_w}}{S_x}}  = \sqrt{\frac{\sqrt{236 {\color{darkBlue}{\mathbf{ \; {\color{darkBlue}{\mathbf{ \; in}}}^{4}}}} \cdot 6020 {\color{darkBlue}{\mathbf{ \; {\color{darkBlue}{\mathbf{ \; in}}}^{6}}}}}}{126 {\color{darkBlue}{\mathbf{ \; {\color{darkBlue}{\mathbf{ \; in}}}^{3}}}}}}  = \mathbf{3.1 {\color{darkBlue}{\mathbf{ \; in}}}}
\end{flalign*}
L\textsubscript{r}, the limiting unbraced length for the limit state of inelastic lateral-torsional buckling, is calculated per AISC/ANSI 360-16 Eq. F2-6 as follows:
\begin{flalign*}
L_r &= 1.95\cdot r_{ts} \cdot \frac{E}{0.7\cdot F_y} \sqrt{\frac{J \cdot c}{S_x \cdot h_0}+\sqrt{{\left(\frac{J \cdot c}{S_x \cdot h_0}\right)}^2+6.76{\left(\frac{0.7\cdot F_y}{E}\right)}^2}} \\ &= 1.95\cdot 3.1 {\color{darkBlue}{\mathbf{ \; in}}} \cdot \frac{29000 {\color{darkBlue}{\mathbf{ \; ksi}}}}{0.7\cdot 50 {\color{darkBlue}{\mathbf{ \; ksi}}}} \sqrt{\frac{15.1 {\color{darkBlue}{\mathbf{ \; {\color{darkBlue}{\mathbf{ \; in}}}^{4}}}} \cdot 1}{126 {\color{darkBlue}{\mathbf{ \; {\color{darkBlue}{\mathbf{ \; in}}}^{3}}}} \cdot 10.2 {\color{darkBlue}{\mathbf{ \; in}}}}+\sqrt{{\left(\frac{15.1 {\color{darkBlue}{\mathbf{ \; {\color{darkBlue}{\mathbf{ \; in}}}^{4}}}} \cdot 1}{126 {\color{darkBlue}{\mathbf{ \; {\color{darkBlue}{\mathbf{ \; in}}}^{3}}}} \cdot 10.2 {\color{darkBlue}{\mathbf{ \; in}}}}\right)}^2+6.76{\left(\frac{0.7\cdot 50 {\color{darkBlue}{\mathbf{ \; ksi}}}}{29000 {\color{darkBlue}{\mathbf{ \; ksi}}}}\right)}^2}} \\ &= \mathbf{64.2 {\color{darkBlue}{\mathbf{ \; ft}}}}
\end{flalign*}
\textlambda\textsubscript{web}, the web width-to-thickness ratio, is calculated per {AISC/ANSI 360-16 Table B4.1b} as follows:
\begin{flalign*}
\lambda_{{web}} = \frac{h}{t_w}  = \frac{7.9 {\color{darkBlue}{\mathbf{ \; in}}}}{0.755 {\color{darkBlue}{\mathbf{ \; in}}}}  = \mathbf{10.5 }
\end{flalign*}
\textlambda\textsubscript{P-web}, the limiting width-to-thickness ratio for compact/noncompact web, is calculated per {AISC/ANSI 360-16 Table B4.1b} as follows:
\begin{flalign*}
\lambda_{P-web} = 3.76\cdot \sqrt{\frac{E}{F_y}} = 3.76\cdot \sqrt{\frac{29000 {\color{darkBlue}{\mathbf{ \; ksi}}}}{50 {\color{darkBlue}{\mathbf{ \; ksi}}}}} = \mathbf{90.6}
\end{flalign*}
\textlambda\textsubscript{R-web}, the limiting width-to-thickness ratio for noncompact/slender web, is calculated per {AISC/ANSI 360-16 Table B4.1b} as follows:
\begin{flalign*}
\lambda_{R-web} = 5.7\cdot \sqrt{\frac{E}{F_y}} = 5.7\cdot \sqrt{\frac{29000 {\color{darkBlue}{\mathbf{ \; ksi}}}}{50 {\color{darkBlue}{\mathbf{ \; ksi}}}}} = \mathbf{137.3}
\end{flalign*}
\textlambda\textsubscript{web} $<$ \textlambda\textsubscript{P-web} \textrightarrow \; \textbf{Compact Web}
\\\\
\textlambda\textsubscript{flange}, the flange width-to-thickness ratio, is calculated per {AISC/ANSI 360-16 Table B4.1b} as follows:
\begin{flalign*}
\lambda_{{flange}} = \frac{b}{t}  = \frac{5.2 {\color{darkBlue}{\mathbf{ \; in}}}}{1.25 {\color{darkBlue}{\mathbf{ \; in}}}}  = \mathbf{4.2 }
\end{flalign*}
\textlambda\textsubscript{P-flange}, the limiting width-to-thickness ratio for compact/noncompact flange, is calculated per {AISC/ANSI 360-16 Table B4.1b} as follows:
\begin{flalign*}
\lambda_{P-flange} = 0.38\cdot \sqrt{\frac{E}{F_y}} = 0.38\cdot \sqrt{\frac{29000 {\color{darkBlue}{\mathbf{ \; ksi}}}}{50 {\color{darkBlue}{\mathbf{ \; ksi}}}}} = \mathbf{9.2}
\end{flalign*}
\textlambda\textsubscript{R-flange}, the limiting width-to-thickness ratio for noncompact/slender flange, is calculated per {AISC/ANSI 360-16 Table B4.1b} as follows:
\begin{flalign*}
\lambda_{R-flange} = 1\cdot \sqrt{\frac{E}{F_y}} = 1\cdot \sqrt{\frac{29000 {\color{darkBlue}{\mathbf{ \; ksi}}}}{50 {\color{darkBlue}{\mathbf{ \; ksi}}}}} = \mathbf{24.1}
\end{flalign*}
\textlambda\textsubscript{flange} $<$ \textlambda\textsubscript{P-flange} \textrightarrow \; \textbf{Compact Flange}
\\\\
Since \(\mathbf{{L_p} < {L_b} <= {L_r}}\) and the beam's flanges are \textbf{compact}, controlling limit state for flexure is \textbf{LTB (not to exceed capacity based on yielding)}.
\\\\
M\textsubscript{p}, the plastic bending moment, is calculated per AISC/ANSI 360-16 Eq. F2-1 as follows:
\begin{flalign*}
M_p = F_y \cdot Z_x  = 50 {\color{darkBlue}{\mathbf{ \; ksi}}} \cdot 147 {\color{darkBlue}{\mathbf{ \; {\color{darkBlue}{\mathbf{ \; in}}}^{3}}}}  = \mathbf{612.5 {\color{darkBlue}{\mathbf{ \; kft}}}}
\end{flalign*}
C\textsubscript{b}, the lateral-torsional buckling modification factor in the critical unbraced span for the critical load combination, is calculated per AISC/ANSI 360- 16 Sec. F1 as follows:
\\
\begin{flalign*}
C_b &= \frac{12.5\cdot M_{max}}{2.5\cdot M_{max}+3\cdot M_A+4\cdot M_B+3\cdot M_C} \\ &= \frac{12.5\cdot 461.6 {\color{darkBlue}{\mathbf{ \; kft}}}}{2.5\cdot 461.6 {\color{darkBlue}{\mathbf{ \; kft}}}+3\cdot 332.6 {\color{darkBlue}{\mathbf{ \; kft}}}+4\cdot 461.6 {\color{darkBlue}{\mathbf{ \; kft}}}+3\cdot 332.1 {\color{darkBlue}{\mathbf{ \; kft}}}} \\ &= \mathbf{1.2 }
\end{flalign*}
\\
For brevity, the C\textsubscript{b} calculation is not shown for each span. The following figure illustrates the value of C\textsubscript{b} for each span.
\begin{figure}[H]
\begin{center}
%% Creator: Matplotlib, PGF backend
%%
%% To include the figure in your LaTeX document, write
%%   \input{<filename>.pgf}
%%
%% Make sure the required packages are loaded in your preamble
%%   \usepackage{pgf}
%%
%% Figures using additional raster images can only be included by \input if
%% they are in the same directory as the main LaTeX file. For loading figures
%% from other directories you can use the `import` package
%%   \usepackage{import}
%%
%% and then include the figures with
%%   \import{<path to file>}{<filename>.pgf}
%%
%% Matplotlib used the following preamble
%%
\begingroup%
\makeatletter%
\begin{pgfpicture}%
\pgfpathrectangle{\pgfpointorigin}{\pgfqpoint{8.000000in}{1.000000in}}%
\pgfusepath{use as bounding box, clip}%
\begin{pgfscope}%
\pgfpathrectangle{\pgfqpoint{1.000000in}{0.110000in}}{\pgfqpoint{6.200000in}{0.770000in}}%
\pgfusepath{clip}%
\pgfsetrectcap%
\pgfsetroundjoin%
\pgfsetlinewidth{0.752812pt}%
\definecolor{currentstroke}{rgb}{0.000000,0.000000,0.000000}%
\pgfsetstrokecolor{currentstroke}%
\pgfsetdash{}{0pt}%
\pgfpathmoveto{\pgfqpoint{1.281818in}{0.880000in}}%
\pgfpathlineto{\pgfqpoint{2.016996in}{0.880000in}}%
\pgfpathlineto{\pgfqpoint{6.264690in}{0.880000in}}%
\pgfpathlineto{\pgfqpoint{6.918182in}{0.880000in}}%
\pgfusepath{stroke}%
\end{pgfscope}%
\begin{pgfscope}%
\pgfpathrectangle{\pgfqpoint{1.000000in}{0.110000in}}{\pgfqpoint{6.200000in}{0.770000in}}%
\pgfusepath{clip}%
\pgfsetrectcap%
\pgfsetroundjoin%
\pgfsetlinewidth{0.752812pt}%
\definecolor{currentstroke}{rgb}{0.000000,0.000000,0.000000}%
\pgfsetstrokecolor{currentstroke}%
\pgfsetdash{}{0pt}%
\pgfpathmoveto{\pgfqpoint{1.281818in}{0.795570in}}%
\pgfpathlineto{\pgfqpoint{2.016996in}{0.795570in}}%
\pgfpathlineto{\pgfqpoint{6.264690in}{0.795570in}}%
\pgfpathlineto{\pgfqpoint{6.918182in}{0.795570in}}%
\pgfusepath{stroke}%
\end{pgfscope}%
\begin{pgfscope}%
\pgfpathrectangle{\pgfqpoint{1.000000in}{0.110000in}}{\pgfqpoint{6.200000in}{0.770000in}}%
\pgfusepath{clip}%
\pgfsetrectcap%
\pgfsetroundjoin%
\pgfsetlinewidth{0.752812pt}%
\definecolor{currentstroke}{rgb}{0.000000,0.000000,0.000000}%
\pgfsetstrokecolor{currentstroke}%
\pgfsetdash{}{0pt}%
\pgfpathmoveto{\pgfqpoint{1.281818in}{0.110000in}}%
\pgfpathlineto{\pgfqpoint{2.016996in}{0.110000in}}%
\pgfpathlineto{\pgfqpoint{6.264690in}{0.110000in}}%
\pgfpathlineto{\pgfqpoint{6.918182in}{0.110000in}}%
\pgfusepath{stroke}%
\end{pgfscope}%
\begin{pgfscope}%
\pgfpathrectangle{\pgfqpoint{1.000000in}{0.110000in}}{\pgfqpoint{6.200000in}{0.770000in}}%
\pgfusepath{clip}%
\pgfsetrectcap%
\pgfsetroundjoin%
\pgfsetlinewidth{0.752812pt}%
\definecolor{currentstroke}{rgb}{0.000000,0.000000,0.000000}%
\pgfsetstrokecolor{currentstroke}%
\pgfsetdash{}{0pt}%
\pgfpathmoveto{\pgfqpoint{1.281818in}{0.194430in}}%
\pgfpathlineto{\pgfqpoint{2.016996in}{0.194430in}}%
\pgfpathlineto{\pgfqpoint{6.264690in}{0.194430in}}%
\pgfpathlineto{\pgfqpoint{6.918182in}{0.194430in}}%
\pgfusepath{stroke}%
\end{pgfscope}%
\begin{pgfscope}%
\pgfpathrectangle{\pgfqpoint{1.000000in}{0.110000in}}{\pgfqpoint{6.200000in}{0.770000in}}%
\pgfusepath{clip}%
\pgfsetbuttcap%
\pgfsetroundjoin%
\pgfsetlinewidth{1.505625pt}%
\definecolor{currentstroke}{rgb}{1.000000,0.000000,0.000000}%
\pgfsetstrokecolor{currentstroke}%
\pgfsetdash{{1.500000pt}{2.475000pt}}{0.000000pt}%
\pgfpathmoveto{\pgfqpoint{1.281818in}{0.110000in}}%
\pgfpathlineto{\pgfqpoint{1.281818in}{0.880000in}}%
\pgfusepath{stroke}%
\end{pgfscope}%
\begin{pgfscope}%
\pgfpathrectangle{\pgfqpoint{1.000000in}{0.110000in}}{\pgfqpoint{6.200000in}{0.770000in}}%
\pgfusepath{clip}%
\pgfsetbuttcap%
\pgfsetroundjoin%
\pgfsetlinewidth{1.505625pt}%
\definecolor{currentstroke}{rgb}{1.000000,0.000000,0.000000}%
\pgfsetstrokecolor{currentstroke}%
\pgfsetdash{{1.500000pt}{2.475000pt}}{0.000000pt}%
\pgfpathmoveto{\pgfqpoint{2.016996in}{0.110000in}}%
\pgfpathlineto{\pgfqpoint{2.016996in}{0.880000in}}%
\pgfusepath{stroke}%
\end{pgfscope}%
\begin{pgfscope}%
\pgfpathrectangle{\pgfqpoint{1.000000in}{0.110000in}}{\pgfqpoint{6.200000in}{0.770000in}}%
\pgfusepath{clip}%
\pgfsetbuttcap%
\pgfsetroundjoin%
\pgfsetlinewidth{1.505625pt}%
\definecolor{currentstroke}{rgb}{1.000000,0.000000,0.000000}%
\pgfsetstrokecolor{currentstroke}%
\pgfsetdash{{1.500000pt}{2.475000pt}}{0.000000pt}%
\pgfpathmoveto{\pgfqpoint{6.264690in}{0.110000in}}%
\pgfpathlineto{\pgfqpoint{6.264690in}{0.880000in}}%
\pgfusepath{stroke}%
\end{pgfscope}%
\begin{pgfscope}%
\pgfpathrectangle{\pgfqpoint{1.000000in}{0.110000in}}{\pgfqpoint{6.200000in}{0.770000in}}%
\pgfusepath{clip}%
\pgfsetbuttcap%
\pgfsetroundjoin%
\pgfsetlinewidth{1.505625pt}%
\definecolor{currentstroke}{rgb}{1.000000,0.000000,0.000000}%
\pgfsetstrokecolor{currentstroke}%
\pgfsetdash{{1.500000pt}{2.475000pt}}{0.000000pt}%
\pgfpathmoveto{\pgfqpoint{6.918182in}{0.110000in}}%
\pgfpathlineto{\pgfqpoint{6.918182in}{0.880000in}}%
\pgfusepath{stroke}%
\end{pgfscope}%
\begin{pgfscope}%
\pgfsetbuttcap%
\pgfsetmiterjoin%
\definecolor{currentfill}{rgb}{0.800000,0.800000,0.800000}%
\pgfsetfillcolor{currentfill}%
\pgfsetlinewidth{1.003750pt}%
\definecolor{currentstroke}{rgb}{0.000000,0.000000,0.000000}%
\pgfsetstrokecolor{currentstroke}%
\pgfsetdash{}{0pt}%
\pgfpathmoveto{\pgfqpoint{1.341343in}{0.396861in}}%
\pgfpathcurveto{\pgfqpoint{1.376066in}{0.362139in}}{\pgfqpoint{1.922749in}{0.362139in}}{\pgfqpoint{1.957471in}{0.396861in}}%
\pgfpathcurveto{\pgfqpoint{1.992193in}{0.431583in}}{\pgfqpoint{1.992193in}{0.555040in}}{\pgfqpoint{1.957471in}{0.589762in}}%
\pgfpathcurveto{\pgfqpoint{1.922749in}{0.624484in}}{\pgfqpoint{1.376066in}{0.624484in}}{\pgfqpoint{1.341343in}{0.589762in}}%
\pgfpathcurveto{\pgfqpoint{1.306621in}{0.555040in}}{\pgfqpoint{1.306621in}{0.431583in}}{\pgfqpoint{1.341343in}{0.396861in}}%
\pgfpathclose%
\pgfusepath{stroke,fill}%
\end{pgfscope}%
\begin{pgfscope}%
\definecolor{textcolor}{rgb}{0.000000,0.000000,0.000000}%
\pgfsetstrokecolor{textcolor}%
\pgfsetfillcolor{textcolor}%
\pgftext[x=1.649407in,y=0.493311in,,]{\color{textcolor}\rmfamily\fontsize{10.000000}{12.000000}\selectfont C\textsubscript{b} = 1.0}%
\end{pgfscope}%
\begin{pgfscope}%
\pgfsetbuttcap%
\pgfsetmiterjoin%
\definecolor{currentfill}{rgb}{0.800000,0.800000,0.800000}%
\pgfsetfillcolor{currentfill}%
\pgfsetlinewidth{1.003750pt}%
\definecolor{currentstroke}{rgb}{0.000000,0.000000,0.000000}%
\pgfsetstrokecolor{currentstroke}%
\pgfsetdash{}{0pt}%
\pgfpathmoveto{\pgfqpoint{3.798057in}{0.396861in}}%
\pgfpathcurveto{\pgfqpoint{3.832779in}{0.362139in}}{\pgfqpoint{4.448907in}{0.362139in}}{\pgfqpoint{4.483629in}{0.396861in}}%
\pgfpathcurveto{\pgfqpoint{4.518352in}{0.431583in}}{\pgfqpoint{4.518352in}{0.555040in}}{\pgfqpoint{4.483629in}{0.589762in}}%
\pgfpathcurveto{\pgfqpoint{4.448907in}{0.624484in}}{\pgfqpoint{3.832779in}{0.624484in}}{\pgfqpoint{3.798057in}{0.589762in}}%
\pgfpathcurveto{\pgfqpoint{3.763335in}{0.555040in}}{\pgfqpoint{3.763335in}{0.431583in}}{\pgfqpoint{3.798057in}{0.396861in}}%
\pgfpathclose%
\pgfusepath{stroke,fill}%
\end{pgfscope}%
\begin{pgfscope}%
\definecolor{textcolor}{rgb}{0.000000,0.000000,0.000000}%
\pgfsetstrokecolor{textcolor}%
\pgfsetfillcolor{textcolor}%
\pgftext[x=4.140843in,y=0.493311in,,]{\color{textcolor}\rmfamily\fontsize{10.000000}{12.000000}\selectfont C\textsubscript{b} = 1.16}%
\end{pgfscope}%
\begin{pgfscope}%
\pgfsetbuttcap%
\pgfsetmiterjoin%
\definecolor{currentfill}{rgb}{0.800000,0.800000,0.800000}%
\pgfsetfillcolor{currentfill}%
\pgfsetlinewidth{1.003750pt}%
\definecolor{currentstroke}{rgb}{0.000000,0.000000,0.000000}%
\pgfsetstrokecolor{currentstroke}%
\pgfsetdash{}{0pt}%
\pgfpathmoveto{\pgfqpoint{6.283372in}{0.396861in}}%
\pgfpathcurveto{\pgfqpoint{6.318095in}{0.362139in}}{\pgfqpoint{6.864778in}{0.362139in}}{\pgfqpoint{6.899500in}{0.396861in}}%
\pgfpathcurveto{\pgfqpoint{6.934222in}{0.431583in}}{\pgfqpoint{6.934222in}{0.555040in}}{\pgfqpoint{6.899500in}{0.589762in}}%
\pgfpathcurveto{\pgfqpoint{6.864778in}{0.624484in}}{\pgfqpoint{6.318095in}{0.624484in}}{\pgfqpoint{6.283372in}{0.589762in}}%
\pgfpathcurveto{\pgfqpoint{6.248650in}{0.555040in}}{\pgfqpoint{6.248650in}{0.431583in}}{\pgfqpoint{6.283372in}{0.396861in}}%
\pgfpathclose%
\pgfusepath{stroke,fill}%
\end{pgfscope}%
\begin{pgfscope}%
\definecolor{textcolor}{rgb}{0.000000,0.000000,0.000000}%
\pgfsetstrokecolor{textcolor}%
\pgfsetfillcolor{textcolor}%
\pgftext[x=6.591436in,y=0.493311in,,]{\color{textcolor}\rmfamily\fontsize{10.000000}{12.000000}\selectfont C\textsubscript{b} = 1.0}%
\end{pgfscope}%
\end{pgfpicture}%
\makeatother%
\endgroup%

\end{center}
\caption{C\textsubscript{b} Along Member}
\end{figure}
F\textsubscript{cr}, the buckling stress for the critical section in the critical unbraced span, is calculated per AISC/ANSI 360- 16 Eq. F2-4 as follows:
\begin{flalign*}
F_{cr} & = \cfrac{C_b \cdot \pi^2 \cdot {{E}}} {\left(\cfrac{L_b}{r_{ts}}\right)^2} \cdot \sqrt{1 + 0.078 \cdot \cfrac{{J} \cdot {c}}{{S_x} \cdot h_0} \cdot \left(\cfrac{L_b}{r_{ts}}\right)^2} \\ & = \cfrac{1.16  \cdot \pi^2 \cdot {{29000 {\color{darkBlue}{\mathbf{ \; ksi}}}}}} {\left(\cfrac{26.0 {\color{darkBlue}{\mathbf{ \; ft}}}}{3.1 {\color{darkBlue}{\mathbf{ \; in}}}}\right)^2} \cdot \sqrt{1 + 0.078 \cdot \cfrac{{15.1 {\color{darkBlue}{\mathbf{ \; {\color{darkBlue}{\mathbf{ \; in}}}^{4}}}}} \cdot {1}}{{126 {\color{darkBlue}{\mathbf{ \; {\color{darkBlue}{\mathbf{ \; in}}}^{3}}}}} \cdot 10.2 {\color{darkBlue}{\mathbf{ \; in}}}} \cdot \left(\cfrac{26.0 {\color{darkBlue}{\mathbf{ \; ft}}}}{3.1 {\color{darkBlue}{\mathbf{ \; in}}}}\right)^2} = \mathbf{104.0 {\color{darkBlue}{\mathbf{ \; ksi}}}}
\end{flalign*}
\\
\textphi\textsubscript{b}, the resistance factor for bending, is determined per AISC/ANSI 360-16 {\S}F1a as \textbf{0.9}.
\\\\
\textphi\textsubscript{b}M\textsubscript{n}, the design flexural strength, is calculated per AISC/ANSI 360-16 Eq. F2-1 as follows:
\begin{flalign*}
{\phi_b}{M_{n}} & = {\phi_b} \cdot {C_b} \cdot {{M_p} - 0.7 \cdot {F_{y}} \cdot {S_{x}} \cdot \frac{{L_b} - {L_p}}{{L_r} - {L_p}}} < {\phi_b} \cdot {M_p} \\ & = {0.9} \cdot {1.16 } \cdot {{612.5 {\color{darkBlue}{\mathbf{ \; kft}}}} - 0.7 \cdot {50 {\color{darkBlue}{\mathbf{ \; ksi}}}} \cdot {126 {\color{darkBlue}{\mathbf{ \; {\color{darkBlue}{\mathbf{ \; in}}}^{3}}}}} \cdot \frac{{{26.0 {\color{darkBlue}{\mathbf{ \; ft}}}}} - {9.5 {\color{darkBlue}{\mathbf{ \; ft}}}}}{{64.2 {\color{darkBlue}{\mathbf{ \; ft}}}} - {9.5 {\color{darkBlue}{\mathbf{ \; ft}}}}}} < {0.9} \cdot {612.5 {\color{darkBlue}{\mathbf{ \; kft}}}} = \mathbf{551.3 {\color{darkBlue}{\mathbf{ \; kft}}}}
\end{flalign*}
\vspace{-20pt}
{\setlength{\mathindent}{0cm}
\begin{flalign*}
\mathbf{|M_u| = 461.6 {\color{darkBlue}{\mathbf{ \; kft}}}  \;  < \phi_b \cdot M_n = 551.3 {\color{darkBlue}{\mathbf{ \; kft}}}  \;  (DCR = 0.84 - OK)}
\end{flalign*}
\textbf{(yielding controls)}
%	-------------------------------- SHEAR CHECK ---------------------------------
\section{Shear Check}
\begin{figure}[H]
\begin{center}
%% Creator: Matplotlib, PGF backend
%%
%% To include the figure in your LaTeX document, write
%%   \input{<filename>.pgf}
%%
%% Make sure the required packages are loaded in your preamble
%%   \usepackage{pgf}
%%
%% Figures using additional raster images can only be included by \input if
%% they are in the same directory as the main LaTeX file. For loading figures
%% from other directories you can use the `import` package
%%   \usepackage{import}
%%
%% and then include the figures with
%%   \import{<path to file>}{<filename>.pgf}
%%
%% Matplotlib used the following preamble
%%
\begingroup%
\makeatletter%
\begin{pgfpicture}%
\pgfpathrectangle{\pgfpointorigin}{\pgfqpoint{8.000000in}{3.000000in}}%
\pgfusepath{use as bounding box, clip}%
\begin{pgfscope}%
\pgfsetbuttcap%
\pgfsetmiterjoin%
\definecolor{currentfill}{rgb}{1.000000,1.000000,1.000000}%
\pgfsetfillcolor{currentfill}%
\pgfsetlinewidth{0.000000pt}%
\definecolor{currentstroke}{rgb}{1.000000,1.000000,1.000000}%
\pgfsetstrokecolor{currentstroke}%
\pgfsetdash{}{0pt}%
\pgfpathmoveto{\pgfqpoint{0.000000in}{0.000000in}}%
\pgfpathlineto{\pgfqpoint{8.000000in}{0.000000in}}%
\pgfpathlineto{\pgfqpoint{8.000000in}{3.000000in}}%
\pgfpathlineto{\pgfqpoint{0.000000in}{3.000000in}}%
\pgfpathclose%
\pgfusepath{fill}%
\end{pgfscope}%
\begin{pgfscope}%
\pgfsetbuttcap%
\pgfsetmiterjoin%
\definecolor{currentfill}{rgb}{1.000000,1.000000,1.000000}%
\pgfsetfillcolor{currentfill}%
\pgfsetlinewidth{0.000000pt}%
\definecolor{currentstroke}{rgb}{0.000000,0.000000,0.000000}%
\pgfsetstrokecolor{currentstroke}%
\pgfsetstrokeopacity{0.000000}%
\pgfsetdash{}{0pt}%
\pgfpathmoveto{\pgfqpoint{1.000000in}{0.330000in}}%
\pgfpathlineto{\pgfqpoint{7.200000in}{0.330000in}}%
\pgfpathlineto{\pgfqpoint{7.200000in}{2.640000in}}%
\pgfpathlineto{\pgfqpoint{1.000000in}{2.640000in}}%
\pgfpathclose%
\pgfusepath{fill}%
\end{pgfscope}%
\begin{pgfscope}%
\pgfpathrectangle{\pgfqpoint{1.000000in}{0.330000in}}{\pgfqpoint{6.200000in}{2.310000in}}%
\pgfusepath{clip}%
\pgfsetbuttcap%
\pgfsetroundjoin%
\pgfsetlinewidth{0.803000pt}%
\definecolor{currentstroke}{rgb}{0.000000,0.000000,0.000000}%
\pgfsetstrokecolor{currentstroke}%
\pgfsetdash{{0.800000pt}{1.320000pt}}{0.000000pt}%
\pgfpathmoveto{\pgfqpoint{1.281818in}{0.330000in}}%
\pgfpathlineto{\pgfqpoint{1.281818in}{2.640000in}}%
\pgfusepath{stroke}%
\end{pgfscope}%
\begin{pgfscope}%
\pgfsetbuttcap%
\pgfsetroundjoin%
\definecolor{currentfill}{rgb}{0.000000,0.000000,0.000000}%
\pgfsetfillcolor{currentfill}%
\pgfsetlinewidth{0.803000pt}%
\definecolor{currentstroke}{rgb}{0.000000,0.000000,0.000000}%
\pgfsetstrokecolor{currentstroke}%
\pgfsetdash{}{0pt}%
\pgfsys@defobject{currentmarker}{\pgfqpoint{0.000000in}{-0.048611in}}{\pgfqpoint{0.000000in}{0.000000in}}{%
\pgfpathmoveto{\pgfqpoint{0.000000in}{0.000000in}}%
\pgfpathlineto{\pgfqpoint{0.000000in}{-0.048611in}}%
\pgfusepath{stroke,fill}%
}%
\begin{pgfscope}%
\pgfsys@transformshift{1.281818in}{0.330000in}%
\pgfsys@useobject{currentmarker}{}%
\end{pgfscope}%
\end{pgfscope}%
\begin{pgfscope}%
\pgfsetbuttcap%
\pgfsetroundjoin%
\definecolor{currentfill}{rgb}{0.000000,0.000000,0.000000}%
\pgfsetfillcolor{currentfill}%
\pgfsetlinewidth{0.803000pt}%
\definecolor{currentstroke}{rgb}{0.000000,0.000000,0.000000}%
\pgfsetstrokecolor{currentstroke}%
\pgfsetdash{}{0pt}%
\pgfsys@defobject{currentmarker}{\pgfqpoint{0.000000in}{0.000000in}}{\pgfqpoint{0.000000in}{0.048611in}}{%
\pgfpathmoveto{\pgfqpoint{0.000000in}{0.000000in}}%
\pgfpathlineto{\pgfqpoint{0.000000in}{0.048611in}}%
\pgfusepath{stroke,fill}%
}%
\begin{pgfscope}%
\pgfsys@transformshift{1.281818in}{2.640000in}%
\pgfsys@useobject{currentmarker}{}%
\end{pgfscope}%
\end{pgfscope}%
\begin{pgfscope}%
\definecolor{textcolor}{rgb}{0.000000,0.000000,0.000000}%
\pgfsetstrokecolor{textcolor}%
\pgfsetfillcolor{textcolor}%
\pgftext[x=1.281818in,y=0.232778in,,top]{\color{textcolor}\rmfamily\fontsize{10.000000}{12.000000}\selectfont \(\displaystyle {0}\)}%
\end{pgfscope}%
\begin{pgfscope}%
\pgfpathrectangle{\pgfqpoint{1.000000in}{0.330000in}}{\pgfqpoint{6.200000in}{2.310000in}}%
\pgfusepath{clip}%
\pgfsetbuttcap%
\pgfsetroundjoin%
\pgfsetlinewidth{0.803000pt}%
\definecolor{currentstroke}{rgb}{0.000000,0.000000,0.000000}%
\pgfsetstrokecolor{currentstroke}%
\pgfsetdash{{0.800000pt}{1.320000pt}}{0.000000pt}%
\pgfpathmoveto{\pgfqpoint{2.098682in}{0.330000in}}%
\pgfpathlineto{\pgfqpoint{2.098682in}{2.640000in}}%
\pgfusepath{stroke}%
\end{pgfscope}%
\begin{pgfscope}%
\pgfsetbuttcap%
\pgfsetroundjoin%
\definecolor{currentfill}{rgb}{0.000000,0.000000,0.000000}%
\pgfsetfillcolor{currentfill}%
\pgfsetlinewidth{0.803000pt}%
\definecolor{currentstroke}{rgb}{0.000000,0.000000,0.000000}%
\pgfsetstrokecolor{currentstroke}%
\pgfsetdash{}{0pt}%
\pgfsys@defobject{currentmarker}{\pgfqpoint{0.000000in}{-0.048611in}}{\pgfqpoint{0.000000in}{0.000000in}}{%
\pgfpathmoveto{\pgfqpoint{0.000000in}{0.000000in}}%
\pgfpathlineto{\pgfqpoint{0.000000in}{-0.048611in}}%
\pgfusepath{stroke,fill}%
}%
\begin{pgfscope}%
\pgfsys@transformshift{2.098682in}{0.330000in}%
\pgfsys@useobject{currentmarker}{}%
\end{pgfscope}%
\end{pgfscope}%
\begin{pgfscope}%
\pgfsetbuttcap%
\pgfsetroundjoin%
\definecolor{currentfill}{rgb}{0.000000,0.000000,0.000000}%
\pgfsetfillcolor{currentfill}%
\pgfsetlinewidth{0.803000pt}%
\definecolor{currentstroke}{rgb}{0.000000,0.000000,0.000000}%
\pgfsetstrokecolor{currentstroke}%
\pgfsetdash{}{0pt}%
\pgfsys@defobject{currentmarker}{\pgfqpoint{0.000000in}{0.000000in}}{\pgfqpoint{0.000000in}{0.048611in}}{%
\pgfpathmoveto{\pgfqpoint{0.000000in}{0.000000in}}%
\pgfpathlineto{\pgfqpoint{0.000000in}{0.048611in}}%
\pgfusepath{stroke,fill}%
}%
\begin{pgfscope}%
\pgfsys@transformshift{2.098682in}{2.640000in}%
\pgfsys@useobject{currentmarker}{}%
\end{pgfscope}%
\end{pgfscope}%
\begin{pgfscope}%
\definecolor{textcolor}{rgb}{0.000000,0.000000,0.000000}%
\pgfsetstrokecolor{textcolor}%
\pgfsetfillcolor{textcolor}%
\pgftext[x=2.098682in,y=0.232778in,,top]{\color{textcolor}\rmfamily\fontsize{10.000000}{12.000000}\selectfont \(\displaystyle {5}\)}%
\end{pgfscope}%
\begin{pgfscope}%
\pgfpathrectangle{\pgfqpoint{1.000000in}{0.330000in}}{\pgfqpoint{6.200000in}{2.310000in}}%
\pgfusepath{clip}%
\pgfsetbuttcap%
\pgfsetroundjoin%
\pgfsetlinewidth{0.803000pt}%
\definecolor{currentstroke}{rgb}{0.000000,0.000000,0.000000}%
\pgfsetstrokecolor{currentstroke}%
\pgfsetdash{{0.800000pt}{1.320000pt}}{0.000000pt}%
\pgfpathmoveto{\pgfqpoint{2.915547in}{0.330000in}}%
\pgfpathlineto{\pgfqpoint{2.915547in}{2.640000in}}%
\pgfusepath{stroke}%
\end{pgfscope}%
\begin{pgfscope}%
\pgfsetbuttcap%
\pgfsetroundjoin%
\definecolor{currentfill}{rgb}{0.000000,0.000000,0.000000}%
\pgfsetfillcolor{currentfill}%
\pgfsetlinewidth{0.803000pt}%
\definecolor{currentstroke}{rgb}{0.000000,0.000000,0.000000}%
\pgfsetstrokecolor{currentstroke}%
\pgfsetdash{}{0pt}%
\pgfsys@defobject{currentmarker}{\pgfqpoint{0.000000in}{-0.048611in}}{\pgfqpoint{0.000000in}{0.000000in}}{%
\pgfpathmoveto{\pgfqpoint{0.000000in}{0.000000in}}%
\pgfpathlineto{\pgfqpoint{0.000000in}{-0.048611in}}%
\pgfusepath{stroke,fill}%
}%
\begin{pgfscope}%
\pgfsys@transformshift{2.915547in}{0.330000in}%
\pgfsys@useobject{currentmarker}{}%
\end{pgfscope}%
\end{pgfscope}%
\begin{pgfscope}%
\pgfsetbuttcap%
\pgfsetroundjoin%
\definecolor{currentfill}{rgb}{0.000000,0.000000,0.000000}%
\pgfsetfillcolor{currentfill}%
\pgfsetlinewidth{0.803000pt}%
\definecolor{currentstroke}{rgb}{0.000000,0.000000,0.000000}%
\pgfsetstrokecolor{currentstroke}%
\pgfsetdash{}{0pt}%
\pgfsys@defobject{currentmarker}{\pgfqpoint{0.000000in}{0.000000in}}{\pgfqpoint{0.000000in}{0.048611in}}{%
\pgfpathmoveto{\pgfqpoint{0.000000in}{0.000000in}}%
\pgfpathlineto{\pgfqpoint{0.000000in}{0.048611in}}%
\pgfusepath{stroke,fill}%
}%
\begin{pgfscope}%
\pgfsys@transformshift{2.915547in}{2.640000in}%
\pgfsys@useobject{currentmarker}{}%
\end{pgfscope}%
\end{pgfscope}%
\begin{pgfscope}%
\definecolor{textcolor}{rgb}{0.000000,0.000000,0.000000}%
\pgfsetstrokecolor{textcolor}%
\pgfsetfillcolor{textcolor}%
\pgftext[x=2.915547in,y=0.232778in,,top]{\color{textcolor}\rmfamily\fontsize{10.000000}{12.000000}\selectfont \(\displaystyle {10}\)}%
\end{pgfscope}%
\begin{pgfscope}%
\pgfpathrectangle{\pgfqpoint{1.000000in}{0.330000in}}{\pgfqpoint{6.200000in}{2.310000in}}%
\pgfusepath{clip}%
\pgfsetbuttcap%
\pgfsetroundjoin%
\pgfsetlinewidth{0.803000pt}%
\definecolor{currentstroke}{rgb}{0.000000,0.000000,0.000000}%
\pgfsetstrokecolor{currentstroke}%
\pgfsetdash{{0.800000pt}{1.320000pt}}{0.000000pt}%
\pgfpathmoveto{\pgfqpoint{3.732411in}{0.330000in}}%
\pgfpathlineto{\pgfqpoint{3.732411in}{2.640000in}}%
\pgfusepath{stroke}%
\end{pgfscope}%
\begin{pgfscope}%
\pgfsetbuttcap%
\pgfsetroundjoin%
\definecolor{currentfill}{rgb}{0.000000,0.000000,0.000000}%
\pgfsetfillcolor{currentfill}%
\pgfsetlinewidth{0.803000pt}%
\definecolor{currentstroke}{rgb}{0.000000,0.000000,0.000000}%
\pgfsetstrokecolor{currentstroke}%
\pgfsetdash{}{0pt}%
\pgfsys@defobject{currentmarker}{\pgfqpoint{0.000000in}{-0.048611in}}{\pgfqpoint{0.000000in}{0.000000in}}{%
\pgfpathmoveto{\pgfqpoint{0.000000in}{0.000000in}}%
\pgfpathlineto{\pgfqpoint{0.000000in}{-0.048611in}}%
\pgfusepath{stroke,fill}%
}%
\begin{pgfscope}%
\pgfsys@transformshift{3.732411in}{0.330000in}%
\pgfsys@useobject{currentmarker}{}%
\end{pgfscope}%
\end{pgfscope}%
\begin{pgfscope}%
\pgfsetbuttcap%
\pgfsetroundjoin%
\definecolor{currentfill}{rgb}{0.000000,0.000000,0.000000}%
\pgfsetfillcolor{currentfill}%
\pgfsetlinewidth{0.803000pt}%
\definecolor{currentstroke}{rgb}{0.000000,0.000000,0.000000}%
\pgfsetstrokecolor{currentstroke}%
\pgfsetdash{}{0pt}%
\pgfsys@defobject{currentmarker}{\pgfqpoint{0.000000in}{0.000000in}}{\pgfqpoint{0.000000in}{0.048611in}}{%
\pgfpathmoveto{\pgfqpoint{0.000000in}{0.000000in}}%
\pgfpathlineto{\pgfqpoint{0.000000in}{0.048611in}}%
\pgfusepath{stroke,fill}%
}%
\begin{pgfscope}%
\pgfsys@transformshift{3.732411in}{2.640000in}%
\pgfsys@useobject{currentmarker}{}%
\end{pgfscope}%
\end{pgfscope}%
\begin{pgfscope}%
\definecolor{textcolor}{rgb}{0.000000,0.000000,0.000000}%
\pgfsetstrokecolor{textcolor}%
\pgfsetfillcolor{textcolor}%
\pgftext[x=3.732411in,y=0.232778in,,top]{\color{textcolor}\rmfamily\fontsize{10.000000}{12.000000}\selectfont \(\displaystyle {15}\)}%
\end{pgfscope}%
\begin{pgfscope}%
\pgfpathrectangle{\pgfqpoint{1.000000in}{0.330000in}}{\pgfqpoint{6.200000in}{2.310000in}}%
\pgfusepath{clip}%
\pgfsetbuttcap%
\pgfsetroundjoin%
\pgfsetlinewidth{0.803000pt}%
\definecolor{currentstroke}{rgb}{0.000000,0.000000,0.000000}%
\pgfsetstrokecolor{currentstroke}%
\pgfsetdash{{0.800000pt}{1.320000pt}}{0.000000pt}%
\pgfpathmoveto{\pgfqpoint{4.549275in}{0.330000in}}%
\pgfpathlineto{\pgfqpoint{4.549275in}{2.640000in}}%
\pgfusepath{stroke}%
\end{pgfscope}%
\begin{pgfscope}%
\pgfsetbuttcap%
\pgfsetroundjoin%
\definecolor{currentfill}{rgb}{0.000000,0.000000,0.000000}%
\pgfsetfillcolor{currentfill}%
\pgfsetlinewidth{0.803000pt}%
\definecolor{currentstroke}{rgb}{0.000000,0.000000,0.000000}%
\pgfsetstrokecolor{currentstroke}%
\pgfsetdash{}{0pt}%
\pgfsys@defobject{currentmarker}{\pgfqpoint{0.000000in}{-0.048611in}}{\pgfqpoint{0.000000in}{0.000000in}}{%
\pgfpathmoveto{\pgfqpoint{0.000000in}{0.000000in}}%
\pgfpathlineto{\pgfqpoint{0.000000in}{-0.048611in}}%
\pgfusepath{stroke,fill}%
}%
\begin{pgfscope}%
\pgfsys@transformshift{4.549275in}{0.330000in}%
\pgfsys@useobject{currentmarker}{}%
\end{pgfscope}%
\end{pgfscope}%
\begin{pgfscope}%
\pgfsetbuttcap%
\pgfsetroundjoin%
\definecolor{currentfill}{rgb}{0.000000,0.000000,0.000000}%
\pgfsetfillcolor{currentfill}%
\pgfsetlinewidth{0.803000pt}%
\definecolor{currentstroke}{rgb}{0.000000,0.000000,0.000000}%
\pgfsetstrokecolor{currentstroke}%
\pgfsetdash{}{0pt}%
\pgfsys@defobject{currentmarker}{\pgfqpoint{0.000000in}{0.000000in}}{\pgfqpoint{0.000000in}{0.048611in}}{%
\pgfpathmoveto{\pgfqpoint{0.000000in}{0.000000in}}%
\pgfpathlineto{\pgfqpoint{0.000000in}{0.048611in}}%
\pgfusepath{stroke,fill}%
}%
\begin{pgfscope}%
\pgfsys@transformshift{4.549275in}{2.640000in}%
\pgfsys@useobject{currentmarker}{}%
\end{pgfscope}%
\end{pgfscope}%
\begin{pgfscope}%
\definecolor{textcolor}{rgb}{0.000000,0.000000,0.000000}%
\pgfsetstrokecolor{textcolor}%
\pgfsetfillcolor{textcolor}%
\pgftext[x=4.549275in,y=0.232778in,,top]{\color{textcolor}\rmfamily\fontsize{10.000000}{12.000000}\selectfont \(\displaystyle {20}\)}%
\end{pgfscope}%
\begin{pgfscope}%
\pgfpathrectangle{\pgfqpoint{1.000000in}{0.330000in}}{\pgfqpoint{6.200000in}{2.310000in}}%
\pgfusepath{clip}%
\pgfsetbuttcap%
\pgfsetroundjoin%
\pgfsetlinewidth{0.803000pt}%
\definecolor{currentstroke}{rgb}{0.000000,0.000000,0.000000}%
\pgfsetstrokecolor{currentstroke}%
\pgfsetdash{{0.800000pt}{1.320000pt}}{0.000000pt}%
\pgfpathmoveto{\pgfqpoint{5.366140in}{0.330000in}}%
\pgfpathlineto{\pgfqpoint{5.366140in}{2.640000in}}%
\pgfusepath{stroke}%
\end{pgfscope}%
\begin{pgfscope}%
\pgfsetbuttcap%
\pgfsetroundjoin%
\definecolor{currentfill}{rgb}{0.000000,0.000000,0.000000}%
\pgfsetfillcolor{currentfill}%
\pgfsetlinewidth{0.803000pt}%
\definecolor{currentstroke}{rgb}{0.000000,0.000000,0.000000}%
\pgfsetstrokecolor{currentstroke}%
\pgfsetdash{}{0pt}%
\pgfsys@defobject{currentmarker}{\pgfqpoint{0.000000in}{-0.048611in}}{\pgfqpoint{0.000000in}{0.000000in}}{%
\pgfpathmoveto{\pgfqpoint{0.000000in}{0.000000in}}%
\pgfpathlineto{\pgfqpoint{0.000000in}{-0.048611in}}%
\pgfusepath{stroke,fill}%
}%
\begin{pgfscope}%
\pgfsys@transformshift{5.366140in}{0.330000in}%
\pgfsys@useobject{currentmarker}{}%
\end{pgfscope}%
\end{pgfscope}%
\begin{pgfscope}%
\pgfsetbuttcap%
\pgfsetroundjoin%
\definecolor{currentfill}{rgb}{0.000000,0.000000,0.000000}%
\pgfsetfillcolor{currentfill}%
\pgfsetlinewidth{0.803000pt}%
\definecolor{currentstroke}{rgb}{0.000000,0.000000,0.000000}%
\pgfsetstrokecolor{currentstroke}%
\pgfsetdash{}{0pt}%
\pgfsys@defobject{currentmarker}{\pgfqpoint{0.000000in}{0.000000in}}{\pgfqpoint{0.000000in}{0.048611in}}{%
\pgfpathmoveto{\pgfqpoint{0.000000in}{0.000000in}}%
\pgfpathlineto{\pgfqpoint{0.000000in}{0.048611in}}%
\pgfusepath{stroke,fill}%
}%
\begin{pgfscope}%
\pgfsys@transformshift{5.366140in}{2.640000in}%
\pgfsys@useobject{currentmarker}{}%
\end{pgfscope}%
\end{pgfscope}%
\begin{pgfscope}%
\definecolor{textcolor}{rgb}{0.000000,0.000000,0.000000}%
\pgfsetstrokecolor{textcolor}%
\pgfsetfillcolor{textcolor}%
\pgftext[x=5.366140in,y=0.232778in,,top]{\color{textcolor}\rmfamily\fontsize{10.000000}{12.000000}\selectfont \(\displaystyle {25}\)}%
\end{pgfscope}%
\begin{pgfscope}%
\pgfpathrectangle{\pgfqpoint{1.000000in}{0.330000in}}{\pgfqpoint{6.200000in}{2.310000in}}%
\pgfusepath{clip}%
\pgfsetbuttcap%
\pgfsetroundjoin%
\pgfsetlinewidth{0.803000pt}%
\definecolor{currentstroke}{rgb}{0.000000,0.000000,0.000000}%
\pgfsetstrokecolor{currentstroke}%
\pgfsetdash{{0.800000pt}{1.320000pt}}{0.000000pt}%
\pgfpathmoveto{\pgfqpoint{6.183004in}{0.330000in}}%
\pgfpathlineto{\pgfqpoint{6.183004in}{2.640000in}}%
\pgfusepath{stroke}%
\end{pgfscope}%
\begin{pgfscope}%
\pgfsetbuttcap%
\pgfsetroundjoin%
\definecolor{currentfill}{rgb}{0.000000,0.000000,0.000000}%
\pgfsetfillcolor{currentfill}%
\pgfsetlinewidth{0.803000pt}%
\definecolor{currentstroke}{rgb}{0.000000,0.000000,0.000000}%
\pgfsetstrokecolor{currentstroke}%
\pgfsetdash{}{0pt}%
\pgfsys@defobject{currentmarker}{\pgfqpoint{0.000000in}{-0.048611in}}{\pgfqpoint{0.000000in}{0.000000in}}{%
\pgfpathmoveto{\pgfqpoint{0.000000in}{0.000000in}}%
\pgfpathlineto{\pgfqpoint{0.000000in}{-0.048611in}}%
\pgfusepath{stroke,fill}%
}%
\begin{pgfscope}%
\pgfsys@transformshift{6.183004in}{0.330000in}%
\pgfsys@useobject{currentmarker}{}%
\end{pgfscope}%
\end{pgfscope}%
\begin{pgfscope}%
\pgfsetbuttcap%
\pgfsetroundjoin%
\definecolor{currentfill}{rgb}{0.000000,0.000000,0.000000}%
\pgfsetfillcolor{currentfill}%
\pgfsetlinewidth{0.803000pt}%
\definecolor{currentstroke}{rgb}{0.000000,0.000000,0.000000}%
\pgfsetstrokecolor{currentstroke}%
\pgfsetdash{}{0pt}%
\pgfsys@defobject{currentmarker}{\pgfqpoint{0.000000in}{0.000000in}}{\pgfqpoint{0.000000in}{0.048611in}}{%
\pgfpathmoveto{\pgfqpoint{0.000000in}{0.000000in}}%
\pgfpathlineto{\pgfqpoint{0.000000in}{0.048611in}}%
\pgfusepath{stroke,fill}%
}%
\begin{pgfscope}%
\pgfsys@transformshift{6.183004in}{2.640000in}%
\pgfsys@useobject{currentmarker}{}%
\end{pgfscope}%
\end{pgfscope}%
\begin{pgfscope}%
\definecolor{textcolor}{rgb}{0.000000,0.000000,0.000000}%
\pgfsetstrokecolor{textcolor}%
\pgfsetfillcolor{textcolor}%
\pgftext[x=6.183004in,y=0.232778in,,top]{\color{textcolor}\rmfamily\fontsize{10.000000}{12.000000}\selectfont \(\displaystyle {30}\)}%
\end{pgfscope}%
\begin{pgfscope}%
\pgfpathrectangle{\pgfqpoint{1.000000in}{0.330000in}}{\pgfqpoint{6.200000in}{2.310000in}}%
\pgfusepath{clip}%
\pgfsetbuttcap%
\pgfsetroundjoin%
\pgfsetlinewidth{0.803000pt}%
\definecolor{currentstroke}{rgb}{0.000000,0.000000,0.000000}%
\pgfsetstrokecolor{currentstroke}%
\pgfsetdash{{0.800000pt}{1.320000pt}}{0.000000pt}%
\pgfpathmoveto{\pgfqpoint{6.999868in}{0.330000in}}%
\pgfpathlineto{\pgfqpoint{6.999868in}{2.640000in}}%
\pgfusepath{stroke}%
\end{pgfscope}%
\begin{pgfscope}%
\pgfsetbuttcap%
\pgfsetroundjoin%
\definecolor{currentfill}{rgb}{0.000000,0.000000,0.000000}%
\pgfsetfillcolor{currentfill}%
\pgfsetlinewidth{0.803000pt}%
\definecolor{currentstroke}{rgb}{0.000000,0.000000,0.000000}%
\pgfsetstrokecolor{currentstroke}%
\pgfsetdash{}{0pt}%
\pgfsys@defobject{currentmarker}{\pgfqpoint{0.000000in}{-0.048611in}}{\pgfqpoint{0.000000in}{0.000000in}}{%
\pgfpathmoveto{\pgfqpoint{0.000000in}{0.000000in}}%
\pgfpathlineto{\pgfqpoint{0.000000in}{-0.048611in}}%
\pgfusepath{stroke,fill}%
}%
\begin{pgfscope}%
\pgfsys@transformshift{6.999868in}{0.330000in}%
\pgfsys@useobject{currentmarker}{}%
\end{pgfscope}%
\end{pgfscope}%
\begin{pgfscope}%
\pgfsetbuttcap%
\pgfsetroundjoin%
\definecolor{currentfill}{rgb}{0.000000,0.000000,0.000000}%
\pgfsetfillcolor{currentfill}%
\pgfsetlinewidth{0.803000pt}%
\definecolor{currentstroke}{rgb}{0.000000,0.000000,0.000000}%
\pgfsetstrokecolor{currentstroke}%
\pgfsetdash{}{0pt}%
\pgfsys@defobject{currentmarker}{\pgfqpoint{0.000000in}{0.000000in}}{\pgfqpoint{0.000000in}{0.048611in}}{%
\pgfpathmoveto{\pgfqpoint{0.000000in}{0.000000in}}%
\pgfpathlineto{\pgfqpoint{0.000000in}{0.048611in}}%
\pgfusepath{stroke,fill}%
}%
\begin{pgfscope}%
\pgfsys@transformshift{6.999868in}{2.640000in}%
\pgfsys@useobject{currentmarker}{}%
\end{pgfscope}%
\end{pgfscope}%
\begin{pgfscope}%
\definecolor{textcolor}{rgb}{0.000000,0.000000,0.000000}%
\pgfsetstrokecolor{textcolor}%
\pgfsetfillcolor{textcolor}%
\pgftext[x=6.999868in,y=0.232778in,,top]{\color{textcolor}\rmfamily\fontsize{10.000000}{12.000000}\selectfont \(\displaystyle {35}\)}%
\end{pgfscope}%
\begin{pgfscope}%
\pgfpathrectangle{\pgfqpoint{1.000000in}{0.330000in}}{\pgfqpoint{6.200000in}{2.310000in}}%
\pgfusepath{clip}%
\pgfsetbuttcap%
\pgfsetroundjoin%
\pgfsetlinewidth{0.803000pt}%
\definecolor{currentstroke}{rgb}{0.000000,0.000000,0.000000}%
\pgfsetstrokecolor{currentstroke}%
\pgfsetdash{{0.800000pt}{1.320000pt}}{0.000000pt}%
\pgfpathmoveto{\pgfqpoint{1.000000in}{0.476629in}}%
\pgfpathlineto{\pgfqpoint{7.200000in}{0.476629in}}%
\pgfusepath{stroke}%
\end{pgfscope}%
\begin{pgfscope}%
\pgfsetbuttcap%
\pgfsetroundjoin%
\definecolor{currentfill}{rgb}{0.000000,0.000000,0.000000}%
\pgfsetfillcolor{currentfill}%
\pgfsetlinewidth{0.803000pt}%
\definecolor{currentstroke}{rgb}{0.000000,0.000000,0.000000}%
\pgfsetstrokecolor{currentstroke}%
\pgfsetdash{}{0pt}%
\pgfsys@defobject{currentmarker}{\pgfqpoint{-0.048611in}{0.000000in}}{\pgfqpoint{-0.000000in}{0.000000in}}{%
\pgfpathmoveto{\pgfqpoint{-0.000000in}{0.000000in}}%
\pgfpathlineto{\pgfqpoint{-0.048611in}{0.000000in}}%
\pgfusepath{stroke,fill}%
}%
\begin{pgfscope}%
\pgfsys@transformshift{1.000000in}{0.476629in}%
\pgfsys@useobject{currentmarker}{}%
\end{pgfscope}%
\end{pgfscope}%
\begin{pgfscope}%
\pgfsetbuttcap%
\pgfsetroundjoin%
\definecolor{currentfill}{rgb}{0.000000,0.000000,0.000000}%
\pgfsetfillcolor{currentfill}%
\pgfsetlinewidth{0.803000pt}%
\definecolor{currentstroke}{rgb}{0.000000,0.000000,0.000000}%
\pgfsetstrokecolor{currentstroke}%
\pgfsetdash{}{0pt}%
\pgfsys@defobject{currentmarker}{\pgfqpoint{0.000000in}{0.000000in}}{\pgfqpoint{0.048611in}{0.000000in}}{%
\pgfpathmoveto{\pgfqpoint{0.000000in}{0.000000in}}%
\pgfpathlineto{\pgfqpoint{0.048611in}{0.000000in}}%
\pgfusepath{stroke,fill}%
}%
\begin{pgfscope}%
\pgfsys@transformshift{7.200000in}{0.476629in}%
\pgfsys@useobject{currentmarker}{}%
\end{pgfscope}%
\end{pgfscope}%
\begin{pgfscope}%
\definecolor{textcolor}{rgb}{0.000000,0.000000,0.000000}%
\pgfsetstrokecolor{textcolor}%
\pgfsetfillcolor{textcolor}%
\pgftext[x=0.655863in, y=0.428404in, left, base]{\color{textcolor}\rmfamily\fontsize{10.000000}{12.000000}\selectfont \(\displaystyle {\ensuremath{-}75}\)}%
\end{pgfscope}%
\begin{pgfscope}%
\pgfpathrectangle{\pgfqpoint{1.000000in}{0.330000in}}{\pgfqpoint{6.200000in}{2.310000in}}%
\pgfusepath{clip}%
\pgfsetbuttcap%
\pgfsetroundjoin%
\pgfsetlinewidth{0.803000pt}%
\definecolor{currentstroke}{rgb}{0.000000,0.000000,0.000000}%
\pgfsetstrokecolor{currentstroke}%
\pgfsetdash{{0.800000pt}{1.320000pt}}{0.000000pt}%
\pgfpathmoveto{\pgfqpoint{1.000000in}{0.808784in}}%
\pgfpathlineto{\pgfqpoint{7.200000in}{0.808784in}}%
\pgfusepath{stroke}%
\end{pgfscope}%
\begin{pgfscope}%
\pgfsetbuttcap%
\pgfsetroundjoin%
\definecolor{currentfill}{rgb}{0.000000,0.000000,0.000000}%
\pgfsetfillcolor{currentfill}%
\pgfsetlinewidth{0.803000pt}%
\definecolor{currentstroke}{rgb}{0.000000,0.000000,0.000000}%
\pgfsetstrokecolor{currentstroke}%
\pgfsetdash{}{0pt}%
\pgfsys@defobject{currentmarker}{\pgfqpoint{-0.048611in}{0.000000in}}{\pgfqpoint{-0.000000in}{0.000000in}}{%
\pgfpathmoveto{\pgfqpoint{-0.000000in}{0.000000in}}%
\pgfpathlineto{\pgfqpoint{-0.048611in}{0.000000in}}%
\pgfusepath{stroke,fill}%
}%
\begin{pgfscope}%
\pgfsys@transformshift{1.000000in}{0.808784in}%
\pgfsys@useobject{currentmarker}{}%
\end{pgfscope}%
\end{pgfscope}%
\begin{pgfscope}%
\pgfsetbuttcap%
\pgfsetroundjoin%
\definecolor{currentfill}{rgb}{0.000000,0.000000,0.000000}%
\pgfsetfillcolor{currentfill}%
\pgfsetlinewidth{0.803000pt}%
\definecolor{currentstroke}{rgb}{0.000000,0.000000,0.000000}%
\pgfsetstrokecolor{currentstroke}%
\pgfsetdash{}{0pt}%
\pgfsys@defobject{currentmarker}{\pgfqpoint{0.000000in}{0.000000in}}{\pgfqpoint{0.048611in}{0.000000in}}{%
\pgfpathmoveto{\pgfqpoint{0.000000in}{0.000000in}}%
\pgfpathlineto{\pgfqpoint{0.048611in}{0.000000in}}%
\pgfusepath{stroke,fill}%
}%
\begin{pgfscope}%
\pgfsys@transformshift{7.200000in}{0.808784in}%
\pgfsys@useobject{currentmarker}{}%
\end{pgfscope}%
\end{pgfscope}%
\begin{pgfscope}%
\definecolor{textcolor}{rgb}{0.000000,0.000000,0.000000}%
\pgfsetstrokecolor{textcolor}%
\pgfsetfillcolor{textcolor}%
\pgftext[x=0.655863in, y=0.760559in, left, base]{\color{textcolor}\rmfamily\fontsize{10.000000}{12.000000}\selectfont \(\displaystyle {\ensuremath{-}50}\)}%
\end{pgfscope}%
\begin{pgfscope}%
\pgfpathrectangle{\pgfqpoint{1.000000in}{0.330000in}}{\pgfqpoint{6.200000in}{2.310000in}}%
\pgfusepath{clip}%
\pgfsetbuttcap%
\pgfsetroundjoin%
\pgfsetlinewidth{0.803000pt}%
\definecolor{currentstroke}{rgb}{0.000000,0.000000,0.000000}%
\pgfsetstrokecolor{currentstroke}%
\pgfsetdash{{0.800000pt}{1.320000pt}}{0.000000pt}%
\pgfpathmoveto{\pgfqpoint{1.000000in}{1.140939in}}%
\pgfpathlineto{\pgfqpoint{7.200000in}{1.140939in}}%
\pgfusepath{stroke}%
\end{pgfscope}%
\begin{pgfscope}%
\pgfsetbuttcap%
\pgfsetroundjoin%
\definecolor{currentfill}{rgb}{0.000000,0.000000,0.000000}%
\pgfsetfillcolor{currentfill}%
\pgfsetlinewidth{0.803000pt}%
\definecolor{currentstroke}{rgb}{0.000000,0.000000,0.000000}%
\pgfsetstrokecolor{currentstroke}%
\pgfsetdash{}{0pt}%
\pgfsys@defobject{currentmarker}{\pgfqpoint{-0.048611in}{0.000000in}}{\pgfqpoint{-0.000000in}{0.000000in}}{%
\pgfpathmoveto{\pgfqpoint{-0.000000in}{0.000000in}}%
\pgfpathlineto{\pgfqpoint{-0.048611in}{0.000000in}}%
\pgfusepath{stroke,fill}%
}%
\begin{pgfscope}%
\pgfsys@transformshift{1.000000in}{1.140939in}%
\pgfsys@useobject{currentmarker}{}%
\end{pgfscope}%
\end{pgfscope}%
\begin{pgfscope}%
\pgfsetbuttcap%
\pgfsetroundjoin%
\definecolor{currentfill}{rgb}{0.000000,0.000000,0.000000}%
\pgfsetfillcolor{currentfill}%
\pgfsetlinewidth{0.803000pt}%
\definecolor{currentstroke}{rgb}{0.000000,0.000000,0.000000}%
\pgfsetstrokecolor{currentstroke}%
\pgfsetdash{}{0pt}%
\pgfsys@defobject{currentmarker}{\pgfqpoint{0.000000in}{0.000000in}}{\pgfqpoint{0.048611in}{0.000000in}}{%
\pgfpathmoveto{\pgfqpoint{0.000000in}{0.000000in}}%
\pgfpathlineto{\pgfqpoint{0.048611in}{0.000000in}}%
\pgfusepath{stroke,fill}%
}%
\begin{pgfscope}%
\pgfsys@transformshift{7.200000in}{1.140939in}%
\pgfsys@useobject{currentmarker}{}%
\end{pgfscope}%
\end{pgfscope}%
\begin{pgfscope}%
\definecolor{textcolor}{rgb}{0.000000,0.000000,0.000000}%
\pgfsetstrokecolor{textcolor}%
\pgfsetfillcolor{textcolor}%
\pgftext[x=0.655863in, y=1.092714in, left, base]{\color{textcolor}\rmfamily\fontsize{10.000000}{12.000000}\selectfont \(\displaystyle {\ensuremath{-}25}\)}%
\end{pgfscope}%
\begin{pgfscope}%
\pgfpathrectangle{\pgfqpoint{1.000000in}{0.330000in}}{\pgfqpoint{6.200000in}{2.310000in}}%
\pgfusepath{clip}%
\pgfsetbuttcap%
\pgfsetroundjoin%
\pgfsetlinewidth{0.803000pt}%
\definecolor{currentstroke}{rgb}{0.000000,0.000000,0.000000}%
\pgfsetstrokecolor{currentstroke}%
\pgfsetdash{{0.800000pt}{1.320000pt}}{0.000000pt}%
\pgfpathmoveto{\pgfqpoint{1.000000in}{1.473094in}}%
\pgfpathlineto{\pgfqpoint{7.200000in}{1.473094in}}%
\pgfusepath{stroke}%
\end{pgfscope}%
\begin{pgfscope}%
\pgfsetbuttcap%
\pgfsetroundjoin%
\definecolor{currentfill}{rgb}{0.000000,0.000000,0.000000}%
\pgfsetfillcolor{currentfill}%
\pgfsetlinewidth{0.803000pt}%
\definecolor{currentstroke}{rgb}{0.000000,0.000000,0.000000}%
\pgfsetstrokecolor{currentstroke}%
\pgfsetdash{}{0pt}%
\pgfsys@defobject{currentmarker}{\pgfqpoint{-0.048611in}{0.000000in}}{\pgfqpoint{-0.000000in}{0.000000in}}{%
\pgfpathmoveto{\pgfqpoint{-0.000000in}{0.000000in}}%
\pgfpathlineto{\pgfqpoint{-0.048611in}{0.000000in}}%
\pgfusepath{stroke,fill}%
}%
\begin{pgfscope}%
\pgfsys@transformshift{1.000000in}{1.473094in}%
\pgfsys@useobject{currentmarker}{}%
\end{pgfscope}%
\end{pgfscope}%
\begin{pgfscope}%
\pgfsetbuttcap%
\pgfsetroundjoin%
\definecolor{currentfill}{rgb}{0.000000,0.000000,0.000000}%
\pgfsetfillcolor{currentfill}%
\pgfsetlinewidth{0.803000pt}%
\definecolor{currentstroke}{rgb}{0.000000,0.000000,0.000000}%
\pgfsetstrokecolor{currentstroke}%
\pgfsetdash{}{0pt}%
\pgfsys@defobject{currentmarker}{\pgfqpoint{0.000000in}{0.000000in}}{\pgfqpoint{0.048611in}{0.000000in}}{%
\pgfpathmoveto{\pgfqpoint{0.000000in}{0.000000in}}%
\pgfpathlineto{\pgfqpoint{0.048611in}{0.000000in}}%
\pgfusepath{stroke,fill}%
}%
\begin{pgfscope}%
\pgfsys@transformshift{7.200000in}{1.473094in}%
\pgfsys@useobject{currentmarker}{}%
\end{pgfscope}%
\end{pgfscope}%
\begin{pgfscope}%
\definecolor{textcolor}{rgb}{0.000000,0.000000,0.000000}%
\pgfsetstrokecolor{textcolor}%
\pgfsetfillcolor{textcolor}%
\pgftext[x=0.833333in, y=1.424869in, left, base]{\color{textcolor}\rmfamily\fontsize{10.000000}{12.000000}\selectfont \(\displaystyle {0}\)}%
\end{pgfscope}%
\begin{pgfscope}%
\pgfpathrectangle{\pgfqpoint{1.000000in}{0.330000in}}{\pgfqpoint{6.200000in}{2.310000in}}%
\pgfusepath{clip}%
\pgfsetbuttcap%
\pgfsetroundjoin%
\pgfsetlinewidth{0.803000pt}%
\definecolor{currentstroke}{rgb}{0.000000,0.000000,0.000000}%
\pgfsetstrokecolor{currentstroke}%
\pgfsetdash{{0.800000pt}{1.320000pt}}{0.000000pt}%
\pgfpathmoveto{\pgfqpoint{1.000000in}{1.805250in}}%
\pgfpathlineto{\pgfqpoint{7.200000in}{1.805250in}}%
\pgfusepath{stroke}%
\end{pgfscope}%
\begin{pgfscope}%
\pgfsetbuttcap%
\pgfsetroundjoin%
\definecolor{currentfill}{rgb}{0.000000,0.000000,0.000000}%
\pgfsetfillcolor{currentfill}%
\pgfsetlinewidth{0.803000pt}%
\definecolor{currentstroke}{rgb}{0.000000,0.000000,0.000000}%
\pgfsetstrokecolor{currentstroke}%
\pgfsetdash{}{0pt}%
\pgfsys@defobject{currentmarker}{\pgfqpoint{-0.048611in}{0.000000in}}{\pgfqpoint{-0.000000in}{0.000000in}}{%
\pgfpathmoveto{\pgfqpoint{-0.000000in}{0.000000in}}%
\pgfpathlineto{\pgfqpoint{-0.048611in}{0.000000in}}%
\pgfusepath{stroke,fill}%
}%
\begin{pgfscope}%
\pgfsys@transformshift{1.000000in}{1.805250in}%
\pgfsys@useobject{currentmarker}{}%
\end{pgfscope}%
\end{pgfscope}%
\begin{pgfscope}%
\pgfsetbuttcap%
\pgfsetroundjoin%
\definecolor{currentfill}{rgb}{0.000000,0.000000,0.000000}%
\pgfsetfillcolor{currentfill}%
\pgfsetlinewidth{0.803000pt}%
\definecolor{currentstroke}{rgb}{0.000000,0.000000,0.000000}%
\pgfsetstrokecolor{currentstroke}%
\pgfsetdash{}{0pt}%
\pgfsys@defobject{currentmarker}{\pgfqpoint{0.000000in}{0.000000in}}{\pgfqpoint{0.048611in}{0.000000in}}{%
\pgfpathmoveto{\pgfqpoint{0.000000in}{0.000000in}}%
\pgfpathlineto{\pgfqpoint{0.048611in}{0.000000in}}%
\pgfusepath{stroke,fill}%
}%
\begin{pgfscope}%
\pgfsys@transformshift{7.200000in}{1.805250in}%
\pgfsys@useobject{currentmarker}{}%
\end{pgfscope}%
\end{pgfscope}%
\begin{pgfscope}%
\definecolor{textcolor}{rgb}{0.000000,0.000000,0.000000}%
\pgfsetstrokecolor{textcolor}%
\pgfsetfillcolor{textcolor}%
\pgftext[x=0.763888in, y=1.757024in, left, base]{\color{textcolor}\rmfamily\fontsize{10.000000}{12.000000}\selectfont \(\displaystyle {25}\)}%
\end{pgfscope}%
\begin{pgfscope}%
\pgfpathrectangle{\pgfqpoint{1.000000in}{0.330000in}}{\pgfqpoint{6.200000in}{2.310000in}}%
\pgfusepath{clip}%
\pgfsetbuttcap%
\pgfsetroundjoin%
\pgfsetlinewidth{0.803000pt}%
\definecolor{currentstroke}{rgb}{0.000000,0.000000,0.000000}%
\pgfsetstrokecolor{currentstroke}%
\pgfsetdash{{0.800000pt}{1.320000pt}}{0.000000pt}%
\pgfpathmoveto{\pgfqpoint{1.000000in}{2.137405in}}%
\pgfpathlineto{\pgfqpoint{7.200000in}{2.137405in}}%
\pgfusepath{stroke}%
\end{pgfscope}%
\begin{pgfscope}%
\pgfsetbuttcap%
\pgfsetroundjoin%
\definecolor{currentfill}{rgb}{0.000000,0.000000,0.000000}%
\pgfsetfillcolor{currentfill}%
\pgfsetlinewidth{0.803000pt}%
\definecolor{currentstroke}{rgb}{0.000000,0.000000,0.000000}%
\pgfsetstrokecolor{currentstroke}%
\pgfsetdash{}{0pt}%
\pgfsys@defobject{currentmarker}{\pgfqpoint{-0.048611in}{0.000000in}}{\pgfqpoint{-0.000000in}{0.000000in}}{%
\pgfpathmoveto{\pgfqpoint{-0.000000in}{0.000000in}}%
\pgfpathlineto{\pgfqpoint{-0.048611in}{0.000000in}}%
\pgfusepath{stroke,fill}%
}%
\begin{pgfscope}%
\pgfsys@transformshift{1.000000in}{2.137405in}%
\pgfsys@useobject{currentmarker}{}%
\end{pgfscope}%
\end{pgfscope}%
\begin{pgfscope}%
\pgfsetbuttcap%
\pgfsetroundjoin%
\definecolor{currentfill}{rgb}{0.000000,0.000000,0.000000}%
\pgfsetfillcolor{currentfill}%
\pgfsetlinewidth{0.803000pt}%
\definecolor{currentstroke}{rgb}{0.000000,0.000000,0.000000}%
\pgfsetstrokecolor{currentstroke}%
\pgfsetdash{}{0pt}%
\pgfsys@defobject{currentmarker}{\pgfqpoint{0.000000in}{0.000000in}}{\pgfqpoint{0.048611in}{0.000000in}}{%
\pgfpathmoveto{\pgfqpoint{0.000000in}{0.000000in}}%
\pgfpathlineto{\pgfqpoint{0.048611in}{0.000000in}}%
\pgfusepath{stroke,fill}%
}%
\begin{pgfscope}%
\pgfsys@transformshift{7.200000in}{2.137405in}%
\pgfsys@useobject{currentmarker}{}%
\end{pgfscope}%
\end{pgfscope}%
\begin{pgfscope}%
\definecolor{textcolor}{rgb}{0.000000,0.000000,0.000000}%
\pgfsetstrokecolor{textcolor}%
\pgfsetfillcolor{textcolor}%
\pgftext[x=0.763888in, y=2.089180in, left, base]{\color{textcolor}\rmfamily\fontsize{10.000000}{12.000000}\selectfont \(\displaystyle {50}\)}%
\end{pgfscope}%
\begin{pgfscope}%
\pgfpathrectangle{\pgfqpoint{1.000000in}{0.330000in}}{\pgfqpoint{6.200000in}{2.310000in}}%
\pgfusepath{clip}%
\pgfsetbuttcap%
\pgfsetroundjoin%
\pgfsetlinewidth{0.803000pt}%
\definecolor{currentstroke}{rgb}{0.000000,0.000000,0.000000}%
\pgfsetstrokecolor{currentstroke}%
\pgfsetdash{{0.800000pt}{1.320000pt}}{0.000000pt}%
\pgfpathmoveto{\pgfqpoint{1.000000in}{2.469560in}}%
\pgfpathlineto{\pgfqpoint{7.200000in}{2.469560in}}%
\pgfusepath{stroke}%
\end{pgfscope}%
\begin{pgfscope}%
\pgfsetbuttcap%
\pgfsetroundjoin%
\definecolor{currentfill}{rgb}{0.000000,0.000000,0.000000}%
\pgfsetfillcolor{currentfill}%
\pgfsetlinewidth{0.803000pt}%
\definecolor{currentstroke}{rgb}{0.000000,0.000000,0.000000}%
\pgfsetstrokecolor{currentstroke}%
\pgfsetdash{}{0pt}%
\pgfsys@defobject{currentmarker}{\pgfqpoint{-0.048611in}{0.000000in}}{\pgfqpoint{-0.000000in}{0.000000in}}{%
\pgfpathmoveto{\pgfqpoint{-0.000000in}{0.000000in}}%
\pgfpathlineto{\pgfqpoint{-0.048611in}{0.000000in}}%
\pgfusepath{stroke,fill}%
}%
\begin{pgfscope}%
\pgfsys@transformshift{1.000000in}{2.469560in}%
\pgfsys@useobject{currentmarker}{}%
\end{pgfscope}%
\end{pgfscope}%
\begin{pgfscope}%
\pgfsetbuttcap%
\pgfsetroundjoin%
\definecolor{currentfill}{rgb}{0.000000,0.000000,0.000000}%
\pgfsetfillcolor{currentfill}%
\pgfsetlinewidth{0.803000pt}%
\definecolor{currentstroke}{rgb}{0.000000,0.000000,0.000000}%
\pgfsetstrokecolor{currentstroke}%
\pgfsetdash{}{0pt}%
\pgfsys@defobject{currentmarker}{\pgfqpoint{0.000000in}{0.000000in}}{\pgfqpoint{0.048611in}{0.000000in}}{%
\pgfpathmoveto{\pgfqpoint{0.000000in}{0.000000in}}%
\pgfpathlineto{\pgfqpoint{0.048611in}{0.000000in}}%
\pgfusepath{stroke,fill}%
}%
\begin{pgfscope}%
\pgfsys@transformshift{7.200000in}{2.469560in}%
\pgfsys@useobject{currentmarker}{}%
\end{pgfscope}%
\end{pgfscope}%
\begin{pgfscope}%
\definecolor{textcolor}{rgb}{0.000000,0.000000,0.000000}%
\pgfsetstrokecolor{textcolor}%
\pgfsetfillcolor{textcolor}%
\pgftext[x=0.763888in, y=2.421335in, left, base]{\color{textcolor}\rmfamily\fontsize{10.000000}{12.000000}\selectfont \(\displaystyle {75}\)}%
\end{pgfscope}%
\begin{pgfscope}%
\pgfpathrectangle{\pgfqpoint{1.000000in}{0.330000in}}{\pgfqpoint{6.200000in}{2.310000in}}%
\pgfusepath{clip}%
\pgfsetrectcap%
\pgfsetroundjoin%
\pgfsetlinewidth{1.505625pt}%
\definecolor{currentstroke}{rgb}{0.121569,0.466667,0.705882}%
\pgfsetstrokecolor{currentstroke}%
\pgfsetdash{}{0pt}%
\pgfpathmoveto{\pgfqpoint{1.281818in}{1.473094in}}%
\pgfpathlineto{\pgfqpoint{1.281818in}{1.473094in}}%
\pgfpathlineto{\pgfqpoint{2.003382in}{1.449716in}}%
\pgfpathlineto{\pgfqpoint{2.016996in}{1.561219in}}%
\pgfpathlineto{\pgfqpoint{6.237462in}{1.386723in}}%
\pgfpathlineto{\pgfqpoint{6.264690in}{1.494267in}}%
\pgfpathlineto{\pgfqpoint{6.918182in}{1.473094in}}%
\pgfpathlineto{\pgfqpoint{6.918182in}{1.473094in}}%
\pgfusepath{stroke}%
\end{pgfscope}%
\begin{pgfscope}%
\pgfpathrectangle{\pgfqpoint{1.000000in}{0.330000in}}{\pgfqpoint{6.200000in}{2.310000in}}%
\pgfusepath{clip}%
\pgfsetrectcap%
\pgfsetroundjoin%
\pgfsetlinewidth{1.505625pt}%
\definecolor{currentstroke}{rgb}{1.000000,0.498039,0.054902}%
\pgfsetstrokecolor{currentstroke}%
\pgfsetdash{}{0pt}%
\pgfpathmoveto{\pgfqpoint{1.281818in}{1.473094in}}%
\pgfpathlineto{\pgfqpoint{1.281818in}{1.473094in}}%
\pgfpathlineto{\pgfqpoint{2.003382in}{1.429060in}}%
\pgfpathlineto{\pgfqpoint{2.016996in}{1.542601in}}%
\pgfpathlineto{\pgfqpoint{6.237462in}{1.405859in}}%
\pgfpathlineto{\pgfqpoint{6.264690in}{1.512974in}}%
\pgfpathlineto{\pgfqpoint{6.918182in}{1.473094in}}%
\pgfpathlineto{\pgfqpoint{6.918182in}{1.473094in}}%
\pgfusepath{stroke}%
\end{pgfscope}%
\begin{pgfscope}%
\pgfpathrectangle{\pgfqpoint{1.000000in}{0.330000in}}{\pgfqpoint{6.200000in}{2.310000in}}%
\pgfusepath{clip}%
\pgfsetrectcap%
\pgfsetroundjoin%
\pgfsetlinewidth{1.505625pt}%
\definecolor{currentstroke}{rgb}{0.172549,0.627451,0.172549}%
\pgfsetstrokecolor{currentstroke}%
\pgfsetdash{}{0pt}%
\pgfpathmoveto{\pgfqpoint{1.281818in}{1.473094in}}%
\pgfpathlineto{\pgfqpoint{1.281818in}{1.473094in}}%
\pgfpathlineto{\pgfqpoint{2.003382in}{1.443261in}}%
\pgfpathlineto{\pgfqpoint{2.016996in}{1.561332in}}%
\pgfpathlineto{\pgfqpoint{6.237462in}{1.386835in}}%
\pgfpathlineto{\pgfqpoint{6.264690in}{1.500113in}}%
\pgfpathlineto{\pgfqpoint{6.918182in}{1.473094in}}%
\pgfpathlineto{\pgfqpoint{6.918182in}{1.473094in}}%
\pgfusepath{stroke}%
\end{pgfscope}%
\begin{pgfscope}%
\pgfpathrectangle{\pgfqpoint{1.000000in}{0.330000in}}{\pgfqpoint{6.200000in}{2.310000in}}%
\pgfusepath{clip}%
\pgfsetrectcap%
\pgfsetroundjoin%
\pgfsetlinewidth{1.505625pt}%
\definecolor{currentstroke}{rgb}{0.839216,0.152941,0.156863}%
\pgfsetstrokecolor{currentstroke}%
\pgfsetdash{}{0pt}%
\pgfpathmoveto{\pgfqpoint{1.281818in}{1.473094in}}%
\pgfpathlineto{\pgfqpoint{1.281818in}{1.473094in}}%
\pgfpathlineto{\pgfqpoint{2.003382in}{1.443261in}}%
\pgfpathlineto{\pgfqpoint{2.016996in}{1.561785in}}%
\pgfpathlineto{\pgfqpoint{6.237462in}{1.387288in}}%
\pgfpathlineto{\pgfqpoint{6.264690in}{1.494267in}}%
\pgfpathlineto{\pgfqpoint{6.918182in}{1.473094in}}%
\pgfpathlineto{\pgfqpoint{6.918182in}{1.473094in}}%
\pgfusepath{stroke}%
\end{pgfscope}%
\begin{pgfscope}%
\pgfpathrectangle{\pgfqpoint{1.000000in}{0.330000in}}{\pgfqpoint{6.200000in}{2.310000in}}%
\pgfusepath{clip}%
\pgfsetrectcap%
\pgfsetroundjoin%
\pgfsetlinewidth{1.505625pt}%
\definecolor{currentstroke}{rgb}{0.580392,0.403922,0.741176}%
\pgfsetstrokecolor{currentstroke}%
\pgfsetdash{}{0pt}%
\pgfpathmoveto{\pgfqpoint{1.281818in}{1.473094in}}%
\pgfpathlineto{\pgfqpoint{1.281818in}{1.473094in}}%
\pgfpathlineto{\pgfqpoint{2.003382in}{1.449716in}}%
\pgfpathlineto{\pgfqpoint{2.016996in}{1.540791in}}%
\pgfpathlineto{\pgfqpoint{6.237462in}{1.404050in}}%
\pgfpathlineto{\pgfqpoint{6.264690in}{1.512974in}}%
\pgfpathlineto{\pgfqpoint{6.918182in}{1.473094in}}%
\pgfpathlineto{\pgfqpoint{6.918182in}{1.473094in}}%
\pgfusepath{stroke}%
\end{pgfscope}%
\begin{pgfscope}%
\pgfpathrectangle{\pgfqpoint{1.000000in}{0.330000in}}{\pgfqpoint{6.200000in}{2.310000in}}%
\pgfusepath{clip}%
\pgfsetrectcap%
\pgfsetroundjoin%
\pgfsetlinewidth{1.505625pt}%
\definecolor{currentstroke}{rgb}{0.549020,0.337255,0.294118}%
\pgfsetstrokecolor{currentstroke}%
\pgfsetdash{}{0pt}%
\pgfpathmoveto{\pgfqpoint{1.281818in}{1.473094in}}%
\pgfpathlineto{\pgfqpoint{1.281818in}{1.473094in}}%
\pgfpathlineto{\pgfqpoint{2.003382in}{1.403863in}}%
\pgfpathlineto{\pgfqpoint{2.016996in}{1.677860in}}%
\pgfpathlineto{\pgfqpoint{6.237462in}{1.272921in}}%
\pgfpathlineto{\pgfqpoint{6.264690in}{1.535795in}}%
\pgfpathlineto{\pgfqpoint{6.918182in}{1.473094in}}%
\pgfpathlineto{\pgfqpoint{6.918182in}{1.473094in}}%
\pgfusepath{stroke}%
\end{pgfscope}%
\begin{pgfscope}%
\pgfpathrectangle{\pgfqpoint{1.000000in}{0.330000in}}{\pgfqpoint{6.200000in}{2.310000in}}%
\pgfusepath{clip}%
\pgfsetrectcap%
\pgfsetroundjoin%
\pgfsetlinewidth{1.505625pt}%
\definecolor{currentstroke}{rgb}{0.890196,0.466667,0.760784}%
\pgfsetstrokecolor{currentstroke}%
\pgfsetdash{}{0pt}%
\pgfpathmoveto{\pgfqpoint{1.281818in}{1.473094in}}%
\pgfpathlineto{\pgfqpoint{1.281818in}{1.473094in}}%
\pgfpathlineto{\pgfqpoint{2.003382in}{1.449716in}}%
\pgfpathlineto{\pgfqpoint{2.016996in}{1.560766in}}%
\pgfpathlineto{\pgfqpoint{6.237462in}{1.386270in}}%
\pgfpathlineto{\pgfqpoint{6.264690in}{1.500113in}}%
\pgfpathlineto{\pgfqpoint{6.918182in}{1.473094in}}%
\pgfpathlineto{\pgfqpoint{6.918182in}{1.473094in}}%
\pgfusepath{stroke}%
\end{pgfscope}%
\begin{pgfscope}%
\pgfpathrectangle{\pgfqpoint{1.000000in}{0.330000in}}{\pgfqpoint{6.200000in}{2.310000in}}%
\pgfusepath{clip}%
\pgfsetrectcap%
\pgfsetroundjoin%
\pgfsetlinewidth{1.505625pt}%
\definecolor{currentstroke}{rgb}{0.498039,0.498039,0.498039}%
\pgfsetstrokecolor{currentstroke}%
\pgfsetdash{}{0pt}%
\pgfpathmoveto{\pgfqpoint{1.281818in}{1.473094in}}%
\pgfpathlineto{\pgfqpoint{1.281818in}{1.473094in}}%
\pgfpathlineto{\pgfqpoint{2.003382in}{1.114062in}}%
\pgfpathlineto{\pgfqpoint{2.016996in}{2.535000in}}%
\pgfpathlineto{\pgfqpoint{6.237462in}{0.435000in}}%
\pgfpathlineto{\pgfqpoint{6.264690in}{1.798256in}}%
\pgfpathlineto{\pgfqpoint{6.918182in}{1.473094in}}%
\pgfpathlineto{\pgfqpoint{6.918182in}{1.473094in}}%
\pgfusepath{stroke}%
\end{pgfscope}%
\begin{pgfscope}%
\pgfpathrectangle{\pgfqpoint{1.000000in}{0.330000in}}{\pgfqpoint{6.200000in}{2.310000in}}%
\pgfusepath{clip}%
\pgfsetrectcap%
\pgfsetroundjoin%
\pgfsetlinewidth{1.505625pt}%
\definecolor{currentstroke}{rgb}{0.737255,0.741176,0.133333}%
\pgfsetstrokecolor{currentstroke}%
\pgfsetdash{}{0pt}%
\pgfpathmoveto{\pgfqpoint{1.281818in}{1.473094in}}%
\pgfpathlineto{\pgfqpoint{1.281818in}{1.473094in}}%
\pgfpathlineto{\pgfqpoint{2.003382in}{1.449716in}}%
\pgfpathlineto{\pgfqpoint{2.016996in}{1.541788in}}%
\pgfpathlineto{\pgfqpoint{6.237462in}{1.405046in}}%
\pgfpathlineto{\pgfqpoint{6.264690in}{1.500113in}}%
\pgfpathlineto{\pgfqpoint{6.918182in}{1.473094in}}%
\pgfpathlineto{\pgfqpoint{6.918182in}{1.473094in}}%
\pgfusepath{stroke}%
\end{pgfscope}%
\begin{pgfscope}%
\pgfpathrectangle{\pgfqpoint{1.000000in}{0.330000in}}{\pgfqpoint{6.200000in}{2.310000in}}%
\pgfusepath{clip}%
\pgfsetrectcap%
\pgfsetroundjoin%
\pgfsetlinewidth{1.505625pt}%
\definecolor{currentstroke}{rgb}{0.090196,0.745098,0.811765}%
\pgfsetstrokecolor{currentstroke}%
\pgfsetdash{}{0pt}%
\pgfpathmoveto{\pgfqpoint{1.281818in}{1.473094in}}%
\pgfpathlineto{\pgfqpoint{1.281818in}{1.473094in}}%
\pgfpathlineto{\pgfqpoint{2.003382in}{1.429060in}}%
\pgfpathlineto{\pgfqpoint{2.016996in}{1.603333in}}%
\pgfpathlineto{\pgfqpoint{6.237462in}{1.345776in}}%
\pgfpathlineto{\pgfqpoint{6.264690in}{1.512974in}}%
\pgfpathlineto{\pgfqpoint{6.918182in}{1.473094in}}%
\pgfpathlineto{\pgfqpoint{6.918182in}{1.473094in}}%
\pgfusepath{stroke}%
\end{pgfscope}%
\begin{pgfscope}%
\pgfpathrectangle{\pgfqpoint{1.000000in}{0.330000in}}{\pgfqpoint{6.200000in}{2.310000in}}%
\pgfusepath{clip}%
\pgfsetrectcap%
\pgfsetroundjoin%
\pgfsetlinewidth{1.505625pt}%
\definecolor{currentstroke}{rgb}{0.121569,0.466667,0.705882}%
\pgfsetstrokecolor{currentstroke}%
\pgfsetdash{}{0pt}%
\pgfpathmoveto{\pgfqpoint{1.281818in}{1.473094in}}%
\pgfpathlineto{\pgfqpoint{1.281818in}{1.473094in}}%
\pgfpathlineto{\pgfqpoint{2.003382in}{1.445820in}}%
\pgfpathlineto{\pgfqpoint{2.016996in}{1.553765in}}%
\pgfpathlineto{\pgfqpoint{6.237462in}{1.394233in}}%
\pgfpathlineto{\pgfqpoint{6.264690in}{1.497796in}}%
\pgfpathlineto{\pgfqpoint{6.918182in}{1.473094in}}%
\pgfpathlineto{\pgfqpoint{6.918182in}{1.473094in}}%
\pgfusepath{stroke}%
\end{pgfscope}%
\begin{pgfscope}%
\pgfpathrectangle{\pgfqpoint{1.000000in}{0.330000in}}{\pgfqpoint{6.200000in}{2.310000in}}%
\pgfusepath{clip}%
\pgfsetrectcap%
\pgfsetroundjoin%
\pgfsetlinewidth{1.505625pt}%
\definecolor{currentstroke}{rgb}{1.000000,0.498039,0.054902}%
\pgfsetstrokecolor{currentstroke}%
\pgfsetdash{}{0pt}%
\pgfpathmoveto{\pgfqpoint{1.281818in}{1.473094in}}%
\pgfpathlineto{\pgfqpoint{1.281818in}{1.473094in}}%
\pgfpathlineto{\pgfqpoint{2.003382in}{1.443261in}}%
\pgfpathlineto{\pgfqpoint{2.016996in}{1.542806in}}%
\pgfpathlineto{\pgfqpoint{6.237462in}{1.406065in}}%
\pgfpathlineto{\pgfqpoint{6.264690in}{1.494267in}}%
\pgfpathlineto{\pgfqpoint{6.918182in}{1.473094in}}%
\pgfpathlineto{\pgfqpoint{6.918182in}{1.473094in}}%
\pgfusepath{stroke}%
\end{pgfscope}%
\begin{pgfscope}%
\pgfpathrectangle{\pgfqpoint{1.000000in}{0.330000in}}{\pgfqpoint{6.200000in}{2.310000in}}%
\pgfusepath{clip}%
\pgfsetrectcap%
\pgfsetroundjoin%
\pgfsetlinewidth{1.505625pt}%
\definecolor{currentstroke}{rgb}{0.172549,0.627451,0.172549}%
\pgfsetstrokecolor{currentstroke}%
\pgfsetdash{}{0pt}%
\pgfpathmoveto{\pgfqpoint{1.281818in}{1.473094in}}%
\pgfpathlineto{\pgfqpoint{1.281818in}{1.473094in}}%
\pgfpathlineto{\pgfqpoint{2.003382in}{1.459457in}}%
\pgfpathlineto{\pgfqpoint{2.016996in}{1.513430in}}%
\pgfpathlineto{\pgfqpoint{6.237462in}{1.433664in}}%
\pgfpathlineto{\pgfqpoint{6.264690in}{1.485445in}}%
\pgfpathlineto{\pgfqpoint{6.918182in}{1.473094in}}%
\pgfpathlineto{\pgfqpoint{6.918182in}{1.473094in}}%
\pgfusepath{stroke}%
\end{pgfscope}%
\begin{pgfscope}%
\pgfpathrectangle{\pgfqpoint{1.000000in}{0.330000in}}{\pgfqpoint{6.200000in}{2.310000in}}%
\pgfusepath{clip}%
\pgfsetrectcap%
\pgfsetroundjoin%
\pgfsetlinewidth{1.505625pt}%
\definecolor{currentstroke}{rgb}{0.839216,0.152941,0.156863}%
\pgfsetstrokecolor{currentstroke}%
\pgfsetdash{}{0pt}%
\pgfpathmoveto{\pgfqpoint{1.281818in}{1.473094in}}%
\pgfpathlineto{\pgfqpoint{1.281818in}{1.473094in}}%
\pgfpathlineto{\pgfqpoint{2.003382in}{1.449716in}}%
\pgfpathlineto{\pgfqpoint{2.016996in}{1.602973in}}%
\pgfpathlineto{\pgfqpoint{6.237462in}{1.345415in}}%
\pgfpathlineto{\pgfqpoint{6.264690in}{1.494267in}}%
\pgfpathlineto{\pgfqpoint{6.918182in}{1.473094in}}%
\pgfpathlineto{\pgfqpoint{6.918182in}{1.473094in}}%
\pgfusepath{stroke}%
\end{pgfscope}%
\begin{pgfscope}%
\pgfpathrectangle{\pgfqpoint{1.000000in}{0.330000in}}{\pgfqpoint{6.200000in}{2.310000in}}%
\pgfusepath{clip}%
\pgfsetrectcap%
\pgfsetroundjoin%
\pgfsetlinewidth{1.505625pt}%
\definecolor{currentstroke}{rgb}{0.580392,0.403922,0.741176}%
\pgfsetstrokecolor{currentstroke}%
\pgfsetdash{}{0pt}%
\pgfpathmoveto{\pgfqpoint{1.281818in}{1.473094in}}%
\pgfpathlineto{\pgfqpoint{1.281818in}{1.473094in}}%
\pgfpathlineto{\pgfqpoint{2.003382in}{1.429060in}}%
\pgfpathlineto{\pgfqpoint{2.016996in}{1.544051in}}%
\pgfpathlineto{\pgfqpoint{6.237462in}{1.407309in}}%
\pgfpathlineto{\pgfqpoint{6.264690in}{1.494267in}}%
\pgfpathlineto{\pgfqpoint{6.918182in}{1.473094in}}%
\pgfpathlineto{\pgfqpoint{6.918182in}{1.473094in}}%
\pgfusepath{stroke}%
\end{pgfscope}%
\begin{pgfscope}%
\pgfpathrectangle{\pgfqpoint{1.000000in}{0.330000in}}{\pgfqpoint{6.200000in}{2.310000in}}%
\pgfusepath{clip}%
\pgfsetrectcap%
\pgfsetroundjoin%
\pgfsetlinewidth{1.505625pt}%
\definecolor{currentstroke}{rgb}{0.549020,0.337255,0.294118}%
\pgfsetstrokecolor{currentstroke}%
\pgfsetdash{}{0pt}%
\pgfpathmoveto{\pgfqpoint{1.281818in}{1.473094in}}%
\pgfpathlineto{\pgfqpoint{1.281818in}{1.473094in}}%
\pgfpathlineto{\pgfqpoint{2.003382in}{1.449716in}}%
\pgfpathlineto{\pgfqpoint{2.016996in}{1.601523in}}%
\pgfpathlineto{\pgfqpoint{6.237462in}{1.343966in}}%
\pgfpathlineto{\pgfqpoint{6.264690in}{1.512974in}}%
\pgfpathlineto{\pgfqpoint{6.918182in}{1.473094in}}%
\pgfpathlineto{\pgfqpoint{6.918182in}{1.473094in}}%
\pgfusepath{stroke}%
\end{pgfscope}%
\begin{pgfscope}%
\pgfpathrectangle{\pgfqpoint{1.000000in}{0.330000in}}{\pgfqpoint{6.200000in}{2.310000in}}%
\pgfusepath{clip}%
\pgfsetrectcap%
\pgfsetroundjoin%
\pgfsetlinewidth{1.505625pt}%
\definecolor{currentstroke}{rgb}{0.890196,0.466667,0.760784}%
\pgfsetstrokecolor{currentstroke}%
\pgfsetdash{}{0pt}%
\pgfpathmoveto{\pgfqpoint{1.281818in}{1.473094in}}%
\pgfpathlineto{\pgfqpoint{1.281818in}{1.473094in}}%
\pgfpathlineto{\pgfqpoint{2.003382in}{1.429060in}}%
\pgfpathlineto{\pgfqpoint{2.016996in}{1.604783in}}%
\pgfpathlineto{\pgfqpoint{6.237462in}{1.347225in}}%
\pgfpathlineto{\pgfqpoint{6.264690in}{1.494267in}}%
\pgfpathlineto{\pgfqpoint{6.918182in}{1.473094in}}%
\pgfpathlineto{\pgfqpoint{6.918182in}{1.473094in}}%
\pgfusepath{stroke}%
\end{pgfscope}%
\begin{pgfscope}%
\pgfpathrectangle{\pgfqpoint{1.000000in}{0.330000in}}{\pgfqpoint{6.200000in}{2.310000in}}%
\pgfusepath{clip}%
\pgfsetrectcap%
\pgfsetroundjoin%
\pgfsetlinewidth{1.505625pt}%
\definecolor{currentstroke}{rgb}{0.498039,0.498039,0.498039}%
\pgfsetstrokecolor{currentstroke}%
\pgfsetdash{}{0pt}%
\pgfpathmoveto{\pgfqpoint{1.281818in}{1.473094in}}%
\pgfpathlineto{\pgfqpoint{1.281818in}{1.473094in}}%
\pgfpathlineto{\pgfqpoint{2.003382in}{1.443261in}}%
\pgfpathlineto{\pgfqpoint{2.016996in}{1.542353in}}%
\pgfpathlineto{\pgfqpoint{6.237462in}{1.405612in}}%
\pgfpathlineto{\pgfqpoint{6.264690in}{1.500113in}}%
\pgfpathlineto{\pgfqpoint{6.918182in}{1.473094in}}%
\pgfpathlineto{\pgfqpoint{6.918182in}{1.473094in}}%
\pgfusepath{stroke}%
\end{pgfscope}%
\begin{pgfscope}%
\pgfpathrectangle{\pgfqpoint{1.000000in}{0.330000in}}{\pgfqpoint{6.200000in}{2.310000in}}%
\pgfusepath{clip}%
\pgfsetrectcap%
\pgfsetroundjoin%
\pgfsetlinewidth{1.505625pt}%
\definecolor{currentstroke}{rgb}{0.737255,0.741176,0.133333}%
\pgfsetstrokecolor{currentstroke}%
\pgfsetdash{}{0pt}%
\pgfpathmoveto{\pgfqpoint{1.281818in}{1.473094in}}%
\pgfpathlineto{\pgfqpoint{1.281818in}{1.473094in}}%
\pgfpathlineto{\pgfqpoint{2.003382in}{1.449716in}}%
\pgfpathlineto{\pgfqpoint{2.016996in}{1.542241in}}%
\pgfpathlineto{\pgfqpoint{6.237462in}{1.405499in}}%
\pgfpathlineto{\pgfqpoint{6.264690in}{1.494267in}}%
\pgfpathlineto{\pgfqpoint{6.918182in}{1.473094in}}%
\pgfpathlineto{\pgfqpoint{6.918182in}{1.473094in}}%
\pgfusepath{stroke}%
\end{pgfscope}%
\begin{pgfscope}%
\pgfpathrectangle{\pgfqpoint{1.000000in}{0.330000in}}{\pgfqpoint{6.200000in}{2.310000in}}%
\pgfusepath{clip}%
\pgfsetrectcap%
\pgfsetroundjoin%
\pgfsetlinewidth{1.505625pt}%
\definecolor{currentstroke}{rgb}{0.090196,0.745098,0.811765}%
\pgfsetstrokecolor{currentstroke}%
\pgfsetdash{}{0pt}%
\pgfpathmoveto{\pgfqpoint{1.281818in}{1.473094in}}%
\pgfpathlineto{\pgfqpoint{1.281818in}{1.473094in}}%
\pgfpathlineto{\pgfqpoint{2.003382in}{1.455561in}}%
\pgfpathlineto{\pgfqpoint{2.016996in}{1.524954in}}%
\pgfpathlineto{\pgfqpoint{6.237462in}{1.422398in}}%
\pgfpathlineto{\pgfqpoint{6.264690in}{1.488974in}}%
\pgfpathlineto{\pgfqpoint{6.918182in}{1.473094in}}%
\pgfpathlineto{\pgfqpoint{6.918182in}{1.473094in}}%
\pgfusepath{stroke}%
\end{pgfscope}%
\begin{pgfscope}%
\pgfpathrectangle{\pgfqpoint{1.000000in}{0.330000in}}{\pgfqpoint{6.200000in}{2.310000in}}%
\pgfusepath{clip}%
\pgfsetrectcap%
\pgfsetroundjoin%
\pgfsetlinewidth{1.505625pt}%
\definecolor{currentstroke}{rgb}{0.121569,0.466667,0.705882}%
\pgfsetstrokecolor{currentstroke}%
\pgfsetdash{}{0pt}%
\pgfpathmoveto{\pgfqpoint{1.281818in}{1.473094in}}%
\pgfpathlineto{\pgfqpoint{1.281818in}{1.473094in}}%
\pgfpathlineto{\pgfqpoint{2.003382in}{1.344824in}}%
\pgfpathlineto{\pgfqpoint{2.016996in}{1.852478in}}%
\pgfpathlineto{\pgfqpoint{6.237462in}{1.102218in}}%
\pgfpathlineto{\pgfqpoint{6.264690in}{1.589264in}}%
\pgfpathlineto{\pgfqpoint{6.918182in}{1.473094in}}%
\pgfpathlineto{\pgfqpoint{6.918182in}{1.473094in}}%
\pgfusepath{stroke}%
\end{pgfscope}%
\begin{pgfscope}%
\pgfsetroundcap%
\pgfsetroundjoin%
\pgfsetlinewidth{1.003750pt}%
\definecolor{currentstroke}{rgb}{0.000000,0.000000,0.000000}%
\pgfsetstrokecolor{currentstroke}%
\pgfsetdash{}{0pt}%
\pgfpathmoveto{\pgfqpoint{2.484633in}{2.535000in}}%
\pgfpathquadraticcurveto{\pgfqpoint{2.264701in}{2.535000in}}{\pgfqpoint{2.044768in}{2.535000in}}%
\pgfusepath{stroke}%
\end{pgfscope}%
\begin{pgfscope}%
\pgfsetbuttcap%
\pgfsetmiterjoin%
\definecolor{currentfill}{rgb}{0.800000,0.800000,0.800000}%
\pgfsetfillcolor{currentfill}%
\pgfsetlinewidth{1.003750pt}%
\definecolor{currentstroke}{rgb}{0.000000,0.000000,0.000000}%
\pgfsetstrokecolor{currentstroke}%
\pgfsetdash{}{0pt}%
\pgfpathmoveto{\pgfqpoint{2.542377in}{2.438549in}}%
\pgfpathcurveto{\pgfqpoint{2.577100in}{2.403827in}}{\pgfqpoint{3.405885in}{2.403827in}}{\pgfqpoint{3.440607in}{2.438549in}}%
\pgfpathcurveto{\pgfqpoint{3.475329in}{2.473272in}}{\pgfqpoint{3.475329in}{2.596728in}}{\pgfqpoint{3.440607in}{2.631451in}}%
\pgfpathcurveto{\pgfqpoint{3.405885in}{2.666173in}}{\pgfqpoint{2.577100in}{2.666173in}}{\pgfqpoint{2.542377in}{2.631451in}}%
\pgfpathcurveto{\pgfqpoint{2.507655in}{2.596728in}}{\pgfqpoint{2.507655in}{2.473272in}}{\pgfqpoint{2.542377in}{2.438549in}}%
\pgfpathclose%
\pgfusepath{stroke,fill}%
\end{pgfscope}%
\begin{pgfscope}%
\definecolor{textcolor}{rgb}{0.000000,0.000000,0.000000}%
\pgfsetstrokecolor{textcolor}%
\pgfsetfillcolor{textcolor}%
\pgftext[x=3.405885in,y=2.535000in,right,]{\color{textcolor}\rmfamily\fontsize{10.000000}{12.000000}\selectfont \(\displaystyle V_u =\) 79.9 kip}%
\end{pgfscope}%
\begin{pgfscope}%
\pgfsetbuttcap%
\pgfsetmiterjoin%
\definecolor{currentfill}{rgb}{0.800000,0.800000,0.800000}%
\pgfsetfillcolor{currentfill}%
\pgfsetlinewidth{1.003750pt}%
\definecolor{currentstroke}{rgb}{0.000000,0.000000,0.000000}%
\pgfsetstrokecolor{currentstroke}%
\pgfsetdash{}{0pt}%
\pgfpathmoveto{\pgfqpoint{0.965278in}{0.358599in}}%
\pgfpathcurveto{\pgfqpoint{1.000000in}{0.323877in}}{\pgfqpoint{2.720682in}{0.323877in}}{\pgfqpoint{2.755404in}{0.358599in}}%
\pgfpathcurveto{\pgfqpoint{2.790127in}{0.393321in}}{\pgfqpoint{2.790127in}{0.668784in}}{\pgfqpoint{2.755404in}{0.703506in}}%
\pgfpathcurveto{\pgfqpoint{2.720682in}{0.738228in}}{\pgfqpoint{1.000000in}{0.738228in}}{\pgfqpoint{0.965278in}{0.703506in}}%
\pgfpathcurveto{\pgfqpoint{0.930556in}{0.668784in}}{\pgfqpoint{0.930556in}{0.393321in}}{\pgfqpoint{0.965278in}{0.358599in}}%
\pgfpathclose%
\pgfusepath{stroke,fill}%
\end{pgfscope}%
\begin{pgfscope}%
\definecolor{textcolor}{rgb}{0.000000,0.000000,0.000000}%
\pgfsetstrokecolor{textcolor}%
\pgfsetfillcolor{textcolor}%
\pgftext[x=1.000000in, y=0.580049in, left, base]{\color{textcolor}\rmfamily\fontsize{10.000000}{12.000000}\selectfont Max combo: 1.2D + 1.6S}%
\end{pgfscope}%
\begin{pgfscope}%
\definecolor{textcolor}{rgb}{0.000000,0.000000,0.000000}%
\pgfsetstrokecolor{textcolor}%
\pgfsetfillcolor{textcolor}%
\pgftext[x=1.000000in, y=0.428043in, left, base]{\color{textcolor}\rmfamily\fontsize{10.000000}{12.000000}\selectfont ASCE7-16 Sec. 2.3.1 (LC 3)}%
\end{pgfscope}%
\end{pgfpicture}%
\makeatother%
\endgroup%

\end{center}
\caption{Shear Demand Envelope}
\end{figure}
C\textsubscript{v1}, the web shear strength coefficient, is calculated per AISC/ANSI 360-16 Eq. G2-2 as follows, based on the ratio of the clear distance between flanges to web thickness:
\begin{flalign*}
\frac{h}{t_w} = \frac{7.9 {\color{darkBlue}{\mathbf{ \; in}}}}{0.8 {\color{darkBlue}{\mathbf{ \; in}}}} = \mathbf{10.5 } <= 2.24\sqrt{\frac{E}{F_y}} = 2.24\sqrt{\frac{29000 {\color{darkBlue}{\mathbf{ \; ksi}}}}{50 {\color{darkBlue}{\mathbf{ \; ksi}}}}} = \mathbf{53.9 } \rightarrow C_{v1} = \mathbf{1.0}
\end{flalign*}
\textphi\textsubscript{v}, the resistance factor for shear, is calculated per AISC/ANSI 360-16 {\S}G2.1.a as follows:
\begin{flalign*}
\frac{h}{t_w} = \frac{7.9 {\color{darkBlue}{\mathbf{ \; in}}}}{0.755 {\color{darkBlue}{\mathbf{ \; in}}}} = \mathbf{10.5 } <= 2.24\cdot \sqrt{\frac{E}{F_y}} = 2.24\cdot \sqrt{\frac{29000 {\color{darkBlue}{\mathbf{ \; ksi}}}}{50 {\color{darkBlue}{\mathbf{ \; ksi}}}}} = \mathbf{53.9} \rightarrow \phi_v = \mathbf{1.0}
\end{flalign*}
\textphi\textsubscript{v}V\textsubscript{n}, the design shear strength, is calculated per AISC/ANSI 360-16 Eq. G2-1 as follows:
\begin{flalign*}
\phi_v V_n = 0.6\cdot F_y \cdot A_w \cdot C_{v1}  = 0.6\cdot 50 {\color{darkBlue}{\mathbf{ \; ksi}}} \cdot 8.61 {\color{darkBlue}{\mathbf{ \; {\color{darkBlue}{\mathbf{ \; in}}}^{2}}}} \cdot 1.0  = \mathbf{258.2 {\color{darkBlue}{\mathbf{ \; kip}}}}
\end{flalign*}
\vspace{-26pt}
{\setlength{\mathindent}{0cm}
\begin{flalign*}
\mathbf{|V_u| = 79.9 {\color{darkBlue}{\mathbf{ \; kip}}}  \;  < \phi_v \cdot V_n = 258.2 {\color{darkBlue}{\mathbf{ \; kip}}}  \;  (DCR = 0.31 - OK)}
\end{flalign*}
%	----------------------------- DEFLECTION CHECK -------------------------------
\section{Deflection Check}
\begin{figure}[H]
\begin{center}
%% Creator: Matplotlib, PGF backend
%%
%% To include the figure in your LaTeX document, write
%%   \input{<filename>.pgf}
%%
%% Make sure the required packages are loaded in your preamble
%%   \usepackage{pgf}
%%
%% Figures using additional raster images can only be included by \input if
%% they are in the same directory as the main LaTeX file. For loading figures
%% from other directories you can use the `import` package
%%   \usepackage{import}
%%
%% and then include the figures with
%%   \import{<path to file>}{<filename>.pgf}
%%
%% Matplotlib used the following preamble
%%
\begingroup%
\makeatletter%
\begin{pgfpicture}%
\pgfpathrectangle{\pgfpointorigin}{\pgfqpoint{8.000000in}{3.000000in}}%
\pgfusepath{use as bounding box, clip}%
\begin{pgfscope}%
\pgfsetbuttcap%
\pgfsetmiterjoin%
\definecolor{currentfill}{rgb}{1.000000,1.000000,1.000000}%
\pgfsetfillcolor{currentfill}%
\pgfsetlinewidth{0.000000pt}%
\definecolor{currentstroke}{rgb}{1.000000,1.000000,1.000000}%
\pgfsetstrokecolor{currentstroke}%
\pgfsetdash{}{0pt}%
\pgfpathmoveto{\pgfqpoint{0.000000in}{0.000000in}}%
\pgfpathlineto{\pgfqpoint{8.000000in}{0.000000in}}%
\pgfpathlineto{\pgfqpoint{8.000000in}{3.000000in}}%
\pgfpathlineto{\pgfqpoint{0.000000in}{3.000000in}}%
\pgfpathclose%
\pgfusepath{fill}%
\end{pgfscope}%
\begin{pgfscope}%
\pgfsetbuttcap%
\pgfsetmiterjoin%
\definecolor{currentfill}{rgb}{1.000000,1.000000,1.000000}%
\pgfsetfillcolor{currentfill}%
\pgfsetlinewidth{0.000000pt}%
\definecolor{currentstroke}{rgb}{0.000000,0.000000,0.000000}%
\pgfsetstrokecolor{currentstroke}%
\pgfsetstrokeopacity{0.000000}%
\pgfsetdash{}{0pt}%
\pgfpathmoveto{\pgfqpoint{1.000000in}{0.330000in}}%
\pgfpathlineto{\pgfqpoint{7.200000in}{0.330000in}}%
\pgfpathlineto{\pgfqpoint{7.200000in}{2.640000in}}%
\pgfpathlineto{\pgfqpoint{1.000000in}{2.640000in}}%
\pgfpathclose%
\pgfusepath{fill}%
\end{pgfscope}%
\begin{pgfscope}%
\pgfpathrectangle{\pgfqpoint{1.000000in}{0.330000in}}{\pgfqpoint{6.200000in}{2.310000in}}%
\pgfusepath{clip}%
\pgfsetbuttcap%
\pgfsetroundjoin%
\pgfsetlinewidth{0.803000pt}%
\definecolor{currentstroke}{rgb}{0.000000,0.000000,0.000000}%
\pgfsetstrokecolor{currentstroke}%
\pgfsetdash{{0.800000pt}{1.320000pt}}{0.000000pt}%
\pgfpathmoveto{\pgfqpoint{1.281818in}{0.330000in}}%
\pgfpathlineto{\pgfqpoint{1.281818in}{2.640000in}}%
\pgfusepath{stroke}%
\end{pgfscope}%
\begin{pgfscope}%
\pgfsetbuttcap%
\pgfsetroundjoin%
\definecolor{currentfill}{rgb}{0.000000,0.000000,0.000000}%
\pgfsetfillcolor{currentfill}%
\pgfsetlinewidth{0.803000pt}%
\definecolor{currentstroke}{rgb}{0.000000,0.000000,0.000000}%
\pgfsetstrokecolor{currentstroke}%
\pgfsetdash{}{0pt}%
\pgfsys@defobject{currentmarker}{\pgfqpoint{0.000000in}{-0.048611in}}{\pgfqpoint{0.000000in}{0.000000in}}{%
\pgfpathmoveto{\pgfqpoint{0.000000in}{0.000000in}}%
\pgfpathlineto{\pgfqpoint{0.000000in}{-0.048611in}}%
\pgfusepath{stroke,fill}%
}%
\begin{pgfscope}%
\pgfsys@transformshift{1.281818in}{0.330000in}%
\pgfsys@useobject{currentmarker}{}%
\end{pgfscope}%
\end{pgfscope}%
\begin{pgfscope}%
\pgfsetbuttcap%
\pgfsetroundjoin%
\definecolor{currentfill}{rgb}{0.000000,0.000000,0.000000}%
\pgfsetfillcolor{currentfill}%
\pgfsetlinewidth{0.803000pt}%
\definecolor{currentstroke}{rgb}{0.000000,0.000000,0.000000}%
\pgfsetstrokecolor{currentstroke}%
\pgfsetdash{}{0pt}%
\pgfsys@defobject{currentmarker}{\pgfqpoint{0.000000in}{0.000000in}}{\pgfqpoint{0.000000in}{0.048611in}}{%
\pgfpathmoveto{\pgfqpoint{0.000000in}{0.000000in}}%
\pgfpathlineto{\pgfqpoint{0.000000in}{0.048611in}}%
\pgfusepath{stroke,fill}%
}%
\begin{pgfscope}%
\pgfsys@transformshift{1.281818in}{2.640000in}%
\pgfsys@useobject{currentmarker}{}%
\end{pgfscope}%
\end{pgfscope}%
\begin{pgfscope}%
\definecolor{textcolor}{rgb}{0.000000,0.000000,0.000000}%
\pgfsetstrokecolor{textcolor}%
\pgfsetfillcolor{textcolor}%
\pgftext[x=1.281818in,y=0.232778in,,top]{\color{textcolor}\rmfamily\fontsize{10.000000}{12.000000}\selectfont \(\displaystyle {0}\)}%
\end{pgfscope}%
\begin{pgfscope}%
\pgfpathrectangle{\pgfqpoint{1.000000in}{0.330000in}}{\pgfqpoint{6.200000in}{2.310000in}}%
\pgfusepath{clip}%
\pgfsetbuttcap%
\pgfsetroundjoin%
\pgfsetlinewidth{0.803000pt}%
\definecolor{currentstroke}{rgb}{0.000000,0.000000,0.000000}%
\pgfsetstrokecolor{currentstroke}%
\pgfsetdash{{0.800000pt}{1.320000pt}}{0.000000pt}%
\pgfpathmoveto{\pgfqpoint{2.098682in}{0.330000in}}%
\pgfpathlineto{\pgfqpoint{2.098682in}{2.640000in}}%
\pgfusepath{stroke}%
\end{pgfscope}%
\begin{pgfscope}%
\pgfsetbuttcap%
\pgfsetroundjoin%
\definecolor{currentfill}{rgb}{0.000000,0.000000,0.000000}%
\pgfsetfillcolor{currentfill}%
\pgfsetlinewidth{0.803000pt}%
\definecolor{currentstroke}{rgb}{0.000000,0.000000,0.000000}%
\pgfsetstrokecolor{currentstroke}%
\pgfsetdash{}{0pt}%
\pgfsys@defobject{currentmarker}{\pgfqpoint{0.000000in}{-0.048611in}}{\pgfqpoint{0.000000in}{0.000000in}}{%
\pgfpathmoveto{\pgfqpoint{0.000000in}{0.000000in}}%
\pgfpathlineto{\pgfqpoint{0.000000in}{-0.048611in}}%
\pgfusepath{stroke,fill}%
}%
\begin{pgfscope}%
\pgfsys@transformshift{2.098682in}{0.330000in}%
\pgfsys@useobject{currentmarker}{}%
\end{pgfscope}%
\end{pgfscope}%
\begin{pgfscope}%
\pgfsetbuttcap%
\pgfsetroundjoin%
\definecolor{currentfill}{rgb}{0.000000,0.000000,0.000000}%
\pgfsetfillcolor{currentfill}%
\pgfsetlinewidth{0.803000pt}%
\definecolor{currentstroke}{rgb}{0.000000,0.000000,0.000000}%
\pgfsetstrokecolor{currentstroke}%
\pgfsetdash{}{0pt}%
\pgfsys@defobject{currentmarker}{\pgfqpoint{0.000000in}{0.000000in}}{\pgfqpoint{0.000000in}{0.048611in}}{%
\pgfpathmoveto{\pgfqpoint{0.000000in}{0.000000in}}%
\pgfpathlineto{\pgfqpoint{0.000000in}{0.048611in}}%
\pgfusepath{stroke,fill}%
}%
\begin{pgfscope}%
\pgfsys@transformshift{2.098682in}{2.640000in}%
\pgfsys@useobject{currentmarker}{}%
\end{pgfscope}%
\end{pgfscope}%
\begin{pgfscope}%
\definecolor{textcolor}{rgb}{0.000000,0.000000,0.000000}%
\pgfsetstrokecolor{textcolor}%
\pgfsetfillcolor{textcolor}%
\pgftext[x=2.098682in,y=0.232778in,,top]{\color{textcolor}\rmfamily\fontsize{10.000000}{12.000000}\selectfont \(\displaystyle {5}\)}%
\end{pgfscope}%
\begin{pgfscope}%
\pgfpathrectangle{\pgfqpoint{1.000000in}{0.330000in}}{\pgfqpoint{6.200000in}{2.310000in}}%
\pgfusepath{clip}%
\pgfsetbuttcap%
\pgfsetroundjoin%
\pgfsetlinewidth{0.803000pt}%
\definecolor{currentstroke}{rgb}{0.000000,0.000000,0.000000}%
\pgfsetstrokecolor{currentstroke}%
\pgfsetdash{{0.800000pt}{1.320000pt}}{0.000000pt}%
\pgfpathmoveto{\pgfqpoint{2.915547in}{0.330000in}}%
\pgfpathlineto{\pgfqpoint{2.915547in}{2.640000in}}%
\pgfusepath{stroke}%
\end{pgfscope}%
\begin{pgfscope}%
\pgfsetbuttcap%
\pgfsetroundjoin%
\definecolor{currentfill}{rgb}{0.000000,0.000000,0.000000}%
\pgfsetfillcolor{currentfill}%
\pgfsetlinewidth{0.803000pt}%
\definecolor{currentstroke}{rgb}{0.000000,0.000000,0.000000}%
\pgfsetstrokecolor{currentstroke}%
\pgfsetdash{}{0pt}%
\pgfsys@defobject{currentmarker}{\pgfqpoint{0.000000in}{-0.048611in}}{\pgfqpoint{0.000000in}{0.000000in}}{%
\pgfpathmoveto{\pgfqpoint{0.000000in}{0.000000in}}%
\pgfpathlineto{\pgfqpoint{0.000000in}{-0.048611in}}%
\pgfusepath{stroke,fill}%
}%
\begin{pgfscope}%
\pgfsys@transformshift{2.915547in}{0.330000in}%
\pgfsys@useobject{currentmarker}{}%
\end{pgfscope}%
\end{pgfscope}%
\begin{pgfscope}%
\pgfsetbuttcap%
\pgfsetroundjoin%
\definecolor{currentfill}{rgb}{0.000000,0.000000,0.000000}%
\pgfsetfillcolor{currentfill}%
\pgfsetlinewidth{0.803000pt}%
\definecolor{currentstroke}{rgb}{0.000000,0.000000,0.000000}%
\pgfsetstrokecolor{currentstroke}%
\pgfsetdash{}{0pt}%
\pgfsys@defobject{currentmarker}{\pgfqpoint{0.000000in}{0.000000in}}{\pgfqpoint{0.000000in}{0.048611in}}{%
\pgfpathmoveto{\pgfqpoint{0.000000in}{0.000000in}}%
\pgfpathlineto{\pgfqpoint{0.000000in}{0.048611in}}%
\pgfusepath{stroke,fill}%
}%
\begin{pgfscope}%
\pgfsys@transformshift{2.915547in}{2.640000in}%
\pgfsys@useobject{currentmarker}{}%
\end{pgfscope}%
\end{pgfscope}%
\begin{pgfscope}%
\definecolor{textcolor}{rgb}{0.000000,0.000000,0.000000}%
\pgfsetstrokecolor{textcolor}%
\pgfsetfillcolor{textcolor}%
\pgftext[x=2.915547in,y=0.232778in,,top]{\color{textcolor}\rmfamily\fontsize{10.000000}{12.000000}\selectfont \(\displaystyle {10}\)}%
\end{pgfscope}%
\begin{pgfscope}%
\pgfpathrectangle{\pgfqpoint{1.000000in}{0.330000in}}{\pgfqpoint{6.200000in}{2.310000in}}%
\pgfusepath{clip}%
\pgfsetbuttcap%
\pgfsetroundjoin%
\pgfsetlinewidth{0.803000pt}%
\definecolor{currentstroke}{rgb}{0.000000,0.000000,0.000000}%
\pgfsetstrokecolor{currentstroke}%
\pgfsetdash{{0.800000pt}{1.320000pt}}{0.000000pt}%
\pgfpathmoveto{\pgfqpoint{3.732411in}{0.330000in}}%
\pgfpathlineto{\pgfqpoint{3.732411in}{2.640000in}}%
\pgfusepath{stroke}%
\end{pgfscope}%
\begin{pgfscope}%
\pgfsetbuttcap%
\pgfsetroundjoin%
\definecolor{currentfill}{rgb}{0.000000,0.000000,0.000000}%
\pgfsetfillcolor{currentfill}%
\pgfsetlinewidth{0.803000pt}%
\definecolor{currentstroke}{rgb}{0.000000,0.000000,0.000000}%
\pgfsetstrokecolor{currentstroke}%
\pgfsetdash{}{0pt}%
\pgfsys@defobject{currentmarker}{\pgfqpoint{0.000000in}{-0.048611in}}{\pgfqpoint{0.000000in}{0.000000in}}{%
\pgfpathmoveto{\pgfqpoint{0.000000in}{0.000000in}}%
\pgfpathlineto{\pgfqpoint{0.000000in}{-0.048611in}}%
\pgfusepath{stroke,fill}%
}%
\begin{pgfscope}%
\pgfsys@transformshift{3.732411in}{0.330000in}%
\pgfsys@useobject{currentmarker}{}%
\end{pgfscope}%
\end{pgfscope}%
\begin{pgfscope}%
\pgfsetbuttcap%
\pgfsetroundjoin%
\definecolor{currentfill}{rgb}{0.000000,0.000000,0.000000}%
\pgfsetfillcolor{currentfill}%
\pgfsetlinewidth{0.803000pt}%
\definecolor{currentstroke}{rgb}{0.000000,0.000000,0.000000}%
\pgfsetstrokecolor{currentstroke}%
\pgfsetdash{}{0pt}%
\pgfsys@defobject{currentmarker}{\pgfqpoint{0.000000in}{0.000000in}}{\pgfqpoint{0.000000in}{0.048611in}}{%
\pgfpathmoveto{\pgfqpoint{0.000000in}{0.000000in}}%
\pgfpathlineto{\pgfqpoint{0.000000in}{0.048611in}}%
\pgfusepath{stroke,fill}%
}%
\begin{pgfscope}%
\pgfsys@transformshift{3.732411in}{2.640000in}%
\pgfsys@useobject{currentmarker}{}%
\end{pgfscope}%
\end{pgfscope}%
\begin{pgfscope}%
\definecolor{textcolor}{rgb}{0.000000,0.000000,0.000000}%
\pgfsetstrokecolor{textcolor}%
\pgfsetfillcolor{textcolor}%
\pgftext[x=3.732411in,y=0.232778in,,top]{\color{textcolor}\rmfamily\fontsize{10.000000}{12.000000}\selectfont \(\displaystyle {15}\)}%
\end{pgfscope}%
\begin{pgfscope}%
\pgfpathrectangle{\pgfqpoint{1.000000in}{0.330000in}}{\pgfqpoint{6.200000in}{2.310000in}}%
\pgfusepath{clip}%
\pgfsetbuttcap%
\pgfsetroundjoin%
\pgfsetlinewidth{0.803000pt}%
\definecolor{currentstroke}{rgb}{0.000000,0.000000,0.000000}%
\pgfsetstrokecolor{currentstroke}%
\pgfsetdash{{0.800000pt}{1.320000pt}}{0.000000pt}%
\pgfpathmoveto{\pgfqpoint{4.549275in}{0.330000in}}%
\pgfpathlineto{\pgfqpoint{4.549275in}{2.640000in}}%
\pgfusepath{stroke}%
\end{pgfscope}%
\begin{pgfscope}%
\pgfsetbuttcap%
\pgfsetroundjoin%
\definecolor{currentfill}{rgb}{0.000000,0.000000,0.000000}%
\pgfsetfillcolor{currentfill}%
\pgfsetlinewidth{0.803000pt}%
\definecolor{currentstroke}{rgb}{0.000000,0.000000,0.000000}%
\pgfsetstrokecolor{currentstroke}%
\pgfsetdash{}{0pt}%
\pgfsys@defobject{currentmarker}{\pgfqpoint{0.000000in}{-0.048611in}}{\pgfqpoint{0.000000in}{0.000000in}}{%
\pgfpathmoveto{\pgfqpoint{0.000000in}{0.000000in}}%
\pgfpathlineto{\pgfqpoint{0.000000in}{-0.048611in}}%
\pgfusepath{stroke,fill}%
}%
\begin{pgfscope}%
\pgfsys@transformshift{4.549275in}{0.330000in}%
\pgfsys@useobject{currentmarker}{}%
\end{pgfscope}%
\end{pgfscope}%
\begin{pgfscope}%
\pgfsetbuttcap%
\pgfsetroundjoin%
\definecolor{currentfill}{rgb}{0.000000,0.000000,0.000000}%
\pgfsetfillcolor{currentfill}%
\pgfsetlinewidth{0.803000pt}%
\definecolor{currentstroke}{rgb}{0.000000,0.000000,0.000000}%
\pgfsetstrokecolor{currentstroke}%
\pgfsetdash{}{0pt}%
\pgfsys@defobject{currentmarker}{\pgfqpoint{0.000000in}{0.000000in}}{\pgfqpoint{0.000000in}{0.048611in}}{%
\pgfpathmoveto{\pgfqpoint{0.000000in}{0.000000in}}%
\pgfpathlineto{\pgfqpoint{0.000000in}{0.048611in}}%
\pgfusepath{stroke,fill}%
}%
\begin{pgfscope}%
\pgfsys@transformshift{4.549275in}{2.640000in}%
\pgfsys@useobject{currentmarker}{}%
\end{pgfscope}%
\end{pgfscope}%
\begin{pgfscope}%
\definecolor{textcolor}{rgb}{0.000000,0.000000,0.000000}%
\pgfsetstrokecolor{textcolor}%
\pgfsetfillcolor{textcolor}%
\pgftext[x=4.549275in,y=0.232778in,,top]{\color{textcolor}\rmfamily\fontsize{10.000000}{12.000000}\selectfont \(\displaystyle {20}\)}%
\end{pgfscope}%
\begin{pgfscope}%
\pgfpathrectangle{\pgfqpoint{1.000000in}{0.330000in}}{\pgfqpoint{6.200000in}{2.310000in}}%
\pgfusepath{clip}%
\pgfsetbuttcap%
\pgfsetroundjoin%
\pgfsetlinewidth{0.803000pt}%
\definecolor{currentstroke}{rgb}{0.000000,0.000000,0.000000}%
\pgfsetstrokecolor{currentstroke}%
\pgfsetdash{{0.800000pt}{1.320000pt}}{0.000000pt}%
\pgfpathmoveto{\pgfqpoint{5.366140in}{0.330000in}}%
\pgfpathlineto{\pgfqpoint{5.366140in}{2.640000in}}%
\pgfusepath{stroke}%
\end{pgfscope}%
\begin{pgfscope}%
\pgfsetbuttcap%
\pgfsetroundjoin%
\definecolor{currentfill}{rgb}{0.000000,0.000000,0.000000}%
\pgfsetfillcolor{currentfill}%
\pgfsetlinewidth{0.803000pt}%
\definecolor{currentstroke}{rgb}{0.000000,0.000000,0.000000}%
\pgfsetstrokecolor{currentstroke}%
\pgfsetdash{}{0pt}%
\pgfsys@defobject{currentmarker}{\pgfqpoint{0.000000in}{-0.048611in}}{\pgfqpoint{0.000000in}{0.000000in}}{%
\pgfpathmoveto{\pgfqpoint{0.000000in}{0.000000in}}%
\pgfpathlineto{\pgfqpoint{0.000000in}{-0.048611in}}%
\pgfusepath{stroke,fill}%
}%
\begin{pgfscope}%
\pgfsys@transformshift{5.366140in}{0.330000in}%
\pgfsys@useobject{currentmarker}{}%
\end{pgfscope}%
\end{pgfscope}%
\begin{pgfscope}%
\pgfsetbuttcap%
\pgfsetroundjoin%
\definecolor{currentfill}{rgb}{0.000000,0.000000,0.000000}%
\pgfsetfillcolor{currentfill}%
\pgfsetlinewidth{0.803000pt}%
\definecolor{currentstroke}{rgb}{0.000000,0.000000,0.000000}%
\pgfsetstrokecolor{currentstroke}%
\pgfsetdash{}{0pt}%
\pgfsys@defobject{currentmarker}{\pgfqpoint{0.000000in}{0.000000in}}{\pgfqpoint{0.000000in}{0.048611in}}{%
\pgfpathmoveto{\pgfqpoint{0.000000in}{0.000000in}}%
\pgfpathlineto{\pgfqpoint{0.000000in}{0.048611in}}%
\pgfusepath{stroke,fill}%
}%
\begin{pgfscope}%
\pgfsys@transformshift{5.366140in}{2.640000in}%
\pgfsys@useobject{currentmarker}{}%
\end{pgfscope}%
\end{pgfscope}%
\begin{pgfscope}%
\definecolor{textcolor}{rgb}{0.000000,0.000000,0.000000}%
\pgfsetstrokecolor{textcolor}%
\pgfsetfillcolor{textcolor}%
\pgftext[x=5.366140in,y=0.232778in,,top]{\color{textcolor}\rmfamily\fontsize{10.000000}{12.000000}\selectfont \(\displaystyle {25}\)}%
\end{pgfscope}%
\begin{pgfscope}%
\pgfpathrectangle{\pgfqpoint{1.000000in}{0.330000in}}{\pgfqpoint{6.200000in}{2.310000in}}%
\pgfusepath{clip}%
\pgfsetbuttcap%
\pgfsetroundjoin%
\pgfsetlinewidth{0.803000pt}%
\definecolor{currentstroke}{rgb}{0.000000,0.000000,0.000000}%
\pgfsetstrokecolor{currentstroke}%
\pgfsetdash{{0.800000pt}{1.320000pt}}{0.000000pt}%
\pgfpathmoveto{\pgfqpoint{6.183004in}{0.330000in}}%
\pgfpathlineto{\pgfqpoint{6.183004in}{2.640000in}}%
\pgfusepath{stroke}%
\end{pgfscope}%
\begin{pgfscope}%
\pgfsetbuttcap%
\pgfsetroundjoin%
\definecolor{currentfill}{rgb}{0.000000,0.000000,0.000000}%
\pgfsetfillcolor{currentfill}%
\pgfsetlinewidth{0.803000pt}%
\definecolor{currentstroke}{rgb}{0.000000,0.000000,0.000000}%
\pgfsetstrokecolor{currentstroke}%
\pgfsetdash{}{0pt}%
\pgfsys@defobject{currentmarker}{\pgfqpoint{0.000000in}{-0.048611in}}{\pgfqpoint{0.000000in}{0.000000in}}{%
\pgfpathmoveto{\pgfqpoint{0.000000in}{0.000000in}}%
\pgfpathlineto{\pgfqpoint{0.000000in}{-0.048611in}}%
\pgfusepath{stroke,fill}%
}%
\begin{pgfscope}%
\pgfsys@transformshift{6.183004in}{0.330000in}%
\pgfsys@useobject{currentmarker}{}%
\end{pgfscope}%
\end{pgfscope}%
\begin{pgfscope}%
\pgfsetbuttcap%
\pgfsetroundjoin%
\definecolor{currentfill}{rgb}{0.000000,0.000000,0.000000}%
\pgfsetfillcolor{currentfill}%
\pgfsetlinewidth{0.803000pt}%
\definecolor{currentstroke}{rgb}{0.000000,0.000000,0.000000}%
\pgfsetstrokecolor{currentstroke}%
\pgfsetdash{}{0pt}%
\pgfsys@defobject{currentmarker}{\pgfqpoint{0.000000in}{0.000000in}}{\pgfqpoint{0.000000in}{0.048611in}}{%
\pgfpathmoveto{\pgfqpoint{0.000000in}{0.000000in}}%
\pgfpathlineto{\pgfqpoint{0.000000in}{0.048611in}}%
\pgfusepath{stroke,fill}%
}%
\begin{pgfscope}%
\pgfsys@transformshift{6.183004in}{2.640000in}%
\pgfsys@useobject{currentmarker}{}%
\end{pgfscope}%
\end{pgfscope}%
\begin{pgfscope}%
\definecolor{textcolor}{rgb}{0.000000,0.000000,0.000000}%
\pgfsetstrokecolor{textcolor}%
\pgfsetfillcolor{textcolor}%
\pgftext[x=6.183004in,y=0.232778in,,top]{\color{textcolor}\rmfamily\fontsize{10.000000}{12.000000}\selectfont \(\displaystyle {30}\)}%
\end{pgfscope}%
\begin{pgfscope}%
\pgfpathrectangle{\pgfqpoint{1.000000in}{0.330000in}}{\pgfqpoint{6.200000in}{2.310000in}}%
\pgfusepath{clip}%
\pgfsetbuttcap%
\pgfsetroundjoin%
\pgfsetlinewidth{0.803000pt}%
\definecolor{currentstroke}{rgb}{0.000000,0.000000,0.000000}%
\pgfsetstrokecolor{currentstroke}%
\pgfsetdash{{0.800000pt}{1.320000pt}}{0.000000pt}%
\pgfpathmoveto{\pgfqpoint{6.999868in}{0.330000in}}%
\pgfpathlineto{\pgfqpoint{6.999868in}{2.640000in}}%
\pgfusepath{stroke}%
\end{pgfscope}%
\begin{pgfscope}%
\pgfsetbuttcap%
\pgfsetroundjoin%
\definecolor{currentfill}{rgb}{0.000000,0.000000,0.000000}%
\pgfsetfillcolor{currentfill}%
\pgfsetlinewidth{0.803000pt}%
\definecolor{currentstroke}{rgb}{0.000000,0.000000,0.000000}%
\pgfsetstrokecolor{currentstroke}%
\pgfsetdash{}{0pt}%
\pgfsys@defobject{currentmarker}{\pgfqpoint{0.000000in}{-0.048611in}}{\pgfqpoint{0.000000in}{0.000000in}}{%
\pgfpathmoveto{\pgfqpoint{0.000000in}{0.000000in}}%
\pgfpathlineto{\pgfqpoint{0.000000in}{-0.048611in}}%
\pgfusepath{stroke,fill}%
}%
\begin{pgfscope}%
\pgfsys@transformshift{6.999868in}{0.330000in}%
\pgfsys@useobject{currentmarker}{}%
\end{pgfscope}%
\end{pgfscope}%
\begin{pgfscope}%
\pgfsetbuttcap%
\pgfsetroundjoin%
\definecolor{currentfill}{rgb}{0.000000,0.000000,0.000000}%
\pgfsetfillcolor{currentfill}%
\pgfsetlinewidth{0.803000pt}%
\definecolor{currentstroke}{rgb}{0.000000,0.000000,0.000000}%
\pgfsetstrokecolor{currentstroke}%
\pgfsetdash{}{0pt}%
\pgfsys@defobject{currentmarker}{\pgfqpoint{0.000000in}{0.000000in}}{\pgfqpoint{0.000000in}{0.048611in}}{%
\pgfpathmoveto{\pgfqpoint{0.000000in}{0.000000in}}%
\pgfpathlineto{\pgfqpoint{0.000000in}{0.048611in}}%
\pgfusepath{stroke,fill}%
}%
\begin{pgfscope}%
\pgfsys@transformshift{6.999868in}{2.640000in}%
\pgfsys@useobject{currentmarker}{}%
\end{pgfscope}%
\end{pgfscope}%
\begin{pgfscope}%
\definecolor{textcolor}{rgb}{0.000000,0.000000,0.000000}%
\pgfsetstrokecolor{textcolor}%
\pgfsetfillcolor{textcolor}%
\pgftext[x=6.999868in,y=0.232778in,,top]{\color{textcolor}\rmfamily\fontsize{10.000000}{12.000000}\selectfont \(\displaystyle {35}\)}%
\end{pgfscope}%
\begin{pgfscope}%
\pgfpathrectangle{\pgfqpoint{1.000000in}{0.330000in}}{\pgfqpoint{6.200000in}{2.310000in}}%
\pgfusepath{clip}%
\pgfsetbuttcap%
\pgfsetroundjoin%
\pgfsetlinewidth{0.803000pt}%
\definecolor{currentstroke}{rgb}{0.000000,0.000000,0.000000}%
\pgfsetstrokecolor{currentstroke}%
\pgfsetdash{{0.800000pt}{1.320000pt}}{0.000000pt}%
\pgfpathmoveto{\pgfqpoint{1.000000in}{0.618325in}}%
\pgfpathlineto{\pgfqpoint{7.200000in}{0.618325in}}%
\pgfusepath{stroke}%
\end{pgfscope}%
\begin{pgfscope}%
\pgfsetbuttcap%
\pgfsetroundjoin%
\definecolor{currentfill}{rgb}{0.000000,0.000000,0.000000}%
\pgfsetfillcolor{currentfill}%
\pgfsetlinewidth{0.803000pt}%
\definecolor{currentstroke}{rgb}{0.000000,0.000000,0.000000}%
\pgfsetstrokecolor{currentstroke}%
\pgfsetdash{}{0pt}%
\pgfsys@defobject{currentmarker}{\pgfqpoint{-0.048611in}{0.000000in}}{\pgfqpoint{-0.000000in}{0.000000in}}{%
\pgfpathmoveto{\pgfqpoint{-0.000000in}{0.000000in}}%
\pgfpathlineto{\pgfqpoint{-0.048611in}{0.000000in}}%
\pgfusepath{stroke,fill}%
}%
\begin{pgfscope}%
\pgfsys@transformshift{1.000000in}{0.618325in}%
\pgfsys@useobject{currentmarker}{}%
\end{pgfscope}%
\end{pgfscope}%
\begin{pgfscope}%
\pgfsetbuttcap%
\pgfsetroundjoin%
\definecolor{currentfill}{rgb}{0.000000,0.000000,0.000000}%
\pgfsetfillcolor{currentfill}%
\pgfsetlinewidth{0.803000pt}%
\definecolor{currentstroke}{rgb}{0.000000,0.000000,0.000000}%
\pgfsetstrokecolor{currentstroke}%
\pgfsetdash{}{0pt}%
\pgfsys@defobject{currentmarker}{\pgfqpoint{0.000000in}{0.000000in}}{\pgfqpoint{0.048611in}{0.000000in}}{%
\pgfpathmoveto{\pgfqpoint{0.000000in}{0.000000in}}%
\pgfpathlineto{\pgfqpoint{0.048611in}{0.000000in}}%
\pgfusepath{stroke,fill}%
}%
\begin{pgfscope}%
\pgfsys@transformshift{7.200000in}{0.618325in}%
\pgfsys@useobject{currentmarker}{}%
\end{pgfscope}%
\end{pgfscope}%
\begin{pgfscope}%
\definecolor{textcolor}{rgb}{0.000000,0.000000,0.000000}%
\pgfsetstrokecolor{textcolor}%
\pgfsetfillcolor{textcolor}%
\pgftext[x=0.617283in, y=0.570100in, left, base]{\color{textcolor}\rmfamily\fontsize{10.000000}{12.000000}\selectfont \(\displaystyle {\ensuremath{-}1.5}\)}%
\end{pgfscope}%
\begin{pgfscope}%
\pgfpathrectangle{\pgfqpoint{1.000000in}{0.330000in}}{\pgfqpoint{6.200000in}{2.310000in}}%
\pgfusepath{clip}%
\pgfsetbuttcap%
\pgfsetroundjoin%
\pgfsetlinewidth{0.803000pt}%
\definecolor{currentstroke}{rgb}{0.000000,0.000000,0.000000}%
\pgfsetstrokecolor{currentstroke}%
\pgfsetdash{{0.800000pt}{1.320000pt}}{0.000000pt}%
\pgfpathmoveto{\pgfqpoint{1.000000in}{1.022481in}}%
\pgfpathlineto{\pgfqpoint{7.200000in}{1.022481in}}%
\pgfusepath{stroke}%
\end{pgfscope}%
\begin{pgfscope}%
\pgfsetbuttcap%
\pgfsetroundjoin%
\definecolor{currentfill}{rgb}{0.000000,0.000000,0.000000}%
\pgfsetfillcolor{currentfill}%
\pgfsetlinewidth{0.803000pt}%
\definecolor{currentstroke}{rgb}{0.000000,0.000000,0.000000}%
\pgfsetstrokecolor{currentstroke}%
\pgfsetdash{}{0pt}%
\pgfsys@defobject{currentmarker}{\pgfqpoint{-0.048611in}{0.000000in}}{\pgfqpoint{-0.000000in}{0.000000in}}{%
\pgfpathmoveto{\pgfqpoint{-0.000000in}{0.000000in}}%
\pgfpathlineto{\pgfqpoint{-0.048611in}{0.000000in}}%
\pgfusepath{stroke,fill}%
}%
\begin{pgfscope}%
\pgfsys@transformshift{1.000000in}{1.022481in}%
\pgfsys@useobject{currentmarker}{}%
\end{pgfscope}%
\end{pgfscope}%
\begin{pgfscope}%
\pgfsetbuttcap%
\pgfsetroundjoin%
\definecolor{currentfill}{rgb}{0.000000,0.000000,0.000000}%
\pgfsetfillcolor{currentfill}%
\pgfsetlinewidth{0.803000pt}%
\definecolor{currentstroke}{rgb}{0.000000,0.000000,0.000000}%
\pgfsetstrokecolor{currentstroke}%
\pgfsetdash{}{0pt}%
\pgfsys@defobject{currentmarker}{\pgfqpoint{0.000000in}{0.000000in}}{\pgfqpoint{0.048611in}{0.000000in}}{%
\pgfpathmoveto{\pgfqpoint{0.000000in}{0.000000in}}%
\pgfpathlineto{\pgfqpoint{0.048611in}{0.000000in}}%
\pgfusepath{stroke,fill}%
}%
\begin{pgfscope}%
\pgfsys@transformshift{7.200000in}{1.022481in}%
\pgfsys@useobject{currentmarker}{}%
\end{pgfscope}%
\end{pgfscope}%
\begin{pgfscope}%
\definecolor{textcolor}{rgb}{0.000000,0.000000,0.000000}%
\pgfsetstrokecolor{textcolor}%
\pgfsetfillcolor{textcolor}%
\pgftext[x=0.617283in, y=0.974256in, left, base]{\color{textcolor}\rmfamily\fontsize{10.000000}{12.000000}\selectfont \(\displaystyle {\ensuremath{-}1.0}\)}%
\end{pgfscope}%
\begin{pgfscope}%
\pgfpathrectangle{\pgfqpoint{1.000000in}{0.330000in}}{\pgfqpoint{6.200000in}{2.310000in}}%
\pgfusepath{clip}%
\pgfsetbuttcap%
\pgfsetroundjoin%
\pgfsetlinewidth{0.803000pt}%
\definecolor{currentstroke}{rgb}{0.000000,0.000000,0.000000}%
\pgfsetstrokecolor{currentstroke}%
\pgfsetdash{{0.800000pt}{1.320000pt}}{0.000000pt}%
\pgfpathmoveto{\pgfqpoint{1.000000in}{1.426638in}}%
\pgfpathlineto{\pgfqpoint{7.200000in}{1.426638in}}%
\pgfusepath{stroke}%
\end{pgfscope}%
\begin{pgfscope}%
\pgfsetbuttcap%
\pgfsetroundjoin%
\definecolor{currentfill}{rgb}{0.000000,0.000000,0.000000}%
\pgfsetfillcolor{currentfill}%
\pgfsetlinewidth{0.803000pt}%
\definecolor{currentstroke}{rgb}{0.000000,0.000000,0.000000}%
\pgfsetstrokecolor{currentstroke}%
\pgfsetdash{}{0pt}%
\pgfsys@defobject{currentmarker}{\pgfqpoint{-0.048611in}{0.000000in}}{\pgfqpoint{-0.000000in}{0.000000in}}{%
\pgfpathmoveto{\pgfqpoint{-0.000000in}{0.000000in}}%
\pgfpathlineto{\pgfqpoint{-0.048611in}{0.000000in}}%
\pgfusepath{stroke,fill}%
}%
\begin{pgfscope}%
\pgfsys@transformshift{1.000000in}{1.426638in}%
\pgfsys@useobject{currentmarker}{}%
\end{pgfscope}%
\end{pgfscope}%
\begin{pgfscope}%
\pgfsetbuttcap%
\pgfsetroundjoin%
\definecolor{currentfill}{rgb}{0.000000,0.000000,0.000000}%
\pgfsetfillcolor{currentfill}%
\pgfsetlinewidth{0.803000pt}%
\definecolor{currentstroke}{rgb}{0.000000,0.000000,0.000000}%
\pgfsetstrokecolor{currentstroke}%
\pgfsetdash{}{0pt}%
\pgfsys@defobject{currentmarker}{\pgfqpoint{0.000000in}{0.000000in}}{\pgfqpoint{0.048611in}{0.000000in}}{%
\pgfpathmoveto{\pgfqpoint{0.000000in}{0.000000in}}%
\pgfpathlineto{\pgfqpoint{0.048611in}{0.000000in}}%
\pgfusepath{stroke,fill}%
}%
\begin{pgfscope}%
\pgfsys@transformshift{7.200000in}{1.426638in}%
\pgfsys@useobject{currentmarker}{}%
\end{pgfscope}%
\end{pgfscope}%
\begin{pgfscope}%
\definecolor{textcolor}{rgb}{0.000000,0.000000,0.000000}%
\pgfsetstrokecolor{textcolor}%
\pgfsetfillcolor{textcolor}%
\pgftext[x=0.617283in, y=1.378412in, left, base]{\color{textcolor}\rmfamily\fontsize{10.000000}{12.000000}\selectfont \(\displaystyle {\ensuremath{-}0.5}\)}%
\end{pgfscope}%
\begin{pgfscope}%
\pgfpathrectangle{\pgfqpoint{1.000000in}{0.330000in}}{\pgfqpoint{6.200000in}{2.310000in}}%
\pgfusepath{clip}%
\pgfsetbuttcap%
\pgfsetroundjoin%
\pgfsetlinewidth{0.803000pt}%
\definecolor{currentstroke}{rgb}{0.000000,0.000000,0.000000}%
\pgfsetstrokecolor{currentstroke}%
\pgfsetdash{{0.800000pt}{1.320000pt}}{0.000000pt}%
\pgfpathmoveto{\pgfqpoint{1.000000in}{1.830794in}}%
\pgfpathlineto{\pgfqpoint{7.200000in}{1.830794in}}%
\pgfusepath{stroke}%
\end{pgfscope}%
\begin{pgfscope}%
\pgfsetbuttcap%
\pgfsetroundjoin%
\definecolor{currentfill}{rgb}{0.000000,0.000000,0.000000}%
\pgfsetfillcolor{currentfill}%
\pgfsetlinewidth{0.803000pt}%
\definecolor{currentstroke}{rgb}{0.000000,0.000000,0.000000}%
\pgfsetstrokecolor{currentstroke}%
\pgfsetdash{}{0pt}%
\pgfsys@defobject{currentmarker}{\pgfqpoint{-0.048611in}{0.000000in}}{\pgfqpoint{-0.000000in}{0.000000in}}{%
\pgfpathmoveto{\pgfqpoint{-0.000000in}{0.000000in}}%
\pgfpathlineto{\pgfqpoint{-0.048611in}{0.000000in}}%
\pgfusepath{stroke,fill}%
}%
\begin{pgfscope}%
\pgfsys@transformshift{1.000000in}{1.830794in}%
\pgfsys@useobject{currentmarker}{}%
\end{pgfscope}%
\end{pgfscope}%
\begin{pgfscope}%
\pgfsetbuttcap%
\pgfsetroundjoin%
\definecolor{currentfill}{rgb}{0.000000,0.000000,0.000000}%
\pgfsetfillcolor{currentfill}%
\pgfsetlinewidth{0.803000pt}%
\definecolor{currentstroke}{rgb}{0.000000,0.000000,0.000000}%
\pgfsetstrokecolor{currentstroke}%
\pgfsetdash{}{0pt}%
\pgfsys@defobject{currentmarker}{\pgfqpoint{0.000000in}{0.000000in}}{\pgfqpoint{0.048611in}{0.000000in}}{%
\pgfpathmoveto{\pgfqpoint{0.000000in}{0.000000in}}%
\pgfpathlineto{\pgfqpoint{0.048611in}{0.000000in}}%
\pgfusepath{stroke,fill}%
}%
\begin{pgfscope}%
\pgfsys@transformshift{7.200000in}{1.830794in}%
\pgfsys@useobject{currentmarker}{}%
\end{pgfscope}%
\end{pgfscope}%
\begin{pgfscope}%
\definecolor{textcolor}{rgb}{0.000000,0.000000,0.000000}%
\pgfsetstrokecolor{textcolor}%
\pgfsetfillcolor{textcolor}%
\pgftext[x=0.725308in, y=1.782568in, left, base]{\color{textcolor}\rmfamily\fontsize{10.000000}{12.000000}\selectfont \(\displaystyle {0.0}\)}%
\end{pgfscope}%
\begin{pgfscope}%
\pgfpathrectangle{\pgfqpoint{1.000000in}{0.330000in}}{\pgfqpoint{6.200000in}{2.310000in}}%
\pgfusepath{clip}%
\pgfsetbuttcap%
\pgfsetroundjoin%
\pgfsetlinewidth{0.803000pt}%
\definecolor{currentstroke}{rgb}{0.000000,0.000000,0.000000}%
\pgfsetstrokecolor{currentstroke}%
\pgfsetdash{{0.800000pt}{1.320000pt}}{0.000000pt}%
\pgfpathmoveto{\pgfqpoint{1.000000in}{2.234950in}}%
\pgfpathlineto{\pgfqpoint{7.200000in}{2.234950in}}%
\pgfusepath{stroke}%
\end{pgfscope}%
\begin{pgfscope}%
\pgfsetbuttcap%
\pgfsetroundjoin%
\definecolor{currentfill}{rgb}{0.000000,0.000000,0.000000}%
\pgfsetfillcolor{currentfill}%
\pgfsetlinewidth{0.803000pt}%
\definecolor{currentstroke}{rgb}{0.000000,0.000000,0.000000}%
\pgfsetstrokecolor{currentstroke}%
\pgfsetdash{}{0pt}%
\pgfsys@defobject{currentmarker}{\pgfqpoint{-0.048611in}{0.000000in}}{\pgfqpoint{-0.000000in}{0.000000in}}{%
\pgfpathmoveto{\pgfqpoint{-0.000000in}{0.000000in}}%
\pgfpathlineto{\pgfqpoint{-0.048611in}{0.000000in}}%
\pgfusepath{stroke,fill}%
}%
\begin{pgfscope}%
\pgfsys@transformshift{1.000000in}{2.234950in}%
\pgfsys@useobject{currentmarker}{}%
\end{pgfscope}%
\end{pgfscope}%
\begin{pgfscope}%
\pgfsetbuttcap%
\pgfsetroundjoin%
\definecolor{currentfill}{rgb}{0.000000,0.000000,0.000000}%
\pgfsetfillcolor{currentfill}%
\pgfsetlinewidth{0.803000pt}%
\definecolor{currentstroke}{rgb}{0.000000,0.000000,0.000000}%
\pgfsetstrokecolor{currentstroke}%
\pgfsetdash{}{0pt}%
\pgfsys@defobject{currentmarker}{\pgfqpoint{0.000000in}{0.000000in}}{\pgfqpoint{0.048611in}{0.000000in}}{%
\pgfpathmoveto{\pgfqpoint{0.000000in}{0.000000in}}%
\pgfpathlineto{\pgfqpoint{0.048611in}{0.000000in}}%
\pgfusepath{stroke,fill}%
}%
\begin{pgfscope}%
\pgfsys@transformshift{7.200000in}{2.234950in}%
\pgfsys@useobject{currentmarker}{}%
\end{pgfscope}%
\end{pgfscope}%
\begin{pgfscope}%
\definecolor{textcolor}{rgb}{0.000000,0.000000,0.000000}%
\pgfsetstrokecolor{textcolor}%
\pgfsetfillcolor{textcolor}%
\pgftext[x=0.725308in, y=2.186725in, left, base]{\color{textcolor}\rmfamily\fontsize{10.000000}{12.000000}\selectfont \(\displaystyle {0.5}\)}%
\end{pgfscope}%
\begin{pgfscope}%
\pgfpathrectangle{\pgfqpoint{1.000000in}{0.330000in}}{\pgfqpoint{6.200000in}{2.310000in}}%
\pgfusepath{clip}%
\pgfsetbuttcap%
\pgfsetroundjoin%
\pgfsetlinewidth{0.803000pt}%
\definecolor{currentstroke}{rgb}{0.000000,0.000000,0.000000}%
\pgfsetstrokecolor{currentstroke}%
\pgfsetdash{{0.800000pt}{1.320000pt}}{0.000000pt}%
\pgfpathmoveto{\pgfqpoint{1.000000in}{2.639106in}}%
\pgfpathlineto{\pgfqpoint{7.200000in}{2.639106in}}%
\pgfusepath{stroke}%
\end{pgfscope}%
\begin{pgfscope}%
\pgfsetbuttcap%
\pgfsetroundjoin%
\definecolor{currentfill}{rgb}{0.000000,0.000000,0.000000}%
\pgfsetfillcolor{currentfill}%
\pgfsetlinewidth{0.803000pt}%
\definecolor{currentstroke}{rgb}{0.000000,0.000000,0.000000}%
\pgfsetstrokecolor{currentstroke}%
\pgfsetdash{}{0pt}%
\pgfsys@defobject{currentmarker}{\pgfqpoint{-0.048611in}{0.000000in}}{\pgfqpoint{-0.000000in}{0.000000in}}{%
\pgfpathmoveto{\pgfqpoint{-0.000000in}{0.000000in}}%
\pgfpathlineto{\pgfqpoint{-0.048611in}{0.000000in}}%
\pgfusepath{stroke,fill}%
}%
\begin{pgfscope}%
\pgfsys@transformshift{1.000000in}{2.639106in}%
\pgfsys@useobject{currentmarker}{}%
\end{pgfscope}%
\end{pgfscope}%
\begin{pgfscope}%
\pgfsetbuttcap%
\pgfsetroundjoin%
\definecolor{currentfill}{rgb}{0.000000,0.000000,0.000000}%
\pgfsetfillcolor{currentfill}%
\pgfsetlinewidth{0.803000pt}%
\definecolor{currentstroke}{rgb}{0.000000,0.000000,0.000000}%
\pgfsetstrokecolor{currentstroke}%
\pgfsetdash{}{0pt}%
\pgfsys@defobject{currentmarker}{\pgfqpoint{0.000000in}{0.000000in}}{\pgfqpoint{0.048611in}{0.000000in}}{%
\pgfpathmoveto{\pgfqpoint{0.000000in}{0.000000in}}%
\pgfpathlineto{\pgfqpoint{0.048611in}{0.000000in}}%
\pgfusepath{stroke,fill}%
}%
\begin{pgfscope}%
\pgfsys@transformshift{7.200000in}{2.639106in}%
\pgfsys@useobject{currentmarker}{}%
\end{pgfscope}%
\end{pgfscope}%
\begin{pgfscope}%
\definecolor{textcolor}{rgb}{0.000000,0.000000,0.000000}%
\pgfsetstrokecolor{textcolor}%
\pgfsetfillcolor{textcolor}%
\pgftext[x=0.725308in, y=2.590881in, left, base]{\color{textcolor}\rmfamily\fontsize{10.000000}{12.000000}\selectfont \(\displaystyle {1.0}\)}%
\end{pgfscope}%
\begin{pgfscope}%
\pgfpathrectangle{\pgfqpoint{1.000000in}{0.330000in}}{\pgfqpoint{6.200000in}{2.310000in}}%
\pgfusepath{clip}%
\pgfsetrectcap%
\pgfsetroundjoin%
\pgfsetlinewidth{1.505625pt}%
\definecolor{currentstroke}{rgb}{0.121569,0.466667,0.705882}%
\pgfsetstrokecolor{currentstroke}%
\pgfsetdash{}{0pt}%
\pgfpathmoveto{\pgfqpoint{1.281818in}{1.899012in}}%
\pgfpathlineto{\pgfqpoint{2.193983in}{1.813369in}}%
\pgfpathlineto{\pgfqpoint{2.588801in}{1.775571in}}%
\pgfpathlineto{\pgfqpoint{2.847475in}{1.753068in}}%
\pgfpathlineto{\pgfqpoint{3.065305in}{1.736303in}}%
\pgfpathlineto{\pgfqpoint{3.269521in}{1.722824in}}%
\pgfpathlineto{\pgfqpoint{3.460123in}{1.712470in}}%
\pgfpathlineto{\pgfqpoint{3.637110in}{1.704956in}}%
\pgfpathlineto{\pgfqpoint{3.814097in}{1.699588in}}%
\pgfpathlineto{\pgfqpoint{3.991085in}{1.696453in}}%
\pgfpathlineto{\pgfqpoint{4.168072in}{1.695598in}}%
\pgfpathlineto{\pgfqpoint{4.345059in}{1.697039in}}%
\pgfpathlineto{\pgfqpoint{4.522047in}{1.700756in}}%
\pgfpathlineto{\pgfqpoint{4.699034in}{1.706692in}}%
\pgfpathlineto{\pgfqpoint{4.876021in}{1.714758in}}%
\pgfpathlineto{\pgfqpoint{5.066623in}{1.725681in}}%
\pgfpathlineto{\pgfqpoint{5.270839in}{1.739731in}}%
\pgfpathlineto{\pgfqpoint{5.488669in}{1.757051in}}%
\pgfpathlineto{\pgfqpoint{5.733729in}{1.778858in}}%
\pgfpathlineto{\pgfqpoint{6.046860in}{1.809103in}}%
\pgfpathlineto{\pgfqpoint{6.536978in}{1.856643in}}%
\pgfpathlineto{\pgfqpoint{6.918182in}{1.892701in}}%
\pgfpathlineto{\pgfqpoint{6.918182in}{1.892701in}}%
\pgfusepath{stroke}%
\end{pgfscope}%
\begin{pgfscope}%
\pgfpathrectangle{\pgfqpoint{1.000000in}{0.330000in}}{\pgfqpoint{6.200000in}{2.310000in}}%
\pgfusepath{clip}%
\pgfsetrectcap%
\pgfsetroundjoin%
\pgfsetlinewidth{1.505625pt}%
\definecolor{currentstroke}{rgb}{1.000000,0.498039,0.054902}%
\pgfsetstrokecolor{currentstroke}%
\pgfsetdash{}{0pt}%
\pgfpathmoveto{\pgfqpoint{1.281818in}{1.866698in}}%
\pgfpathlineto{\pgfqpoint{2.425428in}{1.809777in}}%
\pgfpathlineto{\pgfqpoint{2.820246in}{1.791069in}}%
\pgfpathlineto{\pgfqpoint{3.119763in}{1.779051in}}%
\pgfpathlineto{\pgfqpoint{3.392051in}{1.770327in}}%
\pgfpathlineto{\pgfqpoint{3.650725in}{1.764305in}}%
\pgfpathlineto{\pgfqpoint{3.895784in}{1.760829in}}%
\pgfpathlineto{\pgfqpoint{4.140843in}{1.759629in}}%
\pgfpathlineto{\pgfqpoint{4.385903in}{1.760743in}}%
\pgfpathlineto{\pgfqpoint{4.630962in}{1.764140in}}%
\pgfpathlineto{\pgfqpoint{4.876021in}{1.769722in}}%
\pgfpathlineto{\pgfqpoint{5.134695in}{1.777801in}}%
\pgfpathlineto{\pgfqpoint{5.420597in}{1.789010in}}%
\pgfpathlineto{\pgfqpoint{5.747343in}{1.804127in}}%
\pgfpathlineto{\pgfqpoint{6.237462in}{1.829367in}}%
\pgfpathlineto{\pgfqpoint{6.700351in}{1.852526in}}%
\pgfpathlineto{\pgfqpoint{6.918182in}{1.863376in}}%
\pgfpathlineto{\pgfqpoint{6.918182in}{1.863376in}}%
\pgfusepath{stroke}%
\end{pgfscope}%
\begin{pgfscope}%
\pgfpathrectangle{\pgfqpoint{1.000000in}{0.330000in}}{\pgfqpoint{6.200000in}{2.310000in}}%
\pgfusepath{clip}%
\pgfsetrectcap%
\pgfsetroundjoin%
\pgfsetlinewidth{1.505625pt}%
\definecolor{currentstroke}{rgb}{0.172549,0.627451,0.172549}%
\pgfsetstrokecolor{currentstroke}%
\pgfsetdash{}{0pt}%
\pgfpathmoveto{\pgfqpoint{1.281818in}{1.883796in}}%
\pgfpathlineto{\pgfqpoint{2.180369in}{1.817984in}}%
\pgfpathlineto{\pgfqpoint{2.670487in}{1.780431in}}%
\pgfpathlineto{\pgfqpoint{2.942776in}{1.761850in}}%
\pgfpathlineto{\pgfqpoint{3.174220in}{1.748196in}}%
\pgfpathlineto{\pgfqpoint{3.392051in}{1.737554in}}%
\pgfpathlineto{\pgfqpoint{3.596267in}{1.729780in}}%
\pgfpathlineto{\pgfqpoint{3.800483in}{1.724323in}}%
\pgfpathlineto{\pgfqpoint{3.991085in}{1.721428in}}%
\pgfpathlineto{\pgfqpoint{4.181686in}{1.720715in}}%
\pgfpathlineto{\pgfqpoint{4.372288in}{1.722201in}}%
\pgfpathlineto{\pgfqpoint{4.562890in}{1.725859in}}%
\pgfpathlineto{\pgfqpoint{4.767106in}{1.732113in}}%
\pgfpathlineto{\pgfqpoint{4.971322in}{1.740649in}}%
\pgfpathlineto{\pgfqpoint{5.189152in}{1.752055in}}%
\pgfpathlineto{\pgfqpoint{5.420597in}{1.766446in}}%
\pgfpathlineto{\pgfqpoint{5.679271in}{1.784761in}}%
\pgfpathlineto{\pgfqpoint{6.033245in}{1.812278in}}%
\pgfpathlineto{\pgfqpoint{6.482521in}{1.847135in}}%
\pgfpathlineto{\pgfqpoint{6.918182in}{1.879575in}}%
\pgfpathlineto{\pgfqpoint{6.918182in}{1.879575in}}%
\pgfusepath{stroke}%
\end{pgfscope}%
\begin{pgfscope}%
\pgfpathrectangle{\pgfqpoint{1.000000in}{0.330000in}}{\pgfqpoint{6.200000in}{2.310000in}}%
\pgfusepath{clip}%
\pgfsetrectcap%
\pgfsetroundjoin%
\pgfsetlinewidth{1.505625pt}%
\definecolor{currentstroke}{rgb}{0.839216,0.152941,0.156863}%
\pgfsetstrokecolor{currentstroke}%
\pgfsetdash{}{0pt}%
\pgfpathmoveto{\pgfqpoint{1.281818in}{2.535000in}}%
\pgfpathlineto{\pgfqpoint{2.030610in}{1.816996in}}%
\pgfpathlineto{\pgfqpoint{2.398199in}{1.445497in}}%
\pgfpathlineto{\pgfqpoint{2.520729in}{1.325663in}}%
\pgfpathlineto{\pgfqpoint{2.629644in}{1.222455in}}%
\pgfpathlineto{\pgfqpoint{2.724945in}{1.135323in}}%
\pgfpathlineto{\pgfqpoint{2.806632in}{1.063376in}}%
\pgfpathlineto{\pgfqpoint{2.888318in}{0.994250in}}%
\pgfpathlineto{\pgfqpoint{2.970004in}{0.928207in}}%
\pgfpathlineto{\pgfqpoint{3.038076in}{0.875706in}}%
\pgfpathlineto{\pgfqpoint{3.106148in}{0.825651in}}%
\pgfpathlineto{\pgfqpoint{3.174220in}{0.778168in}}%
\pgfpathlineto{\pgfqpoint{3.242292in}{0.733375in}}%
\pgfpathlineto{\pgfqpoint{3.310365in}{0.691383in}}%
\pgfpathlineto{\pgfqpoint{3.378437in}{0.652293in}}%
\pgfpathlineto{\pgfqpoint{3.446509in}{0.616200in}}%
\pgfpathlineto{\pgfqpoint{3.500966in}{0.589541in}}%
\pgfpathlineto{\pgfqpoint{3.555424in}{0.564895in}}%
\pgfpathlineto{\pgfqpoint{3.609881in}{0.542300in}}%
\pgfpathlineto{\pgfqpoint{3.664339in}{0.521790in}}%
\pgfpathlineto{\pgfqpoint{3.718797in}{0.503396in}}%
\pgfpathlineto{\pgfqpoint{3.773254in}{0.487145in}}%
\pgfpathlineto{\pgfqpoint{3.827712in}{0.473061in}}%
\pgfpathlineto{\pgfqpoint{3.882170in}{0.461165in}}%
\pgfpathlineto{\pgfqpoint{3.936627in}{0.451476in}}%
\pgfpathlineto{\pgfqpoint{3.991085in}{0.444007in}}%
\pgfpathlineto{\pgfqpoint{4.045542in}{0.438771in}}%
\pgfpathlineto{\pgfqpoint{4.100000in}{0.435774in}}%
\pgfpathlineto{\pgfqpoint{4.154458in}{0.435023in}}%
\pgfpathlineto{\pgfqpoint{4.208915in}{0.436518in}}%
\pgfpathlineto{\pgfqpoint{4.263373in}{0.440257in}}%
\pgfpathlineto{\pgfqpoint{4.317830in}{0.446237in}}%
\pgfpathlineto{\pgfqpoint{4.372288in}{0.454448in}}%
\pgfpathlineto{\pgfqpoint{4.426746in}{0.464879in}}%
\pgfpathlineto{\pgfqpoint{4.481203in}{0.477516in}}%
\pgfpathlineto{\pgfqpoint{4.535661in}{0.492340in}}%
\pgfpathlineto{\pgfqpoint{4.590119in}{0.509332in}}%
\pgfpathlineto{\pgfqpoint{4.644576in}{0.528465in}}%
\pgfpathlineto{\pgfqpoint{4.699034in}{0.549712in}}%
\pgfpathlineto{\pgfqpoint{4.753491in}{0.573044in}}%
\pgfpathlineto{\pgfqpoint{4.807949in}{0.598424in}}%
\pgfpathlineto{\pgfqpoint{4.862407in}{0.625817in}}%
\pgfpathlineto{\pgfqpoint{4.930479in}{0.662826in}}%
\pgfpathlineto{\pgfqpoint{4.998551in}{0.702828in}}%
\pgfpathlineto{\pgfqpoint{5.066623in}{0.745729in}}%
\pgfpathlineto{\pgfqpoint{5.134695in}{0.791427in}}%
\pgfpathlineto{\pgfqpoint{5.202767in}{0.839813in}}%
\pgfpathlineto{\pgfqpoint{5.270839in}{0.890766in}}%
\pgfpathlineto{\pgfqpoint{5.338911in}{0.944163in}}%
\pgfpathlineto{\pgfqpoint{5.420597in}{1.011273in}}%
\pgfpathlineto{\pgfqpoint{5.502284in}{1.081461in}}%
\pgfpathlineto{\pgfqpoint{5.583970in}{1.154464in}}%
\pgfpathlineto{\pgfqpoint{5.679271in}{1.242816in}}%
\pgfpathlineto{\pgfqpoint{5.774572in}{1.334146in}}%
\pgfpathlineto{\pgfqpoint{5.883487in}{1.441523in}}%
\pgfpathlineto{\pgfqpoint{6.019631in}{1.579068in}}%
\pgfpathlineto{\pgfqpoint{6.346377in}{1.911150in}}%
\pgfpathlineto{\pgfqpoint{6.713966in}{2.270332in}}%
\pgfpathlineto{\pgfqpoint{6.918182in}{2.469851in}}%
\pgfpathlineto{\pgfqpoint{6.918182in}{2.469851in}}%
\pgfusepath{stroke}%
\end{pgfscope}%
\begin{pgfscope}%
\pgfpathrectangle{\pgfqpoint{1.000000in}{0.330000in}}{\pgfqpoint{6.200000in}{2.310000in}}%
\pgfusepath{clip}%
\pgfsetrectcap%
\pgfsetroundjoin%
\pgfsetlinewidth{1.505625pt}%
\definecolor{currentstroke}{rgb}{0.580392,0.403922,0.741176}%
\pgfsetstrokecolor{currentstroke}%
\pgfsetdash{}{0pt}%
\pgfpathmoveto{\pgfqpoint{1.281818in}{2.380192in}}%
\pgfpathlineto{\pgfqpoint{2.030610in}{1.820029in}}%
\pgfpathlineto{\pgfqpoint{2.411814in}{1.519684in}}%
\pgfpathlineto{\pgfqpoint{2.534343in}{1.426496in}}%
\pgfpathlineto{\pgfqpoint{2.643259in}{1.346325in}}%
\pgfpathlineto{\pgfqpoint{2.738560in}{1.278712in}}%
\pgfpathlineto{\pgfqpoint{2.833860in}{1.213849in}}%
\pgfpathlineto{\pgfqpoint{2.915547in}{1.160699in}}%
\pgfpathlineto{\pgfqpoint{2.997233in}{1.110020in}}%
\pgfpathlineto{\pgfqpoint{3.078920in}{1.062000in}}%
\pgfpathlineto{\pgfqpoint{3.160606in}{1.016814in}}%
\pgfpathlineto{\pgfqpoint{3.228678in}{0.981442in}}%
\pgfpathlineto{\pgfqpoint{3.296750in}{0.948237in}}%
\pgfpathlineto{\pgfqpoint{3.364822in}{0.917282in}}%
\pgfpathlineto{\pgfqpoint{3.432894in}{0.888650in}}%
\pgfpathlineto{\pgfqpoint{3.500966in}{0.862410in}}%
\pgfpathlineto{\pgfqpoint{3.569038in}{0.838624in}}%
\pgfpathlineto{\pgfqpoint{3.637110in}{0.817349in}}%
\pgfpathlineto{\pgfqpoint{3.705182in}{0.798634in}}%
\pgfpathlineto{\pgfqpoint{3.773254in}{0.782524in}}%
\pgfpathlineto{\pgfqpoint{3.841326in}{0.769055in}}%
\pgfpathlineto{\pgfqpoint{3.909398in}{0.758260in}}%
\pgfpathlineto{\pgfqpoint{3.977470in}{0.750163in}}%
\pgfpathlineto{\pgfqpoint{4.045542in}{0.744784in}}%
\pgfpathlineto{\pgfqpoint{4.113614in}{0.742136in}}%
\pgfpathlineto{\pgfqpoint{4.181686in}{0.742224in}}%
\pgfpathlineto{\pgfqpoint{4.249758in}{0.745050in}}%
\pgfpathlineto{\pgfqpoint{4.317830in}{0.750609in}}%
\pgfpathlineto{\pgfqpoint{4.385903in}{0.758887in}}%
\pgfpathlineto{\pgfqpoint{4.453975in}{0.769868in}}%
\pgfpathlineto{\pgfqpoint{4.522047in}{0.783527in}}%
\pgfpathlineto{\pgfqpoint{4.590119in}{0.799833in}}%
\pgfpathlineto{\pgfqpoint{4.658191in}{0.818751in}}%
\pgfpathlineto{\pgfqpoint{4.726263in}{0.840237in}}%
\pgfpathlineto{\pgfqpoint{4.794335in}{0.864242in}}%
\pgfpathlineto{\pgfqpoint{4.862407in}{0.890712in}}%
\pgfpathlineto{\pgfqpoint{4.930479in}{0.919584in}}%
\pgfpathlineto{\pgfqpoint{4.998551in}{0.950793in}}%
\pgfpathlineto{\pgfqpoint{5.066623in}{0.984263in}}%
\pgfpathlineto{\pgfqpoint{5.134695in}{1.019915in}}%
\pgfpathlineto{\pgfqpoint{5.216381in}{1.065457in}}%
\pgfpathlineto{\pgfqpoint{5.298068in}{1.113856in}}%
\pgfpathlineto{\pgfqpoint{5.379754in}{1.164940in}}%
\pgfpathlineto{\pgfqpoint{5.461440in}{1.218523in}}%
\pgfpathlineto{\pgfqpoint{5.556741in}{1.283928in}}%
\pgfpathlineto{\pgfqpoint{5.652042in}{1.352126in}}%
\pgfpathlineto{\pgfqpoint{5.760957in}{1.433019in}}%
\pgfpathlineto{\pgfqpoint{5.883487in}{1.527098in}}%
\pgfpathlineto{\pgfqpoint{6.033245in}{1.645250in}}%
\pgfpathlineto{\pgfqpoint{6.332762in}{1.883056in}}%
\pgfpathlineto{\pgfqpoint{6.713966in}{2.173706in}}%
\pgfpathlineto{\pgfqpoint{6.918182in}{2.329365in}}%
\pgfpathlineto{\pgfqpoint{6.918182in}{2.329365in}}%
\pgfusepath{stroke}%
\end{pgfscope}%
\begin{pgfscope}%
\pgfpathrectangle{\pgfqpoint{1.000000in}{0.330000in}}{\pgfqpoint{6.200000in}{2.310000in}}%
\pgfusepath{clip}%
\pgfsetrectcap%
\pgfsetroundjoin%
\pgfsetlinewidth{1.505625pt}%
\definecolor{currentstroke}{rgb}{0.549020,0.337255,0.294118}%
\pgfsetstrokecolor{currentstroke}%
\pgfsetdash{}{0pt}%
\pgfpathmoveto{\pgfqpoint{1.281818in}{1.888913in}}%
\pgfpathlineto{\pgfqpoint{2.221212in}{1.813643in}}%
\pgfpathlineto{\pgfqpoint{2.629644in}{1.780523in}}%
\pgfpathlineto{\pgfqpoint{2.888318in}{1.761688in}}%
\pgfpathlineto{\pgfqpoint{3.119763in}{1.746996in}}%
\pgfpathlineto{\pgfqpoint{3.337593in}{1.735443in}}%
\pgfpathlineto{\pgfqpoint{3.541809in}{1.726882in}}%
\pgfpathlineto{\pgfqpoint{3.732411in}{1.721040in}}%
\pgfpathlineto{\pgfqpoint{3.923013in}{1.717385in}}%
\pgfpathlineto{\pgfqpoint{4.113614in}{1.715979in}}%
\pgfpathlineto{\pgfqpoint{4.304216in}{1.716848in}}%
\pgfpathlineto{\pgfqpoint{4.494818in}{1.719973in}}%
\pgfpathlineto{\pgfqpoint{4.685419in}{1.725297in}}%
\pgfpathlineto{\pgfqpoint{4.889635in}{1.733325in}}%
\pgfpathlineto{\pgfqpoint{5.093852in}{1.743584in}}%
\pgfpathlineto{\pgfqpoint{5.311682in}{1.756726in}}%
\pgfpathlineto{\pgfqpoint{5.556741in}{1.773795in}}%
\pgfpathlineto{\pgfqpoint{5.856258in}{1.797079in}}%
\pgfpathlineto{\pgfqpoint{6.768423in}{1.869880in}}%
\pgfpathlineto{\pgfqpoint{6.918182in}{1.881485in}}%
\pgfpathlineto{\pgfqpoint{6.918182in}{1.881485in}}%
\pgfusepath{stroke}%
\end{pgfscope}%
\begin{pgfscope}%
\pgfpathrectangle{\pgfqpoint{1.000000in}{0.330000in}}{\pgfqpoint{6.200000in}{2.310000in}}%
\pgfusepath{clip}%
\pgfsetrectcap%
\pgfsetroundjoin%
\pgfsetlinewidth{1.505625pt}%
\definecolor{currentstroke}{rgb}{0.890196,0.466667,0.760784}%
\pgfsetstrokecolor{currentstroke}%
\pgfsetdash{}{0pt}%
\pgfpathmoveto{\pgfqpoint{1.281818in}{2.373909in}}%
\pgfpathlineto{\pgfqpoint{2.030610in}{1.820152in}}%
\pgfpathlineto{\pgfqpoint{2.411814in}{1.523242in}}%
\pgfpathlineto{\pgfqpoint{2.534343in}{1.431120in}}%
\pgfpathlineto{\pgfqpoint{2.643259in}{1.351866in}}%
\pgfpathlineto{\pgfqpoint{2.738560in}{1.285026in}}%
\pgfpathlineto{\pgfqpoint{2.833860in}{1.220904in}}%
\pgfpathlineto{\pgfqpoint{2.915547in}{1.168362in}}%
\pgfpathlineto{\pgfqpoint{2.997233in}{1.118263in}}%
\pgfpathlineto{\pgfqpoint{3.078920in}{1.070792in}}%
\pgfpathlineto{\pgfqpoint{3.160606in}{1.026124in}}%
\pgfpathlineto{\pgfqpoint{3.228678in}{0.991156in}}%
\pgfpathlineto{\pgfqpoint{3.296750in}{0.958331in}}%
\pgfpathlineto{\pgfqpoint{3.364822in}{0.927729in}}%
\pgfpathlineto{\pgfqpoint{3.432894in}{0.899425in}}%
\pgfpathlineto{\pgfqpoint{3.500966in}{0.873485in}}%
\pgfpathlineto{\pgfqpoint{3.569038in}{0.849971in}}%
\pgfpathlineto{\pgfqpoint{3.637110in}{0.828939in}}%
\pgfpathlineto{\pgfqpoint{3.705182in}{0.810439in}}%
\pgfpathlineto{\pgfqpoint{3.773254in}{0.794512in}}%
\pgfpathlineto{\pgfqpoint{3.841326in}{0.781198in}}%
\pgfpathlineto{\pgfqpoint{3.909398in}{0.770526in}}%
\pgfpathlineto{\pgfqpoint{3.977470in}{0.762522in}}%
\pgfpathlineto{\pgfqpoint{4.045542in}{0.757204in}}%
\pgfpathlineto{\pgfqpoint{4.113614in}{0.754586in}}%
\pgfpathlineto{\pgfqpoint{4.181686in}{0.754674in}}%
\pgfpathlineto{\pgfqpoint{4.249758in}{0.757468in}}%
\pgfpathlineto{\pgfqpoint{4.317830in}{0.762962in}}%
\pgfpathlineto{\pgfqpoint{4.385903in}{0.771146in}}%
\pgfpathlineto{\pgfqpoint{4.453975in}{0.782002in}}%
\pgfpathlineto{\pgfqpoint{4.522047in}{0.795504in}}%
\pgfpathlineto{\pgfqpoint{4.590119in}{0.811624in}}%
\pgfpathlineto{\pgfqpoint{4.658191in}{0.830325in}}%
\pgfpathlineto{\pgfqpoint{4.726263in}{0.851565in}}%
\pgfpathlineto{\pgfqpoint{4.794335in}{0.875296in}}%
\pgfpathlineto{\pgfqpoint{4.862407in}{0.901463in}}%
\pgfpathlineto{\pgfqpoint{4.930479in}{0.930005in}}%
\pgfpathlineto{\pgfqpoint{4.998551in}{0.960857in}}%
\pgfpathlineto{\pgfqpoint{5.066623in}{0.993944in}}%
\pgfpathlineto{\pgfqpoint{5.134695in}{1.029189in}}%
\pgfpathlineto{\pgfqpoint{5.216381in}{1.074210in}}%
\pgfpathlineto{\pgfqpoint{5.298068in}{1.122056in}}%
\pgfpathlineto{\pgfqpoint{5.379754in}{1.172555in}}%
\pgfpathlineto{\pgfqpoint{5.461440in}{1.225525in}}%
\pgfpathlineto{\pgfqpoint{5.556741in}{1.290183in}}%
\pgfpathlineto{\pgfqpoint{5.652042in}{1.357600in}}%
\pgfpathlineto{\pgfqpoint{5.760957in}{1.437569in}}%
\pgfpathlineto{\pgfqpoint{5.883487in}{1.530571in}}%
\pgfpathlineto{\pgfqpoint{6.033245in}{1.647372in}}%
\pgfpathlineto{\pgfqpoint{6.332762in}{1.882458in}}%
\pgfpathlineto{\pgfqpoint{6.713966in}{2.169785in}}%
\pgfpathlineto{\pgfqpoint{6.918182in}{2.323663in}}%
\pgfpathlineto{\pgfqpoint{6.918182in}{2.323663in}}%
\pgfusepath{stroke}%
\end{pgfscope}%
\begin{pgfscope}%
\pgfpathrectangle{\pgfqpoint{1.000000in}{0.330000in}}{\pgfqpoint{6.200000in}{2.310000in}}%
\pgfusepath{clip}%
\pgfsetrectcap%
\pgfsetroundjoin%
\pgfsetlinewidth{1.505625pt}%
\definecolor{currentstroke}{rgb}{0.498039,0.498039,0.498039}%
\pgfsetstrokecolor{currentstroke}%
\pgfsetdash{}{0pt}%
\pgfpathmoveto{\pgfqpoint{1.281818in}{1.937110in}}%
\pgfpathlineto{\pgfqpoint{2.180369in}{1.806117in}}%
\pgfpathlineto{\pgfqpoint{2.493500in}{1.759991in}}%
\pgfpathlineto{\pgfqpoint{2.711331in}{1.730051in}}%
\pgfpathlineto{\pgfqpoint{2.901932in}{1.706059in}}%
\pgfpathlineto{\pgfqpoint{3.078920in}{1.686063in}}%
\pgfpathlineto{\pgfqpoint{3.242292in}{1.669851in}}%
\pgfpathlineto{\pgfqpoint{3.392051in}{1.657078in}}%
\pgfpathlineto{\pgfqpoint{3.541809in}{1.646449in}}%
\pgfpathlineto{\pgfqpoint{3.691568in}{1.638080in}}%
\pgfpathlineto{\pgfqpoint{3.841326in}{1.632061in}}%
\pgfpathlineto{\pgfqpoint{3.991085in}{1.628454in}}%
\pgfpathlineto{\pgfqpoint{4.127229in}{1.627301in}}%
\pgfpathlineto{\pgfqpoint{4.263373in}{1.628182in}}%
\pgfpathlineto{\pgfqpoint{4.413131in}{1.631489in}}%
\pgfpathlineto{\pgfqpoint{4.562890in}{1.637210in}}%
\pgfpathlineto{\pgfqpoint{4.712648in}{1.645281in}}%
\pgfpathlineto{\pgfqpoint{4.862407in}{1.655614in}}%
\pgfpathlineto{\pgfqpoint{5.012165in}{1.668092in}}%
\pgfpathlineto{\pgfqpoint{5.175538in}{1.683985in}}%
\pgfpathlineto{\pgfqpoint{5.338911in}{1.702043in}}%
\pgfpathlineto{\pgfqpoint{5.515898in}{1.723751in}}%
\pgfpathlineto{\pgfqpoint{5.720114in}{1.751088in}}%
\pgfpathlineto{\pgfqpoint{5.965173in}{1.786229in}}%
\pgfpathlineto{\pgfqpoint{6.591436in}{1.877276in}}%
\pgfpathlineto{\pgfqpoint{6.918182in}{1.923630in}}%
\pgfpathlineto{\pgfqpoint{6.918182in}{1.923630in}}%
\pgfusepath{stroke}%
\end{pgfscope}%
\begin{pgfscope}%
\pgfpathrectangle{\pgfqpoint{1.000000in}{0.330000in}}{\pgfqpoint{6.200000in}{2.310000in}}%
\pgfusepath{clip}%
\pgfsetrectcap%
\pgfsetroundjoin%
\pgfsetlinewidth{1.505625pt}%
\definecolor{currentstroke}{rgb}{0.737255,0.741176,0.133333}%
\pgfsetstrokecolor{currentstroke}%
\pgfsetdash{}{0pt}%
\pgfpathmoveto{\pgfqpoint{1.281818in}{1.890634in}}%
\pgfpathlineto{\pgfqpoint{2.221212in}{1.813160in}}%
\pgfpathlineto{\pgfqpoint{2.616030in}{1.780179in}}%
\pgfpathlineto{\pgfqpoint{2.874704in}{1.760669in}}%
\pgfpathlineto{\pgfqpoint{3.106148in}{1.745381in}}%
\pgfpathlineto{\pgfqpoint{3.323979in}{1.733287in}}%
\pgfpathlineto{\pgfqpoint{3.528195in}{1.724249in}}%
\pgfpathlineto{\pgfqpoint{3.718797in}{1.717997in}}%
\pgfpathlineto{\pgfqpoint{3.909398in}{1.713973in}}%
\pgfpathlineto{\pgfqpoint{4.100000in}{1.712251in}}%
\pgfpathlineto{\pgfqpoint{4.290602in}{1.712862in}}%
\pgfpathlineto{\pgfqpoint{4.481203in}{1.715798in}}%
\pgfpathlineto{\pgfqpoint{4.671805in}{1.721008in}}%
\pgfpathlineto{\pgfqpoint{4.862407in}{1.728400in}}%
\pgfpathlineto{\pgfqpoint{5.066623in}{1.738590in}}%
\pgfpathlineto{\pgfqpoint{5.284453in}{1.751805in}}%
\pgfpathlineto{\pgfqpoint{5.515898in}{1.768137in}}%
\pgfpathlineto{\pgfqpoint{5.788186in}{1.789718in}}%
\pgfpathlineto{\pgfqpoint{6.169390in}{1.822462in}}%
\pgfpathlineto{\pgfqpoint{6.523364in}{1.852339in}}%
\pgfpathlineto{\pgfqpoint{6.918182in}{1.885098in}}%
\pgfpathlineto{\pgfqpoint{6.918182in}{1.885098in}}%
\pgfusepath{stroke}%
\end{pgfscope}%
\begin{pgfscope}%
\pgfpathrectangle{\pgfqpoint{1.000000in}{0.330000in}}{\pgfqpoint{6.200000in}{2.310000in}}%
\pgfusepath{clip}%
\pgfsetrectcap%
\pgfsetroundjoin%
\pgfsetlinewidth{1.505625pt}%
\definecolor{currentstroke}{rgb}{0.090196,0.745098,0.811765}%
\pgfsetstrokecolor{currentstroke}%
\pgfsetdash{}{0pt}%
\pgfpathmoveto{\pgfqpoint{1.281818in}{1.888340in}}%
\pgfpathlineto{\pgfqpoint{2.221212in}{1.813804in}}%
\pgfpathlineto{\pgfqpoint{2.629644in}{1.780997in}}%
\pgfpathlineto{\pgfqpoint{2.901932in}{1.761431in}}%
\pgfpathlineto{\pgfqpoint{3.133377in}{1.747027in}}%
\pgfpathlineto{\pgfqpoint{3.351208in}{1.735752in}}%
\pgfpathlineto{\pgfqpoint{3.555424in}{1.727451in}}%
\pgfpathlineto{\pgfqpoint{3.746025in}{1.721847in}}%
\pgfpathlineto{\pgfqpoint{3.936627in}{1.718417in}}%
\pgfpathlineto{\pgfqpoint{4.127229in}{1.717222in}}%
\pgfpathlineto{\pgfqpoint{4.317830in}{1.718282in}}%
\pgfpathlineto{\pgfqpoint{4.508432in}{1.721574in}}%
\pgfpathlineto{\pgfqpoint{4.699034in}{1.727034in}}%
\pgfpathlineto{\pgfqpoint{4.903250in}{1.735170in}}%
\pgfpathlineto{\pgfqpoint{5.107466in}{1.745492in}}%
\pgfpathlineto{\pgfqpoint{5.325296in}{1.758645in}}%
\pgfpathlineto{\pgfqpoint{5.570356in}{1.775647in}}%
\pgfpathlineto{\pgfqpoint{5.869873in}{1.798722in}}%
\pgfpathlineto{\pgfqpoint{6.632279in}{1.858682in}}%
\pgfpathlineto{\pgfqpoint{6.918182in}{1.880281in}}%
\pgfpathlineto{\pgfqpoint{6.918182in}{1.880281in}}%
\pgfusepath{stroke}%
\end{pgfscope}%
\begin{pgfscope}%
\pgfpathrectangle{\pgfqpoint{1.000000in}{0.330000in}}{\pgfqpoint{6.200000in}{2.310000in}}%
\pgfusepath{clip}%
\pgfsetrectcap%
\pgfsetroundjoin%
\pgfsetlinewidth{1.505625pt}%
\definecolor{currentstroke}{rgb}{0.121569,0.466667,0.705882}%
\pgfsetstrokecolor{currentstroke}%
\pgfsetdash{}{0pt}%
\pgfpathmoveto{\pgfqpoint{1.281818in}{1.932582in}}%
\pgfpathlineto{\pgfqpoint{2.139526in}{1.812816in}}%
\pgfpathlineto{\pgfqpoint{2.520729in}{1.757855in}}%
\pgfpathlineto{\pgfqpoint{2.738560in}{1.728598in}}%
\pgfpathlineto{\pgfqpoint{2.929161in}{1.705129in}}%
\pgfpathlineto{\pgfqpoint{3.106148in}{1.685564in}}%
\pgfpathlineto{\pgfqpoint{3.269521in}{1.669706in}}%
\pgfpathlineto{\pgfqpoint{3.432894in}{1.656195in}}%
\pgfpathlineto{\pgfqpoint{3.582653in}{1.646029in}}%
\pgfpathlineto{\pgfqpoint{3.732411in}{1.638109in}}%
\pgfpathlineto{\pgfqpoint{3.882170in}{1.632524in}}%
\pgfpathlineto{\pgfqpoint{4.031928in}{1.629338in}}%
\pgfpathlineto{\pgfqpoint{4.168072in}{1.628556in}}%
\pgfpathlineto{\pgfqpoint{4.304216in}{1.629799in}}%
\pgfpathlineto{\pgfqpoint{4.453975in}{1.633494in}}%
\pgfpathlineto{\pgfqpoint{4.603733in}{1.639591in}}%
\pgfpathlineto{\pgfqpoint{4.753491in}{1.648029in}}%
\pgfpathlineto{\pgfqpoint{4.903250in}{1.658720in}}%
\pgfpathlineto{\pgfqpoint{5.053008in}{1.671549in}}%
\pgfpathlineto{\pgfqpoint{5.216381in}{1.687816in}}%
\pgfpathlineto{\pgfqpoint{5.379754in}{1.706240in}}%
\pgfpathlineto{\pgfqpoint{5.556741in}{1.728338in}}%
\pgfpathlineto{\pgfqpoint{5.760957in}{1.756116in}}%
\pgfpathlineto{\pgfqpoint{6.006017in}{1.791778in}}%
\pgfpathlineto{\pgfqpoint{6.673123in}{1.890258in}}%
\pgfpathlineto{\pgfqpoint{6.918182in}{1.925900in}}%
\pgfpathlineto{\pgfqpoint{6.918182in}{1.925900in}}%
\pgfusepath{stroke}%
\end{pgfscope}%
\begin{pgfscope}%
\pgfpathrectangle{\pgfqpoint{1.000000in}{0.330000in}}{\pgfqpoint{6.200000in}{2.310000in}}%
\pgfusepath{clip}%
\pgfsetrectcap%
\pgfsetroundjoin%
\pgfsetlinewidth{1.505625pt}%
\definecolor{currentstroke}{rgb}{1.000000,0.498039,0.054902}%
\pgfsetstrokecolor{currentstroke}%
\pgfsetdash{}{0pt}%
\pgfpathmoveto{\pgfqpoint{1.281818in}{1.920374in}}%
\pgfpathlineto{\pgfqpoint{2.153140in}{1.813194in}}%
\pgfpathlineto{\pgfqpoint{2.547958in}{1.763214in}}%
\pgfpathlineto{\pgfqpoint{2.779403in}{1.736185in}}%
\pgfpathlineto{\pgfqpoint{2.983619in}{1.714618in}}%
\pgfpathlineto{\pgfqpoint{3.160606in}{1.698073in}}%
\pgfpathlineto{\pgfqpoint{3.337593in}{1.683818in}}%
\pgfpathlineto{\pgfqpoint{3.500966in}{1.672897in}}%
\pgfpathlineto{\pgfqpoint{3.664339in}{1.664279in}}%
\pgfpathlineto{\pgfqpoint{3.814097in}{1.658502in}}%
\pgfpathlineto{\pgfqpoint{3.963856in}{1.654824in}}%
\pgfpathlineto{\pgfqpoint{4.113614in}{1.653286in}}%
\pgfpathlineto{\pgfqpoint{4.263373in}{1.653907in}}%
\pgfpathlineto{\pgfqpoint{4.413131in}{1.656681in}}%
\pgfpathlineto{\pgfqpoint{4.562890in}{1.661579in}}%
\pgfpathlineto{\pgfqpoint{4.712648in}{1.668548in}}%
\pgfpathlineto{\pgfqpoint{4.876021in}{1.678422in}}%
\pgfpathlineto{\pgfqpoint{5.039394in}{1.690539in}}%
\pgfpathlineto{\pgfqpoint{5.216381in}{1.706004in}}%
\pgfpathlineto{\pgfqpoint{5.393368in}{1.723654in}}%
\pgfpathlineto{\pgfqpoint{5.597585in}{1.746339in}}%
\pgfpathlineto{\pgfqpoint{5.829029in}{1.774401in}}%
\pgfpathlineto{\pgfqpoint{6.155775in}{1.816539in}}%
\pgfpathlineto{\pgfqpoint{6.441678in}{1.852893in}}%
\pgfpathlineto{\pgfqpoint{6.918182in}{1.912087in}}%
\pgfpathlineto{\pgfqpoint{6.918182in}{1.912087in}}%
\pgfusepath{stroke}%
\end{pgfscope}%
\begin{pgfscope}%
\pgfpathrectangle{\pgfqpoint{1.000000in}{0.330000in}}{\pgfqpoint{6.200000in}{2.310000in}}%
\pgfusepath{clip}%
\pgfsetrectcap%
\pgfsetroundjoin%
\pgfsetlinewidth{1.505625pt}%
\definecolor{currentstroke}{rgb}{0.172549,0.627451,0.172549}%
\pgfsetstrokecolor{currentstroke}%
\pgfsetdash{}{0pt}%
\pgfpathmoveto{\pgfqpoint{1.281818in}{1.930287in}}%
\pgfpathlineto{\pgfqpoint{2.139526in}{1.813203in}}%
\pgfpathlineto{\pgfqpoint{2.520729in}{1.759426in}}%
\pgfpathlineto{\pgfqpoint{2.752174in}{1.729104in}}%
\pgfpathlineto{\pgfqpoint{2.942776in}{1.706332in}}%
\pgfpathlineto{\pgfqpoint{3.119763in}{1.687411in}}%
\pgfpathlineto{\pgfqpoint{3.283136in}{1.672137in}}%
\pgfpathlineto{\pgfqpoint{3.446509in}{1.659190in}}%
\pgfpathlineto{\pgfqpoint{3.596267in}{1.649520in}}%
\pgfpathlineto{\pgfqpoint{3.746025in}{1.642066in}}%
\pgfpathlineto{\pgfqpoint{3.895784in}{1.636914in}}%
\pgfpathlineto{\pgfqpoint{4.045542in}{1.634122in}}%
\pgfpathlineto{\pgfqpoint{4.195301in}{1.633721in}}%
\pgfpathlineto{\pgfqpoint{4.345059in}{1.635717in}}%
\pgfpathlineto{\pgfqpoint{4.494818in}{1.640091in}}%
\pgfpathlineto{\pgfqpoint{4.644576in}{1.646794in}}%
\pgfpathlineto{\pgfqpoint{4.794335in}{1.655755in}}%
\pgfpathlineto{\pgfqpoint{4.944093in}{1.666874in}}%
\pgfpathlineto{\pgfqpoint{5.107466in}{1.681318in}}%
\pgfpathlineto{\pgfqpoint{5.270839in}{1.697982in}}%
\pgfpathlineto{\pgfqpoint{5.447826in}{1.718267in}}%
\pgfpathlineto{\pgfqpoint{5.638428in}{1.742316in}}%
\pgfpathlineto{\pgfqpoint{5.869873in}{1.773878in}}%
\pgfpathlineto{\pgfqpoint{6.210233in}{1.822882in}}%
\pgfpathlineto{\pgfqpoint{6.428063in}{1.853456in}}%
\pgfpathlineto{\pgfqpoint{6.918182in}{1.921083in}}%
\pgfpathlineto{\pgfqpoint{6.918182in}{1.921083in}}%
\pgfusepath{stroke}%
\end{pgfscope}%
\begin{pgfscope}%
\pgfpathrectangle{\pgfqpoint{1.000000in}{0.330000in}}{\pgfqpoint{6.200000in}{2.310000in}}%
\pgfusepath{clip}%
\pgfsetrectcap%
\pgfsetroundjoin%
\pgfsetlinewidth{1.505625pt}%
\definecolor{currentstroke}{rgb}{0.839216,0.152941,0.156863}%
\pgfsetstrokecolor{currentstroke}%
\pgfsetdash{}{0pt}%
\pgfpathmoveto{\pgfqpoint{1.281818in}{1.854730in}}%
\pgfpathlineto{\pgfqpoint{3.078920in}{1.797299in}}%
\pgfpathlineto{\pgfqpoint{3.419280in}{1.789988in}}%
\pgfpathlineto{\pgfqpoint{3.732411in}{1.785530in}}%
\pgfpathlineto{\pgfqpoint{4.031928in}{1.783516in}}%
\pgfpathlineto{\pgfqpoint{4.331445in}{1.783795in}}%
\pgfpathlineto{\pgfqpoint{4.630962in}{1.786358in}}%
\pgfpathlineto{\pgfqpoint{4.944093in}{1.791358in}}%
\pgfpathlineto{\pgfqpoint{5.270839in}{1.798842in}}%
\pgfpathlineto{\pgfqpoint{5.638428in}{1.809508in}}%
\pgfpathlineto{\pgfqpoint{6.155775in}{1.826985in}}%
\pgfpathlineto{\pgfqpoint{6.890953in}{1.851610in}}%
\pgfpathlineto{\pgfqpoint{6.918182in}{1.852515in}}%
\pgfpathlineto{\pgfqpoint{6.918182in}{1.852515in}}%
\pgfusepath{stroke}%
\end{pgfscope}%
\begin{pgfscope}%
\pgfpathrectangle{\pgfqpoint{1.000000in}{0.330000in}}{\pgfqpoint{6.200000in}{2.310000in}}%
\pgfusepath{clip}%
\pgfsetrectcap%
\pgfsetroundjoin%
\pgfsetlinewidth{1.505625pt}%
\definecolor{currentstroke}{rgb}{0.580392,0.403922,0.741176}%
\pgfsetstrokecolor{currentstroke}%
\pgfsetdash{}{0pt}%
\pgfpathmoveto{\pgfqpoint{1.281818in}{1.885517in}}%
\pgfpathlineto{\pgfqpoint{2.166755in}{1.818701in}}%
\pgfpathlineto{\pgfqpoint{2.670487in}{1.778917in}}%
\pgfpathlineto{\pgfqpoint{2.942776in}{1.759758in}}%
\pgfpathlineto{\pgfqpoint{3.174220in}{1.745655in}}%
\pgfpathlineto{\pgfqpoint{3.392051in}{1.734634in}}%
\pgfpathlineto{\pgfqpoint{3.596267in}{1.726552in}}%
\pgfpathlineto{\pgfqpoint{3.786869in}{1.721142in}}%
\pgfpathlineto{\pgfqpoint{3.977470in}{1.717903in}}%
\pgfpathlineto{\pgfqpoint{4.168072in}{1.716899in}}%
\pgfpathlineto{\pgfqpoint{4.358674in}{1.718155in}}%
\pgfpathlineto{\pgfqpoint{4.549275in}{1.721654in}}%
\pgfpathlineto{\pgfqpoint{4.739877in}{1.727338in}}%
\pgfpathlineto{\pgfqpoint{4.944093in}{1.735739in}}%
\pgfpathlineto{\pgfqpoint{5.148309in}{1.746356in}}%
\pgfpathlineto{\pgfqpoint{5.366140in}{1.759865in}}%
\pgfpathlineto{\pgfqpoint{5.611199in}{1.777329in}}%
\pgfpathlineto{\pgfqpoint{5.910716in}{1.801073in}}%
\pgfpathlineto{\pgfqpoint{6.918182in}{1.883188in}}%
\pgfpathlineto{\pgfqpoint{6.918182in}{1.883188in}}%
\pgfusepath{stroke}%
\end{pgfscope}%
\begin{pgfscope}%
\pgfpathrectangle{\pgfqpoint{1.000000in}{0.330000in}}{\pgfqpoint{6.200000in}{2.310000in}}%
\pgfusepath{clip}%
\pgfsetrectcap%
\pgfsetroundjoin%
\pgfsetlinewidth{1.505625pt}%
\definecolor{currentstroke}{rgb}{0.549020,0.337255,0.294118}%
\pgfsetstrokecolor{currentstroke}%
\pgfsetdash{}{0pt}%
\pgfpathmoveto{\pgfqpoint{1.281818in}{1.927212in}}%
\pgfpathlineto{\pgfqpoint{2.180369in}{1.808372in}}%
\pgfpathlineto{\pgfqpoint{2.520729in}{1.762884in}}%
\pgfpathlineto{\pgfqpoint{2.752174in}{1.734254in}}%
\pgfpathlineto{\pgfqpoint{2.942776in}{1.712819in}}%
\pgfpathlineto{\pgfqpoint{3.119763in}{1.695048in}}%
\pgfpathlineto{\pgfqpoint{3.283136in}{1.680728in}}%
\pgfpathlineto{\pgfqpoint{3.446509in}{1.668607in}}%
\pgfpathlineto{\pgfqpoint{3.609881in}{1.658848in}}%
\pgfpathlineto{\pgfqpoint{3.759640in}{1.652084in}}%
\pgfpathlineto{\pgfqpoint{3.909398in}{1.647485in}}%
\pgfpathlineto{\pgfqpoint{4.059157in}{1.645099in}}%
\pgfpathlineto{\pgfqpoint{4.208915in}{1.644953in}}%
\pgfpathlineto{\pgfqpoint{4.358674in}{1.647048in}}%
\pgfpathlineto{\pgfqpoint{4.508432in}{1.651364in}}%
\pgfpathlineto{\pgfqpoint{4.658191in}{1.657857in}}%
\pgfpathlineto{\pgfqpoint{4.807949in}{1.666457in}}%
\pgfpathlineto{\pgfqpoint{4.971322in}{1.678133in}}%
\pgfpathlineto{\pgfqpoint{5.134695in}{1.692056in}}%
\pgfpathlineto{\pgfqpoint{5.311682in}{1.709460in}}%
\pgfpathlineto{\pgfqpoint{5.502284in}{1.730586in}}%
\pgfpathlineto{\pgfqpoint{5.706500in}{1.755508in}}%
\pgfpathlineto{\pgfqpoint{5.951559in}{1.787728in}}%
\pgfpathlineto{\pgfqpoint{6.918182in}{1.917609in}}%
\pgfpathlineto{\pgfqpoint{6.918182in}{1.917609in}}%
\pgfusepath{stroke}%
\end{pgfscope}%
\begin{pgfscope}%
\pgfpathrectangle{\pgfqpoint{1.000000in}{0.330000in}}{\pgfqpoint{6.200000in}{2.310000in}}%
\pgfusepath{clip}%
\pgfsetrectcap%
\pgfsetroundjoin%
\pgfsetlinewidth{1.505625pt}%
\definecolor{currentstroke}{rgb}{0.890196,0.466667,0.760784}%
\pgfsetstrokecolor{currentstroke}%
\pgfsetdash{}{0pt}%
\pgfpathmoveto{\pgfqpoint{1.281818in}{1.925491in}}%
\pgfpathlineto{\pgfqpoint{2.180369in}{1.808759in}}%
\pgfpathlineto{\pgfqpoint{2.520729in}{1.764063in}}%
\pgfpathlineto{\pgfqpoint{2.752174in}{1.735946in}}%
\pgfpathlineto{\pgfqpoint{2.942776in}{1.714911in}}%
\pgfpathlineto{\pgfqpoint{3.119763in}{1.697488in}}%
\pgfpathlineto{\pgfqpoint{3.283136in}{1.683464in}}%
\pgfpathlineto{\pgfqpoint{3.446509in}{1.671614in}}%
\pgfpathlineto{\pgfqpoint{3.609881in}{1.662095in}}%
\pgfpathlineto{\pgfqpoint{3.759640in}{1.655522in}}%
\pgfpathlineto{\pgfqpoint{3.909398in}{1.651083in}}%
\pgfpathlineto{\pgfqpoint{4.059157in}{1.648824in}}%
\pgfpathlineto{\pgfqpoint{4.208915in}{1.648768in}}%
\pgfpathlineto{\pgfqpoint{4.358674in}{1.650915in}}%
\pgfpathlineto{\pgfqpoint{4.508432in}{1.655241in}}%
\pgfpathlineto{\pgfqpoint{4.658191in}{1.661700in}}%
\pgfpathlineto{\pgfqpoint{4.821563in}{1.671093in}}%
\pgfpathlineto{\pgfqpoint{4.984936in}{1.682813in}}%
\pgfpathlineto{\pgfqpoint{5.148309in}{1.696704in}}%
\pgfpathlineto{\pgfqpoint{5.325296in}{1.713978in}}%
\pgfpathlineto{\pgfqpoint{5.515898in}{1.734847in}}%
\pgfpathlineto{\pgfqpoint{5.733729in}{1.761047in}}%
\pgfpathlineto{\pgfqpoint{6.006017in}{1.796243in}}%
\pgfpathlineto{\pgfqpoint{6.536978in}{1.865533in}}%
\pgfpathlineto{\pgfqpoint{6.918182in}{1.913997in}}%
\pgfpathlineto{\pgfqpoint{6.918182in}{1.913997in}}%
\pgfusepath{stroke}%
\end{pgfscope}%
\begin{pgfscope}%
\pgfpathrectangle{\pgfqpoint{1.000000in}{0.330000in}}{\pgfqpoint{6.200000in}{2.310000in}}%
\pgfusepath{clip}%
\pgfsetrectcap%
\pgfsetroundjoin%
\pgfsetlinewidth{1.505625pt}%
\definecolor{currentstroke}{rgb}{0.498039,0.498039,0.498039}%
\pgfsetstrokecolor{currentstroke}%
\pgfsetdash{}{0pt}%
\pgfpathmoveto{\pgfqpoint{1.281818in}{1.881517in}}%
\pgfpathlineto{\pgfqpoint{2.166755in}{1.819455in}}%
\pgfpathlineto{\pgfqpoint{2.697716in}{1.780034in}}%
\pgfpathlineto{\pgfqpoint{2.970004in}{1.762143in}}%
\pgfpathlineto{\pgfqpoint{3.201449in}{1.749058in}}%
\pgfpathlineto{\pgfqpoint{3.419280in}{1.738926in}}%
\pgfpathlineto{\pgfqpoint{3.623496in}{1.731604in}}%
\pgfpathlineto{\pgfqpoint{3.827712in}{1.726563in}}%
\pgfpathlineto{\pgfqpoint{4.018314in}{1.724021in}}%
\pgfpathlineto{\pgfqpoint{4.208915in}{1.723619in}}%
\pgfpathlineto{\pgfqpoint{4.399517in}{1.725369in}}%
\pgfpathlineto{\pgfqpoint{4.590119in}{1.729240in}}%
\pgfpathlineto{\pgfqpoint{4.794335in}{1.735658in}}%
\pgfpathlineto{\pgfqpoint{4.998551in}{1.744285in}}%
\pgfpathlineto{\pgfqpoint{5.216381in}{1.755700in}}%
\pgfpathlineto{\pgfqpoint{5.461440in}{1.770895in}}%
\pgfpathlineto{\pgfqpoint{5.747343in}{1.791062in}}%
\pgfpathlineto{\pgfqpoint{6.183004in}{1.824478in}}%
\pgfpathlineto{\pgfqpoint{6.468906in}{1.845553in}}%
\pgfpathlineto{\pgfqpoint{6.918182in}{1.877734in}}%
\pgfpathlineto{\pgfqpoint{6.918182in}{1.877734in}}%
\pgfusepath{stroke}%
\end{pgfscope}%
\begin{pgfscope}%
\pgfpathrectangle{\pgfqpoint{1.000000in}{0.330000in}}{\pgfqpoint{6.200000in}{2.310000in}}%
\pgfusepath{clip}%
\pgfsetrectcap%
\pgfsetroundjoin%
\pgfsetlinewidth{1.505625pt}%
\definecolor{currentstroke}{rgb}{0.737255,0.741176,0.133333}%
\pgfsetstrokecolor{currentstroke}%
\pgfsetdash{}{0pt}%
\pgfpathmoveto{\pgfqpoint{1.281818in}{1.922095in}}%
\pgfpathlineto{\pgfqpoint{2.153140in}{1.812871in}}%
\pgfpathlineto{\pgfqpoint{2.547958in}{1.761974in}}%
\pgfpathlineto{\pgfqpoint{2.779403in}{1.734435in}}%
\pgfpathlineto{\pgfqpoint{2.983619in}{1.712444in}}%
\pgfpathlineto{\pgfqpoint{3.160606in}{1.695557in}}%
\pgfpathlineto{\pgfqpoint{3.337593in}{1.680988in}}%
\pgfpathlineto{\pgfqpoint{3.500966in}{1.669806in}}%
\pgfpathlineto{\pgfqpoint{3.664339in}{1.660958in}}%
\pgfpathlineto{\pgfqpoint{3.814097in}{1.655002in}}%
\pgfpathlineto{\pgfqpoint{3.963856in}{1.651175in}}%
\pgfpathlineto{\pgfqpoint{4.113614in}{1.649524in}}%
\pgfpathlineto{\pgfqpoint{4.263373in}{1.650068in}}%
\pgfpathlineto{\pgfqpoint{4.413131in}{1.652805in}}%
\pgfpathlineto{\pgfqpoint{4.562890in}{1.657709in}}%
\pgfpathlineto{\pgfqpoint{4.712648in}{1.664728in}}%
\pgfpathlineto{\pgfqpoint{4.876021in}{1.674712in}}%
\pgfpathlineto{\pgfqpoint{5.039394in}{1.686999in}}%
\pgfpathlineto{\pgfqpoint{5.202767in}{1.701427in}}%
\pgfpathlineto{\pgfqpoint{5.379754in}{1.719247in}}%
\pgfpathlineto{\pgfqpoint{5.570356in}{1.740658in}}%
\pgfpathlineto{\pgfqpoint{5.788186in}{1.767409in}}%
\pgfpathlineto{\pgfqpoint{6.074089in}{1.805015in}}%
\pgfpathlineto{\pgfqpoint{6.523364in}{1.864452in}}%
\pgfpathlineto{\pgfqpoint{6.918182in}{1.915699in}}%
\pgfpathlineto{\pgfqpoint{6.918182in}{1.915699in}}%
\pgfusepath{stroke}%
\end{pgfscope}%
\begin{pgfscope}%
\pgfpathrectangle{\pgfqpoint{1.000000in}{0.330000in}}{\pgfqpoint{6.200000in}{2.310000in}}%
\pgfusepath{clip}%
\pgfsetrectcap%
\pgfsetroundjoin%
\pgfsetlinewidth{1.505625pt}%
\definecolor{currentstroke}{rgb}{0.090196,0.745098,0.811765}%
\pgfsetstrokecolor{currentstroke}%
\pgfsetdash{}{0pt}%
\pgfpathmoveto{\pgfqpoint{1.281818in}{1.883811in}}%
\pgfpathlineto{\pgfqpoint{2.166755in}{1.818981in}}%
\pgfpathlineto{\pgfqpoint{2.697716in}{1.777937in}}%
\pgfpathlineto{\pgfqpoint{2.970004in}{1.759280in}}%
\pgfpathlineto{\pgfqpoint{3.201449in}{1.745603in}}%
\pgfpathlineto{\pgfqpoint{3.419280in}{1.734974in}}%
\pgfpathlineto{\pgfqpoint{3.623496in}{1.727250in}}%
\pgfpathlineto{\pgfqpoint{3.814097in}{1.722159in}}%
\pgfpathlineto{\pgfqpoint{4.004699in}{1.719221in}}%
\pgfpathlineto{\pgfqpoint{4.195301in}{1.718498in}}%
\pgfpathlineto{\pgfqpoint{4.385903in}{1.720011in}}%
\pgfpathlineto{\pgfqpoint{4.576504in}{1.723738in}}%
\pgfpathlineto{\pgfqpoint{4.767106in}{1.729619in}}%
\pgfpathlineto{\pgfqpoint{4.971322in}{1.738189in}}%
\pgfpathlineto{\pgfqpoint{5.175538in}{1.748932in}}%
\pgfpathlineto{\pgfqpoint{5.406983in}{1.763435in}}%
\pgfpathlineto{\pgfqpoint{5.665657in}{1.782060in}}%
\pgfpathlineto{\pgfqpoint{5.978788in}{1.806995in}}%
\pgfpathlineto{\pgfqpoint{6.686737in}{1.864241in}}%
\pgfpathlineto{\pgfqpoint{6.918182in}{1.882551in}}%
\pgfpathlineto{\pgfqpoint{6.918182in}{1.882551in}}%
\pgfusepath{stroke}%
\end{pgfscope}%
\begin{pgfscope}%
\pgfpathrectangle{\pgfqpoint{1.000000in}{0.330000in}}{\pgfqpoint{6.200000in}{2.310000in}}%
\pgfusepath{clip}%
\pgfsetrectcap%
\pgfsetroundjoin%
\pgfsetlinewidth{1.505625pt}%
\definecolor{currentstroke}{rgb}{0.121569,0.466667,0.705882}%
\pgfsetstrokecolor{currentstroke}%
\pgfsetdash{}{0pt}%
\pgfpathmoveto{\pgfqpoint{1.281818in}{1.939405in}}%
\pgfpathlineto{\pgfqpoint{2.180369in}{1.805601in}}%
\pgfpathlineto{\pgfqpoint{2.493500in}{1.758503in}}%
\pgfpathlineto{\pgfqpoint{2.711331in}{1.727914in}}%
\pgfpathlineto{\pgfqpoint{2.901932in}{1.703381in}}%
\pgfpathlineto{\pgfqpoint{3.078920in}{1.682914in}}%
\pgfpathlineto{\pgfqpoint{3.242292in}{1.666298in}}%
\pgfpathlineto{\pgfqpoint{3.392051in}{1.653185in}}%
\pgfpathlineto{\pgfqpoint{3.541809in}{1.642248in}}%
\pgfpathlineto{\pgfqpoint{3.691568in}{1.633607in}}%
\pgfpathlineto{\pgfqpoint{3.841326in}{1.627355in}}%
\pgfpathlineto{\pgfqpoint{3.977470in}{1.623801in}}%
\pgfpathlineto{\pgfqpoint{4.113614in}{1.622309in}}%
\pgfpathlineto{\pgfqpoint{4.249758in}{1.622891in}}%
\pgfpathlineto{\pgfqpoint{4.385903in}{1.625544in}}%
\pgfpathlineto{\pgfqpoint{4.522047in}{1.630246in}}%
\pgfpathlineto{\pgfqpoint{4.671805in}{1.637734in}}%
\pgfpathlineto{\pgfqpoint{4.821563in}{1.647574in}}%
\pgfpathlineto{\pgfqpoint{4.971322in}{1.659659in}}%
\pgfpathlineto{\pgfqpoint{5.121080in}{1.673860in}}%
\pgfpathlineto{\pgfqpoint{5.284453in}{1.691579in}}%
\pgfpathlineto{\pgfqpoint{5.461440in}{1.713124in}}%
\pgfpathlineto{\pgfqpoint{5.652042in}{1.738662in}}%
\pgfpathlineto{\pgfqpoint{5.869873in}{1.770172in}}%
\pgfpathlineto{\pgfqpoint{6.169390in}{1.815983in}}%
\pgfpathlineto{\pgfqpoint{6.455292in}{1.859331in}}%
\pgfpathlineto{\pgfqpoint{6.918182in}{1.928447in}}%
\pgfpathlineto{\pgfqpoint{6.918182in}{1.928447in}}%
\pgfusepath{stroke}%
\end{pgfscope}%
\begin{pgfscope}%
\pgfsetroundcap%
\pgfsetroundjoin%
\pgfsetlinewidth{1.003750pt}%
\definecolor{currentstroke}{rgb}{0.000000,0.000000,0.000000}%
\pgfsetstrokecolor{currentstroke}%
\pgfsetdash{}{0pt}%
\pgfpathmoveto{\pgfqpoint{3.673877in}{0.435000in}}%
\pgfpathquadraticcurveto{\pgfqpoint{3.893466in}{0.435000in}}{\pgfqpoint{4.113054in}{0.435000in}}%
\pgfusepath{stroke}%
\end{pgfscope}%
\begin{pgfscope}%
\pgfsetbuttcap%
\pgfsetmiterjoin%
\definecolor{currentfill}{rgb}{0.800000,0.800000,0.800000}%
\pgfsetfillcolor{currentfill}%
\pgfsetlinewidth{1.003750pt}%
\definecolor{currentstroke}{rgb}{0.000000,0.000000,0.000000}%
\pgfsetstrokecolor{currentstroke}%
\pgfsetdash{}{0pt}%
\pgfpathmoveto{\pgfqpoint{2.717232in}{0.338549in}}%
\pgfpathcurveto{\pgfqpoint{2.751954in}{0.303827in}}{\pgfqpoint{3.581430in}{0.303827in}}{\pgfqpoint{3.616152in}{0.338549in}}%
\pgfpathcurveto{\pgfqpoint{3.650875in}{0.373272in}}{\pgfqpoint{3.650875in}{0.496728in}}{\pgfqpoint{3.616152in}{0.531451in}}%
\pgfpathcurveto{\pgfqpoint{3.581430in}{0.566173in}}{\pgfqpoint{2.751954in}{0.566173in}}{\pgfqpoint{2.717232in}{0.531451in}}%
\pgfpathcurveto{\pgfqpoint{2.682510in}{0.496728in}}{\pgfqpoint{2.682510in}{0.373272in}}{\pgfqpoint{2.717232in}{0.338549in}}%
\pgfpathclose%
\pgfusepath{stroke,fill}%
\end{pgfscope}%
\begin{pgfscope}%
\definecolor{textcolor}{rgb}{0.000000,0.000000,0.000000}%
\pgfsetstrokecolor{textcolor}%
\pgfsetfillcolor{textcolor}%
\pgftext[x=2.751954in,y=0.435000in,left,]{\color{textcolor}\rmfamily\fontsize{10.000000}{12.000000}\selectfont \(\displaystyle \Delta =\) -1.7 inch}%
\end{pgfscope}%
\begin{pgfscope}%
\pgfsetbuttcap%
\pgfsetmiterjoin%
\definecolor{currentfill}{rgb}{0.800000,0.800000,0.800000}%
\pgfsetfillcolor{currentfill}%
\pgfsetlinewidth{1.003750pt}%
\definecolor{currentstroke}{rgb}{0.000000,0.000000,0.000000}%
\pgfsetstrokecolor{currentstroke}%
\pgfsetdash{}{0pt}%
\pgfpathmoveto{\pgfqpoint{0.965278in}{0.358599in}}%
\pgfpathcurveto{\pgfqpoint{1.000000in}{0.323877in}}{\pgfqpoint{2.720682in}{0.323877in}}{\pgfqpoint{2.755404in}{0.358599in}}%
\pgfpathcurveto{\pgfqpoint{2.790127in}{0.393321in}}{\pgfqpoint{2.790127in}{0.668784in}}{\pgfqpoint{2.755404in}{0.703506in}}%
\pgfpathcurveto{\pgfqpoint{2.720682in}{0.738228in}}{\pgfqpoint{1.000000in}{0.738228in}}{\pgfqpoint{0.965278in}{0.703506in}}%
\pgfpathcurveto{\pgfqpoint{0.930556in}{0.668784in}}{\pgfqpoint{0.930556in}{0.393321in}}{\pgfqpoint{0.965278in}{0.358599in}}%
\pgfpathclose%
\pgfusepath{stroke,fill}%
\end{pgfscope}%
\begin{pgfscope}%
\definecolor{textcolor}{rgb}{0.000000,0.000000,0.000000}%
\pgfsetstrokecolor{textcolor}%
\pgfsetfillcolor{textcolor}%
\pgftext[x=1.000000in, y=0.580049in, left, base]{\color{textcolor}\rmfamily\fontsize{10.000000}{12.000000}\selectfont Max combo: 1.0D + 1.0S}%
\end{pgfscope}%
\begin{pgfscope}%
\definecolor{textcolor}{rgb}{0.000000,0.000000,0.000000}%
\pgfsetstrokecolor{textcolor}%
\pgfsetfillcolor{textcolor}%
\pgftext[x=1.000000in, y=0.428043in, left, base]{\color{textcolor}\rmfamily\fontsize{10.000000}{12.000000}\selectfont ASCE7-16 Sec. 2.4.1 (LC 3)}%
\end{pgfscope}%
\end{pgfpicture}%
\makeatother%
\endgroup%

\end{center}
\caption{Deflection Envelope}
\end{figure}
Tl Deflection Check: 
$\Delta_{max} = -1.73 {\color{darkBlue}{\mathbf{ \; in}}} = \cfrac{L}{180} < \cfrac{L}{1.0}  \;  \mathbf{(OK)}$\\
\bigbreak
\vspace{-30pt}
%	---------------------------------- REACTIONS ---------------------------------
\section{Reactions}
The following is a summary of service-level reactions at each support:
\begin{table}[ht]
\caption{Reactions at Supports}
\centering
\begin{tabular}{l l l l l l l }
\hline
Loc. & Type & D & S & Lr0 & Lr1 & Lr2\\
\hline
4.5 {\color{darkBlue}{\textbf{ft}}} & Shear & 5.8 {\color{darkBlue}{\textbf{kip}}} & 62.2 {\color{darkBlue}{\textbf{kip}}} & 1.1 {\color{darkBlue}{\textbf{kip}}} & 2.8 {\color{darkBlue}{\textbf{kip}}} & -0.1 {\color{darkBlue}{\textbf{kip}}}\\ 
30.5 {\color{darkBlue}{\textbf{ft}}} & Shear & 5.5 {\color{darkBlue}{\textbf{kip}}} & 59.7 {\color{darkBlue}{\textbf{kip}}} & -0.1 {\color{darkBlue}{\textbf{kip}}} & 2.8 {\color{darkBlue}{\textbf{kip}}} & 0.9 {\color{darkBlue}{\textbf{kip}}}\\ 
\hline
\end{tabular}
\end{table}
\end{document}