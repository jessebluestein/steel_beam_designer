\documentclass[12pt, fleqn]{article}
\usepackage{pgfplots}
\usepackage{bm}
\usepackage{marginnote}
\usepackage{wallpaper}
\usepackage{lastpage}
\usepackage[left=1.3cm,right=2.0cm,top=1.8cm,bottom=5.0cm,marginparwidth=3.4cm]{geometry}
\usepackage{amsmath}
\usepackage{amssymb}
\usepackage{xcolor}
\usepackage{enumitem}
\usepackage{float}
\usepackage{textgreek}
\usepackage{textcomp}
\usepackage{fancyhdr}
\usepackage{graphicx}
\usepackage{pstricks}
\usepackage{subfigure}
\usepackage{caption}
\captionsetup{justification=centering,labelfont=bf, belowskip=12pt,aboveskip=12pt}
\usepackage{textcomp}
\setlength{\headheight}{70pt}
\setlength{\textfloatsep}{12pt}
\setlength{\intextsep}{0pt}
\pagestyle{fancy}\fancyhf{}
\renewcommand{\headrulewidth}{0pt}
\definecolor{darkBlue}{cmyk}{.80, .32, 0, 0}
\setlength{\parindent}{0cm}
\newcommand{\tab}{\hspace*{2em}}
\newcommand\BackgroundStructure{
\setlength{\unitlength}{1mm}
\setlength\fboxsep{0mm}
\setlength\fboxrule{0.5mm}
\put(10, 20pr){\fcolorbox{black}{gray!5}{\framebox(155,247){}}}
\put(165, 20){\fcolorbox{black}{gray!10}{\framebox(37,247){}}}
\put(10, 262){\fcolorbox{black}{white!10}{\framebox(192, 25){}}}
\put(175, 263){\includegraphics{}}}
\setlength{\abovedisplayskip}{0pt}
\setlength{\belowdisplayskip}{0pt}
%	----------------------------------- HEADER -----------------------------------
\fancyhead[L]{\begin{tabular}{l l | l l}
\textbf{Member:} & {Test3} & \textbf{Firm:} & {ABC Company} \\
\textbf{Project:} & {123 Maple Street, San Francisco CA} & \textbf{Engineer:} & {Jesse} \\
\textbf{Level:} & {2} & \textbf{Checker:} & {Joey}  \\
\textbf{Date:} & {2021-05-23} & \textbf{Page:} & \thepage\\
\end{tabular}}
%	---------------------------- APPLIED LOADS SECTION ---------------------------
\begin{document}
\begin{center}
\textbf{\LARGE W10x12 Design Report}
\end{center}
\section{Applied Loading}
\vspace{-30pt}
\begin{figure}[H]
\begin{center}
%% Creator: Matplotlib, PGF backend
%%
%% To include the figure in your LaTeX document, write
%%   \input{<filename>.pgf}
%%
%% Make sure the required packages are loaded in your preamble
%%   \usepackage{pgf}
%%
%% Figures using additional raster images can only be included by \input if
%% they are in the same directory as the main LaTeX file. For loading figures
%% from other directories you can use the `import` package
%%   \usepackage{import}
%%
%% and then include the figures with
%%   \import{<path to file>}{<filename>.pgf}
%%
%% Matplotlib used the following preamble
%%
\begingroup%
\makeatletter%
\begin{pgfpicture}%
\pgfpathrectangle{\pgfpointorigin}{\pgfqpoint{8.000000in}{3.000000in}}%
\pgfusepath{use as bounding box, clip}%
\begin{pgfscope}%
\pgfpathrectangle{\pgfqpoint{1.000000in}{0.825864in}}{\pgfqpoint{6.200000in}{1.814136in}}%
\pgfusepath{clip}%
\pgfsetbuttcap%
\pgfsetmiterjoin%
\definecolor{currentfill}{rgb}{0.121569,0.466667,0.705882}%
\pgfsetfillcolor{currentfill}%
\pgfsetfillopacity{0.400000}%
\pgfsetlinewidth{1.003750pt}%
\definecolor{currentstroke}{rgb}{0.121569,0.466667,0.705882}%
\pgfsetstrokecolor{currentstroke}%
\pgfsetstrokeopacity{0.400000}%
\pgfsetdash{}{0pt}%
\pgfpathmoveto{\pgfqpoint{1.281818in}{0.908325in}}%
\pgfpathlineto{\pgfqpoint{1.281818in}{1.647628in}}%
\pgfpathlineto{\pgfqpoint{6.918182in}{1.647628in}}%
\pgfpathlineto{\pgfqpoint{6.918182in}{0.908325in}}%
\pgfpathclose%
\pgfusepath{stroke,fill}%
\end{pgfscope}%
\begin{pgfscope}%
\pgfpathrectangle{\pgfqpoint{1.000000in}{0.825864in}}{\pgfqpoint{6.200000in}{1.814136in}}%
\pgfusepath{clip}%
\pgfsetbuttcap%
\pgfsetmiterjoin%
\definecolor{currentfill}{rgb}{1.000000,0.498039,0.054902}%
\pgfsetfillcolor{currentfill}%
\pgfsetfillopacity{0.400000}%
\pgfsetlinewidth{1.003750pt}%
\definecolor{currentstroke}{rgb}{1.000000,0.498039,0.054902}%
\pgfsetstrokecolor{currentstroke}%
\pgfsetstrokeopacity{0.400000}%
\pgfsetdash{}{0pt}%
\pgfpathmoveto{\pgfqpoint{1.281818in}{1.647628in}}%
\pgfpathlineto{\pgfqpoint{1.281818in}{2.216322in}}%
\pgfpathlineto{\pgfqpoint{6.918182in}{2.216322in}}%
\pgfpathlineto{\pgfqpoint{6.918182in}{1.647628in}}%
\pgfpathclose%
\pgfusepath{stroke,fill}%
\end{pgfscope}%
\begin{pgfscope}%
\pgfpathrectangle{\pgfqpoint{1.000000in}{0.825864in}}{\pgfqpoint{6.200000in}{1.814136in}}%
\pgfusepath{clip}%
\pgfsetbuttcap%
\pgfsetmiterjoin%
\definecolor{currentfill}{rgb}{0.172549,0.627451,0.172549}%
\pgfsetfillcolor{currentfill}%
\pgfsetfillopacity{0.400000}%
\pgfsetlinewidth{1.003750pt}%
\definecolor{currentstroke}{rgb}{0.172549,0.627451,0.172549}%
\pgfsetstrokecolor{currentstroke}%
\pgfsetstrokeopacity{0.400000}%
\pgfsetdash{}{0pt}%
\pgfpathmoveto{\pgfqpoint{1.281818in}{2.216322in}}%
\pgfpathlineto{\pgfqpoint{1.281818in}{2.557539in}}%
\pgfpathlineto{\pgfqpoint{6.918182in}{2.557539in}}%
\pgfpathlineto{\pgfqpoint{6.918182in}{2.216322in}}%
\pgfpathclose%
\pgfusepath{stroke,fill}%
\end{pgfscope}%
\begin{pgfscope}%
\pgfsetbuttcap%
\pgfsetmiterjoin%
\definecolor{currentfill}{rgb}{0.800000,0.800000,0.800000}%
\pgfsetfillcolor{currentfill}%
\pgfsetlinewidth{1.003750pt}%
\definecolor{currentstroke}{rgb}{0.000000,0.000000,0.000000}%
\pgfsetstrokecolor{currentstroke}%
\pgfsetdash{}{0pt}%
\pgfpathmoveto{\pgfqpoint{3.983970in}{1.181526in}}%
\pgfpathcurveto{\pgfqpoint{4.018692in}{1.146803in}}{\pgfqpoint{4.181308in}{1.146803in}}{\pgfqpoint{4.216030in}{1.181526in}}%
\pgfpathcurveto{\pgfqpoint{4.250752in}{1.216248in}}{\pgfqpoint{4.250752in}{1.339705in}}{\pgfqpoint{4.216030in}{1.374427in}}%
\pgfpathcurveto{\pgfqpoint{4.181308in}{1.409149in}}{\pgfqpoint{4.018692in}{1.409149in}}{\pgfqpoint{3.983970in}{1.374427in}}%
\pgfpathcurveto{\pgfqpoint{3.949247in}{1.339705in}}{\pgfqpoint{3.949247in}{1.216248in}}{\pgfqpoint{3.983970in}{1.181526in}}%
\pgfpathclose%
\pgfusepath{stroke,fill}%
\end{pgfscope}%
\begin{pgfscope}%
\definecolor{textcolor}{rgb}{0.000000,0.000000,0.000000}%
\pgfsetstrokecolor{textcolor}%
\pgfsetfillcolor{textcolor}%
\pgftext[x=4.100000in,y=1.277976in,,]{\color{textcolor}\rmfamily\fontsize{10.000000}{12.000000}\selectfont w\textsubscript{1}}%
\end{pgfscope}%
\begin{pgfscope}%
\pgfsetbuttcap%
\pgfsetmiterjoin%
\definecolor{currentfill}{rgb}{0.800000,0.800000,0.800000}%
\pgfsetfillcolor{currentfill}%
\pgfsetlinewidth{1.003750pt}%
\definecolor{currentstroke}{rgb}{0.000000,0.000000,0.000000}%
\pgfsetstrokecolor{currentstroke}%
\pgfsetdash{}{0pt}%
\pgfpathmoveto{\pgfqpoint{3.983970in}{1.835525in}}%
\pgfpathcurveto{\pgfqpoint{4.018692in}{1.800802in}}{\pgfqpoint{4.181308in}{1.800802in}}{\pgfqpoint{4.216030in}{1.835525in}}%
\pgfpathcurveto{\pgfqpoint{4.250752in}{1.870247in}}{\pgfqpoint{4.250752in}{1.993703in}}{\pgfqpoint{4.216030in}{2.028426in}}%
\pgfpathcurveto{\pgfqpoint{4.181308in}{2.063148in}}{\pgfqpoint{4.018692in}{2.063148in}}{\pgfqpoint{3.983970in}{2.028426in}}%
\pgfpathcurveto{\pgfqpoint{3.949247in}{1.993703in}}{\pgfqpoint{3.949247in}{1.870247in}}{\pgfqpoint{3.983970in}{1.835525in}}%
\pgfpathclose%
\pgfusepath{stroke,fill}%
\end{pgfscope}%
\begin{pgfscope}%
\definecolor{textcolor}{rgb}{0.000000,0.000000,0.000000}%
\pgfsetstrokecolor{textcolor}%
\pgfsetfillcolor{textcolor}%
\pgftext[x=4.100000in,y=1.931975in,,]{\color{textcolor}\rmfamily\fontsize{10.000000}{12.000000}\selectfont w\textsubscript{2}}%
\end{pgfscope}%
\begin{pgfscope}%
\pgfsetbuttcap%
\pgfsetmiterjoin%
\definecolor{currentfill}{rgb}{0.800000,0.800000,0.800000}%
\pgfsetfillcolor{currentfill}%
\pgfsetlinewidth{1.003750pt}%
\definecolor{currentstroke}{rgb}{0.000000,0.000000,0.000000}%
\pgfsetstrokecolor{currentstroke}%
\pgfsetdash{}{0pt}%
\pgfpathmoveto{\pgfqpoint{3.983970in}{2.290480in}}%
\pgfpathcurveto{\pgfqpoint{4.018692in}{2.255758in}}{\pgfqpoint{4.181308in}{2.255758in}}{\pgfqpoint{4.216030in}{2.290480in}}%
\pgfpathcurveto{\pgfqpoint{4.250752in}{2.325203in}}{\pgfqpoint{4.250752in}{2.448659in}}{\pgfqpoint{4.216030in}{2.483381in}}%
\pgfpathcurveto{\pgfqpoint{4.181308in}{2.518104in}}{\pgfqpoint{4.018692in}{2.518104in}}{\pgfqpoint{3.983970in}{2.483381in}}%
\pgfpathcurveto{\pgfqpoint{3.949247in}{2.448659in}}{\pgfqpoint{3.949247in}{2.325203in}}{\pgfqpoint{3.983970in}{2.290480in}}%
\pgfpathclose%
\pgfusepath{stroke,fill}%
\end{pgfscope}%
\begin{pgfscope}%
\definecolor{textcolor}{rgb}{0.000000,0.000000,0.000000}%
\pgfsetstrokecolor{textcolor}%
\pgfsetfillcolor{textcolor}%
\pgftext[x=4.100000in,y=2.386931in,,]{\color{textcolor}\rmfamily\fontsize{10.000000}{12.000000}\selectfont w\textsubscript{3}}%
\end{pgfscope}%
\begin{pgfscope}%
\pgfsetroundcap%
\pgfsetroundjoin%
\pgfsetlinewidth{1.003750pt}%
\definecolor{currentstroke}{rgb}{0.000000,0.000000,0.000000}%
\pgfsetstrokecolor{currentstroke}%
\pgfsetdash{}{0pt}%
\pgfpathmoveto{\pgfqpoint{1.281818in}{2.158833in}}%
\pgfpathquadraticcurveto{\pgfqpoint{1.281818in}{1.547451in}}{\pgfqpoint{1.281818in}{0.944962in}}%
\pgfusepath{stroke}%
\end{pgfscope}%
\begin{pgfscope}%
\pgfsetroundcap%
\pgfsetroundjoin%
\pgfsetlinewidth{1.003750pt}%
\definecolor{currentstroke}{rgb}{0.000000,0.000000,0.000000}%
\pgfsetstrokecolor{currentstroke}%
\pgfsetdash{}{0pt}%
\pgfpathmoveto{\pgfqpoint{1.365152in}{1.011629in}}%
\pgfpathlineto{\pgfqpoint{1.281818in}{0.944962in}}%
\pgfpathlineto{\pgfqpoint{1.198485in}{1.011629in}}%
\pgfusepath{stroke}%
\end{pgfscope}%
\begin{pgfscope}%
\pgfsetbuttcap%
\pgfsetmiterjoin%
\definecolor{currentfill}{rgb}{0.800000,0.800000,0.800000}%
\pgfsetfillcolor{currentfill}%
\pgfsetlinewidth{1.003750pt}%
\definecolor{currentstroke}{rgb}{0.000000,0.000000,0.000000}%
\pgfsetstrokecolor{currentstroke}%
\pgfsetdash{}{0pt}%
\pgfpathmoveto{\pgfqpoint{1.150857in}{2.223140in}}%
\pgfpathcurveto{\pgfqpoint{1.192524in}{2.181473in}}{\pgfqpoint{1.371112in}{2.181473in}}{\pgfqpoint{1.412779in}{2.223140in}}%
\pgfpathcurveto{\pgfqpoint{1.454446in}{2.264806in}}{\pgfqpoint{1.454446in}{2.412954in}}{\pgfqpoint{1.412779in}{2.454621in}}%
\pgfpathcurveto{\pgfqpoint{1.371112in}{2.496287in}}{\pgfqpoint{1.192524in}{2.496287in}}{\pgfqpoint{1.150857in}{2.454621in}}%
\pgfpathcurveto{\pgfqpoint{1.109191in}{2.412954in}}{\pgfqpoint{1.109191in}{2.264806in}}{\pgfqpoint{1.150857in}{2.223140in}}%
\pgfpathclose%
\pgfusepath{stroke,fill}%
\end{pgfscope}%
\begin{pgfscope}%
\definecolor{textcolor}{rgb}{0.000000,0.000000,0.000000}%
\pgfsetstrokecolor{textcolor}%
\pgfsetfillcolor{textcolor}%
\pgftext[x=1.281818in,y=2.297213in,,base]{\color{textcolor}\rmfamily\fontsize{12.000000}{14.400000}\selectfont \(\displaystyle P_x\)}%
\end{pgfscope}%
\begin{pgfscope}%
\pgfsetroundcap%
\pgfsetroundjoin%
\pgfsetlinewidth{1.003750pt}%
\definecolor{currentstroke}{rgb}{0.000000,0.000000,0.000000}%
\pgfsetstrokecolor{currentstroke}%
\pgfsetdash{}{0pt}%
\pgfpathmoveto{\pgfqpoint{6.918182in}{2.158833in}}%
\pgfpathquadraticcurveto{\pgfqpoint{6.918182in}{1.547451in}}{\pgfqpoint{6.918182in}{0.944962in}}%
\pgfusepath{stroke}%
\end{pgfscope}%
\begin{pgfscope}%
\pgfsetroundcap%
\pgfsetroundjoin%
\pgfsetlinewidth{1.003750pt}%
\definecolor{currentstroke}{rgb}{0.000000,0.000000,0.000000}%
\pgfsetstrokecolor{currentstroke}%
\pgfsetdash{}{0pt}%
\pgfpathmoveto{\pgfqpoint{7.001515in}{1.011629in}}%
\pgfpathlineto{\pgfqpoint{6.918182in}{0.944962in}}%
\pgfpathlineto{\pgfqpoint{6.834848in}{1.011629in}}%
\pgfusepath{stroke}%
\end{pgfscope}%
\begin{pgfscope}%
\pgfsetbuttcap%
\pgfsetmiterjoin%
\definecolor{currentfill}{rgb}{0.800000,0.800000,0.800000}%
\pgfsetfillcolor{currentfill}%
\pgfsetlinewidth{1.003750pt}%
\definecolor{currentstroke}{rgb}{0.000000,0.000000,0.000000}%
\pgfsetstrokecolor{currentstroke}%
\pgfsetdash{}{0pt}%
\pgfpathmoveto{\pgfqpoint{6.787221in}{2.223140in}}%
\pgfpathcurveto{\pgfqpoint{6.828888in}{2.181473in}}{\pgfqpoint{7.007476in}{2.181473in}}{\pgfqpoint{7.049143in}{2.223140in}}%
\pgfpathcurveto{\pgfqpoint{7.090809in}{2.264806in}}{\pgfqpoint{7.090809in}{2.412954in}}{\pgfqpoint{7.049143in}{2.454621in}}%
\pgfpathcurveto{\pgfqpoint{7.007476in}{2.496287in}}{\pgfqpoint{6.828888in}{2.496287in}}{\pgfqpoint{6.787221in}{2.454621in}}%
\pgfpathcurveto{\pgfqpoint{6.745554in}{2.412954in}}{\pgfqpoint{6.745554in}{2.264806in}}{\pgfqpoint{6.787221in}{2.223140in}}%
\pgfpathclose%
\pgfusepath{stroke,fill}%
\end{pgfscope}%
\begin{pgfscope}%
\definecolor{textcolor}{rgb}{0.000000,0.000000,0.000000}%
\pgfsetstrokecolor{textcolor}%
\pgfsetfillcolor{textcolor}%
\pgftext[x=6.918182in,y=2.297213in,,base]{\color{textcolor}\rmfamily\fontsize{12.000000}{14.400000}\selectfont \(\displaystyle P_x\)}%
\end{pgfscope}%
\begin{pgfscope}%
\pgfsetroundcap%
\pgfsetroundjoin%
\pgfsetlinewidth{1.003750pt}%
\definecolor{currentstroke}{rgb}{0.000000,0.000000,0.000000}%
\pgfsetstrokecolor{currentstroke}%
\pgfsetdash{}{0pt}%
\pgfpathmoveto{\pgfqpoint{3.883217in}{2.158833in}}%
\pgfpathquadraticcurveto{\pgfqpoint{3.883217in}{1.547451in}}{\pgfqpoint{3.883217in}{0.944962in}}%
\pgfusepath{stroke}%
\end{pgfscope}%
\begin{pgfscope}%
\pgfsetroundcap%
\pgfsetroundjoin%
\pgfsetlinewidth{1.003750pt}%
\definecolor{currentstroke}{rgb}{0.000000,0.000000,0.000000}%
\pgfsetstrokecolor{currentstroke}%
\pgfsetdash{}{0pt}%
\pgfpathmoveto{\pgfqpoint{3.966550in}{1.011629in}}%
\pgfpathlineto{\pgfqpoint{3.883217in}{0.944962in}}%
\pgfpathlineto{\pgfqpoint{3.799883in}{1.011629in}}%
\pgfusepath{stroke}%
\end{pgfscope}%
\begin{pgfscope}%
\pgfsetbuttcap%
\pgfsetmiterjoin%
\definecolor{currentfill}{rgb}{0.800000,0.800000,0.800000}%
\pgfsetfillcolor{currentfill}%
\pgfsetlinewidth{1.003750pt}%
\definecolor{currentstroke}{rgb}{0.000000,0.000000,0.000000}%
\pgfsetstrokecolor{currentstroke}%
\pgfsetdash{}{0pt}%
\pgfpathmoveto{\pgfqpoint{3.752256in}{2.223140in}}%
\pgfpathcurveto{\pgfqpoint{3.793923in}{2.181473in}}{\pgfqpoint{3.972511in}{2.181473in}}{\pgfqpoint{4.014177in}{2.223140in}}%
\pgfpathcurveto{\pgfqpoint{4.055844in}{2.264806in}}{\pgfqpoint{4.055844in}{2.412954in}}{\pgfqpoint{4.014177in}{2.454621in}}%
\pgfpathcurveto{\pgfqpoint{3.972511in}{2.496287in}}{\pgfqpoint{3.793923in}{2.496287in}}{\pgfqpoint{3.752256in}{2.454621in}}%
\pgfpathcurveto{\pgfqpoint{3.710589in}{2.412954in}}{\pgfqpoint{3.710589in}{2.264806in}}{\pgfqpoint{3.752256in}{2.223140in}}%
\pgfpathclose%
\pgfusepath{stroke,fill}%
\end{pgfscope}%
\begin{pgfscope}%
\definecolor{textcolor}{rgb}{0.000000,0.000000,0.000000}%
\pgfsetstrokecolor{textcolor}%
\pgfsetfillcolor{textcolor}%
\pgftext[x=3.883217in,y=2.297213in,,base]{\color{textcolor}\rmfamily\fontsize{12.000000}{14.400000}\selectfont \(\displaystyle P_x\)}%
\end{pgfscope}%
\begin{pgfscope}%
\pgfpathrectangle{\pgfqpoint{1.000000in}{0.330000in}}{\pgfqpoint{6.200000in}{0.604712in}}%
\pgfusepath{clip}%
\pgfsetbuttcap%
\pgfsetroundjoin%
\definecolor{currentfill}{rgb}{1.000000,0.000000,0.000000}%
\pgfsetfillcolor{currentfill}%
\pgfsetlinewidth{1.003750pt}%
\definecolor{currentstroke}{rgb}{1.000000,0.000000,0.000000}%
\pgfsetstrokecolor{currentstroke}%
\pgfsetdash{}{0pt}%
\pgfsys@defobject{currentmarker}{\pgfqpoint{-0.098209in}{-0.098209in}}{\pgfqpoint{0.098209in}{0.098209in}}{%
\pgfpathmoveto{\pgfqpoint{0.000000in}{0.098209in}}%
\pgfpathlineto{\pgfqpoint{-0.098209in}{-0.098209in}}%
\pgfpathlineto{\pgfqpoint{0.098209in}{-0.098209in}}%
\pgfpathclose%
\pgfusepath{stroke,fill}%
}%
\begin{pgfscope}%
\pgfsys@transformshift{3.883217in}{0.357487in}%
\pgfsys@useobject{currentmarker}{}%
\end{pgfscope}%
\begin{pgfscope}%
\pgfsys@transformshift{6.918182in}{0.357487in}%
\pgfsys@useobject{currentmarker}{}%
\end{pgfscope}%
\end{pgfscope}%
\begin{pgfscope}%
\pgfpathrectangle{\pgfqpoint{1.000000in}{0.330000in}}{\pgfqpoint{6.200000in}{0.604712in}}%
\pgfusepath{clip}%
\pgfsetbuttcap%
\pgfsetroundjoin%
\definecolor{currentfill}{rgb}{0.000000,0.000000,1.000000}%
\pgfsetfillcolor{currentfill}%
\pgfsetlinewidth{1.003750pt}%
\definecolor{currentstroke}{rgb}{0.000000,0.000000,1.000000}%
\pgfsetstrokecolor{currentstroke}%
\pgfsetdash{}{0pt}%
\pgfsys@defobject{currentmarker}{\pgfqpoint{-0.098209in}{-0.098209in}}{\pgfqpoint{0.098209in}{0.098209in}}{%
\pgfpathmoveto{\pgfqpoint{-0.098209in}{-0.098209in}}%
\pgfpathlineto{\pgfqpoint{0.098209in}{-0.098209in}}%
\pgfpathlineto{\pgfqpoint{0.098209in}{0.098209in}}%
\pgfpathlineto{\pgfqpoint{-0.098209in}{0.098209in}}%
\pgfpathclose%
\pgfusepath{stroke,fill}%
}%
\end{pgfscope}%
\begin{pgfscope}%
\pgfpathrectangle{\pgfqpoint{1.000000in}{0.330000in}}{\pgfqpoint{6.200000in}{0.604712in}}%
\pgfusepath{clip}%
\pgfsetrectcap%
\pgfsetroundjoin%
\pgfsetlinewidth{1.003750pt}%
\definecolor{currentstroke}{rgb}{0.000000,0.000000,0.000000}%
\pgfsetstrokecolor{currentstroke}%
\pgfsetdash{}{0pt}%
\pgfpathmoveto{\pgfqpoint{1.281818in}{0.907225in}}%
\pgfpathlineto{\pgfqpoint{6.918182in}{0.907225in}}%
\pgfusepath{stroke}%
\end{pgfscope}%
\begin{pgfscope}%
\pgfpathrectangle{\pgfqpoint{1.000000in}{0.330000in}}{\pgfqpoint{6.200000in}{0.604712in}}%
\pgfusepath{clip}%
\pgfsetrectcap%
\pgfsetroundjoin%
\pgfsetlinewidth{1.003750pt}%
\definecolor{currentstroke}{rgb}{0.000000,0.000000,0.000000}%
\pgfsetstrokecolor{currentstroke}%
\pgfsetdash{}{0pt}%
\pgfpathmoveto{\pgfqpoint{1.281818in}{0.872092in}}%
\pgfpathlineto{\pgfqpoint{6.918182in}{0.872092in}}%
\pgfusepath{stroke}%
\end{pgfscope}%
\begin{pgfscope}%
\pgfpathrectangle{\pgfqpoint{1.000000in}{0.330000in}}{\pgfqpoint{6.200000in}{0.604712in}}%
\pgfusepath{clip}%
\pgfsetrectcap%
\pgfsetroundjoin%
\pgfsetlinewidth{1.003750pt}%
\definecolor{currentstroke}{rgb}{0.000000,0.000000,0.000000}%
\pgfsetstrokecolor{currentstroke}%
\pgfsetdash{}{0pt}%
\pgfpathmoveto{\pgfqpoint{1.281818in}{0.485631in}}%
\pgfpathlineto{\pgfqpoint{6.918182in}{0.485631in}}%
\pgfusepath{stroke}%
\end{pgfscope}%
\begin{pgfscope}%
\pgfpathrectangle{\pgfqpoint{1.000000in}{0.330000in}}{\pgfqpoint{6.200000in}{0.604712in}}%
\pgfusepath{clip}%
\pgfsetrectcap%
\pgfsetroundjoin%
\pgfsetlinewidth{1.003750pt}%
\definecolor{currentstroke}{rgb}{0.000000,0.000000,0.000000}%
\pgfsetstrokecolor{currentstroke}%
\pgfsetdash{}{0pt}%
\pgfpathmoveto{\pgfqpoint{1.281818in}{0.520764in}}%
\pgfpathlineto{\pgfqpoint{6.918182in}{0.520764in}}%
\pgfusepath{stroke}%
\end{pgfscope}%
\begin{pgfscope}%
\pgfpathrectangle{\pgfqpoint{1.000000in}{0.330000in}}{\pgfqpoint{6.200000in}{0.604712in}}%
\pgfusepath{clip}%
\pgfsetrectcap%
\pgfsetroundjoin%
\pgfsetlinewidth{1.003750pt}%
\definecolor{currentstroke}{rgb}{0.000000,0.000000,0.000000}%
\pgfsetstrokecolor{currentstroke}%
\pgfsetdash{}{0pt}%
\pgfpathmoveto{\pgfqpoint{1.281818in}{0.485631in}}%
\pgfpathlineto{\pgfqpoint{1.281818in}{0.907225in}}%
\pgfusepath{stroke}%
\end{pgfscope}%
\begin{pgfscope}%
\pgfpathrectangle{\pgfqpoint{1.000000in}{0.330000in}}{\pgfqpoint{6.200000in}{0.604712in}}%
\pgfusepath{clip}%
\pgfsetrectcap%
\pgfsetroundjoin%
\pgfsetlinewidth{1.003750pt}%
\definecolor{currentstroke}{rgb}{0.000000,0.000000,0.000000}%
\pgfsetstrokecolor{currentstroke}%
\pgfsetdash{}{0pt}%
\pgfpathmoveto{\pgfqpoint{6.918182in}{0.485631in}}%
\pgfpathlineto{\pgfqpoint{6.918182in}{0.907225in}}%
\pgfusepath{stroke}%
\end{pgfscope}%
\begin{pgfscope}%
\pgfsetbuttcap%
\pgfsetmiterjoin%
\definecolor{currentfill}{rgb}{0.800000,0.800000,0.800000}%
\pgfsetfillcolor{currentfill}%
\pgfsetlinewidth{1.003750pt}%
\definecolor{currentstroke}{rgb}{0.000000,0.000000,0.000000}%
\pgfsetstrokecolor{currentstroke}%
\pgfsetdash{}{0pt}%
\pgfpathmoveto{\pgfqpoint{3.818363in}{0.599978in}}%
\pgfpathcurveto{\pgfqpoint{3.853086in}{0.565255in}}{\pgfqpoint{4.346914in}{0.565255in}}{\pgfqpoint{4.381637in}{0.599978in}}%
\pgfpathcurveto{\pgfqpoint{4.416359in}{0.634700in}}{\pgfqpoint{4.416359in}{0.758156in}}{\pgfqpoint{4.381637in}{0.792879in}}%
\pgfpathcurveto{\pgfqpoint{4.346914in}{0.827601in}}{\pgfqpoint{3.853086in}{0.827601in}}{\pgfqpoint{3.818363in}{0.792879in}}%
\pgfpathcurveto{\pgfqpoint{3.783641in}{0.758156in}}{\pgfqpoint{3.783641in}{0.634700in}}{\pgfqpoint{3.818363in}{0.599978in}}%
\pgfpathclose%
\pgfusepath{stroke,fill}%
\end{pgfscope}%
\begin{pgfscope}%
\definecolor{textcolor}{rgb}{0.000000,0.000000,0.000000}%
\pgfsetstrokecolor{textcolor}%
\pgfsetfillcolor{textcolor}%
\pgftext[x=4.100000in,y=0.696428in,,]{\color{textcolor}\rmfamily\fontsize{10.000000}{12.000000}\selectfont W10x12}%
\end{pgfscope}%
\begin{pgfscope}%
\pgfsetbuttcap%
\pgfsetmiterjoin%
\definecolor{currentfill}{rgb}{0.800000,0.800000,0.800000}%
\pgfsetfillcolor{currentfill}%
\pgfsetlinewidth{1.003750pt}%
\definecolor{currentstroke}{rgb}{0.000000,0.000000,0.000000}%
\pgfsetstrokecolor{currentstroke}%
\pgfsetdash{}{0pt}%
\pgfpathmoveto{\pgfqpoint{2.133057in}{0.175607in}}%
\pgfpathcurveto{\pgfqpoint{2.167779in}{0.140885in}}{\pgfqpoint{2.997256in}{0.140885in}}{\pgfqpoint{3.031978in}{0.175607in}}%
\pgfpathcurveto{\pgfqpoint{3.066700in}{0.210329in}}{\pgfqpoint{3.066700in}{0.333786in}}{\pgfqpoint{3.031978in}{0.368508in}}%
\pgfpathcurveto{\pgfqpoint{2.997256in}{0.403230in}}{\pgfqpoint{2.167779in}{0.403230in}}{\pgfqpoint{2.133057in}{0.368508in}}%
\pgfpathcurveto{\pgfqpoint{2.098334in}{0.333786in}}{\pgfqpoint{2.098334in}{0.210329in}}{\pgfqpoint{2.133057in}{0.175607in}}%
\pgfpathclose%
\pgfusepath{stroke,fill}%
\end{pgfscope}%
\begin{pgfscope}%
\definecolor{textcolor}{rgb}{0.000000,0.000000,0.000000}%
\pgfsetstrokecolor{textcolor}%
\pgfsetfillcolor{textcolor}%
\pgftext[x=2.582517in,y=0.272058in,,]{\color{textcolor}\rmfamily\fontsize{10.000000}{12.000000}\selectfont Span 0 = 6 ft}%
\end{pgfscope}%
\begin{pgfscope}%
\pgfsetbuttcap%
\pgfsetmiterjoin%
\definecolor{currentfill}{rgb}{0.800000,0.800000,0.800000}%
\pgfsetfillcolor{currentfill}%
\pgfsetlinewidth{1.003750pt}%
\definecolor{currentstroke}{rgb}{0.000000,0.000000,0.000000}%
\pgfsetstrokecolor{currentstroke}%
\pgfsetdash{}{0pt}%
\pgfpathmoveto{\pgfqpoint{4.951239in}{0.175607in}}%
\pgfpathcurveto{\pgfqpoint{4.985961in}{0.140885in}}{\pgfqpoint{5.815438in}{0.140885in}}{\pgfqpoint{5.850160in}{0.175607in}}%
\pgfpathcurveto{\pgfqpoint{5.884882in}{0.210329in}}{\pgfqpoint{5.884882in}{0.333786in}}{\pgfqpoint{5.850160in}{0.368508in}}%
\pgfpathcurveto{\pgfqpoint{5.815438in}{0.403230in}}{\pgfqpoint{4.985961in}{0.403230in}}{\pgfqpoint{4.951239in}{0.368508in}}%
\pgfpathcurveto{\pgfqpoint{4.916516in}{0.333786in}}{\pgfqpoint{4.916516in}{0.210329in}}{\pgfqpoint{4.951239in}{0.175607in}}%
\pgfpathclose%
\pgfusepath{stroke,fill}%
\end{pgfscope}%
\begin{pgfscope}%
\definecolor{textcolor}{rgb}{0.000000,0.000000,0.000000}%
\pgfsetstrokecolor{textcolor}%
\pgfsetfillcolor{textcolor}%
\pgftext[x=5.400699in,y=0.272058in,,]{\color{textcolor}\rmfamily\fontsize{10.000000}{12.000000}\selectfont Span 1 = 7 ft}%
\end{pgfscope}%
\end{pgfpicture}%
\makeatother%
\endgroup%

\end{center}
\vspace{-18pt}
\caption{Applied Loads}
\end{figure}
The following distributed loads are applied to the beam. The program can handle all possible mass and force units in both metric and imperial systems simultaneously. Loads are plotted to scale according to their relative magnitudes. A "positive" load is defined as a load acting in the direction of gravity.
\begin{table}[ht]
\caption{Applied Distributed Loads}
\centering
\begin{tabular}{l l l l l l l}
\hline
Load & Start Loc. & Start Mag. & End Loc. & End Mag. & Type & Description\\
\hline
w\textsubscript{1} & 0 {\color{darkBlue}{\textbf{ft}}} & 0.026 {\color{darkBlue}{\textbf{klf}}} & 13 {\color{darkBlue}{\textbf{ft}}} & 0.026 {\color{darkBlue}{\textbf{klf}}} & D & floor\\
w\textsubscript{2} & 0 {\color{darkBlue}{\textbf{ft}}} & 0.02 {\color{darkBlue}{\textbf{klf}}} & 13 {\color{darkBlue}{\textbf{ft}}} & 0.02 {\color{darkBlue}{\textbf{klf}}} & Lr & floor\\
w\textsubscript{3} & 0 {\color{darkBlue}{\textbf{ft}}} & 12.0 {\color{darkBlue}{\textbf{plf}}} & 13 {\color{darkBlue}{\textbf{ft}}} & 12.0 {\color{darkBlue}{\textbf{plf}}} & D & Self weight\\
\hline
\end{tabular}
\end{table}
\begin{table}[ht]
\caption{Applied Point Loads}
\centering
\begin{tabular}{l l l l l l}
\hline
Load & Loc. & Shear & Type & Description \\
\hline
P\textsubscript{1} & 0 {\color{darkBlue}{\textbf{ft}}} & 0.39 {\color{darkBlue}{\textbf{kip}}} & D & No description\\
P\textsubscript{2} & 0 {\color{darkBlue}{\textbf{ft}}} & 0.34 {\color{darkBlue}{\textbf{kip}}} & Lr & No description\\
P\textsubscript{3} & 0 {\color{darkBlue}{\textbf{ft}}} & 0.01 {\color{darkBlue}{\textbf{kip}}} & L & No description\\
P\textsubscript{4} & 6 {\color{darkBlue}{\textbf{ft}}} & 1.3 {\color{darkBlue}{\textbf{kip}}} & D & No description\\
P\textsubscript{5} & 6 {\color{darkBlue}{\textbf{ft}}} & 0.7 {\color{darkBlue}{\textbf{kip}}} & Lr & No description\\
P\textsubscript{6} & 6 {\color{darkBlue}{\textbf{ft}}} & 0.3 {\color{darkBlue}{\textbf{kip}}} & L & No description\\
P\textsubscript{7} & 13 {\color{darkBlue}{\textbf{ft}}} & 3.57 {\color{darkBlue}{\textbf{kip}}} & D & No description\\
P\textsubscript{8} & 13 {\color{darkBlue}{\textbf{ft}}} & 0.64 {\color{darkBlue}{\textbf{kip}}} & Lr & No description\\
P\textsubscript{9} & 13 {\color{darkBlue}{\textbf{ft}}} & 2.67 {\color{darkBlue}{\textbf{kip}}} & L & No description\\
P\textsubscript{10} & 13 {\color{darkBlue}{\textbf{ft}}} & 0.11 {\color{darkBlue}{\textbf{kip}}} & E & No description\\
\hline
\end{tabular}
\end{table}
%	-------------------------------- LOAD COMBOS	--------------------------------
\section{Load Combinations}
The following load combinations are used for the design. Duplicate load combinations are not listed and only loads that are used on the beam are included in the load combinations (i.e. If soil load is not included as a load type in any of the applied loads, then "H" loads will not be included in the listed load combinations). S\textsubscript{DS} is input as 1.0 and \textOmega\textsubscript{0} is input as 2.5 for use in seismic load combinations. Any load designated as a pattern load is applied to spans in all possible permutations to create the most extreme loading condition. Numbers after a load indicate the span over which the pattern load is applied (i.e. L0 indicates that live load is applied only on the first span).
\begin{table}[H]
\caption{Strength (LRFD) Load Combinations}
\centering
\begin{tabular}{l l l}
\hline
Load Combo & Loads and Factors & Reference\\
\hline
LC 1 & 1.4D + -2.5E + 0.5L0 & ASCE7-16 \S2.3.6 (LC 6)\\
LC 2 & 1.2D + 1.0L1 & ASCE7-16 \S2.3.1 (LC 3)\\
LC 3 & 1.4D + 2.5E + 0.5L0 & ASCE7-16 \S2.3.6 (LC 6)\\
LC 4 & 1.2D + 1.6L0 + 1.6L1 & ASCE7-16 \S2.3.1 (LC 2)\\
LC 5 & 0.7D + -2.5E & ASCE7-16 \S2.3.6 (LC 7)\\
LC 6 & 1.2D + 1.0L0 + 1.0L1 & ASCE7-16 \S2.3.1 (LC 3)\\
LC 7 & 1.4D + -2.5E + 0.5L0 + 0.5L1 & ASCE7-16 \S2.3.6 (LC 6)\\
LC 8 & 1.2D + 1.6L1 & ASCE7-16 \S2.3.1 (LC 2)\\
LC 9 & 1.2D + 1.6L\textsubscriptr0 & ASCE7-16 \S2.3.1 (LC 3)\\
LC 10 & 1.4D + 2.5E + 0.5L0 + 0.5L1 & ASCE7-16 \S2.3.6 (LC 6)\\
LC 11 & 1.2D + 1.6L\textsubscriptr1 & ASCE7-16 \S2.3.1 (LC 3)\\
LC 12 & 1.4D + 2.5E + 0.5L1 & ASCE7-16 \S2.3.6 (LC 6)\\
LC 13 & 1.2D + 1.6L\textsubscriptr1 + 1.0L1 & ASCE7-16 \S2.3.1 (LC 3)\\
LC 14 & 1.2D + 0.5L\textsubscriptr0 + 1.0L0 & ASCE7-16 \S2.3.1 (LC 4)\\
LC 15 & 1.4D & ASCE7-16 \S2.3.1 (LC 1)\\
LC 16 & 1.2D + 1.0L0 & ASCE7-16 \S2.3.1 (LC 3)\\
LC 17 & 1.2D + 1.6L\textsubscriptr0 + 1.0L0 & ASCE7-16 \S2.3.1 (LC 3)\\
LC 18 & 1.2D + 1.6L\textsubscriptr0 + 1.0L0 + 1.6L\textsubscriptr1 + 1.0L1 & ASCE7-16 \S2.3.1 (LC 3)\\
LC 19 & 1.2D + 0.5L\textsubscriptr0 + 1.6L0 & ASCE7-16 \S2.3.1 (LC 2)\\
LC 20 & 1.2D + 1.6L\textsubscriptr0 + 1.6L\textsubscriptr1 & ASCE7-16 \S2.3.1 (LC 3)\\
LC 21 & 1.4D + -2.5E + 0.5L1 & ASCE7-16 \S2.3.6 (LC 6)\\
LC 22 & 1.2D + 1.6L0 & ASCE7-16 \S2.3.1 (LC 2)\\
LC 23 & 1.2D + 0.5L\textsubscriptr0 + 1.0L0 + 0.5L\textsubscriptr1 + 1.0L1 & ASCE7-16 \S2.3.1 (LC 4)\\
LC 24 & 0.7D + 2.5E & ASCE7-16 \S2.3.6 (LC 7)\\
LC 25 & 1.2D + 0.5L\textsubscriptr0 + 1.6L0 + 0.5L\textsubscriptr1 + 1.6L1 & ASCE7-16 \S2.3.1 (LC 2)\\
LC 26 & 1.2D & ASCE7-16 \S2.3.1 (LC 3)\\
LC 27 & 1.2D + 0.5L\textsubscriptr1 + 1.6L1 & ASCE7-16 \S2.3.1 (LC 2)\\
LC 28 & 1.2D + 0.5L\textsubscriptr1 + 1.0L1 & ASCE7-16 \S2.3.1 (LC 4)\\
LC 29 & 0.9D & ASCE7-16 \S2.3.1 (LC 5)\\
\hline
\end{tabular}
\end{table}
\begin{table}[H]
\caption{Deflection (ASD) Load Combinations}
\centering
\begin{tabular}{l l l}
\hline
Load Combo & Loads and Factors & Reference\\
\hline
LC 1 & 1.0D & ASCE7-16 \S2.4.1 (LC 1)\\
LC 2 & 1.0D + 0.75L\textsubscriptr0 + 0.75L0 & ASCE7-16 \S2.4.1 (LC 4)\\
LC 3 & 1.0D + 1.0L1 & ASCE7-16 \S2.4.1 (LC 2)\\
LC 4 & 1.0D + 1.0L\textsubscriptr0 + 1.0L\textsubscriptr1 & ASCE7-16 \S2.4.1 (LC 3)\\
LC 5 & 1.0D + 0.75L0 & ASCE7-16 \S2.4.1 (LC 4)\\
LC 6 & 1.1D + -1.31E + 0.75L1 & ASCE7-16 \S2.4.5 (LC 8)\\
LC 7 & 1.0D + 1.0L\textsubscriptr0 & ASCE7-16 \S2.4.1 (LC 3)\\
LC 8 & 0.4D + 1.75E & ASCE7-16 \S2.4.5 (LC 9)\\
LC 9 & 0.4D + -1.75E & ASCE7-16 \S2.4.5 (LC 9)\\
LC 10 & 1.0L1 & L only deflection check\\
LC 11 & 1.0L0 + 1.0L1 & L only deflection check\\
LC 12 & 1.0D + 1.0L\textsubscriptr1 & ASCE7-16 \S2.4.1 (LC 3)\\
LC 13 & 1.0D + 0.75L\textsubscriptr1 + 0.75L1 & ASCE7-16 \S2.4.1 (LC 4)\\
LC 14 & 1.14D + 1.75E & ASCE7-16 \S2.4.5 (LC 8)\\
LC 15 & 1.0D + 0.75L\textsubscriptr0 + 0.75L0 + 0.75L\textsubscriptr1 + 0.75L1 & ASCE7-16 \S2.4.1 (LC 4)\\
LC 16 & 1.0D + 0.75L1 & ASCE7-16 \S2.4.1 (LC 4)\\
LC 17 & 0.6D & ASCE7-16 \S2.4.1 (LC 7)\\
LC 18 & 1.1D + -1.31E + 0.75L0 + 0.75L1 & ASCE7-16 \S2.4.5 (LC 8)\\
LC 19 & 1.1D + 1.31E + 0.75L0 & ASCE7-16 \S2.4.5 (LC 8)\\
LC 20 & 1.0D + 1.0L0 & ASCE7-16 \S2.4.1 (LC 2)\\
LC 21 & 1.0L0 & L only deflection check\\
LC 22 & 1.0D + 0.75L0 + 0.75L1 & ASCE7-16 \S2.4.1 (LC 4)\\
LC 23 & 1.1D + 1.31E + 0.75L1 & ASCE7-16 \S2.4.5 (LC 8)\\
LC 24 & 1.1D + 1.31E + 0.75L0 + 0.75L1 & ASCE7-16 \S2.4.5 (LC 8)\\
LC 25 & 1.0D + 1.0L0 + 1.0L1 & ASCE7-16 \S2.4.1 (LC 2)\\
LC 26 & 1.14D + -1.75E & ASCE7-16 \S2.4.5 (LC 8)\\
LC 27 & 1.1D + -1.31E + 0.75L0 & ASCE7-16 \S2.4.5 (LC 8)\\
\hline
\end{tabular}
\end{table}
%	---------------------- SECTIONAL & MATERIAL PROPERTIES -----------------------
\section{Sectional and Material Properties}
The following are sectional and material properties used for analysis \textbf{(W10x12, Grade A992)}:
\begin{table}[ht]
\caption{Sectional and Material Properties}
\vspace{-10pt}
\centering
\begin{tabular}{lll}
\centering
\begin{tabular}[t]{ll}
\cline{1-2}
Property & Value \\
\cline{1-2}
A\textsubscript{w} & 1.9 {\color{darkBlue}{\textbf{{\color{darkBlue}{\textbf{in}}}\textsuperscript{2}}}} \\
C\textsubscript{w} & 50.9 {\color{darkBlue}{\textbf{{\color{darkBlue}{\textbf{in}}}\textsuperscript{6}}}} \\
F\textsubscript{u} & 65 {\color{darkBlue}{\textbf{ksi}}} \\
F\textsubscript{y} & 50 {\color{darkBlue}{\textbf{ksi}}} \\
I\textsubscript{x} & 53.8 {\color{darkBlue}{\textbf{{\color{darkBlue}{\textbf{in}}}\textsuperscript{4}}}} \\
I\textsubscript{y} & 2.2 {\color{darkBlue}{\textbf{{\color{darkBlue}{\textbf{in}}}\textsuperscript{4}}}} \\
\cline{1-2}
\end{tabular}
&
\begin{tabular}[t]{ll}
\cline{1-2}
Property & Value \\
\cline{1-2}
S\textsubscript{x} & 10.9 {\color{darkBlue}{\textbf{{\color{darkBlue}{\textbf{in}}}\textsuperscript{3}}}} \\
S\textsubscript{y} & 1.1 {\color{darkBlue}{\textbf{{\color{darkBlue}{\textbf{in}}}\textsuperscript{3}}}} \\
Z\textsubscript{x} & 12.6 {\color{darkBlue}{\textbf{{\color{darkBlue}{\textbf{in}}}\textsuperscript{3}}}} \\
Z\textsubscript{y} & 1.7 {\color{darkBlue}{\textbf{{\color{darkBlue}{\textbf{in}}}\textsuperscript{3}}}} \\
r\textsubscript{x} & 3.9 {\color{darkBlue}{\textbf{in}}} \\
r\textsubscript{y} & 0.8 {\color{darkBlue}{\textbf{in}}} \\
\cline{1-2}
\end{tabular}
&
\begin{tabular}[t]{ll}
\cline{1-2}
Property & Value \\
\cline{1-2}
b\textsubscript{f} & 4.0 {\color{darkBlue}{\textbf{in}}} \\
t\textsubscript{f} & 0.2 {\color{darkBlue}{\textbf{in}}} \\
t\textsubscript{w} & 0.2 {\color{darkBlue}{\textbf{in}}} \\
h\textsubscript{0} & 9.7 {\color{darkBlue}{\textbf{in}}} \\
U.W. & 490 {\color{darkBlue}{\textbf{pcf}}} \\
\cline{1-2}
\end{tabular}
\end{tabular}
\end{table}
%	-------------------------------- BENDING CHECK -------------------------------
\section{Bending Check}
\begin{figure}[H]
\begin{center}
%% Creator: Matplotlib, PGF backend
%%
%% To include the figure in your LaTeX document, write
%%   \input{<filename>.pgf}
%%
%% Make sure the required packages are loaded in your preamble
%%   \usepackage{pgf}
%%
%% Figures using additional raster images can only be included by \input if
%% they are in the same directory as the main LaTeX file. For loading figures
%% from other directories you can use the `import` package
%%   \usepackage{import}
%%
%% and then include the figures with
%%   \import{<path to file>}{<filename>.pgf}
%%
%% Matplotlib used the following preamble
%%
\begingroup%
\makeatletter%
\begin{pgfpicture}%
\pgfpathrectangle{\pgfpointorigin}{\pgfqpoint{8.000000in}{3.000000in}}%
\pgfusepath{use as bounding box, clip}%
\begin{pgfscope}%
\pgfsetbuttcap%
\pgfsetmiterjoin%
\definecolor{currentfill}{rgb}{1.000000,1.000000,1.000000}%
\pgfsetfillcolor{currentfill}%
\pgfsetlinewidth{0.000000pt}%
\definecolor{currentstroke}{rgb}{1.000000,1.000000,1.000000}%
\pgfsetstrokecolor{currentstroke}%
\pgfsetdash{}{0pt}%
\pgfpathmoveto{\pgfqpoint{0.000000in}{0.000000in}}%
\pgfpathlineto{\pgfqpoint{8.000000in}{0.000000in}}%
\pgfpathlineto{\pgfqpoint{8.000000in}{3.000000in}}%
\pgfpathlineto{\pgfqpoint{0.000000in}{3.000000in}}%
\pgfpathclose%
\pgfusepath{fill}%
\end{pgfscope}%
\begin{pgfscope}%
\pgfsetbuttcap%
\pgfsetmiterjoin%
\definecolor{currentfill}{rgb}{1.000000,1.000000,1.000000}%
\pgfsetfillcolor{currentfill}%
\pgfsetlinewidth{0.000000pt}%
\definecolor{currentstroke}{rgb}{0.000000,0.000000,0.000000}%
\pgfsetstrokecolor{currentstroke}%
\pgfsetstrokeopacity{0.000000}%
\pgfsetdash{}{0pt}%
\pgfpathmoveto{\pgfqpoint{1.000000in}{0.330000in}}%
\pgfpathlineto{\pgfqpoint{7.200000in}{0.330000in}}%
\pgfpathlineto{\pgfqpoint{7.200000in}{2.640000in}}%
\pgfpathlineto{\pgfqpoint{1.000000in}{2.640000in}}%
\pgfpathclose%
\pgfusepath{fill}%
\end{pgfscope}%
\begin{pgfscope}%
\pgfpathrectangle{\pgfqpoint{1.000000in}{0.330000in}}{\pgfqpoint{6.200000in}{2.310000in}}%
\pgfusepath{clip}%
\pgfsetbuttcap%
\pgfsetroundjoin%
\pgfsetlinewidth{0.803000pt}%
\definecolor{currentstroke}{rgb}{0.000000,0.000000,0.000000}%
\pgfsetstrokecolor{currentstroke}%
\pgfsetdash{{0.800000pt}{1.320000pt}}{0.000000pt}%
\pgfpathmoveto{\pgfqpoint{1.281818in}{0.330000in}}%
\pgfpathlineto{\pgfqpoint{1.281818in}{2.640000in}}%
\pgfusepath{stroke}%
\end{pgfscope}%
\begin{pgfscope}%
\pgfsetbuttcap%
\pgfsetroundjoin%
\definecolor{currentfill}{rgb}{0.000000,0.000000,0.000000}%
\pgfsetfillcolor{currentfill}%
\pgfsetlinewidth{0.803000pt}%
\definecolor{currentstroke}{rgb}{0.000000,0.000000,0.000000}%
\pgfsetstrokecolor{currentstroke}%
\pgfsetdash{}{0pt}%
\pgfsys@defobject{currentmarker}{\pgfqpoint{0.000000in}{-0.048611in}}{\pgfqpoint{0.000000in}{0.000000in}}{%
\pgfpathmoveto{\pgfqpoint{0.000000in}{0.000000in}}%
\pgfpathlineto{\pgfqpoint{0.000000in}{-0.048611in}}%
\pgfusepath{stroke,fill}%
}%
\begin{pgfscope}%
\pgfsys@transformshift{1.281818in}{0.330000in}%
\pgfsys@useobject{currentmarker}{}%
\end{pgfscope}%
\end{pgfscope}%
\begin{pgfscope}%
\pgfsetbuttcap%
\pgfsetroundjoin%
\definecolor{currentfill}{rgb}{0.000000,0.000000,0.000000}%
\pgfsetfillcolor{currentfill}%
\pgfsetlinewidth{0.803000pt}%
\definecolor{currentstroke}{rgb}{0.000000,0.000000,0.000000}%
\pgfsetstrokecolor{currentstroke}%
\pgfsetdash{}{0pt}%
\pgfsys@defobject{currentmarker}{\pgfqpoint{0.000000in}{0.000000in}}{\pgfqpoint{0.000000in}{0.048611in}}{%
\pgfpathmoveto{\pgfqpoint{0.000000in}{0.000000in}}%
\pgfpathlineto{\pgfqpoint{0.000000in}{0.048611in}}%
\pgfusepath{stroke,fill}%
}%
\begin{pgfscope}%
\pgfsys@transformshift{1.281818in}{2.640000in}%
\pgfsys@useobject{currentmarker}{}%
\end{pgfscope}%
\end{pgfscope}%
\begin{pgfscope}%
\definecolor{textcolor}{rgb}{0.000000,0.000000,0.000000}%
\pgfsetstrokecolor{textcolor}%
\pgfsetfillcolor{textcolor}%
\pgftext[x=1.281818in,y=0.232778in,,top]{\color{textcolor}\rmfamily\fontsize{10.000000}{12.000000}\selectfont \(\displaystyle {0}\)}%
\end{pgfscope}%
\begin{pgfscope}%
\pgfpathrectangle{\pgfqpoint{1.000000in}{0.330000in}}{\pgfqpoint{6.200000in}{2.310000in}}%
\pgfusepath{clip}%
\pgfsetbuttcap%
\pgfsetroundjoin%
\pgfsetlinewidth{0.803000pt}%
\definecolor{currentstroke}{rgb}{0.000000,0.000000,0.000000}%
\pgfsetstrokecolor{currentstroke}%
\pgfsetdash{{0.800000pt}{1.320000pt}}{0.000000pt}%
\pgfpathmoveto{\pgfqpoint{2.148951in}{0.330000in}}%
\pgfpathlineto{\pgfqpoint{2.148951in}{2.640000in}}%
\pgfusepath{stroke}%
\end{pgfscope}%
\begin{pgfscope}%
\pgfsetbuttcap%
\pgfsetroundjoin%
\definecolor{currentfill}{rgb}{0.000000,0.000000,0.000000}%
\pgfsetfillcolor{currentfill}%
\pgfsetlinewidth{0.803000pt}%
\definecolor{currentstroke}{rgb}{0.000000,0.000000,0.000000}%
\pgfsetstrokecolor{currentstroke}%
\pgfsetdash{}{0pt}%
\pgfsys@defobject{currentmarker}{\pgfqpoint{0.000000in}{-0.048611in}}{\pgfqpoint{0.000000in}{0.000000in}}{%
\pgfpathmoveto{\pgfqpoint{0.000000in}{0.000000in}}%
\pgfpathlineto{\pgfqpoint{0.000000in}{-0.048611in}}%
\pgfusepath{stroke,fill}%
}%
\begin{pgfscope}%
\pgfsys@transformshift{2.148951in}{0.330000in}%
\pgfsys@useobject{currentmarker}{}%
\end{pgfscope}%
\end{pgfscope}%
\begin{pgfscope}%
\pgfsetbuttcap%
\pgfsetroundjoin%
\definecolor{currentfill}{rgb}{0.000000,0.000000,0.000000}%
\pgfsetfillcolor{currentfill}%
\pgfsetlinewidth{0.803000pt}%
\definecolor{currentstroke}{rgb}{0.000000,0.000000,0.000000}%
\pgfsetstrokecolor{currentstroke}%
\pgfsetdash{}{0pt}%
\pgfsys@defobject{currentmarker}{\pgfqpoint{0.000000in}{0.000000in}}{\pgfqpoint{0.000000in}{0.048611in}}{%
\pgfpathmoveto{\pgfqpoint{0.000000in}{0.000000in}}%
\pgfpathlineto{\pgfqpoint{0.000000in}{0.048611in}}%
\pgfusepath{stroke,fill}%
}%
\begin{pgfscope}%
\pgfsys@transformshift{2.148951in}{2.640000in}%
\pgfsys@useobject{currentmarker}{}%
\end{pgfscope}%
\end{pgfscope}%
\begin{pgfscope}%
\definecolor{textcolor}{rgb}{0.000000,0.000000,0.000000}%
\pgfsetstrokecolor{textcolor}%
\pgfsetfillcolor{textcolor}%
\pgftext[x=2.148951in,y=0.232778in,,top]{\color{textcolor}\rmfamily\fontsize{10.000000}{12.000000}\selectfont \(\displaystyle {2}\)}%
\end{pgfscope}%
\begin{pgfscope}%
\pgfpathrectangle{\pgfqpoint{1.000000in}{0.330000in}}{\pgfqpoint{6.200000in}{2.310000in}}%
\pgfusepath{clip}%
\pgfsetbuttcap%
\pgfsetroundjoin%
\pgfsetlinewidth{0.803000pt}%
\definecolor{currentstroke}{rgb}{0.000000,0.000000,0.000000}%
\pgfsetstrokecolor{currentstroke}%
\pgfsetdash{{0.800000pt}{1.320000pt}}{0.000000pt}%
\pgfpathmoveto{\pgfqpoint{3.016084in}{0.330000in}}%
\pgfpathlineto{\pgfqpoint{3.016084in}{2.640000in}}%
\pgfusepath{stroke}%
\end{pgfscope}%
\begin{pgfscope}%
\pgfsetbuttcap%
\pgfsetroundjoin%
\definecolor{currentfill}{rgb}{0.000000,0.000000,0.000000}%
\pgfsetfillcolor{currentfill}%
\pgfsetlinewidth{0.803000pt}%
\definecolor{currentstroke}{rgb}{0.000000,0.000000,0.000000}%
\pgfsetstrokecolor{currentstroke}%
\pgfsetdash{}{0pt}%
\pgfsys@defobject{currentmarker}{\pgfqpoint{0.000000in}{-0.048611in}}{\pgfqpoint{0.000000in}{0.000000in}}{%
\pgfpathmoveto{\pgfqpoint{0.000000in}{0.000000in}}%
\pgfpathlineto{\pgfqpoint{0.000000in}{-0.048611in}}%
\pgfusepath{stroke,fill}%
}%
\begin{pgfscope}%
\pgfsys@transformshift{3.016084in}{0.330000in}%
\pgfsys@useobject{currentmarker}{}%
\end{pgfscope}%
\end{pgfscope}%
\begin{pgfscope}%
\pgfsetbuttcap%
\pgfsetroundjoin%
\definecolor{currentfill}{rgb}{0.000000,0.000000,0.000000}%
\pgfsetfillcolor{currentfill}%
\pgfsetlinewidth{0.803000pt}%
\definecolor{currentstroke}{rgb}{0.000000,0.000000,0.000000}%
\pgfsetstrokecolor{currentstroke}%
\pgfsetdash{}{0pt}%
\pgfsys@defobject{currentmarker}{\pgfqpoint{0.000000in}{0.000000in}}{\pgfqpoint{0.000000in}{0.048611in}}{%
\pgfpathmoveto{\pgfqpoint{0.000000in}{0.000000in}}%
\pgfpathlineto{\pgfqpoint{0.000000in}{0.048611in}}%
\pgfusepath{stroke,fill}%
}%
\begin{pgfscope}%
\pgfsys@transformshift{3.016084in}{2.640000in}%
\pgfsys@useobject{currentmarker}{}%
\end{pgfscope}%
\end{pgfscope}%
\begin{pgfscope}%
\definecolor{textcolor}{rgb}{0.000000,0.000000,0.000000}%
\pgfsetstrokecolor{textcolor}%
\pgfsetfillcolor{textcolor}%
\pgftext[x=3.016084in,y=0.232778in,,top]{\color{textcolor}\rmfamily\fontsize{10.000000}{12.000000}\selectfont \(\displaystyle {4}\)}%
\end{pgfscope}%
\begin{pgfscope}%
\pgfpathrectangle{\pgfqpoint{1.000000in}{0.330000in}}{\pgfqpoint{6.200000in}{2.310000in}}%
\pgfusepath{clip}%
\pgfsetbuttcap%
\pgfsetroundjoin%
\pgfsetlinewidth{0.803000pt}%
\definecolor{currentstroke}{rgb}{0.000000,0.000000,0.000000}%
\pgfsetstrokecolor{currentstroke}%
\pgfsetdash{{0.800000pt}{1.320000pt}}{0.000000pt}%
\pgfpathmoveto{\pgfqpoint{3.883217in}{0.330000in}}%
\pgfpathlineto{\pgfqpoint{3.883217in}{2.640000in}}%
\pgfusepath{stroke}%
\end{pgfscope}%
\begin{pgfscope}%
\pgfsetbuttcap%
\pgfsetroundjoin%
\definecolor{currentfill}{rgb}{0.000000,0.000000,0.000000}%
\pgfsetfillcolor{currentfill}%
\pgfsetlinewidth{0.803000pt}%
\definecolor{currentstroke}{rgb}{0.000000,0.000000,0.000000}%
\pgfsetstrokecolor{currentstroke}%
\pgfsetdash{}{0pt}%
\pgfsys@defobject{currentmarker}{\pgfqpoint{0.000000in}{-0.048611in}}{\pgfqpoint{0.000000in}{0.000000in}}{%
\pgfpathmoveto{\pgfqpoint{0.000000in}{0.000000in}}%
\pgfpathlineto{\pgfqpoint{0.000000in}{-0.048611in}}%
\pgfusepath{stroke,fill}%
}%
\begin{pgfscope}%
\pgfsys@transformshift{3.883217in}{0.330000in}%
\pgfsys@useobject{currentmarker}{}%
\end{pgfscope}%
\end{pgfscope}%
\begin{pgfscope}%
\pgfsetbuttcap%
\pgfsetroundjoin%
\definecolor{currentfill}{rgb}{0.000000,0.000000,0.000000}%
\pgfsetfillcolor{currentfill}%
\pgfsetlinewidth{0.803000pt}%
\definecolor{currentstroke}{rgb}{0.000000,0.000000,0.000000}%
\pgfsetstrokecolor{currentstroke}%
\pgfsetdash{}{0pt}%
\pgfsys@defobject{currentmarker}{\pgfqpoint{0.000000in}{0.000000in}}{\pgfqpoint{0.000000in}{0.048611in}}{%
\pgfpathmoveto{\pgfqpoint{0.000000in}{0.000000in}}%
\pgfpathlineto{\pgfqpoint{0.000000in}{0.048611in}}%
\pgfusepath{stroke,fill}%
}%
\begin{pgfscope}%
\pgfsys@transformshift{3.883217in}{2.640000in}%
\pgfsys@useobject{currentmarker}{}%
\end{pgfscope}%
\end{pgfscope}%
\begin{pgfscope}%
\definecolor{textcolor}{rgb}{0.000000,0.000000,0.000000}%
\pgfsetstrokecolor{textcolor}%
\pgfsetfillcolor{textcolor}%
\pgftext[x=3.883217in,y=0.232778in,,top]{\color{textcolor}\rmfamily\fontsize{10.000000}{12.000000}\selectfont \(\displaystyle {6}\)}%
\end{pgfscope}%
\begin{pgfscope}%
\pgfpathrectangle{\pgfqpoint{1.000000in}{0.330000in}}{\pgfqpoint{6.200000in}{2.310000in}}%
\pgfusepath{clip}%
\pgfsetbuttcap%
\pgfsetroundjoin%
\pgfsetlinewidth{0.803000pt}%
\definecolor{currentstroke}{rgb}{0.000000,0.000000,0.000000}%
\pgfsetstrokecolor{currentstroke}%
\pgfsetdash{{0.800000pt}{1.320000pt}}{0.000000pt}%
\pgfpathmoveto{\pgfqpoint{4.750350in}{0.330000in}}%
\pgfpathlineto{\pgfqpoint{4.750350in}{2.640000in}}%
\pgfusepath{stroke}%
\end{pgfscope}%
\begin{pgfscope}%
\pgfsetbuttcap%
\pgfsetroundjoin%
\definecolor{currentfill}{rgb}{0.000000,0.000000,0.000000}%
\pgfsetfillcolor{currentfill}%
\pgfsetlinewidth{0.803000pt}%
\definecolor{currentstroke}{rgb}{0.000000,0.000000,0.000000}%
\pgfsetstrokecolor{currentstroke}%
\pgfsetdash{}{0pt}%
\pgfsys@defobject{currentmarker}{\pgfqpoint{0.000000in}{-0.048611in}}{\pgfqpoint{0.000000in}{0.000000in}}{%
\pgfpathmoveto{\pgfqpoint{0.000000in}{0.000000in}}%
\pgfpathlineto{\pgfqpoint{0.000000in}{-0.048611in}}%
\pgfusepath{stroke,fill}%
}%
\begin{pgfscope}%
\pgfsys@transformshift{4.750350in}{0.330000in}%
\pgfsys@useobject{currentmarker}{}%
\end{pgfscope}%
\end{pgfscope}%
\begin{pgfscope}%
\pgfsetbuttcap%
\pgfsetroundjoin%
\definecolor{currentfill}{rgb}{0.000000,0.000000,0.000000}%
\pgfsetfillcolor{currentfill}%
\pgfsetlinewidth{0.803000pt}%
\definecolor{currentstroke}{rgb}{0.000000,0.000000,0.000000}%
\pgfsetstrokecolor{currentstroke}%
\pgfsetdash{}{0pt}%
\pgfsys@defobject{currentmarker}{\pgfqpoint{0.000000in}{0.000000in}}{\pgfqpoint{0.000000in}{0.048611in}}{%
\pgfpathmoveto{\pgfqpoint{0.000000in}{0.000000in}}%
\pgfpathlineto{\pgfqpoint{0.000000in}{0.048611in}}%
\pgfusepath{stroke,fill}%
}%
\begin{pgfscope}%
\pgfsys@transformshift{4.750350in}{2.640000in}%
\pgfsys@useobject{currentmarker}{}%
\end{pgfscope}%
\end{pgfscope}%
\begin{pgfscope}%
\definecolor{textcolor}{rgb}{0.000000,0.000000,0.000000}%
\pgfsetstrokecolor{textcolor}%
\pgfsetfillcolor{textcolor}%
\pgftext[x=4.750350in,y=0.232778in,,top]{\color{textcolor}\rmfamily\fontsize{10.000000}{12.000000}\selectfont \(\displaystyle {8}\)}%
\end{pgfscope}%
\begin{pgfscope}%
\pgfpathrectangle{\pgfqpoint{1.000000in}{0.330000in}}{\pgfqpoint{6.200000in}{2.310000in}}%
\pgfusepath{clip}%
\pgfsetbuttcap%
\pgfsetroundjoin%
\pgfsetlinewidth{0.803000pt}%
\definecolor{currentstroke}{rgb}{0.000000,0.000000,0.000000}%
\pgfsetstrokecolor{currentstroke}%
\pgfsetdash{{0.800000pt}{1.320000pt}}{0.000000pt}%
\pgfpathmoveto{\pgfqpoint{5.617483in}{0.330000in}}%
\pgfpathlineto{\pgfqpoint{5.617483in}{2.640000in}}%
\pgfusepath{stroke}%
\end{pgfscope}%
\begin{pgfscope}%
\pgfsetbuttcap%
\pgfsetroundjoin%
\definecolor{currentfill}{rgb}{0.000000,0.000000,0.000000}%
\pgfsetfillcolor{currentfill}%
\pgfsetlinewidth{0.803000pt}%
\definecolor{currentstroke}{rgb}{0.000000,0.000000,0.000000}%
\pgfsetstrokecolor{currentstroke}%
\pgfsetdash{}{0pt}%
\pgfsys@defobject{currentmarker}{\pgfqpoint{0.000000in}{-0.048611in}}{\pgfqpoint{0.000000in}{0.000000in}}{%
\pgfpathmoveto{\pgfqpoint{0.000000in}{0.000000in}}%
\pgfpathlineto{\pgfqpoint{0.000000in}{-0.048611in}}%
\pgfusepath{stroke,fill}%
}%
\begin{pgfscope}%
\pgfsys@transformshift{5.617483in}{0.330000in}%
\pgfsys@useobject{currentmarker}{}%
\end{pgfscope}%
\end{pgfscope}%
\begin{pgfscope}%
\pgfsetbuttcap%
\pgfsetroundjoin%
\definecolor{currentfill}{rgb}{0.000000,0.000000,0.000000}%
\pgfsetfillcolor{currentfill}%
\pgfsetlinewidth{0.803000pt}%
\definecolor{currentstroke}{rgb}{0.000000,0.000000,0.000000}%
\pgfsetstrokecolor{currentstroke}%
\pgfsetdash{}{0pt}%
\pgfsys@defobject{currentmarker}{\pgfqpoint{0.000000in}{0.000000in}}{\pgfqpoint{0.000000in}{0.048611in}}{%
\pgfpathmoveto{\pgfqpoint{0.000000in}{0.000000in}}%
\pgfpathlineto{\pgfqpoint{0.000000in}{0.048611in}}%
\pgfusepath{stroke,fill}%
}%
\begin{pgfscope}%
\pgfsys@transformshift{5.617483in}{2.640000in}%
\pgfsys@useobject{currentmarker}{}%
\end{pgfscope}%
\end{pgfscope}%
\begin{pgfscope}%
\definecolor{textcolor}{rgb}{0.000000,0.000000,0.000000}%
\pgfsetstrokecolor{textcolor}%
\pgfsetfillcolor{textcolor}%
\pgftext[x=5.617483in,y=0.232778in,,top]{\color{textcolor}\rmfamily\fontsize{10.000000}{12.000000}\selectfont \(\displaystyle {10}\)}%
\end{pgfscope}%
\begin{pgfscope}%
\pgfpathrectangle{\pgfqpoint{1.000000in}{0.330000in}}{\pgfqpoint{6.200000in}{2.310000in}}%
\pgfusepath{clip}%
\pgfsetbuttcap%
\pgfsetroundjoin%
\pgfsetlinewidth{0.803000pt}%
\definecolor{currentstroke}{rgb}{0.000000,0.000000,0.000000}%
\pgfsetstrokecolor{currentstroke}%
\pgfsetdash{{0.800000pt}{1.320000pt}}{0.000000pt}%
\pgfpathmoveto{\pgfqpoint{6.484615in}{0.330000in}}%
\pgfpathlineto{\pgfqpoint{6.484615in}{2.640000in}}%
\pgfusepath{stroke}%
\end{pgfscope}%
\begin{pgfscope}%
\pgfsetbuttcap%
\pgfsetroundjoin%
\definecolor{currentfill}{rgb}{0.000000,0.000000,0.000000}%
\pgfsetfillcolor{currentfill}%
\pgfsetlinewidth{0.803000pt}%
\definecolor{currentstroke}{rgb}{0.000000,0.000000,0.000000}%
\pgfsetstrokecolor{currentstroke}%
\pgfsetdash{}{0pt}%
\pgfsys@defobject{currentmarker}{\pgfqpoint{0.000000in}{-0.048611in}}{\pgfqpoint{0.000000in}{0.000000in}}{%
\pgfpathmoveto{\pgfqpoint{0.000000in}{0.000000in}}%
\pgfpathlineto{\pgfqpoint{0.000000in}{-0.048611in}}%
\pgfusepath{stroke,fill}%
}%
\begin{pgfscope}%
\pgfsys@transformshift{6.484615in}{0.330000in}%
\pgfsys@useobject{currentmarker}{}%
\end{pgfscope}%
\end{pgfscope}%
\begin{pgfscope}%
\pgfsetbuttcap%
\pgfsetroundjoin%
\definecolor{currentfill}{rgb}{0.000000,0.000000,0.000000}%
\pgfsetfillcolor{currentfill}%
\pgfsetlinewidth{0.803000pt}%
\definecolor{currentstroke}{rgb}{0.000000,0.000000,0.000000}%
\pgfsetstrokecolor{currentstroke}%
\pgfsetdash{}{0pt}%
\pgfsys@defobject{currentmarker}{\pgfqpoint{0.000000in}{0.000000in}}{\pgfqpoint{0.000000in}{0.048611in}}{%
\pgfpathmoveto{\pgfqpoint{0.000000in}{0.000000in}}%
\pgfpathlineto{\pgfqpoint{0.000000in}{0.048611in}}%
\pgfusepath{stroke,fill}%
}%
\begin{pgfscope}%
\pgfsys@transformshift{6.484615in}{2.640000in}%
\pgfsys@useobject{currentmarker}{}%
\end{pgfscope}%
\end{pgfscope}%
\begin{pgfscope}%
\definecolor{textcolor}{rgb}{0.000000,0.000000,0.000000}%
\pgfsetstrokecolor{textcolor}%
\pgfsetfillcolor{textcolor}%
\pgftext[x=6.484615in,y=0.232778in,,top]{\color{textcolor}\rmfamily\fontsize{10.000000}{12.000000}\selectfont \(\displaystyle {12}\)}%
\end{pgfscope}%
\begin{pgfscope}%
\pgfpathrectangle{\pgfqpoint{1.000000in}{0.330000in}}{\pgfqpoint{6.200000in}{2.310000in}}%
\pgfusepath{clip}%
\pgfsetbuttcap%
\pgfsetroundjoin%
\pgfsetlinewidth{0.803000pt}%
\definecolor{currentstroke}{rgb}{0.000000,0.000000,0.000000}%
\pgfsetstrokecolor{currentstroke}%
\pgfsetdash{{0.800000pt}{1.320000pt}}{0.000000pt}%
\pgfpathmoveto{\pgfqpoint{1.000000in}{0.858386in}}%
\pgfpathlineto{\pgfqpoint{7.200000in}{0.858386in}}%
\pgfusepath{stroke}%
\end{pgfscope}%
\begin{pgfscope}%
\pgfsetbuttcap%
\pgfsetroundjoin%
\definecolor{currentfill}{rgb}{0.000000,0.000000,0.000000}%
\pgfsetfillcolor{currentfill}%
\pgfsetlinewidth{0.803000pt}%
\definecolor{currentstroke}{rgb}{0.000000,0.000000,0.000000}%
\pgfsetstrokecolor{currentstroke}%
\pgfsetdash{}{0pt}%
\pgfsys@defobject{currentmarker}{\pgfqpoint{-0.048611in}{0.000000in}}{\pgfqpoint{-0.000000in}{0.000000in}}{%
\pgfpathmoveto{\pgfqpoint{-0.000000in}{0.000000in}}%
\pgfpathlineto{\pgfqpoint{-0.048611in}{0.000000in}}%
\pgfusepath{stroke,fill}%
}%
\begin{pgfscope}%
\pgfsys@transformshift{1.000000in}{0.858386in}%
\pgfsys@useobject{currentmarker}{}%
\end{pgfscope}%
\end{pgfscope}%
\begin{pgfscope}%
\pgfsetbuttcap%
\pgfsetroundjoin%
\definecolor{currentfill}{rgb}{0.000000,0.000000,0.000000}%
\pgfsetfillcolor{currentfill}%
\pgfsetlinewidth{0.803000pt}%
\definecolor{currentstroke}{rgb}{0.000000,0.000000,0.000000}%
\pgfsetstrokecolor{currentstroke}%
\pgfsetdash{}{0pt}%
\pgfsys@defobject{currentmarker}{\pgfqpoint{0.000000in}{0.000000in}}{\pgfqpoint{0.048611in}{0.000000in}}{%
\pgfpathmoveto{\pgfqpoint{0.000000in}{0.000000in}}%
\pgfpathlineto{\pgfqpoint{0.048611in}{0.000000in}}%
\pgfusepath{stroke,fill}%
}%
\begin{pgfscope}%
\pgfsys@transformshift{7.200000in}{0.858386in}%
\pgfsys@useobject{currentmarker}{}%
\end{pgfscope}%
\end{pgfscope}%
\begin{pgfscope}%
\definecolor{textcolor}{rgb}{0.000000,0.000000,0.000000}%
\pgfsetstrokecolor{textcolor}%
\pgfsetfillcolor{textcolor}%
\pgftext[x=0.725308in, y=0.810161in, left, base]{\color{textcolor}\rmfamily\fontsize{10.000000}{12.000000}\selectfont \(\displaystyle {\ensuremath{-}6}\)}%
\end{pgfscope}%
\begin{pgfscope}%
\pgfpathrectangle{\pgfqpoint{1.000000in}{0.330000in}}{\pgfqpoint{6.200000in}{2.310000in}}%
\pgfusepath{clip}%
\pgfsetbuttcap%
\pgfsetroundjoin%
\pgfsetlinewidth{0.803000pt}%
\definecolor{currentstroke}{rgb}{0.000000,0.000000,0.000000}%
\pgfsetstrokecolor{currentstroke}%
\pgfsetdash{{0.800000pt}{1.320000pt}}{0.000000pt}%
\pgfpathmoveto{\pgfqpoint{1.000000in}{1.417258in}}%
\pgfpathlineto{\pgfqpoint{7.200000in}{1.417258in}}%
\pgfusepath{stroke}%
\end{pgfscope}%
\begin{pgfscope}%
\pgfsetbuttcap%
\pgfsetroundjoin%
\definecolor{currentfill}{rgb}{0.000000,0.000000,0.000000}%
\pgfsetfillcolor{currentfill}%
\pgfsetlinewidth{0.803000pt}%
\definecolor{currentstroke}{rgb}{0.000000,0.000000,0.000000}%
\pgfsetstrokecolor{currentstroke}%
\pgfsetdash{}{0pt}%
\pgfsys@defobject{currentmarker}{\pgfqpoint{-0.048611in}{0.000000in}}{\pgfqpoint{-0.000000in}{0.000000in}}{%
\pgfpathmoveto{\pgfqpoint{-0.000000in}{0.000000in}}%
\pgfpathlineto{\pgfqpoint{-0.048611in}{0.000000in}}%
\pgfusepath{stroke,fill}%
}%
\begin{pgfscope}%
\pgfsys@transformshift{1.000000in}{1.417258in}%
\pgfsys@useobject{currentmarker}{}%
\end{pgfscope}%
\end{pgfscope}%
\begin{pgfscope}%
\pgfsetbuttcap%
\pgfsetroundjoin%
\definecolor{currentfill}{rgb}{0.000000,0.000000,0.000000}%
\pgfsetfillcolor{currentfill}%
\pgfsetlinewidth{0.803000pt}%
\definecolor{currentstroke}{rgb}{0.000000,0.000000,0.000000}%
\pgfsetstrokecolor{currentstroke}%
\pgfsetdash{}{0pt}%
\pgfsys@defobject{currentmarker}{\pgfqpoint{0.000000in}{0.000000in}}{\pgfqpoint{0.048611in}{0.000000in}}{%
\pgfpathmoveto{\pgfqpoint{0.000000in}{0.000000in}}%
\pgfpathlineto{\pgfqpoint{0.048611in}{0.000000in}}%
\pgfusepath{stroke,fill}%
}%
\begin{pgfscope}%
\pgfsys@transformshift{7.200000in}{1.417258in}%
\pgfsys@useobject{currentmarker}{}%
\end{pgfscope}%
\end{pgfscope}%
\begin{pgfscope}%
\definecolor{textcolor}{rgb}{0.000000,0.000000,0.000000}%
\pgfsetstrokecolor{textcolor}%
\pgfsetfillcolor{textcolor}%
\pgftext[x=0.725308in, y=1.369032in, left, base]{\color{textcolor}\rmfamily\fontsize{10.000000}{12.000000}\selectfont \(\displaystyle {\ensuremath{-}4}\)}%
\end{pgfscope}%
\begin{pgfscope}%
\pgfpathrectangle{\pgfqpoint{1.000000in}{0.330000in}}{\pgfqpoint{6.200000in}{2.310000in}}%
\pgfusepath{clip}%
\pgfsetbuttcap%
\pgfsetroundjoin%
\pgfsetlinewidth{0.803000pt}%
\definecolor{currentstroke}{rgb}{0.000000,0.000000,0.000000}%
\pgfsetstrokecolor{currentstroke}%
\pgfsetdash{{0.800000pt}{1.320000pt}}{0.000000pt}%
\pgfpathmoveto{\pgfqpoint{1.000000in}{1.976129in}}%
\pgfpathlineto{\pgfqpoint{7.200000in}{1.976129in}}%
\pgfusepath{stroke}%
\end{pgfscope}%
\begin{pgfscope}%
\pgfsetbuttcap%
\pgfsetroundjoin%
\definecolor{currentfill}{rgb}{0.000000,0.000000,0.000000}%
\pgfsetfillcolor{currentfill}%
\pgfsetlinewidth{0.803000pt}%
\definecolor{currentstroke}{rgb}{0.000000,0.000000,0.000000}%
\pgfsetstrokecolor{currentstroke}%
\pgfsetdash{}{0pt}%
\pgfsys@defobject{currentmarker}{\pgfqpoint{-0.048611in}{0.000000in}}{\pgfqpoint{-0.000000in}{0.000000in}}{%
\pgfpathmoveto{\pgfqpoint{-0.000000in}{0.000000in}}%
\pgfpathlineto{\pgfqpoint{-0.048611in}{0.000000in}}%
\pgfusepath{stroke,fill}%
}%
\begin{pgfscope}%
\pgfsys@transformshift{1.000000in}{1.976129in}%
\pgfsys@useobject{currentmarker}{}%
\end{pgfscope}%
\end{pgfscope}%
\begin{pgfscope}%
\pgfsetbuttcap%
\pgfsetroundjoin%
\definecolor{currentfill}{rgb}{0.000000,0.000000,0.000000}%
\pgfsetfillcolor{currentfill}%
\pgfsetlinewidth{0.803000pt}%
\definecolor{currentstroke}{rgb}{0.000000,0.000000,0.000000}%
\pgfsetstrokecolor{currentstroke}%
\pgfsetdash{}{0pt}%
\pgfsys@defobject{currentmarker}{\pgfqpoint{0.000000in}{0.000000in}}{\pgfqpoint{0.048611in}{0.000000in}}{%
\pgfpathmoveto{\pgfqpoint{0.000000in}{0.000000in}}%
\pgfpathlineto{\pgfqpoint{0.048611in}{0.000000in}}%
\pgfusepath{stroke,fill}%
}%
\begin{pgfscope}%
\pgfsys@transformshift{7.200000in}{1.976129in}%
\pgfsys@useobject{currentmarker}{}%
\end{pgfscope}%
\end{pgfscope}%
\begin{pgfscope}%
\definecolor{textcolor}{rgb}{0.000000,0.000000,0.000000}%
\pgfsetstrokecolor{textcolor}%
\pgfsetfillcolor{textcolor}%
\pgftext[x=0.725308in, y=1.927904in, left, base]{\color{textcolor}\rmfamily\fontsize{10.000000}{12.000000}\selectfont \(\displaystyle {\ensuremath{-}2}\)}%
\end{pgfscope}%
\begin{pgfscope}%
\pgfpathrectangle{\pgfqpoint{1.000000in}{0.330000in}}{\pgfqpoint{6.200000in}{2.310000in}}%
\pgfusepath{clip}%
\pgfsetbuttcap%
\pgfsetroundjoin%
\pgfsetlinewidth{0.803000pt}%
\definecolor{currentstroke}{rgb}{0.000000,0.000000,0.000000}%
\pgfsetstrokecolor{currentstroke}%
\pgfsetdash{{0.800000pt}{1.320000pt}}{0.000000pt}%
\pgfpathmoveto{\pgfqpoint{1.000000in}{2.535000in}}%
\pgfpathlineto{\pgfqpoint{7.200000in}{2.535000in}}%
\pgfusepath{stroke}%
\end{pgfscope}%
\begin{pgfscope}%
\pgfsetbuttcap%
\pgfsetroundjoin%
\definecolor{currentfill}{rgb}{0.000000,0.000000,0.000000}%
\pgfsetfillcolor{currentfill}%
\pgfsetlinewidth{0.803000pt}%
\definecolor{currentstroke}{rgb}{0.000000,0.000000,0.000000}%
\pgfsetstrokecolor{currentstroke}%
\pgfsetdash{}{0pt}%
\pgfsys@defobject{currentmarker}{\pgfqpoint{-0.048611in}{0.000000in}}{\pgfqpoint{-0.000000in}{0.000000in}}{%
\pgfpathmoveto{\pgfqpoint{-0.000000in}{0.000000in}}%
\pgfpathlineto{\pgfqpoint{-0.048611in}{0.000000in}}%
\pgfusepath{stroke,fill}%
}%
\begin{pgfscope}%
\pgfsys@transformshift{1.000000in}{2.535000in}%
\pgfsys@useobject{currentmarker}{}%
\end{pgfscope}%
\end{pgfscope}%
\begin{pgfscope}%
\pgfsetbuttcap%
\pgfsetroundjoin%
\definecolor{currentfill}{rgb}{0.000000,0.000000,0.000000}%
\pgfsetfillcolor{currentfill}%
\pgfsetlinewidth{0.803000pt}%
\definecolor{currentstroke}{rgb}{0.000000,0.000000,0.000000}%
\pgfsetstrokecolor{currentstroke}%
\pgfsetdash{}{0pt}%
\pgfsys@defobject{currentmarker}{\pgfqpoint{0.000000in}{0.000000in}}{\pgfqpoint{0.048611in}{0.000000in}}{%
\pgfpathmoveto{\pgfqpoint{0.000000in}{0.000000in}}%
\pgfpathlineto{\pgfqpoint{0.048611in}{0.000000in}}%
\pgfusepath{stroke,fill}%
}%
\begin{pgfscope}%
\pgfsys@transformshift{7.200000in}{2.535000in}%
\pgfsys@useobject{currentmarker}{}%
\end{pgfscope}%
\end{pgfscope}%
\begin{pgfscope}%
\definecolor{textcolor}{rgb}{0.000000,0.000000,0.000000}%
\pgfsetstrokecolor{textcolor}%
\pgfsetfillcolor{textcolor}%
\pgftext[x=0.833333in, y=2.486775in, left, base]{\color{textcolor}\rmfamily\fontsize{10.000000}{12.000000}\selectfont \(\displaystyle {0}\)}%
\end{pgfscope}%
\begin{pgfscope}%
\pgfpathrectangle{\pgfqpoint{1.000000in}{0.330000in}}{\pgfqpoint{6.200000in}{2.310000in}}%
\pgfusepath{clip}%
\pgfsetrectcap%
\pgfsetroundjoin%
\pgfsetlinewidth{1.505625pt}%
\definecolor{currentstroke}{rgb}{0.121569,0.466667,0.705882}%
\pgfsetstrokecolor{currentstroke}%
\pgfsetdash{}{0pt}%
\pgfpathmoveto{\pgfqpoint{1.281818in}{2.535000in}}%
\pgfpathlineto{\pgfqpoint{1.281818in}{2.535000in}}%
\pgfpathlineto{\pgfqpoint{1.354079in}{2.509201in}}%
\pgfpathlineto{\pgfqpoint{1.426340in}{2.482989in}}%
\pgfpathlineto{\pgfqpoint{1.498601in}{2.456364in}}%
\pgfpathlineto{\pgfqpoint{1.570862in}{2.429326in}}%
\pgfpathlineto{\pgfqpoint{1.643124in}{2.401875in}}%
\pgfpathlineto{\pgfqpoint{1.715385in}{2.374011in}}%
\pgfpathlineto{\pgfqpoint{1.787646in}{2.345734in}}%
\pgfpathlineto{\pgfqpoint{1.859907in}{2.317044in}}%
\pgfpathlineto{\pgfqpoint{1.932168in}{2.287942in}}%
\pgfpathlineto{\pgfqpoint{2.004429in}{2.258426in}}%
\pgfpathlineto{\pgfqpoint{2.076690in}{2.228497in}}%
\pgfpathlineto{\pgfqpoint{2.148951in}{2.198156in}}%
\pgfpathlineto{\pgfqpoint{2.221212in}{2.167401in}}%
\pgfpathlineto{\pgfqpoint{2.293473in}{2.136234in}}%
\pgfpathlineto{\pgfqpoint{2.365734in}{2.104654in}}%
\pgfpathlineto{\pgfqpoint{2.437995in}{2.072660in}}%
\pgfpathlineto{\pgfqpoint{2.510256in}{2.040254in}}%
\pgfpathlineto{\pgfqpoint{2.582517in}{2.007435in}}%
\pgfpathlineto{\pgfqpoint{2.654779in}{1.974203in}}%
\pgfpathlineto{\pgfqpoint{2.727040in}{1.940558in}}%
\pgfpathlineto{\pgfqpoint{2.799301in}{1.906500in}}%
\pgfpathlineto{\pgfqpoint{2.871562in}{1.872029in}}%
\pgfpathlineto{\pgfqpoint{2.943823in}{1.837145in}}%
\pgfpathlineto{\pgfqpoint{3.016084in}{1.801848in}}%
\pgfpathlineto{\pgfqpoint{3.088345in}{1.766138in}}%
\pgfpathlineto{\pgfqpoint{3.160606in}{1.730015in}}%
\pgfpathlineto{\pgfqpoint{3.232867in}{1.693480in}}%
\pgfpathlineto{\pgfqpoint{3.305128in}{1.656531in}}%
\pgfpathlineto{\pgfqpoint{3.377389in}{1.619169in}}%
\pgfpathlineto{\pgfqpoint{3.449650in}{1.581395in}}%
\pgfpathlineto{\pgfqpoint{3.521911in}{1.543207in}}%
\pgfpathlineto{\pgfqpoint{3.594172in}{1.504607in}}%
\pgfpathlineto{\pgfqpoint{3.666434in}{1.465594in}}%
\pgfpathlineto{\pgfqpoint{3.738695in}{1.426167in}}%
\pgfpathlineto{\pgfqpoint{3.883217in}{1.346214in}}%
\pgfpathlineto{\pgfqpoint{3.955478in}{1.382980in}}%
\pgfpathlineto{\pgfqpoint{4.027739in}{1.419334in}}%
\pgfpathlineto{\pgfqpoint{4.100000in}{1.455274in}}%
\pgfpathlineto{\pgfqpoint{4.172261in}{1.490802in}}%
\pgfpathlineto{\pgfqpoint{4.244522in}{1.525917in}}%
\pgfpathlineto{\pgfqpoint{4.316783in}{1.560619in}}%
\pgfpathlineto{\pgfqpoint{4.389044in}{1.594907in}}%
\pgfpathlineto{\pgfqpoint{4.461305in}{1.628783in}}%
\pgfpathlineto{\pgfqpoint{4.533566in}{1.662246in}}%
\pgfpathlineto{\pgfqpoint{4.605828in}{1.695296in}}%
\pgfpathlineto{\pgfqpoint{4.678089in}{1.727933in}}%
\pgfpathlineto{\pgfqpoint{4.750350in}{1.760157in}}%
\pgfpathlineto{\pgfqpoint{4.822611in}{1.791969in}}%
\pgfpathlineto{\pgfqpoint{4.894872in}{1.823367in}}%
\pgfpathlineto{\pgfqpoint{4.967133in}{1.854352in}}%
\pgfpathlineto{\pgfqpoint{5.039394in}{1.884924in}}%
\pgfpathlineto{\pgfqpoint{5.111655in}{1.915084in}}%
\pgfpathlineto{\pgfqpoint{5.183916in}{1.944830in}}%
\pgfpathlineto{\pgfqpoint{5.256177in}{1.974164in}}%
\pgfpathlineto{\pgfqpoint{5.328438in}{2.003084in}}%
\pgfpathlineto{\pgfqpoint{5.400699in}{2.031592in}}%
\pgfpathlineto{\pgfqpoint{5.472960in}{2.059687in}}%
\pgfpathlineto{\pgfqpoint{5.545221in}{2.087369in}}%
\pgfpathlineto{\pgfqpoint{5.617483in}{2.114637in}}%
\pgfpathlineto{\pgfqpoint{5.689744in}{2.141493in}}%
\pgfpathlineto{\pgfqpoint{5.762005in}{2.167936in}}%
\pgfpathlineto{\pgfqpoint{5.834266in}{2.193966in}}%
\pgfpathlineto{\pgfqpoint{5.906527in}{2.219583in}}%
\pgfpathlineto{\pgfqpoint{5.978788in}{2.244787in}}%
\pgfpathlineto{\pgfqpoint{6.051049in}{2.269578in}}%
\pgfpathlineto{\pgfqpoint{6.123310in}{2.293957in}}%
\pgfpathlineto{\pgfqpoint{6.195571in}{2.317922in}}%
\pgfpathlineto{\pgfqpoint{6.267832in}{2.341474in}}%
\pgfpathlineto{\pgfqpoint{6.340093in}{2.364613in}}%
\pgfpathlineto{\pgfqpoint{6.412354in}{2.387340in}}%
\pgfpathlineto{\pgfqpoint{6.484615in}{2.409653in}}%
\pgfpathlineto{\pgfqpoint{6.556876in}{2.431554in}}%
\pgfpathlineto{\pgfqpoint{6.629138in}{2.453041in}}%
\pgfpathlineto{\pgfqpoint{6.701399in}{2.474116in}}%
\pgfpathlineto{\pgfqpoint{6.773660in}{2.494778in}}%
\pgfpathlineto{\pgfqpoint{6.918182in}{2.535000in}}%
\pgfpathlineto{\pgfqpoint{6.918182in}{2.535000in}}%
\pgfusepath{stroke}%
\end{pgfscope}%
\begin{pgfscope}%
\pgfpathrectangle{\pgfqpoint{1.000000in}{0.330000in}}{\pgfqpoint{6.200000in}{2.310000in}}%
\pgfusepath{clip}%
\pgfsetrectcap%
\pgfsetroundjoin%
\pgfsetlinewidth{1.505625pt}%
\definecolor{currentstroke}{rgb}{1.000000,0.498039,0.054902}%
\pgfsetstrokecolor{currentstroke}%
\pgfsetdash{}{0pt}%
\pgfpathmoveto{\pgfqpoint{1.281818in}{2.535000in}}%
\pgfpathlineto{\pgfqpoint{1.281818in}{2.535000in}}%
\pgfpathlineto{\pgfqpoint{1.354079in}{2.513086in}}%
\pgfpathlineto{\pgfqpoint{1.426340in}{2.490818in}}%
\pgfpathlineto{\pgfqpoint{1.498601in}{2.468196in}}%
\pgfpathlineto{\pgfqpoint{1.570862in}{2.445220in}}%
\pgfpathlineto{\pgfqpoint{1.643124in}{2.421891in}}%
\pgfpathlineto{\pgfqpoint{1.715385in}{2.398207in}}%
\pgfpathlineto{\pgfqpoint{1.787646in}{2.374169in}}%
\pgfpathlineto{\pgfqpoint{1.859907in}{2.349778in}}%
\pgfpathlineto{\pgfqpoint{1.932168in}{2.325032in}}%
\pgfpathlineto{\pgfqpoint{2.004429in}{2.299933in}}%
\pgfpathlineto{\pgfqpoint{2.076690in}{2.274479in}}%
\pgfpathlineto{\pgfqpoint{2.148951in}{2.248672in}}%
\pgfpathlineto{\pgfqpoint{2.221212in}{2.222510in}}%
\pgfpathlineto{\pgfqpoint{2.293473in}{2.195995in}}%
\pgfpathlineto{\pgfqpoint{2.365734in}{2.169126in}}%
\pgfpathlineto{\pgfqpoint{2.437995in}{2.141902in}}%
\pgfpathlineto{\pgfqpoint{2.510256in}{2.114325in}}%
\pgfpathlineto{\pgfqpoint{2.582517in}{2.086394in}}%
\pgfpathlineto{\pgfqpoint{2.654779in}{2.058109in}}%
\pgfpathlineto{\pgfqpoint{2.727040in}{2.029470in}}%
\pgfpathlineto{\pgfqpoint{2.799301in}{2.000477in}}%
\pgfpathlineto{\pgfqpoint{2.871562in}{1.971130in}}%
\pgfpathlineto{\pgfqpoint{2.943823in}{1.941429in}}%
\pgfpathlineto{\pgfqpoint{3.016084in}{1.911374in}}%
\pgfpathlineto{\pgfqpoint{3.088345in}{1.880965in}}%
\pgfpathlineto{\pgfqpoint{3.160606in}{1.850203in}}%
\pgfpathlineto{\pgfqpoint{3.232867in}{1.819086in}}%
\pgfpathlineto{\pgfqpoint{3.305128in}{1.787615in}}%
\pgfpathlineto{\pgfqpoint{3.377389in}{1.755791in}}%
\pgfpathlineto{\pgfqpoint{3.449650in}{1.723612in}}%
\pgfpathlineto{\pgfqpoint{3.521911in}{1.691080in}}%
\pgfpathlineto{\pgfqpoint{3.594172in}{1.658193in}}%
\pgfpathlineto{\pgfqpoint{3.666434in}{1.624953in}}%
\pgfpathlineto{\pgfqpoint{3.738695in}{1.591358in}}%
\pgfpathlineto{\pgfqpoint{3.883217in}{1.523226in}}%
\pgfpathlineto{\pgfqpoint{3.955478in}{1.554569in}}%
\pgfpathlineto{\pgfqpoint{4.027739in}{1.585558in}}%
\pgfpathlineto{\pgfqpoint{4.100000in}{1.616193in}}%
\pgfpathlineto{\pgfqpoint{4.172261in}{1.646474in}}%
\pgfpathlineto{\pgfqpoint{4.244522in}{1.676402in}}%
\pgfpathlineto{\pgfqpoint{4.316783in}{1.705975in}}%
\pgfpathlineto{\pgfqpoint{4.389044in}{1.735194in}}%
\pgfpathlineto{\pgfqpoint{4.461305in}{1.764060in}}%
\pgfpathlineto{\pgfqpoint{4.533566in}{1.792571in}}%
\pgfpathlineto{\pgfqpoint{4.605828in}{1.820729in}}%
\pgfpathlineto{\pgfqpoint{4.678089in}{1.848532in}}%
\pgfpathlineto{\pgfqpoint{4.750350in}{1.875982in}}%
\pgfpathlineto{\pgfqpoint{4.822611in}{1.903077in}}%
\pgfpathlineto{\pgfqpoint{4.894872in}{1.929819in}}%
\pgfpathlineto{\pgfqpoint{4.967133in}{1.956207in}}%
\pgfpathlineto{\pgfqpoint{5.039394in}{1.982241in}}%
\pgfpathlineto{\pgfqpoint{5.111655in}{2.007920in}}%
\pgfpathlineto{\pgfqpoint{5.183916in}{2.033246in}}%
\pgfpathlineto{\pgfqpoint{5.256177in}{2.058218in}}%
\pgfpathlineto{\pgfqpoint{5.328438in}{2.082836in}}%
\pgfpathlineto{\pgfqpoint{5.400699in}{2.107100in}}%
\pgfpathlineto{\pgfqpoint{5.472960in}{2.131010in}}%
\pgfpathlineto{\pgfqpoint{5.545221in}{2.154566in}}%
\pgfpathlineto{\pgfqpoint{5.617483in}{2.177769in}}%
\pgfpathlineto{\pgfqpoint{5.689744in}{2.200617in}}%
\pgfpathlineto{\pgfqpoint{5.762005in}{2.223111in}}%
\pgfpathlineto{\pgfqpoint{5.834266in}{2.245251in}}%
\pgfpathlineto{\pgfqpoint{5.906527in}{2.267038in}}%
\pgfpathlineto{\pgfqpoint{5.978788in}{2.288470in}}%
\pgfpathlineto{\pgfqpoint{6.051049in}{2.309549in}}%
\pgfpathlineto{\pgfqpoint{6.123310in}{2.330273in}}%
\pgfpathlineto{\pgfqpoint{6.195571in}{2.350644in}}%
\pgfpathlineto{\pgfqpoint{6.267832in}{2.370660in}}%
\pgfpathlineto{\pgfqpoint{6.340093in}{2.390323in}}%
\pgfpathlineto{\pgfqpoint{6.412354in}{2.409632in}}%
\pgfpathlineto{\pgfqpoint{6.484615in}{2.428586in}}%
\pgfpathlineto{\pgfqpoint{6.556876in}{2.447187in}}%
\pgfpathlineto{\pgfqpoint{6.629138in}{2.465434in}}%
\pgfpathlineto{\pgfqpoint{6.701399in}{2.483327in}}%
\pgfpathlineto{\pgfqpoint{6.773660in}{2.500866in}}%
\pgfpathlineto{\pgfqpoint{6.918182in}{2.535000in}}%
\pgfpathlineto{\pgfqpoint{6.918182in}{2.535000in}}%
\pgfusepath{stroke}%
\end{pgfscope}%
\begin{pgfscope}%
\pgfpathrectangle{\pgfqpoint{1.000000in}{0.330000in}}{\pgfqpoint{6.200000in}{2.310000in}}%
\pgfusepath{clip}%
\pgfsetrectcap%
\pgfsetroundjoin%
\pgfsetlinewidth{1.505625pt}%
\definecolor{currentstroke}{rgb}{0.172549,0.627451,0.172549}%
\pgfsetstrokecolor{currentstroke}%
\pgfsetdash{}{0pt}%
\pgfpathmoveto{\pgfqpoint{1.281818in}{2.535000in}}%
\pgfpathlineto{\pgfqpoint{1.281818in}{2.535000in}}%
\pgfpathlineto{\pgfqpoint{1.354079in}{2.509201in}}%
\pgfpathlineto{\pgfqpoint{1.426340in}{2.482989in}}%
\pgfpathlineto{\pgfqpoint{1.498601in}{2.456364in}}%
\pgfpathlineto{\pgfqpoint{1.570862in}{2.429326in}}%
\pgfpathlineto{\pgfqpoint{1.643124in}{2.401875in}}%
\pgfpathlineto{\pgfqpoint{1.715385in}{2.374011in}}%
\pgfpathlineto{\pgfqpoint{1.787646in}{2.345734in}}%
\pgfpathlineto{\pgfqpoint{1.859907in}{2.317044in}}%
\pgfpathlineto{\pgfqpoint{1.932168in}{2.287942in}}%
\pgfpathlineto{\pgfqpoint{2.004429in}{2.258426in}}%
\pgfpathlineto{\pgfqpoint{2.076690in}{2.228497in}}%
\pgfpathlineto{\pgfqpoint{2.148951in}{2.198156in}}%
\pgfpathlineto{\pgfqpoint{2.221212in}{2.167401in}}%
\pgfpathlineto{\pgfqpoint{2.293473in}{2.136234in}}%
\pgfpathlineto{\pgfqpoint{2.365734in}{2.104654in}}%
\pgfpathlineto{\pgfqpoint{2.437995in}{2.072660in}}%
\pgfpathlineto{\pgfqpoint{2.510256in}{2.040254in}}%
\pgfpathlineto{\pgfqpoint{2.582517in}{2.007435in}}%
\pgfpathlineto{\pgfqpoint{2.654779in}{1.974203in}}%
\pgfpathlineto{\pgfqpoint{2.727040in}{1.940558in}}%
\pgfpathlineto{\pgfqpoint{2.799301in}{1.906500in}}%
\pgfpathlineto{\pgfqpoint{2.871562in}{1.872029in}}%
\pgfpathlineto{\pgfqpoint{2.943823in}{1.837145in}}%
\pgfpathlineto{\pgfqpoint{3.016084in}{1.801848in}}%
\pgfpathlineto{\pgfqpoint{3.088345in}{1.766138in}}%
\pgfpathlineto{\pgfqpoint{3.160606in}{1.730015in}}%
\pgfpathlineto{\pgfqpoint{3.232867in}{1.693480in}}%
\pgfpathlineto{\pgfqpoint{3.305128in}{1.656531in}}%
\pgfpathlineto{\pgfqpoint{3.377389in}{1.619169in}}%
\pgfpathlineto{\pgfqpoint{3.449650in}{1.581395in}}%
\pgfpathlineto{\pgfqpoint{3.521911in}{1.543207in}}%
\pgfpathlineto{\pgfqpoint{3.594172in}{1.504607in}}%
\pgfpathlineto{\pgfqpoint{3.666434in}{1.465594in}}%
\pgfpathlineto{\pgfqpoint{3.738695in}{1.426167in}}%
\pgfpathlineto{\pgfqpoint{3.883217in}{1.346214in}}%
\pgfpathlineto{\pgfqpoint{3.955478in}{1.382980in}}%
\pgfpathlineto{\pgfqpoint{4.027739in}{1.419334in}}%
\pgfpathlineto{\pgfqpoint{4.100000in}{1.455274in}}%
\pgfpathlineto{\pgfqpoint{4.172261in}{1.490802in}}%
\pgfpathlineto{\pgfqpoint{4.244522in}{1.525917in}}%
\pgfpathlineto{\pgfqpoint{4.316783in}{1.560619in}}%
\pgfpathlineto{\pgfqpoint{4.389044in}{1.594907in}}%
\pgfpathlineto{\pgfqpoint{4.461305in}{1.628783in}}%
\pgfpathlineto{\pgfqpoint{4.533566in}{1.662246in}}%
\pgfpathlineto{\pgfqpoint{4.605828in}{1.695296in}}%
\pgfpathlineto{\pgfqpoint{4.678089in}{1.727933in}}%
\pgfpathlineto{\pgfqpoint{4.750350in}{1.760157in}}%
\pgfpathlineto{\pgfqpoint{4.822611in}{1.791969in}}%
\pgfpathlineto{\pgfqpoint{4.894872in}{1.823367in}}%
\pgfpathlineto{\pgfqpoint{4.967133in}{1.854352in}}%
\pgfpathlineto{\pgfqpoint{5.039394in}{1.884924in}}%
\pgfpathlineto{\pgfqpoint{5.111655in}{1.915084in}}%
\pgfpathlineto{\pgfqpoint{5.183916in}{1.944830in}}%
\pgfpathlineto{\pgfqpoint{5.256177in}{1.974164in}}%
\pgfpathlineto{\pgfqpoint{5.328438in}{2.003084in}}%
\pgfpathlineto{\pgfqpoint{5.400699in}{2.031592in}}%
\pgfpathlineto{\pgfqpoint{5.472960in}{2.059687in}}%
\pgfpathlineto{\pgfqpoint{5.545221in}{2.087369in}}%
\pgfpathlineto{\pgfqpoint{5.617483in}{2.114637in}}%
\pgfpathlineto{\pgfqpoint{5.689744in}{2.141493in}}%
\pgfpathlineto{\pgfqpoint{5.762005in}{2.167936in}}%
\pgfpathlineto{\pgfqpoint{5.834266in}{2.193966in}}%
\pgfpathlineto{\pgfqpoint{5.906527in}{2.219583in}}%
\pgfpathlineto{\pgfqpoint{5.978788in}{2.244787in}}%
\pgfpathlineto{\pgfqpoint{6.051049in}{2.269578in}}%
\pgfpathlineto{\pgfqpoint{6.123310in}{2.293957in}}%
\pgfpathlineto{\pgfqpoint{6.195571in}{2.317922in}}%
\pgfpathlineto{\pgfqpoint{6.267832in}{2.341474in}}%
\pgfpathlineto{\pgfqpoint{6.340093in}{2.364613in}}%
\pgfpathlineto{\pgfqpoint{6.412354in}{2.387340in}}%
\pgfpathlineto{\pgfqpoint{6.484615in}{2.409653in}}%
\pgfpathlineto{\pgfqpoint{6.556876in}{2.431554in}}%
\pgfpathlineto{\pgfqpoint{6.629138in}{2.453041in}}%
\pgfpathlineto{\pgfqpoint{6.701399in}{2.474116in}}%
\pgfpathlineto{\pgfqpoint{6.773660in}{2.494778in}}%
\pgfpathlineto{\pgfqpoint{6.918182in}{2.535000in}}%
\pgfpathlineto{\pgfqpoint{6.918182in}{2.535000in}}%
\pgfusepath{stroke}%
\end{pgfscope}%
\begin{pgfscope}%
\pgfpathrectangle{\pgfqpoint{1.000000in}{0.330000in}}{\pgfqpoint{6.200000in}{2.310000in}}%
\pgfusepath{clip}%
\pgfsetrectcap%
\pgfsetroundjoin%
\pgfsetlinewidth{1.505625pt}%
\definecolor{currentstroke}{rgb}{0.839216,0.152941,0.156863}%
\pgfsetstrokecolor{currentstroke}%
\pgfsetdash{}{0pt}%
\pgfpathmoveto{\pgfqpoint{1.281818in}{2.535000in}}%
\pgfpathlineto{\pgfqpoint{1.281818in}{2.535000in}}%
\pgfpathlineto{\pgfqpoint{1.354079in}{2.512341in}}%
\pgfpathlineto{\pgfqpoint{1.426340in}{2.489328in}}%
\pgfpathlineto{\pgfqpoint{1.498601in}{2.465961in}}%
\pgfpathlineto{\pgfqpoint{1.570862in}{2.442240in}}%
\pgfpathlineto{\pgfqpoint{1.643124in}{2.418165in}}%
\pgfpathlineto{\pgfqpoint{1.715385in}{2.393736in}}%
\pgfpathlineto{\pgfqpoint{1.787646in}{2.368953in}}%
\pgfpathlineto{\pgfqpoint{1.859907in}{2.343816in}}%
\pgfpathlineto{\pgfqpoint{1.932168in}{2.318326in}}%
\pgfpathlineto{\pgfqpoint{2.004429in}{2.292481in}}%
\pgfpathlineto{\pgfqpoint{2.076690in}{2.266282in}}%
\pgfpathlineto{\pgfqpoint{2.148951in}{2.239730in}}%
\pgfpathlineto{\pgfqpoint{2.221212in}{2.212823in}}%
\pgfpathlineto{\pgfqpoint{2.293473in}{2.185563in}}%
\pgfpathlineto{\pgfqpoint{2.365734in}{2.157948in}}%
\pgfpathlineto{\pgfqpoint{2.437995in}{2.129980in}}%
\pgfpathlineto{\pgfqpoint{2.510256in}{2.101657in}}%
\pgfpathlineto{\pgfqpoint{2.582517in}{2.072981in}}%
\pgfpathlineto{\pgfqpoint{2.654779in}{2.043951in}}%
\pgfpathlineto{\pgfqpoint{2.727040in}{2.014567in}}%
\pgfpathlineto{\pgfqpoint{2.799301in}{1.984829in}}%
\pgfpathlineto{\pgfqpoint{2.871562in}{1.954736in}}%
\pgfpathlineto{\pgfqpoint{2.943823in}{1.924290in}}%
\pgfpathlineto{\pgfqpoint{3.016084in}{1.893490in}}%
\pgfpathlineto{\pgfqpoint{3.088345in}{1.862336in}}%
\pgfpathlineto{\pgfqpoint{3.160606in}{1.830828in}}%
\pgfpathlineto{\pgfqpoint{3.232867in}{1.798967in}}%
\pgfpathlineto{\pgfqpoint{3.305128in}{1.766751in}}%
\pgfpathlineto{\pgfqpoint{3.377389in}{1.734181in}}%
\pgfpathlineto{\pgfqpoint{3.449650in}{1.701257in}}%
\pgfpathlineto{\pgfqpoint{3.521911in}{1.667980in}}%
\pgfpathlineto{\pgfqpoint{3.594172in}{1.634348in}}%
\pgfpathlineto{\pgfqpoint{3.666434in}{1.600362in}}%
\pgfpathlineto{\pgfqpoint{3.738695in}{1.566023in}}%
\pgfpathlineto{\pgfqpoint{3.883217in}{1.496400in}}%
\pgfpathlineto{\pgfqpoint{3.955478in}{1.528382in}}%
\pgfpathlineto{\pgfqpoint{4.027739in}{1.560010in}}%
\pgfpathlineto{\pgfqpoint{4.100000in}{1.591283in}}%
\pgfpathlineto{\pgfqpoint{4.172261in}{1.622203in}}%
\pgfpathlineto{\pgfqpoint{4.244522in}{1.652769in}}%
\pgfpathlineto{\pgfqpoint{4.316783in}{1.682981in}}%
\pgfpathlineto{\pgfqpoint{4.389044in}{1.712839in}}%
\pgfpathlineto{\pgfqpoint{4.461305in}{1.742343in}}%
\pgfpathlineto{\pgfqpoint{4.533566in}{1.771494in}}%
\pgfpathlineto{\pgfqpoint{4.605828in}{1.800290in}}%
\pgfpathlineto{\pgfqpoint{4.678089in}{1.828732in}}%
\pgfpathlineto{\pgfqpoint{4.750350in}{1.856820in}}%
\pgfpathlineto{\pgfqpoint{4.822611in}{1.884555in}}%
\pgfpathlineto{\pgfqpoint{4.894872in}{1.911935in}}%
\pgfpathlineto{\pgfqpoint{4.967133in}{1.938962in}}%
\pgfpathlineto{\pgfqpoint{5.039394in}{1.965634in}}%
\pgfpathlineto{\pgfqpoint{5.111655in}{1.991953in}}%
\pgfpathlineto{\pgfqpoint{5.183916in}{2.017917in}}%
\pgfpathlineto{\pgfqpoint{5.256177in}{2.043528in}}%
\pgfpathlineto{\pgfqpoint{5.328438in}{2.068785in}}%
\pgfpathlineto{\pgfqpoint{5.400699in}{2.093687in}}%
\pgfpathlineto{\pgfqpoint{5.472960in}{2.118236in}}%
\pgfpathlineto{\pgfqpoint{5.545221in}{2.142431in}}%
\pgfpathlineto{\pgfqpoint{5.617483in}{2.166272in}}%
\pgfpathlineto{\pgfqpoint{5.689744in}{2.189759in}}%
\pgfpathlineto{\pgfqpoint{5.762005in}{2.212892in}}%
\pgfpathlineto{\pgfqpoint{5.834266in}{2.235671in}}%
\pgfpathlineto{\pgfqpoint{5.906527in}{2.258096in}}%
\pgfpathlineto{\pgfqpoint{5.978788in}{2.280167in}}%
\pgfpathlineto{\pgfqpoint{6.051049in}{2.301884in}}%
\pgfpathlineto{\pgfqpoint{6.123310in}{2.323247in}}%
\pgfpathlineto{\pgfqpoint{6.195571in}{2.344257in}}%
\pgfpathlineto{\pgfqpoint{6.267832in}{2.364912in}}%
\pgfpathlineto{\pgfqpoint{6.340093in}{2.385213in}}%
\pgfpathlineto{\pgfqpoint{6.412354in}{2.405161in}}%
\pgfpathlineto{\pgfqpoint{6.484615in}{2.424754in}}%
\pgfpathlineto{\pgfqpoint{6.556876in}{2.443994in}}%
\pgfpathlineto{\pgfqpoint{6.629138in}{2.462879in}}%
\pgfpathlineto{\pgfqpoint{6.701399in}{2.481411in}}%
\pgfpathlineto{\pgfqpoint{6.773660in}{2.499589in}}%
\pgfpathlineto{\pgfqpoint{6.918182in}{2.535000in}}%
\pgfpathlineto{\pgfqpoint{6.918182in}{2.535000in}}%
\pgfusepath{stroke}%
\end{pgfscope}%
\begin{pgfscope}%
\pgfpathrectangle{\pgfqpoint{1.000000in}{0.330000in}}{\pgfqpoint{6.200000in}{2.310000in}}%
\pgfusepath{clip}%
\pgfsetrectcap%
\pgfsetroundjoin%
\pgfsetlinewidth{1.505625pt}%
\definecolor{currentstroke}{rgb}{0.580392,0.403922,0.741176}%
\pgfsetstrokecolor{currentstroke}%
\pgfsetdash{}{0pt}%
\pgfpathmoveto{\pgfqpoint{1.281818in}{2.535000in}}%
\pgfpathlineto{\pgfqpoint{1.281818in}{2.535000in}}%
\pgfpathlineto{\pgfqpoint{1.354079in}{2.522217in}}%
\pgfpathlineto{\pgfqpoint{1.426340in}{2.509227in}}%
\pgfpathlineto{\pgfqpoint{1.498601in}{2.496031in}}%
\pgfpathlineto{\pgfqpoint{1.570862in}{2.482629in}}%
\pgfpathlineto{\pgfqpoint{1.643124in}{2.469020in}}%
\pgfpathlineto{\pgfqpoint{1.715385in}{2.455204in}}%
\pgfpathlineto{\pgfqpoint{1.787646in}{2.441182in}}%
\pgfpathlineto{\pgfqpoint{1.859907in}{2.426954in}}%
\pgfpathlineto{\pgfqpoint{1.932168in}{2.412519in}}%
\pgfpathlineto{\pgfqpoint{2.004429in}{2.397877in}}%
\pgfpathlineto{\pgfqpoint{2.076690in}{2.383029in}}%
\pgfpathlineto{\pgfqpoint{2.148951in}{2.367975in}}%
\pgfpathlineto{\pgfqpoint{2.221212in}{2.352714in}}%
\pgfpathlineto{\pgfqpoint{2.293473in}{2.337247in}}%
\pgfpathlineto{\pgfqpoint{2.365734in}{2.321573in}}%
\pgfpathlineto{\pgfqpoint{2.437995in}{2.305693in}}%
\pgfpathlineto{\pgfqpoint{2.510256in}{2.289606in}}%
\pgfpathlineto{\pgfqpoint{2.582517in}{2.273313in}}%
\pgfpathlineto{\pgfqpoint{2.654779in}{2.256814in}}%
\pgfpathlineto{\pgfqpoint{2.727040in}{2.240107in}}%
\pgfpathlineto{\pgfqpoint{2.799301in}{2.223195in}}%
\pgfpathlineto{\pgfqpoint{2.871562in}{2.206076in}}%
\pgfpathlineto{\pgfqpoint{2.943823in}{2.188750in}}%
\pgfpathlineto{\pgfqpoint{3.016084in}{2.171218in}}%
\pgfpathlineto{\pgfqpoint{3.088345in}{2.153480in}}%
\pgfpathlineto{\pgfqpoint{3.160606in}{2.135535in}}%
\pgfpathlineto{\pgfqpoint{3.232867in}{2.117383in}}%
\pgfpathlineto{\pgfqpoint{3.305128in}{2.099026in}}%
\pgfpathlineto{\pgfqpoint{3.377389in}{2.080461in}}%
\pgfpathlineto{\pgfqpoint{3.449650in}{2.061690in}}%
\pgfpathlineto{\pgfqpoint{3.521911in}{2.042713in}}%
\pgfpathlineto{\pgfqpoint{3.594172in}{2.023529in}}%
\pgfpathlineto{\pgfqpoint{3.666434in}{2.004139in}}%
\pgfpathlineto{\pgfqpoint{3.738695in}{1.984542in}}%
\pgfpathlineto{\pgfqpoint{3.883217in}{1.944798in}}%
\pgfpathlineto{\pgfqpoint{3.955478in}{1.963082in}}%
\pgfpathlineto{\pgfqpoint{4.027739in}{1.981159in}}%
\pgfpathlineto{\pgfqpoint{4.100000in}{1.999029in}}%
\pgfpathlineto{\pgfqpoint{4.172261in}{2.016693in}}%
\pgfpathlineto{\pgfqpoint{4.244522in}{2.034151in}}%
\pgfpathlineto{\pgfqpoint{4.316783in}{2.051402in}}%
\pgfpathlineto{\pgfqpoint{4.389044in}{2.068447in}}%
\pgfpathlineto{\pgfqpoint{4.461305in}{2.085285in}}%
\pgfpathlineto{\pgfqpoint{4.533566in}{2.101916in}}%
\pgfpathlineto{\pgfqpoint{4.605828in}{2.118342in}}%
\pgfpathlineto{\pgfqpoint{4.678089in}{2.134560in}}%
\pgfpathlineto{\pgfqpoint{4.750350in}{2.150573in}}%
\pgfpathlineto{\pgfqpoint{4.822611in}{2.166378in}}%
\pgfpathlineto{\pgfqpoint{4.894872in}{2.181978in}}%
\pgfpathlineto{\pgfqpoint{4.967133in}{2.197371in}}%
\pgfpathlineto{\pgfqpoint{5.039394in}{2.212557in}}%
\pgfpathlineto{\pgfqpoint{5.111655in}{2.227537in}}%
\pgfpathlineto{\pgfqpoint{5.183916in}{2.242310in}}%
\pgfpathlineto{\pgfqpoint{5.256177in}{2.256877in}}%
\pgfpathlineto{\pgfqpoint{5.328438in}{2.271238in}}%
\pgfpathlineto{\pgfqpoint{5.400699in}{2.285392in}}%
\pgfpathlineto{\pgfqpoint{5.472960in}{2.299339in}}%
\pgfpathlineto{\pgfqpoint{5.545221in}{2.313080in}}%
\pgfpathlineto{\pgfqpoint{5.617483in}{2.326615in}}%
\pgfpathlineto{\pgfqpoint{5.689744in}{2.339943in}}%
\pgfpathlineto{\pgfqpoint{5.762005in}{2.353065in}}%
\pgfpathlineto{\pgfqpoint{5.834266in}{2.365980in}}%
\pgfpathlineto{\pgfqpoint{5.906527in}{2.378689in}}%
\pgfpathlineto{\pgfqpoint{5.978788in}{2.391191in}}%
\pgfpathlineto{\pgfqpoint{6.051049in}{2.403487in}}%
\pgfpathlineto{\pgfqpoint{6.123310in}{2.415576in}}%
\pgfpathlineto{\pgfqpoint{6.195571in}{2.427459in}}%
\pgfpathlineto{\pgfqpoint{6.267832in}{2.439135in}}%
\pgfpathlineto{\pgfqpoint{6.340093in}{2.450605in}}%
\pgfpathlineto{\pgfqpoint{6.412354in}{2.461869in}}%
\pgfpathlineto{\pgfqpoint{6.484615in}{2.472925in}}%
\pgfpathlineto{\pgfqpoint{6.556876in}{2.483776in}}%
\pgfpathlineto{\pgfqpoint{6.629138in}{2.494420in}}%
\pgfpathlineto{\pgfqpoint{6.701399in}{2.504857in}}%
\pgfpathlineto{\pgfqpoint{6.773660in}{2.515088in}}%
\pgfpathlineto{\pgfqpoint{6.918182in}{2.535000in}}%
\pgfpathlineto{\pgfqpoint{6.918182in}{2.535000in}}%
\pgfusepath{stroke}%
\end{pgfscope}%
\begin{pgfscope}%
\pgfpathrectangle{\pgfqpoint{1.000000in}{0.330000in}}{\pgfqpoint{6.200000in}{2.310000in}}%
\pgfusepath{clip}%
\pgfsetrectcap%
\pgfsetroundjoin%
\pgfsetlinewidth{1.505625pt}%
\definecolor{currentstroke}{rgb}{0.549020,0.337255,0.294118}%
\pgfsetstrokecolor{currentstroke}%
\pgfsetdash{}{0pt}%
\pgfpathmoveto{\pgfqpoint{1.281818in}{2.535000in}}%
\pgfpathlineto{\pgfqpoint{1.281818in}{2.535000in}}%
\pgfpathlineto{\pgfqpoint{1.354079in}{2.512620in}}%
\pgfpathlineto{\pgfqpoint{1.426340in}{2.489887in}}%
\pgfpathlineto{\pgfqpoint{1.498601in}{2.466799in}}%
\pgfpathlineto{\pgfqpoint{1.570862in}{2.443358in}}%
\pgfpathlineto{\pgfqpoint{1.643124in}{2.419562in}}%
\pgfpathlineto{\pgfqpoint{1.715385in}{2.395413in}}%
\pgfpathlineto{\pgfqpoint{1.787646in}{2.370909in}}%
\pgfpathlineto{\pgfqpoint{1.859907in}{2.346052in}}%
\pgfpathlineto{\pgfqpoint{1.932168in}{2.320841in}}%
\pgfpathlineto{\pgfqpoint{2.004429in}{2.295275in}}%
\pgfpathlineto{\pgfqpoint{2.076690in}{2.269356in}}%
\pgfpathlineto{\pgfqpoint{2.148951in}{2.243083in}}%
\pgfpathlineto{\pgfqpoint{2.221212in}{2.216456in}}%
\pgfpathlineto{\pgfqpoint{2.293473in}{2.189475in}}%
\pgfpathlineto{\pgfqpoint{2.365734in}{2.162140in}}%
\pgfpathlineto{\pgfqpoint{2.437995in}{2.134451in}}%
\pgfpathlineto{\pgfqpoint{2.510256in}{2.106408in}}%
\pgfpathlineto{\pgfqpoint{2.582517in}{2.078011in}}%
\pgfpathlineto{\pgfqpoint{2.654779in}{2.049260in}}%
\pgfpathlineto{\pgfqpoint{2.727040in}{2.020155in}}%
\pgfpathlineto{\pgfqpoint{2.799301in}{1.990697in}}%
\pgfpathlineto{\pgfqpoint{2.871562in}{1.960884in}}%
\pgfpathlineto{\pgfqpoint{2.943823in}{1.930717in}}%
\pgfpathlineto{\pgfqpoint{3.016084in}{1.900197in}}%
\pgfpathlineto{\pgfqpoint{3.088345in}{1.869322in}}%
\pgfpathlineto{\pgfqpoint{3.160606in}{1.838094in}}%
\pgfpathlineto{\pgfqpoint{3.232867in}{1.806511in}}%
\pgfpathlineto{\pgfqpoint{3.305128in}{1.774575in}}%
\pgfpathlineto{\pgfqpoint{3.377389in}{1.742285in}}%
\pgfpathlineto{\pgfqpoint{3.449650in}{1.709640in}}%
\pgfpathlineto{\pgfqpoint{3.521911in}{1.676642in}}%
\pgfpathlineto{\pgfqpoint{3.594172in}{1.643290in}}%
\pgfpathlineto{\pgfqpoint{3.666434in}{1.609584in}}%
\pgfpathlineto{\pgfqpoint{3.738695in}{1.575524in}}%
\pgfpathlineto{\pgfqpoint{3.883217in}{1.506460in}}%
\pgfpathlineto{\pgfqpoint{3.955478in}{1.538202in}}%
\pgfpathlineto{\pgfqpoint{4.027739in}{1.569590in}}%
\pgfpathlineto{\pgfqpoint{4.100000in}{1.600625in}}%
\pgfpathlineto{\pgfqpoint{4.172261in}{1.631305in}}%
\pgfpathlineto{\pgfqpoint{4.244522in}{1.661631in}}%
\pgfpathlineto{\pgfqpoint{4.316783in}{1.691604in}}%
\pgfpathlineto{\pgfqpoint{4.389044in}{1.721222in}}%
\pgfpathlineto{\pgfqpoint{4.461305in}{1.750487in}}%
\pgfpathlineto{\pgfqpoint{4.533566in}{1.779398in}}%
\pgfpathlineto{\pgfqpoint{4.605828in}{1.807954in}}%
\pgfpathlineto{\pgfqpoint{4.678089in}{1.836157in}}%
\pgfpathlineto{\pgfqpoint{4.750350in}{1.864006in}}%
\pgfpathlineto{\pgfqpoint{4.822611in}{1.891501in}}%
\pgfpathlineto{\pgfqpoint{4.894872in}{1.918642in}}%
\pgfpathlineto{\pgfqpoint{4.967133in}{1.945429in}}%
\pgfpathlineto{\pgfqpoint{5.039394in}{1.971862in}}%
\pgfpathlineto{\pgfqpoint{5.111655in}{1.997941in}}%
\pgfpathlineto{\pgfqpoint{5.183916in}{2.023666in}}%
\pgfpathlineto{\pgfqpoint{5.256177in}{2.049037in}}%
\pgfpathlineto{\pgfqpoint{5.328438in}{2.074054in}}%
\pgfpathlineto{\pgfqpoint{5.400699in}{2.098717in}}%
\pgfpathlineto{\pgfqpoint{5.472960in}{2.123026in}}%
\pgfpathlineto{\pgfqpoint{5.545221in}{2.146982in}}%
\pgfpathlineto{\pgfqpoint{5.617483in}{2.170583in}}%
\pgfpathlineto{\pgfqpoint{5.689744in}{2.193831in}}%
\pgfpathlineto{\pgfqpoint{5.762005in}{2.216724in}}%
\pgfpathlineto{\pgfqpoint{5.834266in}{2.239264in}}%
\pgfpathlineto{\pgfqpoint{5.906527in}{2.261449in}}%
\pgfpathlineto{\pgfqpoint{5.978788in}{2.283281in}}%
\pgfpathlineto{\pgfqpoint{6.051049in}{2.304758in}}%
\pgfpathlineto{\pgfqpoint{6.123310in}{2.325882in}}%
\pgfpathlineto{\pgfqpoint{6.195571in}{2.346652in}}%
\pgfpathlineto{\pgfqpoint{6.267832in}{2.367068in}}%
\pgfpathlineto{\pgfqpoint{6.340093in}{2.387129in}}%
\pgfpathlineto{\pgfqpoint{6.412354in}{2.406837in}}%
\pgfpathlineto{\pgfqpoint{6.484615in}{2.426191in}}%
\pgfpathlineto{\pgfqpoint{6.556876in}{2.445191in}}%
\pgfpathlineto{\pgfqpoint{6.629138in}{2.463837in}}%
\pgfpathlineto{\pgfqpoint{6.701399in}{2.482129in}}%
\pgfpathlineto{\pgfqpoint{6.773660in}{2.500068in}}%
\pgfpathlineto{\pgfqpoint{6.918182in}{2.535000in}}%
\pgfpathlineto{\pgfqpoint{6.918182in}{2.535000in}}%
\pgfusepath{stroke}%
\end{pgfscope}%
\begin{pgfscope}%
\pgfpathrectangle{\pgfqpoint{1.000000in}{0.330000in}}{\pgfqpoint{6.200000in}{2.310000in}}%
\pgfusepath{clip}%
\pgfsetrectcap%
\pgfsetroundjoin%
\pgfsetlinewidth{1.505625pt}%
\definecolor{currentstroke}{rgb}{0.890196,0.466667,0.760784}%
\pgfsetstrokecolor{currentstroke}%
\pgfsetdash{}{0pt}%
\pgfpathmoveto{\pgfqpoint{1.281818in}{2.535000in}}%
\pgfpathlineto{\pgfqpoint{1.281818in}{2.535000in}}%
\pgfpathlineto{\pgfqpoint{1.354079in}{2.509201in}}%
\pgfpathlineto{\pgfqpoint{1.426340in}{2.482989in}}%
\pgfpathlineto{\pgfqpoint{1.498601in}{2.456364in}}%
\pgfpathlineto{\pgfqpoint{1.570862in}{2.429326in}}%
\pgfpathlineto{\pgfqpoint{1.643124in}{2.401875in}}%
\pgfpathlineto{\pgfqpoint{1.715385in}{2.374011in}}%
\pgfpathlineto{\pgfqpoint{1.787646in}{2.345734in}}%
\pgfpathlineto{\pgfqpoint{1.859907in}{2.317044in}}%
\pgfpathlineto{\pgfqpoint{1.932168in}{2.287942in}}%
\pgfpathlineto{\pgfqpoint{2.004429in}{2.258426in}}%
\pgfpathlineto{\pgfqpoint{2.076690in}{2.228497in}}%
\pgfpathlineto{\pgfqpoint{2.148951in}{2.198156in}}%
\pgfpathlineto{\pgfqpoint{2.221212in}{2.167401in}}%
\pgfpathlineto{\pgfqpoint{2.293473in}{2.136234in}}%
\pgfpathlineto{\pgfqpoint{2.365734in}{2.104654in}}%
\pgfpathlineto{\pgfqpoint{2.437995in}{2.072660in}}%
\pgfpathlineto{\pgfqpoint{2.510256in}{2.040254in}}%
\pgfpathlineto{\pgfqpoint{2.582517in}{2.007435in}}%
\pgfpathlineto{\pgfqpoint{2.654779in}{1.974203in}}%
\pgfpathlineto{\pgfqpoint{2.727040in}{1.940558in}}%
\pgfpathlineto{\pgfqpoint{2.799301in}{1.906500in}}%
\pgfpathlineto{\pgfqpoint{2.871562in}{1.872029in}}%
\pgfpathlineto{\pgfqpoint{2.943823in}{1.837145in}}%
\pgfpathlineto{\pgfqpoint{3.016084in}{1.801848in}}%
\pgfpathlineto{\pgfqpoint{3.088345in}{1.766138in}}%
\pgfpathlineto{\pgfqpoint{3.160606in}{1.730015in}}%
\pgfpathlineto{\pgfqpoint{3.232867in}{1.693480in}}%
\pgfpathlineto{\pgfqpoint{3.305128in}{1.656531in}}%
\pgfpathlineto{\pgfqpoint{3.377389in}{1.619169in}}%
\pgfpathlineto{\pgfqpoint{3.449650in}{1.581395in}}%
\pgfpathlineto{\pgfqpoint{3.521911in}{1.543207in}}%
\pgfpathlineto{\pgfqpoint{3.594172in}{1.504607in}}%
\pgfpathlineto{\pgfqpoint{3.666434in}{1.465594in}}%
\pgfpathlineto{\pgfqpoint{3.738695in}{1.426167in}}%
\pgfpathlineto{\pgfqpoint{3.883217in}{1.346214in}}%
\pgfpathlineto{\pgfqpoint{3.955478in}{1.382980in}}%
\pgfpathlineto{\pgfqpoint{4.027739in}{1.419334in}}%
\pgfpathlineto{\pgfqpoint{4.100000in}{1.455274in}}%
\pgfpathlineto{\pgfqpoint{4.172261in}{1.490802in}}%
\pgfpathlineto{\pgfqpoint{4.244522in}{1.525917in}}%
\pgfpathlineto{\pgfqpoint{4.316783in}{1.560619in}}%
\pgfpathlineto{\pgfqpoint{4.389044in}{1.594907in}}%
\pgfpathlineto{\pgfqpoint{4.461305in}{1.628783in}}%
\pgfpathlineto{\pgfqpoint{4.533566in}{1.662246in}}%
\pgfpathlineto{\pgfqpoint{4.605828in}{1.695296in}}%
\pgfpathlineto{\pgfqpoint{4.678089in}{1.727933in}}%
\pgfpathlineto{\pgfqpoint{4.750350in}{1.760157in}}%
\pgfpathlineto{\pgfqpoint{4.822611in}{1.791969in}}%
\pgfpathlineto{\pgfqpoint{4.894872in}{1.823367in}}%
\pgfpathlineto{\pgfqpoint{4.967133in}{1.854352in}}%
\pgfpathlineto{\pgfqpoint{5.039394in}{1.884924in}}%
\pgfpathlineto{\pgfqpoint{5.111655in}{1.915084in}}%
\pgfpathlineto{\pgfqpoint{5.183916in}{1.944830in}}%
\pgfpathlineto{\pgfqpoint{5.256177in}{1.974164in}}%
\pgfpathlineto{\pgfqpoint{5.328438in}{2.003084in}}%
\pgfpathlineto{\pgfqpoint{5.400699in}{2.031592in}}%
\pgfpathlineto{\pgfqpoint{5.472960in}{2.059687in}}%
\pgfpathlineto{\pgfqpoint{5.545221in}{2.087369in}}%
\pgfpathlineto{\pgfqpoint{5.617483in}{2.114637in}}%
\pgfpathlineto{\pgfqpoint{5.689744in}{2.141493in}}%
\pgfpathlineto{\pgfqpoint{5.762005in}{2.167936in}}%
\pgfpathlineto{\pgfqpoint{5.834266in}{2.193966in}}%
\pgfpathlineto{\pgfqpoint{5.906527in}{2.219583in}}%
\pgfpathlineto{\pgfqpoint{5.978788in}{2.244787in}}%
\pgfpathlineto{\pgfqpoint{6.051049in}{2.269578in}}%
\pgfpathlineto{\pgfqpoint{6.123310in}{2.293957in}}%
\pgfpathlineto{\pgfqpoint{6.195571in}{2.317922in}}%
\pgfpathlineto{\pgfqpoint{6.267832in}{2.341474in}}%
\pgfpathlineto{\pgfqpoint{6.340093in}{2.364613in}}%
\pgfpathlineto{\pgfqpoint{6.412354in}{2.387340in}}%
\pgfpathlineto{\pgfqpoint{6.484615in}{2.409653in}}%
\pgfpathlineto{\pgfqpoint{6.556876in}{2.431554in}}%
\pgfpathlineto{\pgfqpoint{6.629138in}{2.453041in}}%
\pgfpathlineto{\pgfqpoint{6.701399in}{2.474116in}}%
\pgfpathlineto{\pgfqpoint{6.773660in}{2.494778in}}%
\pgfpathlineto{\pgfqpoint{6.918182in}{2.535000in}}%
\pgfpathlineto{\pgfqpoint{6.918182in}{2.535000in}}%
\pgfusepath{stroke}%
\end{pgfscope}%
\begin{pgfscope}%
\pgfpathrectangle{\pgfqpoint{1.000000in}{0.330000in}}{\pgfqpoint{6.200000in}{2.310000in}}%
\pgfusepath{clip}%
\pgfsetrectcap%
\pgfsetroundjoin%
\pgfsetlinewidth{1.505625pt}%
\definecolor{currentstroke}{rgb}{0.498039,0.498039,0.498039}%
\pgfsetstrokecolor{currentstroke}%
\pgfsetdash{}{0pt}%
\pgfpathmoveto{\pgfqpoint{1.281818in}{2.535000in}}%
\pgfpathlineto{\pgfqpoint{1.281818in}{2.535000in}}%
\pgfpathlineto{\pgfqpoint{1.354079in}{2.513086in}}%
\pgfpathlineto{\pgfqpoint{1.426340in}{2.490818in}}%
\pgfpathlineto{\pgfqpoint{1.498601in}{2.468196in}}%
\pgfpathlineto{\pgfqpoint{1.570862in}{2.445220in}}%
\pgfpathlineto{\pgfqpoint{1.643124in}{2.421891in}}%
\pgfpathlineto{\pgfqpoint{1.715385in}{2.398207in}}%
\pgfpathlineto{\pgfqpoint{1.787646in}{2.374169in}}%
\pgfpathlineto{\pgfqpoint{1.859907in}{2.349778in}}%
\pgfpathlineto{\pgfqpoint{1.932168in}{2.325032in}}%
\pgfpathlineto{\pgfqpoint{2.004429in}{2.299933in}}%
\pgfpathlineto{\pgfqpoint{2.076690in}{2.274479in}}%
\pgfpathlineto{\pgfqpoint{2.148951in}{2.248672in}}%
\pgfpathlineto{\pgfqpoint{2.221212in}{2.222510in}}%
\pgfpathlineto{\pgfqpoint{2.293473in}{2.195995in}}%
\pgfpathlineto{\pgfqpoint{2.365734in}{2.169126in}}%
\pgfpathlineto{\pgfqpoint{2.437995in}{2.141902in}}%
\pgfpathlineto{\pgfqpoint{2.510256in}{2.114325in}}%
\pgfpathlineto{\pgfqpoint{2.582517in}{2.086394in}}%
\pgfpathlineto{\pgfqpoint{2.654779in}{2.058109in}}%
\pgfpathlineto{\pgfqpoint{2.727040in}{2.029470in}}%
\pgfpathlineto{\pgfqpoint{2.799301in}{2.000477in}}%
\pgfpathlineto{\pgfqpoint{2.871562in}{1.971130in}}%
\pgfpathlineto{\pgfqpoint{2.943823in}{1.941429in}}%
\pgfpathlineto{\pgfqpoint{3.016084in}{1.911374in}}%
\pgfpathlineto{\pgfqpoint{3.088345in}{1.880965in}}%
\pgfpathlineto{\pgfqpoint{3.160606in}{1.850203in}}%
\pgfpathlineto{\pgfqpoint{3.232867in}{1.819086in}}%
\pgfpathlineto{\pgfqpoint{3.305128in}{1.787615in}}%
\pgfpathlineto{\pgfqpoint{3.377389in}{1.755791in}}%
\pgfpathlineto{\pgfqpoint{3.449650in}{1.723612in}}%
\pgfpathlineto{\pgfqpoint{3.521911in}{1.691080in}}%
\pgfpathlineto{\pgfqpoint{3.594172in}{1.658193in}}%
\pgfpathlineto{\pgfqpoint{3.666434in}{1.624953in}}%
\pgfpathlineto{\pgfqpoint{3.738695in}{1.591358in}}%
\pgfpathlineto{\pgfqpoint{3.883217in}{1.523226in}}%
\pgfpathlineto{\pgfqpoint{3.955478in}{1.554569in}}%
\pgfpathlineto{\pgfqpoint{4.027739in}{1.585558in}}%
\pgfpathlineto{\pgfqpoint{4.100000in}{1.616193in}}%
\pgfpathlineto{\pgfqpoint{4.172261in}{1.646474in}}%
\pgfpathlineto{\pgfqpoint{4.244522in}{1.676402in}}%
\pgfpathlineto{\pgfqpoint{4.316783in}{1.705975in}}%
\pgfpathlineto{\pgfqpoint{4.389044in}{1.735194in}}%
\pgfpathlineto{\pgfqpoint{4.461305in}{1.764060in}}%
\pgfpathlineto{\pgfqpoint{4.533566in}{1.792571in}}%
\pgfpathlineto{\pgfqpoint{4.605828in}{1.820729in}}%
\pgfpathlineto{\pgfqpoint{4.678089in}{1.848532in}}%
\pgfpathlineto{\pgfqpoint{4.750350in}{1.875982in}}%
\pgfpathlineto{\pgfqpoint{4.822611in}{1.903077in}}%
\pgfpathlineto{\pgfqpoint{4.894872in}{1.929819in}}%
\pgfpathlineto{\pgfqpoint{4.967133in}{1.956207in}}%
\pgfpathlineto{\pgfqpoint{5.039394in}{1.982241in}}%
\pgfpathlineto{\pgfqpoint{5.111655in}{2.007920in}}%
\pgfpathlineto{\pgfqpoint{5.183916in}{2.033246in}}%
\pgfpathlineto{\pgfqpoint{5.256177in}{2.058218in}}%
\pgfpathlineto{\pgfqpoint{5.328438in}{2.082836in}}%
\pgfpathlineto{\pgfqpoint{5.400699in}{2.107100in}}%
\pgfpathlineto{\pgfqpoint{5.472960in}{2.131010in}}%
\pgfpathlineto{\pgfqpoint{5.545221in}{2.154566in}}%
\pgfpathlineto{\pgfqpoint{5.617483in}{2.177769in}}%
\pgfpathlineto{\pgfqpoint{5.689744in}{2.200617in}}%
\pgfpathlineto{\pgfqpoint{5.762005in}{2.223111in}}%
\pgfpathlineto{\pgfqpoint{5.834266in}{2.245251in}}%
\pgfpathlineto{\pgfqpoint{5.906527in}{2.267038in}}%
\pgfpathlineto{\pgfqpoint{5.978788in}{2.288470in}}%
\pgfpathlineto{\pgfqpoint{6.051049in}{2.309549in}}%
\pgfpathlineto{\pgfqpoint{6.123310in}{2.330273in}}%
\pgfpathlineto{\pgfqpoint{6.195571in}{2.350644in}}%
\pgfpathlineto{\pgfqpoint{6.267832in}{2.370660in}}%
\pgfpathlineto{\pgfqpoint{6.340093in}{2.390323in}}%
\pgfpathlineto{\pgfqpoint{6.412354in}{2.409632in}}%
\pgfpathlineto{\pgfqpoint{6.484615in}{2.428586in}}%
\pgfpathlineto{\pgfqpoint{6.556876in}{2.447187in}}%
\pgfpathlineto{\pgfqpoint{6.629138in}{2.465434in}}%
\pgfpathlineto{\pgfqpoint{6.701399in}{2.483327in}}%
\pgfpathlineto{\pgfqpoint{6.773660in}{2.500866in}}%
\pgfpathlineto{\pgfqpoint{6.918182in}{2.535000in}}%
\pgfpathlineto{\pgfqpoint{6.918182in}{2.535000in}}%
\pgfusepath{stroke}%
\end{pgfscope}%
\begin{pgfscope}%
\pgfpathrectangle{\pgfqpoint{1.000000in}{0.330000in}}{\pgfqpoint{6.200000in}{2.310000in}}%
\pgfusepath{clip}%
\pgfsetrectcap%
\pgfsetroundjoin%
\pgfsetlinewidth{1.505625pt}%
\definecolor{currentstroke}{rgb}{0.737255,0.741176,0.133333}%
\pgfsetstrokecolor{currentstroke}%
\pgfsetdash{}{0pt}%
\pgfpathmoveto{\pgfqpoint{1.281818in}{2.535000in}}%
\pgfpathlineto{\pgfqpoint{1.281818in}{2.535000in}}%
\pgfpathlineto{\pgfqpoint{1.354079in}{2.487668in}}%
\pgfpathlineto{\pgfqpoint{1.426340in}{2.439733in}}%
\pgfpathlineto{\pgfqpoint{1.498601in}{2.391196in}}%
\pgfpathlineto{\pgfqpoint{1.570862in}{2.342057in}}%
\pgfpathlineto{\pgfqpoint{1.643124in}{2.292315in}}%
\pgfpathlineto{\pgfqpoint{1.715385in}{2.241971in}}%
\pgfpathlineto{\pgfqpoint{1.787646in}{2.191025in}}%
\pgfpathlineto{\pgfqpoint{1.859907in}{2.139476in}}%
\pgfpathlineto{\pgfqpoint{1.932168in}{2.087326in}}%
\pgfpathlineto{\pgfqpoint{2.004429in}{2.034572in}}%
\pgfpathlineto{\pgfqpoint{2.076690in}{1.981217in}}%
\pgfpathlineto{\pgfqpoint{2.148951in}{1.927259in}}%
\pgfpathlineto{\pgfqpoint{2.221212in}{1.872698in}}%
\pgfpathlineto{\pgfqpoint{2.293473in}{1.817536in}}%
\pgfpathlineto{\pgfqpoint{2.365734in}{1.761771in}}%
\pgfpathlineto{\pgfqpoint{2.437995in}{1.705403in}}%
\pgfpathlineto{\pgfqpoint{2.510256in}{1.648434in}}%
\pgfpathlineto{\pgfqpoint{2.582517in}{1.590862in}}%
\pgfpathlineto{\pgfqpoint{2.654779in}{1.532687in}}%
\pgfpathlineto{\pgfqpoint{2.727040in}{1.473911in}}%
\pgfpathlineto{\pgfqpoint{2.799301in}{1.414532in}}%
\pgfpathlineto{\pgfqpoint{2.871562in}{1.354550in}}%
\pgfpathlineto{\pgfqpoint{2.943823in}{1.293966in}}%
\pgfpathlineto{\pgfqpoint{3.016084in}{1.232780in}}%
\pgfpathlineto{\pgfqpoint{3.088345in}{1.170992in}}%
\pgfpathlineto{\pgfqpoint{3.160606in}{1.108601in}}%
\pgfpathlineto{\pgfqpoint{3.232867in}{1.045608in}}%
\pgfpathlineto{\pgfqpoint{3.305128in}{0.982013in}}%
\pgfpathlineto{\pgfqpoint{3.377389in}{0.917815in}}%
\pgfpathlineto{\pgfqpoint{3.449650in}{0.853015in}}%
\pgfpathlineto{\pgfqpoint{3.521911in}{0.787613in}}%
\pgfpathlineto{\pgfqpoint{3.594172in}{0.721608in}}%
\pgfpathlineto{\pgfqpoint{3.666434in}{0.655001in}}%
\pgfpathlineto{\pgfqpoint{3.738695in}{0.587791in}}%
\pgfpathlineto{\pgfqpoint{3.883217in}{0.451766in}}%
\pgfpathlineto{\pgfqpoint{3.955478in}{0.508620in}}%
\pgfpathlineto{\pgfqpoint{4.027739in}{0.565120in}}%
\pgfpathlineto{\pgfqpoint{4.100000in}{0.621266in}}%
\pgfpathlineto{\pgfqpoint{4.172261in}{0.677058in}}%
\pgfpathlineto{\pgfqpoint{4.244522in}{0.732497in}}%
\pgfpathlineto{\pgfqpoint{4.316783in}{0.787581in}}%
\pgfpathlineto{\pgfqpoint{4.389044in}{0.842311in}}%
\pgfpathlineto{\pgfqpoint{4.461305in}{0.896688in}}%
\pgfpathlineto{\pgfqpoint{4.533566in}{0.950710in}}%
\pgfpathlineto{\pgfqpoint{4.605828in}{1.004378in}}%
\pgfpathlineto{\pgfqpoint{4.678089in}{1.057693in}}%
\pgfpathlineto{\pgfqpoint{4.750350in}{1.110653in}}%
\pgfpathlineto{\pgfqpoint{4.822611in}{1.163260in}}%
\pgfpathlineto{\pgfqpoint{4.894872in}{1.215513in}}%
\pgfpathlineto{\pgfqpoint{4.967133in}{1.267411in}}%
\pgfpathlineto{\pgfqpoint{5.039394in}{1.318956in}}%
\pgfpathlineto{\pgfqpoint{5.111655in}{1.370147in}}%
\pgfpathlineto{\pgfqpoint{5.183916in}{1.420984in}}%
\pgfpathlineto{\pgfqpoint{5.256177in}{1.471467in}}%
\pgfpathlineto{\pgfqpoint{5.328438in}{1.521595in}}%
\pgfpathlineto{\pgfqpoint{5.400699in}{1.571370in}}%
\pgfpathlineto{\pgfqpoint{5.472960in}{1.620791in}}%
\pgfpathlineto{\pgfqpoint{5.545221in}{1.669859in}}%
\pgfpathlineto{\pgfqpoint{5.617483in}{1.718572in}}%
\pgfpathlineto{\pgfqpoint{5.689744in}{1.766931in}}%
\pgfpathlineto{\pgfqpoint{5.762005in}{1.814936in}}%
\pgfpathlineto{\pgfqpoint{5.834266in}{1.862587in}}%
\pgfpathlineto{\pgfqpoint{5.906527in}{1.909885in}}%
\pgfpathlineto{\pgfqpoint{5.978788in}{1.956828in}}%
\pgfpathlineto{\pgfqpoint{6.051049in}{2.003417in}}%
\pgfpathlineto{\pgfqpoint{6.123310in}{2.049653in}}%
\pgfpathlineto{\pgfqpoint{6.195571in}{2.095534in}}%
\pgfpathlineto{\pgfqpoint{6.267832in}{2.141062in}}%
\pgfpathlineto{\pgfqpoint{6.340093in}{2.186235in}}%
\pgfpathlineto{\pgfqpoint{6.412354in}{2.231055in}}%
\pgfpathlineto{\pgfqpoint{6.484615in}{2.275521in}}%
\pgfpathlineto{\pgfqpoint{6.556876in}{2.319633in}}%
\pgfpathlineto{\pgfqpoint{6.629138in}{2.363390in}}%
\pgfpathlineto{\pgfqpoint{6.701399in}{2.406794in}}%
\pgfpathlineto{\pgfqpoint{6.773660in}{2.449844in}}%
\pgfpathlineto{\pgfqpoint{6.918182in}{2.535000in}}%
\pgfpathlineto{\pgfqpoint{6.918182in}{2.535000in}}%
\pgfusepath{stroke}%
\end{pgfscope}%
\begin{pgfscope}%
\pgfpathrectangle{\pgfqpoint{1.000000in}{0.330000in}}{\pgfqpoint{6.200000in}{2.310000in}}%
\pgfusepath{clip}%
\pgfsetrectcap%
\pgfsetroundjoin%
\pgfsetlinewidth{1.505625pt}%
\definecolor{currentstroke}{rgb}{0.090196,0.745098,0.811765}%
\pgfsetstrokecolor{currentstroke}%
\pgfsetdash{}{0pt}%
\pgfpathmoveto{\pgfqpoint{1.281818in}{2.535000in}}%
\pgfpathlineto{\pgfqpoint{1.281818in}{2.535000in}}%
\pgfpathlineto{\pgfqpoint{1.354079in}{2.509201in}}%
\pgfpathlineto{\pgfqpoint{1.426340in}{2.482989in}}%
\pgfpathlineto{\pgfqpoint{1.498601in}{2.456364in}}%
\pgfpathlineto{\pgfqpoint{1.570862in}{2.429326in}}%
\pgfpathlineto{\pgfqpoint{1.643124in}{2.401875in}}%
\pgfpathlineto{\pgfqpoint{1.715385in}{2.374011in}}%
\pgfpathlineto{\pgfqpoint{1.787646in}{2.345734in}}%
\pgfpathlineto{\pgfqpoint{1.859907in}{2.317044in}}%
\pgfpathlineto{\pgfqpoint{1.932168in}{2.287942in}}%
\pgfpathlineto{\pgfqpoint{2.004429in}{2.258426in}}%
\pgfpathlineto{\pgfqpoint{2.076690in}{2.228497in}}%
\pgfpathlineto{\pgfqpoint{2.148951in}{2.198156in}}%
\pgfpathlineto{\pgfqpoint{2.221212in}{2.167401in}}%
\pgfpathlineto{\pgfqpoint{2.293473in}{2.136234in}}%
\pgfpathlineto{\pgfqpoint{2.365734in}{2.104654in}}%
\pgfpathlineto{\pgfqpoint{2.437995in}{2.072660in}}%
\pgfpathlineto{\pgfqpoint{2.510256in}{2.040254in}}%
\pgfpathlineto{\pgfqpoint{2.582517in}{2.007435in}}%
\pgfpathlineto{\pgfqpoint{2.654779in}{1.974203in}}%
\pgfpathlineto{\pgfqpoint{2.727040in}{1.940558in}}%
\pgfpathlineto{\pgfqpoint{2.799301in}{1.906500in}}%
\pgfpathlineto{\pgfqpoint{2.871562in}{1.872029in}}%
\pgfpathlineto{\pgfqpoint{2.943823in}{1.837145in}}%
\pgfpathlineto{\pgfqpoint{3.016084in}{1.801848in}}%
\pgfpathlineto{\pgfqpoint{3.088345in}{1.766138in}}%
\pgfpathlineto{\pgfqpoint{3.160606in}{1.730015in}}%
\pgfpathlineto{\pgfqpoint{3.232867in}{1.693480in}}%
\pgfpathlineto{\pgfqpoint{3.305128in}{1.656531in}}%
\pgfpathlineto{\pgfqpoint{3.377389in}{1.619169in}}%
\pgfpathlineto{\pgfqpoint{3.449650in}{1.581395in}}%
\pgfpathlineto{\pgfqpoint{3.521911in}{1.543207in}}%
\pgfpathlineto{\pgfqpoint{3.594172in}{1.504607in}}%
\pgfpathlineto{\pgfqpoint{3.666434in}{1.465594in}}%
\pgfpathlineto{\pgfqpoint{3.738695in}{1.426167in}}%
\pgfpathlineto{\pgfqpoint{3.883217in}{1.346214in}}%
\pgfpathlineto{\pgfqpoint{3.955478in}{1.382980in}}%
\pgfpathlineto{\pgfqpoint{4.027739in}{1.419334in}}%
\pgfpathlineto{\pgfqpoint{4.100000in}{1.455274in}}%
\pgfpathlineto{\pgfqpoint{4.172261in}{1.490802in}}%
\pgfpathlineto{\pgfqpoint{4.244522in}{1.525917in}}%
\pgfpathlineto{\pgfqpoint{4.316783in}{1.560619in}}%
\pgfpathlineto{\pgfqpoint{4.389044in}{1.594907in}}%
\pgfpathlineto{\pgfqpoint{4.461305in}{1.628783in}}%
\pgfpathlineto{\pgfqpoint{4.533566in}{1.662246in}}%
\pgfpathlineto{\pgfqpoint{4.605828in}{1.695296in}}%
\pgfpathlineto{\pgfqpoint{4.678089in}{1.727933in}}%
\pgfpathlineto{\pgfqpoint{4.750350in}{1.760157in}}%
\pgfpathlineto{\pgfqpoint{4.822611in}{1.791969in}}%
\pgfpathlineto{\pgfqpoint{4.894872in}{1.823367in}}%
\pgfpathlineto{\pgfqpoint{4.967133in}{1.854352in}}%
\pgfpathlineto{\pgfqpoint{5.039394in}{1.884924in}}%
\pgfpathlineto{\pgfqpoint{5.111655in}{1.915084in}}%
\pgfpathlineto{\pgfqpoint{5.183916in}{1.944830in}}%
\pgfpathlineto{\pgfqpoint{5.256177in}{1.974164in}}%
\pgfpathlineto{\pgfqpoint{5.328438in}{2.003084in}}%
\pgfpathlineto{\pgfqpoint{5.400699in}{2.031592in}}%
\pgfpathlineto{\pgfqpoint{5.472960in}{2.059687in}}%
\pgfpathlineto{\pgfqpoint{5.545221in}{2.087369in}}%
\pgfpathlineto{\pgfqpoint{5.617483in}{2.114637in}}%
\pgfpathlineto{\pgfqpoint{5.689744in}{2.141493in}}%
\pgfpathlineto{\pgfqpoint{5.762005in}{2.167936in}}%
\pgfpathlineto{\pgfqpoint{5.834266in}{2.193966in}}%
\pgfpathlineto{\pgfqpoint{5.906527in}{2.219583in}}%
\pgfpathlineto{\pgfqpoint{5.978788in}{2.244787in}}%
\pgfpathlineto{\pgfqpoint{6.051049in}{2.269578in}}%
\pgfpathlineto{\pgfqpoint{6.123310in}{2.293957in}}%
\pgfpathlineto{\pgfqpoint{6.195571in}{2.317922in}}%
\pgfpathlineto{\pgfqpoint{6.267832in}{2.341474in}}%
\pgfpathlineto{\pgfqpoint{6.340093in}{2.364613in}}%
\pgfpathlineto{\pgfqpoint{6.412354in}{2.387340in}}%
\pgfpathlineto{\pgfqpoint{6.484615in}{2.409653in}}%
\pgfpathlineto{\pgfqpoint{6.556876in}{2.431554in}}%
\pgfpathlineto{\pgfqpoint{6.629138in}{2.453041in}}%
\pgfpathlineto{\pgfqpoint{6.701399in}{2.474116in}}%
\pgfpathlineto{\pgfqpoint{6.773660in}{2.494778in}}%
\pgfpathlineto{\pgfqpoint{6.918182in}{2.535000in}}%
\pgfpathlineto{\pgfqpoint{6.918182in}{2.535000in}}%
\pgfusepath{stroke}%
\end{pgfscope}%
\begin{pgfscope}%
\pgfpathrectangle{\pgfqpoint{1.000000in}{0.330000in}}{\pgfqpoint{6.200000in}{2.310000in}}%
\pgfusepath{clip}%
\pgfsetrectcap%
\pgfsetroundjoin%
\pgfsetlinewidth{1.505625pt}%
\definecolor{currentstroke}{rgb}{0.121569,0.466667,0.705882}%
\pgfsetstrokecolor{currentstroke}%
\pgfsetdash{}{0pt}%
\pgfpathmoveto{\pgfqpoint{1.281818in}{2.535000in}}%
\pgfpathlineto{\pgfqpoint{1.281818in}{2.535000in}}%
\pgfpathlineto{\pgfqpoint{1.354079in}{2.513086in}}%
\pgfpathlineto{\pgfqpoint{1.426340in}{2.490818in}}%
\pgfpathlineto{\pgfqpoint{1.498601in}{2.468196in}}%
\pgfpathlineto{\pgfqpoint{1.570862in}{2.445220in}}%
\pgfpathlineto{\pgfqpoint{1.643124in}{2.421891in}}%
\pgfpathlineto{\pgfqpoint{1.715385in}{2.398207in}}%
\pgfpathlineto{\pgfqpoint{1.787646in}{2.374169in}}%
\pgfpathlineto{\pgfqpoint{1.859907in}{2.349778in}}%
\pgfpathlineto{\pgfqpoint{1.932168in}{2.325032in}}%
\pgfpathlineto{\pgfqpoint{2.004429in}{2.299933in}}%
\pgfpathlineto{\pgfqpoint{2.076690in}{2.274479in}}%
\pgfpathlineto{\pgfqpoint{2.148951in}{2.248672in}}%
\pgfpathlineto{\pgfqpoint{2.221212in}{2.222510in}}%
\pgfpathlineto{\pgfqpoint{2.293473in}{2.195995in}}%
\pgfpathlineto{\pgfqpoint{2.365734in}{2.169126in}}%
\pgfpathlineto{\pgfqpoint{2.437995in}{2.141902in}}%
\pgfpathlineto{\pgfqpoint{2.510256in}{2.114325in}}%
\pgfpathlineto{\pgfqpoint{2.582517in}{2.086394in}}%
\pgfpathlineto{\pgfqpoint{2.654779in}{2.058109in}}%
\pgfpathlineto{\pgfqpoint{2.727040in}{2.029470in}}%
\pgfpathlineto{\pgfqpoint{2.799301in}{2.000477in}}%
\pgfpathlineto{\pgfqpoint{2.871562in}{1.971130in}}%
\pgfpathlineto{\pgfqpoint{2.943823in}{1.941429in}}%
\pgfpathlineto{\pgfqpoint{3.016084in}{1.911374in}}%
\pgfpathlineto{\pgfqpoint{3.088345in}{1.880965in}}%
\pgfpathlineto{\pgfqpoint{3.160606in}{1.850203in}}%
\pgfpathlineto{\pgfqpoint{3.232867in}{1.819086in}}%
\pgfpathlineto{\pgfqpoint{3.305128in}{1.787615in}}%
\pgfpathlineto{\pgfqpoint{3.377389in}{1.755791in}}%
\pgfpathlineto{\pgfqpoint{3.449650in}{1.723612in}}%
\pgfpathlineto{\pgfqpoint{3.521911in}{1.691080in}}%
\pgfpathlineto{\pgfqpoint{3.594172in}{1.658193in}}%
\pgfpathlineto{\pgfqpoint{3.666434in}{1.624953in}}%
\pgfpathlineto{\pgfqpoint{3.738695in}{1.591358in}}%
\pgfpathlineto{\pgfqpoint{3.883217in}{1.523226in}}%
\pgfpathlineto{\pgfqpoint{3.955478in}{1.559659in}}%
\pgfpathlineto{\pgfqpoint{4.027739in}{1.595489in}}%
\pgfpathlineto{\pgfqpoint{4.100000in}{1.630718in}}%
\pgfpathlineto{\pgfqpoint{4.172261in}{1.665344in}}%
\pgfpathlineto{\pgfqpoint{4.244522in}{1.699368in}}%
\pgfpathlineto{\pgfqpoint{4.316783in}{1.732789in}}%
\pgfpathlineto{\pgfqpoint{4.389044in}{1.765608in}}%
\pgfpathlineto{\pgfqpoint{4.461305in}{1.797825in}}%
\pgfpathlineto{\pgfqpoint{4.533566in}{1.829439in}}%
\pgfpathlineto{\pgfqpoint{4.605828in}{1.860451in}}%
\pgfpathlineto{\pgfqpoint{4.678089in}{1.890860in}}%
\pgfpathlineto{\pgfqpoint{4.750350in}{1.920668in}}%
\pgfpathlineto{\pgfqpoint{4.822611in}{1.949873in}}%
\pgfpathlineto{\pgfqpoint{4.894872in}{1.978475in}}%
\pgfpathlineto{\pgfqpoint{4.967133in}{2.006476in}}%
\pgfpathlineto{\pgfqpoint{5.039394in}{2.033874in}}%
\pgfpathlineto{\pgfqpoint{5.111655in}{2.060669in}}%
\pgfpathlineto{\pgfqpoint{5.183916in}{2.086862in}}%
\pgfpathlineto{\pgfqpoint{5.256177in}{2.112453in}}%
\pgfpathlineto{\pgfqpoint{5.328438in}{2.137442in}}%
\pgfpathlineto{\pgfqpoint{5.400699in}{2.161828in}}%
\pgfpathlineto{\pgfqpoint{5.472960in}{2.185612in}}%
\pgfpathlineto{\pgfqpoint{5.545221in}{2.208794in}}%
\pgfpathlineto{\pgfqpoint{5.617483in}{2.231373in}}%
\pgfpathlineto{\pgfqpoint{5.689744in}{2.253350in}}%
\pgfpathlineto{\pgfqpoint{5.762005in}{2.274724in}}%
\pgfpathlineto{\pgfqpoint{5.834266in}{2.295497in}}%
\pgfpathlineto{\pgfqpoint{5.906527in}{2.315667in}}%
\pgfpathlineto{\pgfqpoint{5.978788in}{2.335234in}}%
\pgfpathlineto{\pgfqpoint{6.051049in}{2.354199in}}%
\pgfpathlineto{\pgfqpoint{6.123310in}{2.372562in}}%
\pgfpathlineto{\pgfqpoint{6.195571in}{2.390323in}}%
\pgfpathlineto{\pgfqpoint{6.267832in}{2.407481in}}%
\pgfpathlineto{\pgfqpoint{6.340093in}{2.424037in}}%
\pgfpathlineto{\pgfqpoint{6.412354in}{2.439990in}}%
\pgfpathlineto{\pgfqpoint{6.484615in}{2.455341in}}%
\pgfpathlineto{\pgfqpoint{6.556876in}{2.470090in}}%
\pgfpathlineto{\pgfqpoint{6.629138in}{2.484237in}}%
\pgfpathlineto{\pgfqpoint{6.701399in}{2.497781in}}%
\pgfpathlineto{\pgfqpoint{6.773660in}{2.510723in}}%
\pgfpathlineto{\pgfqpoint{6.918182in}{2.535000in}}%
\pgfpathlineto{\pgfqpoint{6.918182in}{2.535000in}}%
\pgfusepath{stroke}%
\end{pgfscope}%
\begin{pgfscope}%
\pgfpathrectangle{\pgfqpoint{1.000000in}{0.330000in}}{\pgfqpoint{6.200000in}{2.310000in}}%
\pgfusepath{clip}%
\pgfsetrectcap%
\pgfsetroundjoin%
\pgfsetlinewidth{1.505625pt}%
\definecolor{currentstroke}{rgb}{1.000000,0.498039,0.054902}%
\pgfsetstrokecolor{currentstroke}%
\pgfsetdash{}{0pt}%
\pgfpathmoveto{\pgfqpoint{1.281818in}{2.535000in}}%
\pgfpathlineto{\pgfqpoint{1.281818in}{2.535000in}}%
\pgfpathlineto{\pgfqpoint{1.354079in}{2.509434in}}%
\pgfpathlineto{\pgfqpoint{1.426340in}{2.483454in}}%
\pgfpathlineto{\pgfqpoint{1.498601in}{2.457062in}}%
\pgfpathlineto{\pgfqpoint{1.570862in}{2.430257in}}%
\pgfpathlineto{\pgfqpoint{1.643124in}{2.403039in}}%
\pgfpathlineto{\pgfqpoint{1.715385in}{2.375408in}}%
\pgfpathlineto{\pgfqpoint{1.787646in}{2.347364in}}%
\pgfpathlineto{\pgfqpoint{1.859907in}{2.318907in}}%
\pgfpathlineto{\pgfqpoint{1.932168in}{2.290037in}}%
\pgfpathlineto{\pgfqpoint{2.004429in}{2.260755in}}%
\pgfpathlineto{\pgfqpoint{2.076690in}{2.231059in}}%
\pgfpathlineto{\pgfqpoint{2.148951in}{2.200950in}}%
\pgfpathlineto{\pgfqpoint{2.221212in}{2.170429in}}%
\pgfpathlineto{\pgfqpoint{2.293473in}{2.139494in}}%
\pgfpathlineto{\pgfqpoint{2.365734in}{2.108147in}}%
\pgfpathlineto{\pgfqpoint{2.437995in}{2.076386in}}%
\pgfpathlineto{\pgfqpoint{2.510256in}{2.044213in}}%
\pgfpathlineto{\pgfqpoint{2.582517in}{2.011626in}}%
\pgfpathlineto{\pgfqpoint{2.654779in}{1.978627in}}%
\pgfpathlineto{\pgfqpoint{2.727040in}{1.945215in}}%
\pgfpathlineto{\pgfqpoint{2.799301in}{1.911390in}}%
\pgfpathlineto{\pgfqpoint{2.871562in}{1.877152in}}%
\pgfpathlineto{\pgfqpoint{2.943823in}{1.842501in}}%
\pgfpathlineto{\pgfqpoint{3.016084in}{1.807437in}}%
\pgfpathlineto{\pgfqpoint{3.088345in}{1.771960in}}%
\pgfpathlineto{\pgfqpoint{3.160606in}{1.736070in}}%
\pgfpathlineto{\pgfqpoint{3.232867in}{1.699767in}}%
\pgfpathlineto{\pgfqpoint{3.305128in}{1.663051in}}%
\pgfpathlineto{\pgfqpoint{3.377389in}{1.625922in}}%
\pgfpathlineto{\pgfqpoint{3.449650in}{1.588381in}}%
\pgfpathlineto{\pgfqpoint{3.521911in}{1.550426in}}%
\pgfpathlineto{\pgfqpoint{3.594172in}{1.512059in}}%
\pgfpathlineto{\pgfqpoint{3.666434in}{1.473278in}}%
\pgfpathlineto{\pgfqpoint{3.738695in}{1.434085in}}%
\pgfpathlineto{\pgfqpoint{3.883217in}{1.354597in}}%
\pgfpathlineto{\pgfqpoint{3.955478in}{1.391164in}}%
\pgfpathlineto{\pgfqpoint{4.027739in}{1.427318in}}%
\pgfpathlineto{\pgfqpoint{4.100000in}{1.463059in}}%
\pgfpathlineto{\pgfqpoint{4.172261in}{1.498387in}}%
\pgfpathlineto{\pgfqpoint{4.244522in}{1.533302in}}%
\pgfpathlineto{\pgfqpoint{4.316783in}{1.567804in}}%
\pgfpathlineto{\pgfqpoint{4.389044in}{1.601893in}}%
\pgfpathlineto{\pgfqpoint{4.461305in}{1.635570in}}%
\pgfpathlineto{\pgfqpoint{4.533566in}{1.668833in}}%
\pgfpathlineto{\pgfqpoint{4.605828in}{1.701683in}}%
\pgfpathlineto{\pgfqpoint{4.678089in}{1.734121in}}%
\pgfpathlineto{\pgfqpoint{4.750350in}{1.766145in}}%
\pgfpathlineto{\pgfqpoint{4.822611in}{1.797757in}}%
\pgfpathlineto{\pgfqpoint{4.894872in}{1.828956in}}%
\pgfpathlineto{\pgfqpoint{4.967133in}{1.859741in}}%
\pgfpathlineto{\pgfqpoint{5.039394in}{1.890114in}}%
\pgfpathlineto{\pgfqpoint{5.111655in}{1.920074in}}%
\pgfpathlineto{\pgfqpoint{5.183916in}{1.949621in}}%
\pgfpathlineto{\pgfqpoint{5.256177in}{1.978755in}}%
\pgfpathlineto{\pgfqpoint{5.328438in}{2.007476in}}%
\pgfpathlineto{\pgfqpoint{5.400699in}{2.035784in}}%
\pgfpathlineto{\pgfqpoint{5.472960in}{2.063679in}}%
\pgfpathlineto{\pgfqpoint{5.545221in}{2.091161in}}%
\pgfpathlineto{\pgfqpoint{5.617483in}{2.118230in}}%
\pgfpathlineto{\pgfqpoint{5.689744in}{2.144886in}}%
\pgfpathlineto{\pgfqpoint{5.762005in}{2.171130in}}%
\pgfpathlineto{\pgfqpoint{5.834266in}{2.196960in}}%
\pgfpathlineto{\pgfqpoint{5.906527in}{2.222377in}}%
\pgfpathlineto{\pgfqpoint{5.978788in}{2.247382in}}%
\pgfpathlineto{\pgfqpoint{6.051049in}{2.271973in}}%
\pgfpathlineto{\pgfqpoint{6.123310in}{2.296152in}}%
\pgfpathlineto{\pgfqpoint{6.195571in}{2.319918in}}%
\pgfpathlineto{\pgfqpoint{6.267832in}{2.343270in}}%
\pgfpathlineto{\pgfqpoint{6.340093in}{2.366210in}}%
\pgfpathlineto{\pgfqpoint{6.412354in}{2.388737in}}%
\pgfpathlineto{\pgfqpoint{6.484615in}{2.410851in}}%
\pgfpathlineto{\pgfqpoint{6.556876in}{2.432552in}}%
\pgfpathlineto{\pgfqpoint{6.629138in}{2.453840in}}%
\pgfpathlineto{\pgfqpoint{6.701399in}{2.474715in}}%
\pgfpathlineto{\pgfqpoint{6.773660in}{2.495177in}}%
\pgfpathlineto{\pgfqpoint{6.918182in}{2.535000in}}%
\pgfpathlineto{\pgfqpoint{6.918182in}{2.535000in}}%
\pgfusepath{stroke}%
\end{pgfscope}%
\begin{pgfscope}%
\pgfpathrectangle{\pgfqpoint{1.000000in}{0.330000in}}{\pgfqpoint{6.200000in}{2.310000in}}%
\pgfusepath{clip}%
\pgfsetrectcap%
\pgfsetroundjoin%
\pgfsetlinewidth{1.505625pt}%
\definecolor{currentstroke}{rgb}{0.172549,0.627451,0.172549}%
\pgfsetstrokecolor{currentstroke}%
\pgfsetdash{}{0pt}%
\pgfpathmoveto{\pgfqpoint{1.281818in}{2.535000in}}%
\pgfpathlineto{\pgfqpoint{1.281818in}{2.535000in}}%
\pgfpathlineto{\pgfqpoint{1.354079in}{2.513086in}}%
\pgfpathlineto{\pgfqpoint{1.426340in}{2.490818in}}%
\pgfpathlineto{\pgfqpoint{1.498601in}{2.468196in}}%
\pgfpathlineto{\pgfqpoint{1.570862in}{2.445220in}}%
\pgfpathlineto{\pgfqpoint{1.643124in}{2.421891in}}%
\pgfpathlineto{\pgfqpoint{1.715385in}{2.398207in}}%
\pgfpathlineto{\pgfqpoint{1.787646in}{2.374169in}}%
\pgfpathlineto{\pgfqpoint{1.859907in}{2.349778in}}%
\pgfpathlineto{\pgfqpoint{1.932168in}{2.325032in}}%
\pgfpathlineto{\pgfqpoint{2.004429in}{2.299933in}}%
\pgfpathlineto{\pgfqpoint{2.076690in}{2.274479in}}%
\pgfpathlineto{\pgfqpoint{2.148951in}{2.248672in}}%
\pgfpathlineto{\pgfqpoint{2.221212in}{2.222510in}}%
\pgfpathlineto{\pgfqpoint{2.293473in}{2.195995in}}%
\pgfpathlineto{\pgfqpoint{2.365734in}{2.169126in}}%
\pgfpathlineto{\pgfqpoint{2.437995in}{2.141902in}}%
\pgfpathlineto{\pgfqpoint{2.510256in}{2.114325in}}%
\pgfpathlineto{\pgfqpoint{2.582517in}{2.086394in}}%
\pgfpathlineto{\pgfqpoint{2.654779in}{2.058109in}}%
\pgfpathlineto{\pgfqpoint{2.727040in}{2.029470in}}%
\pgfpathlineto{\pgfqpoint{2.799301in}{2.000477in}}%
\pgfpathlineto{\pgfqpoint{2.871562in}{1.971130in}}%
\pgfpathlineto{\pgfqpoint{2.943823in}{1.941429in}}%
\pgfpathlineto{\pgfqpoint{3.016084in}{1.911374in}}%
\pgfpathlineto{\pgfqpoint{3.088345in}{1.880965in}}%
\pgfpathlineto{\pgfqpoint{3.160606in}{1.850203in}}%
\pgfpathlineto{\pgfqpoint{3.232867in}{1.819086in}}%
\pgfpathlineto{\pgfqpoint{3.305128in}{1.787615in}}%
\pgfpathlineto{\pgfqpoint{3.377389in}{1.755791in}}%
\pgfpathlineto{\pgfqpoint{3.449650in}{1.723612in}}%
\pgfpathlineto{\pgfqpoint{3.521911in}{1.691080in}}%
\pgfpathlineto{\pgfqpoint{3.594172in}{1.658193in}}%
\pgfpathlineto{\pgfqpoint{3.666434in}{1.624953in}}%
\pgfpathlineto{\pgfqpoint{3.738695in}{1.591358in}}%
\pgfpathlineto{\pgfqpoint{3.883217in}{1.523226in}}%
\pgfpathlineto{\pgfqpoint{3.955478in}{1.559659in}}%
\pgfpathlineto{\pgfqpoint{4.027739in}{1.595489in}}%
\pgfpathlineto{\pgfqpoint{4.100000in}{1.630718in}}%
\pgfpathlineto{\pgfqpoint{4.172261in}{1.665344in}}%
\pgfpathlineto{\pgfqpoint{4.244522in}{1.699368in}}%
\pgfpathlineto{\pgfqpoint{4.316783in}{1.732789in}}%
\pgfpathlineto{\pgfqpoint{4.389044in}{1.765608in}}%
\pgfpathlineto{\pgfqpoint{4.461305in}{1.797825in}}%
\pgfpathlineto{\pgfqpoint{4.533566in}{1.829439in}}%
\pgfpathlineto{\pgfqpoint{4.605828in}{1.860451in}}%
\pgfpathlineto{\pgfqpoint{4.678089in}{1.890860in}}%
\pgfpathlineto{\pgfqpoint{4.750350in}{1.920668in}}%
\pgfpathlineto{\pgfqpoint{4.822611in}{1.949873in}}%
\pgfpathlineto{\pgfqpoint{4.894872in}{1.978475in}}%
\pgfpathlineto{\pgfqpoint{4.967133in}{2.006476in}}%
\pgfpathlineto{\pgfqpoint{5.039394in}{2.033874in}}%
\pgfpathlineto{\pgfqpoint{5.111655in}{2.060669in}}%
\pgfpathlineto{\pgfqpoint{5.183916in}{2.086862in}}%
\pgfpathlineto{\pgfqpoint{5.256177in}{2.112453in}}%
\pgfpathlineto{\pgfqpoint{5.328438in}{2.137442in}}%
\pgfpathlineto{\pgfqpoint{5.400699in}{2.161828in}}%
\pgfpathlineto{\pgfqpoint{5.472960in}{2.185612in}}%
\pgfpathlineto{\pgfqpoint{5.545221in}{2.208794in}}%
\pgfpathlineto{\pgfqpoint{5.617483in}{2.231373in}}%
\pgfpathlineto{\pgfqpoint{5.689744in}{2.253350in}}%
\pgfpathlineto{\pgfqpoint{5.762005in}{2.274724in}}%
\pgfpathlineto{\pgfqpoint{5.834266in}{2.295497in}}%
\pgfpathlineto{\pgfqpoint{5.906527in}{2.315667in}}%
\pgfpathlineto{\pgfqpoint{5.978788in}{2.335234in}}%
\pgfpathlineto{\pgfqpoint{6.051049in}{2.354199in}}%
\pgfpathlineto{\pgfqpoint{6.123310in}{2.372562in}}%
\pgfpathlineto{\pgfqpoint{6.195571in}{2.390323in}}%
\pgfpathlineto{\pgfqpoint{6.267832in}{2.407481in}}%
\pgfpathlineto{\pgfqpoint{6.340093in}{2.424037in}}%
\pgfpathlineto{\pgfqpoint{6.412354in}{2.439990in}}%
\pgfpathlineto{\pgfqpoint{6.484615in}{2.455341in}}%
\pgfpathlineto{\pgfqpoint{6.556876in}{2.470090in}}%
\pgfpathlineto{\pgfqpoint{6.629138in}{2.484237in}}%
\pgfpathlineto{\pgfqpoint{6.701399in}{2.497781in}}%
\pgfpathlineto{\pgfqpoint{6.773660in}{2.510723in}}%
\pgfpathlineto{\pgfqpoint{6.918182in}{2.535000in}}%
\pgfpathlineto{\pgfqpoint{6.918182in}{2.535000in}}%
\pgfusepath{stroke}%
\end{pgfscope}%
\begin{pgfscope}%
\pgfpathrectangle{\pgfqpoint{1.000000in}{0.330000in}}{\pgfqpoint{6.200000in}{2.310000in}}%
\pgfusepath{clip}%
\pgfsetrectcap%
\pgfsetroundjoin%
\pgfsetlinewidth{1.505625pt}%
\definecolor{currentstroke}{rgb}{0.839216,0.152941,0.156863}%
\pgfsetstrokecolor{currentstroke}%
\pgfsetdash{}{0pt}%
\pgfpathmoveto{\pgfqpoint{1.281818in}{2.535000in}}%
\pgfpathlineto{\pgfqpoint{1.281818in}{2.535000in}}%
\pgfpathlineto{\pgfqpoint{1.354079in}{2.504677in}}%
\pgfpathlineto{\pgfqpoint{1.426340in}{2.473923in}}%
\pgfpathlineto{\pgfqpoint{1.498601in}{2.442737in}}%
\pgfpathlineto{\pgfqpoint{1.570862in}{2.411119in}}%
\pgfpathlineto{\pgfqpoint{1.643124in}{2.379070in}}%
\pgfpathlineto{\pgfqpoint{1.715385in}{2.346589in}}%
\pgfpathlineto{\pgfqpoint{1.787646in}{2.313677in}}%
\pgfpathlineto{\pgfqpoint{1.859907in}{2.280333in}}%
\pgfpathlineto{\pgfqpoint{1.932168in}{2.246557in}}%
\pgfpathlineto{\pgfqpoint{2.004429in}{2.212350in}}%
\pgfpathlineto{\pgfqpoint{2.076690in}{2.177712in}}%
\pgfpathlineto{\pgfqpoint{2.148951in}{2.142641in}}%
\pgfpathlineto{\pgfqpoint{2.221212in}{2.107140in}}%
\pgfpathlineto{\pgfqpoint{2.293473in}{2.071206in}}%
\pgfpathlineto{\pgfqpoint{2.365734in}{2.034841in}}%
\pgfpathlineto{\pgfqpoint{2.437995in}{1.998045in}}%
\pgfpathlineto{\pgfqpoint{2.510256in}{1.960817in}}%
\pgfpathlineto{\pgfqpoint{2.582517in}{1.923157in}}%
\pgfpathlineto{\pgfqpoint{2.654779in}{1.885066in}}%
\pgfpathlineto{\pgfqpoint{2.727040in}{1.846543in}}%
\pgfpathlineto{\pgfqpoint{2.799301in}{1.807589in}}%
\pgfpathlineto{\pgfqpoint{2.871562in}{1.768203in}}%
\pgfpathlineto{\pgfqpoint{2.943823in}{1.728385in}}%
\pgfpathlineto{\pgfqpoint{3.016084in}{1.688136in}}%
\pgfpathlineto{\pgfqpoint{3.088345in}{1.647456in}}%
\pgfpathlineto{\pgfqpoint{3.160606in}{1.606343in}}%
\pgfpathlineto{\pgfqpoint{3.232867in}{1.564800in}}%
\pgfpathlineto{\pgfqpoint{3.305128in}{1.522824in}}%
\pgfpathlineto{\pgfqpoint{3.377389in}{1.480417in}}%
\pgfpathlineto{\pgfqpoint{3.449650in}{1.437579in}}%
\pgfpathlineto{\pgfqpoint{3.521911in}{1.394309in}}%
\pgfpathlineto{\pgfqpoint{3.594172in}{1.350607in}}%
\pgfpathlineto{\pgfqpoint{3.666434in}{1.306474in}}%
\pgfpathlineto{\pgfqpoint{3.738695in}{1.261909in}}%
\pgfpathlineto{\pgfqpoint{3.883217in}{1.171628in}}%
\pgfpathlineto{\pgfqpoint{3.955478in}{1.211343in}}%
\pgfpathlineto{\pgfqpoint{4.027739in}{1.250703in}}%
\pgfpathlineto{\pgfqpoint{4.100000in}{1.289710in}}%
\pgfpathlineto{\pgfqpoint{4.172261in}{1.328362in}}%
\pgfpathlineto{\pgfqpoint{4.244522in}{1.366661in}}%
\pgfpathlineto{\pgfqpoint{4.316783in}{1.404606in}}%
\pgfpathlineto{\pgfqpoint{4.389044in}{1.442197in}}%
\pgfpathlineto{\pgfqpoint{4.461305in}{1.479433in}}%
\pgfpathlineto{\pgfqpoint{4.533566in}{1.516316in}}%
\pgfpathlineto{\pgfqpoint{4.605828in}{1.552845in}}%
\pgfpathlineto{\pgfqpoint{4.678089in}{1.589020in}}%
\pgfpathlineto{\pgfqpoint{4.750350in}{1.624841in}}%
\pgfpathlineto{\pgfqpoint{4.822611in}{1.660308in}}%
\pgfpathlineto{\pgfqpoint{4.894872in}{1.695421in}}%
\pgfpathlineto{\pgfqpoint{4.967133in}{1.730180in}}%
\pgfpathlineto{\pgfqpoint{5.039394in}{1.764585in}}%
\pgfpathlineto{\pgfqpoint{5.111655in}{1.798636in}}%
\pgfpathlineto{\pgfqpoint{5.183916in}{1.832334in}}%
\pgfpathlineto{\pgfqpoint{5.256177in}{1.865677in}}%
\pgfpathlineto{\pgfqpoint{5.328438in}{1.898666in}}%
\pgfpathlineto{\pgfqpoint{5.400699in}{1.931302in}}%
\pgfpathlineto{\pgfqpoint{5.472960in}{1.963583in}}%
\pgfpathlineto{\pgfqpoint{5.545221in}{1.995511in}}%
\pgfpathlineto{\pgfqpoint{5.617483in}{2.027084in}}%
\pgfpathlineto{\pgfqpoint{5.689744in}{2.058304in}}%
\pgfpathlineto{\pgfqpoint{5.762005in}{2.089169in}}%
\pgfpathlineto{\pgfqpoint{5.834266in}{2.119681in}}%
\pgfpathlineto{\pgfqpoint{5.906527in}{2.149839in}}%
\pgfpathlineto{\pgfqpoint{5.978788in}{2.179642in}}%
\pgfpathlineto{\pgfqpoint{6.051049in}{2.209092in}}%
\pgfpathlineto{\pgfqpoint{6.123310in}{2.238188in}}%
\pgfpathlineto{\pgfqpoint{6.195571in}{2.266930in}}%
\pgfpathlineto{\pgfqpoint{6.267832in}{2.295318in}}%
\pgfpathlineto{\pgfqpoint{6.340093in}{2.323352in}}%
\pgfpathlineto{\pgfqpoint{6.412354in}{2.351032in}}%
\pgfpathlineto{\pgfqpoint{6.484615in}{2.378358in}}%
\pgfpathlineto{\pgfqpoint{6.556876in}{2.405330in}}%
\pgfpathlineto{\pgfqpoint{6.629138in}{2.431949in}}%
\pgfpathlineto{\pgfqpoint{6.701399in}{2.458213in}}%
\pgfpathlineto{\pgfqpoint{6.773660in}{2.484123in}}%
\pgfpathlineto{\pgfqpoint{6.918182in}{2.535000in}}%
\pgfpathlineto{\pgfqpoint{6.918182in}{2.535000in}}%
\pgfusepath{stroke}%
\end{pgfscope}%
\begin{pgfscope}%
\pgfpathrectangle{\pgfqpoint{1.000000in}{0.330000in}}{\pgfqpoint{6.200000in}{2.310000in}}%
\pgfusepath{clip}%
\pgfsetrectcap%
\pgfsetroundjoin%
\pgfsetlinewidth{1.505625pt}%
\definecolor{currentstroke}{rgb}{0.580392,0.403922,0.741176}%
\pgfsetstrokecolor{currentstroke}%
\pgfsetdash{}{0pt}%
\pgfpathmoveto{\pgfqpoint{1.281818in}{2.535000in}}%
\pgfpathlineto{\pgfqpoint{1.281818in}{2.535000in}}%
\pgfpathlineto{\pgfqpoint{1.354079in}{2.509434in}}%
\pgfpathlineto{\pgfqpoint{1.426340in}{2.483454in}}%
\pgfpathlineto{\pgfqpoint{1.498601in}{2.457062in}}%
\pgfpathlineto{\pgfqpoint{1.570862in}{2.430257in}}%
\pgfpathlineto{\pgfqpoint{1.643124in}{2.403039in}}%
\pgfpathlineto{\pgfqpoint{1.715385in}{2.375408in}}%
\pgfpathlineto{\pgfqpoint{1.787646in}{2.347364in}}%
\pgfpathlineto{\pgfqpoint{1.859907in}{2.318907in}}%
\pgfpathlineto{\pgfqpoint{1.932168in}{2.290037in}}%
\pgfpathlineto{\pgfqpoint{2.004429in}{2.260755in}}%
\pgfpathlineto{\pgfqpoint{2.076690in}{2.231059in}}%
\pgfpathlineto{\pgfqpoint{2.148951in}{2.200950in}}%
\pgfpathlineto{\pgfqpoint{2.221212in}{2.170429in}}%
\pgfpathlineto{\pgfqpoint{2.293473in}{2.139494in}}%
\pgfpathlineto{\pgfqpoint{2.365734in}{2.108147in}}%
\pgfpathlineto{\pgfqpoint{2.437995in}{2.076386in}}%
\pgfpathlineto{\pgfqpoint{2.510256in}{2.044213in}}%
\pgfpathlineto{\pgfqpoint{2.582517in}{2.011626in}}%
\pgfpathlineto{\pgfqpoint{2.654779in}{1.978627in}}%
\pgfpathlineto{\pgfqpoint{2.727040in}{1.945215in}}%
\pgfpathlineto{\pgfqpoint{2.799301in}{1.911390in}}%
\pgfpathlineto{\pgfqpoint{2.871562in}{1.877152in}}%
\pgfpathlineto{\pgfqpoint{2.943823in}{1.842501in}}%
\pgfpathlineto{\pgfqpoint{3.016084in}{1.807437in}}%
\pgfpathlineto{\pgfqpoint{3.088345in}{1.771960in}}%
\pgfpathlineto{\pgfqpoint{3.160606in}{1.736070in}}%
\pgfpathlineto{\pgfqpoint{3.232867in}{1.699767in}}%
\pgfpathlineto{\pgfqpoint{3.305128in}{1.663051in}}%
\pgfpathlineto{\pgfqpoint{3.377389in}{1.625922in}}%
\pgfpathlineto{\pgfqpoint{3.449650in}{1.588381in}}%
\pgfpathlineto{\pgfqpoint{3.521911in}{1.550426in}}%
\pgfpathlineto{\pgfqpoint{3.594172in}{1.512059in}}%
\pgfpathlineto{\pgfqpoint{3.666434in}{1.473278in}}%
\pgfpathlineto{\pgfqpoint{3.738695in}{1.434085in}}%
\pgfpathlineto{\pgfqpoint{3.883217in}{1.354597in}}%
\pgfpathlineto{\pgfqpoint{3.955478in}{1.391164in}}%
\pgfpathlineto{\pgfqpoint{4.027739in}{1.427318in}}%
\pgfpathlineto{\pgfqpoint{4.100000in}{1.463059in}}%
\pgfpathlineto{\pgfqpoint{4.172261in}{1.498387in}}%
\pgfpathlineto{\pgfqpoint{4.244522in}{1.533302in}}%
\pgfpathlineto{\pgfqpoint{4.316783in}{1.567804in}}%
\pgfpathlineto{\pgfqpoint{4.389044in}{1.601893in}}%
\pgfpathlineto{\pgfqpoint{4.461305in}{1.635570in}}%
\pgfpathlineto{\pgfqpoint{4.533566in}{1.668833in}}%
\pgfpathlineto{\pgfqpoint{4.605828in}{1.701683in}}%
\pgfpathlineto{\pgfqpoint{4.678089in}{1.734121in}}%
\pgfpathlineto{\pgfqpoint{4.750350in}{1.766145in}}%
\pgfpathlineto{\pgfqpoint{4.822611in}{1.797757in}}%
\pgfpathlineto{\pgfqpoint{4.894872in}{1.828956in}}%
\pgfpathlineto{\pgfqpoint{4.967133in}{1.859741in}}%
\pgfpathlineto{\pgfqpoint{5.039394in}{1.890114in}}%
\pgfpathlineto{\pgfqpoint{5.111655in}{1.920074in}}%
\pgfpathlineto{\pgfqpoint{5.183916in}{1.949621in}}%
\pgfpathlineto{\pgfqpoint{5.256177in}{1.978755in}}%
\pgfpathlineto{\pgfqpoint{5.328438in}{2.007476in}}%
\pgfpathlineto{\pgfqpoint{5.400699in}{2.035784in}}%
\pgfpathlineto{\pgfqpoint{5.472960in}{2.063679in}}%
\pgfpathlineto{\pgfqpoint{5.545221in}{2.091161in}}%
\pgfpathlineto{\pgfqpoint{5.617483in}{2.118230in}}%
\pgfpathlineto{\pgfqpoint{5.689744in}{2.144886in}}%
\pgfpathlineto{\pgfqpoint{5.762005in}{2.171130in}}%
\pgfpathlineto{\pgfqpoint{5.834266in}{2.196960in}}%
\pgfpathlineto{\pgfqpoint{5.906527in}{2.222377in}}%
\pgfpathlineto{\pgfqpoint{5.978788in}{2.247382in}}%
\pgfpathlineto{\pgfqpoint{6.051049in}{2.271973in}}%
\pgfpathlineto{\pgfqpoint{6.123310in}{2.296152in}}%
\pgfpathlineto{\pgfqpoint{6.195571in}{2.319918in}}%
\pgfpathlineto{\pgfqpoint{6.267832in}{2.343270in}}%
\pgfpathlineto{\pgfqpoint{6.340093in}{2.366210in}}%
\pgfpathlineto{\pgfqpoint{6.412354in}{2.388737in}}%
\pgfpathlineto{\pgfqpoint{6.484615in}{2.410851in}}%
\pgfpathlineto{\pgfqpoint{6.556876in}{2.432552in}}%
\pgfpathlineto{\pgfqpoint{6.629138in}{2.453840in}}%
\pgfpathlineto{\pgfqpoint{6.701399in}{2.474715in}}%
\pgfpathlineto{\pgfqpoint{6.773660in}{2.495177in}}%
\pgfpathlineto{\pgfqpoint{6.918182in}{2.535000in}}%
\pgfpathlineto{\pgfqpoint{6.918182in}{2.535000in}}%
\pgfusepath{stroke}%
\end{pgfscope}%
\begin{pgfscope}%
\pgfpathrectangle{\pgfqpoint{1.000000in}{0.330000in}}{\pgfqpoint{6.200000in}{2.310000in}}%
\pgfusepath{clip}%
\pgfsetrectcap%
\pgfsetroundjoin%
\pgfsetlinewidth{1.505625pt}%
\definecolor{currentstroke}{rgb}{0.549020,0.337255,0.294118}%
\pgfsetstrokecolor{currentstroke}%
\pgfsetdash{}{0pt}%
\pgfpathmoveto{\pgfqpoint{1.281818in}{2.535000in}}%
\pgfpathlineto{\pgfqpoint{1.281818in}{2.535000in}}%
\pgfpathlineto{\pgfqpoint{1.354079in}{2.512620in}}%
\pgfpathlineto{\pgfqpoint{1.426340in}{2.489887in}}%
\pgfpathlineto{\pgfqpoint{1.498601in}{2.466799in}}%
\pgfpathlineto{\pgfqpoint{1.570862in}{2.443358in}}%
\pgfpathlineto{\pgfqpoint{1.643124in}{2.419562in}}%
\pgfpathlineto{\pgfqpoint{1.715385in}{2.395413in}}%
\pgfpathlineto{\pgfqpoint{1.787646in}{2.370909in}}%
\pgfpathlineto{\pgfqpoint{1.859907in}{2.346052in}}%
\pgfpathlineto{\pgfqpoint{1.932168in}{2.320841in}}%
\pgfpathlineto{\pgfqpoint{2.004429in}{2.295275in}}%
\pgfpathlineto{\pgfqpoint{2.076690in}{2.269356in}}%
\pgfpathlineto{\pgfqpoint{2.148951in}{2.243083in}}%
\pgfpathlineto{\pgfqpoint{2.221212in}{2.216456in}}%
\pgfpathlineto{\pgfqpoint{2.293473in}{2.189475in}}%
\pgfpathlineto{\pgfqpoint{2.365734in}{2.162140in}}%
\pgfpathlineto{\pgfqpoint{2.437995in}{2.134451in}}%
\pgfpathlineto{\pgfqpoint{2.510256in}{2.106408in}}%
\pgfpathlineto{\pgfqpoint{2.582517in}{2.078011in}}%
\pgfpathlineto{\pgfqpoint{2.654779in}{2.049260in}}%
\pgfpathlineto{\pgfqpoint{2.727040in}{2.020155in}}%
\pgfpathlineto{\pgfqpoint{2.799301in}{1.990697in}}%
\pgfpathlineto{\pgfqpoint{2.871562in}{1.960884in}}%
\pgfpathlineto{\pgfqpoint{2.943823in}{1.930717in}}%
\pgfpathlineto{\pgfqpoint{3.016084in}{1.900197in}}%
\pgfpathlineto{\pgfqpoint{3.088345in}{1.869322in}}%
\pgfpathlineto{\pgfqpoint{3.160606in}{1.838094in}}%
\pgfpathlineto{\pgfqpoint{3.232867in}{1.806511in}}%
\pgfpathlineto{\pgfqpoint{3.305128in}{1.774575in}}%
\pgfpathlineto{\pgfqpoint{3.377389in}{1.742285in}}%
\pgfpathlineto{\pgfqpoint{3.449650in}{1.709640in}}%
\pgfpathlineto{\pgfqpoint{3.521911in}{1.676642in}}%
\pgfpathlineto{\pgfqpoint{3.594172in}{1.643290in}}%
\pgfpathlineto{\pgfqpoint{3.666434in}{1.609584in}}%
\pgfpathlineto{\pgfqpoint{3.738695in}{1.575524in}}%
\pgfpathlineto{\pgfqpoint{3.883217in}{1.506460in}}%
\pgfpathlineto{\pgfqpoint{3.955478in}{1.538202in}}%
\pgfpathlineto{\pgfqpoint{4.027739in}{1.569590in}}%
\pgfpathlineto{\pgfqpoint{4.100000in}{1.600625in}}%
\pgfpathlineto{\pgfqpoint{4.172261in}{1.631305in}}%
\pgfpathlineto{\pgfqpoint{4.244522in}{1.661631in}}%
\pgfpathlineto{\pgfqpoint{4.316783in}{1.691604in}}%
\pgfpathlineto{\pgfqpoint{4.389044in}{1.721222in}}%
\pgfpathlineto{\pgfqpoint{4.461305in}{1.750487in}}%
\pgfpathlineto{\pgfqpoint{4.533566in}{1.779398in}}%
\pgfpathlineto{\pgfqpoint{4.605828in}{1.807954in}}%
\pgfpathlineto{\pgfqpoint{4.678089in}{1.836157in}}%
\pgfpathlineto{\pgfqpoint{4.750350in}{1.864006in}}%
\pgfpathlineto{\pgfqpoint{4.822611in}{1.891501in}}%
\pgfpathlineto{\pgfqpoint{4.894872in}{1.918642in}}%
\pgfpathlineto{\pgfqpoint{4.967133in}{1.945429in}}%
\pgfpathlineto{\pgfqpoint{5.039394in}{1.971862in}}%
\pgfpathlineto{\pgfqpoint{5.111655in}{1.997941in}}%
\pgfpathlineto{\pgfqpoint{5.183916in}{2.023666in}}%
\pgfpathlineto{\pgfqpoint{5.256177in}{2.049037in}}%
\pgfpathlineto{\pgfqpoint{5.328438in}{2.074054in}}%
\pgfpathlineto{\pgfqpoint{5.400699in}{2.098717in}}%
\pgfpathlineto{\pgfqpoint{5.472960in}{2.123026in}}%
\pgfpathlineto{\pgfqpoint{5.545221in}{2.146982in}}%
\pgfpathlineto{\pgfqpoint{5.617483in}{2.170583in}}%
\pgfpathlineto{\pgfqpoint{5.689744in}{2.193831in}}%
\pgfpathlineto{\pgfqpoint{5.762005in}{2.216724in}}%
\pgfpathlineto{\pgfqpoint{5.834266in}{2.239264in}}%
\pgfpathlineto{\pgfqpoint{5.906527in}{2.261449in}}%
\pgfpathlineto{\pgfqpoint{5.978788in}{2.283281in}}%
\pgfpathlineto{\pgfqpoint{6.051049in}{2.304758in}}%
\pgfpathlineto{\pgfqpoint{6.123310in}{2.325882in}}%
\pgfpathlineto{\pgfqpoint{6.195571in}{2.346652in}}%
\pgfpathlineto{\pgfqpoint{6.267832in}{2.367068in}}%
\pgfpathlineto{\pgfqpoint{6.340093in}{2.387129in}}%
\pgfpathlineto{\pgfqpoint{6.412354in}{2.406837in}}%
\pgfpathlineto{\pgfqpoint{6.484615in}{2.426191in}}%
\pgfpathlineto{\pgfqpoint{6.556876in}{2.445191in}}%
\pgfpathlineto{\pgfqpoint{6.629138in}{2.463837in}}%
\pgfpathlineto{\pgfqpoint{6.701399in}{2.482129in}}%
\pgfpathlineto{\pgfqpoint{6.773660in}{2.500068in}}%
\pgfpathlineto{\pgfqpoint{6.918182in}{2.535000in}}%
\pgfpathlineto{\pgfqpoint{6.918182in}{2.535000in}}%
\pgfusepath{stroke}%
\end{pgfscope}%
\begin{pgfscope}%
\pgfpathrectangle{\pgfqpoint{1.000000in}{0.330000in}}{\pgfqpoint{6.200000in}{2.310000in}}%
\pgfusepath{clip}%
\pgfsetrectcap%
\pgfsetroundjoin%
\pgfsetlinewidth{1.505625pt}%
\definecolor{currentstroke}{rgb}{0.890196,0.466667,0.760784}%
\pgfsetstrokecolor{currentstroke}%
\pgfsetdash{}{0pt}%
\pgfpathmoveto{\pgfqpoint{1.281818in}{2.535000in}}%
\pgfpathlineto{\pgfqpoint{1.281818in}{2.535000in}}%
\pgfpathlineto{\pgfqpoint{1.354079in}{2.487202in}}%
\pgfpathlineto{\pgfqpoint{1.426340in}{2.438802in}}%
\pgfpathlineto{\pgfqpoint{1.498601in}{2.389799in}}%
\pgfpathlineto{\pgfqpoint{1.570862in}{2.340194in}}%
\pgfpathlineto{\pgfqpoint{1.643124in}{2.289987in}}%
\pgfpathlineto{\pgfqpoint{1.715385in}{2.239177in}}%
\pgfpathlineto{\pgfqpoint{1.787646in}{2.187765in}}%
\pgfpathlineto{\pgfqpoint{1.859907in}{2.135751in}}%
\pgfpathlineto{\pgfqpoint{1.932168in}{2.083134in}}%
\pgfpathlineto{\pgfqpoint{2.004429in}{2.029915in}}%
\pgfpathlineto{\pgfqpoint{2.076690in}{1.976094in}}%
\pgfpathlineto{\pgfqpoint{2.148951in}{1.921670in}}%
\pgfpathlineto{\pgfqpoint{2.221212in}{1.866644in}}%
\pgfpathlineto{\pgfqpoint{2.293473in}{1.811015in}}%
\pgfpathlineto{\pgfqpoint{2.365734in}{1.754785in}}%
\pgfpathlineto{\pgfqpoint{2.437995in}{1.697952in}}%
\pgfpathlineto{\pgfqpoint{2.510256in}{1.640516in}}%
\pgfpathlineto{\pgfqpoint{2.582517in}{1.582479in}}%
\pgfpathlineto{\pgfqpoint{2.654779in}{1.523838in}}%
\pgfpathlineto{\pgfqpoint{2.727040in}{1.464596in}}%
\pgfpathlineto{\pgfqpoint{2.799301in}{1.404751in}}%
\pgfpathlineto{\pgfqpoint{2.871562in}{1.344304in}}%
\pgfpathlineto{\pgfqpoint{2.943823in}{1.283255in}}%
\pgfpathlineto{\pgfqpoint{3.016084in}{1.221603in}}%
\pgfpathlineto{\pgfqpoint{3.088345in}{1.159349in}}%
\pgfpathlineto{\pgfqpoint{3.160606in}{1.096492in}}%
\pgfpathlineto{\pgfqpoint{3.232867in}{1.033034in}}%
\pgfpathlineto{\pgfqpoint{3.305128in}{0.968972in}}%
\pgfpathlineto{\pgfqpoint{3.377389in}{0.904309in}}%
\pgfpathlineto{\pgfqpoint{3.449650in}{0.839043in}}%
\pgfpathlineto{\pgfqpoint{3.521911in}{0.773175in}}%
\pgfpathlineto{\pgfqpoint{3.594172in}{0.706705in}}%
\pgfpathlineto{\pgfqpoint{3.666434in}{0.639632in}}%
\pgfpathlineto{\pgfqpoint{3.738695in}{0.571957in}}%
\pgfpathlineto{\pgfqpoint{3.883217in}{0.435000in}}%
\pgfpathlineto{\pgfqpoint{3.955478in}{0.492253in}}%
\pgfpathlineto{\pgfqpoint{4.027739in}{0.549152in}}%
\pgfpathlineto{\pgfqpoint{4.100000in}{0.605698in}}%
\pgfpathlineto{\pgfqpoint{4.172261in}{0.661889in}}%
\pgfpathlineto{\pgfqpoint{4.244522in}{0.717726in}}%
\pgfpathlineto{\pgfqpoint{4.316783in}{0.773210in}}%
\pgfpathlineto{\pgfqpoint{4.389044in}{0.828339in}}%
\pgfpathlineto{\pgfqpoint{4.461305in}{0.883115in}}%
\pgfpathlineto{\pgfqpoint{4.533566in}{0.937537in}}%
\pgfpathlineto{\pgfqpoint{4.605828in}{0.991604in}}%
\pgfpathlineto{\pgfqpoint{4.678089in}{1.045318in}}%
\pgfpathlineto{\pgfqpoint{4.750350in}{1.098678in}}%
\pgfpathlineto{\pgfqpoint{4.822611in}{1.151683in}}%
\pgfpathlineto{\pgfqpoint{4.894872in}{1.204335in}}%
\pgfpathlineto{\pgfqpoint{4.967133in}{1.256633in}}%
\pgfpathlineto{\pgfqpoint{5.039394in}{1.308577in}}%
\pgfpathlineto{\pgfqpoint{5.111655in}{1.360167in}}%
\pgfpathlineto{\pgfqpoint{5.183916in}{1.411403in}}%
\pgfpathlineto{\pgfqpoint{5.256177in}{1.462285in}}%
\pgfpathlineto{\pgfqpoint{5.328438in}{1.512813in}}%
\pgfpathlineto{\pgfqpoint{5.400699in}{1.562987in}}%
\pgfpathlineto{\pgfqpoint{5.472960in}{1.612808in}}%
\pgfpathlineto{\pgfqpoint{5.545221in}{1.662274in}}%
\pgfpathlineto{\pgfqpoint{5.617483in}{1.711386in}}%
\pgfpathlineto{\pgfqpoint{5.689744in}{1.760145in}}%
\pgfpathlineto{\pgfqpoint{5.762005in}{1.808549in}}%
\pgfpathlineto{\pgfqpoint{5.834266in}{1.856599in}}%
\pgfpathlineto{\pgfqpoint{5.906527in}{1.904296in}}%
\pgfpathlineto{\pgfqpoint{5.978788in}{1.951638in}}%
\pgfpathlineto{\pgfqpoint{6.051049in}{1.998627in}}%
\pgfpathlineto{\pgfqpoint{6.123310in}{2.045262in}}%
\pgfpathlineto{\pgfqpoint{6.195571in}{2.091542in}}%
\pgfpathlineto{\pgfqpoint{6.267832in}{2.137469in}}%
\pgfpathlineto{\pgfqpoint{6.340093in}{2.183042in}}%
\pgfpathlineto{\pgfqpoint{6.412354in}{2.228261in}}%
\pgfpathlineto{\pgfqpoint{6.484615in}{2.273126in}}%
\pgfpathlineto{\pgfqpoint{6.556876in}{2.317637in}}%
\pgfpathlineto{\pgfqpoint{6.629138in}{2.361794in}}%
\pgfpathlineto{\pgfqpoint{6.701399in}{2.405597in}}%
\pgfpathlineto{\pgfqpoint{6.773660in}{2.449046in}}%
\pgfpathlineto{\pgfqpoint{6.918182in}{2.535000in}}%
\pgfpathlineto{\pgfqpoint{6.918182in}{2.535000in}}%
\pgfusepath{stroke}%
\end{pgfscope}%
\begin{pgfscope}%
\pgfpathrectangle{\pgfqpoint{1.000000in}{0.330000in}}{\pgfqpoint{6.200000in}{2.310000in}}%
\pgfusepath{clip}%
\pgfsetrectcap%
\pgfsetroundjoin%
\pgfsetlinewidth{1.505625pt}%
\definecolor{currentstroke}{rgb}{0.498039,0.498039,0.498039}%
\pgfsetstrokecolor{currentstroke}%
\pgfsetdash{}{0pt}%
\pgfpathmoveto{\pgfqpoint{1.281818in}{2.535000in}}%
\pgfpathlineto{\pgfqpoint{1.281818in}{2.535000in}}%
\pgfpathlineto{\pgfqpoint{1.354079in}{2.487202in}}%
\pgfpathlineto{\pgfqpoint{1.426340in}{2.438802in}}%
\pgfpathlineto{\pgfqpoint{1.498601in}{2.389799in}}%
\pgfpathlineto{\pgfqpoint{1.570862in}{2.340194in}}%
\pgfpathlineto{\pgfqpoint{1.643124in}{2.289987in}}%
\pgfpathlineto{\pgfqpoint{1.715385in}{2.239177in}}%
\pgfpathlineto{\pgfqpoint{1.787646in}{2.187765in}}%
\pgfpathlineto{\pgfqpoint{1.859907in}{2.135751in}}%
\pgfpathlineto{\pgfqpoint{1.932168in}{2.083134in}}%
\pgfpathlineto{\pgfqpoint{2.004429in}{2.029915in}}%
\pgfpathlineto{\pgfqpoint{2.076690in}{1.976094in}}%
\pgfpathlineto{\pgfqpoint{2.148951in}{1.921670in}}%
\pgfpathlineto{\pgfqpoint{2.221212in}{1.866644in}}%
\pgfpathlineto{\pgfqpoint{2.293473in}{1.811015in}}%
\pgfpathlineto{\pgfqpoint{2.365734in}{1.754785in}}%
\pgfpathlineto{\pgfqpoint{2.437995in}{1.697952in}}%
\pgfpathlineto{\pgfqpoint{2.510256in}{1.640516in}}%
\pgfpathlineto{\pgfqpoint{2.582517in}{1.582479in}}%
\pgfpathlineto{\pgfqpoint{2.654779in}{1.523838in}}%
\pgfpathlineto{\pgfqpoint{2.727040in}{1.464596in}}%
\pgfpathlineto{\pgfqpoint{2.799301in}{1.404751in}}%
\pgfpathlineto{\pgfqpoint{2.871562in}{1.344304in}}%
\pgfpathlineto{\pgfqpoint{2.943823in}{1.283255in}}%
\pgfpathlineto{\pgfqpoint{3.016084in}{1.221603in}}%
\pgfpathlineto{\pgfqpoint{3.088345in}{1.159349in}}%
\pgfpathlineto{\pgfqpoint{3.160606in}{1.096492in}}%
\pgfpathlineto{\pgfqpoint{3.232867in}{1.033034in}}%
\pgfpathlineto{\pgfqpoint{3.305128in}{0.968972in}}%
\pgfpathlineto{\pgfqpoint{3.377389in}{0.904309in}}%
\pgfpathlineto{\pgfqpoint{3.449650in}{0.839043in}}%
\pgfpathlineto{\pgfqpoint{3.521911in}{0.773175in}}%
\pgfpathlineto{\pgfqpoint{3.594172in}{0.706705in}}%
\pgfpathlineto{\pgfqpoint{3.666434in}{0.639632in}}%
\pgfpathlineto{\pgfqpoint{3.738695in}{0.571957in}}%
\pgfpathlineto{\pgfqpoint{3.883217in}{0.435000in}}%
\pgfpathlineto{\pgfqpoint{3.955478in}{0.497343in}}%
\pgfpathlineto{\pgfqpoint{4.027739in}{0.559084in}}%
\pgfpathlineto{\pgfqpoint{4.100000in}{0.620222in}}%
\pgfpathlineto{\pgfqpoint{4.172261in}{0.680759in}}%
\pgfpathlineto{\pgfqpoint{4.244522in}{0.740692in}}%
\pgfpathlineto{\pgfqpoint{4.316783in}{0.800024in}}%
\pgfpathlineto{\pgfqpoint{4.389044in}{0.858753in}}%
\pgfpathlineto{\pgfqpoint{4.461305in}{0.916880in}}%
\pgfpathlineto{\pgfqpoint{4.533566in}{0.974404in}}%
\pgfpathlineto{\pgfqpoint{4.605828in}{1.031326in}}%
\pgfpathlineto{\pgfqpoint{4.678089in}{1.087646in}}%
\pgfpathlineto{\pgfqpoint{4.750350in}{1.143364in}}%
\pgfpathlineto{\pgfqpoint{4.822611in}{1.198479in}}%
\pgfpathlineto{\pgfqpoint{4.894872in}{1.252992in}}%
\pgfpathlineto{\pgfqpoint{4.967133in}{1.306902in}}%
\pgfpathlineto{\pgfqpoint{5.039394in}{1.360210in}}%
\pgfpathlineto{\pgfqpoint{5.111655in}{1.412916in}}%
\pgfpathlineto{\pgfqpoint{5.183916in}{1.465019in}}%
\pgfpathlineto{\pgfqpoint{5.256177in}{1.516520in}}%
\pgfpathlineto{\pgfqpoint{5.328438in}{1.567419in}}%
\pgfpathlineto{\pgfqpoint{5.400699in}{1.617715in}}%
\pgfpathlineto{\pgfqpoint{5.472960in}{1.667409in}}%
\pgfpathlineto{\pgfqpoint{5.545221in}{1.716501in}}%
\pgfpathlineto{\pgfqpoint{5.617483in}{1.764990in}}%
\pgfpathlineto{\pgfqpoint{5.689744in}{1.812878in}}%
\pgfpathlineto{\pgfqpoint{5.762005in}{1.860162in}}%
\pgfpathlineto{\pgfqpoint{5.834266in}{1.906845in}}%
\pgfpathlineto{\pgfqpoint{5.906527in}{1.952925in}}%
\pgfpathlineto{\pgfqpoint{5.978788in}{1.998402in}}%
\pgfpathlineto{\pgfqpoint{6.051049in}{2.043278in}}%
\pgfpathlineto{\pgfqpoint{6.123310in}{2.087551in}}%
\pgfpathlineto{\pgfqpoint{6.195571in}{2.131221in}}%
\pgfpathlineto{\pgfqpoint{6.267832in}{2.174290in}}%
\pgfpathlineto{\pgfqpoint{6.340093in}{2.216756in}}%
\pgfpathlineto{\pgfqpoint{6.412354in}{2.258619in}}%
\pgfpathlineto{\pgfqpoint{6.484615in}{2.299881in}}%
\pgfpathlineto{\pgfqpoint{6.556876in}{2.340539in}}%
\pgfpathlineto{\pgfqpoint{6.629138in}{2.380596in}}%
\pgfpathlineto{\pgfqpoint{6.701399in}{2.420050in}}%
\pgfpathlineto{\pgfqpoint{6.773660in}{2.458902in}}%
\pgfpathlineto{\pgfqpoint{6.918182in}{2.535000in}}%
\pgfpathlineto{\pgfqpoint{6.918182in}{2.535000in}}%
\pgfusepath{stroke}%
\end{pgfscope}%
\begin{pgfscope}%
\pgfpathrectangle{\pgfqpoint{1.000000in}{0.330000in}}{\pgfqpoint{6.200000in}{2.310000in}}%
\pgfusepath{clip}%
\pgfsetrectcap%
\pgfsetroundjoin%
\pgfsetlinewidth{1.505625pt}%
\definecolor{currentstroke}{rgb}{0.737255,0.741176,0.133333}%
\pgfsetstrokecolor{currentstroke}%
\pgfsetdash{}{0pt}%
\pgfpathmoveto{\pgfqpoint{1.281818in}{2.535000in}}%
\pgfpathlineto{\pgfqpoint{1.281818in}{2.535000in}}%
\pgfpathlineto{\pgfqpoint{1.354079in}{2.504398in}}%
\pgfpathlineto{\pgfqpoint{1.426340in}{2.473364in}}%
\pgfpathlineto{\pgfqpoint{1.498601in}{2.441898in}}%
\pgfpathlineto{\pgfqpoint{1.570862in}{2.410001in}}%
\pgfpathlineto{\pgfqpoint{1.643124in}{2.377673in}}%
\pgfpathlineto{\pgfqpoint{1.715385in}{2.344912in}}%
\pgfpathlineto{\pgfqpoint{1.787646in}{2.311721in}}%
\pgfpathlineto{\pgfqpoint{1.859907in}{2.278097in}}%
\pgfpathlineto{\pgfqpoint{1.932168in}{2.244042in}}%
\pgfpathlineto{\pgfqpoint{2.004429in}{2.209556in}}%
\pgfpathlineto{\pgfqpoint{2.076690in}{2.174638in}}%
\pgfpathlineto{\pgfqpoint{2.148951in}{2.139288in}}%
\pgfpathlineto{\pgfqpoint{2.221212in}{2.103507in}}%
\pgfpathlineto{\pgfqpoint{2.293473in}{2.067294in}}%
\pgfpathlineto{\pgfqpoint{2.365734in}{2.030650in}}%
\pgfpathlineto{\pgfqpoint{2.437995in}{1.993574in}}%
\pgfpathlineto{\pgfqpoint{2.510256in}{1.956066in}}%
\pgfpathlineto{\pgfqpoint{2.582517in}{1.918127in}}%
\pgfpathlineto{\pgfqpoint{2.654779in}{1.879757in}}%
\pgfpathlineto{\pgfqpoint{2.727040in}{1.840954in}}%
\pgfpathlineto{\pgfqpoint{2.799301in}{1.801721in}}%
\pgfpathlineto{\pgfqpoint{2.871562in}{1.762055in}}%
\pgfpathlineto{\pgfqpoint{2.943823in}{1.721958in}}%
\pgfpathlineto{\pgfqpoint{3.016084in}{1.681430in}}%
\pgfpathlineto{\pgfqpoint{3.088345in}{1.640470in}}%
\pgfpathlineto{\pgfqpoint{3.160606in}{1.599078in}}%
\pgfpathlineto{\pgfqpoint{3.232867in}{1.557255in}}%
\pgfpathlineto{\pgfqpoint{3.305128in}{1.515000in}}%
\pgfpathlineto{\pgfqpoint{3.377389in}{1.472314in}}%
\pgfpathlineto{\pgfqpoint{3.449650in}{1.429196in}}%
\pgfpathlineto{\pgfqpoint{3.521911in}{1.385646in}}%
\pgfpathlineto{\pgfqpoint{3.594172in}{1.341665in}}%
\pgfpathlineto{\pgfqpoint{3.666434in}{1.297252in}}%
\pgfpathlineto{\pgfqpoint{3.738695in}{1.252408in}}%
\pgfpathlineto{\pgfqpoint{3.883217in}{1.161569in}}%
\pgfpathlineto{\pgfqpoint{3.955478in}{1.201523in}}%
\pgfpathlineto{\pgfqpoint{4.027739in}{1.241123in}}%
\pgfpathlineto{\pgfqpoint{4.100000in}{1.280369in}}%
\pgfpathlineto{\pgfqpoint{4.172261in}{1.319261in}}%
\pgfpathlineto{\pgfqpoint{4.244522in}{1.357799in}}%
\pgfpathlineto{\pgfqpoint{4.316783in}{1.395983in}}%
\pgfpathlineto{\pgfqpoint{4.389044in}{1.433813in}}%
\pgfpathlineto{\pgfqpoint{4.461305in}{1.471290in}}%
\pgfpathlineto{\pgfqpoint{4.533566in}{1.508412in}}%
\pgfpathlineto{\pgfqpoint{4.605828in}{1.545180in}}%
\pgfpathlineto{\pgfqpoint{4.678089in}{1.581595in}}%
\pgfpathlineto{\pgfqpoint{4.750350in}{1.617655in}}%
\pgfpathlineto{\pgfqpoint{4.822611in}{1.653362in}}%
\pgfpathlineto{\pgfqpoint{4.894872in}{1.688714in}}%
\pgfpathlineto{\pgfqpoint{4.967133in}{1.723713in}}%
\pgfpathlineto{\pgfqpoint{5.039394in}{1.758358in}}%
\pgfpathlineto{\pgfqpoint{5.111655in}{1.792648in}}%
\pgfpathlineto{\pgfqpoint{5.183916in}{1.826585in}}%
\pgfpathlineto{\pgfqpoint{5.256177in}{1.860168in}}%
\pgfpathlineto{\pgfqpoint{5.328438in}{1.893397in}}%
\pgfpathlineto{\pgfqpoint{5.400699in}{1.926272in}}%
\pgfpathlineto{\pgfqpoint{5.472960in}{1.958793in}}%
\pgfpathlineto{\pgfqpoint{5.545221in}{1.990960in}}%
\pgfpathlineto{\pgfqpoint{5.617483in}{2.022773in}}%
\pgfpathlineto{\pgfqpoint{5.689744in}{2.054232in}}%
\pgfpathlineto{\pgfqpoint{5.762005in}{2.085337in}}%
\pgfpathlineto{\pgfqpoint{5.834266in}{2.116088in}}%
\pgfpathlineto{\pgfqpoint{5.906527in}{2.146485in}}%
\pgfpathlineto{\pgfqpoint{5.978788in}{2.176529in}}%
\pgfpathlineto{\pgfqpoint{6.051049in}{2.206218in}}%
\pgfpathlineto{\pgfqpoint{6.123310in}{2.235554in}}%
\pgfpathlineto{\pgfqpoint{6.195571in}{2.264535in}}%
\pgfpathlineto{\pgfqpoint{6.267832in}{2.293162in}}%
\pgfpathlineto{\pgfqpoint{6.340093in}{2.321436in}}%
\pgfpathlineto{\pgfqpoint{6.412354in}{2.349356in}}%
\pgfpathlineto{\pgfqpoint{6.484615in}{2.376921in}}%
\pgfpathlineto{\pgfqpoint{6.556876in}{2.404133in}}%
\pgfpathlineto{\pgfqpoint{6.629138in}{2.430991in}}%
\pgfpathlineto{\pgfqpoint{6.701399in}{2.457494in}}%
\pgfpathlineto{\pgfqpoint{6.773660in}{2.483644in}}%
\pgfpathlineto{\pgfqpoint{6.918182in}{2.535000in}}%
\pgfpathlineto{\pgfqpoint{6.918182in}{2.535000in}}%
\pgfusepath{stroke}%
\end{pgfscope}%
\begin{pgfscope}%
\pgfpathrectangle{\pgfqpoint{1.000000in}{0.330000in}}{\pgfqpoint{6.200000in}{2.310000in}}%
\pgfusepath{clip}%
\pgfsetrectcap%
\pgfsetroundjoin%
\pgfsetlinewidth{1.505625pt}%
\definecolor{currentstroke}{rgb}{0.090196,0.745098,0.811765}%
\pgfsetstrokecolor{currentstroke}%
\pgfsetdash{}{0pt}%
\pgfpathmoveto{\pgfqpoint{1.281818in}{2.535000in}}%
\pgfpathlineto{\pgfqpoint{1.281818in}{2.535000in}}%
\pgfpathlineto{\pgfqpoint{1.354079in}{2.487668in}}%
\pgfpathlineto{\pgfqpoint{1.426340in}{2.439733in}}%
\pgfpathlineto{\pgfqpoint{1.498601in}{2.391196in}}%
\pgfpathlineto{\pgfqpoint{1.570862in}{2.342057in}}%
\pgfpathlineto{\pgfqpoint{1.643124in}{2.292315in}}%
\pgfpathlineto{\pgfqpoint{1.715385in}{2.241971in}}%
\pgfpathlineto{\pgfqpoint{1.787646in}{2.191025in}}%
\pgfpathlineto{\pgfqpoint{1.859907in}{2.139476in}}%
\pgfpathlineto{\pgfqpoint{1.932168in}{2.087326in}}%
\pgfpathlineto{\pgfqpoint{2.004429in}{2.034572in}}%
\pgfpathlineto{\pgfqpoint{2.076690in}{1.981217in}}%
\pgfpathlineto{\pgfqpoint{2.148951in}{1.927259in}}%
\pgfpathlineto{\pgfqpoint{2.221212in}{1.872698in}}%
\pgfpathlineto{\pgfqpoint{2.293473in}{1.817536in}}%
\pgfpathlineto{\pgfqpoint{2.365734in}{1.761771in}}%
\pgfpathlineto{\pgfqpoint{2.437995in}{1.705403in}}%
\pgfpathlineto{\pgfqpoint{2.510256in}{1.648434in}}%
\pgfpathlineto{\pgfqpoint{2.582517in}{1.590862in}}%
\pgfpathlineto{\pgfqpoint{2.654779in}{1.532687in}}%
\pgfpathlineto{\pgfqpoint{2.727040in}{1.473911in}}%
\pgfpathlineto{\pgfqpoint{2.799301in}{1.414532in}}%
\pgfpathlineto{\pgfqpoint{2.871562in}{1.354550in}}%
\pgfpathlineto{\pgfqpoint{2.943823in}{1.293966in}}%
\pgfpathlineto{\pgfqpoint{3.016084in}{1.232780in}}%
\pgfpathlineto{\pgfqpoint{3.088345in}{1.170992in}}%
\pgfpathlineto{\pgfqpoint{3.160606in}{1.108601in}}%
\pgfpathlineto{\pgfqpoint{3.232867in}{1.045608in}}%
\pgfpathlineto{\pgfqpoint{3.305128in}{0.982013in}}%
\pgfpathlineto{\pgfqpoint{3.377389in}{0.917815in}}%
\pgfpathlineto{\pgfqpoint{3.449650in}{0.853015in}}%
\pgfpathlineto{\pgfqpoint{3.521911in}{0.787613in}}%
\pgfpathlineto{\pgfqpoint{3.594172in}{0.721608in}}%
\pgfpathlineto{\pgfqpoint{3.666434in}{0.655001in}}%
\pgfpathlineto{\pgfqpoint{3.738695in}{0.587791in}}%
\pgfpathlineto{\pgfqpoint{3.883217in}{0.451766in}}%
\pgfpathlineto{\pgfqpoint{3.955478in}{0.513710in}}%
\pgfpathlineto{\pgfqpoint{4.027739in}{0.575052in}}%
\pgfpathlineto{\pgfqpoint{4.100000in}{0.635791in}}%
\pgfpathlineto{\pgfqpoint{4.172261in}{0.695928in}}%
\pgfpathlineto{\pgfqpoint{4.244522in}{0.755463in}}%
\pgfpathlineto{\pgfqpoint{4.316783in}{0.814395in}}%
\pgfpathlineto{\pgfqpoint{4.389044in}{0.872725in}}%
\pgfpathlineto{\pgfqpoint{4.461305in}{0.930452in}}%
\pgfpathlineto{\pgfqpoint{4.533566in}{0.987578in}}%
\pgfpathlineto{\pgfqpoint{4.605828in}{1.044101in}}%
\pgfpathlineto{\pgfqpoint{4.678089in}{1.100021in}}%
\pgfpathlineto{\pgfqpoint{4.750350in}{1.155339in}}%
\pgfpathlineto{\pgfqpoint{4.822611in}{1.210055in}}%
\pgfpathlineto{\pgfqpoint{4.894872in}{1.264169in}}%
\pgfpathlineto{\pgfqpoint{4.967133in}{1.317680in}}%
\pgfpathlineto{\pgfqpoint{5.039394in}{1.370589in}}%
\pgfpathlineto{\pgfqpoint{5.111655in}{1.422896in}}%
\pgfpathlineto{\pgfqpoint{5.183916in}{1.474600in}}%
\pgfpathlineto{\pgfqpoint{5.256177in}{1.525702in}}%
\pgfpathlineto{\pgfqpoint{5.328438in}{1.576201in}}%
\pgfpathlineto{\pgfqpoint{5.400699in}{1.626098in}}%
\pgfpathlineto{\pgfqpoint{5.472960in}{1.675393in}}%
\pgfpathlineto{\pgfqpoint{5.545221in}{1.724086in}}%
\pgfpathlineto{\pgfqpoint{5.617483in}{1.772176in}}%
\pgfpathlineto{\pgfqpoint{5.689744in}{1.819664in}}%
\pgfpathlineto{\pgfqpoint{5.762005in}{1.866549in}}%
\pgfpathlineto{\pgfqpoint{5.834266in}{1.912832in}}%
\pgfpathlineto{\pgfqpoint{5.906527in}{1.958513in}}%
\pgfpathlineto{\pgfqpoint{5.978788in}{2.003592in}}%
\pgfpathlineto{\pgfqpoint{6.051049in}{2.048068in}}%
\pgfpathlineto{\pgfqpoint{6.123310in}{2.091942in}}%
\pgfpathlineto{\pgfqpoint{6.195571in}{2.135213in}}%
\pgfpathlineto{\pgfqpoint{6.267832in}{2.177882in}}%
\pgfpathlineto{\pgfqpoint{6.340093in}{2.219949in}}%
\pgfpathlineto{\pgfqpoint{6.412354in}{2.261414in}}%
\pgfpathlineto{\pgfqpoint{6.484615in}{2.302276in}}%
\pgfpathlineto{\pgfqpoint{6.556876in}{2.342535in}}%
\pgfpathlineto{\pgfqpoint{6.629138in}{2.382193in}}%
\pgfpathlineto{\pgfqpoint{6.701399in}{2.421248in}}%
\pgfpathlineto{\pgfqpoint{6.773660in}{2.459701in}}%
\pgfpathlineto{\pgfqpoint{6.918182in}{2.535000in}}%
\pgfpathlineto{\pgfqpoint{6.918182in}{2.535000in}}%
\pgfusepath{stroke}%
\end{pgfscope}%
\begin{pgfscope}%
\pgfpathrectangle{\pgfqpoint{1.000000in}{0.330000in}}{\pgfqpoint{6.200000in}{2.310000in}}%
\pgfusepath{clip}%
\pgfsetrectcap%
\pgfsetroundjoin%
\pgfsetlinewidth{1.505625pt}%
\definecolor{currentstroke}{rgb}{0.121569,0.466667,0.705882}%
\pgfsetstrokecolor{currentstroke}%
\pgfsetdash{}{0pt}%
\pgfpathmoveto{\pgfqpoint{1.281818in}{2.535000in}}%
\pgfpathlineto{\pgfqpoint{1.281818in}{2.535000in}}%
\pgfpathlineto{\pgfqpoint{1.354079in}{2.509434in}}%
\pgfpathlineto{\pgfqpoint{1.426340in}{2.483454in}}%
\pgfpathlineto{\pgfqpoint{1.498601in}{2.457062in}}%
\pgfpathlineto{\pgfqpoint{1.570862in}{2.430257in}}%
\pgfpathlineto{\pgfqpoint{1.643124in}{2.403039in}}%
\pgfpathlineto{\pgfqpoint{1.715385in}{2.375408in}}%
\pgfpathlineto{\pgfqpoint{1.787646in}{2.347364in}}%
\pgfpathlineto{\pgfqpoint{1.859907in}{2.318907in}}%
\pgfpathlineto{\pgfqpoint{1.932168in}{2.290037in}}%
\pgfpathlineto{\pgfqpoint{2.004429in}{2.260755in}}%
\pgfpathlineto{\pgfqpoint{2.076690in}{2.231059in}}%
\pgfpathlineto{\pgfqpoint{2.148951in}{2.200950in}}%
\pgfpathlineto{\pgfqpoint{2.221212in}{2.170429in}}%
\pgfpathlineto{\pgfqpoint{2.293473in}{2.139494in}}%
\pgfpathlineto{\pgfqpoint{2.365734in}{2.108147in}}%
\pgfpathlineto{\pgfqpoint{2.437995in}{2.076386in}}%
\pgfpathlineto{\pgfqpoint{2.510256in}{2.044213in}}%
\pgfpathlineto{\pgfqpoint{2.582517in}{2.011626in}}%
\pgfpathlineto{\pgfqpoint{2.654779in}{1.978627in}}%
\pgfpathlineto{\pgfqpoint{2.727040in}{1.945215in}}%
\pgfpathlineto{\pgfqpoint{2.799301in}{1.911390in}}%
\pgfpathlineto{\pgfqpoint{2.871562in}{1.877152in}}%
\pgfpathlineto{\pgfqpoint{2.943823in}{1.842501in}}%
\pgfpathlineto{\pgfqpoint{3.016084in}{1.807437in}}%
\pgfpathlineto{\pgfqpoint{3.088345in}{1.771960in}}%
\pgfpathlineto{\pgfqpoint{3.160606in}{1.736070in}}%
\pgfpathlineto{\pgfqpoint{3.232867in}{1.699767in}}%
\pgfpathlineto{\pgfqpoint{3.305128in}{1.663051in}}%
\pgfpathlineto{\pgfqpoint{3.377389in}{1.625922in}}%
\pgfpathlineto{\pgfqpoint{3.449650in}{1.588381in}}%
\pgfpathlineto{\pgfqpoint{3.521911in}{1.550426in}}%
\pgfpathlineto{\pgfqpoint{3.594172in}{1.512059in}}%
\pgfpathlineto{\pgfqpoint{3.666434in}{1.473278in}}%
\pgfpathlineto{\pgfqpoint{3.738695in}{1.434085in}}%
\pgfpathlineto{\pgfqpoint{3.883217in}{1.354597in}}%
\pgfpathlineto{\pgfqpoint{3.955478in}{1.391164in}}%
\pgfpathlineto{\pgfqpoint{4.027739in}{1.427318in}}%
\pgfpathlineto{\pgfqpoint{4.100000in}{1.463059in}}%
\pgfpathlineto{\pgfqpoint{4.172261in}{1.498387in}}%
\pgfpathlineto{\pgfqpoint{4.244522in}{1.533302in}}%
\pgfpathlineto{\pgfqpoint{4.316783in}{1.567804in}}%
\pgfpathlineto{\pgfqpoint{4.389044in}{1.601893in}}%
\pgfpathlineto{\pgfqpoint{4.461305in}{1.635570in}}%
\pgfpathlineto{\pgfqpoint{4.533566in}{1.668833in}}%
\pgfpathlineto{\pgfqpoint{4.605828in}{1.701683in}}%
\pgfpathlineto{\pgfqpoint{4.678089in}{1.734121in}}%
\pgfpathlineto{\pgfqpoint{4.750350in}{1.766145in}}%
\pgfpathlineto{\pgfqpoint{4.822611in}{1.797757in}}%
\pgfpathlineto{\pgfqpoint{4.894872in}{1.828956in}}%
\pgfpathlineto{\pgfqpoint{4.967133in}{1.859741in}}%
\pgfpathlineto{\pgfqpoint{5.039394in}{1.890114in}}%
\pgfpathlineto{\pgfqpoint{5.111655in}{1.920074in}}%
\pgfpathlineto{\pgfqpoint{5.183916in}{1.949621in}}%
\pgfpathlineto{\pgfqpoint{5.256177in}{1.978755in}}%
\pgfpathlineto{\pgfqpoint{5.328438in}{2.007476in}}%
\pgfpathlineto{\pgfqpoint{5.400699in}{2.035784in}}%
\pgfpathlineto{\pgfqpoint{5.472960in}{2.063679in}}%
\pgfpathlineto{\pgfqpoint{5.545221in}{2.091161in}}%
\pgfpathlineto{\pgfqpoint{5.617483in}{2.118230in}}%
\pgfpathlineto{\pgfqpoint{5.689744in}{2.144886in}}%
\pgfpathlineto{\pgfqpoint{5.762005in}{2.171130in}}%
\pgfpathlineto{\pgfqpoint{5.834266in}{2.196960in}}%
\pgfpathlineto{\pgfqpoint{5.906527in}{2.222377in}}%
\pgfpathlineto{\pgfqpoint{5.978788in}{2.247382in}}%
\pgfpathlineto{\pgfqpoint{6.051049in}{2.271973in}}%
\pgfpathlineto{\pgfqpoint{6.123310in}{2.296152in}}%
\pgfpathlineto{\pgfqpoint{6.195571in}{2.319918in}}%
\pgfpathlineto{\pgfqpoint{6.267832in}{2.343270in}}%
\pgfpathlineto{\pgfqpoint{6.340093in}{2.366210in}}%
\pgfpathlineto{\pgfqpoint{6.412354in}{2.388737in}}%
\pgfpathlineto{\pgfqpoint{6.484615in}{2.410851in}}%
\pgfpathlineto{\pgfqpoint{6.556876in}{2.432552in}}%
\pgfpathlineto{\pgfqpoint{6.629138in}{2.453840in}}%
\pgfpathlineto{\pgfqpoint{6.701399in}{2.474715in}}%
\pgfpathlineto{\pgfqpoint{6.773660in}{2.495177in}}%
\pgfpathlineto{\pgfqpoint{6.918182in}{2.535000in}}%
\pgfpathlineto{\pgfqpoint{6.918182in}{2.535000in}}%
\pgfusepath{stroke}%
\end{pgfscope}%
\begin{pgfscope}%
\pgfpathrectangle{\pgfqpoint{1.000000in}{0.330000in}}{\pgfqpoint{6.200000in}{2.310000in}}%
\pgfusepath{clip}%
\pgfsetrectcap%
\pgfsetroundjoin%
\pgfsetlinewidth{1.505625pt}%
\definecolor{currentstroke}{rgb}{1.000000,0.498039,0.054902}%
\pgfsetstrokecolor{currentstroke}%
\pgfsetdash{}{0pt}%
\pgfpathmoveto{\pgfqpoint{1.281818in}{2.535000in}}%
\pgfpathlineto{\pgfqpoint{1.281818in}{2.535000in}}%
\pgfpathlineto{\pgfqpoint{1.354079in}{2.512341in}}%
\pgfpathlineto{\pgfqpoint{1.426340in}{2.489328in}}%
\pgfpathlineto{\pgfqpoint{1.498601in}{2.465961in}}%
\pgfpathlineto{\pgfqpoint{1.570862in}{2.442240in}}%
\pgfpathlineto{\pgfqpoint{1.643124in}{2.418165in}}%
\pgfpathlineto{\pgfqpoint{1.715385in}{2.393736in}}%
\pgfpathlineto{\pgfqpoint{1.787646in}{2.368953in}}%
\pgfpathlineto{\pgfqpoint{1.859907in}{2.343816in}}%
\pgfpathlineto{\pgfqpoint{1.932168in}{2.318326in}}%
\pgfpathlineto{\pgfqpoint{2.004429in}{2.292481in}}%
\pgfpathlineto{\pgfqpoint{2.076690in}{2.266282in}}%
\pgfpathlineto{\pgfqpoint{2.148951in}{2.239730in}}%
\pgfpathlineto{\pgfqpoint{2.221212in}{2.212823in}}%
\pgfpathlineto{\pgfqpoint{2.293473in}{2.185563in}}%
\pgfpathlineto{\pgfqpoint{2.365734in}{2.157948in}}%
\pgfpathlineto{\pgfqpoint{2.437995in}{2.129980in}}%
\pgfpathlineto{\pgfqpoint{2.510256in}{2.101657in}}%
\pgfpathlineto{\pgfqpoint{2.582517in}{2.072981in}}%
\pgfpathlineto{\pgfqpoint{2.654779in}{2.043951in}}%
\pgfpathlineto{\pgfqpoint{2.727040in}{2.014567in}}%
\pgfpathlineto{\pgfqpoint{2.799301in}{1.984829in}}%
\pgfpathlineto{\pgfqpoint{2.871562in}{1.954736in}}%
\pgfpathlineto{\pgfqpoint{2.943823in}{1.924290in}}%
\pgfpathlineto{\pgfqpoint{3.016084in}{1.893490in}}%
\pgfpathlineto{\pgfqpoint{3.088345in}{1.862336in}}%
\pgfpathlineto{\pgfqpoint{3.160606in}{1.830828in}}%
\pgfpathlineto{\pgfqpoint{3.232867in}{1.798967in}}%
\pgfpathlineto{\pgfqpoint{3.305128in}{1.766751in}}%
\pgfpathlineto{\pgfqpoint{3.377389in}{1.734181in}}%
\pgfpathlineto{\pgfqpoint{3.449650in}{1.701257in}}%
\pgfpathlineto{\pgfqpoint{3.521911in}{1.667980in}}%
\pgfpathlineto{\pgfqpoint{3.594172in}{1.634348in}}%
\pgfpathlineto{\pgfqpoint{3.666434in}{1.600362in}}%
\pgfpathlineto{\pgfqpoint{3.738695in}{1.566023in}}%
\pgfpathlineto{\pgfqpoint{3.883217in}{1.496400in}}%
\pgfpathlineto{\pgfqpoint{3.955478in}{1.528382in}}%
\pgfpathlineto{\pgfqpoint{4.027739in}{1.560010in}}%
\pgfpathlineto{\pgfqpoint{4.100000in}{1.591283in}}%
\pgfpathlineto{\pgfqpoint{4.172261in}{1.622203in}}%
\pgfpathlineto{\pgfqpoint{4.244522in}{1.652769in}}%
\pgfpathlineto{\pgfqpoint{4.316783in}{1.682981in}}%
\pgfpathlineto{\pgfqpoint{4.389044in}{1.712839in}}%
\pgfpathlineto{\pgfqpoint{4.461305in}{1.742343in}}%
\pgfpathlineto{\pgfqpoint{4.533566in}{1.771494in}}%
\pgfpathlineto{\pgfqpoint{4.605828in}{1.800290in}}%
\pgfpathlineto{\pgfqpoint{4.678089in}{1.828732in}}%
\pgfpathlineto{\pgfqpoint{4.750350in}{1.856820in}}%
\pgfpathlineto{\pgfqpoint{4.822611in}{1.884555in}}%
\pgfpathlineto{\pgfqpoint{4.894872in}{1.911935in}}%
\pgfpathlineto{\pgfqpoint{4.967133in}{1.938962in}}%
\pgfpathlineto{\pgfqpoint{5.039394in}{1.965634in}}%
\pgfpathlineto{\pgfqpoint{5.111655in}{1.991953in}}%
\pgfpathlineto{\pgfqpoint{5.183916in}{2.017917in}}%
\pgfpathlineto{\pgfqpoint{5.256177in}{2.043528in}}%
\pgfpathlineto{\pgfqpoint{5.328438in}{2.068785in}}%
\pgfpathlineto{\pgfqpoint{5.400699in}{2.093687in}}%
\pgfpathlineto{\pgfqpoint{5.472960in}{2.118236in}}%
\pgfpathlineto{\pgfqpoint{5.545221in}{2.142431in}}%
\pgfpathlineto{\pgfqpoint{5.617483in}{2.166272in}}%
\pgfpathlineto{\pgfqpoint{5.689744in}{2.189759in}}%
\pgfpathlineto{\pgfqpoint{5.762005in}{2.212892in}}%
\pgfpathlineto{\pgfqpoint{5.834266in}{2.235671in}}%
\pgfpathlineto{\pgfqpoint{5.906527in}{2.258096in}}%
\pgfpathlineto{\pgfqpoint{5.978788in}{2.280167in}}%
\pgfpathlineto{\pgfqpoint{6.051049in}{2.301884in}}%
\pgfpathlineto{\pgfqpoint{6.123310in}{2.323247in}}%
\pgfpathlineto{\pgfqpoint{6.195571in}{2.344257in}}%
\pgfpathlineto{\pgfqpoint{6.267832in}{2.364912in}}%
\pgfpathlineto{\pgfqpoint{6.340093in}{2.385213in}}%
\pgfpathlineto{\pgfqpoint{6.412354in}{2.405161in}}%
\pgfpathlineto{\pgfqpoint{6.484615in}{2.424754in}}%
\pgfpathlineto{\pgfqpoint{6.556876in}{2.443994in}}%
\pgfpathlineto{\pgfqpoint{6.629138in}{2.462879in}}%
\pgfpathlineto{\pgfqpoint{6.701399in}{2.481411in}}%
\pgfpathlineto{\pgfqpoint{6.773660in}{2.499589in}}%
\pgfpathlineto{\pgfqpoint{6.918182in}{2.535000in}}%
\pgfpathlineto{\pgfqpoint{6.918182in}{2.535000in}}%
\pgfusepath{stroke}%
\end{pgfscope}%
\begin{pgfscope}%
\pgfpathrectangle{\pgfqpoint{1.000000in}{0.330000in}}{\pgfqpoint{6.200000in}{2.310000in}}%
\pgfusepath{clip}%
\pgfsetrectcap%
\pgfsetroundjoin%
\pgfsetlinewidth{1.505625pt}%
\definecolor{currentstroke}{rgb}{0.172549,0.627451,0.172549}%
\pgfsetstrokecolor{currentstroke}%
\pgfsetdash{}{0pt}%
\pgfpathmoveto{\pgfqpoint{1.281818in}{2.535000in}}%
\pgfpathlineto{\pgfqpoint{1.281818in}{2.535000in}}%
\pgfpathlineto{\pgfqpoint{1.354079in}{2.504677in}}%
\pgfpathlineto{\pgfqpoint{1.426340in}{2.473923in}}%
\pgfpathlineto{\pgfqpoint{1.498601in}{2.442737in}}%
\pgfpathlineto{\pgfqpoint{1.570862in}{2.411119in}}%
\pgfpathlineto{\pgfqpoint{1.643124in}{2.379070in}}%
\pgfpathlineto{\pgfqpoint{1.715385in}{2.346589in}}%
\pgfpathlineto{\pgfqpoint{1.787646in}{2.313677in}}%
\pgfpathlineto{\pgfqpoint{1.859907in}{2.280333in}}%
\pgfpathlineto{\pgfqpoint{1.932168in}{2.246557in}}%
\pgfpathlineto{\pgfqpoint{2.004429in}{2.212350in}}%
\pgfpathlineto{\pgfqpoint{2.076690in}{2.177712in}}%
\pgfpathlineto{\pgfqpoint{2.148951in}{2.142641in}}%
\pgfpathlineto{\pgfqpoint{2.221212in}{2.107140in}}%
\pgfpathlineto{\pgfqpoint{2.293473in}{2.071206in}}%
\pgfpathlineto{\pgfqpoint{2.365734in}{2.034841in}}%
\pgfpathlineto{\pgfqpoint{2.437995in}{1.998045in}}%
\pgfpathlineto{\pgfqpoint{2.510256in}{1.960817in}}%
\pgfpathlineto{\pgfqpoint{2.582517in}{1.923157in}}%
\pgfpathlineto{\pgfqpoint{2.654779in}{1.885066in}}%
\pgfpathlineto{\pgfqpoint{2.727040in}{1.846543in}}%
\pgfpathlineto{\pgfqpoint{2.799301in}{1.807589in}}%
\pgfpathlineto{\pgfqpoint{2.871562in}{1.768203in}}%
\pgfpathlineto{\pgfqpoint{2.943823in}{1.728385in}}%
\pgfpathlineto{\pgfqpoint{3.016084in}{1.688136in}}%
\pgfpathlineto{\pgfqpoint{3.088345in}{1.647456in}}%
\pgfpathlineto{\pgfqpoint{3.160606in}{1.606343in}}%
\pgfpathlineto{\pgfqpoint{3.232867in}{1.564800in}}%
\pgfpathlineto{\pgfqpoint{3.305128in}{1.522824in}}%
\pgfpathlineto{\pgfqpoint{3.377389in}{1.480417in}}%
\pgfpathlineto{\pgfqpoint{3.449650in}{1.437579in}}%
\pgfpathlineto{\pgfqpoint{3.521911in}{1.394309in}}%
\pgfpathlineto{\pgfqpoint{3.594172in}{1.350607in}}%
\pgfpathlineto{\pgfqpoint{3.666434in}{1.306474in}}%
\pgfpathlineto{\pgfqpoint{3.738695in}{1.261909in}}%
\pgfpathlineto{\pgfqpoint{3.883217in}{1.171628in}}%
\pgfpathlineto{\pgfqpoint{3.955478in}{1.212934in}}%
\pgfpathlineto{\pgfqpoint{4.027739in}{1.253807in}}%
\pgfpathlineto{\pgfqpoint{4.100000in}{1.294249in}}%
\pgfpathlineto{\pgfqpoint{4.172261in}{1.334259in}}%
\pgfpathlineto{\pgfqpoint{4.244522in}{1.373838in}}%
\pgfpathlineto{\pgfqpoint{4.316783in}{1.412985in}}%
\pgfpathlineto{\pgfqpoint{4.389044in}{1.451701in}}%
\pgfpathlineto{\pgfqpoint{4.461305in}{1.489985in}}%
\pgfpathlineto{\pgfqpoint{4.533566in}{1.527837in}}%
\pgfpathlineto{\pgfqpoint{4.605828in}{1.565258in}}%
\pgfpathlineto{\pgfqpoint{4.678089in}{1.602247in}}%
\pgfpathlineto{\pgfqpoint{4.750350in}{1.638805in}}%
\pgfpathlineto{\pgfqpoint{4.822611in}{1.674931in}}%
\pgfpathlineto{\pgfqpoint{4.894872in}{1.710626in}}%
\pgfpathlineto{\pgfqpoint{4.967133in}{1.745889in}}%
\pgfpathlineto{\pgfqpoint{5.039394in}{1.780720in}}%
\pgfpathlineto{\pgfqpoint{5.111655in}{1.815120in}}%
\pgfpathlineto{\pgfqpoint{5.183916in}{1.849089in}}%
\pgfpathlineto{\pgfqpoint{5.256177in}{1.882625in}}%
\pgfpathlineto{\pgfqpoint{5.328438in}{1.915731in}}%
\pgfpathlineto{\pgfqpoint{5.400699in}{1.948404in}}%
\pgfpathlineto{\pgfqpoint{5.472960in}{1.980646in}}%
\pgfpathlineto{\pgfqpoint{5.545221in}{2.012457in}}%
\pgfpathlineto{\pgfqpoint{5.617483in}{2.043835in}}%
\pgfpathlineto{\pgfqpoint{5.689744in}{2.074783in}}%
\pgfpathlineto{\pgfqpoint{5.762005in}{2.105298in}}%
\pgfpathlineto{\pgfqpoint{5.834266in}{2.135383in}}%
\pgfpathlineto{\pgfqpoint{5.906527in}{2.165035in}}%
\pgfpathlineto{\pgfqpoint{5.978788in}{2.194256in}}%
\pgfpathlineto{\pgfqpoint{6.051049in}{2.223046in}}%
\pgfpathlineto{\pgfqpoint{6.123310in}{2.251403in}}%
\pgfpathlineto{\pgfqpoint{6.195571in}{2.279330in}}%
\pgfpathlineto{\pgfqpoint{6.267832in}{2.306824in}}%
\pgfpathlineto{\pgfqpoint{6.340093in}{2.333888in}}%
\pgfpathlineto{\pgfqpoint{6.412354in}{2.360519in}}%
\pgfpathlineto{\pgfqpoint{6.484615in}{2.386719in}}%
\pgfpathlineto{\pgfqpoint{6.556876in}{2.412488in}}%
\pgfpathlineto{\pgfqpoint{6.629138in}{2.437824in}}%
\pgfpathlineto{\pgfqpoint{6.701399in}{2.462730in}}%
\pgfpathlineto{\pgfqpoint{6.773660in}{2.487203in}}%
\pgfpathlineto{\pgfqpoint{6.918182in}{2.535000in}}%
\pgfpathlineto{\pgfqpoint{6.918182in}{2.535000in}}%
\pgfusepath{stroke}%
\end{pgfscope}%
\begin{pgfscope}%
\pgfpathrectangle{\pgfqpoint{1.000000in}{0.330000in}}{\pgfqpoint{6.200000in}{2.310000in}}%
\pgfusepath{clip}%
\pgfsetrectcap%
\pgfsetroundjoin%
\pgfsetlinewidth{1.505625pt}%
\definecolor{currentstroke}{rgb}{0.839216,0.152941,0.156863}%
\pgfsetstrokecolor{currentstroke}%
\pgfsetdash{}{0pt}%
\pgfpathmoveto{\pgfqpoint{1.281818in}{2.535000in}}%
\pgfpathlineto{\pgfqpoint{1.281818in}{2.535000in}}%
\pgfpathlineto{\pgfqpoint{1.354079in}{2.522217in}}%
\pgfpathlineto{\pgfqpoint{1.426340in}{2.509227in}}%
\pgfpathlineto{\pgfqpoint{1.498601in}{2.496031in}}%
\pgfpathlineto{\pgfqpoint{1.570862in}{2.482629in}}%
\pgfpathlineto{\pgfqpoint{1.643124in}{2.469020in}}%
\pgfpathlineto{\pgfqpoint{1.715385in}{2.455204in}}%
\pgfpathlineto{\pgfqpoint{1.787646in}{2.441182in}}%
\pgfpathlineto{\pgfqpoint{1.859907in}{2.426954in}}%
\pgfpathlineto{\pgfqpoint{1.932168in}{2.412519in}}%
\pgfpathlineto{\pgfqpoint{2.004429in}{2.397877in}}%
\pgfpathlineto{\pgfqpoint{2.076690in}{2.383029in}}%
\pgfpathlineto{\pgfqpoint{2.148951in}{2.367975in}}%
\pgfpathlineto{\pgfqpoint{2.221212in}{2.352714in}}%
\pgfpathlineto{\pgfqpoint{2.293473in}{2.337247in}}%
\pgfpathlineto{\pgfqpoint{2.365734in}{2.321573in}}%
\pgfpathlineto{\pgfqpoint{2.437995in}{2.305693in}}%
\pgfpathlineto{\pgfqpoint{2.510256in}{2.289606in}}%
\pgfpathlineto{\pgfqpoint{2.582517in}{2.273313in}}%
\pgfpathlineto{\pgfqpoint{2.654779in}{2.256814in}}%
\pgfpathlineto{\pgfqpoint{2.727040in}{2.240107in}}%
\pgfpathlineto{\pgfqpoint{2.799301in}{2.223195in}}%
\pgfpathlineto{\pgfqpoint{2.871562in}{2.206076in}}%
\pgfpathlineto{\pgfqpoint{2.943823in}{2.188750in}}%
\pgfpathlineto{\pgfqpoint{3.016084in}{2.171218in}}%
\pgfpathlineto{\pgfqpoint{3.088345in}{2.153480in}}%
\pgfpathlineto{\pgfqpoint{3.160606in}{2.135535in}}%
\pgfpathlineto{\pgfqpoint{3.232867in}{2.117383in}}%
\pgfpathlineto{\pgfqpoint{3.305128in}{2.099026in}}%
\pgfpathlineto{\pgfqpoint{3.377389in}{2.080461in}}%
\pgfpathlineto{\pgfqpoint{3.449650in}{2.061690in}}%
\pgfpathlineto{\pgfqpoint{3.521911in}{2.042713in}}%
\pgfpathlineto{\pgfqpoint{3.594172in}{2.023529in}}%
\pgfpathlineto{\pgfqpoint{3.666434in}{2.004139in}}%
\pgfpathlineto{\pgfqpoint{3.738695in}{1.984542in}}%
\pgfpathlineto{\pgfqpoint{3.883217in}{1.944798in}}%
\pgfpathlineto{\pgfqpoint{3.955478in}{1.963082in}}%
\pgfpathlineto{\pgfqpoint{4.027739in}{1.981159in}}%
\pgfpathlineto{\pgfqpoint{4.100000in}{1.999029in}}%
\pgfpathlineto{\pgfqpoint{4.172261in}{2.016693in}}%
\pgfpathlineto{\pgfqpoint{4.244522in}{2.034151in}}%
\pgfpathlineto{\pgfqpoint{4.316783in}{2.051402in}}%
\pgfpathlineto{\pgfqpoint{4.389044in}{2.068447in}}%
\pgfpathlineto{\pgfqpoint{4.461305in}{2.085285in}}%
\pgfpathlineto{\pgfqpoint{4.533566in}{2.101916in}}%
\pgfpathlineto{\pgfqpoint{4.605828in}{2.118342in}}%
\pgfpathlineto{\pgfqpoint{4.678089in}{2.134560in}}%
\pgfpathlineto{\pgfqpoint{4.750350in}{2.150573in}}%
\pgfpathlineto{\pgfqpoint{4.822611in}{2.166378in}}%
\pgfpathlineto{\pgfqpoint{4.894872in}{2.181978in}}%
\pgfpathlineto{\pgfqpoint{4.967133in}{2.197371in}}%
\pgfpathlineto{\pgfqpoint{5.039394in}{2.212557in}}%
\pgfpathlineto{\pgfqpoint{5.111655in}{2.227537in}}%
\pgfpathlineto{\pgfqpoint{5.183916in}{2.242310in}}%
\pgfpathlineto{\pgfqpoint{5.256177in}{2.256877in}}%
\pgfpathlineto{\pgfqpoint{5.328438in}{2.271238in}}%
\pgfpathlineto{\pgfqpoint{5.400699in}{2.285392in}}%
\pgfpathlineto{\pgfqpoint{5.472960in}{2.299339in}}%
\pgfpathlineto{\pgfqpoint{5.545221in}{2.313080in}}%
\pgfpathlineto{\pgfqpoint{5.617483in}{2.326615in}}%
\pgfpathlineto{\pgfqpoint{5.689744in}{2.339943in}}%
\pgfpathlineto{\pgfqpoint{5.762005in}{2.353065in}}%
\pgfpathlineto{\pgfqpoint{5.834266in}{2.365980in}}%
\pgfpathlineto{\pgfqpoint{5.906527in}{2.378689in}}%
\pgfpathlineto{\pgfqpoint{5.978788in}{2.391191in}}%
\pgfpathlineto{\pgfqpoint{6.051049in}{2.403487in}}%
\pgfpathlineto{\pgfqpoint{6.123310in}{2.415576in}}%
\pgfpathlineto{\pgfqpoint{6.195571in}{2.427459in}}%
\pgfpathlineto{\pgfqpoint{6.267832in}{2.439135in}}%
\pgfpathlineto{\pgfqpoint{6.340093in}{2.450605in}}%
\pgfpathlineto{\pgfqpoint{6.412354in}{2.461869in}}%
\pgfpathlineto{\pgfqpoint{6.484615in}{2.472925in}}%
\pgfpathlineto{\pgfqpoint{6.556876in}{2.483776in}}%
\pgfpathlineto{\pgfqpoint{6.629138in}{2.494420in}}%
\pgfpathlineto{\pgfqpoint{6.701399in}{2.504857in}}%
\pgfpathlineto{\pgfqpoint{6.773660in}{2.515088in}}%
\pgfpathlineto{\pgfqpoint{6.918182in}{2.535000in}}%
\pgfpathlineto{\pgfqpoint{6.918182in}{2.535000in}}%
\pgfusepath{stroke}%
\end{pgfscope}%
\begin{pgfscope}%
\pgfpathrectangle{\pgfqpoint{1.000000in}{0.330000in}}{\pgfqpoint{6.200000in}{2.310000in}}%
\pgfusepath{clip}%
\pgfsetrectcap%
\pgfsetroundjoin%
\pgfsetlinewidth{1.505625pt}%
\definecolor{currentstroke}{rgb}{0.580392,0.403922,0.741176}%
\pgfsetstrokecolor{currentstroke}%
\pgfsetdash{}{0pt}%
\pgfpathmoveto{\pgfqpoint{1.281818in}{2.535000in}}%
\pgfpathlineto{\pgfqpoint{1.281818in}{2.535000in}}%
\pgfpathlineto{\pgfqpoint{1.354079in}{2.504398in}}%
\pgfpathlineto{\pgfqpoint{1.426340in}{2.473364in}}%
\pgfpathlineto{\pgfqpoint{1.498601in}{2.441898in}}%
\pgfpathlineto{\pgfqpoint{1.570862in}{2.410001in}}%
\pgfpathlineto{\pgfqpoint{1.643124in}{2.377673in}}%
\pgfpathlineto{\pgfqpoint{1.715385in}{2.344912in}}%
\pgfpathlineto{\pgfqpoint{1.787646in}{2.311721in}}%
\pgfpathlineto{\pgfqpoint{1.859907in}{2.278097in}}%
\pgfpathlineto{\pgfqpoint{1.932168in}{2.244042in}}%
\pgfpathlineto{\pgfqpoint{2.004429in}{2.209556in}}%
\pgfpathlineto{\pgfqpoint{2.076690in}{2.174638in}}%
\pgfpathlineto{\pgfqpoint{2.148951in}{2.139288in}}%
\pgfpathlineto{\pgfqpoint{2.221212in}{2.103507in}}%
\pgfpathlineto{\pgfqpoint{2.293473in}{2.067294in}}%
\pgfpathlineto{\pgfqpoint{2.365734in}{2.030650in}}%
\pgfpathlineto{\pgfqpoint{2.437995in}{1.993574in}}%
\pgfpathlineto{\pgfqpoint{2.510256in}{1.956066in}}%
\pgfpathlineto{\pgfqpoint{2.582517in}{1.918127in}}%
\pgfpathlineto{\pgfqpoint{2.654779in}{1.879757in}}%
\pgfpathlineto{\pgfqpoint{2.727040in}{1.840954in}}%
\pgfpathlineto{\pgfqpoint{2.799301in}{1.801721in}}%
\pgfpathlineto{\pgfqpoint{2.871562in}{1.762055in}}%
\pgfpathlineto{\pgfqpoint{2.943823in}{1.721958in}}%
\pgfpathlineto{\pgfqpoint{3.016084in}{1.681430in}}%
\pgfpathlineto{\pgfqpoint{3.088345in}{1.640470in}}%
\pgfpathlineto{\pgfqpoint{3.160606in}{1.599078in}}%
\pgfpathlineto{\pgfqpoint{3.232867in}{1.557255in}}%
\pgfpathlineto{\pgfqpoint{3.305128in}{1.515000in}}%
\pgfpathlineto{\pgfqpoint{3.377389in}{1.472314in}}%
\pgfpathlineto{\pgfqpoint{3.449650in}{1.429196in}}%
\pgfpathlineto{\pgfqpoint{3.521911in}{1.385646in}}%
\pgfpathlineto{\pgfqpoint{3.594172in}{1.341665in}}%
\pgfpathlineto{\pgfqpoint{3.666434in}{1.297252in}}%
\pgfpathlineto{\pgfqpoint{3.738695in}{1.252408in}}%
\pgfpathlineto{\pgfqpoint{3.883217in}{1.161569in}}%
\pgfpathlineto{\pgfqpoint{3.955478in}{1.203113in}}%
\pgfpathlineto{\pgfqpoint{4.027739in}{1.244226in}}%
\pgfpathlineto{\pgfqpoint{4.100000in}{1.284908in}}%
\pgfpathlineto{\pgfqpoint{4.172261in}{1.325158in}}%
\pgfpathlineto{\pgfqpoint{4.244522in}{1.364976in}}%
\pgfpathlineto{\pgfqpoint{4.316783in}{1.404363in}}%
\pgfpathlineto{\pgfqpoint{4.389044in}{1.443318in}}%
\pgfpathlineto{\pgfqpoint{4.461305in}{1.481841in}}%
\pgfpathlineto{\pgfqpoint{4.533566in}{1.519933in}}%
\pgfpathlineto{\pgfqpoint{4.605828in}{1.557594in}}%
\pgfpathlineto{\pgfqpoint{4.678089in}{1.594822in}}%
\pgfpathlineto{\pgfqpoint{4.750350in}{1.631620in}}%
\pgfpathlineto{\pgfqpoint{4.822611in}{1.667985in}}%
\pgfpathlineto{\pgfqpoint{4.894872in}{1.703920in}}%
\pgfpathlineto{\pgfqpoint{4.967133in}{1.739422in}}%
\pgfpathlineto{\pgfqpoint{5.039394in}{1.774493in}}%
\pgfpathlineto{\pgfqpoint{5.111655in}{1.809132in}}%
\pgfpathlineto{\pgfqpoint{5.183916in}{1.843340in}}%
\pgfpathlineto{\pgfqpoint{5.256177in}{1.877116in}}%
\pgfpathlineto{\pgfqpoint{5.328438in}{1.910461in}}%
\pgfpathlineto{\pgfqpoint{5.400699in}{1.943374in}}%
\pgfpathlineto{\pgfqpoint{5.472960in}{1.975856in}}%
\pgfpathlineto{\pgfqpoint{5.545221in}{2.007906in}}%
\pgfpathlineto{\pgfqpoint{5.617483in}{2.039524in}}%
\pgfpathlineto{\pgfqpoint{5.689744in}{2.070711in}}%
\pgfpathlineto{\pgfqpoint{5.762005in}{2.101466in}}%
\pgfpathlineto{\pgfqpoint{5.834266in}{2.131790in}}%
\pgfpathlineto{\pgfqpoint{5.906527in}{2.161682in}}%
\pgfpathlineto{\pgfqpoint{5.978788in}{2.191142in}}%
\pgfpathlineto{\pgfqpoint{6.051049in}{2.220171in}}%
\pgfpathlineto{\pgfqpoint{6.123310in}{2.248769in}}%
\pgfpathlineto{\pgfqpoint{6.195571in}{2.276935in}}%
\pgfpathlineto{\pgfqpoint{6.267832in}{2.304669in}}%
\pgfpathlineto{\pgfqpoint{6.340093in}{2.331971in}}%
\pgfpathlineto{\pgfqpoint{6.412354in}{2.358843in}}%
\pgfpathlineto{\pgfqpoint{6.484615in}{2.385282in}}%
\pgfpathlineto{\pgfqpoint{6.556876in}{2.411290in}}%
\pgfpathlineto{\pgfqpoint{6.629138in}{2.436866in}}%
\pgfpathlineto{\pgfqpoint{6.701399in}{2.462011in}}%
\pgfpathlineto{\pgfqpoint{6.773660in}{2.486724in}}%
\pgfpathlineto{\pgfqpoint{6.918182in}{2.535000in}}%
\pgfpathlineto{\pgfqpoint{6.918182in}{2.535000in}}%
\pgfusepath{stroke}%
\end{pgfscope}%
\begin{pgfscope}%
\pgfpathrectangle{\pgfqpoint{1.000000in}{0.330000in}}{\pgfqpoint{6.200000in}{2.310000in}}%
\pgfusepath{clip}%
\pgfsetrectcap%
\pgfsetroundjoin%
\pgfsetlinewidth{1.505625pt}%
\definecolor{currentstroke}{rgb}{0.549020,0.337255,0.294118}%
\pgfsetstrokecolor{currentstroke}%
\pgfsetdash{}{0pt}%
\pgfpathmoveto{\pgfqpoint{1.281818in}{2.535000in}}%
\pgfpathlineto{\pgfqpoint{1.281818in}{2.535000in}}%
\pgfpathlineto{\pgfqpoint{1.354079in}{2.513086in}}%
\pgfpathlineto{\pgfqpoint{1.426340in}{2.490818in}}%
\pgfpathlineto{\pgfqpoint{1.498601in}{2.468196in}}%
\pgfpathlineto{\pgfqpoint{1.570862in}{2.445220in}}%
\pgfpathlineto{\pgfqpoint{1.643124in}{2.421891in}}%
\pgfpathlineto{\pgfqpoint{1.715385in}{2.398207in}}%
\pgfpathlineto{\pgfqpoint{1.787646in}{2.374169in}}%
\pgfpathlineto{\pgfqpoint{1.859907in}{2.349778in}}%
\pgfpathlineto{\pgfqpoint{1.932168in}{2.325032in}}%
\pgfpathlineto{\pgfqpoint{2.004429in}{2.299933in}}%
\pgfpathlineto{\pgfqpoint{2.076690in}{2.274479in}}%
\pgfpathlineto{\pgfqpoint{2.148951in}{2.248672in}}%
\pgfpathlineto{\pgfqpoint{2.221212in}{2.222510in}}%
\pgfpathlineto{\pgfqpoint{2.293473in}{2.195995in}}%
\pgfpathlineto{\pgfqpoint{2.365734in}{2.169126in}}%
\pgfpathlineto{\pgfqpoint{2.437995in}{2.141902in}}%
\pgfpathlineto{\pgfqpoint{2.510256in}{2.114325in}}%
\pgfpathlineto{\pgfqpoint{2.582517in}{2.086394in}}%
\pgfpathlineto{\pgfqpoint{2.654779in}{2.058109in}}%
\pgfpathlineto{\pgfqpoint{2.727040in}{2.029470in}}%
\pgfpathlineto{\pgfqpoint{2.799301in}{2.000477in}}%
\pgfpathlineto{\pgfqpoint{2.871562in}{1.971130in}}%
\pgfpathlineto{\pgfqpoint{2.943823in}{1.941429in}}%
\pgfpathlineto{\pgfqpoint{3.016084in}{1.911374in}}%
\pgfpathlineto{\pgfqpoint{3.088345in}{1.880965in}}%
\pgfpathlineto{\pgfqpoint{3.160606in}{1.850203in}}%
\pgfpathlineto{\pgfqpoint{3.232867in}{1.819086in}}%
\pgfpathlineto{\pgfqpoint{3.305128in}{1.787615in}}%
\pgfpathlineto{\pgfqpoint{3.377389in}{1.755791in}}%
\pgfpathlineto{\pgfqpoint{3.449650in}{1.723612in}}%
\pgfpathlineto{\pgfqpoint{3.521911in}{1.691080in}}%
\pgfpathlineto{\pgfqpoint{3.594172in}{1.658193in}}%
\pgfpathlineto{\pgfqpoint{3.666434in}{1.624953in}}%
\pgfpathlineto{\pgfqpoint{3.738695in}{1.591358in}}%
\pgfpathlineto{\pgfqpoint{3.883217in}{1.523226in}}%
\pgfpathlineto{\pgfqpoint{3.955478in}{1.554569in}}%
\pgfpathlineto{\pgfqpoint{4.027739in}{1.585558in}}%
\pgfpathlineto{\pgfqpoint{4.100000in}{1.616193in}}%
\pgfpathlineto{\pgfqpoint{4.172261in}{1.646474in}}%
\pgfpathlineto{\pgfqpoint{4.244522in}{1.676402in}}%
\pgfpathlineto{\pgfqpoint{4.316783in}{1.705975in}}%
\pgfpathlineto{\pgfqpoint{4.389044in}{1.735194in}}%
\pgfpathlineto{\pgfqpoint{4.461305in}{1.764060in}}%
\pgfpathlineto{\pgfqpoint{4.533566in}{1.792571in}}%
\pgfpathlineto{\pgfqpoint{4.605828in}{1.820729in}}%
\pgfpathlineto{\pgfqpoint{4.678089in}{1.848532in}}%
\pgfpathlineto{\pgfqpoint{4.750350in}{1.875982in}}%
\pgfpathlineto{\pgfqpoint{4.822611in}{1.903077in}}%
\pgfpathlineto{\pgfqpoint{4.894872in}{1.929819in}}%
\pgfpathlineto{\pgfqpoint{4.967133in}{1.956207in}}%
\pgfpathlineto{\pgfqpoint{5.039394in}{1.982241in}}%
\pgfpathlineto{\pgfqpoint{5.111655in}{2.007920in}}%
\pgfpathlineto{\pgfqpoint{5.183916in}{2.033246in}}%
\pgfpathlineto{\pgfqpoint{5.256177in}{2.058218in}}%
\pgfpathlineto{\pgfqpoint{5.328438in}{2.082836in}}%
\pgfpathlineto{\pgfqpoint{5.400699in}{2.107100in}}%
\pgfpathlineto{\pgfqpoint{5.472960in}{2.131010in}}%
\pgfpathlineto{\pgfqpoint{5.545221in}{2.154566in}}%
\pgfpathlineto{\pgfqpoint{5.617483in}{2.177769in}}%
\pgfpathlineto{\pgfqpoint{5.689744in}{2.200617in}}%
\pgfpathlineto{\pgfqpoint{5.762005in}{2.223111in}}%
\pgfpathlineto{\pgfqpoint{5.834266in}{2.245251in}}%
\pgfpathlineto{\pgfqpoint{5.906527in}{2.267038in}}%
\pgfpathlineto{\pgfqpoint{5.978788in}{2.288470in}}%
\pgfpathlineto{\pgfqpoint{6.051049in}{2.309549in}}%
\pgfpathlineto{\pgfqpoint{6.123310in}{2.330273in}}%
\pgfpathlineto{\pgfqpoint{6.195571in}{2.350644in}}%
\pgfpathlineto{\pgfqpoint{6.267832in}{2.370660in}}%
\pgfpathlineto{\pgfqpoint{6.340093in}{2.390323in}}%
\pgfpathlineto{\pgfqpoint{6.412354in}{2.409632in}}%
\pgfpathlineto{\pgfqpoint{6.484615in}{2.428586in}}%
\pgfpathlineto{\pgfqpoint{6.556876in}{2.447187in}}%
\pgfpathlineto{\pgfqpoint{6.629138in}{2.465434in}}%
\pgfpathlineto{\pgfqpoint{6.701399in}{2.483327in}}%
\pgfpathlineto{\pgfqpoint{6.773660in}{2.500866in}}%
\pgfpathlineto{\pgfqpoint{6.918182in}{2.535000in}}%
\pgfpathlineto{\pgfqpoint{6.918182in}{2.535000in}}%
\pgfusepath{stroke}%
\end{pgfscope}%
\begin{pgfscope}%
\pgfpathrectangle{\pgfqpoint{1.000000in}{0.330000in}}{\pgfqpoint{6.200000in}{2.310000in}}%
\pgfusepath{clip}%
\pgfsetrectcap%
\pgfsetroundjoin%
\pgfsetlinewidth{1.505625pt}%
\definecolor{currentstroke}{rgb}{0.890196,0.466667,0.760784}%
\pgfsetstrokecolor{currentstroke}%
\pgfsetdash{}{0pt}%
\pgfpathmoveto{\pgfqpoint{1.281818in}{2.535000in}}%
\pgfpathlineto{\pgfqpoint{1.281818in}{2.535000in}}%
\pgfpathlineto{\pgfqpoint{1.354079in}{2.513086in}}%
\pgfpathlineto{\pgfqpoint{1.426340in}{2.490818in}}%
\pgfpathlineto{\pgfqpoint{1.498601in}{2.468196in}}%
\pgfpathlineto{\pgfqpoint{1.570862in}{2.445220in}}%
\pgfpathlineto{\pgfqpoint{1.643124in}{2.421891in}}%
\pgfpathlineto{\pgfqpoint{1.715385in}{2.398207in}}%
\pgfpathlineto{\pgfqpoint{1.787646in}{2.374169in}}%
\pgfpathlineto{\pgfqpoint{1.859907in}{2.349778in}}%
\pgfpathlineto{\pgfqpoint{1.932168in}{2.325032in}}%
\pgfpathlineto{\pgfqpoint{2.004429in}{2.299933in}}%
\pgfpathlineto{\pgfqpoint{2.076690in}{2.274479in}}%
\pgfpathlineto{\pgfqpoint{2.148951in}{2.248672in}}%
\pgfpathlineto{\pgfqpoint{2.221212in}{2.222510in}}%
\pgfpathlineto{\pgfqpoint{2.293473in}{2.195995in}}%
\pgfpathlineto{\pgfqpoint{2.365734in}{2.169126in}}%
\pgfpathlineto{\pgfqpoint{2.437995in}{2.141902in}}%
\pgfpathlineto{\pgfqpoint{2.510256in}{2.114325in}}%
\pgfpathlineto{\pgfqpoint{2.582517in}{2.086394in}}%
\pgfpathlineto{\pgfqpoint{2.654779in}{2.058109in}}%
\pgfpathlineto{\pgfqpoint{2.727040in}{2.029470in}}%
\pgfpathlineto{\pgfqpoint{2.799301in}{2.000477in}}%
\pgfpathlineto{\pgfqpoint{2.871562in}{1.971130in}}%
\pgfpathlineto{\pgfqpoint{2.943823in}{1.941429in}}%
\pgfpathlineto{\pgfqpoint{3.016084in}{1.911374in}}%
\pgfpathlineto{\pgfqpoint{3.088345in}{1.880965in}}%
\pgfpathlineto{\pgfqpoint{3.160606in}{1.850203in}}%
\pgfpathlineto{\pgfqpoint{3.232867in}{1.819086in}}%
\pgfpathlineto{\pgfqpoint{3.305128in}{1.787615in}}%
\pgfpathlineto{\pgfqpoint{3.377389in}{1.755791in}}%
\pgfpathlineto{\pgfqpoint{3.449650in}{1.723612in}}%
\pgfpathlineto{\pgfqpoint{3.521911in}{1.691080in}}%
\pgfpathlineto{\pgfqpoint{3.594172in}{1.658193in}}%
\pgfpathlineto{\pgfqpoint{3.666434in}{1.624953in}}%
\pgfpathlineto{\pgfqpoint{3.738695in}{1.591358in}}%
\pgfpathlineto{\pgfqpoint{3.883217in}{1.523226in}}%
\pgfpathlineto{\pgfqpoint{3.955478in}{1.556159in}}%
\pgfpathlineto{\pgfqpoint{4.027739in}{1.588662in}}%
\pgfpathlineto{\pgfqpoint{4.100000in}{1.620732in}}%
\pgfpathlineto{\pgfqpoint{4.172261in}{1.652371in}}%
\pgfpathlineto{\pgfqpoint{4.244522in}{1.683578in}}%
\pgfpathlineto{\pgfqpoint{4.316783in}{1.714354in}}%
\pgfpathlineto{\pgfqpoint{4.389044in}{1.744698in}}%
\pgfpathlineto{\pgfqpoint{4.461305in}{1.774611in}}%
\pgfpathlineto{\pgfqpoint{4.533566in}{1.804092in}}%
\pgfpathlineto{\pgfqpoint{4.605828in}{1.833142in}}%
\pgfpathlineto{\pgfqpoint{4.678089in}{1.861760in}}%
\pgfpathlineto{\pgfqpoint{4.750350in}{1.889946in}}%
\pgfpathlineto{\pgfqpoint{4.822611in}{1.917701in}}%
\pgfpathlineto{\pgfqpoint{4.894872in}{1.945024in}}%
\pgfpathlineto{\pgfqpoint{4.967133in}{1.971916in}}%
\pgfpathlineto{\pgfqpoint{5.039394in}{1.998376in}}%
\pgfpathlineto{\pgfqpoint{5.111655in}{2.024404in}}%
\pgfpathlineto{\pgfqpoint{5.183916in}{2.050001in}}%
\pgfpathlineto{\pgfqpoint{5.256177in}{2.075167in}}%
\pgfpathlineto{\pgfqpoint{5.328438in}{2.099901in}}%
\pgfpathlineto{\pgfqpoint{5.400699in}{2.124203in}}%
\pgfpathlineto{\pgfqpoint{5.472960in}{2.148073in}}%
\pgfpathlineto{\pgfqpoint{5.545221in}{2.171512in}}%
\pgfpathlineto{\pgfqpoint{5.617483in}{2.194520in}}%
\pgfpathlineto{\pgfqpoint{5.689744in}{2.217096in}}%
\pgfpathlineto{\pgfqpoint{5.762005in}{2.239240in}}%
\pgfpathlineto{\pgfqpoint{5.834266in}{2.260953in}}%
\pgfpathlineto{\pgfqpoint{5.906527in}{2.282234in}}%
\pgfpathlineto{\pgfqpoint{5.978788in}{2.303084in}}%
\pgfpathlineto{\pgfqpoint{6.051049in}{2.323502in}}%
\pgfpathlineto{\pgfqpoint{6.123310in}{2.343488in}}%
\pgfpathlineto{\pgfqpoint{6.195571in}{2.363043in}}%
\pgfpathlineto{\pgfqpoint{6.267832in}{2.382167in}}%
\pgfpathlineto{\pgfqpoint{6.340093in}{2.400859in}}%
\pgfpathlineto{\pgfqpoint{6.412354in}{2.419119in}}%
\pgfpathlineto{\pgfqpoint{6.484615in}{2.436947in}}%
\pgfpathlineto{\pgfqpoint{6.556876in}{2.454344in}}%
\pgfpathlineto{\pgfqpoint{6.629138in}{2.471310in}}%
\pgfpathlineto{\pgfqpoint{6.701399in}{2.487844in}}%
\pgfpathlineto{\pgfqpoint{6.773660in}{2.503946in}}%
\pgfpathlineto{\pgfqpoint{6.918182in}{2.535000in}}%
\pgfpathlineto{\pgfqpoint{6.918182in}{2.535000in}}%
\pgfusepath{stroke}%
\end{pgfscope}%
\begin{pgfscope}%
\pgfpathrectangle{\pgfqpoint{1.000000in}{0.330000in}}{\pgfqpoint{6.200000in}{2.310000in}}%
\pgfusepath{clip}%
\pgfsetrectcap%
\pgfsetroundjoin%
\pgfsetlinewidth{1.505625pt}%
\definecolor{currentstroke}{rgb}{0.498039,0.498039,0.498039}%
\pgfsetstrokecolor{currentstroke}%
\pgfsetdash{}{0pt}%
\pgfpathmoveto{\pgfqpoint{1.281818in}{2.535000in}}%
\pgfpathlineto{\pgfqpoint{1.281818in}{2.535000in}}%
\pgfpathlineto{\pgfqpoint{1.354079in}{2.513086in}}%
\pgfpathlineto{\pgfqpoint{1.426340in}{2.490818in}}%
\pgfpathlineto{\pgfqpoint{1.498601in}{2.468196in}}%
\pgfpathlineto{\pgfqpoint{1.570862in}{2.445220in}}%
\pgfpathlineto{\pgfqpoint{1.643124in}{2.421891in}}%
\pgfpathlineto{\pgfqpoint{1.715385in}{2.398207in}}%
\pgfpathlineto{\pgfqpoint{1.787646in}{2.374169in}}%
\pgfpathlineto{\pgfqpoint{1.859907in}{2.349778in}}%
\pgfpathlineto{\pgfqpoint{1.932168in}{2.325032in}}%
\pgfpathlineto{\pgfqpoint{2.004429in}{2.299933in}}%
\pgfpathlineto{\pgfqpoint{2.076690in}{2.274479in}}%
\pgfpathlineto{\pgfqpoint{2.148951in}{2.248672in}}%
\pgfpathlineto{\pgfqpoint{2.221212in}{2.222510in}}%
\pgfpathlineto{\pgfqpoint{2.293473in}{2.195995in}}%
\pgfpathlineto{\pgfqpoint{2.365734in}{2.169126in}}%
\pgfpathlineto{\pgfqpoint{2.437995in}{2.141902in}}%
\pgfpathlineto{\pgfqpoint{2.510256in}{2.114325in}}%
\pgfpathlineto{\pgfqpoint{2.582517in}{2.086394in}}%
\pgfpathlineto{\pgfqpoint{2.654779in}{2.058109in}}%
\pgfpathlineto{\pgfqpoint{2.727040in}{2.029470in}}%
\pgfpathlineto{\pgfqpoint{2.799301in}{2.000477in}}%
\pgfpathlineto{\pgfqpoint{2.871562in}{1.971130in}}%
\pgfpathlineto{\pgfqpoint{2.943823in}{1.941429in}}%
\pgfpathlineto{\pgfqpoint{3.016084in}{1.911374in}}%
\pgfpathlineto{\pgfqpoint{3.088345in}{1.880965in}}%
\pgfpathlineto{\pgfqpoint{3.160606in}{1.850203in}}%
\pgfpathlineto{\pgfqpoint{3.232867in}{1.819086in}}%
\pgfpathlineto{\pgfqpoint{3.305128in}{1.787615in}}%
\pgfpathlineto{\pgfqpoint{3.377389in}{1.755791in}}%
\pgfpathlineto{\pgfqpoint{3.449650in}{1.723612in}}%
\pgfpathlineto{\pgfqpoint{3.521911in}{1.691080in}}%
\pgfpathlineto{\pgfqpoint{3.594172in}{1.658193in}}%
\pgfpathlineto{\pgfqpoint{3.666434in}{1.624953in}}%
\pgfpathlineto{\pgfqpoint{3.738695in}{1.591358in}}%
\pgfpathlineto{\pgfqpoint{3.883217in}{1.523226in}}%
\pgfpathlineto{\pgfqpoint{3.955478in}{1.556159in}}%
\pgfpathlineto{\pgfqpoint{4.027739in}{1.588662in}}%
\pgfpathlineto{\pgfqpoint{4.100000in}{1.620732in}}%
\pgfpathlineto{\pgfqpoint{4.172261in}{1.652371in}}%
\pgfpathlineto{\pgfqpoint{4.244522in}{1.683578in}}%
\pgfpathlineto{\pgfqpoint{4.316783in}{1.714354in}}%
\pgfpathlineto{\pgfqpoint{4.389044in}{1.744698in}}%
\pgfpathlineto{\pgfqpoint{4.461305in}{1.774611in}}%
\pgfpathlineto{\pgfqpoint{4.533566in}{1.804092in}}%
\pgfpathlineto{\pgfqpoint{4.605828in}{1.833142in}}%
\pgfpathlineto{\pgfqpoint{4.678089in}{1.861760in}}%
\pgfpathlineto{\pgfqpoint{4.750350in}{1.889946in}}%
\pgfpathlineto{\pgfqpoint{4.822611in}{1.917701in}}%
\pgfpathlineto{\pgfqpoint{4.894872in}{1.945024in}}%
\pgfpathlineto{\pgfqpoint{4.967133in}{1.971916in}}%
\pgfpathlineto{\pgfqpoint{5.039394in}{1.998376in}}%
\pgfpathlineto{\pgfqpoint{5.111655in}{2.024404in}}%
\pgfpathlineto{\pgfqpoint{5.183916in}{2.050001in}}%
\pgfpathlineto{\pgfqpoint{5.256177in}{2.075167in}}%
\pgfpathlineto{\pgfqpoint{5.328438in}{2.099901in}}%
\pgfpathlineto{\pgfqpoint{5.400699in}{2.124203in}}%
\pgfpathlineto{\pgfqpoint{5.472960in}{2.148073in}}%
\pgfpathlineto{\pgfqpoint{5.545221in}{2.171512in}}%
\pgfpathlineto{\pgfqpoint{5.617483in}{2.194520in}}%
\pgfpathlineto{\pgfqpoint{5.689744in}{2.217096in}}%
\pgfpathlineto{\pgfqpoint{5.762005in}{2.239240in}}%
\pgfpathlineto{\pgfqpoint{5.834266in}{2.260953in}}%
\pgfpathlineto{\pgfqpoint{5.906527in}{2.282234in}}%
\pgfpathlineto{\pgfqpoint{5.978788in}{2.303084in}}%
\pgfpathlineto{\pgfqpoint{6.051049in}{2.323502in}}%
\pgfpathlineto{\pgfqpoint{6.123310in}{2.343488in}}%
\pgfpathlineto{\pgfqpoint{6.195571in}{2.363043in}}%
\pgfpathlineto{\pgfqpoint{6.267832in}{2.382167in}}%
\pgfpathlineto{\pgfqpoint{6.340093in}{2.400859in}}%
\pgfpathlineto{\pgfqpoint{6.412354in}{2.419119in}}%
\pgfpathlineto{\pgfqpoint{6.484615in}{2.436947in}}%
\pgfpathlineto{\pgfqpoint{6.556876in}{2.454344in}}%
\pgfpathlineto{\pgfqpoint{6.629138in}{2.471310in}}%
\pgfpathlineto{\pgfqpoint{6.701399in}{2.487844in}}%
\pgfpathlineto{\pgfqpoint{6.773660in}{2.503946in}}%
\pgfpathlineto{\pgfqpoint{6.918182in}{2.535000in}}%
\pgfpathlineto{\pgfqpoint{6.918182in}{2.535000in}}%
\pgfusepath{stroke}%
\end{pgfscope}%
\begin{pgfscope}%
\pgfpathrectangle{\pgfqpoint{1.000000in}{0.330000in}}{\pgfqpoint{6.200000in}{2.310000in}}%
\pgfusepath{clip}%
\pgfsetrectcap%
\pgfsetroundjoin%
\pgfsetlinewidth{1.505625pt}%
\definecolor{currentstroke}{rgb}{0.737255,0.741176,0.133333}%
\pgfsetstrokecolor{currentstroke}%
\pgfsetdash{}{0pt}%
\pgfpathmoveto{\pgfqpoint{1.281818in}{2.535000in}}%
\pgfpathlineto{\pgfqpoint{1.281818in}{2.535000in}}%
\pgfpathlineto{\pgfqpoint{1.354079in}{2.518565in}}%
\pgfpathlineto{\pgfqpoint{1.426340in}{2.501864in}}%
\pgfpathlineto{\pgfqpoint{1.498601in}{2.484897in}}%
\pgfpathlineto{\pgfqpoint{1.570862in}{2.467665in}}%
\pgfpathlineto{\pgfqpoint{1.643124in}{2.450168in}}%
\pgfpathlineto{\pgfqpoint{1.715385in}{2.432405in}}%
\pgfpathlineto{\pgfqpoint{1.787646in}{2.414377in}}%
\pgfpathlineto{\pgfqpoint{1.859907in}{2.396083in}}%
\pgfpathlineto{\pgfqpoint{1.932168in}{2.377524in}}%
\pgfpathlineto{\pgfqpoint{2.004429in}{2.358699in}}%
\pgfpathlineto{\pgfqpoint{2.076690in}{2.339609in}}%
\pgfpathlineto{\pgfqpoint{2.148951in}{2.320254in}}%
\pgfpathlineto{\pgfqpoint{2.221212in}{2.300633in}}%
\pgfpathlineto{\pgfqpoint{2.293473in}{2.280746in}}%
\pgfpathlineto{\pgfqpoint{2.365734in}{2.260594in}}%
\pgfpathlineto{\pgfqpoint{2.437995in}{2.240177in}}%
\pgfpathlineto{\pgfqpoint{2.510256in}{2.219494in}}%
\pgfpathlineto{\pgfqpoint{2.582517in}{2.198546in}}%
\pgfpathlineto{\pgfqpoint{2.654779in}{2.177332in}}%
\pgfpathlineto{\pgfqpoint{2.727040in}{2.155852in}}%
\pgfpathlineto{\pgfqpoint{2.799301in}{2.134108in}}%
\pgfpathlineto{\pgfqpoint{2.871562in}{2.112097in}}%
\pgfpathlineto{\pgfqpoint{2.943823in}{2.089822in}}%
\pgfpathlineto{\pgfqpoint{3.016084in}{2.067281in}}%
\pgfpathlineto{\pgfqpoint{3.088345in}{2.044474in}}%
\pgfpathlineto{\pgfqpoint{3.160606in}{2.021402in}}%
\pgfpathlineto{\pgfqpoint{3.232867in}{1.998064in}}%
\pgfpathlineto{\pgfqpoint{3.305128in}{1.974461in}}%
\pgfpathlineto{\pgfqpoint{3.377389in}{1.950593in}}%
\pgfpathlineto{\pgfqpoint{3.449650in}{1.926459in}}%
\pgfpathlineto{\pgfqpoint{3.521911in}{1.902060in}}%
\pgfpathlineto{\pgfqpoint{3.594172in}{1.877395in}}%
\pgfpathlineto{\pgfqpoint{3.666434in}{1.852465in}}%
\pgfpathlineto{\pgfqpoint{3.738695in}{1.827269in}}%
\pgfpathlineto{\pgfqpoint{3.883217in}{1.776169in}}%
\pgfpathlineto{\pgfqpoint{3.955478in}{1.799677in}}%
\pgfpathlineto{\pgfqpoint{4.027739in}{1.822918in}}%
\pgfpathlineto{\pgfqpoint{4.100000in}{1.845895in}}%
\pgfpathlineto{\pgfqpoint{4.172261in}{1.868606in}}%
\pgfpathlineto{\pgfqpoint{4.244522in}{1.891051in}}%
\pgfpathlineto{\pgfqpoint{4.316783in}{1.913231in}}%
\pgfpathlineto{\pgfqpoint{4.389044in}{1.935146in}}%
\pgfpathlineto{\pgfqpoint{4.461305in}{1.956795in}}%
\pgfpathlineto{\pgfqpoint{4.533566in}{1.978178in}}%
\pgfpathlineto{\pgfqpoint{4.605828in}{1.999296in}}%
\pgfpathlineto{\pgfqpoint{4.678089in}{2.020149in}}%
\pgfpathlineto{\pgfqpoint{4.750350in}{2.040736in}}%
\pgfpathlineto{\pgfqpoint{4.822611in}{2.061058in}}%
\pgfpathlineto{\pgfqpoint{4.894872in}{2.081114in}}%
\pgfpathlineto{\pgfqpoint{4.967133in}{2.100905in}}%
\pgfpathlineto{\pgfqpoint{5.039394in}{2.120430in}}%
\pgfpathlineto{\pgfqpoint{5.111655in}{2.139690in}}%
\pgfpathlineto{\pgfqpoint{5.183916in}{2.158685in}}%
\pgfpathlineto{\pgfqpoint{5.256177in}{2.177414in}}%
\pgfpathlineto{\pgfqpoint{5.328438in}{2.195877in}}%
\pgfpathlineto{\pgfqpoint{5.400699in}{2.214075in}}%
\pgfpathlineto{\pgfqpoint{5.472960in}{2.232008in}}%
\pgfpathlineto{\pgfqpoint{5.545221in}{2.249675in}}%
\pgfpathlineto{\pgfqpoint{5.617483in}{2.267076in}}%
\pgfpathlineto{\pgfqpoint{5.689744in}{2.284213in}}%
\pgfpathlineto{\pgfqpoint{5.762005in}{2.301083in}}%
\pgfpathlineto{\pgfqpoint{5.834266in}{2.317689in}}%
\pgfpathlineto{\pgfqpoint{5.906527in}{2.334028in}}%
\pgfpathlineto{\pgfqpoint{5.978788in}{2.350103in}}%
\pgfpathlineto{\pgfqpoint{6.051049in}{2.365912in}}%
\pgfpathlineto{\pgfqpoint{6.123310in}{2.381455in}}%
\pgfpathlineto{\pgfqpoint{6.195571in}{2.396733in}}%
\pgfpathlineto{\pgfqpoint{6.267832in}{2.411745in}}%
\pgfpathlineto{\pgfqpoint{6.340093in}{2.426492in}}%
\pgfpathlineto{\pgfqpoint{6.412354in}{2.440974in}}%
\pgfpathlineto{\pgfqpoint{6.484615in}{2.455190in}}%
\pgfpathlineto{\pgfqpoint{6.556876in}{2.469140in}}%
\pgfpathlineto{\pgfqpoint{6.629138in}{2.482826in}}%
\pgfpathlineto{\pgfqpoint{6.701399in}{2.496245in}}%
\pgfpathlineto{\pgfqpoint{6.773660in}{2.509399in}}%
\pgfpathlineto{\pgfqpoint{6.918182in}{2.535000in}}%
\pgfpathlineto{\pgfqpoint{6.918182in}{2.535000in}}%
\pgfusepath{stroke}%
\end{pgfscope}%
\begin{pgfscope}%
\pgfsetroundcap%
\pgfsetroundjoin%
\pgfsetlinewidth{1.003750pt}%
\definecolor{currentstroke}{rgb}{0.000000,0.000000,0.000000}%
\pgfsetstrokecolor{currentstroke}%
\pgfsetdash{}{0pt}%
\pgfpathmoveto{\pgfqpoint{4.339629in}{0.435000in}}%
\pgfpathquadraticcurveto{\pgfqpoint{4.125310in}{0.435000in}}{\pgfqpoint{3.910990in}{0.435000in}}%
\pgfusepath{stroke}%
\end{pgfscope}%
\begin{pgfscope}%
\pgfsetbuttcap%
\pgfsetmiterjoin%
\definecolor{currentfill}{rgb}{0.800000,0.800000,0.800000}%
\pgfsetfillcolor{currentfill}%
\pgfsetlinewidth{1.003750pt}%
\definecolor{currentstroke}{rgb}{0.000000,0.000000,0.000000}%
\pgfsetstrokecolor{currentstroke}%
\pgfsetdash{}{0pt}%
\pgfpathmoveto{\pgfqpoint{4.397313in}{0.338549in}}%
\pgfpathcurveto{\pgfqpoint{4.432036in}{0.303827in}}{\pgfqpoint{5.272106in}{0.303827in}}{\pgfqpoint{5.306828in}{0.338549in}}%
\pgfpathcurveto{\pgfqpoint{5.341550in}{0.373272in}}{\pgfqpoint{5.341550in}{0.496728in}}{\pgfqpoint{5.306828in}{0.531451in}}%
\pgfpathcurveto{\pgfqpoint{5.272106in}{0.566173in}}{\pgfqpoint{4.432036in}{0.566173in}}{\pgfqpoint{4.397313in}{0.531451in}}%
\pgfpathcurveto{\pgfqpoint{4.362591in}{0.496728in}}{\pgfqpoint{4.362591in}{0.373272in}}{\pgfqpoint{4.397313in}{0.338549in}}%
\pgfpathclose%
\pgfusepath{stroke,fill}%
\end{pgfscope}%
\begin{pgfscope}%
\definecolor{textcolor}{rgb}{0.000000,0.000000,0.000000}%
\pgfsetstrokecolor{textcolor}%
\pgfsetfillcolor{textcolor}%
\pgftext[x=5.272106in,y=0.435000in,right,]{\color{textcolor}\rmfamily\fontsize{10.000000}{12.000000}\selectfont \(\displaystyle M_u =\) -7.5 kft}%
\end{pgfscope}%
\begin{pgfscope}%
\pgfsetbuttcap%
\pgfsetmiterjoin%
\definecolor{currentfill}{rgb}{0.800000,0.800000,0.800000}%
\pgfsetfillcolor{currentfill}%
\pgfsetlinewidth{1.003750pt}%
\definecolor{currentstroke}{rgb}{0.000000,0.000000,0.000000}%
\pgfsetstrokecolor{currentstroke}%
\pgfsetdash{}{0pt}%
\pgfpathmoveto{\pgfqpoint{0.965278in}{0.358599in}}%
\pgfpathcurveto{\pgfqpoint{1.000000in}{0.323877in}}{\pgfqpoint{4.339899in}{0.323877in}}{\pgfqpoint{4.374622in}{0.358599in}}%
\pgfpathcurveto{\pgfqpoint{4.409344in}{0.393321in}}{\pgfqpoint{4.409344in}{0.668784in}}{\pgfqpoint{4.374622in}{0.703506in}}%
\pgfpathcurveto{\pgfqpoint{4.339899in}{0.738228in}}{\pgfqpoint{1.000000in}{0.738228in}}{\pgfqpoint{0.965278in}{0.703506in}}%
\pgfpathcurveto{\pgfqpoint{0.930556in}{0.668784in}}{\pgfqpoint{0.930556in}{0.393321in}}{\pgfqpoint{0.965278in}{0.358599in}}%
\pgfpathclose%
\pgfusepath{stroke,fill}%
\end{pgfscope}%
\begin{pgfscope}%
\definecolor{textcolor}{rgb}{0.000000,0.000000,0.000000}%
\pgfsetstrokecolor{textcolor}%
\pgfsetfillcolor{textcolor}%
\pgftext[x=1.000000in, y=0.580049in, left, base]{\color{textcolor}\rmfamily\fontsize{10.000000}{12.000000}\selectfont Max combo: 1.2D + 1.6Lr0 + 1.0L0 + 1.6Lr1 + 1.0L1}%
\end{pgfscope}%
\begin{pgfscope}%
\definecolor{textcolor}{rgb}{0.000000,0.000000,0.000000}%
\pgfsetstrokecolor{textcolor}%
\pgfsetfillcolor{textcolor}%
\pgftext[x=1.000000in, y=0.428043in, left, base]{\color{textcolor}\rmfamily\fontsize{10.000000}{12.000000}\selectfont ASCE7-16 Sec. 2.3.1 (LC 3)}%
\end{pgfscope}%
\end{pgfpicture}%
\makeatother%
\endgroup%

\end{center}
\caption{Moment Demand Envelope}
\end{figure}
L\textsubscript{p}, the limiting laterally unbraced length for the limit state of yielding, is calculated per AISC/ANSI 360-16 Eq. F2-5 as follows:
\begin{flalign*}
L_p = 1.76\cdot r_y \cdot \sqrt{\frac{E}{F_y}}  = 1.76\cdot 0.785 {\color{darkBlue}{\mathbf{ \; in}}} \cdot \sqrt{\frac{29000 {\color{darkBlue}{\mathbf{ \; ksi}}}}{50 {\color{darkBlue}{\mathbf{ \; ksi}}}}}  = \mathbf{2.8 {\color{darkBlue}{\mathbf{ \; ft}}}}
\end{flalign*}
r\textsubscript{ts}, a coefficient used in the calculation of L\textsubscript{r} and C\textsubscript{b}, is calculated per AISC/ANSI 360-16 Eq. F2-7 as follows:
\begin{flalign*}
r_{{ts}} = \sqrt{\frac{\sqrt{I_y \cdot C_w}}{S_x}}  = \sqrt{\frac{\sqrt{2.18 {\color{darkBlue}{\mathbf{ \; {\color{darkBlue}{\mathbf{ \; in}}}^{4}}}} \cdot 50.9 {\color{darkBlue}{\mathbf{ \; {\color{darkBlue}{\mathbf{ \; in}}}^{6}}}}}}{10.9 {\color{darkBlue}{\mathbf{ \; {\color{darkBlue}{\mathbf{ \; in}}}^{3}}}}}}  = \mathbf{1.0 {\color{darkBlue}{\mathbf{ \; in}}}}
\end{flalign*}
L\textsubscript{r}, the limiting unbraced length for the limit state of inelastic lateral-torsional buckling, is calculated per AISC/ANSI 360-16 Eq. F2-6 as follows:
\begin{flalign*}
L_r &= 1.95\cdot r_{ts} \cdot \frac{E}{0.7\cdot F_y} \sqrt{\frac{J \cdot c}{S_x \cdot h_0}+\sqrt{{\left(\frac{J \cdot c}{S_x \cdot h_0}\right)}^2+6.76{\left(\frac{0.7\cdot F_y}{E}\right)}^2}} \\ &= 1.95\cdot 1.0 {\color{darkBlue}{\mathbf{ \; in}}} \cdot \frac{29000 {\color{darkBlue}{\mathbf{ \; ksi}}}}{0.7\cdot 50 {\color{darkBlue}{\mathbf{ \; ksi}}}} \sqrt{\frac{0.05 {\color{darkBlue}{\mathbf{ \; {\color{darkBlue}{\mathbf{ \; in}}}^{4}}}} \cdot 1}{10.9 {\color{darkBlue}{\mathbf{ \; {\color{darkBlue}{\mathbf{ \; in}}}^{3}}}} \cdot 9.7 {\color{darkBlue}{\mathbf{ \; in}}}}+\sqrt{{\left(\frac{0.05 {\color{darkBlue}{\mathbf{ \; {\color{darkBlue}{\mathbf{ \; in}}}^{4}}}} \cdot 1}{10.9 {\color{darkBlue}{\mathbf{ \; {\color{darkBlue}{\mathbf{ \; in}}}^{3}}}} \cdot 9.7 {\color{darkBlue}{\mathbf{ \; in}}}}\right)}^2+6.76{\left(\frac{0.7\cdot 50 {\color{darkBlue}{\mathbf{ \; ksi}}}}{29000 {\color{darkBlue}{\mathbf{ \; ksi}}}}\right)}^2}} \\ &= \mathbf{8.1 {\color{darkBlue}{\mathbf{ \; ft}}}}
\end{flalign*}
\textlambda\textsubscript{web}, the web width-to-thickness ratio, is calculated per {AISC/ANSI 360-16 Table B4.1b} as follows:
\begin{flalign*}
\lambda_{{web}} = \frac{h}{t_w}  = \frac{8.85 {\color{darkBlue}{\mathbf{ \; in}}}}{0.19 {\color{darkBlue}{\mathbf{ \; in}}}}  = \mathbf{46.6 }
\end{flalign*}
\textlambda\textsubscript{P-web}, the limiting width-to-thickness ratio for compact/noncompact web, is calculated per {AISC/ANSI 360-16 Table B4.1b} as follows:
\begin{flalign*}
\lambda_{P-web} = 3.76\cdot \sqrt{\frac{E}{F_y}} = 3.76\cdot \sqrt{\frac{29000 {\color{darkBlue}{\mathbf{ \; ksi}}}}{50 {\color{darkBlue}{\mathbf{ \; ksi}}}}} = \mathbf{90.6}
\end{flalign*}
\textlambda\textsubscript{R-web}, the limiting width-to-thickness ratio for noncompact/slender web, is calculated per {AISC/ANSI 360-16 Table B4.1b} as follows:
\begin{flalign*}
\lambda_{R-web} = 5.7\cdot \sqrt{\frac{E}{F_y}} = 5.7\cdot \sqrt{\frac{29000 {\color{darkBlue}{\mathbf{ \; ksi}}}}{50 {\color{darkBlue}{\mathbf{ \; ksi}}}}} = \mathbf{137.3}
\end{flalign*}
\textlambda\textsubscript{web} $<$ \textlambda\textsubscript{P-web} \textrightarrow \; \textbf{Compact Web}
\\\\
\textlambda\textsubscript{flange}, the flange width-to-thickness ratio, is calculated per {AISC/ANSI 360-16 Table B4.1b} as follows:
\begin{flalign*}
\lambda_{{flange}} = \frac{b}{t}  = \frac{1.98 {\color{darkBlue}{\mathbf{ \; in}}}}{0.21 {\color{darkBlue}{\mathbf{ \; in}}}}  = \mathbf{9.4 }
\end{flalign*}
\textlambda\textsubscript{P-flange}, the limiting width-to-thickness ratio for compact/noncompact flange, is calculated per {AISC/ANSI 360-16 Table B4.1b} as follows:
\begin{flalign*}
\lambda_{P-flange} = 0.38\cdot \sqrt{\frac{E}{F_y}} = 0.38\cdot \sqrt{\frac{29000 {\color{darkBlue}{\mathbf{ \; ksi}}}}{50 {\color{darkBlue}{\mathbf{ \; ksi}}}}} = \mathbf{9.2}
\end{flalign*}
\textlambda\textsubscript{R-flange}, the limiting width-to-thickness ratio for noncompact/slender flange, is calculated per {AISC/ANSI 360-16 Table B4.1b} as follows:
\begin{flalign*}
\lambda_{R-flange} = 1\cdot \sqrt{\frac{E}{F_y}} = 1\cdot \sqrt{\frac{29000 {\color{darkBlue}{\mathbf{ \; ksi}}}}{50 {\color{darkBlue}{\mathbf{ \; ksi}}}}} = \mathbf{24.1}
\end{flalign*}
\textlambda\textsubscript{P-flange} $<$ \textlambda\textsubscript{flange} $<$ \textlambda\textsubscript{R-flange} \textrightarrow \; \textbf{Noncompact Flange}
\\\\
Since \(\mathbf{{L_p} < {L_b} <= {L_r}}\) and the beam's flanges are \textbf{noncompact}, controlling limit state for flexure is \textbf{LTB (not to exceed capacity based on compression flange local buckling)}.
\\\\
M\textsubscript{p}, the plastic bending moment, is calculated per AISC/ANSI 360-16 Eq. F2-1 as follows:
\begin{flalign*}
M_p = F_y \cdot Z_x  = 50 {\color{darkBlue}{\mathbf{ \; ksi}}} \cdot 12.6 {\color{darkBlue}{\mathbf{ \; {\color{darkBlue}{\mathbf{ \; in}}}^{3}}}}  = \mathbf{52.5 {\color{darkBlue}{\mathbf{ \; kft}}}}
\end{flalign*}
C\textsubscript{b}, the lateral-torsional buckling modification factor in the critical unbraced span for the critical load combination, is calculated per AISC/ANSI 360- 16 Sec. F1 as follows:
\\
\begin{flalign*}
C_b = _bf_{1.0 dimensionless}
\end{flalign*}
\\
For brevity, the C\textsubscript{b} calculation is not shown for each span. The following figure illustrates the value of C\textsubscript{b} for each span.
\begin{figure}[H]
\begin{center}
%% Creator: Matplotlib, PGF backend
%%
%% To include the figure in your LaTeX document, write
%%   \input{<filename>.pgf}
%%
%% Make sure the required packages are loaded in your preamble
%%   \usepackage{pgf}
%%
%% Figures using additional raster images can only be included by \input if
%% they are in the same directory as the main LaTeX file. For loading figures
%% from other directories you can use the `import` package
%%   \usepackage{import}
%%
%% and then include the figures with
%%   \import{<path to file>}{<filename>.pgf}
%%
%% Matplotlib used the following preamble
%%
\begingroup%
\makeatletter%
\begin{pgfpicture}%
\pgfpathrectangle{\pgfpointorigin}{\pgfqpoint{8.000000in}{1.000000in}}%
\pgfusepath{use as bounding box, clip}%
\begin{pgfscope}%
\pgfpathrectangle{\pgfqpoint{1.000000in}{0.110000in}}{\pgfqpoint{6.200000in}{0.770000in}}%
\pgfusepath{clip}%
\pgfsetrectcap%
\pgfsetroundjoin%
\pgfsetlinewidth{0.752812pt}%
\definecolor{currentstroke}{rgb}{0.000000,0.000000,0.000000}%
\pgfsetstrokecolor{currentstroke}%
\pgfsetdash{}{0pt}%
\pgfpathmoveto{\pgfqpoint{1.281818in}{0.880000in}}%
\pgfpathlineto{\pgfqpoint{3.883217in}{0.880000in}}%
\pgfpathlineto{\pgfqpoint{6.918182in}{0.880000in}}%
\pgfusepath{stroke}%
\end{pgfscope}%
\begin{pgfscope}%
\pgfpathrectangle{\pgfqpoint{1.000000in}{0.110000in}}{\pgfqpoint{6.200000in}{0.770000in}}%
\pgfusepath{clip}%
\pgfsetrectcap%
\pgfsetroundjoin%
\pgfsetlinewidth{0.752812pt}%
\definecolor{currentstroke}{rgb}{0.000000,0.000000,0.000000}%
\pgfsetstrokecolor{currentstroke}%
\pgfsetdash{}{0pt}%
\pgfpathmoveto{\pgfqpoint{1.281818in}{0.863617in}}%
\pgfpathlineto{\pgfqpoint{3.883217in}{0.863617in}}%
\pgfpathlineto{\pgfqpoint{6.918182in}{0.863617in}}%
\pgfusepath{stroke}%
\end{pgfscope}%
\begin{pgfscope}%
\pgfpathrectangle{\pgfqpoint{1.000000in}{0.110000in}}{\pgfqpoint{6.200000in}{0.770000in}}%
\pgfusepath{clip}%
\pgfsetrectcap%
\pgfsetroundjoin%
\pgfsetlinewidth{0.752812pt}%
\definecolor{currentstroke}{rgb}{0.000000,0.000000,0.000000}%
\pgfsetstrokecolor{currentstroke}%
\pgfsetdash{}{0pt}%
\pgfpathmoveto{\pgfqpoint{1.281818in}{0.110000in}}%
\pgfpathlineto{\pgfqpoint{3.883217in}{0.110000in}}%
\pgfpathlineto{\pgfqpoint{6.918182in}{0.110000in}}%
\pgfusepath{stroke}%
\end{pgfscope}%
\begin{pgfscope}%
\pgfpathrectangle{\pgfqpoint{1.000000in}{0.110000in}}{\pgfqpoint{6.200000in}{0.770000in}}%
\pgfusepath{clip}%
\pgfsetrectcap%
\pgfsetroundjoin%
\pgfsetlinewidth{0.752812pt}%
\definecolor{currentstroke}{rgb}{0.000000,0.000000,0.000000}%
\pgfsetstrokecolor{currentstroke}%
\pgfsetdash{}{0pt}%
\pgfpathmoveto{\pgfqpoint{1.281818in}{0.126383in}}%
\pgfpathlineto{\pgfqpoint{3.883217in}{0.126383in}}%
\pgfpathlineto{\pgfqpoint{6.918182in}{0.126383in}}%
\pgfusepath{stroke}%
\end{pgfscope}%
\begin{pgfscope}%
\pgfpathrectangle{\pgfqpoint{1.000000in}{0.110000in}}{\pgfqpoint{6.200000in}{0.770000in}}%
\pgfusepath{clip}%
\pgfsetbuttcap%
\pgfsetroundjoin%
\pgfsetlinewidth{1.505625pt}%
\definecolor{currentstroke}{rgb}{1.000000,0.000000,0.000000}%
\pgfsetstrokecolor{currentstroke}%
\pgfsetdash{{1.500000pt}{2.475000pt}}{0.000000pt}%
\pgfpathmoveto{\pgfqpoint{1.281818in}{0.110000in}}%
\pgfpathlineto{\pgfqpoint{1.281818in}{0.880000in}}%
\pgfusepath{stroke}%
\end{pgfscope}%
\begin{pgfscope}%
\pgfpathrectangle{\pgfqpoint{1.000000in}{0.110000in}}{\pgfqpoint{6.200000in}{0.770000in}}%
\pgfusepath{clip}%
\pgfsetbuttcap%
\pgfsetroundjoin%
\pgfsetlinewidth{1.505625pt}%
\definecolor{currentstroke}{rgb}{1.000000,0.000000,0.000000}%
\pgfsetstrokecolor{currentstroke}%
\pgfsetdash{{1.500000pt}{2.475000pt}}{0.000000pt}%
\pgfpathmoveto{\pgfqpoint{3.883217in}{0.110000in}}%
\pgfpathlineto{\pgfqpoint{3.883217in}{0.880000in}}%
\pgfusepath{stroke}%
\end{pgfscope}%
\begin{pgfscope}%
\pgfpathrectangle{\pgfqpoint{1.000000in}{0.110000in}}{\pgfqpoint{6.200000in}{0.770000in}}%
\pgfusepath{clip}%
\pgfsetbuttcap%
\pgfsetroundjoin%
\pgfsetlinewidth{1.505625pt}%
\definecolor{currentstroke}{rgb}{1.000000,0.000000,0.000000}%
\pgfsetstrokecolor{currentstroke}%
\pgfsetdash{{1.500000pt}{2.475000pt}}{0.000000pt}%
\pgfpathmoveto{\pgfqpoint{6.918182in}{0.110000in}}%
\pgfpathlineto{\pgfqpoint{6.918182in}{0.880000in}}%
\pgfusepath{stroke}%
\end{pgfscope}%
\begin{pgfscope}%
\pgfsetbuttcap%
\pgfsetmiterjoin%
\definecolor{currentfill}{rgb}{0.800000,0.800000,0.800000}%
\pgfsetfillcolor{currentfill}%
\pgfsetlinewidth{1.003750pt}%
\definecolor{currentstroke}{rgb}{0.000000,0.000000,0.000000}%
\pgfsetstrokecolor{currentstroke}%
\pgfsetdash{}{0pt}%
\pgfpathmoveto{\pgfqpoint{2.274454in}{0.396599in}}%
\pgfpathcurveto{\pgfqpoint{2.309176in}{0.361877in}}{\pgfqpoint{2.855859in}{0.361877in}}{\pgfqpoint{2.890581in}{0.396599in}}%
\pgfpathcurveto{\pgfqpoint{2.925304in}{0.431321in}}{\pgfqpoint{2.925304in}{0.554778in}}{\pgfqpoint{2.890581in}{0.589500in}}%
\pgfpathcurveto{\pgfqpoint{2.855859in}{0.624222in}}{\pgfqpoint{2.309176in}{0.624222in}}{\pgfqpoint{2.274454in}{0.589500in}}%
\pgfpathcurveto{\pgfqpoint{2.239731in}{0.554778in}}{\pgfqpoint{2.239731in}{0.431321in}}{\pgfqpoint{2.274454in}{0.396599in}}%
\pgfpathclose%
\pgfusepath{stroke,fill}%
\end{pgfscope}%
\begin{pgfscope}%
\definecolor{textcolor}{rgb}{0.000000,0.000000,0.000000}%
\pgfsetstrokecolor{textcolor}%
\pgfsetfillcolor{textcolor}%
\pgftext[x=2.582517in,y=0.493050in,,]{\color{textcolor}\rmfamily\fontsize{10.000000}{12.000000}\selectfont C\textsubscript{b} = 1.0}%
\end{pgfscope}%
\begin{pgfscope}%
\pgfsetbuttcap%
\pgfsetmiterjoin%
\definecolor{currentfill}{rgb}{0.800000,0.800000,0.800000}%
\pgfsetfillcolor{currentfill}%
\pgfsetlinewidth{1.003750pt}%
\definecolor{currentstroke}{rgb}{0.000000,0.000000,0.000000}%
\pgfsetstrokecolor{currentstroke}%
\pgfsetdash{}{0pt}%
\pgfpathmoveto{\pgfqpoint{5.057913in}{0.396599in}}%
\pgfpathcurveto{\pgfqpoint{5.092635in}{0.361877in}}{\pgfqpoint{5.708763in}{0.361877in}}{\pgfqpoint{5.743485in}{0.396599in}}%
\pgfpathcurveto{\pgfqpoint{5.778208in}{0.431321in}}{\pgfqpoint{5.778208in}{0.554778in}}{\pgfqpoint{5.743485in}{0.589500in}}%
\pgfpathcurveto{\pgfqpoint{5.708763in}{0.624222in}}{\pgfqpoint{5.092635in}{0.624222in}}{\pgfqpoint{5.057913in}{0.589500in}}%
\pgfpathcurveto{\pgfqpoint{5.023191in}{0.554778in}}{\pgfqpoint{5.023191in}{0.431321in}}{\pgfqpoint{5.057913in}{0.396599in}}%
\pgfpathclose%
\pgfusepath{stroke,fill}%
\end{pgfscope}%
\begin{pgfscope}%
\definecolor{textcolor}{rgb}{0.000000,0.000000,0.000000}%
\pgfsetstrokecolor{textcolor}%
\pgfsetfillcolor{textcolor}%
\pgftext[x=5.400699in,y=0.493050in,,]{\color{textcolor}\rmfamily\fontsize{10.000000}{12.000000}\selectfont C\textsubscript{b} = 1.79}%
\end{pgfscope}%
\end{pgfpicture}%
\makeatother%
\endgroup%

\end{center}
\caption{C\textsubscript{b} Along Member}
\end{figure}
F\textsubscript{cr}, the buckling stress for the critical section in the critical unbraced span, is calculated per AISC/ANSI 360- 16 Eq. F2-4 as follows:
\begin{flalign*}
F_{cr} & = \cfrac{C_b \cdot \pi^2 \cdot {{E}}} {\left(\cfrac{L_b}{r_{ts}}\right)^2} \cdot \sqrt{1 + 0.078 \cdot \cfrac{{J} \cdot {c}}{{S_x} \cdot h_0} \cdot \left(\cfrac{L_b}{r_{ts}}\right)^2} \\ & = \cfrac{1.0  \cdot \pi^2 \cdot {{29000 {\color{darkBlue}{\mathbf{ \; ksi}}}}}} {\left(\cfrac{6 {\color{darkBlue}{\mathbf{ \; ft}}}}{1.0 {\color{darkBlue}{\mathbf{ \; in}}}}\right)^2} \cdot \sqrt{1 + 0.078 \cdot \cfrac{{0.0547 {\color{darkBlue}{\mathbf{ \; {\color{darkBlue}{\mathbf{ \; in}}}^{4}}}}} \cdot {1}}{{10.9 {\color{darkBlue}{\mathbf{ \; {\color{darkBlue}{\mathbf{ \; in}}}^{3}}}}} \cdot 9.7 {\color{darkBlue}{\mathbf{ \; in}}}} \cdot \left(\cfrac{6 {\color{darkBlue}{\mathbf{ \; ft}}}}{1.0 {\color{darkBlue}{\mathbf{ \; in}}}}\right)^2} = \mathbf{58.9 {\color{darkBlue}{\mathbf{ \; ksi}}}}
\end{flalign*}
\\
\textphi\textsubscript{b}, the resistance factor for bending, is determined per AISC/ANSI 360-16 {\S}F1a as \textbf{0.9}.
\\\\
\textphi\textsubscript{b}M\textsubscript{n}, the design flexural strength, is calculated per AISC/ANSI 360-16 Eq. F2-2 as follows:
\begin{flalign*}
{\phi_b}{M_{n}} & = {\phi_b} \cdot {C_b} \cdot {{M_p} - 0.7 \cdot {F_{y}} \cdot {S_{x}} \cdot \frac{{L_b} - {L_p}}{{L_r} - {L_p}}} < {\phi_b} \cdot {{{M_p} - 0.7 \cdot {F_{y}} \cdot {S_{x}} \cdot {\frac{{\lambda} - {\lambda_p}}{{\lambda_r} - {\lambda_p}}}}} \\ & = {0.9} \cdot {1.0 } \cdot {{52.5 {\color{darkBlue}{\mathbf{ \; kft}}}} - 0.7 \cdot {50 {\color{darkBlue}{\mathbf{ \; ksi}}}} \cdot {10.9 {\color{darkBlue}{\mathbf{ \; {\color{darkBlue}{\mathbf{ \; in}}}^{3}}}}} \cdot \frac{{{6 {\color{darkBlue}{\mathbf{ \; ft}}}}} - {2.8 {\color{darkBlue}{\mathbf{ \; ft}}}}}{{8.1 {\color{darkBlue}{\mathbf{ \; ft}}}} - {2.8 {\color{darkBlue}{\mathbf{ \; ft}}}}}} < {0.9} \cdot {{{52.5 {\color{darkBlue}{\mathbf{ \; kft}}}} - 0.7 \cdot {50 {\color{darkBlue}{\mathbf{ \; ksi}}}} \cdot {10.9 {\color{darkBlue}{\mathbf{ \; {\color{darkBlue}{\mathbf{ \; in}}}^{3}}}}} \cdot {\frac{{9.4 } - {9.2 }}{{24.1 } - {9.2 }}}}} = \mathbf{35.9 {\color{darkBlue}{\mathbf{ \; kft}}}}
\end{flalign*}
\vspace{-20pt}
{\setlength{\mathindent}{0cm}
\begin{flalign*}
\mathbf{|M_u| = 7.5 {\color{darkBlue}{\mathbf{ \; kft}}}  \;  < \phi_b \cdot M_n = 35.9 {\color{darkBlue}{\mathbf{ \; kft}}}  \;  (DCR = 0.21 - OK)}
\end{flalign*}
\textbf{(LTB controls)}
%	-------------------------------- SHEAR CHECK ---------------------------------
\section{Shear Check}
\begin{figure}[H]
\begin{center}
%% Creator: Matplotlib, PGF backend
%%
%% To include the figure in your LaTeX document, write
%%   \input{<filename>.pgf}
%%
%% Make sure the required packages are loaded in your preamble
%%   \usepackage{pgf}
%%
%% Figures using additional raster images can only be included by \input if
%% they are in the same directory as the main LaTeX file. For loading figures
%% from other directories you can use the `import` package
%%   \usepackage{import}
%%
%% and then include the figures with
%%   \import{<path to file>}{<filename>.pgf}
%%
%% Matplotlib used the following preamble
%%
\begingroup%
\makeatletter%
\begin{pgfpicture}%
\pgfpathrectangle{\pgfpointorigin}{\pgfqpoint{8.000000in}{3.000000in}}%
\pgfusepath{use as bounding box, clip}%
\begin{pgfscope}%
\pgfsetbuttcap%
\pgfsetmiterjoin%
\definecolor{currentfill}{rgb}{1.000000,1.000000,1.000000}%
\pgfsetfillcolor{currentfill}%
\pgfsetlinewidth{0.000000pt}%
\definecolor{currentstroke}{rgb}{1.000000,1.000000,1.000000}%
\pgfsetstrokecolor{currentstroke}%
\pgfsetdash{}{0pt}%
\pgfpathmoveto{\pgfqpoint{0.000000in}{0.000000in}}%
\pgfpathlineto{\pgfqpoint{8.000000in}{0.000000in}}%
\pgfpathlineto{\pgfqpoint{8.000000in}{3.000000in}}%
\pgfpathlineto{\pgfqpoint{0.000000in}{3.000000in}}%
\pgfpathclose%
\pgfusepath{fill}%
\end{pgfscope}%
\begin{pgfscope}%
\pgfsetbuttcap%
\pgfsetmiterjoin%
\definecolor{currentfill}{rgb}{1.000000,1.000000,1.000000}%
\pgfsetfillcolor{currentfill}%
\pgfsetlinewidth{0.000000pt}%
\definecolor{currentstroke}{rgb}{0.000000,0.000000,0.000000}%
\pgfsetstrokecolor{currentstroke}%
\pgfsetstrokeopacity{0.000000}%
\pgfsetdash{}{0pt}%
\pgfpathmoveto{\pgfqpoint{1.000000in}{0.330000in}}%
\pgfpathlineto{\pgfqpoint{7.200000in}{0.330000in}}%
\pgfpathlineto{\pgfqpoint{7.200000in}{2.640000in}}%
\pgfpathlineto{\pgfqpoint{1.000000in}{2.640000in}}%
\pgfpathclose%
\pgfusepath{fill}%
\end{pgfscope}%
\begin{pgfscope}%
\pgfpathrectangle{\pgfqpoint{1.000000in}{0.330000in}}{\pgfqpoint{6.200000in}{2.310000in}}%
\pgfusepath{clip}%
\pgfsetbuttcap%
\pgfsetroundjoin%
\pgfsetlinewidth{0.803000pt}%
\definecolor{currentstroke}{rgb}{0.000000,0.000000,0.000000}%
\pgfsetstrokecolor{currentstroke}%
\pgfsetdash{{0.800000pt}{1.320000pt}}{0.000000pt}%
\pgfpathmoveto{\pgfqpoint{1.281818in}{0.330000in}}%
\pgfpathlineto{\pgfqpoint{1.281818in}{2.640000in}}%
\pgfusepath{stroke}%
\end{pgfscope}%
\begin{pgfscope}%
\pgfsetbuttcap%
\pgfsetroundjoin%
\definecolor{currentfill}{rgb}{0.000000,0.000000,0.000000}%
\pgfsetfillcolor{currentfill}%
\pgfsetlinewidth{0.803000pt}%
\definecolor{currentstroke}{rgb}{0.000000,0.000000,0.000000}%
\pgfsetstrokecolor{currentstroke}%
\pgfsetdash{}{0pt}%
\pgfsys@defobject{currentmarker}{\pgfqpoint{0.000000in}{-0.048611in}}{\pgfqpoint{0.000000in}{0.000000in}}{%
\pgfpathmoveto{\pgfqpoint{0.000000in}{0.000000in}}%
\pgfpathlineto{\pgfqpoint{0.000000in}{-0.048611in}}%
\pgfusepath{stroke,fill}%
}%
\begin{pgfscope}%
\pgfsys@transformshift{1.281818in}{0.330000in}%
\pgfsys@useobject{currentmarker}{}%
\end{pgfscope}%
\end{pgfscope}%
\begin{pgfscope}%
\pgfsetbuttcap%
\pgfsetroundjoin%
\definecolor{currentfill}{rgb}{0.000000,0.000000,0.000000}%
\pgfsetfillcolor{currentfill}%
\pgfsetlinewidth{0.803000pt}%
\definecolor{currentstroke}{rgb}{0.000000,0.000000,0.000000}%
\pgfsetstrokecolor{currentstroke}%
\pgfsetdash{}{0pt}%
\pgfsys@defobject{currentmarker}{\pgfqpoint{0.000000in}{0.000000in}}{\pgfqpoint{0.000000in}{0.048611in}}{%
\pgfpathmoveto{\pgfqpoint{0.000000in}{0.000000in}}%
\pgfpathlineto{\pgfqpoint{0.000000in}{0.048611in}}%
\pgfusepath{stroke,fill}%
}%
\begin{pgfscope}%
\pgfsys@transformshift{1.281818in}{2.640000in}%
\pgfsys@useobject{currentmarker}{}%
\end{pgfscope}%
\end{pgfscope}%
\begin{pgfscope}%
\definecolor{textcolor}{rgb}{0.000000,0.000000,0.000000}%
\pgfsetstrokecolor{textcolor}%
\pgfsetfillcolor{textcolor}%
\pgftext[x=1.281818in,y=0.232778in,,top]{\color{textcolor}\rmfamily\fontsize{10.000000}{12.000000}\selectfont \(\displaystyle {0}\)}%
\end{pgfscope}%
\begin{pgfscope}%
\pgfpathrectangle{\pgfqpoint{1.000000in}{0.330000in}}{\pgfqpoint{6.200000in}{2.310000in}}%
\pgfusepath{clip}%
\pgfsetbuttcap%
\pgfsetroundjoin%
\pgfsetlinewidth{0.803000pt}%
\definecolor{currentstroke}{rgb}{0.000000,0.000000,0.000000}%
\pgfsetstrokecolor{currentstroke}%
\pgfsetdash{{0.800000pt}{1.320000pt}}{0.000000pt}%
\pgfpathmoveto{\pgfqpoint{2.148951in}{0.330000in}}%
\pgfpathlineto{\pgfqpoint{2.148951in}{2.640000in}}%
\pgfusepath{stroke}%
\end{pgfscope}%
\begin{pgfscope}%
\pgfsetbuttcap%
\pgfsetroundjoin%
\definecolor{currentfill}{rgb}{0.000000,0.000000,0.000000}%
\pgfsetfillcolor{currentfill}%
\pgfsetlinewidth{0.803000pt}%
\definecolor{currentstroke}{rgb}{0.000000,0.000000,0.000000}%
\pgfsetstrokecolor{currentstroke}%
\pgfsetdash{}{0pt}%
\pgfsys@defobject{currentmarker}{\pgfqpoint{0.000000in}{-0.048611in}}{\pgfqpoint{0.000000in}{0.000000in}}{%
\pgfpathmoveto{\pgfqpoint{0.000000in}{0.000000in}}%
\pgfpathlineto{\pgfqpoint{0.000000in}{-0.048611in}}%
\pgfusepath{stroke,fill}%
}%
\begin{pgfscope}%
\pgfsys@transformshift{2.148951in}{0.330000in}%
\pgfsys@useobject{currentmarker}{}%
\end{pgfscope}%
\end{pgfscope}%
\begin{pgfscope}%
\pgfsetbuttcap%
\pgfsetroundjoin%
\definecolor{currentfill}{rgb}{0.000000,0.000000,0.000000}%
\pgfsetfillcolor{currentfill}%
\pgfsetlinewidth{0.803000pt}%
\definecolor{currentstroke}{rgb}{0.000000,0.000000,0.000000}%
\pgfsetstrokecolor{currentstroke}%
\pgfsetdash{}{0pt}%
\pgfsys@defobject{currentmarker}{\pgfqpoint{0.000000in}{0.000000in}}{\pgfqpoint{0.000000in}{0.048611in}}{%
\pgfpathmoveto{\pgfqpoint{0.000000in}{0.000000in}}%
\pgfpathlineto{\pgfqpoint{0.000000in}{0.048611in}}%
\pgfusepath{stroke,fill}%
}%
\begin{pgfscope}%
\pgfsys@transformshift{2.148951in}{2.640000in}%
\pgfsys@useobject{currentmarker}{}%
\end{pgfscope}%
\end{pgfscope}%
\begin{pgfscope}%
\definecolor{textcolor}{rgb}{0.000000,0.000000,0.000000}%
\pgfsetstrokecolor{textcolor}%
\pgfsetfillcolor{textcolor}%
\pgftext[x=2.148951in,y=0.232778in,,top]{\color{textcolor}\rmfamily\fontsize{10.000000}{12.000000}\selectfont \(\displaystyle {2}\)}%
\end{pgfscope}%
\begin{pgfscope}%
\pgfpathrectangle{\pgfqpoint{1.000000in}{0.330000in}}{\pgfqpoint{6.200000in}{2.310000in}}%
\pgfusepath{clip}%
\pgfsetbuttcap%
\pgfsetroundjoin%
\pgfsetlinewidth{0.803000pt}%
\definecolor{currentstroke}{rgb}{0.000000,0.000000,0.000000}%
\pgfsetstrokecolor{currentstroke}%
\pgfsetdash{{0.800000pt}{1.320000pt}}{0.000000pt}%
\pgfpathmoveto{\pgfqpoint{3.016084in}{0.330000in}}%
\pgfpathlineto{\pgfqpoint{3.016084in}{2.640000in}}%
\pgfusepath{stroke}%
\end{pgfscope}%
\begin{pgfscope}%
\pgfsetbuttcap%
\pgfsetroundjoin%
\definecolor{currentfill}{rgb}{0.000000,0.000000,0.000000}%
\pgfsetfillcolor{currentfill}%
\pgfsetlinewidth{0.803000pt}%
\definecolor{currentstroke}{rgb}{0.000000,0.000000,0.000000}%
\pgfsetstrokecolor{currentstroke}%
\pgfsetdash{}{0pt}%
\pgfsys@defobject{currentmarker}{\pgfqpoint{0.000000in}{-0.048611in}}{\pgfqpoint{0.000000in}{0.000000in}}{%
\pgfpathmoveto{\pgfqpoint{0.000000in}{0.000000in}}%
\pgfpathlineto{\pgfqpoint{0.000000in}{-0.048611in}}%
\pgfusepath{stroke,fill}%
}%
\begin{pgfscope}%
\pgfsys@transformshift{3.016084in}{0.330000in}%
\pgfsys@useobject{currentmarker}{}%
\end{pgfscope}%
\end{pgfscope}%
\begin{pgfscope}%
\pgfsetbuttcap%
\pgfsetroundjoin%
\definecolor{currentfill}{rgb}{0.000000,0.000000,0.000000}%
\pgfsetfillcolor{currentfill}%
\pgfsetlinewidth{0.803000pt}%
\definecolor{currentstroke}{rgb}{0.000000,0.000000,0.000000}%
\pgfsetstrokecolor{currentstroke}%
\pgfsetdash{}{0pt}%
\pgfsys@defobject{currentmarker}{\pgfqpoint{0.000000in}{0.000000in}}{\pgfqpoint{0.000000in}{0.048611in}}{%
\pgfpathmoveto{\pgfqpoint{0.000000in}{0.000000in}}%
\pgfpathlineto{\pgfqpoint{0.000000in}{0.048611in}}%
\pgfusepath{stroke,fill}%
}%
\begin{pgfscope}%
\pgfsys@transformshift{3.016084in}{2.640000in}%
\pgfsys@useobject{currentmarker}{}%
\end{pgfscope}%
\end{pgfscope}%
\begin{pgfscope}%
\definecolor{textcolor}{rgb}{0.000000,0.000000,0.000000}%
\pgfsetstrokecolor{textcolor}%
\pgfsetfillcolor{textcolor}%
\pgftext[x=3.016084in,y=0.232778in,,top]{\color{textcolor}\rmfamily\fontsize{10.000000}{12.000000}\selectfont \(\displaystyle {4}\)}%
\end{pgfscope}%
\begin{pgfscope}%
\pgfpathrectangle{\pgfqpoint{1.000000in}{0.330000in}}{\pgfqpoint{6.200000in}{2.310000in}}%
\pgfusepath{clip}%
\pgfsetbuttcap%
\pgfsetroundjoin%
\pgfsetlinewidth{0.803000pt}%
\definecolor{currentstroke}{rgb}{0.000000,0.000000,0.000000}%
\pgfsetstrokecolor{currentstroke}%
\pgfsetdash{{0.800000pt}{1.320000pt}}{0.000000pt}%
\pgfpathmoveto{\pgfqpoint{3.883217in}{0.330000in}}%
\pgfpathlineto{\pgfqpoint{3.883217in}{2.640000in}}%
\pgfusepath{stroke}%
\end{pgfscope}%
\begin{pgfscope}%
\pgfsetbuttcap%
\pgfsetroundjoin%
\definecolor{currentfill}{rgb}{0.000000,0.000000,0.000000}%
\pgfsetfillcolor{currentfill}%
\pgfsetlinewidth{0.803000pt}%
\definecolor{currentstroke}{rgb}{0.000000,0.000000,0.000000}%
\pgfsetstrokecolor{currentstroke}%
\pgfsetdash{}{0pt}%
\pgfsys@defobject{currentmarker}{\pgfqpoint{0.000000in}{-0.048611in}}{\pgfqpoint{0.000000in}{0.000000in}}{%
\pgfpathmoveto{\pgfqpoint{0.000000in}{0.000000in}}%
\pgfpathlineto{\pgfqpoint{0.000000in}{-0.048611in}}%
\pgfusepath{stroke,fill}%
}%
\begin{pgfscope}%
\pgfsys@transformshift{3.883217in}{0.330000in}%
\pgfsys@useobject{currentmarker}{}%
\end{pgfscope}%
\end{pgfscope}%
\begin{pgfscope}%
\pgfsetbuttcap%
\pgfsetroundjoin%
\definecolor{currentfill}{rgb}{0.000000,0.000000,0.000000}%
\pgfsetfillcolor{currentfill}%
\pgfsetlinewidth{0.803000pt}%
\definecolor{currentstroke}{rgb}{0.000000,0.000000,0.000000}%
\pgfsetstrokecolor{currentstroke}%
\pgfsetdash{}{0pt}%
\pgfsys@defobject{currentmarker}{\pgfqpoint{0.000000in}{0.000000in}}{\pgfqpoint{0.000000in}{0.048611in}}{%
\pgfpathmoveto{\pgfqpoint{0.000000in}{0.000000in}}%
\pgfpathlineto{\pgfqpoint{0.000000in}{0.048611in}}%
\pgfusepath{stroke,fill}%
}%
\begin{pgfscope}%
\pgfsys@transformshift{3.883217in}{2.640000in}%
\pgfsys@useobject{currentmarker}{}%
\end{pgfscope}%
\end{pgfscope}%
\begin{pgfscope}%
\definecolor{textcolor}{rgb}{0.000000,0.000000,0.000000}%
\pgfsetstrokecolor{textcolor}%
\pgfsetfillcolor{textcolor}%
\pgftext[x=3.883217in,y=0.232778in,,top]{\color{textcolor}\rmfamily\fontsize{10.000000}{12.000000}\selectfont \(\displaystyle {6}\)}%
\end{pgfscope}%
\begin{pgfscope}%
\pgfpathrectangle{\pgfqpoint{1.000000in}{0.330000in}}{\pgfqpoint{6.200000in}{2.310000in}}%
\pgfusepath{clip}%
\pgfsetbuttcap%
\pgfsetroundjoin%
\pgfsetlinewidth{0.803000pt}%
\definecolor{currentstroke}{rgb}{0.000000,0.000000,0.000000}%
\pgfsetstrokecolor{currentstroke}%
\pgfsetdash{{0.800000pt}{1.320000pt}}{0.000000pt}%
\pgfpathmoveto{\pgfqpoint{4.750350in}{0.330000in}}%
\pgfpathlineto{\pgfqpoint{4.750350in}{2.640000in}}%
\pgfusepath{stroke}%
\end{pgfscope}%
\begin{pgfscope}%
\pgfsetbuttcap%
\pgfsetroundjoin%
\definecolor{currentfill}{rgb}{0.000000,0.000000,0.000000}%
\pgfsetfillcolor{currentfill}%
\pgfsetlinewidth{0.803000pt}%
\definecolor{currentstroke}{rgb}{0.000000,0.000000,0.000000}%
\pgfsetstrokecolor{currentstroke}%
\pgfsetdash{}{0pt}%
\pgfsys@defobject{currentmarker}{\pgfqpoint{0.000000in}{-0.048611in}}{\pgfqpoint{0.000000in}{0.000000in}}{%
\pgfpathmoveto{\pgfqpoint{0.000000in}{0.000000in}}%
\pgfpathlineto{\pgfqpoint{0.000000in}{-0.048611in}}%
\pgfusepath{stroke,fill}%
}%
\begin{pgfscope}%
\pgfsys@transformshift{4.750350in}{0.330000in}%
\pgfsys@useobject{currentmarker}{}%
\end{pgfscope}%
\end{pgfscope}%
\begin{pgfscope}%
\pgfsetbuttcap%
\pgfsetroundjoin%
\definecolor{currentfill}{rgb}{0.000000,0.000000,0.000000}%
\pgfsetfillcolor{currentfill}%
\pgfsetlinewidth{0.803000pt}%
\definecolor{currentstroke}{rgb}{0.000000,0.000000,0.000000}%
\pgfsetstrokecolor{currentstroke}%
\pgfsetdash{}{0pt}%
\pgfsys@defobject{currentmarker}{\pgfqpoint{0.000000in}{0.000000in}}{\pgfqpoint{0.000000in}{0.048611in}}{%
\pgfpathmoveto{\pgfqpoint{0.000000in}{0.000000in}}%
\pgfpathlineto{\pgfqpoint{0.000000in}{0.048611in}}%
\pgfusepath{stroke,fill}%
}%
\begin{pgfscope}%
\pgfsys@transformshift{4.750350in}{2.640000in}%
\pgfsys@useobject{currentmarker}{}%
\end{pgfscope}%
\end{pgfscope}%
\begin{pgfscope}%
\definecolor{textcolor}{rgb}{0.000000,0.000000,0.000000}%
\pgfsetstrokecolor{textcolor}%
\pgfsetfillcolor{textcolor}%
\pgftext[x=4.750350in,y=0.232778in,,top]{\color{textcolor}\rmfamily\fontsize{10.000000}{12.000000}\selectfont \(\displaystyle {8}\)}%
\end{pgfscope}%
\begin{pgfscope}%
\pgfpathrectangle{\pgfqpoint{1.000000in}{0.330000in}}{\pgfqpoint{6.200000in}{2.310000in}}%
\pgfusepath{clip}%
\pgfsetbuttcap%
\pgfsetroundjoin%
\pgfsetlinewidth{0.803000pt}%
\definecolor{currentstroke}{rgb}{0.000000,0.000000,0.000000}%
\pgfsetstrokecolor{currentstroke}%
\pgfsetdash{{0.800000pt}{1.320000pt}}{0.000000pt}%
\pgfpathmoveto{\pgfqpoint{5.617483in}{0.330000in}}%
\pgfpathlineto{\pgfqpoint{5.617483in}{2.640000in}}%
\pgfusepath{stroke}%
\end{pgfscope}%
\begin{pgfscope}%
\pgfsetbuttcap%
\pgfsetroundjoin%
\definecolor{currentfill}{rgb}{0.000000,0.000000,0.000000}%
\pgfsetfillcolor{currentfill}%
\pgfsetlinewidth{0.803000pt}%
\definecolor{currentstroke}{rgb}{0.000000,0.000000,0.000000}%
\pgfsetstrokecolor{currentstroke}%
\pgfsetdash{}{0pt}%
\pgfsys@defobject{currentmarker}{\pgfqpoint{0.000000in}{-0.048611in}}{\pgfqpoint{0.000000in}{0.000000in}}{%
\pgfpathmoveto{\pgfqpoint{0.000000in}{0.000000in}}%
\pgfpathlineto{\pgfqpoint{0.000000in}{-0.048611in}}%
\pgfusepath{stroke,fill}%
}%
\begin{pgfscope}%
\pgfsys@transformshift{5.617483in}{0.330000in}%
\pgfsys@useobject{currentmarker}{}%
\end{pgfscope}%
\end{pgfscope}%
\begin{pgfscope}%
\pgfsetbuttcap%
\pgfsetroundjoin%
\definecolor{currentfill}{rgb}{0.000000,0.000000,0.000000}%
\pgfsetfillcolor{currentfill}%
\pgfsetlinewidth{0.803000pt}%
\definecolor{currentstroke}{rgb}{0.000000,0.000000,0.000000}%
\pgfsetstrokecolor{currentstroke}%
\pgfsetdash{}{0pt}%
\pgfsys@defobject{currentmarker}{\pgfqpoint{0.000000in}{0.000000in}}{\pgfqpoint{0.000000in}{0.048611in}}{%
\pgfpathmoveto{\pgfqpoint{0.000000in}{0.000000in}}%
\pgfpathlineto{\pgfqpoint{0.000000in}{0.048611in}}%
\pgfusepath{stroke,fill}%
}%
\begin{pgfscope}%
\pgfsys@transformshift{5.617483in}{2.640000in}%
\pgfsys@useobject{currentmarker}{}%
\end{pgfscope}%
\end{pgfscope}%
\begin{pgfscope}%
\definecolor{textcolor}{rgb}{0.000000,0.000000,0.000000}%
\pgfsetstrokecolor{textcolor}%
\pgfsetfillcolor{textcolor}%
\pgftext[x=5.617483in,y=0.232778in,,top]{\color{textcolor}\rmfamily\fontsize{10.000000}{12.000000}\selectfont \(\displaystyle {10}\)}%
\end{pgfscope}%
\begin{pgfscope}%
\pgfpathrectangle{\pgfqpoint{1.000000in}{0.330000in}}{\pgfqpoint{6.200000in}{2.310000in}}%
\pgfusepath{clip}%
\pgfsetbuttcap%
\pgfsetroundjoin%
\pgfsetlinewidth{0.803000pt}%
\definecolor{currentstroke}{rgb}{0.000000,0.000000,0.000000}%
\pgfsetstrokecolor{currentstroke}%
\pgfsetdash{{0.800000pt}{1.320000pt}}{0.000000pt}%
\pgfpathmoveto{\pgfqpoint{6.484615in}{0.330000in}}%
\pgfpathlineto{\pgfqpoint{6.484615in}{2.640000in}}%
\pgfusepath{stroke}%
\end{pgfscope}%
\begin{pgfscope}%
\pgfsetbuttcap%
\pgfsetroundjoin%
\definecolor{currentfill}{rgb}{0.000000,0.000000,0.000000}%
\pgfsetfillcolor{currentfill}%
\pgfsetlinewidth{0.803000pt}%
\definecolor{currentstroke}{rgb}{0.000000,0.000000,0.000000}%
\pgfsetstrokecolor{currentstroke}%
\pgfsetdash{}{0pt}%
\pgfsys@defobject{currentmarker}{\pgfqpoint{0.000000in}{-0.048611in}}{\pgfqpoint{0.000000in}{0.000000in}}{%
\pgfpathmoveto{\pgfqpoint{0.000000in}{0.000000in}}%
\pgfpathlineto{\pgfqpoint{0.000000in}{-0.048611in}}%
\pgfusepath{stroke,fill}%
}%
\begin{pgfscope}%
\pgfsys@transformshift{6.484615in}{0.330000in}%
\pgfsys@useobject{currentmarker}{}%
\end{pgfscope}%
\end{pgfscope}%
\begin{pgfscope}%
\pgfsetbuttcap%
\pgfsetroundjoin%
\definecolor{currentfill}{rgb}{0.000000,0.000000,0.000000}%
\pgfsetfillcolor{currentfill}%
\pgfsetlinewidth{0.803000pt}%
\definecolor{currentstroke}{rgb}{0.000000,0.000000,0.000000}%
\pgfsetstrokecolor{currentstroke}%
\pgfsetdash{}{0pt}%
\pgfsys@defobject{currentmarker}{\pgfqpoint{0.000000in}{0.000000in}}{\pgfqpoint{0.000000in}{0.048611in}}{%
\pgfpathmoveto{\pgfqpoint{0.000000in}{0.000000in}}%
\pgfpathlineto{\pgfqpoint{0.000000in}{0.048611in}}%
\pgfusepath{stroke,fill}%
}%
\begin{pgfscope}%
\pgfsys@transformshift{6.484615in}{2.640000in}%
\pgfsys@useobject{currentmarker}{}%
\end{pgfscope}%
\end{pgfscope}%
\begin{pgfscope}%
\definecolor{textcolor}{rgb}{0.000000,0.000000,0.000000}%
\pgfsetstrokecolor{textcolor}%
\pgfsetfillcolor{textcolor}%
\pgftext[x=6.484615in,y=0.232778in,,top]{\color{textcolor}\rmfamily\fontsize{10.000000}{12.000000}\selectfont \(\displaystyle {12}\)}%
\end{pgfscope}%
\begin{pgfscope}%
\pgfpathrectangle{\pgfqpoint{1.000000in}{0.330000in}}{\pgfqpoint{6.200000in}{2.310000in}}%
\pgfusepath{clip}%
\pgfsetbuttcap%
\pgfsetroundjoin%
\pgfsetlinewidth{0.803000pt}%
\definecolor{currentstroke}{rgb}{0.000000,0.000000,0.000000}%
\pgfsetstrokecolor{currentstroke}%
\pgfsetdash{{0.800000pt}{1.320000pt}}{0.000000pt}%
\pgfpathmoveto{\pgfqpoint{1.000000in}{0.406348in}}%
\pgfpathlineto{\pgfqpoint{7.200000in}{0.406348in}}%
\pgfusepath{stroke}%
\end{pgfscope}%
\begin{pgfscope}%
\pgfsetbuttcap%
\pgfsetroundjoin%
\definecolor{currentfill}{rgb}{0.000000,0.000000,0.000000}%
\pgfsetfillcolor{currentfill}%
\pgfsetlinewidth{0.803000pt}%
\definecolor{currentstroke}{rgb}{0.000000,0.000000,0.000000}%
\pgfsetstrokecolor{currentstroke}%
\pgfsetdash{}{0pt}%
\pgfsys@defobject{currentmarker}{\pgfqpoint{-0.048611in}{0.000000in}}{\pgfqpoint{-0.000000in}{0.000000in}}{%
\pgfpathmoveto{\pgfqpoint{-0.000000in}{0.000000in}}%
\pgfpathlineto{\pgfqpoint{-0.048611in}{0.000000in}}%
\pgfusepath{stroke,fill}%
}%
\begin{pgfscope}%
\pgfsys@transformshift{1.000000in}{0.406348in}%
\pgfsys@useobject{currentmarker}{}%
\end{pgfscope}%
\end{pgfscope}%
\begin{pgfscope}%
\pgfsetbuttcap%
\pgfsetroundjoin%
\definecolor{currentfill}{rgb}{0.000000,0.000000,0.000000}%
\pgfsetfillcolor{currentfill}%
\pgfsetlinewidth{0.803000pt}%
\definecolor{currentstroke}{rgb}{0.000000,0.000000,0.000000}%
\pgfsetstrokecolor{currentstroke}%
\pgfsetdash{}{0pt}%
\pgfsys@defobject{currentmarker}{\pgfqpoint{0.000000in}{0.000000in}}{\pgfqpoint{0.048611in}{0.000000in}}{%
\pgfpathmoveto{\pgfqpoint{0.000000in}{0.000000in}}%
\pgfpathlineto{\pgfqpoint{0.048611in}{0.000000in}}%
\pgfusepath{stroke,fill}%
}%
\begin{pgfscope}%
\pgfsys@transformshift{7.200000in}{0.406348in}%
\pgfsys@useobject{currentmarker}{}%
\end{pgfscope}%
\end{pgfscope}%
\begin{pgfscope}%
\definecolor{textcolor}{rgb}{0.000000,0.000000,0.000000}%
\pgfsetstrokecolor{textcolor}%
\pgfsetfillcolor{textcolor}%
\pgftext[x=0.617283in, y=0.358122in, left, base]{\color{textcolor}\rmfamily\fontsize{10.000000}{12.000000}\selectfont \(\displaystyle {\ensuremath{-}1.5}\)}%
\end{pgfscope}%
\begin{pgfscope}%
\pgfpathrectangle{\pgfqpoint{1.000000in}{0.330000in}}{\pgfqpoint{6.200000in}{2.310000in}}%
\pgfusepath{clip}%
\pgfsetbuttcap%
\pgfsetroundjoin%
\pgfsetlinewidth{0.803000pt}%
\definecolor{currentstroke}{rgb}{0.000000,0.000000,0.000000}%
\pgfsetstrokecolor{currentstroke}%
\pgfsetdash{{0.800000pt}{1.320000pt}}{0.000000pt}%
\pgfpathmoveto{\pgfqpoint{1.000000in}{0.780724in}}%
\pgfpathlineto{\pgfqpoint{7.200000in}{0.780724in}}%
\pgfusepath{stroke}%
\end{pgfscope}%
\begin{pgfscope}%
\pgfsetbuttcap%
\pgfsetroundjoin%
\definecolor{currentfill}{rgb}{0.000000,0.000000,0.000000}%
\pgfsetfillcolor{currentfill}%
\pgfsetlinewidth{0.803000pt}%
\definecolor{currentstroke}{rgb}{0.000000,0.000000,0.000000}%
\pgfsetstrokecolor{currentstroke}%
\pgfsetdash{}{0pt}%
\pgfsys@defobject{currentmarker}{\pgfqpoint{-0.048611in}{0.000000in}}{\pgfqpoint{-0.000000in}{0.000000in}}{%
\pgfpathmoveto{\pgfqpoint{-0.000000in}{0.000000in}}%
\pgfpathlineto{\pgfqpoint{-0.048611in}{0.000000in}}%
\pgfusepath{stroke,fill}%
}%
\begin{pgfscope}%
\pgfsys@transformshift{1.000000in}{0.780724in}%
\pgfsys@useobject{currentmarker}{}%
\end{pgfscope}%
\end{pgfscope}%
\begin{pgfscope}%
\pgfsetbuttcap%
\pgfsetroundjoin%
\definecolor{currentfill}{rgb}{0.000000,0.000000,0.000000}%
\pgfsetfillcolor{currentfill}%
\pgfsetlinewidth{0.803000pt}%
\definecolor{currentstroke}{rgb}{0.000000,0.000000,0.000000}%
\pgfsetstrokecolor{currentstroke}%
\pgfsetdash{}{0pt}%
\pgfsys@defobject{currentmarker}{\pgfqpoint{0.000000in}{0.000000in}}{\pgfqpoint{0.048611in}{0.000000in}}{%
\pgfpathmoveto{\pgfqpoint{0.000000in}{0.000000in}}%
\pgfpathlineto{\pgfqpoint{0.048611in}{0.000000in}}%
\pgfusepath{stroke,fill}%
}%
\begin{pgfscope}%
\pgfsys@transformshift{7.200000in}{0.780724in}%
\pgfsys@useobject{currentmarker}{}%
\end{pgfscope}%
\end{pgfscope}%
\begin{pgfscope}%
\definecolor{textcolor}{rgb}{0.000000,0.000000,0.000000}%
\pgfsetstrokecolor{textcolor}%
\pgfsetfillcolor{textcolor}%
\pgftext[x=0.617283in, y=0.732498in, left, base]{\color{textcolor}\rmfamily\fontsize{10.000000}{12.000000}\selectfont \(\displaystyle {\ensuremath{-}1.0}\)}%
\end{pgfscope}%
\begin{pgfscope}%
\pgfpathrectangle{\pgfqpoint{1.000000in}{0.330000in}}{\pgfqpoint{6.200000in}{2.310000in}}%
\pgfusepath{clip}%
\pgfsetbuttcap%
\pgfsetroundjoin%
\pgfsetlinewidth{0.803000pt}%
\definecolor{currentstroke}{rgb}{0.000000,0.000000,0.000000}%
\pgfsetstrokecolor{currentstroke}%
\pgfsetdash{{0.800000pt}{1.320000pt}}{0.000000pt}%
\pgfpathmoveto{\pgfqpoint{1.000000in}{1.155100in}}%
\pgfpathlineto{\pgfqpoint{7.200000in}{1.155100in}}%
\pgfusepath{stroke}%
\end{pgfscope}%
\begin{pgfscope}%
\pgfsetbuttcap%
\pgfsetroundjoin%
\definecolor{currentfill}{rgb}{0.000000,0.000000,0.000000}%
\pgfsetfillcolor{currentfill}%
\pgfsetlinewidth{0.803000pt}%
\definecolor{currentstroke}{rgb}{0.000000,0.000000,0.000000}%
\pgfsetstrokecolor{currentstroke}%
\pgfsetdash{}{0pt}%
\pgfsys@defobject{currentmarker}{\pgfqpoint{-0.048611in}{0.000000in}}{\pgfqpoint{-0.000000in}{0.000000in}}{%
\pgfpathmoveto{\pgfqpoint{-0.000000in}{0.000000in}}%
\pgfpathlineto{\pgfqpoint{-0.048611in}{0.000000in}}%
\pgfusepath{stroke,fill}%
}%
\begin{pgfscope}%
\pgfsys@transformshift{1.000000in}{1.155100in}%
\pgfsys@useobject{currentmarker}{}%
\end{pgfscope}%
\end{pgfscope}%
\begin{pgfscope}%
\pgfsetbuttcap%
\pgfsetroundjoin%
\definecolor{currentfill}{rgb}{0.000000,0.000000,0.000000}%
\pgfsetfillcolor{currentfill}%
\pgfsetlinewidth{0.803000pt}%
\definecolor{currentstroke}{rgb}{0.000000,0.000000,0.000000}%
\pgfsetstrokecolor{currentstroke}%
\pgfsetdash{}{0pt}%
\pgfsys@defobject{currentmarker}{\pgfqpoint{0.000000in}{0.000000in}}{\pgfqpoint{0.048611in}{0.000000in}}{%
\pgfpathmoveto{\pgfqpoint{0.000000in}{0.000000in}}%
\pgfpathlineto{\pgfqpoint{0.048611in}{0.000000in}}%
\pgfusepath{stroke,fill}%
}%
\begin{pgfscope}%
\pgfsys@transformshift{7.200000in}{1.155100in}%
\pgfsys@useobject{currentmarker}{}%
\end{pgfscope}%
\end{pgfscope}%
\begin{pgfscope}%
\definecolor{textcolor}{rgb}{0.000000,0.000000,0.000000}%
\pgfsetstrokecolor{textcolor}%
\pgfsetfillcolor{textcolor}%
\pgftext[x=0.617283in, y=1.106874in, left, base]{\color{textcolor}\rmfamily\fontsize{10.000000}{12.000000}\selectfont \(\displaystyle {\ensuremath{-}0.5}\)}%
\end{pgfscope}%
\begin{pgfscope}%
\pgfpathrectangle{\pgfqpoint{1.000000in}{0.330000in}}{\pgfqpoint{6.200000in}{2.310000in}}%
\pgfusepath{clip}%
\pgfsetbuttcap%
\pgfsetroundjoin%
\pgfsetlinewidth{0.803000pt}%
\definecolor{currentstroke}{rgb}{0.000000,0.000000,0.000000}%
\pgfsetstrokecolor{currentstroke}%
\pgfsetdash{{0.800000pt}{1.320000pt}}{0.000000pt}%
\pgfpathmoveto{\pgfqpoint{1.000000in}{1.529475in}}%
\pgfpathlineto{\pgfqpoint{7.200000in}{1.529475in}}%
\pgfusepath{stroke}%
\end{pgfscope}%
\begin{pgfscope}%
\pgfsetbuttcap%
\pgfsetroundjoin%
\definecolor{currentfill}{rgb}{0.000000,0.000000,0.000000}%
\pgfsetfillcolor{currentfill}%
\pgfsetlinewidth{0.803000pt}%
\definecolor{currentstroke}{rgb}{0.000000,0.000000,0.000000}%
\pgfsetstrokecolor{currentstroke}%
\pgfsetdash{}{0pt}%
\pgfsys@defobject{currentmarker}{\pgfqpoint{-0.048611in}{0.000000in}}{\pgfqpoint{-0.000000in}{0.000000in}}{%
\pgfpathmoveto{\pgfqpoint{-0.000000in}{0.000000in}}%
\pgfpathlineto{\pgfqpoint{-0.048611in}{0.000000in}}%
\pgfusepath{stroke,fill}%
}%
\begin{pgfscope}%
\pgfsys@transformshift{1.000000in}{1.529475in}%
\pgfsys@useobject{currentmarker}{}%
\end{pgfscope}%
\end{pgfscope}%
\begin{pgfscope}%
\pgfsetbuttcap%
\pgfsetroundjoin%
\definecolor{currentfill}{rgb}{0.000000,0.000000,0.000000}%
\pgfsetfillcolor{currentfill}%
\pgfsetlinewidth{0.803000pt}%
\definecolor{currentstroke}{rgb}{0.000000,0.000000,0.000000}%
\pgfsetstrokecolor{currentstroke}%
\pgfsetdash{}{0pt}%
\pgfsys@defobject{currentmarker}{\pgfqpoint{0.000000in}{0.000000in}}{\pgfqpoint{0.048611in}{0.000000in}}{%
\pgfpathmoveto{\pgfqpoint{0.000000in}{0.000000in}}%
\pgfpathlineto{\pgfqpoint{0.048611in}{0.000000in}}%
\pgfusepath{stroke,fill}%
}%
\begin{pgfscope}%
\pgfsys@transformshift{7.200000in}{1.529475in}%
\pgfsys@useobject{currentmarker}{}%
\end{pgfscope}%
\end{pgfscope}%
\begin{pgfscope}%
\definecolor{textcolor}{rgb}{0.000000,0.000000,0.000000}%
\pgfsetstrokecolor{textcolor}%
\pgfsetfillcolor{textcolor}%
\pgftext[x=0.725308in, y=1.481250in, left, base]{\color{textcolor}\rmfamily\fontsize{10.000000}{12.000000}\selectfont \(\displaystyle {0.0}\)}%
\end{pgfscope}%
\begin{pgfscope}%
\pgfpathrectangle{\pgfqpoint{1.000000in}{0.330000in}}{\pgfqpoint{6.200000in}{2.310000in}}%
\pgfusepath{clip}%
\pgfsetbuttcap%
\pgfsetroundjoin%
\pgfsetlinewidth{0.803000pt}%
\definecolor{currentstroke}{rgb}{0.000000,0.000000,0.000000}%
\pgfsetstrokecolor{currentstroke}%
\pgfsetdash{{0.800000pt}{1.320000pt}}{0.000000pt}%
\pgfpathmoveto{\pgfqpoint{1.000000in}{1.903851in}}%
\pgfpathlineto{\pgfqpoint{7.200000in}{1.903851in}}%
\pgfusepath{stroke}%
\end{pgfscope}%
\begin{pgfscope}%
\pgfsetbuttcap%
\pgfsetroundjoin%
\definecolor{currentfill}{rgb}{0.000000,0.000000,0.000000}%
\pgfsetfillcolor{currentfill}%
\pgfsetlinewidth{0.803000pt}%
\definecolor{currentstroke}{rgb}{0.000000,0.000000,0.000000}%
\pgfsetstrokecolor{currentstroke}%
\pgfsetdash{}{0pt}%
\pgfsys@defobject{currentmarker}{\pgfqpoint{-0.048611in}{0.000000in}}{\pgfqpoint{-0.000000in}{0.000000in}}{%
\pgfpathmoveto{\pgfqpoint{-0.000000in}{0.000000in}}%
\pgfpathlineto{\pgfqpoint{-0.048611in}{0.000000in}}%
\pgfusepath{stroke,fill}%
}%
\begin{pgfscope}%
\pgfsys@transformshift{1.000000in}{1.903851in}%
\pgfsys@useobject{currentmarker}{}%
\end{pgfscope}%
\end{pgfscope}%
\begin{pgfscope}%
\pgfsetbuttcap%
\pgfsetroundjoin%
\definecolor{currentfill}{rgb}{0.000000,0.000000,0.000000}%
\pgfsetfillcolor{currentfill}%
\pgfsetlinewidth{0.803000pt}%
\definecolor{currentstroke}{rgb}{0.000000,0.000000,0.000000}%
\pgfsetstrokecolor{currentstroke}%
\pgfsetdash{}{0pt}%
\pgfsys@defobject{currentmarker}{\pgfqpoint{0.000000in}{0.000000in}}{\pgfqpoint{0.048611in}{0.000000in}}{%
\pgfpathmoveto{\pgfqpoint{0.000000in}{0.000000in}}%
\pgfpathlineto{\pgfqpoint{0.048611in}{0.000000in}}%
\pgfusepath{stroke,fill}%
}%
\begin{pgfscope}%
\pgfsys@transformshift{7.200000in}{1.903851in}%
\pgfsys@useobject{currentmarker}{}%
\end{pgfscope}%
\end{pgfscope}%
\begin{pgfscope}%
\definecolor{textcolor}{rgb}{0.000000,0.000000,0.000000}%
\pgfsetstrokecolor{textcolor}%
\pgfsetfillcolor{textcolor}%
\pgftext[x=0.725308in, y=1.855626in, left, base]{\color{textcolor}\rmfamily\fontsize{10.000000}{12.000000}\selectfont \(\displaystyle {0.5}\)}%
\end{pgfscope}%
\begin{pgfscope}%
\pgfpathrectangle{\pgfqpoint{1.000000in}{0.330000in}}{\pgfqpoint{6.200000in}{2.310000in}}%
\pgfusepath{clip}%
\pgfsetbuttcap%
\pgfsetroundjoin%
\pgfsetlinewidth{0.803000pt}%
\definecolor{currentstroke}{rgb}{0.000000,0.000000,0.000000}%
\pgfsetstrokecolor{currentstroke}%
\pgfsetdash{{0.800000pt}{1.320000pt}}{0.000000pt}%
\pgfpathmoveto{\pgfqpoint{1.000000in}{2.278227in}}%
\pgfpathlineto{\pgfqpoint{7.200000in}{2.278227in}}%
\pgfusepath{stroke}%
\end{pgfscope}%
\begin{pgfscope}%
\pgfsetbuttcap%
\pgfsetroundjoin%
\definecolor{currentfill}{rgb}{0.000000,0.000000,0.000000}%
\pgfsetfillcolor{currentfill}%
\pgfsetlinewidth{0.803000pt}%
\definecolor{currentstroke}{rgb}{0.000000,0.000000,0.000000}%
\pgfsetstrokecolor{currentstroke}%
\pgfsetdash{}{0pt}%
\pgfsys@defobject{currentmarker}{\pgfqpoint{-0.048611in}{0.000000in}}{\pgfqpoint{-0.000000in}{0.000000in}}{%
\pgfpathmoveto{\pgfqpoint{-0.000000in}{0.000000in}}%
\pgfpathlineto{\pgfqpoint{-0.048611in}{0.000000in}}%
\pgfusepath{stroke,fill}%
}%
\begin{pgfscope}%
\pgfsys@transformshift{1.000000in}{2.278227in}%
\pgfsys@useobject{currentmarker}{}%
\end{pgfscope}%
\end{pgfscope}%
\begin{pgfscope}%
\pgfsetbuttcap%
\pgfsetroundjoin%
\definecolor{currentfill}{rgb}{0.000000,0.000000,0.000000}%
\pgfsetfillcolor{currentfill}%
\pgfsetlinewidth{0.803000pt}%
\definecolor{currentstroke}{rgb}{0.000000,0.000000,0.000000}%
\pgfsetstrokecolor{currentstroke}%
\pgfsetdash{}{0pt}%
\pgfsys@defobject{currentmarker}{\pgfqpoint{0.000000in}{0.000000in}}{\pgfqpoint{0.048611in}{0.000000in}}{%
\pgfpathmoveto{\pgfqpoint{0.000000in}{0.000000in}}%
\pgfpathlineto{\pgfqpoint{0.048611in}{0.000000in}}%
\pgfusepath{stroke,fill}%
}%
\begin{pgfscope}%
\pgfsys@transformshift{7.200000in}{2.278227in}%
\pgfsys@useobject{currentmarker}{}%
\end{pgfscope}%
\end{pgfscope}%
\begin{pgfscope}%
\definecolor{textcolor}{rgb}{0.000000,0.000000,0.000000}%
\pgfsetstrokecolor{textcolor}%
\pgfsetfillcolor{textcolor}%
\pgftext[x=0.725308in, y=2.230002in, left, base]{\color{textcolor}\rmfamily\fontsize{10.000000}{12.000000}\selectfont \(\displaystyle {1.0}\)}%
\end{pgfscope}%
\begin{pgfscope}%
\pgfpathrectangle{\pgfqpoint{1.000000in}{0.330000in}}{\pgfqpoint{6.200000in}{2.310000in}}%
\pgfusepath{clip}%
\pgfsetrectcap%
\pgfsetroundjoin%
\pgfsetlinewidth{1.505625pt}%
\definecolor{currentstroke}{rgb}{0.121569,0.466667,0.705882}%
\pgfsetstrokecolor{currentstroke}%
\pgfsetdash{}{0pt}%
\pgfpathmoveto{\pgfqpoint{1.281818in}{1.529475in}}%
\pgfpathlineto{\pgfqpoint{1.281818in}{1.116913in}}%
\pgfpathlineto{\pgfqpoint{1.354079in}{1.110274in}}%
\pgfpathlineto{\pgfqpoint{1.426340in}{1.103635in}}%
\pgfpathlineto{\pgfqpoint{1.498601in}{1.096996in}}%
\pgfpathlineto{\pgfqpoint{1.570862in}{1.090357in}}%
\pgfpathlineto{\pgfqpoint{1.643124in}{1.083719in}}%
\pgfpathlineto{\pgfqpoint{1.715385in}{1.077080in}}%
\pgfpathlineto{\pgfqpoint{1.787646in}{1.070441in}}%
\pgfpathlineto{\pgfqpoint{1.859907in}{1.063802in}}%
\pgfpathlineto{\pgfqpoint{1.932168in}{1.057163in}}%
\pgfpathlineto{\pgfqpoint{2.004429in}{1.050524in}}%
\pgfpathlineto{\pgfqpoint{2.076690in}{1.043885in}}%
\pgfpathlineto{\pgfqpoint{2.148951in}{1.037246in}}%
\pgfpathlineto{\pgfqpoint{2.221212in}{1.030607in}}%
\pgfpathlineto{\pgfqpoint{2.293473in}{1.023968in}}%
\pgfpathlineto{\pgfqpoint{2.365734in}{1.017329in}}%
\pgfpathlineto{\pgfqpoint{2.437995in}{1.010690in}}%
\pgfpathlineto{\pgfqpoint{2.510256in}{1.004051in}}%
\pgfpathlineto{\pgfqpoint{2.582517in}{0.997412in}}%
\pgfpathlineto{\pgfqpoint{2.654779in}{0.990774in}}%
\pgfpathlineto{\pgfqpoint{2.727040in}{0.984135in}}%
\pgfpathlineto{\pgfqpoint{2.799301in}{0.977496in}}%
\pgfpathlineto{\pgfqpoint{2.871562in}{0.970857in}}%
\pgfpathlineto{\pgfqpoint{2.943823in}{0.964218in}}%
\pgfpathlineto{\pgfqpoint{3.016084in}{0.957579in}}%
\pgfpathlineto{\pgfqpoint{3.088345in}{0.950940in}}%
\pgfpathlineto{\pgfqpoint{3.160606in}{0.944301in}}%
\pgfpathlineto{\pgfqpoint{3.232867in}{0.937662in}}%
\pgfpathlineto{\pgfqpoint{3.305128in}{0.931023in}}%
\pgfpathlineto{\pgfqpoint{3.377389in}{0.924384in}}%
\pgfpathlineto{\pgfqpoint{3.449650in}{0.917745in}}%
\pgfpathlineto{\pgfqpoint{3.521911in}{0.911106in}}%
\pgfpathlineto{\pgfqpoint{3.594172in}{0.904467in}}%
\pgfpathlineto{\pgfqpoint{3.666434in}{0.897828in}}%
\pgfpathlineto{\pgfqpoint{3.738695in}{0.891190in}}%
\pgfpathlineto{\pgfqpoint{3.883217in}{2.122787in}}%
\pgfpathlineto{\pgfqpoint{3.955478in}{2.116148in}}%
\pgfpathlineto{\pgfqpoint{4.027739in}{2.109509in}}%
\pgfpathlineto{\pgfqpoint{4.100000in}{2.102870in}}%
\pgfpathlineto{\pgfqpoint{4.172261in}{2.096231in}}%
\pgfpathlineto{\pgfqpoint{4.244522in}{2.089592in}}%
\pgfpathlineto{\pgfqpoint{4.316783in}{2.082953in}}%
\pgfpathlineto{\pgfqpoint{4.389044in}{2.076314in}}%
\pgfpathlineto{\pgfqpoint{4.461305in}{2.069675in}}%
\pgfpathlineto{\pgfqpoint{4.533566in}{2.063036in}}%
\pgfpathlineto{\pgfqpoint{4.605828in}{2.056397in}}%
\pgfpathlineto{\pgfqpoint{4.678089in}{2.049759in}}%
\pgfpathlineto{\pgfqpoint{4.750350in}{2.043120in}}%
\pgfpathlineto{\pgfqpoint{4.822611in}{2.036481in}}%
\pgfpathlineto{\pgfqpoint{4.894872in}{2.029842in}}%
\pgfpathlineto{\pgfqpoint{4.967133in}{2.023203in}}%
\pgfpathlineto{\pgfqpoint{5.039394in}{2.016564in}}%
\pgfpathlineto{\pgfqpoint{5.111655in}{2.009925in}}%
\pgfpathlineto{\pgfqpoint{5.183916in}{2.003286in}}%
\pgfpathlineto{\pgfqpoint{5.256177in}{1.996647in}}%
\pgfpathlineto{\pgfqpoint{5.328438in}{1.990008in}}%
\pgfpathlineto{\pgfqpoint{5.400699in}{1.983369in}}%
\pgfpathlineto{\pgfqpoint{5.472960in}{1.976730in}}%
\pgfpathlineto{\pgfqpoint{5.545221in}{1.970091in}}%
\pgfpathlineto{\pgfqpoint{5.617483in}{1.963452in}}%
\pgfpathlineto{\pgfqpoint{5.689744in}{1.956813in}}%
\pgfpathlineto{\pgfqpoint{5.762005in}{1.950175in}}%
\pgfpathlineto{\pgfqpoint{5.834266in}{1.943536in}}%
\pgfpathlineto{\pgfqpoint{5.906527in}{1.936897in}}%
\pgfpathlineto{\pgfqpoint{5.978788in}{1.930258in}}%
\pgfpathlineto{\pgfqpoint{6.051049in}{1.923619in}}%
\pgfpathlineto{\pgfqpoint{6.123310in}{1.916980in}}%
\pgfpathlineto{\pgfqpoint{6.195571in}{1.910341in}}%
\pgfpathlineto{\pgfqpoint{6.267832in}{1.903702in}}%
\pgfpathlineto{\pgfqpoint{6.340093in}{1.897063in}}%
\pgfpathlineto{\pgfqpoint{6.412354in}{1.890424in}}%
\pgfpathlineto{\pgfqpoint{6.484615in}{1.883785in}}%
\pgfpathlineto{\pgfqpoint{6.556876in}{1.877146in}}%
\pgfpathlineto{\pgfqpoint{6.629138in}{1.870507in}}%
\pgfpathlineto{\pgfqpoint{6.701399in}{1.863868in}}%
\pgfpathlineto{\pgfqpoint{6.773660in}{1.857229in}}%
\pgfpathlineto{\pgfqpoint{6.918182in}{1.843952in}}%
\pgfpathlineto{\pgfqpoint{6.918182in}{1.529475in}}%
\pgfusepath{stroke}%
\end{pgfscope}%
\begin{pgfscope}%
\pgfpathrectangle{\pgfqpoint{1.000000in}{0.330000in}}{\pgfqpoint{6.200000in}{2.310000in}}%
\pgfusepath{clip}%
\pgfsetrectcap%
\pgfsetroundjoin%
\pgfsetlinewidth{1.505625pt}%
\definecolor{currentstroke}{rgb}{1.000000,0.498039,0.054902}%
\pgfsetstrokecolor{currentstroke}%
\pgfsetdash{}{0pt}%
\pgfpathmoveto{\pgfqpoint{1.281818in}{1.529475in}}%
\pgfpathlineto{\pgfqpoint{1.281818in}{1.179060in}}%
\pgfpathlineto{\pgfqpoint{1.354079in}{1.173369in}}%
\pgfpathlineto{\pgfqpoint{1.426340in}{1.167679in}}%
\pgfpathlineto{\pgfqpoint{1.498601in}{1.161988in}}%
\pgfpathlineto{\pgfqpoint{1.570862in}{1.156298in}}%
\pgfpathlineto{\pgfqpoint{1.643124in}{1.150607in}}%
\pgfpathlineto{\pgfqpoint{1.715385in}{1.144917in}}%
\pgfpathlineto{\pgfqpoint{1.787646in}{1.139226in}}%
\pgfpathlineto{\pgfqpoint{1.859907in}{1.133536in}}%
\pgfpathlineto{\pgfqpoint{1.932168in}{1.127845in}}%
\pgfpathlineto{\pgfqpoint{2.004429in}{1.122154in}}%
\pgfpathlineto{\pgfqpoint{2.076690in}{1.116464in}}%
\pgfpathlineto{\pgfqpoint{2.148951in}{1.110773in}}%
\pgfpathlineto{\pgfqpoint{2.221212in}{1.105083in}}%
\pgfpathlineto{\pgfqpoint{2.293473in}{1.099392in}}%
\pgfpathlineto{\pgfqpoint{2.365734in}{1.093702in}}%
\pgfpathlineto{\pgfqpoint{2.437995in}{1.088011in}}%
\pgfpathlineto{\pgfqpoint{2.510256in}{1.082321in}}%
\pgfpathlineto{\pgfqpoint{2.582517in}{1.076630in}}%
\pgfpathlineto{\pgfqpoint{2.654779in}{1.070940in}}%
\pgfpathlineto{\pgfqpoint{2.727040in}{1.065249in}}%
\pgfpathlineto{\pgfqpoint{2.799301in}{1.059559in}}%
\pgfpathlineto{\pgfqpoint{2.871562in}{1.053868in}}%
\pgfpathlineto{\pgfqpoint{2.943823in}{1.048178in}}%
\pgfpathlineto{\pgfqpoint{3.016084in}{1.042487in}}%
\pgfpathlineto{\pgfqpoint{3.088345in}{1.036797in}}%
\pgfpathlineto{\pgfqpoint{3.160606in}{1.031106in}}%
\pgfpathlineto{\pgfqpoint{3.232867in}{1.025416in}}%
\pgfpathlineto{\pgfqpoint{3.305128in}{1.019725in}}%
\pgfpathlineto{\pgfqpoint{3.377389in}{1.014035in}}%
\pgfpathlineto{\pgfqpoint{3.449650in}{1.008344in}}%
\pgfpathlineto{\pgfqpoint{3.521911in}{1.002654in}}%
\pgfpathlineto{\pgfqpoint{3.594172in}{0.996963in}}%
\pgfpathlineto{\pgfqpoint{3.666434in}{0.991273in}}%
\pgfpathlineto{\pgfqpoint{3.738695in}{0.985582in}}%
\pgfpathlineto{\pgfqpoint{3.883217in}{2.035278in}}%
\pgfpathlineto{\pgfqpoint{3.955478in}{2.029587in}}%
\pgfpathlineto{\pgfqpoint{4.027739in}{2.023896in}}%
\pgfpathlineto{\pgfqpoint{4.100000in}{2.018206in}}%
\pgfpathlineto{\pgfqpoint{4.172261in}{2.012515in}}%
\pgfpathlineto{\pgfqpoint{4.244522in}{2.006825in}}%
\pgfpathlineto{\pgfqpoint{4.316783in}{2.001134in}}%
\pgfpathlineto{\pgfqpoint{4.389044in}{1.995444in}}%
\pgfpathlineto{\pgfqpoint{4.461305in}{1.989753in}}%
\pgfpathlineto{\pgfqpoint{4.533566in}{1.984063in}}%
\pgfpathlineto{\pgfqpoint{4.605828in}{1.978372in}}%
\pgfpathlineto{\pgfqpoint{4.678089in}{1.972682in}}%
\pgfpathlineto{\pgfqpoint{4.750350in}{1.966991in}}%
\pgfpathlineto{\pgfqpoint{4.822611in}{1.961301in}}%
\pgfpathlineto{\pgfqpoint{4.894872in}{1.955610in}}%
\pgfpathlineto{\pgfqpoint{4.967133in}{1.949920in}}%
\pgfpathlineto{\pgfqpoint{5.039394in}{1.944229in}}%
\pgfpathlineto{\pgfqpoint{5.111655in}{1.938539in}}%
\pgfpathlineto{\pgfqpoint{5.183916in}{1.932848in}}%
\pgfpathlineto{\pgfqpoint{5.256177in}{1.927158in}}%
\pgfpathlineto{\pgfqpoint{5.328438in}{1.921467in}}%
\pgfpathlineto{\pgfqpoint{5.400699in}{1.915777in}}%
\pgfpathlineto{\pgfqpoint{5.472960in}{1.910086in}}%
\pgfpathlineto{\pgfqpoint{5.545221in}{1.904396in}}%
\pgfpathlineto{\pgfqpoint{5.617483in}{1.898705in}}%
\pgfpathlineto{\pgfqpoint{5.689744in}{1.893015in}}%
\pgfpathlineto{\pgfqpoint{5.762005in}{1.887324in}}%
\pgfpathlineto{\pgfqpoint{5.834266in}{1.881634in}}%
\pgfpathlineto{\pgfqpoint{5.906527in}{1.875943in}}%
\pgfpathlineto{\pgfqpoint{5.978788in}{1.870253in}}%
\pgfpathlineto{\pgfqpoint{6.051049in}{1.864562in}}%
\pgfpathlineto{\pgfqpoint{6.123310in}{1.858872in}}%
\pgfpathlineto{\pgfqpoint{6.195571in}{1.853181in}}%
\pgfpathlineto{\pgfqpoint{6.267832in}{1.847491in}}%
\pgfpathlineto{\pgfqpoint{6.340093in}{1.841800in}}%
\pgfpathlineto{\pgfqpoint{6.412354in}{1.836110in}}%
\pgfpathlineto{\pgfqpoint{6.484615in}{1.830419in}}%
\pgfpathlineto{\pgfqpoint{6.556876in}{1.824728in}}%
\pgfpathlineto{\pgfqpoint{6.629138in}{1.819038in}}%
\pgfpathlineto{\pgfqpoint{6.701399in}{1.813347in}}%
\pgfpathlineto{\pgfqpoint{6.773660in}{1.807657in}}%
\pgfpathlineto{\pgfqpoint{6.918182in}{1.796276in}}%
\pgfpathlineto{\pgfqpoint{6.918182in}{1.529475in}}%
\pgfusepath{stroke}%
\end{pgfscope}%
\begin{pgfscope}%
\pgfpathrectangle{\pgfqpoint{1.000000in}{0.330000in}}{\pgfqpoint{6.200000in}{2.310000in}}%
\pgfusepath{clip}%
\pgfsetrectcap%
\pgfsetroundjoin%
\pgfsetlinewidth{1.505625pt}%
\definecolor{currentstroke}{rgb}{0.172549,0.627451,0.172549}%
\pgfsetstrokecolor{currentstroke}%
\pgfsetdash{}{0pt}%
\pgfpathmoveto{\pgfqpoint{1.281818in}{1.529475in}}%
\pgfpathlineto{\pgfqpoint{1.281818in}{1.116913in}}%
\pgfpathlineto{\pgfqpoint{1.354079in}{1.110274in}}%
\pgfpathlineto{\pgfqpoint{1.426340in}{1.103635in}}%
\pgfpathlineto{\pgfqpoint{1.498601in}{1.096996in}}%
\pgfpathlineto{\pgfqpoint{1.570862in}{1.090357in}}%
\pgfpathlineto{\pgfqpoint{1.643124in}{1.083719in}}%
\pgfpathlineto{\pgfqpoint{1.715385in}{1.077080in}}%
\pgfpathlineto{\pgfqpoint{1.787646in}{1.070441in}}%
\pgfpathlineto{\pgfqpoint{1.859907in}{1.063802in}}%
\pgfpathlineto{\pgfqpoint{1.932168in}{1.057163in}}%
\pgfpathlineto{\pgfqpoint{2.004429in}{1.050524in}}%
\pgfpathlineto{\pgfqpoint{2.076690in}{1.043885in}}%
\pgfpathlineto{\pgfqpoint{2.148951in}{1.037246in}}%
\pgfpathlineto{\pgfqpoint{2.221212in}{1.030607in}}%
\pgfpathlineto{\pgfqpoint{2.293473in}{1.023968in}}%
\pgfpathlineto{\pgfqpoint{2.365734in}{1.017329in}}%
\pgfpathlineto{\pgfqpoint{2.437995in}{1.010690in}}%
\pgfpathlineto{\pgfqpoint{2.510256in}{1.004051in}}%
\pgfpathlineto{\pgfqpoint{2.582517in}{0.997412in}}%
\pgfpathlineto{\pgfqpoint{2.654779in}{0.990774in}}%
\pgfpathlineto{\pgfqpoint{2.727040in}{0.984135in}}%
\pgfpathlineto{\pgfqpoint{2.799301in}{0.977496in}}%
\pgfpathlineto{\pgfqpoint{2.871562in}{0.970857in}}%
\pgfpathlineto{\pgfqpoint{2.943823in}{0.964218in}}%
\pgfpathlineto{\pgfqpoint{3.016084in}{0.957579in}}%
\pgfpathlineto{\pgfqpoint{3.088345in}{0.950940in}}%
\pgfpathlineto{\pgfqpoint{3.160606in}{0.944301in}}%
\pgfpathlineto{\pgfqpoint{3.232867in}{0.937662in}}%
\pgfpathlineto{\pgfqpoint{3.305128in}{0.931023in}}%
\pgfpathlineto{\pgfqpoint{3.377389in}{0.924384in}}%
\pgfpathlineto{\pgfqpoint{3.449650in}{0.917745in}}%
\pgfpathlineto{\pgfqpoint{3.521911in}{0.911106in}}%
\pgfpathlineto{\pgfqpoint{3.594172in}{0.904467in}}%
\pgfpathlineto{\pgfqpoint{3.666434in}{0.897828in}}%
\pgfpathlineto{\pgfqpoint{3.738695in}{0.891190in}}%
\pgfpathlineto{\pgfqpoint{3.883217in}{2.122787in}}%
\pgfpathlineto{\pgfqpoint{3.955478in}{2.116148in}}%
\pgfpathlineto{\pgfqpoint{4.027739in}{2.109509in}}%
\pgfpathlineto{\pgfqpoint{4.100000in}{2.102870in}}%
\pgfpathlineto{\pgfqpoint{4.172261in}{2.096231in}}%
\pgfpathlineto{\pgfqpoint{4.244522in}{2.089592in}}%
\pgfpathlineto{\pgfqpoint{4.316783in}{2.082953in}}%
\pgfpathlineto{\pgfqpoint{4.389044in}{2.076314in}}%
\pgfpathlineto{\pgfqpoint{4.461305in}{2.069675in}}%
\pgfpathlineto{\pgfqpoint{4.533566in}{2.063036in}}%
\pgfpathlineto{\pgfqpoint{4.605828in}{2.056397in}}%
\pgfpathlineto{\pgfqpoint{4.678089in}{2.049759in}}%
\pgfpathlineto{\pgfqpoint{4.750350in}{2.043120in}}%
\pgfpathlineto{\pgfqpoint{4.822611in}{2.036481in}}%
\pgfpathlineto{\pgfqpoint{4.894872in}{2.029842in}}%
\pgfpathlineto{\pgfqpoint{4.967133in}{2.023203in}}%
\pgfpathlineto{\pgfqpoint{5.039394in}{2.016564in}}%
\pgfpathlineto{\pgfqpoint{5.111655in}{2.009925in}}%
\pgfpathlineto{\pgfqpoint{5.183916in}{2.003286in}}%
\pgfpathlineto{\pgfqpoint{5.256177in}{1.996647in}}%
\pgfpathlineto{\pgfqpoint{5.328438in}{1.990008in}}%
\pgfpathlineto{\pgfqpoint{5.400699in}{1.983369in}}%
\pgfpathlineto{\pgfqpoint{5.472960in}{1.976730in}}%
\pgfpathlineto{\pgfqpoint{5.545221in}{1.970091in}}%
\pgfpathlineto{\pgfqpoint{5.617483in}{1.963452in}}%
\pgfpathlineto{\pgfqpoint{5.689744in}{1.956813in}}%
\pgfpathlineto{\pgfqpoint{5.762005in}{1.950175in}}%
\pgfpathlineto{\pgfqpoint{5.834266in}{1.943536in}}%
\pgfpathlineto{\pgfqpoint{5.906527in}{1.936897in}}%
\pgfpathlineto{\pgfqpoint{5.978788in}{1.930258in}}%
\pgfpathlineto{\pgfqpoint{6.051049in}{1.923619in}}%
\pgfpathlineto{\pgfqpoint{6.123310in}{1.916980in}}%
\pgfpathlineto{\pgfqpoint{6.195571in}{1.910341in}}%
\pgfpathlineto{\pgfqpoint{6.267832in}{1.903702in}}%
\pgfpathlineto{\pgfqpoint{6.340093in}{1.897063in}}%
\pgfpathlineto{\pgfqpoint{6.412354in}{1.890424in}}%
\pgfpathlineto{\pgfqpoint{6.484615in}{1.883785in}}%
\pgfpathlineto{\pgfqpoint{6.556876in}{1.877146in}}%
\pgfpathlineto{\pgfqpoint{6.629138in}{1.870507in}}%
\pgfpathlineto{\pgfqpoint{6.701399in}{1.863868in}}%
\pgfpathlineto{\pgfqpoint{6.773660in}{1.857229in}}%
\pgfpathlineto{\pgfqpoint{6.918182in}{1.843952in}}%
\pgfpathlineto{\pgfqpoint{6.918182in}{1.529475in}}%
\pgfusepath{stroke}%
\end{pgfscope}%
\begin{pgfscope}%
\pgfpathrectangle{\pgfqpoint{1.000000in}{0.330000in}}{\pgfqpoint{6.200000in}{2.310000in}}%
\pgfusepath{clip}%
\pgfsetrectcap%
\pgfsetroundjoin%
\pgfsetlinewidth{1.505625pt}%
\definecolor{currentstroke}{rgb}{0.839216,0.152941,0.156863}%
\pgfsetstrokecolor{currentstroke}%
\pgfsetdash{}{0pt}%
\pgfpathmoveto{\pgfqpoint{1.281818in}{1.529475in}}%
\pgfpathlineto{\pgfqpoint{1.281818in}{1.167080in}}%
\pgfpathlineto{\pgfqpoint{1.354079in}{1.161389in}}%
\pgfpathlineto{\pgfqpoint{1.426340in}{1.155699in}}%
\pgfpathlineto{\pgfqpoint{1.498601in}{1.150008in}}%
\pgfpathlineto{\pgfqpoint{1.570862in}{1.144318in}}%
\pgfpathlineto{\pgfqpoint{1.643124in}{1.138627in}}%
\pgfpathlineto{\pgfqpoint{1.715385in}{1.132937in}}%
\pgfpathlineto{\pgfqpoint{1.787646in}{1.127246in}}%
\pgfpathlineto{\pgfqpoint{1.859907in}{1.121555in}}%
\pgfpathlineto{\pgfqpoint{1.932168in}{1.115865in}}%
\pgfpathlineto{\pgfqpoint{2.004429in}{1.110174in}}%
\pgfpathlineto{\pgfqpoint{2.076690in}{1.104484in}}%
\pgfpathlineto{\pgfqpoint{2.148951in}{1.098793in}}%
\pgfpathlineto{\pgfqpoint{2.221212in}{1.093103in}}%
\pgfpathlineto{\pgfqpoint{2.293473in}{1.087412in}}%
\pgfpathlineto{\pgfqpoint{2.365734in}{1.081722in}}%
\pgfpathlineto{\pgfqpoint{2.437995in}{1.076031in}}%
\pgfpathlineto{\pgfqpoint{2.510256in}{1.070341in}}%
\pgfpathlineto{\pgfqpoint{2.582517in}{1.064650in}}%
\pgfpathlineto{\pgfqpoint{2.654779in}{1.058960in}}%
\pgfpathlineto{\pgfqpoint{2.727040in}{1.053269in}}%
\pgfpathlineto{\pgfqpoint{2.799301in}{1.047579in}}%
\pgfpathlineto{\pgfqpoint{2.871562in}{1.041888in}}%
\pgfpathlineto{\pgfqpoint{2.943823in}{1.036198in}}%
\pgfpathlineto{\pgfqpoint{3.016084in}{1.030507in}}%
\pgfpathlineto{\pgfqpoint{3.088345in}{1.024817in}}%
\pgfpathlineto{\pgfqpoint{3.160606in}{1.019126in}}%
\pgfpathlineto{\pgfqpoint{3.232867in}{1.013436in}}%
\pgfpathlineto{\pgfqpoint{3.305128in}{1.007745in}}%
\pgfpathlineto{\pgfqpoint{3.377389in}{1.002055in}}%
\pgfpathlineto{\pgfqpoint{3.449650in}{0.996364in}}%
\pgfpathlineto{\pgfqpoint{3.521911in}{0.990674in}}%
\pgfpathlineto{\pgfqpoint{3.594172in}{0.984983in}}%
\pgfpathlineto{\pgfqpoint{3.666434in}{0.979293in}}%
\pgfpathlineto{\pgfqpoint{3.738695in}{0.973602in}}%
\pgfpathlineto{\pgfqpoint{3.883217in}{2.045546in}}%
\pgfpathlineto{\pgfqpoint{3.955478in}{2.039856in}}%
\pgfpathlineto{\pgfqpoint{4.027739in}{2.034165in}}%
\pgfpathlineto{\pgfqpoint{4.100000in}{2.028475in}}%
\pgfpathlineto{\pgfqpoint{4.172261in}{2.022784in}}%
\pgfpathlineto{\pgfqpoint{4.244522in}{2.017094in}}%
\pgfpathlineto{\pgfqpoint{4.316783in}{2.011403in}}%
\pgfpathlineto{\pgfqpoint{4.389044in}{2.005713in}}%
\pgfpathlineto{\pgfqpoint{4.461305in}{2.000022in}}%
\pgfpathlineto{\pgfqpoint{4.533566in}{1.994331in}}%
\pgfpathlineto{\pgfqpoint{4.605828in}{1.988641in}}%
\pgfpathlineto{\pgfqpoint{4.678089in}{1.982950in}}%
\pgfpathlineto{\pgfqpoint{4.750350in}{1.977260in}}%
\pgfpathlineto{\pgfqpoint{4.822611in}{1.971569in}}%
\pgfpathlineto{\pgfqpoint{4.894872in}{1.965879in}}%
\pgfpathlineto{\pgfqpoint{4.967133in}{1.960188in}}%
\pgfpathlineto{\pgfqpoint{5.039394in}{1.954498in}}%
\pgfpathlineto{\pgfqpoint{5.111655in}{1.948807in}}%
\pgfpathlineto{\pgfqpoint{5.183916in}{1.943117in}}%
\pgfpathlineto{\pgfqpoint{5.256177in}{1.937426in}}%
\pgfpathlineto{\pgfqpoint{5.328438in}{1.931736in}}%
\pgfpathlineto{\pgfqpoint{5.400699in}{1.926045in}}%
\pgfpathlineto{\pgfqpoint{5.472960in}{1.920355in}}%
\pgfpathlineto{\pgfqpoint{5.545221in}{1.914664in}}%
\pgfpathlineto{\pgfqpoint{5.617483in}{1.908974in}}%
\pgfpathlineto{\pgfqpoint{5.689744in}{1.903283in}}%
\pgfpathlineto{\pgfqpoint{5.762005in}{1.897593in}}%
\pgfpathlineto{\pgfqpoint{5.834266in}{1.891902in}}%
\pgfpathlineto{\pgfqpoint{5.906527in}{1.886212in}}%
\pgfpathlineto{\pgfqpoint{5.978788in}{1.880521in}}%
\pgfpathlineto{\pgfqpoint{6.051049in}{1.874831in}}%
\pgfpathlineto{\pgfqpoint{6.123310in}{1.869140in}}%
\pgfpathlineto{\pgfqpoint{6.195571in}{1.863450in}}%
\pgfpathlineto{\pgfqpoint{6.267832in}{1.857759in}}%
\pgfpathlineto{\pgfqpoint{6.340093in}{1.852069in}}%
\pgfpathlineto{\pgfqpoint{6.412354in}{1.846378in}}%
\pgfpathlineto{\pgfqpoint{6.484615in}{1.840688in}}%
\pgfpathlineto{\pgfqpoint{6.556876in}{1.834997in}}%
\pgfpathlineto{\pgfqpoint{6.629138in}{1.829307in}}%
\pgfpathlineto{\pgfqpoint{6.701399in}{1.823616in}}%
\pgfpathlineto{\pgfqpoint{6.773660in}{1.817926in}}%
\pgfpathlineto{\pgfqpoint{6.918182in}{1.806545in}}%
\pgfpathlineto{\pgfqpoint{6.918182in}{1.529475in}}%
\pgfusepath{stroke}%
\end{pgfscope}%
\begin{pgfscope}%
\pgfpathrectangle{\pgfqpoint{1.000000in}{0.330000in}}{\pgfqpoint{6.200000in}{2.310000in}}%
\pgfusepath{clip}%
\pgfsetrectcap%
\pgfsetroundjoin%
\pgfsetlinewidth{1.505625pt}%
\definecolor{currentstroke}{rgb}{0.580392,0.403922,0.741176}%
\pgfsetstrokecolor{currentstroke}%
\pgfsetdash{}{0pt}%
\pgfpathmoveto{\pgfqpoint{1.281818in}{1.529475in}}%
\pgfpathlineto{\pgfqpoint{1.281818in}{1.325066in}}%
\pgfpathlineto{\pgfqpoint{1.354079in}{1.321747in}}%
\pgfpathlineto{\pgfqpoint{1.426340in}{1.318427in}}%
\pgfpathlineto{\pgfqpoint{1.498601in}{1.315108in}}%
\pgfpathlineto{\pgfqpoint{1.570862in}{1.311788in}}%
\pgfpathlineto{\pgfqpoint{1.643124in}{1.308469in}}%
\pgfpathlineto{\pgfqpoint{1.715385in}{1.305149in}}%
\pgfpathlineto{\pgfqpoint{1.787646in}{1.301830in}}%
\pgfpathlineto{\pgfqpoint{1.859907in}{1.298510in}}%
\pgfpathlineto{\pgfqpoint{1.932168in}{1.295191in}}%
\pgfpathlineto{\pgfqpoint{2.004429in}{1.291872in}}%
\pgfpathlineto{\pgfqpoint{2.076690in}{1.288552in}}%
\pgfpathlineto{\pgfqpoint{2.148951in}{1.285233in}}%
\pgfpathlineto{\pgfqpoint{2.221212in}{1.281913in}}%
\pgfpathlineto{\pgfqpoint{2.293473in}{1.278594in}}%
\pgfpathlineto{\pgfqpoint{2.365734in}{1.275274in}}%
\pgfpathlineto{\pgfqpoint{2.437995in}{1.271955in}}%
\pgfpathlineto{\pgfqpoint{2.510256in}{1.268635in}}%
\pgfpathlineto{\pgfqpoint{2.582517in}{1.265316in}}%
\pgfpathlineto{\pgfqpoint{2.654779in}{1.261996in}}%
\pgfpathlineto{\pgfqpoint{2.727040in}{1.258677in}}%
\pgfpathlineto{\pgfqpoint{2.799301in}{1.255357in}}%
\pgfpathlineto{\pgfqpoint{2.871562in}{1.252038in}}%
\pgfpathlineto{\pgfqpoint{2.943823in}{1.248718in}}%
\pgfpathlineto{\pgfqpoint{3.016084in}{1.245399in}}%
\pgfpathlineto{\pgfqpoint{3.088345in}{1.242080in}}%
\pgfpathlineto{\pgfqpoint{3.160606in}{1.238760in}}%
\pgfpathlineto{\pgfqpoint{3.232867in}{1.235441in}}%
\pgfpathlineto{\pgfqpoint{3.305128in}{1.232121in}}%
\pgfpathlineto{\pgfqpoint{3.377389in}{1.228802in}}%
\pgfpathlineto{\pgfqpoint{3.449650in}{1.225482in}}%
\pgfpathlineto{\pgfqpoint{3.521911in}{1.222163in}}%
\pgfpathlineto{\pgfqpoint{3.594172in}{1.218843in}}%
\pgfpathlineto{\pgfqpoint{3.666434in}{1.215524in}}%
\pgfpathlineto{\pgfqpoint{3.738695in}{1.212204in}}%
\pgfpathlineto{\pgfqpoint{3.883217in}{1.824527in}}%
\pgfpathlineto{\pgfqpoint{3.955478in}{1.821207in}}%
\pgfpathlineto{\pgfqpoint{4.027739in}{1.817888in}}%
\pgfpathlineto{\pgfqpoint{4.100000in}{1.814568in}}%
\pgfpathlineto{\pgfqpoint{4.172261in}{1.811249in}}%
\pgfpathlineto{\pgfqpoint{4.244522in}{1.807929in}}%
\pgfpathlineto{\pgfqpoint{4.316783in}{1.804610in}}%
\pgfpathlineto{\pgfqpoint{4.389044in}{1.801290in}}%
\pgfpathlineto{\pgfqpoint{4.461305in}{1.797971in}}%
\pgfpathlineto{\pgfqpoint{4.533566in}{1.794651in}}%
\pgfpathlineto{\pgfqpoint{4.605828in}{1.791332in}}%
\pgfpathlineto{\pgfqpoint{4.678089in}{1.788013in}}%
\pgfpathlineto{\pgfqpoint{4.750350in}{1.784693in}}%
\pgfpathlineto{\pgfqpoint{4.822611in}{1.781374in}}%
\pgfpathlineto{\pgfqpoint{4.894872in}{1.778054in}}%
\pgfpathlineto{\pgfqpoint{4.967133in}{1.774735in}}%
\pgfpathlineto{\pgfqpoint{5.039394in}{1.771415in}}%
\pgfpathlineto{\pgfqpoint{5.111655in}{1.768096in}}%
\pgfpathlineto{\pgfqpoint{5.183916in}{1.764776in}}%
\pgfpathlineto{\pgfqpoint{5.256177in}{1.761457in}}%
\pgfpathlineto{\pgfqpoint{5.328438in}{1.758137in}}%
\pgfpathlineto{\pgfqpoint{5.400699in}{1.754818in}}%
\pgfpathlineto{\pgfqpoint{5.472960in}{1.751498in}}%
\pgfpathlineto{\pgfqpoint{5.545221in}{1.748179in}}%
\pgfpathlineto{\pgfqpoint{5.617483in}{1.744859in}}%
\pgfpathlineto{\pgfqpoint{5.689744in}{1.741540in}}%
\pgfpathlineto{\pgfqpoint{5.762005in}{1.738221in}}%
\pgfpathlineto{\pgfqpoint{5.834266in}{1.734901in}}%
\pgfpathlineto{\pgfqpoint{5.906527in}{1.731582in}}%
\pgfpathlineto{\pgfqpoint{5.978788in}{1.728262in}}%
\pgfpathlineto{\pgfqpoint{6.051049in}{1.724943in}}%
\pgfpathlineto{\pgfqpoint{6.123310in}{1.721623in}}%
\pgfpathlineto{\pgfqpoint{6.195571in}{1.718304in}}%
\pgfpathlineto{\pgfqpoint{6.267832in}{1.714984in}}%
\pgfpathlineto{\pgfqpoint{6.340093in}{1.711665in}}%
\pgfpathlineto{\pgfqpoint{6.412354in}{1.708345in}}%
\pgfpathlineto{\pgfqpoint{6.484615in}{1.705026in}}%
\pgfpathlineto{\pgfqpoint{6.556876in}{1.701706in}}%
\pgfpathlineto{\pgfqpoint{6.629138in}{1.698387in}}%
\pgfpathlineto{\pgfqpoint{6.701399in}{1.695067in}}%
\pgfpathlineto{\pgfqpoint{6.773660in}{1.691748in}}%
\pgfpathlineto{\pgfqpoint{6.918182in}{1.685109in}}%
\pgfpathlineto{\pgfqpoint{6.918182in}{1.529475in}}%
\pgfusepath{stroke}%
\end{pgfscope}%
\begin{pgfscope}%
\pgfpathrectangle{\pgfqpoint{1.000000in}{0.330000in}}{\pgfqpoint{6.200000in}{2.310000in}}%
\pgfusepath{clip}%
\pgfsetrectcap%
\pgfsetroundjoin%
\pgfsetlinewidth{1.505625pt}%
\definecolor{currentstroke}{rgb}{0.549020,0.337255,0.294118}%
\pgfsetstrokecolor{currentstroke}%
\pgfsetdash{}{0pt}%
\pgfpathmoveto{\pgfqpoint{1.281818in}{1.529475in}}%
\pgfpathlineto{\pgfqpoint{1.281818in}{1.171572in}}%
\pgfpathlineto{\pgfqpoint{1.354079in}{1.165882in}}%
\pgfpathlineto{\pgfqpoint{1.426340in}{1.160191in}}%
\pgfpathlineto{\pgfqpoint{1.498601in}{1.154501in}}%
\pgfpathlineto{\pgfqpoint{1.570862in}{1.148810in}}%
\pgfpathlineto{\pgfqpoint{1.643124in}{1.143120in}}%
\pgfpathlineto{\pgfqpoint{1.715385in}{1.137429in}}%
\pgfpathlineto{\pgfqpoint{1.787646in}{1.131739in}}%
\pgfpathlineto{\pgfqpoint{1.859907in}{1.126048in}}%
\pgfpathlineto{\pgfqpoint{1.932168in}{1.120357in}}%
\pgfpathlineto{\pgfqpoint{2.004429in}{1.114667in}}%
\pgfpathlineto{\pgfqpoint{2.076690in}{1.108976in}}%
\pgfpathlineto{\pgfqpoint{2.148951in}{1.103286in}}%
\pgfpathlineto{\pgfqpoint{2.221212in}{1.097595in}}%
\pgfpathlineto{\pgfqpoint{2.293473in}{1.091905in}}%
\pgfpathlineto{\pgfqpoint{2.365734in}{1.086214in}}%
\pgfpathlineto{\pgfqpoint{2.437995in}{1.080524in}}%
\pgfpathlineto{\pgfqpoint{2.510256in}{1.074833in}}%
\pgfpathlineto{\pgfqpoint{2.582517in}{1.069143in}}%
\pgfpathlineto{\pgfqpoint{2.654779in}{1.063452in}}%
\pgfpathlineto{\pgfqpoint{2.727040in}{1.057762in}}%
\pgfpathlineto{\pgfqpoint{2.799301in}{1.052071in}}%
\pgfpathlineto{\pgfqpoint{2.871562in}{1.046381in}}%
\pgfpathlineto{\pgfqpoint{2.943823in}{1.040690in}}%
\pgfpathlineto{\pgfqpoint{3.016084in}{1.035000in}}%
\pgfpathlineto{\pgfqpoint{3.088345in}{1.029309in}}%
\pgfpathlineto{\pgfqpoint{3.160606in}{1.023619in}}%
\pgfpathlineto{\pgfqpoint{3.232867in}{1.017928in}}%
\pgfpathlineto{\pgfqpoint{3.305128in}{1.012238in}}%
\pgfpathlineto{\pgfqpoint{3.377389in}{1.006547in}}%
\pgfpathlineto{\pgfqpoint{3.449650in}{1.000857in}}%
\pgfpathlineto{\pgfqpoint{3.521911in}{0.995166in}}%
\pgfpathlineto{\pgfqpoint{3.594172in}{0.989476in}}%
\pgfpathlineto{\pgfqpoint{3.666434in}{0.983785in}}%
\pgfpathlineto{\pgfqpoint{3.738695in}{0.978095in}}%
\pgfpathlineto{\pgfqpoint{3.883217in}{2.041695in}}%
\pgfpathlineto{\pgfqpoint{3.955478in}{2.036005in}}%
\pgfpathlineto{\pgfqpoint{4.027739in}{2.030314in}}%
\pgfpathlineto{\pgfqpoint{4.100000in}{2.024624in}}%
\pgfpathlineto{\pgfqpoint{4.172261in}{2.018933in}}%
\pgfpathlineto{\pgfqpoint{4.244522in}{2.013243in}}%
\pgfpathlineto{\pgfqpoint{4.316783in}{2.007552in}}%
\pgfpathlineto{\pgfqpoint{4.389044in}{2.001862in}}%
\pgfpathlineto{\pgfqpoint{4.461305in}{1.996171in}}%
\pgfpathlineto{\pgfqpoint{4.533566in}{1.990481in}}%
\pgfpathlineto{\pgfqpoint{4.605828in}{1.984790in}}%
\pgfpathlineto{\pgfqpoint{4.678089in}{1.979100in}}%
\pgfpathlineto{\pgfqpoint{4.750350in}{1.973409in}}%
\pgfpathlineto{\pgfqpoint{4.822611in}{1.967719in}}%
\pgfpathlineto{\pgfqpoint{4.894872in}{1.962028in}}%
\pgfpathlineto{\pgfqpoint{4.967133in}{1.956338in}}%
\pgfpathlineto{\pgfqpoint{5.039394in}{1.950647in}}%
\pgfpathlineto{\pgfqpoint{5.111655in}{1.944957in}}%
\pgfpathlineto{\pgfqpoint{5.183916in}{1.939266in}}%
\pgfpathlineto{\pgfqpoint{5.256177in}{1.933576in}}%
\pgfpathlineto{\pgfqpoint{5.328438in}{1.927885in}}%
\pgfpathlineto{\pgfqpoint{5.400699in}{1.922195in}}%
\pgfpathlineto{\pgfqpoint{5.472960in}{1.916504in}}%
\pgfpathlineto{\pgfqpoint{5.545221in}{1.910814in}}%
\pgfpathlineto{\pgfqpoint{5.617483in}{1.905123in}}%
\pgfpathlineto{\pgfqpoint{5.689744in}{1.899433in}}%
\pgfpathlineto{\pgfqpoint{5.762005in}{1.893742in}}%
\pgfpathlineto{\pgfqpoint{5.834266in}{1.888052in}}%
\pgfpathlineto{\pgfqpoint{5.906527in}{1.882361in}}%
\pgfpathlineto{\pgfqpoint{5.978788in}{1.876670in}}%
\pgfpathlineto{\pgfqpoint{6.051049in}{1.870980in}}%
\pgfpathlineto{\pgfqpoint{6.123310in}{1.865289in}}%
\pgfpathlineto{\pgfqpoint{6.195571in}{1.859599in}}%
\pgfpathlineto{\pgfqpoint{6.267832in}{1.853908in}}%
\pgfpathlineto{\pgfqpoint{6.340093in}{1.848218in}}%
\pgfpathlineto{\pgfqpoint{6.412354in}{1.842527in}}%
\pgfpathlineto{\pgfqpoint{6.484615in}{1.836837in}}%
\pgfpathlineto{\pgfqpoint{6.556876in}{1.831146in}}%
\pgfpathlineto{\pgfqpoint{6.629138in}{1.825456in}}%
\pgfpathlineto{\pgfqpoint{6.701399in}{1.819765in}}%
\pgfpathlineto{\pgfqpoint{6.773660in}{1.814075in}}%
\pgfpathlineto{\pgfqpoint{6.918182in}{1.802694in}}%
\pgfpathlineto{\pgfqpoint{6.918182in}{1.529475in}}%
\pgfusepath{stroke}%
\end{pgfscope}%
\begin{pgfscope}%
\pgfpathrectangle{\pgfqpoint{1.000000in}{0.330000in}}{\pgfqpoint{6.200000in}{2.310000in}}%
\pgfusepath{clip}%
\pgfsetrectcap%
\pgfsetroundjoin%
\pgfsetlinewidth{1.505625pt}%
\definecolor{currentstroke}{rgb}{0.890196,0.466667,0.760784}%
\pgfsetstrokecolor{currentstroke}%
\pgfsetdash{}{0pt}%
\pgfpathmoveto{\pgfqpoint{1.281818in}{1.529475in}}%
\pgfpathlineto{\pgfqpoint{1.281818in}{1.116913in}}%
\pgfpathlineto{\pgfqpoint{1.354079in}{1.110274in}}%
\pgfpathlineto{\pgfqpoint{1.426340in}{1.103635in}}%
\pgfpathlineto{\pgfqpoint{1.498601in}{1.096996in}}%
\pgfpathlineto{\pgfqpoint{1.570862in}{1.090357in}}%
\pgfpathlineto{\pgfqpoint{1.643124in}{1.083719in}}%
\pgfpathlineto{\pgfqpoint{1.715385in}{1.077080in}}%
\pgfpathlineto{\pgfqpoint{1.787646in}{1.070441in}}%
\pgfpathlineto{\pgfqpoint{1.859907in}{1.063802in}}%
\pgfpathlineto{\pgfqpoint{1.932168in}{1.057163in}}%
\pgfpathlineto{\pgfqpoint{2.004429in}{1.050524in}}%
\pgfpathlineto{\pgfqpoint{2.076690in}{1.043885in}}%
\pgfpathlineto{\pgfqpoint{2.148951in}{1.037246in}}%
\pgfpathlineto{\pgfqpoint{2.221212in}{1.030607in}}%
\pgfpathlineto{\pgfqpoint{2.293473in}{1.023968in}}%
\pgfpathlineto{\pgfqpoint{2.365734in}{1.017329in}}%
\pgfpathlineto{\pgfqpoint{2.437995in}{1.010690in}}%
\pgfpathlineto{\pgfqpoint{2.510256in}{1.004051in}}%
\pgfpathlineto{\pgfqpoint{2.582517in}{0.997412in}}%
\pgfpathlineto{\pgfqpoint{2.654779in}{0.990774in}}%
\pgfpathlineto{\pgfqpoint{2.727040in}{0.984135in}}%
\pgfpathlineto{\pgfqpoint{2.799301in}{0.977496in}}%
\pgfpathlineto{\pgfqpoint{2.871562in}{0.970857in}}%
\pgfpathlineto{\pgfqpoint{2.943823in}{0.964218in}}%
\pgfpathlineto{\pgfqpoint{3.016084in}{0.957579in}}%
\pgfpathlineto{\pgfqpoint{3.088345in}{0.950940in}}%
\pgfpathlineto{\pgfqpoint{3.160606in}{0.944301in}}%
\pgfpathlineto{\pgfqpoint{3.232867in}{0.937662in}}%
\pgfpathlineto{\pgfqpoint{3.305128in}{0.931023in}}%
\pgfpathlineto{\pgfqpoint{3.377389in}{0.924384in}}%
\pgfpathlineto{\pgfqpoint{3.449650in}{0.917745in}}%
\pgfpathlineto{\pgfqpoint{3.521911in}{0.911106in}}%
\pgfpathlineto{\pgfqpoint{3.594172in}{0.904467in}}%
\pgfpathlineto{\pgfqpoint{3.666434in}{0.897828in}}%
\pgfpathlineto{\pgfqpoint{3.738695in}{0.891190in}}%
\pgfpathlineto{\pgfqpoint{3.883217in}{2.122787in}}%
\pgfpathlineto{\pgfqpoint{3.955478in}{2.116148in}}%
\pgfpathlineto{\pgfqpoint{4.027739in}{2.109509in}}%
\pgfpathlineto{\pgfqpoint{4.100000in}{2.102870in}}%
\pgfpathlineto{\pgfqpoint{4.172261in}{2.096231in}}%
\pgfpathlineto{\pgfqpoint{4.244522in}{2.089592in}}%
\pgfpathlineto{\pgfqpoint{4.316783in}{2.082953in}}%
\pgfpathlineto{\pgfqpoint{4.389044in}{2.076314in}}%
\pgfpathlineto{\pgfqpoint{4.461305in}{2.069675in}}%
\pgfpathlineto{\pgfqpoint{4.533566in}{2.063036in}}%
\pgfpathlineto{\pgfqpoint{4.605828in}{2.056397in}}%
\pgfpathlineto{\pgfqpoint{4.678089in}{2.049759in}}%
\pgfpathlineto{\pgfqpoint{4.750350in}{2.043120in}}%
\pgfpathlineto{\pgfqpoint{4.822611in}{2.036481in}}%
\pgfpathlineto{\pgfqpoint{4.894872in}{2.029842in}}%
\pgfpathlineto{\pgfqpoint{4.967133in}{2.023203in}}%
\pgfpathlineto{\pgfqpoint{5.039394in}{2.016564in}}%
\pgfpathlineto{\pgfqpoint{5.111655in}{2.009925in}}%
\pgfpathlineto{\pgfqpoint{5.183916in}{2.003286in}}%
\pgfpathlineto{\pgfqpoint{5.256177in}{1.996647in}}%
\pgfpathlineto{\pgfqpoint{5.328438in}{1.990008in}}%
\pgfpathlineto{\pgfqpoint{5.400699in}{1.983369in}}%
\pgfpathlineto{\pgfqpoint{5.472960in}{1.976730in}}%
\pgfpathlineto{\pgfqpoint{5.545221in}{1.970091in}}%
\pgfpathlineto{\pgfqpoint{5.617483in}{1.963452in}}%
\pgfpathlineto{\pgfqpoint{5.689744in}{1.956813in}}%
\pgfpathlineto{\pgfqpoint{5.762005in}{1.950175in}}%
\pgfpathlineto{\pgfqpoint{5.834266in}{1.943536in}}%
\pgfpathlineto{\pgfqpoint{5.906527in}{1.936897in}}%
\pgfpathlineto{\pgfqpoint{5.978788in}{1.930258in}}%
\pgfpathlineto{\pgfqpoint{6.051049in}{1.923619in}}%
\pgfpathlineto{\pgfqpoint{6.123310in}{1.916980in}}%
\pgfpathlineto{\pgfqpoint{6.195571in}{1.910341in}}%
\pgfpathlineto{\pgfqpoint{6.267832in}{1.903702in}}%
\pgfpathlineto{\pgfqpoint{6.340093in}{1.897063in}}%
\pgfpathlineto{\pgfqpoint{6.412354in}{1.890424in}}%
\pgfpathlineto{\pgfqpoint{6.484615in}{1.883785in}}%
\pgfpathlineto{\pgfqpoint{6.556876in}{1.877146in}}%
\pgfpathlineto{\pgfqpoint{6.629138in}{1.870507in}}%
\pgfpathlineto{\pgfqpoint{6.701399in}{1.863868in}}%
\pgfpathlineto{\pgfqpoint{6.773660in}{1.857229in}}%
\pgfpathlineto{\pgfqpoint{6.918182in}{1.843952in}}%
\pgfpathlineto{\pgfqpoint{6.918182in}{1.529475in}}%
\pgfusepath{stroke}%
\end{pgfscope}%
\begin{pgfscope}%
\pgfpathrectangle{\pgfqpoint{1.000000in}{0.330000in}}{\pgfqpoint{6.200000in}{2.310000in}}%
\pgfusepath{clip}%
\pgfsetrectcap%
\pgfsetroundjoin%
\pgfsetlinewidth{1.505625pt}%
\definecolor{currentstroke}{rgb}{0.498039,0.498039,0.498039}%
\pgfsetstrokecolor{currentstroke}%
\pgfsetdash{}{0pt}%
\pgfpathmoveto{\pgfqpoint{1.281818in}{1.529475in}}%
\pgfpathlineto{\pgfqpoint{1.281818in}{1.179060in}}%
\pgfpathlineto{\pgfqpoint{1.354079in}{1.173369in}}%
\pgfpathlineto{\pgfqpoint{1.426340in}{1.167679in}}%
\pgfpathlineto{\pgfqpoint{1.498601in}{1.161988in}}%
\pgfpathlineto{\pgfqpoint{1.570862in}{1.156298in}}%
\pgfpathlineto{\pgfqpoint{1.643124in}{1.150607in}}%
\pgfpathlineto{\pgfqpoint{1.715385in}{1.144917in}}%
\pgfpathlineto{\pgfqpoint{1.787646in}{1.139226in}}%
\pgfpathlineto{\pgfqpoint{1.859907in}{1.133536in}}%
\pgfpathlineto{\pgfqpoint{1.932168in}{1.127845in}}%
\pgfpathlineto{\pgfqpoint{2.004429in}{1.122154in}}%
\pgfpathlineto{\pgfqpoint{2.076690in}{1.116464in}}%
\pgfpathlineto{\pgfqpoint{2.148951in}{1.110773in}}%
\pgfpathlineto{\pgfqpoint{2.221212in}{1.105083in}}%
\pgfpathlineto{\pgfqpoint{2.293473in}{1.099392in}}%
\pgfpathlineto{\pgfqpoint{2.365734in}{1.093702in}}%
\pgfpathlineto{\pgfqpoint{2.437995in}{1.088011in}}%
\pgfpathlineto{\pgfqpoint{2.510256in}{1.082321in}}%
\pgfpathlineto{\pgfqpoint{2.582517in}{1.076630in}}%
\pgfpathlineto{\pgfqpoint{2.654779in}{1.070940in}}%
\pgfpathlineto{\pgfqpoint{2.727040in}{1.065249in}}%
\pgfpathlineto{\pgfqpoint{2.799301in}{1.059559in}}%
\pgfpathlineto{\pgfqpoint{2.871562in}{1.053868in}}%
\pgfpathlineto{\pgfqpoint{2.943823in}{1.048178in}}%
\pgfpathlineto{\pgfqpoint{3.016084in}{1.042487in}}%
\pgfpathlineto{\pgfqpoint{3.088345in}{1.036797in}}%
\pgfpathlineto{\pgfqpoint{3.160606in}{1.031106in}}%
\pgfpathlineto{\pgfqpoint{3.232867in}{1.025416in}}%
\pgfpathlineto{\pgfqpoint{3.305128in}{1.019725in}}%
\pgfpathlineto{\pgfqpoint{3.377389in}{1.014035in}}%
\pgfpathlineto{\pgfqpoint{3.449650in}{1.008344in}}%
\pgfpathlineto{\pgfqpoint{3.521911in}{1.002654in}}%
\pgfpathlineto{\pgfqpoint{3.594172in}{0.996963in}}%
\pgfpathlineto{\pgfqpoint{3.666434in}{0.991273in}}%
\pgfpathlineto{\pgfqpoint{3.738695in}{0.985582in}}%
\pgfpathlineto{\pgfqpoint{3.883217in}{2.035278in}}%
\pgfpathlineto{\pgfqpoint{3.955478in}{2.029587in}}%
\pgfpathlineto{\pgfqpoint{4.027739in}{2.023896in}}%
\pgfpathlineto{\pgfqpoint{4.100000in}{2.018206in}}%
\pgfpathlineto{\pgfqpoint{4.172261in}{2.012515in}}%
\pgfpathlineto{\pgfqpoint{4.244522in}{2.006825in}}%
\pgfpathlineto{\pgfqpoint{4.316783in}{2.001134in}}%
\pgfpathlineto{\pgfqpoint{4.389044in}{1.995444in}}%
\pgfpathlineto{\pgfqpoint{4.461305in}{1.989753in}}%
\pgfpathlineto{\pgfqpoint{4.533566in}{1.984063in}}%
\pgfpathlineto{\pgfqpoint{4.605828in}{1.978372in}}%
\pgfpathlineto{\pgfqpoint{4.678089in}{1.972682in}}%
\pgfpathlineto{\pgfqpoint{4.750350in}{1.966991in}}%
\pgfpathlineto{\pgfqpoint{4.822611in}{1.961301in}}%
\pgfpathlineto{\pgfqpoint{4.894872in}{1.955610in}}%
\pgfpathlineto{\pgfqpoint{4.967133in}{1.949920in}}%
\pgfpathlineto{\pgfqpoint{5.039394in}{1.944229in}}%
\pgfpathlineto{\pgfqpoint{5.111655in}{1.938539in}}%
\pgfpathlineto{\pgfqpoint{5.183916in}{1.932848in}}%
\pgfpathlineto{\pgfqpoint{5.256177in}{1.927158in}}%
\pgfpathlineto{\pgfqpoint{5.328438in}{1.921467in}}%
\pgfpathlineto{\pgfqpoint{5.400699in}{1.915777in}}%
\pgfpathlineto{\pgfqpoint{5.472960in}{1.910086in}}%
\pgfpathlineto{\pgfqpoint{5.545221in}{1.904396in}}%
\pgfpathlineto{\pgfqpoint{5.617483in}{1.898705in}}%
\pgfpathlineto{\pgfqpoint{5.689744in}{1.893015in}}%
\pgfpathlineto{\pgfqpoint{5.762005in}{1.887324in}}%
\pgfpathlineto{\pgfqpoint{5.834266in}{1.881634in}}%
\pgfpathlineto{\pgfqpoint{5.906527in}{1.875943in}}%
\pgfpathlineto{\pgfqpoint{5.978788in}{1.870253in}}%
\pgfpathlineto{\pgfqpoint{6.051049in}{1.864562in}}%
\pgfpathlineto{\pgfqpoint{6.123310in}{1.858872in}}%
\pgfpathlineto{\pgfqpoint{6.195571in}{1.853181in}}%
\pgfpathlineto{\pgfqpoint{6.267832in}{1.847491in}}%
\pgfpathlineto{\pgfqpoint{6.340093in}{1.841800in}}%
\pgfpathlineto{\pgfqpoint{6.412354in}{1.836110in}}%
\pgfpathlineto{\pgfqpoint{6.484615in}{1.830419in}}%
\pgfpathlineto{\pgfqpoint{6.556876in}{1.824728in}}%
\pgfpathlineto{\pgfqpoint{6.629138in}{1.819038in}}%
\pgfpathlineto{\pgfqpoint{6.701399in}{1.813347in}}%
\pgfpathlineto{\pgfqpoint{6.773660in}{1.807657in}}%
\pgfpathlineto{\pgfqpoint{6.918182in}{1.796276in}}%
\pgfpathlineto{\pgfqpoint{6.918182in}{1.529475in}}%
\pgfusepath{stroke}%
\end{pgfscope}%
\begin{pgfscope}%
\pgfpathrectangle{\pgfqpoint{1.000000in}{0.330000in}}{\pgfqpoint{6.200000in}{2.310000in}}%
\pgfusepath{clip}%
\pgfsetrectcap%
\pgfsetroundjoin%
\pgfsetlinewidth{1.505625pt}%
\definecolor{currentstroke}{rgb}{0.737255,0.741176,0.133333}%
\pgfsetstrokecolor{currentstroke}%
\pgfsetdash{}{0pt}%
\pgfpathmoveto{\pgfqpoint{1.281818in}{1.529475in}}%
\pgfpathlineto{\pgfqpoint{1.281818in}{0.771739in}}%
\pgfpathlineto{\pgfqpoint{1.354079in}{0.762055in}}%
\pgfpathlineto{\pgfqpoint{1.426340in}{0.752371in}}%
\pgfpathlineto{\pgfqpoint{1.498601in}{0.742687in}}%
\pgfpathlineto{\pgfqpoint{1.570862in}{0.733003in}}%
\pgfpathlineto{\pgfqpoint{1.643124in}{0.723319in}}%
\pgfpathlineto{\pgfqpoint{1.715385in}{0.713636in}}%
\pgfpathlineto{\pgfqpoint{1.787646in}{0.703952in}}%
\pgfpathlineto{\pgfqpoint{1.859907in}{0.694268in}}%
\pgfpathlineto{\pgfqpoint{1.932168in}{0.684584in}}%
\pgfpathlineto{\pgfqpoint{2.004429in}{0.674900in}}%
\pgfpathlineto{\pgfqpoint{2.076690in}{0.665216in}}%
\pgfpathlineto{\pgfqpoint{2.148951in}{0.655532in}}%
\pgfpathlineto{\pgfqpoint{2.221212in}{0.645849in}}%
\pgfpathlineto{\pgfqpoint{2.293473in}{0.636165in}}%
\pgfpathlineto{\pgfqpoint{2.365734in}{0.626481in}}%
\pgfpathlineto{\pgfqpoint{2.437995in}{0.616797in}}%
\pgfpathlineto{\pgfqpoint{2.510256in}{0.607113in}}%
\pgfpathlineto{\pgfqpoint{2.582517in}{0.597429in}}%
\pgfpathlineto{\pgfqpoint{2.654779in}{0.587745in}}%
\pgfpathlineto{\pgfqpoint{2.727040in}{0.578062in}}%
\pgfpathlineto{\pgfqpoint{2.799301in}{0.568378in}}%
\pgfpathlineto{\pgfqpoint{2.871562in}{0.558694in}}%
\pgfpathlineto{\pgfqpoint{2.943823in}{0.549010in}}%
\pgfpathlineto{\pgfqpoint{3.016084in}{0.539326in}}%
\pgfpathlineto{\pgfqpoint{3.088345in}{0.529642in}}%
\pgfpathlineto{\pgfqpoint{3.160606in}{0.519958in}}%
\pgfpathlineto{\pgfqpoint{3.232867in}{0.510275in}}%
\pgfpathlineto{\pgfqpoint{3.305128in}{0.500591in}}%
\pgfpathlineto{\pgfqpoint{3.377389in}{0.490907in}}%
\pgfpathlineto{\pgfqpoint{3.449650in}{0.481223in}}%
\pgfpathlineto{\pgfqpoint{3.521911in}{0.471539in}}%
\pgfpathlineto{\pgfqpoint{3.594172in}{0.461855in}}%
\pgfpathlineto{\pgfqpoint{3.666434in}{0.452171in}}%
\pgfpathlineto{\pgfqpoint{3.738695in}{0.442488in}}%
\pgfpathlineto{\pgfqpoint{3.883217in}{2.445419in}}%
\pgfpathlineto{\pgfqpoint{3.955478in}{2.439729in}}%
\pgfpathlineto{\pgfqpoint{4.027739in}{2.434038in}}%
\pgfpathlineto{\pgfqpoint{4.100000in}{2.428348in}}%
\pgfpathlineto{\pgfqpoint{4.172261in}{2.422657in}}%
\pgfpathlineto{\pgfqpoint{4.244522in}{2.416967in}}%
\pgfpathlineto{\pgfqpoint{4.316783in}{2.411276in}}%
\pgfpathlineto{\pgfqpoint{4.389044in}{2.405586in}}%
\pgfpathlineto{\pgfqpoint{4.461305in}{2.399895in}}%
\pgfpathlineto{\pgfqpoint{4.533566in}{2.394205in}}%
\pgfpathlineto{\pgfqpoint{4.605828in}{2.388514in}}%
\pgfpathlineto{\pgfqpoint{4.678089in}{2.382824in}}%
\pgfpathlineto{\pgfqpoint{4.750350in}{2.377133in}}%
\pgfpathlineto{\pgfqpoint{4.822611in}{2.371442in}}%
\pgfpathlineto{\pgfqpoint{4.894872in}{2.365752in}}%
\pgfpathlineto{\pgfqpoint{4.967133in}{2.360061in}}%
\pgfpathlineto{\pgfqpoint{5.039394in}{2.354371in}}%
\pgfpathlineto{\pgfqpoint{5.111655in}{2.348680in}}%
\pgfpathlineto{\pgfqpoint{5.183916in}{2.342990in}}%
\pgfpathlineto{\pgfqpoint{5.256177in}{2.337299in}}%
\pgfpathlineto{\pgfqpoint{5.328438in}{2.331609in}}%
\pgfpathlineto{\pgfqpoint{5.400699in}{2.325918in}}%
\pgfpathlineto{\pgfqpoint{5.472960in}{2.320228in}}%
\pgfpathlineto{\pgfqpoint{5.545221in}{2.314537in}}%
\pgfpathlineto{\pgfqpoint{5.617483in}{2.308847in}}%
\pgfpathlineto{\pgfqpoint{5.689744in}{2.303156in}}%
\pgfpathlineto{\pgfqpoint{5.762005in}{2.297466in}}%
\pgfpathlineto{\pgfqpoint{5.834266in}{2.291775in}}%
\pgfpathlineto{\pgfqpoint{5.906527in}{2.286085in}}%
\pgfpathlineto{\pgfqpoint{5.978788in}{2.280394in}}%
\pgfpathlineto{\pgfqpoint{6.051049in}{2.274704in}}%
\pgfpathlineto{\pgfqpoint{6.123310in}{2.269013in}}%
\pgfpathlineto{\pgfqpoint{6.195571in}{2.263323in}}%
\pgfpathlineto{\pgfqpoint{6.267832in}{2.257632in}}%
\pgfpathlineto{\pgfqpoint{6.340093in}{2.251942in}}%
\pgfpathlineto{\pgfqpoint{6.412354in}{2.246251in}}%
\pgfpathlineto{\pgfqpoint{6.484615in}{2.240561in}}%
\pgfpathlineto{\pgfqpoint{6.556876in}{2.234870in}}%
\pgfpathlineto{\pgfqpoint{6.629138in}{2.229180in}}%
\pgfpathlineto{\pgfqpoint{6.701399in}{2.223489in}}%
\pgfpathlineto{\pgfqpoint{6.773660in}{2.217799in}}%
\pgfpathlineto{\pgfqpoint{6.918182in}{2.206418in}}%
\pgfpathlineto{\pgfqpoint{6.918182in}{1.529475in}}%
\pgfusepath{stroke}%
\end{pgfscope}%
\begin{pgfscope}%
\pgfpathrectangle{\pgfqpoint{1.000000in}{0.330000in}}{\pgfqpoint{6.200000in}{2.310000in}}%
\pgfusepath{clip}%
\pgfsetrectcap%
\pgfsetroundjoin%
\pgfsetlinewidth{1.505625pt}%
\definecolor{currentstroke}{rgb}{0.090196,0.745098,0.811765}%
\pgfsetstrokecolor{currentstroke}%
\pgfsetdash{}{0pt}%
\pgfpathmoveto{\pgfqpoint{1.281818in}{1.529475in}}%
\pgfpathlineto{\pgfqpoint{1.281818in}{1.116913in}}%
\pgfpathlineto{\pgfqpoint{1.354079in}{1.110274in}}%
\pgfpathlineto{\pgfqpoint{1.426340in}{1.103635in}}%
\pgfpathlineto{\pgfqpoint{1.498601in}{1.096996in}}%
\pgfpathlineto{\pgfqpoint{1.570862in}{1.090357in}}%
\pgfpathlineto{\pgfqpoint{1.643124in}{1.083719in}}%
\pgfpathlineto{\pgfqpoint{1.715385in}{1.077080in}}%
\pgfpathlineto{\pgfqpoint{1.787646in}{1.070441in}}%
\pgfpathlineto{\pgfqpoint{1.859907in}{1.063802in}}%
\pgfpathlineto{\pgfqpoint{1.932168in}{1.057163in}}%
\pgfpathlineto{\pgfqpoint{2.004429in}{1.050524in}}%
\pgfpathlineto{\pgfqpoint{2.076690in}{1.043885in}}%
\pgfpathlineto{\pgfqpoint{2.148951in}{1.037246in}}%
\pgfpathlineto{\pgfqpoint{2.221212in}{1.030607in}}%
\pgfpathlineto{\pgfqpoint{2.293473in}{1.023968in}}%
\pgfpathlineto{\pgfqpoint{2.365734in}{1.017329in}}%
\pgfpathlineto{\pgfqpoint{2.437995in}{1.010690in}}%
\pgfpathlineto{\pgfqpoint{2.510256in}{1.004051in}}%
\pgfpathlineto{\pgfqpoint{2.582517in}{0.997412in}}%
\pgfpathlineto{\pgfqpoint{2.654779in}{0.990774in}}%
\pgfpathlineto{\pgfqpoint{2.727040in}{0.984135in}}%
\pgfpathlineto{\pgfqpoint{2.799301in}{0.977496in}}%
\pgfpathlineto{\pgfqpoint{2.871562in}{0.970857in}}%
\pgfpathlineto{\pgfqpoint{2.943823in}{0.964218in}}%
\pgfpathlineto{\pgfqpoint{3.016084in}{0.957579in}}%
\pgfpathlineto{\pgfqpoint{3.088345in}{0.950940in}}%
\pgfpathlineto{\pgfqpoint{3.160606in}{0.944301in}}%
\pgfpathlineto{\pgfqpoint{3.232867in}{0.937662in}}%
\pgfpathlineto{\pgfqpoint{3.305128in}{0.931023in}}%
\pgfpathlineto{\pgfqpoint{3.377389in}{0.924384in}}%
\pgfpathlineto{\pgfqpoint{3.449650in}{0.917745in}}%
\pgfpathlineto{\pgfqpoint{3.521911in}{0.911106in}}%
\pgfpathlineto{\pgfqpoint{3.594172in}{0.904467in}}%
\pgfpathlineto{\pgfqpoint{3.666434in}{0.897828in}}%
\pgfpathlineto{\pgfqpoint{3.738695in}{0.891190in}}%
\pgfpathlineto{\pgfqpoint{3.883217in}{2.122787in}}%
\pgfpathlineto{\pgfqpoint{3.955478in}{2.116148in}}%
\pgfpathlineto{\pgfqpoint{4.027739in}{2.109509in}}%
\pgfpathlineto{\pgfqpoint{4.100000in}{2.102870in}}%
\pgfpathlineto{\pgfqpoint{4.172261in}{2.096231in}}%
\pgfpathlineto{\pgfqpoint{4.244522in}{2.089592in}}%
\pgfpathlineto{\pgfqpoint{4.316783in}{2.082953in}}%
\pgfpathlineto{\pgfqpoint{4.389044in}{2.076314in}}%
\pgfpathlineto{\pgfqpoint{4.461305in}{2.069675in}}%
\pgfpathlineto{\pgfqpoint{4.533566in}{2.063036in}}%
\pgfpathlineto{\pgfqpoint{4.605828in}{2.056397in}}%
\pgfpathlineto{\pgfqpoint{4.678089in}{2.049759in}}%
\pgfpathlineto{\pgfqpoint{4.750350in}{2.043120in}}%
\pgfpathlineto{\pgfqpoint{4.822611in}{2.036481in}}%
\pgfpathlineto{\pgfqpoint{4.894872in}{2.029842in}}%
\pgfpathlineto{\pgfqpoint{4.967133in}{2.023203in}}%
\pgfpathlineto{\pgfqpoint{5.039394in}{2.016564in}}%
\pgfpathlineto{\pgfqpoint{5.111655in}{2.009925in}}%
\pgfpathlineto{\pgfqpoint{5.183916in}{2.003286in}}%
\pgfpathlineto{\pgfqpoint{5.256177in}{1.996647in}}%
\pgfpathlineto{\pgfqpoint{5.328438in}{1.990008in}}%
\pgfpathlineto{\pgfqpoint{5.400699in}{1.983369in}}%
\pgfpathlineto{\pgfqpoint{5.472960in}{1.976730in}}%
\pgfpathlineto{\pgfqpoint{5.545221in}{1.970091in}}%
\pgfpathlineto{\pgfqpoint{5.617483in}{1.963452in}}%
\pgfpathlineto{\pgfqpoint{5.689744in}{1.956813in}}%
\pgfpathlineto{\pgfqpoint{5.762005in}{1.950175in}}%
\pgfpathlineto{\pgfqpoint{5.834266in}{1.943536in}}%
\pgfpathlineto{\pgfqpoint{5.906527in}{1.936897in}}%
\pgfpathlineto{\pgfqpoint{5.978788in}{1.930258in}}%
\pgfpathlineto{\pgfqpoint{6.051049in}{1.923619in}}%
\pgfpathlineto{\pgfqpoint{6.123310in}{1.916980in}}%
\pgfpathlineto{\pgfqpoint{6.195571in}{1.910341in}}%
\pgfpathlineto{\pgfqpoint{6.267832in}{1.903702in}}%
\pgfpathlineto{\pgfqpoint{6.340093in}{1.897063in}}%
\pgfpathlineto{\pgfqpoint{6.412354in}{1.890424in}}%
\pgfpathlineto{\pgfqpoint{6.484615in}{1.883785in}}%
\pgfpathlineto{\pgfqpoint{6.556876in}{1.877146in}}%
\pgfpathlineto{\pgfqpoint{6.629138in}{1.870507in}}%
\pgfpathlineto{\pgfqpoint{6.701399in}{1.863868in}}%
\pgfpathlineto{\pgfqpoint{6.773660in}{1.857229in}}%
\pgfpathlineto{\pgfqpoint{6.918182in}{1.843952in}}%
\pgfpathlineto{\pgfqpoint{6.918182in}{1.529475in}}%
\pgfusepath{stroke}%
\end{pgfscope}%
\begin{pgfscope}%
\pgfpathrectangle{\pgfqpoint{1.000000in}{0.330000in}}{\pgfqpoint{6.200000in}{2.310000in}}%
\pgfusepath{clip}%
\pgfsetrectcap%
\pgfsetroundjoin%
\pgfsetlinewidth{1.505625pt}%
\definecolor{currentstroke}{rgb}{0.121569,0.466667,0.705882}%
\pgfsetstrokecolor{currentstroke}%
\pgfsetdash{}{0pt}%
\pgfpathmoveto{\pgfqpoint{1.281818in}{1.529475in}}%
\pgfpathlineto{\pgfqpoint{1.281818in}{1.179060in}}%
\pgfpathlineto{\pgfqpoint{1.354079in}{1.173369in}}%
\pgfpathlineto{\pgfqpoint{1.426340in}{1.167679in}}%
\pgfpathlineto{\pgfqpoint{1.498601in}{1.161988in}}%
\pgfpathlineto{\pgfqpoint{1.570862in}{1.156298in}}%
\pgfpathlineto{\pgfqpoint{1.643124in}{1.150607in}}%
\pgfpathlineto{\pgfqpoint{1.715385in}{1.144917in}}%
\pgfpathlineto{\pgfqpoint{1.787646in}{1.139226in}}%
\pgfpathlineto{\pgfqpoint{1.859907in}{1.133536in}}%
\pgfpathlineto{\pgfqpoint{1.932168in}{1.127845in}}%
\pgfpathlineto{\pgfqpoint{2.004429in}{1.122154in}}%
\pgfpathlineto{\pgfqpoint{2.076690in}{1.116464in}}%
\pgfpathlineto{\pgfqpoint{2.148951in}{1.110773in}}%
\pgfpathlineto{\pgfqpoint{2.221212in}{1.105083in}}%
\pgfpathlineto{\pgfqpoint{2.293473in}{1.099392in}}%
\pgfpathlineto{\pgfqpoint{2.365734in}{1.093702in}}%
\pgfpathlineto{\pgfqpoint{2.437995in}{1.088011in}}%
\pgfpathlineto{\pgfqpoint{2.510256in}{1.082321in}}%
\pgfpathlineto{\pgfqpoint{2.582517in}{1.076630in}}%
\pgfpathlineto{\pgfqpoint{2.654779in}{1.070940in}}%
\pgfpathlineto{\pgfqpoint{2.727040in}{1.065249in}}%
\pgfpathlineto{\pgfqpoint{2.799301in}{1.059559in}}%
\pgfpathlineto{\pgfqpoint{2.871562in}{1.053868in}}%
\pgfpathlineto{\pgfqpoint{2.943823in}{1.048178in}}%
\pgfpathlineto{\pgfqpoint{3.016084in}{1.042487in}}%
\pgfpathlineto{\pgfqpoint{3.088345in}{1.036797in}}%
\pgfpathlineto{\pgfqpoint{3.160606in}{1.031106in}}%
\pgfpathlineto{\pgfqpoint{3.232867in}{1.025416in}}%
\pgfpathlineto{\pgfqpoint{3.305128in}{1.019725in}}%
\pgfpathlineto{\pgfqpoint{3.377389in}{1.014035in}}%
\pgfpathlineto{\pgfqpoint{3.449650in}{1.008344in}}%
\pgfpathlineto{\pgfqpoint{3.521911in}{1.002654in}}%
\pgfpathlineto{\pgfqpoint{3.594172in}{0.996963in}}%
\pgfpathlineto{\pgfqpoint{3.666434in}{0.991273in}}%
\pgfpathlineto{\pgfqpoint{3.738695in}{0.985582in}}%
\pgfpathlineto{\pgfqpoint{3.883217in}{2.118440in}}%
\pgfpathlineto{\pgfqpoint{3.955478in}{2.108757in}}%
\pgfpathlineto{\pgfqpoint{4.027739in}{2.099073in}}%
\pgfpathlineto{\pgfqpoint{4.100000in}{2.089389in}}%
\pgfpathlineto{\pgfqpoint{4.172261in}{2.079705in}}%
\pgfpathlineto{\pgfqpoint{4.244522in}{2.070021in}}%
\pgfpathlineto{\pgfqpoint{4.316783in}{2.060337in}}%
\pgfpathlineto{\pgfqpoint{4.389044in}{2.050653in}}%
\pgfpathlineto{\pgfqpoint{4.461305in}{2.040970in}}%
\pgfpathlineto{\pgfqpoint{4.533566in}{2.031286in}}%
\pgfpathlineto{\pgfqpoint{4.605828in}{2.021602in}}%
\pgfpathlineto{\pgfqpoint{4.678089in}{2.011918in}}%
\pgfpathlineto{\pgfqpoint{4.750350in}{2.002234in}}%
\pgfpathlineto{\pgfqpoint{4.822611in}{1.992550in}}%
\pgfpathlineto{\pgfqpoint{4.894872in}{1.982866in}}%
\pgfpathlineto{\pgfqpoint{4.967133in}{1.973183in}}%
\pgfpathlineto{\pgfqpoint{5.039394in}{1.963499in}}%
\pgfpathlineto{\pgfqpoint{5.111655in}{1.953815in}}%
\pgfpathlineto{\pgfqpoint{5.183916in}{1.944131in}}%
\pgfpathlineto{\pgfqpoint{5.256177in}{1.934447in}}%
\pgfpathlineto{\pgfqpoint{5.328438in}{1.924763in}}%
\pgfpathlineto{\pgfqpoint{5.400699in}{1.915079in}}%
\pgfpathlineto{\pgfqpoint{5.472960in}{1.905396in}}%
\pgfpathlineto{\pgfqpoint{5.545221in}{1.895712in}}%
\pgfpathlineto{\pgfqpoint{5.617483in}{1.886028in}}%
\pgfpathlineto{\pgfqpoint{5.689744in}{1.876344in}}%
\pgfpathlineto{\pgfqpoint{5.762005in}{1.866660in}}%
\pgfpathlineto{\pgfqpoint{5.834266in}{1.856976in}}%
\pgfpathlineto{\pgfqpoint{5.906527in}{1.847292in}}%
\pgfpathlineto{\pgfqpoint{5.978788in}{1.837609in}}%
\pgfpathlineto{\pgfqpoint{6.051049in}{1.827925in}}%
\pgfpathlineto{\pgfqpoint{6.123310in}{1.818241in}}%
\pgfpathlineto{\pgfqpoint{6.195571in}{1.808557in}}%
\pgfpathlineto{\pgfqpoint{6.267832in}{1.798873in}}%
\pgfpathlineto{\pgfqpoint{6.340093in}{1.789189in}}%
\pgfpathlineto{\pgfqpoint{6.412354in}{1.779505in}}%
\pgfpathlineto{\pgfqpoint{6.484615in}{1.769822in}}%
\pgfpathlineto{\pgfqpoint{6.556876in}{1.760138in}}%
\pgfpathlineto{\pgfqpoint{6.629138in}{1.750454in}}%
\pgfpathlineto{\pgfqpoint{6.701399in}{1.740770in}}%
\pgfpathlineto{\pgfqpoint{6.773660in}{1.731086in}}%
\pgfpathlineto{\pgfqpoint{6.918182in}{1.711718in}}%
\pgfpathlineto{\pgfqpoint{6.918182in}{1.529475in}}%
\pgfusepath{stroke}%
\end{pgfscope}%
\begin{pgfscope}%
\pgfpathrectangle{\pgfqpoint{1.000000in}{0.330000in}}{\pgfqpoint{6.200000in}{2.310000in}}%
\pgfusepath{clip}%
\pgfsetrectcap%
\pgfsetroundjoin%
\pgfsetlinewidth{1.505625pt}%
\definecolor{currentstroke}{rgb}{1.000000,0.498039,0.054902}%
\pgfsetstrokecolor{currentstroke}%
\pgfsetdash{}{0pt}%
\pgfpathmoveto{\pgfqpoint{1.281818in}{1.529475in}}%
\pgfpathlineto{\pgfqpoint{1.281818in}{1.120657in}}%
\pgfpathlineto{\pgfqpoint{1.354079in}{1.114018in}}%
\pgfpathlineto{\pgfqpoint{1.426340in}{1.107379in}}%
\pgfpathlineto{\pgfqpoint{1.498601in}{1.100740in}}%
\pgfpathlineto{\pgfqpoint{1.570862in}{1.094101in}}%
\pgfpathlineto{\pgfqpoint{1.643124in}{1.087462in}}%
\pgfpathlineto{\pgfqpoint{1.715385in}{1.080823in}}%
\pgfpathlineto{\pgfqpoint{1.787646in}{1.074184in}}%
\pgfpathlineto{\pgfqpoint{1.859907in}{1.067546in}}%
\pgfpathlineto{\pgfqpoint{1.932168in}{1.060907in}}%
\pgfpathlineto{\pgfqpoint{2.004429in}{1.054268in}}%
\pgfpathlineto{\pgfqpoint{2.076690in}{1.047629in}}%
\pgfpathlineto{\pgfqpoint{2.148951in}{1.040990in}}%
\pgfpathlineto{\pgfqpoint{2.221212in}{1.034351in}}%
\pgfpathlineto{\pgfqpoint{2.293473in}{1.027712in}}%
\pgfpathlineto{\pgfqpoint{2.365734in}{1.021073in}}%
\pgfpathlineto{\pgfqpoint{2.437995in}{1.014434in}}%
\pgfpathlineto{\pgfqpoint{2.510256in}{1.007795in}}%
\pgfpathlineto{\pgfqpoint{2.582517in}{1.001156in}}%
\pgfpathlineto{\pgfqpoint{2.654779in}{0.994517in}}%
\pgfpathlineto{\pgfqpoint{2.727040in}{0.987878in}}%
\pgfpathlineto{\pgfqpoint{2.799301in}{0.981239in}}%
\pgfpathlineto{\pgfqpoint{2.871562in}{0.974600in}}%
\pgfpathlineto{\pgfqpoint{2.943823in}{0.967962in}}%
\pgfpathlineto{\pgfqpoint{3.016084in}{0.961323in}}%
\pgfpathlineto{\pgfqpoint{3.088345in}{0.954684in}}%
\pgfpathlineto{\pgfqpoint{3.160606in}{0.948045in}}%
\pgfpathlineto{\pgfqpoint{3.232867in}{0.941406in}}%
\pgfpathlineto{\pgfqpoint{3.305128in}{0.934767in}}%
\pgfpathlineto{\pgfqpoint{3.377389in}{0.928128in}}%
\pgfpathlineto{\pgfqpoint{3.449650in}{0.921489in}}%
\pgfpathlineto{\pgfqpoint{3.521911in}{0.914850in}}%
\pgfpathlineto{\pgfqpoint{3.594172in}{0.908211in}}%
\pgfpathlineto{\pgfqpoint{3.666434in}{0.901572in}}%
\pgfpathlineto{\pgfqpoint{3.738695in}{0.894933in}}%
\pgfpathlineto{\pgfqpoint{3.883217in}{2.119578in}}%
\pgfpathlineto{\pgfqpoint{3.955478in}{2.112939in}}%
\pgfpathlineto{\pgfqpoint{4.027739in}{2.106300in}}%
\pgfpathlineto{\pgfqpoint{4.100000in}{2.099661in}}%
\pgfpathlineto{\pgfqpoint{4.172261in}{2.093022in}}%
\pgfpathlineto{\pgfqpoint{4.244522in}{2.086383in}}%
\pgfpathlineto{\pgfqpoint{4.316783in}{2.079744in}}%
\pgfpathlineto{\pgfqpoint{4.389044in}{2.073105in}}%
\pgfpathlineto{\pgfqpoint{4.461305in}{2.066466in}}%
\pgfpathlineto{\pgfqpoint{4.533566in}{2.059827in}}%
\pgfpathlineto{\pgfqpoint{4.605828in}{2.053189in}}%
\pgfpathlineto{\pgfqpoint{4.678089in}{2.046550in}}%
\pgfpathlineto{\pgfqpoint{4.750350in}{2.039911in}}%
\pgfpathlineto{\pgfqpoint{4.822611in}{2.033272in}}%
\pgfpathlineto{\pgfqpoint{4.894872in}{2.026633in}}%
\pgfpathlineto{\pgfqpoint{4.967133in}{2.019994in}}%
\pgfpathlineto{\pgfqpoint{5.039394in}{2.013355in}}%
\pgfpathlineto{\pgfqpoint{5.111655in}{2.006716in}}%
\pgfpathlineto{\pgfqpoint{5.183916in}{2.000077in}}%
\pgfpathlineto{\pgfqpoint{5.256177in}{1.993438in}}%
\pgfpathlineto{\pgfqpoint{5.328438in}{1.986799in}}%
\pgfpathlineto{\pgfqpoint{5.400699in}{1.980160in}}%
\pgfpathlineto{\pgfqpoint{5.472960in}{1.973521in}}%
\pgfpathlineto{\pgfqpoint{5.545221in}{1.966882in}}%
\pgfpathlineto{\pgfqpoint{5.617483in}{1.960243in}}%
\pgfpathlineto{\pgfqpoint{5.689744in}{1.953605in}}%
\pgfpathlineto{\pgfqpoint{5.762005in}{1.946966in}}%
\pgfpathlineto{\pgfqpoint{5.834266in}{1.940327in}}%
\pgfpathlineto{\pgfqpoint{5.906527in}{1.933688in}}%
\pgfpathlineto{\pgfqpoint{5.978788in}{1.927049in}}%
\pgfpathlineto{\pgfqpoint{6.051049in}{1.920410in}}%
\pgfpathlineto{\pgfqpoint{6.123310in}{1.913771in}}%
\pgfpathlineto{\pgfqpoint{6.195571in}{1.907132in}}%
\pgfpathlineto{\pgfqpoint{6.267832in}{1.900493in}}%
\pgfpathlineto{\pgfqpoint{6.340093in}{1.893854in}}%
\pgfpathlineto{\pgfqpoint{6.412354in}{1.887215in}}%
\pgfpathlineto{\pgfqpoint{6.484615in}{1.880576in}}%
\pgfpathlineto{\pgfqpoint{6.556876in}{1.873937in}}%
\pgfpathlineto{\pgfqpoint{6.629138in}{1.867298in}}%
\pgfpathlineto{\pgfqpoint{6.701399in}{1.860659in}}%
\pgfpathlineto{\pgfqpoint{6.773660in}{1.854021in}}%
\pgfpathlineto{\pgfqpoint{6.918182in}{1.840743in}}%
\pgfpathlineto{\pgfqpoint{6.918182in}{1.529475in}}%
\pgfusepath{stroke}%
\end{pgfscope}%
\begin{pgfscope}%
\pgfpathrectangle{\pgfqpoint{1.000000in}{0.330000in}}{\pgfqpoint{6.200000in}{2.310000in}}%
\pgfusepath{clip}%
\pgfsetrectcap%
\pgfsetroundjoin%
\pgfsetlinewidth{1.505625pt}%
\definecolor{currentstroke}{rgb}{0.172549,0.627451,0.172549}%
\pgfsetstrokecolor{currentstroke}%
\pgfsetdash{}{0pt}%
\pgfpathmoveto{\pgfqpoint{1.281818in}{1.529475in}}%
\pgfpathlineto{\pgfqpoint{1.281818in}{1.179060in}}%
\pgfpathlineto{\pgfqpoint{1.354079in}{1.173369in}}%
\pgfpathlineto{\pgfqpoint{1.426340in}{1.167679in}}%
\pgfpathlineto{\pgfqpoint{1.498601in}{1.161988in}}%
\pgfpathlineto{\pgfqpoint{1.570862in}{1.156298in}}%
\pgfpathlineto{\pgfqpoint{1.643124in}{1.150607in}}%
\pgfpathlineto{\pgfqpoint{1.715385in}{1.144917in}}%
\pgfpathlineto{\pgfqpoint{1.787646in}{1.139226in}}%
\pgfpathlineto{\pgfqpoint{1.859907in}{1.133536in}}%
\pgfpathlineto{\pgfqpoint{1.932168in}{1.127845in}}%
\pgfpathlineto{\pgfqpoint{2.004429in}{1.122154in}}%
\pgfpathlineto{\pgfqpoint{2.076690in}{1.116464in}}%
\pgfpathlineto{\pgfqpoint{2.148951in}{1.110773in}}%
\pgfpathlineto{\pgfqpoint{2.221212in}{1.105083in}}%
\pgfpathlineto{\pgfqpoint{2.293473in}{1.099392in}}%
\pgfpathlineto{\pgfqpoint{2.365734in}{1.093702in}}%
\pgfpathlineto{\pgfqpoint{2.437995in}{1.088011in}}%
\pgfpathlineto{\pgfqpoint{2.510256in}{1.082321in}}%
\pgfpathlineto{\pgfqpoint{2.582517in}{1.076630in}}%
\pgfpathlineto{\pgfqpoint{2.654779in}{1.070940in}}%
\pgfpathlineto{\pgfqpoint{2.727040in}{1.065249in}}%
\pgfpathlineto{\pgfqpoint{2.799301in}{1.059559in}}%
\pgfpathlineto{\pgfqpoint{2.871562in}{1.053868in}}%
\pgfpathlineto{\pgfqpoint{2.943823in}{1.048178in}}%
\pgfpathlineto{\pgfqpoint{3.016084in}{1.042487in}}%
\pgfpathlineto{\pgfqpoint{3.088345in}{1.036797in}}%
\pgfpathlineto{\pgfqpoint{3.160606in}{1.031106in}}%
\pgfpathlineto{\pgfqpoint{3.232867in}{1.025416in}}%
\pgfpathlineto{\pgfqpoint{3.305128in}{1.019725in}}%
\pgfpathlineto{\pgfqpoint{3.377389in}{1.014035in}}%
\pgfpathlineto{\pgfqpoint{3.449650in}{1.008344in}}%
\pgfpathlineto{\pgfqpoint{3.521911in}{1.002654in}}%
\pgfpathlineto{\pgfqpoint{3.594172in}{0.996963in}}%
\pgfpathlineto{\pgfqpoint{3.666434in}{0.991273in}}%
\pgfpathlineto{\pgfqpoint{3.738695in}{0.985582in}}%
\pgfpathlineto{\pgfqpoint{3.883217in}{2.118440in}}%
\pgfpathlineto{\pgfqpoint{3.955478in}{2.108757in}}%
\pgfpathlineto{\pgfqpoint{4.027739in}{2.099073in}}%
\pgfpathlineto{\pgfqpoint{4.100000in}{2.089389in}}%
\pgfpathlineto{\pgfqpoint{4.172261in}{2.079705in}}%
\pgfpathlineto{\pgfqpoint{4.244522in}{2.070021in}}%
\pgfpathlineto{\pgfqpoint{4.316783in}{2.060337in}}%
\pgfpathlineto{\pgfqpoint{4.389044in}{2.050653in}}%
\pgfpathlineto{\pgfqpoint{4.461305in}{2.040970in}}%
\pgfpathlineto{\pgfqpoint{4.533566in}{2.031286in}}%
\pgfpathlineto{\pgfqpoint{4.605828in}{2.021602in}}%
\pgfpathlineto{\pgfqpoint{4.678089in}{2.011918in}}%
\pgfpathlineto{\pgfqpoint{4.750350in}{2.002234in}}%
\pgfpathlineto{\pgfqpoint{4.822611in}{1.992550in}}%
\pgfpathlineto{\pgfqpoint{4.894872in}{1.982866in}}%
\pgfpathlineto{\pgfqpoint{4.967133in}{1.973183in}}%
\pgfpathlineto{\pgfqpoint{5.039394in}{1.963499in}}%
\pgfpathlineto{\pgfqpoint{5.111655in}{1.953815in}}%
\pgfpathlineto{\pgfqpoint{5.183916in}{1.944131in}}%
\pgfpathlineto{\pgfqpoint{5.256177in}{1.934447in}}%
\pgfpathlineto{\pgfqpoint{5.328438in}{1.924763in}}%
\pgfpathlineto{\pgfqpoint{5.400699in}{1.915079in}}%
\pgfpathlineto{\pgfqpoint{5.472960in}{1.905396in}}%
\pgfpathlineto{\pgfqpoint{5.545221in}{1.895712in}}%
\pgfpathlineto{\pgfqpoint{5.617483in}{1.886028in}}%
\pgfpathlineto{\pgfqpoint{5.689744in}{1.876344in}}%
\pgfpathlineto{\pgfqpoint{5.762005in}{1.866660in}}%
\pgfpathlineto{\pgfqpoint{5.834266in}{1.856976in}}%
\pgfpathlineto{\pgfqpoint{5.906527in}{1.847292in}}%
\pgfpathlineto{\pgfqpoint{5.978788in}{1.837609in}}%
\pgfpathlineto{\pgfqpoint{6.051049in}{1.827925in}}%
\pgfpathlineto{\pgfqpoint{6.123310in}{1.818241in}}%
\pgfpathlineto{\pgfqpoint{6.195571in}{1.808557in}}%
\pgfpathlineto{\pgfqpoint{6.267832in}{1.798873in}}%
\pgfpathlineto{\pgfqpoint{6.340093in}{1.789189in}}%
\pgfpathlineto{\pgfqpoint{6.412354in}{1.779505in}}%
\pgfpathlineto{\pgfqpoint{6.484615in}{1.769822in}}%
\pgfpathlineto{\pgfqpoint{6.556876in}{1.760138in}}%
\pgfpathlineto{\pgfqpoint{6.629138in}{1.750454in}}%
\pgfpathlineto{\pgfqpoint{6.701399in}{1.740770in}}%
\pgfpathlineto{\pgfqpoint{6.773660in}{1.731086in}}%
\pgfpathlineto{\pgfqpoint{6.918182in}{1.711718in}}%
\pgfpathlineto{\pgfqpoint{6.918182in}{1.529475in}}%
\pgfusepath{stroke}%
\end{pgfscope}%
\begin{pgfscope}%
\pgfpathrectangle{\pgfqpoint{1.000000in}{0.330000in}}{\pgfqpoint{6.200000in}{2.310000in}}%
\pgfusepath{clip}%
\pgfsetrectcap%
\pgfsetroundjoin%
\pgfsetlinewidth{1.505625pt}%
\definecolor{currentstroke}{rgb}{0.839216,0.152941,0.156863}%
\pgfsetstrokecolor{currentstroke}%
\pgfsetdash{}{0pt}%
\pgfpathmoveto{\pgfqpoint{1.281818in}{1.529475in}}%
\pgfpathlineto{\pgfqpoint{1.281818in}{1.044284in}}%
\pgfpathlineto{\pgfqpoint{1.354079in}{1.037346in}}%
\pgfpathlineto{\pgfqpoint{1.426340in}{1.030407in}}%
\pgfpathlineto{\pgfqpoint{1.498601in}{1.023469in}}%
\pgfpathlineto{\pgfqpoint{1.570862in}{1.016531in}}%
\pgfpathlineto{\pgfqpoint{1.643124in}{1.009592in}}%
\pgfpathlineto{\pgfqpoint{1.715385in}{1.002654in}}%
\pgfpathlineto{\pgfqpoint{1.787646in}{0.995715in}}%
\pgfpathlineto{\pgfqpoint{1.859907in}{0.988777in}}%
\pgfpathlineto{\pgfqpoint{1.932168in}{0.981838in}}%
\pgfpathlineto{\pgfqpoint{2.004429in}{0.974900in}}%
\pgfpathlineto{\pgfqpoint{2.076690in}{0.967962in}}%
\pgfpathlineto{\pgfqpoint{2.148951in}{0.961023in}}%
\pgfpathlineto{\pgfqpoint{2.221212in}{0.954085in}}%
\pgfpathlineto{\pgfqpoint{2.293473in}{0.947146in}}%
\pgfpathlineto{\pgfqpoint{2.365734in}{0.940208in}}%
\pgfpathlineto{\pgfqpoint{2.437995in}{0.933269in}}%
\pgfpathlineto{\pgfqpoint{2.510256in}{0.926331in}}%
\pgfpathlineto{\pgfqpoint{2.582517in}{0.919392in}}%
\pgfpathlineto{\pgfqpoint{2.654779in}{0.912454in}}%
\pgfpathlineto{\pgfqpoint{2.727040in}{0.905516in}}%
\pgfpathlineto{\pgfqpoint{2.799301in}{0.898577in}}%
\pgfpathlineto{\pgfqpoint{2.871562in}{0.891639in}}%
\pgfpathlineto{\pgfqpoint{2.943823in}{0.884700in}}%
\pgfpathlineto{\pgfqpoint{3.016084in}{0.877762in}}%
\pgfpathlineto{\pgfqpoint{3.088345in}{0.870823in}}%
\pgfpathlineto{\pgfqpoint{3.160606in}{0.863885in}}%
\pgfpathlineto{\pgfqpoint{3.232867in}{0.856947in}}%
\pgfpathlineto{\pgfqpoint{3.305128in}{0.850008in}}%
\pgfpathlineto{\pgfqpoint{3.377389in}{0.843070in}}%
\pgfpathlineto{\pgfqpoint{3.449650in}{0.836131in}}%
\pgfpathlineto{\pgfqpoint{3.521911in}{0.829193in}}%
\pgfpathlineto{\pgfqpoint{3.594172in}{0.822254in}}%
\pgfpathlineto{\pgfqpoint{3.666434in}{0.815316in}}%
\pgfpathlineto{\pgfqpoint{3.738695in}{0.808378in}}%
\pgfpathlineto{\pgfqpoint{3.883217in}{2.169865in}}%
\pgfpathlineto{\pgfqpoint{3.955478in}{2.164174in}}%
\pgfpathlineto{\pgfqpoint{4.027739in}{2.158484in}}%
\pgfpathlineto{\pgfqpoint{4.100000in}{2.152793in}}%
\pgfpathlineto{\pgfqpoint{4.172261in}{2.147103in}}%
\pgfpathlineto{\pgfqpoint{4.244522in}{2.141412in}}%
\pgfpathlineto{\pgfqpoint{4.316783in}{2.135722in}}%
\pgfpathlineto{\pgfqpoint{4.389044in}{2.130031in}}%
\pgfpathlineto{\pgfqpoint{4.461305in}{2.124341in}}%
\pgfpathlineto{\pgfqpoint{4.533566in}{2.118650in}}%
\pgfpathlineto{\pgfqpoint{4.605828in}{2.112960in}}%
\pgfpathlineto{\pgfqpoint{4.678089in}{2.107269in}}%
\pgfpathlineto{\pgfqpoint{4.750350in}{2.101578in}}%
\pgfpathlineto{\pgfqpoint{4.822611in}{2.095888in}}%
\pgfpathlineto{\pgfqpoint{4.894872in}{2.090197in}}%
\pgfpathlineto{\pgfqpoint{4.967133in}{2.084507in}}%
\pgfpathlineto{\pgfqpoint{5.039394in}{2.078816in}}%
\pgfpathlineto{\pgfqpoint{5.111655in}{2.073126in}}%
\pgfpathlineto{\pgfqpoint{5.183916in}{2.067435in}}%
\pgfpathlineto{\pgfqpoint{5.256177in}{2.061745in}}%
\pgfpathlineto{\pgfqpoint{5.328438in}{2.056054in}}%
\pgfpathlineto{\pgfqpoint{5.400699in}{2.050364in}}%
\pgfpathlineto{\pgfqpoint{5.472960in}{2.044673in}}%
\pgfpathlineto{\pgfqpoint{5.545221in}{2.038983in}}%
\pgfpathlineto{\pgfqpoint{5.617483in}{2.033292in}}%
\pgfpathlineto{\pgfqpoint{5.689744in}{2.027602in}}%
\pgfpathlineto{\pgfqpoint{5.762005in}{2.021911in}}%
\pgfpathlineto{\pgfqpoint{5.834266in}{2.016221in}}%
\pgfpathlineto{\pgfqpoint{5.906527in}{2.010530in}}%
\pgfpathlineto{\pgfqpoint{5.978788in}{2.004840in}}%
\pgfpathlineto{\pgfqpoint{6.051049in}{1.999149in}}%
\pgfpathlineto{\pgfqpoint{6.123310in}{1.993459in}}%
\pgfpathlineto{\pgfqpoint{6.195571in}{1.987768in}}%
\pgfpathlineto{\pgfqpoint{6.267832in}{1.982078in}}%
\pgfpathlineto{\pgfqpoint{6.340093in}{1.976387in}}%
\pgfpathlineto{\pgfqpoint{6.412354in}{1.970697in}}%
\pgfpathlineto{\pgfqpoint{6.484615in}{1.965006in}}%
\pgfpathlineto{\pgfqpoint{6.556876in}{1.959316in}}%
\pgfpathlineto{\pgfqpoint{6.629138in}{1.953625in}}%
\pgfpathlineto{\pgfqpoint{6.701399in}{1.947935in}}%
\pgfpathlineto{\pgfqpoint{6.773660in}{1.942244in}}%
\pgfpathlineto{\pgfqpoint{6.918182in}{1.930863in}}%
\pgfpathlineto{\pgfqpoint{6.918182in}{1.529475in}}%
\pgfusepath{stroke}%
\end{pgfscope}%
\begin{pgfscope}%
\pgfpathrectangle{\pgfqpoint{1.000000in}{0.330000in}}{\pgfqpoint{6.200000in}{2.310000in}}%
\pgfusepath{clip}%
\pgfsetrectcap%
\pgfsetroundjoin%
\pgfsetlinewidth{1.505625pt}%
\definecolor{currentstroke}{rgb}{0.580392,0.403922,0.741176}%
\pgfsetstrokecolor{currentstroke}%
\pgfsetdash{}{0pt}%
\pgfpathmoveto{\pgfqpoint{1.281818in}{1.529475in}}%
\pgfpathlineto{\pgfqpoint{1.281818in}{1.120657in}}%
\pgfpathlineto{\pgfqpoint{1.354079in}{1.114018in}}%
\pgfpathlineto{\pgfqpoint{1.426340in}{1.107379in}}%
\pgfpathlineto{\pgfqpoint{1.498601in}{1.100740in}}%
\pgfpathlineto{\pgfqpoint{1.570862in}{1.094101in}}%
\pgfpathlineto{\pgfqpoint{1.643124in}{1.087462in}}%
\pgfpathlineto{\pgfqpoint{1.715385in}{1.080823in}}%
\pgfpathlineto{\pgfqpoint{1.787646in}{1.074184in}}%
\pgfpathlineto{\pgfqpoint{1.859907in}{1.067546in}}%
\pgfpathlineto{\pgfqpoint{1.932168in}{1.060907in}}%
\pgfpathlineto{\pgfqpoint{2.004429in}{1.054268in}}%
\pgfpathlineto{\pgfqpoint{2.076690in}{1.047629in}}%
\pgfpathlineto{\pgfqpoint{2.148951in}{1.040990in}}%
\pgfpathlineto{\pgfqpoint{2.221212in}{1.034351in}}%
\pgfpathlineto{\pgfqpoint{2.293473in}{1.027712in}}%
\pgfpathlineto{\pgfqpoint{2.365734in}{1.021073in}}%
\pgfpathlineto{\pgfqpoint{2.437995in}{1.014434in}}%
\pgfpathlineto{\pgfqpoint{2.510256in}{1.007795in}}%
\pgfpathlineto{\pgfqpoint{2.582517in}{1.001156in}}%
\pgfpathlineto{\pgfqpoint{2.654779in}{0.994517in}}%
\pgfpathlineto{\pgfqpoint{2.727040in}{0.987878in}}%
\pgfpathlineto{\pgfqpoint{2.799301in}{0.981239in}}%
\pgfpathlineto{\pgfqpoint{2.871562in}{0.974600in}}%
\pgfpathlineto{\pgfqpoint{2.943823in}{0.967962in}}%
\pgfpathlineto{\pgfqpoint{3.016084in}{0.961323in}}%
\pgfpathlineto{\pgfqpoint{3.088345in}{0.954684in}}%
\pgfpathlineto{\pgfqpoint{3.160606in}{0.948045in}}%
\pgfpathlineto{\pgfqpoint{3.232867in}{0.941406in}}%
\pgfpathlineto{\pgfqpoint{3.305128in}{0.934767in}}%
\pgfpathlineto{\pgfqpoint{3.377389in}{0.928128in}}%
\pgfpathlineto{\pgfqpoint{3.449650in}{0.921489in}}%
\pgfpathlineto{\pgfqpoint{3.521911in}{0.914850in}}%
\pgfpathlineto{\pgfqpoint{3.594172in}{0.908211in}}%
\pgfpathlineto{\pgfqpoint{3.666434in}{0.901572in}}%
\pgfpathlineto{\pgfqpoint{3.738695in}{0.894933in}}%
\pgfpathlineto{\pgfqpoint{3.883217in}{2.119578in}}%
\pgfpathlineto{\pgfqpoint{3.955478in}{2.112939in}}%
\pgfpathlineto{\pgfqpoint{4.027739in}{2.106300in}}%
\pgfpathlineto{\pgfqpoint{4.100000in}{2.099661in}}%
\pgfpathlineto{\pgfqpoint{4.172261in}{2.093022in}}%
\pgfpathlineto{\pgfqpoint{4.244522in}{2.086383in}}%
\pgfpathlineto{\pgfqpoint{4.316783in}{2.079744in}}%
\pgfpathlineto{\pgfqpoint{4.389044in}{2.073105in}}%
\pgfpathlineto{\pgfqpoint{4.461305in}{2.066466in}}%
\pgfpathlineto{\pgfqpoint{4.533566in}{2.059827in}}%
\pgfpathlineto{\pgfqpoint{4.605828in}{2.053189in}}%
\pgfpathlineto{\pgfqpoint{4.678089in}{2.046550in}}%
\pgfpathlineto{\pgfqpoint{4.750350in}{2.039911in}}%
\pgfpathlineto{\pgfqpoint{4.822611in}{2.033272in}}%
\pgfpathlineto{\pgfqpoint{4.894872in}{2.026633in}}%
\pgfpathlineto{\pgfqpoint{4.967133in}{2.019994in}}%
\pgfpathlineto{\pgfqpoint{5.039394in}{2.013355in}}%
\pgfpathlineto{\pgfqpoint{5.111655in}{2.006716in}}%
\pgfpathlineto{\pgfqpoint{5.183916in}{2.000077in}}%
\pgfpathlineto{\pgfqpoint{5.256177in}{1.993438in}}%
\pgfpathlineto{\pgfqpoint{5.328438in}{1.986799in}}%
\pgfpathlineto{\pgfqpoint{5.400699in}{1.980160in}}%
\pgfpathlineto{\pgfqpoint{5.472960in}{1.973521in}}%
\pgfpathlineto{\pgfqpoint{5.545221in}{1.966882in}}%
\pgfpathlineto{\pgfqpoint{5.617483in}{1.960243in}}%
\pgfpathlineto{\pgfqpoint{5.689744in}{1.953605in}}%
\pgfpathlineto{\pgfqpoint{5.762005in}{1.946966in}}%
\pgfpathlineto{\pgfqpoint{5.834266in}{1.940327in}}%
\pgfpathlineto{\pgfqpoint{5.906527in}{1.933688in}}%
\pgfpathlineto{\pgfqpoint{5.978788in}{1.927049in}}%
\pgfpathlineto{\pgfqpoint{6.051049in}{1.920410in}}%
\pgfpathlineto{\pgfqpoint{6.123310in}{1.913771in}}%
\pgfpathlineto{\pgfqpoint{6.195571in}{1.907132in}}%
\pgfpathlineto{\pgfqpoint{6.267832in}{1.900493in}}%
\pgfpathlineto{\pgfqpoint{6.340093in}{1.893854in}}%
\pgfpathlineto{\pgfqpoint{6.412354in}{1.887215in}}%
\pgfpathlineto{\pgfqpoint{6.484615in}{1.880576in}}%
\pgfpathlineto{\pgfqpoint{6.556876in}{1.873937in}}%
\pgfpathlineto{\pgfqpoint{6.629138in}{1.867298in}}%
\pgfpathlineto{\pgfqpoint{6.701399in}{1.860659in}}%
\pgfpathlineto{\pgfqpoint{6.773660in}{1.854021in}}%
\pgfpathlineto{\pgfqpoint{6.918182in}{1.840743in}}%
\pgfpathlineto{\pgfqpoint{6.918182in}{1.529475in}}%
\pgfusepath{stroke}%
\end{pgfscope}%
\begin{pgfscope}%
\pgfpathrectangle{\pgfqpoint{1.000000in}{0.330000in}}{\pgfqpoint{6.200000in}{2.310000in}}%
\pgfusepath{clip}%
\pgfsetrectcap%
\pgfsetroundjoin%
\pgfsetlinewidth{1.505625pt}%
\definecolor{currentstroke}{rgb}{0.549020,0.337255,0.294118}%
\pgfsetstrokecolor{currentstroke}%
\pgfsetdash{}{0pt}%
\pgfpathmoveto{\pgfqpoint{1.281818in}{1.529475in}}%
\pgfpathlineto{\pgfqpoint{1.281818in}{1.171572in}}%
\pgfpathlineto{\pgfqpoint{1.354079in}{1.165882in}}%
\pgfpathlineto{\pgfqpoint{1.426340in}{1.160191in}}%
\pgfpathlineto{\pgfqpoint{1.498601in}{1.154501in}}%
\pgfpathlineto{\pgfqpoint{1.570862in}{1.148810in}}%
\pgfpathlineto{\pgfqpoint{1.643124in}{1.143120in}}%
\pgfpathlineto{\pgfqpoint{1.715385in}{1.137429in}}%
\pgfpathlineto{\pgfqpoint{1.787646in}{1.131739in}}%
\pgfpathlineto{\pgfqpoint{1.859907in}{1.126048in}}%
\pgfpathlineto{\pgfqpoint{1.932168in}{1.120357in}}%
\pgfpathlineto{\pgfqpoint{2.004429in}{1.114667in}}%
\pgfpathlineto{\pgfqpoint{2.076690in}{1.108976in}}%
\pgfpathlineto{\pgfqpoint{2.148951in}{1.103286in}}%
\pgfpathlineto{\pgfqpoint{2.221212in}{1.097595in}}%
\pgfpathlineto{\pgfqpoint{2.293473in}{1.091905in}}%
\pgfpathlineto{\pgfqpoint{2.365734in}{1.086214in}}%
\pgfpathlineto{\pgfqpoint{2.437995in}{1.080524in}}%
\pgfpathlineto{\pgfqpoint{2.510256in}{1.074833in}}%
\pgfpathlineto{\pgfqpoint{2.582517in}{1.069143in}}%
\pgfpathlineto{\pgfqpoint{2.654779in}{1.063452in}}%
\pgfpathlineto{\pgfqpoint{2.727040in}{1.057762in}}%
\pgfpathlineto{\pgfqpoint{2.799301in}{1.052071in}}%
\pgfpathlineto{\pgfqpoint{2.871562in}{1.046381in}}%
\pgfpathlineto{\pgfqpoint{2.943823in}{1.040690in}}%
\pgfpathlineto{\pgfqpoint{3.016084in}{1.035000in}}%
\pgfpathlineto{\pgfqpoint{3.088345in}{1.029309in}}%
\pgfpathlineto{\pgfqpoint{3.160606in}{1.023619in}}%
\pgfpathlineto{\pgfqpoint{3.232867in}{1.017928in}}%
\pgfpathlineto{\pgfqpoint{3.305128in}{1.012238in}}%
\pgfpathlineto{\pgfqpoint{3.377389in}{1.006547in}}%
\pgfpathlineto{\pgfqpoint{3.449650in}{1.000857in}}%
\pgfpathlineto{\pgfqpoint{3.521911in}{0.995166in}}%
\pgfpathlineto{\pgfqpoint{3.594172in}{0.989476in}}%
\pgfpathlineto{\pgfqpoint{3.666434in}{0.983785in}}%
\pgfpathlineto{\pgfqpoint{3.738695in}{0.978095in}}%
\pgfpathlineto{\pgfqpoint{3.883217in}{2.041695in}}%
\pgfpathlineto{\pgfqpoint{3.955478in}{2.036005in}}%
\pgfpathlineto{\pgfqpoint{4.027739in}{2.030314in}}%
\pgfpathlineto{\pgfqpoint{4.100000in}{2.024624in}}%
\pgfpathlineto{\pgfqpoint{4.172261in}{2.018933in}}%
\pgfpathlineto{\pgfqpoint{4.244522in}{2.013243in}}%
\pgfpathlineto{\pgfqpoint{4.316783in}{2.007552in}}%
\pgfpathlineto{\pgfqpoint{4.389044in}{2.001862in}}%
\pgfpathlineto{\pgfqpoint{4.461305in}{1.996171in}}%
\pgfpathlineto{\pgfqpoint{4.533566in}{1.990481in}}%
\pgfpathlineto{\pgfqpoint{4.605828in}{1.984790in}}%
\pgfpathlineto{\pgfqpoint{4.678089in}{1.979100in}}%
\pgfpathlineto{\pgfqpoint{4.750350in}{1.973409in}}%
\pgfpathlineto{\pgfqpoint{4.822611in}{1.967719in}}%
\pgfpathlineto{\pgfqpoint{4.894872in}{1.962028in}}%
\pgfpathlineto{\pgfqpoint{4.967133in}{1.956338in}}%
\pgfpathlineto{\pgfqpoint{5.039394in}{1.950647in}}%
\pgfpathlineto{\pgfqpoint{5.111655in}{1.944957in}}%
\pgfpathlineto{\pgfqpoint{5.183916in}{1.939266in}}%
\pgfpathlineto{\pgfqpoint{5.256177in}{1.933576in}}%
\pgfpathlineto{\pgfqpoint{5.328438in}{1.927885in}}%
\pgfpathlineto{\pgfqpoint{5.400699in}{1.922195in}}%
\pgfpathlineto{\pgfqpoint{5.472960in}{1.916504in}}%
\pgfpathlineto{\pgfqpoint{5.545221in}{1.910814in}}%
\pgfpathlineto{\pgfqpoint{5.617483in}{1.905123in}}%
\pgfpathlineto{\pgfqpoint{5.689744in}{1.899433in}}%
\pgfpathlineto{\pgfqpoint{5.762005in}{1.893742in}}%
\pgfpathlineto{\pgfqpoint{5.834266in}{1.888052in}}%
\pgfpathlineto{\pgfqpoint{5.906527in}{1.882361in}}%
\pgfpathlineto{\pgfqpoint{5.978788in}{1.876670in}}%
\pgfpathlineto{\pgfqpoint{6.051049in}{1.870980in}}%
\pgfpathlineto{\pgfqpoint{6.123310in}{1.865289in}}%
\pgfpathlineto{\pgfqpoint{6.195571in}{1.859599in}}%
\pgfpathlineto{\pgfqpoint{6.267832in}{1.853908in}}%
\pgfpathlineto{\pgfqpoint{6.340093in}{1.848218in}}%
\pgfpathlineto{\pgfqpoint{6.412354in}{1.842527in}}%
\pgfpathlineto{\pgfqpoint{6.484615in}{1.836837in}}%
\pgfpathlineto{\pgfqpoint{6.556876in}{1.831146in}}%
\pgfpathlineto{\pgfqpoint{6.629138in}{1.825456in}}%
\pgfpathlineto{\pgfqpoint{6.701399in}{1.819765in}}%
\pgfpathlineto{\pgfqpoint{6.773660in}{1.814075in}}%
\pgfpathlineto{\pgfqpoint{6.918182in}{1.802694in}}%
\pgfpathlineto{\pgfqpoint{6.918182in}{1.529475in}}%
\pgfusepath{stroke}%
\end{pgfscope}%
\begin{pgfscope}%
\pgfpathrectangle{\pgfqpoint{1.000000in}{0.330000in}}{\pgfqpoint{6.200000in}{2.310000in}}%
\pgfusepath{clip}%
\pgfsetrectcap%
\pgfsetroundjoin%
\pgfsetlinewidth{1.505625pt}%
\definecolor{currentstroke}{rgb}{0.890196,0.466667,0.760784}%
\pgfsetstrokecolor{currentstroke}%
\pgfsetdash{}{0pt}%
\pgfpathmoveto{\pgfqpoint{1.281818in}{1.529475in}}%
\pgfpathlineto{\pgfqpoint{1.281818in}{0.764251in}}%
\pgfpathlineto{\pgfqpoint{1.354079in}{0.754567in}}%
\pgfpathlineto{\pgfqpoint{1.426340in}{0.744883in}}%
\pgfpathlineto{\pgfqpoint{1.498601in}{0.735200in}}%
\pgfpathlineto{\pgfqpoint{1.570862in}{0.725516in}}%
\pgfpathlineto{\pgfqpoint{1.643124in}{0.715832in}}%
\pgfpathlineto{\pgfqpoint{1.715385in}{0.706148in}}%
\pgfpathlineto{\pgfqpoint{1.787646in}{0.696464in}}%
\pgfpathlineto{\pgfqpoint{1.859907in}{0.686780in}}%
\pgfpathlineto{\pgfqpoint{1.932168in}{0.677096in}}%
\pgfpathlineto{\pgfqpoint{2.004429in}{0.667413in}}%
\pgfpathlineto{\pgfqpoint{2.076690in}{0.657729in}}%
\pgfpathlineto{\pgfqpoint{2.148951in}{0.648045in}}%
\pgfpathlineto{\pgfqpoint{2.221212in}{0.638361in}}%
\pgfpathlineto{\pgfqpoint{2.293473in}{0.628677in}}%
\pgfpathlineto{\pgfqpoint{2.365734in}{0.618993in}}%
\pgfpathlineto{\pgfqpoint{2.437995in}{0.609309in}}%
\pgfpathlineto{\pgfqpoint{2.510256in}{0.599626in}}%
\pgfpathlineto{\pgfqpoint{2.582517in}{0.589942in}}%
\pgfpathlineto{\pgfqpoint{2.654779in}{0.580258in}}%
\pgfpathlineto{\pgfqpoint{2.727040in}{0.570574in}}%
\pgfpathlineto{\pgfqpoint{2.799301in}{0.560890in}}%
\pgfpathlineto{\pgfqpoint{2.871562in}{0.551206in}}%
\pgfpathlineto{\pgfqpoint{2.943823in}{0.541522in}}%
\pgfpathlineto{\pgfqpoint{3.016084in}{0.531839in}}%
\pgfpathlineto{\pgfqpoint{3.088345in}{0.522155in}}%
\pgfpathlineto{\pgfqpoint{3.160606in}{0.512471in}}%
\pgfpathlineto{\pgfqpoint{3.232867in}{0.502787in}}%
\pgfpathlineto{\pgfqpoint{3.305128in}{0.493103in}}%
\pgfpathlineto{\pgfqpoint{3.377389in}{0.483419in}}%
\pgfpathlineto{\pgfqpoint{3.449650in}{0.473735in}}%
\pgfpathlineto{\pgfqpoint{3.521911in}{0.464052in}}%
\pgfpathlineto{\pgfqpoint{3.594172in}{0.454368in}}%
\pgfpathlineto{\pgfqpoint{3.666434in}{0.444684in}}%
\pgfpathlineto{\pgfqpoint{3.738695in}{0.435000in}}%
\pgfpathlineto{\pgfqpoint{3.883217in}{2.451837in}}%
\pgfpathlineto{\pgfqpoint{3.955478in}{2.446147in}}%
\pgfpathlineto{\pgfqpoint{4.027739in}{2.440456in}}%
\pgfpathlineto{\pgfqpoint{4.100000in}{2.434766in}}%
\pgfpathlineto{\pgfqpoint{4.172261in}{2.429075in}}%
\pgfpathlineto{\pgfqpoint{4.244522in}{2.423384in}}%
\pgfpathlineto{\pgfqpoint{4.316783in}{2.417694in}}%
\pgfpathlineto{\pgfqpoint{4.389044in}{2.412003in}}%
\pgfpathlineto{\pgfqpoint{4.461305in}{2.406313in}}%
\pgfpathlineto{\pgfqpoint{4.533566in}{2.400622in}}%
\pgfpathlineto{\pgfqpoint{4.605828in}{2.394932in}}%
\pgfpathlineto{\pgfqpoint{4.678089in}{2.389241in}}%
\pgfpathlineto{\pgfqpoint{4.750350in}{2.383551in}}%
\pgfpathlineto{\pgfqpoint{4.822611in}{2.377860in}}%
\pgfpathlineto{\pgfqpoint{4.894872in}{2.372170in}}%
\pgfpathlineto{\pgfqpoint{4.967133in}{2.366479in}}%
\pgfpathlineto{\pgfqpoint{5.039394in}{2.360789in}}%
\pgfpathlineto{\pgfqpoint{5.111655in}{2.355098in}}%
\pgfpathlineto{\pgfqpoint{5.183916in}{2.349408in}}%
\pgfpathlineto{\pgfqpoint{5.256177in}{2.343717in}}%
\pgfpathlineto{\pgfqpoint{5.328438in}{2.338027in}}%
\pgfpathlineto{\pgfqpoint{5.400699in}{2.332336in}}%
\pgfpathlineto{\pgfqpoint{5.472960in}{2.326646in}}%
\pgfpathlineto{\pgfqpoint{5.545221in}{2.320955in}}%
\pgfpathlineto{\pgfqpoint{5.617483in}{2.315265in}}%
\pgfpathlineto{\pgfqpoint{5.689744in}{2.309574in}}%
\pgfpathlineto{\pgfqpoint{5.762005in}{2.303884in}}%
\pgfpathlineto{\pgfqpoint{5.834266in}{2.298193in}}%
\pgfpathlineto{\pgfqpoint{5.906527in}{2.292503in}}%
\pgfpathlineto{\pgfqpoint{5.978788in}{2.286812in}}%
\pgfpathlineto{\pgfqpoint{6.051049in}{2.281122in}}%
\pgfpathlineto{\pgfqpoint{6.123310in}{2.275431in}}%
\pgfpathlineto{\pgfqpoint{6.195571in}{2.269741in}}%
\pgfpathlineto{\pgfqpoint{6.267832in}{2.264050in}}%
\pgfpathlineto{\pgfqpoint{6.340093in}{2.258360in}}%
\pgfpathlineto{\pgfqpoint{6.412354in}{2.252669in}}%
\pgfpathlineto{\pgfqpoint{6.484615in}{2.246979in}}%
\pgfpathlineto{\pgfqpoint{6.556876in}{2.241288in}}%
\pgfpathlineto{\pgfqpoint{6.629138in}{2.235598in}}%
\pgfpathlineto{\pgfqpoint{6.701399in}{2.229907in}}%
\pgfpathlineto{\pgfqpoint{6.773660in}{2.224217in}}%
\pgfpathlineto{\pgfqpoint{6.918182in}{2.212835in}}%
\pgfpathlineto{\pgfqpoint{6.918182in}{1.529475in}}%
\pgfusepath{stroke}%
\end{pgfscope}%
\begin{pgfscope}%
\pgfpathrectangle{\pgfqpoint{1.000000in}{0.330000in}}{\pgfqpoint{6.200000in}{2.310000in}}%
\pgfusepath{clip}%
\pgfsetrectcap%
\pgfsetroundjoin%
\pgfsetlinewidth{1.505625pt}%
\definecolor{currentstroke}{rgb}{0.498039,0.498039,0.498039}%
\pgfsetstrokecolor{currentstroke}%
\pgfsetdash{}{0pt}%
\pgfpathmoveto{\pgfqpoint{1.281818in}{1.529475in}}%
\pgfpathlineto{\pgfqpoint{1.281818in}{0.764251in}}%
\pgfpathlineto{\pgfqpoint{1.354079in}{0.754567in}}%
\pgfpathlineto{\pgfqpoint{1.426340in}{0.744883in}}%
\pgfpathlineto{\pgfqpoint{1.498601in}{0.735200in}}%
\pgfpathlineto{\pgfqpoint{1.570862in}{0.725516in}}%
\pgfpathlineto{\pgfqpoint{1.643124in}{0.715832in}}%
\pgfpathlineto{\pgfqpoint{1.715385in}{0.706148in}}%
\pgfpathlineto{\pgfqpoint{1.787646in}{0.696464in}}%
\pgfpathlineto{\pgfqpoint{1.859907in}{0.686780in}}%
\pgfpathlineto{\pgfqpoint{1.932168in}{0.677096in}}%
\pgfpathlineto{\pgfqpoint{2.004429in}{0.667413in}}%
\pgfpathlineto{\pgfqpoint{2.076690in}{0.657729in}}%
\pgfpathlineto{\pgfqpoint{2.148951in}{0.648045in}}%
\pgfpathlineto{\pgfqpoint{2.221212in}{0.638361in}}%
\pgfpathlineto{\pgfqpoint{2.293473in}{0.628677in}}%
\pgfpathlineto{\pgfqpoint{2.365734in}{0.618993in}}%
\pgfpathlineto{\pgfqpoint{2.437995in}{0.609309in}}%
\pgfpathlineto{\pgfqpoint{2.510256in}{0.599626in}}%
\pgfpathlineto{\pgfqpoint{2.582517in}{0.589942in}}%
\pgfpathlineto{\pgfqpoint{2.654779in}{0.580258in}}%
\pgfpathlineto{\pgfqpoint{2.727040in}{0.570574in}}%
\pgfpathlineto{\pgfqpoint{2.799301in}{0.560890in}}%
\pgfpathlineto{\pgfqpoint{2.871562in}{0.551206in}}%
\pgfpathlineto{\pgfqpoint{2.943823in}{0.541522in}}%
\pgfpathlineto{\pgfqpoint{3.016084in}{0.531839in}}%
\pgfpathlineto{\pgfqpoint{3.088345in}{0.522155in}}%
\pgfpathlineto{\pgfqpoint{3.160606in}{0.512471in}}%
\pgfpathlineto{\pgfqpoint{3.232867in}{0.502787in}}%
\pgfpathlineto{\pgfqpoint{3.305128in}{0.493103in}}%
\pgfpathlineto{\pgfqpoint{3.377389in}{0.483419in}}%
\pgfpathlineto{\pgfqpoint{3.449650in}{0.473735in}}%
\pgfpathlineto{\pgfqpoint{3.521911in}{0.464052in}}%
\pgfpathlineto{\pgfqpoint{3.594172in}{0.454368in}}%
\pgfpathlineto{\pgfqpoint{3.666434in}{0.444684in}}%
\pgfpathlineto{\pgfqpoint{3.738695in}{0.435000in}}%
\pgfpathlineto{\pgfqpoint{3.883217in}{2.535000in}}%
\pgfpathlineto{\pgfqpoint{3.955478in}{2.525316in}}%
\pgfpathlineto{\pgfqpoint{4.027739in}{2.515632in}}%
\pgfpathlineto{\pgfqpoint{4.100000in}{2.505948in}}%
\pgfpathlineto{\pgfqpoint{4.172261in}{2.496265in}}%
\pgfpathlineto{\pgfqpoint{4.244522in}{2.486581in}}%
\pgfpathlineto{\pgfqpoint{4.316783in}{2.476897in}}%
\pgfpathlineto{\pgfqpoint{4.389044in}{2.467213in}}%
\pgfpathlineto{\pgfqpoint{4.461305in}{2.457529in}}%
\pgfpathlineto{\pgfqpoint{4.533566in}{2.447845in}}%
\pgfpathlineto{\pgfqpoint{4.605828in}{2.438161in}}%
\pgfpathlineto{\pgfqpoint{4.678089in}{2.428478in}}%
\pgfpathlineto{\pgfqpoint{4.750350in}{2.418794in}}%
\pgfpathlineto{\pgfqpoint{4.822611in}{2.409110in}}%
\pgfpathlineto{\pgfqpoint{4.894872in}{2.399426in}}%
\pgfpathlineto{\pgfqpoint{4.967133in}{2.389742in}}%
\pgfpathlineto{\pgfqpoint{5.039394in}{2.380058in}}%
\pgfpathlineto{\pgfqpoint{5.111655in}{2.370374in}}%
\pgfpathlineto{\pgfqpoint{5.183916in}{2.360691in}}%
\pgfpathlineto{\pgfqpoint{5.256177in}{2.351007in}}%
\pgfpathlineto{\pgfqpoint{5.328438in}{2.341323in}}%
\pgfpathlineto{\pgfqpoint{5.400699in}{2.331639in}}%
\pgfpathlineto{\pgfqpoint{5.472960in}{2.321955in}}%
\pgfpathlineto{\pgfqpoint{5.545221in}{2.312271in}}%
\pgfpathlineto{\pgfqpoint{5.617483in}{2.302587in}}%
\pgfpathlineto{\pgfqpoint{5.689744in}{2.292904in}}%
\pgfpathlineto{\pgfqpoint{5.762005in}{2.283220in}}%
\pgfpathlineto{\pgfqpoint{5.834266in}{2.273536in}}%
\pgfpathlineto{\pgfqpoint{5.906527in}{2.263852in}}%
\pgfpathlineto{\pgfqpoint{5.978788in}{2.254168in}}%
\pgfpathlineto{\pgfqpoint{6.051049in}{2.244484in}}%
\pgfpathlineto{\pgfqpoint{6.123310in}{2.234800in}}%
\pgfpathlineto{\pgfqpoint{6.195571in}{2.225117in}}%
\pgfpathlineto{\pgfqpoint{6.267832in}{2.215433in}}%
\pgfpathlineto{\pgfqpoint{6.340093in}{2.205749in}}%
\pgfpathlineto{\pgfqpoint{6.412354in}{2.196065in}}%
\pgfpathlineto{\pgfqpoint{6.484615in}{2.186381in}}%
\pgfpathlineto{\pgfqpoint{6.556876in}{2.176697in}}%
\pgfpathlineto{\pgfqpoint{6.629138in}{2.167013in}}%
\pgfpathlineto{\pgfqpoint{6.701399in}{2.157330in}}%
\pgfpathlineto{\pgfqpoint{6.773660in}{2.147646in}}%
\pgfpathlineto{\pgfqpoint{6.918182in}{2.128278in}}%
\pgfpathlineto{\pgfqpoint{6.918182in}{1.529475in}}%
\pgfusepath{stroke}%
\end{pgfscope}%
\begin{pgfscope}%
\pgfpathrectangle{\pgfqpoint{1.000000in}{0.330000in}}{\pgfqpoint{6.200000in}{2.310000in}}%
\pgfusepath{clip}%
\pgfsetrectcap%
\pgfsetroundjoin%
\pgfsetlinewidth{1.505625pt}%
\definecolor{currentstroke}{rgb}{0.737255,0.741176,0.133333}%
\pgfsetstrokecolor{currentstroke}%
\pgfsetdash{}{0pt}%
\pgfpathmoveto{\pgfqpoint{1.281818in}{1.529475in}}%
\pgfpathlineto{\pgfqpoint{1.281818in}{1.039792in}}%
\pgfpathlineto{\pgfqpoint{1.354079in}{1.032853in}}%
\pgfpathlineto{\pgfqpoint{1.426340in}{1.025915in}}%
\pgfpathlineto{\pgfqpoint{1.498601in}{1.018976in}}%
\pgfpathlineto{\pgfqpoint{1.570862in}{1.012038in}}%
\pgfpathlineto{\pgfqpoint{1.643124in}{1.005100in}}%
\pgfpathlineto{\pgfqpoint{1.715385in}{0.998161in}}%
\pgfpathlineto{\pgfqpoint{1.787646in}{0.991223in}}%
\pgfpathlineto{\pgfqpoint{1.859907in}{0.984284in}}%
\pgfpathlineto{\pgfqpoint{1.932168in}{0.977346in}}%
\pgfpathlineto{\pgfqpoint{2.004429in}{0.970407in}}%
\pgfpathlineto{\pgfqpoint{2.076690in}{0.963469in}}%
\pgfpathlineto{\pgfqpoint{2.148951in}{0.956531in}}%
\pgfpathlineto{\pgfqpoint{2.221212in}{0.949592in}}%
\pgfpathlineto{\pgfqpoint{2.293473in}{0.942654in}}%
\pgfpathlineto{\pgfqpoint{2.365734in}{0.935715in}}%
\pgfpathlineto{\pgfqpoint{2.437995in}{0.928777in}}%
\pgfpathlineto{\pgfqpoint{2.510256in}{0.921838in}}%
\pgfpathlineto{\pgfqpoint{2.582517in}{0.914900in}}%
\pgfpathlineto{\pgfqpoint{2.654779in}{0.907962in}}%
\pgfpathlineto{\pgfqpoint{2.727040in}{0.901023in}}%
\pgfpathlineto{\pgfqpoint{2.799301in}{0.894085in}}%
\pgfpathlineto{\pgfqpoint{2.871562in}{0.887146in}}%
\pgfpathlineto{\pgfqpoint{2.943823in}{0.880208in}}%
\pgfpathlineto{\pgfqpoint{3.016084in}{0.873269in}}%
\pgfpathlineto{\pgfqpoint{3.088345in}{0.866331in}}%
\pgfpathlineto{\pgfqpoint{3.160606in}{0.859393in}}%
\pgfpathlineto{\pgfqpoint{3.232867in}{0.852454in}}%
\pgfpathlineto{\pgfqpoint{3.305128in}{0.845516in}}%
\pgfpathlineto{\pgfqpoint{3.377389in}{0.838577in}}%
\pgfpathlineto{\pgfqpoint{3.449650in}{0.831639in}}%
\pgfpathlineto{\pgfqpoint{3.521911in}{0.824700in}}%
\pgfpathlineto{\pgfqpoint{3.594172in}{0.817762in}}%
\pgfpathlineto{\pgfqpoint{3.666434in}{0.810823in}}%
\pgfpathlineto{\pgfqpoint{3.738695in}{0.803885in}}%
\pgfpathlineto{\pgfqpoint{3.883217in}{2.173715in}}%
\pgfpathlineto{\pgfqpoint{3.955478in}{2.168025in}}%
\pgfpathlineto{\pgfqpoint{4.027739in}{2.162334in}}%
\pgfpathlineto{\pgfqpoint{4.100000in}{2.156644in}}%
\pgfpathlineto{\pgfqpoint{4.172261in}{2.150953in}}%
\pgfpathlineto{\pgfqpoint{4.244522in}{2.145263in}}%
\pgfpathlineto{\pgfqpoint{4.316783in}{2.139572in}}%
\pgfpathlineto{\pgfqpoint{4.389044in}{2.133882in}}%
\pgfpathlineto{\pgfqpoint{4.461305in}{2.128191in}}%
\pgfpathlineto{\pgfqpoint{4.533566in}{2.122501in}}%
\pgfpathlineto{\pgfqpoint{4.605828in}{2.116810in}}%
\pgfpathlineto{\pgfqpoint{4.678089in}{2.111120in}}%
\pgfpathlineto{\pgfqpoint{4.750350in}{2.105429in}}%
\pgfpathlineto{\pgfqpoint{4.822611in}{2.099739in}}%
\pgfpathlineto{\pgfqpoint{4.894872in}{2.094048in}}%
\pgfpathlineto{\pgfqpoint{4.967133in}{2.088358in}}%
\pgfpathlineto{\pgfqpoint{5.039394in}{2.082667in}}%
\pgfpathlineto{\pgfqpoint{5.111655in}{2.076977in}}%
\pgfpathlineto{\pgfqpoint{5.183916in}{2.071286in}}%
\pgfpathlineto{\pgfqpoint{5.256177in}{2.065596in}}%
\pgfpathlineto{\pgfqpoint{5.328438in}{2.059905in}}%
\pgfpathlineto{\pgfqpoint{5.400699in}{2.054215in}}%
\pgfpathlineto{\pgfqpoint{5.472960in}{2.048524in}}%
\pgfpathlineto{\pgfqpoint{5.545221in}{2.042834in}}%
\pgfpathlineto{\pgfqpoint{5.617483in}{2.037143in}}%
\pgfpathlineto{\pgfqpoint{5.689744in}{2.031453in}}%
\pgfpathlineto{\pgfqpoint{5.762005in}{2.025762in}}%
\pgfpathlineto{\pgfqpoint{5.834266in}{2.020072in}}%
\pgfpathlineto{\pgfqpoint{5.906527in}{2.014381in}}%
\pgfpathlineto{\pgfqpoint{5.978788in}{2.008690in}}%
\pgfpathlineto{\pgfqpoint{6.051049in}{2.003000in}}%
\pgfpathlineto{\pgfqpoint{6.123310in}{1.997309in}}%
\pgfpathlineto{\pgfqpoint{6.195571in}{1.991619in}}%
\pgfpathlineto{\pgfqpoint{6.267832in}{1.985928in}}%
\pgfpathlineto{\pgfqpoint{6.340093in}{1.980238in}}%
\pgfpathlineto{\pgfqpoint{6.412354in}{1.974547in}}%
\pgfpathlineto{\pgfqpoint{6.484615in}{1.968857in}}%
\pgfpathlineto{\pgfqpoint{6.556876in}{1.963166in}}%
\pgfpathlineto{\pgfqpoint{6.629138in}{1.957476in}}%
\pgfpathlineto{\pgfqpoint{6.701399in}{1.951785in}}%
\pgfpathlineto{\pgfqpoint{6.773660in}{1.946095in}}%
\pgfpathlineto{\pgfqpoint{6.918182in}{1.934714in}}%
\pgfpathlineto{\pgfqpoint{6.918182in}{1.529475in}}%
\pgfusepath{stroke}%
\end{pgfscope}%
\begin{pgfscope}%
\pgfpathrectangle{\pgfqpoint{1.000000in}{0.330000in}}{\pgfqpoint{6.200000in}{2.310000in}}%
\pgfusepath{clip}%
\pgfsetrectcap%
\pgfsetroundjoin%
\pgfsetlinewidth{1.505625pt}%
\definecolor{currentstroke}{rgb}{0.090196,0.745098,0.811765}%
\pgfsetstrokecolor{currentstroke}%
\pgfsetdash{}{0pt}%
\pgfpathmoveto{\pgfqpoint{1.281818in}{1.529475in}}%
\pgfpathlineto{\pgfqpoint{1.281818in}{0.771739in}}%
\pgfpathlineto{\pgfqpoint{1.354079in}{0.762055in}}%
\pgfpathlineto{\pgfqpoint{1.426340in}{0.752371in}}%
\pgfpathlineto{\pgfqpoint{1.498601in}{0.742687in}}%
\pgfpathlineto{\pgfqpoint{1.570862in}{0.733003in}}%
\pgfpathlineto{\pgfqpoint{1.643124in}{0.723319in}}%
\pgfpathlineto{\pgfqpoint{1.715385in}{0.713636in}}%
\pgfpathlineto{\pgfqpoint{1.787646in}{0.703952in}}%
\pgfpathlineto{\pgfqpoint{1.859907in}{0.694268in}}%
\pgfpathlineto{\pgfqpoint{1.932168in}{0.684584in}}%
\pgfpathlineto{\pgfqpoint{2.004429in}{0.674900in}}%
\pgfpathlineto{\pgfqpoint{2.076690in}{0.665216in}}%
\pgfpathlineto{\pgfqpoint{2.148951in}{0.655532in}}%
\pgfpathlineto{\pgfqpoint{2.221212in}{0.645849in}}%
\pgfpathlineto{\pgfqpoint{2.293473in}{0.636165in}}%
\pgfpathlineto{\pgfqpoint{2.365734in}{0.626481in}}%
\pgfpathlineto{\pgfqpoint{2.437995in}{0.616797in}}%
\pgfpathlineto{\pgfqpoint{2.510256in}{0.607113in}}%
\pgfpathlineto{\pgfqpoint{2.582517in}{0.597429in}}%
\pgfpathlineto{\pgfqpoint{2.654779in}{0.587745in}}%
\pgfpathlineto{\pgfqpoint{2.727040in}{0.578062in}}%
\pgfpathlineto{\pgfqpoint{2.799301in}{0.568378in}}%
\pgfpathlineto{\pgfqpoint{2.871562in}{0.558694in}}%
\pgfpathlineto{\pgfqpoint{2.943823in}{0.549010in}}%
\pgfpathlineto{\pgfqpoint{3.016084in}{0.539326in}}%
\pgfpathlineto{\pgfqpoint{3.088345in}{0.529642in}}%
\pgfpathlineto{\pgfqpoint{3.160606in}{0.519958in}}%
\pgfpathlineto{\pgfqpoint{3.232867in}{0.510275in}}%
\pgfpathlineto{\pgfqpoint{3.305128in}{0.500591in}}%
\pgfpathlineto{\pgfqpoint{3.377389in}{0.490907in}}%
\pgfpathlineto{\pgfqpoint{3.449650in}{0.481223in}}%
\pgfpathlineto{\pgfqpoint{3.521911in}{0.471539in}}%
\pgfpathlineto{\pgfqpoint{3.594172in}{0.461855in}}%
\pgfpathlineto{\pgfqpoint{3.666434in}{0.452171in}}%
\pgfpathlineto{\pgfqpoint{3.738695in}{0.442488in}}%
\pgfpathlineto{\pgfqpoint{3.883217in}{2.528582in}}%
\pgfpathlineto{\pgfqpoint{3.955478in}{2.518898in}}%
\pgfpathlineto{\pgfqpoint{4.027739in}{2.509214in}}%
\pgfpathlineto{\pgfqpoint{4.100000in}{2.499531in}}%
\pgfpathlineto{\pgfqpoint{4.172261in}{2.489847in}}%
\pgfpathlineto{\pgfqpoint{4.244522in}{2.480163in}}%
\pgfpathlineto{\pgfqpoint{4.316783in}{2.470479in}}%
\pgfpathlineto{\pgfqpoint{4.389044in}{2.460795in}}%
\pgfpathlineto{\pgfqpoint{4.461305in}{2.451111in}}%
\pgfpathlineto{\pgfqpoint{4.533566in}{2.441427in}}%
\pgfpathlineto{\pgfqpoint{4.605828in}{2.431744in}}%
\pgfpathlineto{\pgfqpoint{4.678089in}{2.422060in}}%
\pgfpathlineto{\pgfqpoint{4.750350in}{2.412376in}}%
\pgfpathlineto{\pgfqpoint{4.822611in}{2.402692in}}%
\pgfpathlineto{\pgfqpoint{4.894872in}{2.393008in}}%
\pgfpathlineto{\pgfqpoint{4.967133in}{2.383324in}}%
\pgfpathlineto{\pgfqpoint{5.039394in}{2.373640in}}%
\pgfpathlineto{\pgfqpoint{5.111655in}{2.363957in}}%
\pgfpathlineto{\pgfqpoint{5.183916in}{2.354273in}}%
\pgfpathlineto{\pgfqpoint{5.256177in}{2.344589in}}%
\pgfpathlineto{\pgfqpoint{5.328438in}{2.334905in}}%
\pgfpathlineto{\pgfqpoint{5.400699in}{2.325221in}}%
\pgfpathlineto{\pgfqpoint{5.472960in}{2.315537in}}%
\pgfpathlineto{\pgfqpoint{5.545221in}{2.305853in}}%
\pgfpathlineto{\pgfqpoint{5.617483in}{2.296170in}}%
\pgfpathlineto{\pgfqpoint{5.689744in}{2.286486in}}%
\pgfpathlineto{\pgfqpoint{5.762005in}{2.276802in}}%
\pgfpathlineto{\pgfqpoint{5.834266in}{2.267118in}}%
\pgfpathlineto{\pgfqpoint{5.906527in}{2.257434in}}%
\pgfpathlineto{\pgfqpoint{5.978788in}{2.247750in}}%
\pgfpathlineto{\pgfqpoint{6.051049in}{2.238066in}}%
\pgfpathlineto{\pgfqpoint{6.123310in}{2.228383in}}%
\pgfpathlineto{\pgfqpoint{6.195571in}{2.218699in}}%
\pgfpathlineto{\pgfqpoint{6.267832in}{2.209015in}}%
\pgfpathlineto{\pgfqpoint{6.340093in}{2.199331in}}%
\pgfpathlineto{\pgfqpoint{6.412354in}{2.189647in}}%
\pgfpathlineto{\pgfqpoint{6.484615in}{2.179963in}}%
\pgfpathlineto{\pgfqpoint{6.556876in}{2.170279in}}%
\pgfpathlineto{\pgfqpoint{6.629138in}{2.160596in}}%
\pgfpathlineto{\pgfqpoint{6.701399in}{2.150912in}}%
\pgfpathlineto{\pgfqpoint{6.773660in}{2.141228in}}%
\pgfpathlineto{\pgfqpoint{6.918182in}{2.121860in}}%
\pgfpathlineto{\pgfqpoint{6.918182in}{1.529475in}}%
\pgfusepath{stroke}%
\end{pgfscope}%
\begin{pgfscope}%
\pgfpathrectangle{\pgfqpoint{1.000000in}{0.330000in}}{\pgfqpoint{6.200000in}{2.310000in}}%
\pgfusepath{clip}%
\pgfsetrectcap%
\pgfsetroundjoin%
\pgfsetlinewidth{1.505625pt}%
\definecolor{currentstroke}{rgb}{0.121569,0.466667,0.705882}%
\pgfsetstrokecolor{currentstroke}%
\pgfsetdash{}{0pt}%
\pgfpathmoveto{\pgfqpoint{1.281818in}{1.529475in}}%
\pgfpathlineto{\pgfqpoint{1.281818in}{1.120657in}}%
\pgfpathlineto{\pgfqpoint{1.354079in}{1.114018in}}%
\pgfpathlineto{\pgfqpoint{1.426340in}{1.107379in}}%
\pgfpathlineto{\pgfqpoint{1.498601in}{1.100740in}}%
\pgfpathlineto{\pgfqpoint{1.570862in}{1.094101in}}%
\pgfpathlineto{\pgfqpoint{1.643124in}{1.087462in}}%
\pgfpathlineto{\pgfqpoint{1.715385in}{1.080823in}}%
\pgfpathlineto{\pgfqpoint{1.787646in}{1.074184in}}%
\pgfpathlineto{\pgfqpoint{1.859907in}{1.067546in}}%
\pgfpathlineto{\pgfqpoint{1.932168in}{1.060907in}}%
\pgfpathlineto{\pgfqpoint{2.004429in}{1.054268in}}%
\pgfpathlineto{\pgfqpoint{2.076690in}{1.047629in}}%
\pgfpathlineto{\pgfqpoint{2.148951in}{1.040990in}}%
\pgfpathlineto{\pgfqpoint{2.221212in}{1.034351in}}%
\pgfpathlineto{\pgfqpoint{2.293473in}{1.027712in}}%
\pgfpathlineto{\pgfqpoint{2.365734in}{1.021073in}}%
\pgfpathlineto{\pgfqpoint{2.437995in}{1.014434in}}%
\pgfpathlineto{\pgfqpoint{2.510256in}{1.007795in}}%
\pgfpathlineto{\pgfqpoint{2.582517in}{1.001156in}}%
\pgfpathlineto{\pgfqpoint{2.654779in}{0.994517in}}%
\pgfpathlineto{\pgfqpoint{2.727040in}{0.987878in}}%
\pgfpathlineto{\pgfqpoint{2.799301in}{0.981239in}}%
\pgfpathlineto{\pgfqpoint{2.871562in}{0.974600in}}%
\pgfpathlineto{\pgfqpoint{2.943823in}{0.967962in}}%
\pgfpathlineto{\pgfqpoint{3.016084in}{0.961323in}}%
\pgfpathlineto{\pgfqpoint{3.088345in}{0.954684in}}%
\pgfpathlineto{\pgfqpoint{3.160606in}{0.948045in}}%
\pgfpathlineto{\pgfqpoint{3.232867in}{0.941406in}}%
\pgfpathlineto{\pgfqpoint{3.305128in}{0.934767in}}%
\pgfpathlineto{\pgfqpoint{3.377389in}{0.928128in}}%
\pgfpathlineto{\pgfqpoint{3.449650in}{0.921489in}}%
\pgfpathlineto{\pgfqpoint{3.521911in}{0.914850in}}%
\pgfpathlineto{\pgfqpoint{3.594172in}{0.908211in}}%
\pgfpathlineto{\pgfqpoint{3.666434in}{0.901572in}}%
\pgfpathlineto{\pgfqpoint{3.738695in}{0.894933in}}%
\pgfpathlineto{\pgfqpoint{3.883217in}{2.119578in}}%
\pgfpathlineto{\pgfqpoint{3.955478in}{2.112939in}}%
\pgfpathlineto{\pgfqpoint{4.027739in}{2.106300in}}%
\pgfpathlineto{\pgfqpoint{4.100000in}{2.099661in}}%
\pgfpathlineto{\pgfqpoint{4.172261in}{2.093022in}}%
\pgfpathlineto{\pgfqpoint{4.244522in}{2.086383in}}%
\pgfpathlineto{\pgfqpoint{4.316783in}{2.079744in}}%
\pgfpathlineto{\pgfqpoint{4.389044in}{2.073105in}}%
\pgfpathlineto{\pgfqpoint{4.461305in}{2.066466in}}%
\pgfpathlineto{\pgfqpoint{4.533566in}{2.059827in}}%
\pgfpathlineto{\pgfqpoint{4.605828in}{2.053189in}}%
\pgfpathlineto{\pgfqpoint{4.678089in}{2.046550in}}%
\pgfpathlineto{\pgfqpoint{4.750350in}{2.039911in}}%
\pgfpathlineto{\pgfqpoint{4.822611in}{2.033272in}}%
\pgfpathlineto{\pgfqpoint{4.894872in}{2.026633in}}%
\pgfpathlineto{\pgfqpoint{4.967133in}{2.019994in}}%
\pgfpathlineto{\pgfqpoint{5.039394in}{2.013355in}}%
\pgfpathlineto{\pgfqpoint{5.111655in}{2.006716in}}%
\pgfpathlineto{\pgfqpoint{5.183916in}{2.000077in}}%
\pgfpathlineto{\pgfqpoint{5.256177in}{1.993438in}}%
\pgfpathlineto{\pgfqpoint{5.328438in}{1.986799in}}%
\pgfpathlineto{\pgfqpoint{5.400699in}{1.980160in}}%
\pgfpathlineto{\pgfqpoint{5.472960in}{1.973521in}}%
\pgfpathlineto{\pgfqpoint{5.545221in}{1.966882in}}%
\pgfpathlineto{\pgfqpoint{5.617483in}{1.960243in}}%
\pgfpathlineto{\pgfqpoint{5.689744in}{1.953605in}}%
\pgfpathlineto{\pgfqpoint{5.762005in}{1.946966in}}%
\pgfpathlineto{\pgfqpoint{5.834266in}{1.940327in}}%
\pgfpathlineto{\pgfqpoint{5.906527in}{1.933688in}}%
\pgfpathlineto{\pgfqpoint{5.978788in}{1.927049in}}%
\pgfpathlineto{\pgfqpoint{6.051049in}{1.920410in}}%
\pgfpathlineto{\pgfqpoint{6.123310in}{1.913771in}}%
\pgfpathlineto{\pgfqpoint{6.195571in}{1.907132in}}%
\pgfpathlineto{\pgfqpoint{6.267832in}{1.900493in}}%
\pgfpathlineto{\pgfqpoint{6.340093in}{1.893854in}}%
\pgfpathlineto{\pgfqpoint{6.412354in}{1.887215in}}%
\pgfpathlineto{\pgfqpoint{6.484615in}{1.880576in}}%
\pgfpathlineto{\pgfqpoint{6.556876in}{1.873937in}}%
\pgfpathlineto{\pgfqpoint{6.629138in}{1.867298in}}%
\pgfpathlineto{\pgfqpoint{6.701399in}{1.860659in}}%
\pgfpathlineto{\pgfqpoint{6.773660in}{1.854021in}}%
\pgfpathlineto{\pgfqpoint{6.918182in}{1.840743in}}%
\pgfpathlineto{\pgfqpoint{6.918182in}{1.529475in}}%
\pgfusepath{stroke}%
\end{pgfscope}%
\begin{pgfscope}%
\pgfpathrectangle{\pgfqpoint{1.000000in}{0.330000in}}{\pgfqpoint{6.200000in}{2.310000in}}%
\pgfusepath{clip}%
\pgfsetrectcap%
\pgfsetroundjoin%
\pgfsetlinewidth{1.505625pt}%
\definecolor{currentstroke}{rgb}{1.000000,0.498039,0.054902}%
\pgfsetstrokecolor{currentstroke}%
\pgfsetdash{}{0pt}%
\pgfpathmoveto{\pgfqpoint{1.281818in}{1.529475in}}%
\pgfpathlineto{\pgfqpoint{1.281818in}{1.167080in}}%
\pgfpathlineto{\pgfqpoint{1.354079in}{1.161389in}}%
\pgfpathlineto{\pgfqpoint{1.426340in}{1.155699in}}%
\pgfpathlineto{\pgfqpoint{1.498601in}{1.150008in}}%
\pgfpathlineto{\pgfqpoint{1.570862in}{1.144318in}}%
\pgfpathlineto{\pgfqpoint{1.643124in}{1.138627in}}%
\pgfpathlineto{\pgfqpoint{1.715385in}{1.132937in}}%
\pgfpathlineto{\pgfqpoint{1.787646in}{1.127246in}}%
\pgfpathlineto{\pgfqpoint{1.859907in}{1.121555in}}%
\pgfpathlineto{\pgfqpoint{1.932168in}{1.115865in}}%
\pgfpathlineto{\pgfqpoint{2.004429in}{1.110174in}}%
\pgfpathlineto{\pgfqpoint{2.076690in}{1.104484in}}%
\pgfpathlineto{\pgfqpoint{2.148951in}{1.098793in}}%
\pgfpathlineto{\pgfqpoint{2.221212in}{1.093103in}}%
\pgfpathlineto{\pgfqpoint{2.293473in}{1.087412in}}%
\pgfpathlineto{\pgfqpoint{2.365734in}{1.081722in}}%
\pgfpathlineto{\pgfqpoint{2.437995in}{1.076031in}}%
\pgfpathlineto{\pgfqpoint{2.510256in}{1.070341in}}%
\pgfpathlineto{\pgfqpoint{2.582517in}{1.064650in}}%
\pgfpathlineto{\pgfqpoint{2.654779in}{1.058960in}}%
\pgfpathlineto{\pgfqpoint{2.727040in}{1.053269in}}%
\pgfpathlineto{\pgfqpoint{2.799301in}{1.047579in}}%
\pgfpathlineto{\pgfqpoint{2.871562in}{1.041888in}}%
\pgfpathlineto{\pgfqpoint{2.943823in}{1.036198in}}%
\pgfpathlineto{\pgfqpoint{3.016084in}{1.030507in}}%
\pgfpathlineto{\pgfqpoint{3.088345in}{1.024817in}}%
\pgfpathlineto{\pgfqpoint{3.160606in}{1.019126in}}%
\pgfpathlineto{\pgfqpoint{3.232867in}{1.013436in}}%
\pgfpathlineto{\pgfqpoint{3.305128in}{1.007745in}}%
\pgfpathlineto{\pgfqpoint{3.377389in}{1.002055in}}%
\pgfpathlineto{\pgfqpoint{3.449650in}{0.996364in}}%
\pgfpathlineto{\pgfqpoint{3.521911in}{0.990674in}}%
\pgfpathlineto{\pgfqpoint{3.594172in}{0.984983in}}%
\pgfpathlineto{\pgfqpoint{3.666434in}{0.979293in}}%
\pgfpathlineto{\pgfqpoint{3.738695in}{0.973602in}}%
\pgfpathlineto{\pgfqpoint{3.883217in}{2.045546in}}%
\pgfpathlineto{\pgfqpoint{3.955478in}{2.039856in}}%
\pgfpathlineto{\pgfqpoint{4.027739in}{2.034165in}}%
\pgfpathlineto{\pgfqpoint{4.100000in}{2.028475in}}%
\pgfpathlineto{\pgfqpoint{4.172261in}{2.022784in}}%
\pgfpathlineto{\pgfqpoint{4.244522in}{2.017094in}}%
\pgfpathlineto{\pgfqpoint{4.316783in}{2.011403in}}%
\pgfpathlineto{\pgfqpoint{4.389044in}{2.005713in}}%
\pgfpathlineto{\pgfqpoint{4.461305in}{2.000022in}}%
\pgfpathlineto{\pgfqpoint{4.533566in}{1.994331in}}%
\pgfpathlineto{\pgfqpoint{4.605828in}{1.988641in}}%
\pgfpathlineto{\pgfqpoint{4.678089in}{1.982950in}}%
\pgfpathlineto{\pgfqpoint{4.750350in}{1.977260in}}%
\pgfpathlineto{\pgfqpoint{4.822611in}{1.971569in}}%
\pgfpathlineto{\pgfqpoint{4.894872in}{1.965879in}}%
\pgfpathlineto{\pgfqpoint{4.967133in}{1.960188in}}%
\pgfpathlineto{\pgfqpoint{5.039394in}{1.954498in}}%
\pgfpathlineto{\pgfqpoint{5.111655in}{1.948807in}}%
\pgfpathlineto{\pgfqpoint{5.183916in}{1.943117in}}%
\pgfpathlineto{\pgfqpoint{5.256177in}{1.937426in}}%
\pgfpathlineto{\pgfqpoint{5.328438in}{1.931736in}}%
\pgfpathlineto{\pgfqpoint{5.400699in}{1.926045in}}%
\pgfpathlineto{\pgfqpoint{5.472960in}{1.920355in}}%
\pgfpathlineto{\pgfqpoint{5.545221in}{1.914664in}}%
\pgfpathlineto{\pgfqpoint{5.617483in}{1.908974in}}%
\pgfpathlineto{\pgfqpoint{5.689744in}{1.903283in}}%
\pgfpathlineto{\pgfqpoint{5.762005in}{1.897593in}}%
\pgfpathlineto{\pgfqpoint{5.834266in}{1.891902in}}%
\pgfpathlineto{\pgfqpoint{5.906527in}{1.886212in}}%
\pgfpathlineto{\pgfqpoint{5.978788in}{1.880521in}}%
\pgfpathlineto{\pgfqpoint{6.051049in}{1.874831in}}%
\pgfpathlineto{\pgfqpoint{6.123310in}{1.869140in}}%
\pgfpathlineto{\pgfqpoint{6.195571in}{1.863450in}}%
\pgfpathlineto{\pgfqpoint{6.267832in}{1.857759in}}%
\pgfpathlineto{\pgfqpoint{6.340093in}{1.852069in}}%
\pgfpathlineto{\pgfqpoint{6.412354in}{1.846378in}}%
\pgfpathlineto{\pgfqpoint{6.484615in}{1.840688in}}%
\pgfpathlineto{\pgfqpoint{6.556876in}{1.834997in}}%
\pgfpathlineto{\pgfqpoint{6.629138in}{1.829307in}}%
\pgfpathlineto{\pgfqpoint{6.701399in}{1.823616in}}%
\pgfpathlineto{\pgfqpoint{6.773660in}{1.817926in}}%
\pgfpathlineto{\pgfqpoint{6.918182in}{1.806545in}}%
\pgfpathlineto{\pgfqpoint{6.918182in}{1.529475in}}%
\pgfusepath{stroke}%
\end{pgfscope}%
\begin{pgfscope}%
\pgfpathrectangle{\pgfqpoint{1.000000in}{0.330000in}}{\pgfqpoint{6.200000in}{2.310000in}}%
\pgfusepath{clip}%
\pgfsetrectcap%
\pgfsetroundjoin%
\pgfsetlinewidth{1.505625pt}%
\definecolor{currentstroke}{rgb}{0.172549,0.627451,0.172549}%
\pgfsetstrokecolor{currentstroke}%
\pgfsetdash{}{0pt}%
\pgfpathmoveto{\pgfqpoint{1.281818in}{1.529475in}}%
\pgfpathlineto{\pgfqpoint{1.281818in}{1.044284in}}%
\pgfpathlineto{\pgfqpoint{1.354079in}{1.037346in}}%
\pgfpathlineto{\pgfqpoint{1.426340in}{1.030407in}}%
\pgfpathlineto{\pgfqpoint{1.498601in}{1.023469in}}%
\pgfpathlineto{\pgfqpoint{1.570862in}{1.016531in}}%
\pgfpathlineto{\pgfqpoint{1.643124in}{1.009592in}}%
\pgfpathlineto{\pgfqpoint{1.715385in}{1.002654in}}%
\pgfpathlineto{\pgfqpoint{1.787646in}{0.995715in}}%
\pgfpathlineto{\pgfqpoint{1.859907in}{0.988777in}}%
\pgfpathlineto{\pgfqpoint{1.932168in}{0.981838in}}%
\pgfpathlineto{\pgfqpoint{2.004429in}{0.974900in}}%
\pgfpathlineto{\pgfqpoint{2.076690in}{0.967962in}}%
\pgfpathlineto{\pgfqpoint{2.148951in}{0.961023in}}%
\pgfpathlineto{\pgfqpoint{2.221212in}{0.954085in}}%
\pgfpathlineto{\pgfqpoint{2.293473in}{0.947146in}}%
\pgfpathlineto{\pgfqpoint{2.365734in}{0.940208in}}%
\pgfpathlineto{\pgfqpoint{2.437995in}{0.933269in}}%
\pgfpathlineto{\pgfqpoint{2.510256in}{0.926331in}}%
\pgfpathlineto{\pgfqpoint{2.582517in}{0.919392in}}%
\pgfpathlineto{\pgfqpoint{2.654779in}{0.912454in}}%
\pgfpathlineto{\pgfqpoint{2.727040in}{0.905516in}}%
\pgfpathlineto{\pgfqpoint{2.799301in}{0.898577in}}%
\pgfpathlineto{\pgfqpoint{2.871562in}{0.891639in}}%
\pgfpathlineto{\pgfqpoint{2.943823in}{0.884700in}}%
\pgfpathlineto{\pgfqpoint{3.016084in}{0.877762in}}%
\pgfpathlineto{\pgfqpoint{3.088345in}{0.870823in}}%
\pgfpathlineto{\pgfqpoint{3.160606in}{0.863885in}}%
\pgfpathlineto{\pgfqpoint{3.232867in}{0.856947in}}%
\pgfpathlineto{\pgfqpoint{3.305128in}{0.850008in}}%
\pgfpathlineto{\pgfqpoint{3.377389in}{0.843070in}}%
\pgfpathlineto{\pgfqpoint{3.449650in}{0.836131in}}%
\pgfpathlineto{\pgfqpoint{3.521911in}{0.829193in}}%
\pgfpathlineto{\pgfqpoint{3.594172in}{0.822254in}}%
\pgfpathlineto{\pgfqpoint{3.666434in}{0.815316in}}%
\pgfpathlineto{\pgfqpoint{3.738695in}{0.808378in}}%
\pgfpathlineto{\pgfqpoint{3.883217in}{2.195853in}}%
\pgfpathlineto{\pgfqpoint{3.955478in}{2.188915in}}%
\pgfpathlineto{\pgfqpoint{4.027739in}{2.181976in}}%
\pgfpathlineto{\pgfqpoint{4.100000in}{2.175038in}}%
\pgfpathlineto{\pgfqpoint{4.172261in}{2.168099in}}%
\pgfpathlineto{\pgfqpoint{4.244522in}{2.161161in}}%
\pgfpathlineto{\pgfqpoint{4.316783in}{2.154222in}}%
\pgfpathlineto{\pgfqpoint{4.389044in}{2.147284in}}%
\pgfpathlineto{\pgfqpoint{4.461305in}{2.140346in}}%
\pgfpathlineto{\pgfqpoint{4.533566in}{2.133407in}}%
\pgfpathlineto{\pgfqpoint{4.605828in}{2.126469in}}%
\pgfpathlineto{\pgfqpoint{4.678089in}{2.119530in}}%
\pgfpathlineto{\pgfqpoint{4.750350in}{2.112592in}}%
\pgfpathlineto{\pgfqpoint{4.822611in}{2.105653in}}%
\pgfpathlineto{\pgfqpoint{4.894872in}{2.098715in}}%
\pgfpathlineto{\pgfqpoint{4.967133in}{2.091777in}}%
\pgfpathlineto{\pgfqpoint{5.039394in}{2.084838in}}%
\pgfpathlineto{\pgfqpoint{5.111655in}{2.077900in}}%
\pgfpathlineto{\pgfqpoint{5.183916in}{2.070961in}}%
\pgfpathlineto{\pgfqpoint{5.256177in}{2.064023in}}%
\pgfpathlineto{\pgfqpoint{5.328438in}{2.057084in}}%
\pgfpathlineto{\pgfqpoint{5.400699in}{2.050146in}}%
\pgfpathlineto{\pgfqpoint{5.472960in}{2.043208in}}%
\pgfpathlineto{\pgfqpoint{5.545221in}{2.036269in}}%
\pgfpathlineto{\pgfqpoint{5.617483in}{2.029331in}}%
\pgfpathlineto{\pgfqpoint{5.689744in}{2.022392in}}%
\pgfpathlineto{\pgfqpoint{5.762005in}{2.015454in}}%
\pgfpathlineto{\pgfqpoint{5.834266in}{2.008515in}}%
\pgfpathlineto{\pgfqpoint{5.906527in}{2.001577in}}%
\pgfpathlineto{\pgfqpoint{5.978788in}{1.994639in}}%
\pgfpathlineto{\pgfqpoint{6.051049in}{1.987700in}}%
\pgfpathlineto{\pgfqpoint{6.123310in}{1.980762in}}%
\pgfpathlineto{\pgfqpoint{6.195571in}{1.973823in}}%
\pgfpathlineto{\pgfqpoint{6.267832in}{1.966885in}}%
\pgfpathlineto{\pgfqpoint{6.340093in}{1.959946in}}%
\pgfpathlineto{\pgfqpoint{6.412354in}{1.953008in}}%
\pgfpathlineto{\pgfqpoint{6.484615in}{1.946069in}}%
\pgfpathlineto{\pgfqpoint{6.556876in}{1.939131in}}%
\pgfpathlineto{\pgfqpoint{6.629138in}{1.932193in}}%
\pgfpathlineto{\pgfqpoint{6.701399in}{1.925254in}}%
\pgfpathlineto{\pgfqpoint{6.773660in}{1.918316in}}%
\pgfpathlineto{\pgfqpoint{6.918182in}{1.904439in}}%
\pgfpathlineto{\pgfqpoint{6.918182in}{1.529475in}}%
\pgfusepath{stroke}%
\end{pgfscope}%
\begin{pgfscope}%
\pgfpathrectangle{\pgfqpoint{1.000000in}{0.330000in}}{\pgfqpoint{6.200000in}{2.310000in}}%
\pgfusepath{clip}%
\pgfsetrectcap%
\pgfsetroundjoin%
\pgfsetlinewidth{1.505625pt}%
\definecolor{currentstroke}{rgb}{0.839216,0.152941,0.156863}%
\pgfsetstrokecolor{currentstroke}%
\pgfsetdash{}{0pt}%
\pgfpathmoveto{\pgfqpoint{1.281818in}{1.529475in}}%
\pgfpathlineto{\pgfqpoint{1.281818in}{1.325066in}}%
\pgfpathlineto{\pgfqpoint{1.354079in}{1.321747in}}%
\pgfpathlineto{\pgfqpoint{1.426340in}{1.318427in}}%
\pgfpathlineto{\pgfqpoint{1.498601in}{1.315108in}}%
\pgfpathlineto{\pgfqpoint{1.570862in}{1.311788in}}%
\pgfpathlineto{\pgfqpoint{1.643124in}{1.308469in}}%
\pgfpathlineto{\pgfqpoint{1.715385in}{1.305149in}}%
\pgfpathlineto{\pgfqpoint{1.787646in}{1.301830in}}%
\pgfpathlineto{\pgfqpoint{1.859907in}{1.298510in}}%
\pgfpathlineto{\pgfqpoint{1.932168in}{1.295191in}}%
\pgfpathlineto{\pgfqpoint{2.004429in}{1.291872in}}%
\pgfpathlineto{\pgfqpoint{2.076690in}{1.288552in}}%
\pgfpathlineto{\pgfqpoint{2.148951in}{1.285233in}}%
\pgfpathlineto{\pgfqpoint{2.221212in}{1.281913in}}%
\pgfpathlineto{\pgfqpoint{2.293473in}{1.278594in}}%
\pgfpathlineto{\pgfqpoint{2.365734in}{1.275274in}}%
\pgfpathlineto{\pgfqpoint{2.437995in}{1.271955in}}%
\pgfpathlineto{\pgfqpoint{2.510256in}{1.268635in}}%
\pgfpathlineto{\pgfqpoint{2.582517in}{1.265316in}}%
\pgfpathlineto{\pgfqpoint{2.654779in}{1.261996in}}%
\pgfpathlineto{\pgfqpoint{2.727040in}{1.258677in}}%
\pgfpathlineto{\pgfqpoint{2.799301in}{1.255357in}}%
\pgfpathlineto{\pgfqpoint{2.871562in}{1.252038in}}%
\pgfpathlineto{\pgfqpoint{2.943823in}{1.248718in}}%
\pgfpathlineto{\pgfqpoint{3.016084in}{1.245399in}}%
\pgfpathlineto{\pgfqpoint{3.088345in}{1.242080in}}%
\pgfpathlineto{\pgfqpoint{3.160606in}{1.238760in}}%
\pgfpathlineto{\pgfqpoint{3.232867in}{1.235441in}}%
\pgfpathlineto{\pgfqpoint{3.305128in}{1.232121in}}%
\pgfpathlineto{\pgfqpoint{3.377389in}{1.228802in}}%
\pgfpathlineto{\pgfqpoint{3.449650in}{1.225482in}}%
\pgfpathlineto{\pgfqpoint{3.521911in}{1.222163in}}%
\pgfpathlineto{\pgfqpoint{3.594172in}{1.218843in}}%
\pgfpathlineto{\pgfqpoint{3.666434in}{1.215524in}}%
\pgfpathlineto{\pgfqpoint{3.738695in}{1.212204in}}%
\pgfpathlineto{\pgfqpoint{3.883217in}{1.824527in}}%
\pgfpathlineto{\pgfqpoint{3.955478in}{1.821207in}}%
\pgfpathlineto{\pgfqpoint{4.027739in}{1.817888in}}%
\pgfpathlineto{\pgfqpoint{4.100000in}{1.814568in}}%
\pgfpathlineto{\pgfqpoint{4.172261in}{1.811249in}}%
\pgfpathlineto{\pgfqpoint{4.244522in}{1.807929in}}%
\pgfpathlineto{\pgfqpoint{4.316783in}{1.804610in}}%
\pgfpathlineto{\pgfqpoint{4.389044in}{1.801290in}}%
\pgfpathlineto{\pgfqpoint{4.461305in}{1.797971in}}%
\pgfpathlineto{\pgfqpoint{4.533566in}{1.794651in}}%
\pgfpathlineto{\pgfqpoint{4.605828in}{1.791332in}}%
\pgfpathlineto{\pgfqpoint{4.678089in}{1.788013in}}%
\pgfpathlineto{\pgfqpoint{4.750350in}{1.784693in}}%
\pgfpathlineto{\pgfqpoint{4.822611in}{1.781374in}}%
\pgfpathlineto{\pgfqpoint{4.894872in}{1.778054in}}%
\pgfpathlineto{\pgfqpoint{4.967133in}{1.774735in}}%
\pgfpathlineto{\pgfqpoint{5.039394in}{1.771415in}}%
\pgfpathlineto{\pgfqpoint{5.111655in}{1.768096in}}%
\pgfpathlineto{\pgfqpoint{5.183916in}{1.764776in}}%
\pgfpathlineto{\pgfqpoint{5.256177in}{1.761457in}}%
\pgfpathlineto{\pgfqpoint{5.328438in}{1.758137in}}%
\pgfpathlineto{\pgfqpoint{5.400699in}{1.754818in}}%
\pgfpathlineto{\pgfqpoint{5.472960in}{1.751498in}}%
\pgfpathlineto{\pgfqpoint{5.545221in}{1.748179in}}%
\pgfpathlineto{\pgfqpoint{5.617483in}{1.744859in}}%
\pgfpathlineto{\pgfqpoint{5.689744in}{1.741540in}}%
\pgfpathlineto{\pgfqpoint{5.762005in}{1.738221in}}%
\pgfpathlineto{\pgfqpoint{5.834266in}{1.734901in}}%
\pgfpathlineto{\pgfqpoint{5.906527in}{1.731582in}}%
\pgfpathlineto{\pgfqpoint{5.978788in}{1.728262in}}%
\pgfpathlineto{\pgfqpoint{6.051049in}{1.724943in}}%
\pgfpathlineto{\pgfqpoint{6.123310in}{1.721623in}}%
\pgfpathlineto{\pgfqpoint{6.195571in}{1.718304in}}%
\pgfpathlineto{\pgfqpoint{6.267832in}{1.714984in}}%
\pgfpathlineto{\pgfqpoint{6.340093in}{1.711665in}}%
\pgfpathlineto{\pgfqpoint{6.412354in}{1.708345in}}%
\pgfpathlineto{\pgfqpoint{6.484615in}{1.705026in}}%
\pgfpathlineto{\pgfqpoint{6.556876in}{1.701706in}}%
\pgfpathlineto{\pgfqpoint{6.629138in}{1.698387in}}%
\pgfpathlineto{\pgfqpoint{6.701399in}{1.695067in}}%
\pgfpathlineto{\pgfqpoint{6.773660in}{1.691748in}}%
\pgfpathlineto{\pgfqpoint{6.918182in}{1.685109in}}%
\pgfpathlineto{\pgfqpoint{6.918182in}{1.529475in}}%
\pgfusepath{stroke}%
\end{pgfscope}%
\begin{pgfscope}%
\pgfpathrectangle{\pgfqpoint{1.000000in}{0.330000in}}{\pgfqpoint{6.200000in}{2.310000in}}%
\pgfusepath{clip}%
\pgfsetrectcap%
\pgfsetroundjoin%
\pgfsetlinewidth{1.505625pt}%
\definecolor{currentstroke}{rgb}{0.580392,0.403922,0.741176}%
\pgfsetstrokecolor{currentstroke}%
\pgfsetdash{}{0pt}%
\pgfpathmoveto{\pgfqpoint{1.281818in}{1.529475in}}%
\pgfpathlineto{\pgfqpoint{1.281818in}{1.039792in}}%
\pgfpathlineto{\pgfqpoint{1.354079in}{1.032853in}}%
\pgfpathlineto{\pgfqpoint{1.426340in}{1.025915in}}%
\pgfpathlineto{\pgfqpoint{1.498601in}{1.018976in}}%
\pgfpathlineto{\pgfqpoint{1.570862in}{1.012038in}}%
\pgfpathlineto{\pgfqpoint{1.643124in}{1.005100in}}%
\pgfpathlineto{\pgfqpoint{1.715385in}{0.998161in}}%
\pgfpathlineto{\pgfqpoint{1.787646in}{0.991223in}}%
\pgfpathlineto{\pgfqpoint{1.859907in}{0.984284in}}%
\pgfpathlineto{\pgfqpoint{1.932168in}{0.977346in}}%
\pgfpathlineto{\pgfqpoint{2.004429in}{0.970407in}}%
\pgfpathlineto{\pgfqpoint{2.076690in}{0.963469in}}%
\pgfpathlineto{\pgfqpoint{2.148951in}{0.956531in}}%
\pgfpathlineto{\pgfqpoint{2.221212in}{0.949592in}}%
\pgfpathlineto{\pgfqpoint{2.293473in}{0.942654in}}%
\pgfpathlineto{\pgfqpoint{2.365734in}{0.935715in}}%
\pgfpathlineto{\pgfqpoint{2.437995in}{0.928777in}}%
\pgfpathlineto{\pgfqpoint{2.510256in}{0.921838in}}%
\pgfpathlineto{\pgfqpoint{2.582517in}{0.914900in}}%
\pgfpathlineto{\pgfqpoint{2.654779in}{0.907962in}}%
\pgfpathlineto{\pgfqpoint{2.727040in}{0.901023in}}%
\pgfpathlineto{\pgfqpoint{2.799301in}{0.894085in}}%
\pgfpathlineto{\pgfqpoint{2.871562in}{0.887146in}}%
\pgfpathlineto{\pgfqpoint{2.943823in}{0.880208in}}%
\pgfpathlineto{\pgfqpoint{3.016084in}{0.873269in}}%
\pgfpathlineto{\pgfqpoint{3.088345in}{0.866331in}}%
\pgfpathlineto{\pgfqpoint{3.160606in}{0.859393in}}%
\pgfpathlineto{\pgfqpoint{3.232867in}{0.852454in}}%
\pgfpathlineto{\pgfqpoint{3.305128in}{0.845516in}}%
\pgfpathlineto{\pgfqpoint{3.377389in}{0.838577in}}%
\pgfpathlineto{\pgfqpoint{3.449650in}{0.831639in}}%
\pgfpathlineto{\pgfqpoint{3.521911in}{0.824700in}}%
\pgfpathlineto{\pgfqpoint{3.594172in}{0.817762in}}%
\pgfpathlineto{\pgfqpoint{3.666434in}{0.810823in}}%
\pgfpathlineto{\pgfqpoint{3.738695in}{0.803885in}}%
\pgfpathlineto{\pgfqpoint{3.883217in}{2.199704in}}%
\pgfpathlineto{\pgfqpoint{3.955478in}{2.192765in}}%
\pgfpathlineto{\pgfqpoint{4.027739in}{2.185827in}}%
\pgfpathlineto{\pgfqpoint{4.100000in}{2.178888in}}%
\pgfpathlineto{\pgfqpoint{4.172261in}{2.171950in}}%
\pgfpathlineto{\pgfqpoint{4.244522in}{2.165012in}}%
\pgfpathlineto{\pgfqpoint{4.316783in}{2.158073in}}%
\pgfpathlineto{\pgfqpoint{4.389044in}{2.151135in}}%
\pgfpathlineto{\pgfqpoint{4.461305in}{2.144196in}}%
\pgfpathlineto{\pgfqpoint{4.533566in}{2.137258in}}%
\pgfpathlineto{\pgfqpoint{4.605828in}{2.130319in}}%
\pgfpathlineto{\pgfqpoint{4.678089in}{2.123381in}}%
\pgfpathlineto{\pgfqpoint{4.750350in}{2.116443in}}%
\pgfpathlineto{\pgfqpoint{4.822611in}{2.109504in}}%
\pgfpathlineto{\pgfqpoint{4.894872in}{2.102566in}}%
\pgfpathlineto{\pgfqpoint{4.967133in}{2.095627in}}%
\pgfpathlineto{\pgfqpoint{5.039394in}{2.088689in}}%
\pgfpathlineto{\pgfqpoint{5.111655in}{2.081750in}}%
\pgfpathlineto{\pgfqpoint{5.183916in}{2.074812in}}%
\pgfpathlineto{\pgfqpoint{5.256177in}{2.067874in}}%
\pgfpathlineto{\pgfqpoint{5.328438in}{2.060935in}}%
\pgfpathlineto{\pgfqpoint{5.400699in}{2.053997in}}%
\pgfpathlineto{\pgfqpoint{5.472960in}{2.047058in}}%
\pgfpathlineto{\pgfqpoint{5.545221in}{2.040120in}}%
\pgfpathlineto{\pgfqpoint{5.617483in}{2.033181in}}%
\pgfpathlineto{\pgfqpoint{5.689744in}{2.026243in}}%
\pgfpathlineto{\pgfqpoint{5.762005in}{2.019305in}}%
\pgfpathlineto{\pgfqpoint{5.834266in}{2.012366in}}%
\pgfpathlineto{\pgfqpoint{5.906527in}{2.005428in}}%
\pgfpathlineto{\pgfqpoint{5.978788in}{1.998489in}}%
\pgfpathlineto{\pgfqpoint{6.051049in}{1.991551in}}%
\pgfpathlineto{\pgfqpoint{6.123310in}{1.984612in}}%
\pgfpathlineto{\pgfqpoint{6.195571in}{1.977674in}}%
\pgfpathlineto{\pgfqpoint{6.267832in}{1.970735in}}%
\pgfpathlineto{\pgfqpoint{6.340093in}{1.963797in}}%
\pgfpathlineto{\pgfqpoint{6.412354in}{1.956859in}}%
\pgfpathlineto{\pgfqpoint{6.484615in}{1.949920in}}%
\pgfpathlineto{\pgfqpoint{6.556876in}{1.942982in}}%
\pgfpathlineto{\pgfqpoint{6.629138in}{1.936043in}}%
\pgfpathlineto{\pgfqpoint{6.701399in}{1.929105in}}%
\pgfpathlineto{\pgfqpoint{6.773660in}{1.922166in}}%
\pgfpathlineto{\pgfqpoint{6.918182in}{1.908290in}}%
\pgfpathlineto{\pgfqpoint{6.918182in}{1.529475in}}%
\pgfusepath{stroke}%
\end{pgfscope}%
\begin{pgfscope}%
\pgfpathrectangle{\pgfqpoint{1.000000in}{0.330000in}}{\pgfqpoint{6.200000in}{2.310000in}}%
\pgfusepath{clip}%
\pgfsetrectcap%
\pgfsetroundjoin%
\pgfsetlinewidth{1.505625pt}%
\definecolor{currentstroke}{rgb}{0.549020,0.337255,0.294118}%
\pgfsetstrokecolor{currentstroke}%
\pgfsetdash{}{0pt}%
\pgfpathmoveto{\pgfqpoint{1.281818in}{1.529475in}}%
\pgfpathlineto{\pgfqpoint{1.281818in}{1.179060in}}%
\pgfpathlineto{\pgfqpoint{1.354079in}{1.173369in}}%
\pgfpathlineto{\pgfqpoint{1.426340in}{1.167679in}}%
\pgfpathlineto{\pgfqpoint{1.498601in}{1.161988in}}%
\pgfpathlineto{\pgfqpoint{1.570862in}{1.156298in}}%
\pgfpathlineto{\pgfqpoint{1.643124in}{1.150607in}}%
\pgfpathlineto{\pgfqpoint{1.715385in}{1.144917in}}%
\pgfpathlineto{\pgfqpoint{1.787646in}{1.139226in}}%
\pgfpathlineto{\pgfqpoint{1.859907in}{1.133536in}}%
\pgfpathlineto{\pgfqpoint{1.932168in}{1.127845in}}%
\pgfpathlineto{\pgfqpoint{2.004429in}{1.122154in}}%
\pgfpathlineto{\pgfqpoint{2.076690in}{1.116464in}}%
\pgfpathlineto{\pgfqpoint{2.148951in}{1.110773in}}%
\pgfpathlineto{\pgfqpoint{2.221212in}{1.105083in}}%
\pgfpathlineto{\pgfqpoint{2.293473in}{1.099392in}}%
\pgfpathlineto{\pgfqpoint{2.365734in}{1.093702in}}%
\pgfpathlineto{\pgfqpoint{2.437995in}{1.088011in}}%
\pgfpathlineto{\pgfqpoint{2.510256in}{1.082321in}}%
\pgfpathlineto{\pgfqpoint{2.582517in}{1.076630in}}%
\pgfpathlineto{\pgfqpoint{2.654779in}{1.070940in}}%
\pgfpathlineto{\pgfqpoint{2.727040in}{1.065249in}}%
\pgfpathlineto{\pgfqpoint{2.799301in}{1.059559in}}%
\pgfpathlineto{\pgfqpoint{2.871562in}{1.053868in}}%
\pgfpathlineto{\pgfqpoint{2.943823in}{1.048178in}}%
\pgfpathlineto{\pgfqpoint{3.016084in}{1.042487in}}%
\pgfpathlineto{\pgfqpoint{3.088345in}{1.036797in}}%
\pgfpathlineto{\pgfqpoint{3.160606in}{1.031106in}}%
\pgfpathlineto{\pgfqpoint{3.232867in}{1.025416in}}%
\pgfpathlineto{\pgfqpoint{3.305128in}{1.019725in}}%
\pgfpathlineto{\pgfqpoint{3.377389in}{1.014035in}}%
\pgfpathlineto{\pgfqpoint{3.449650in}{1.008344in}}%
\pgfpathlineto{\pgfqpoint{3.521911in}{1.002654in}}%
\pgfpathlineto{\pgfqpoint{3.594172in}{0.996963in}}%
\pgfpathlineto{\pgfqpoint{3.666434in}{0.991273in}}%
\pgfpathlineto{\pgfqpoint{3.738695in}{0.985582in}}%
\pgfpathlineto{\pgfqpoint{3.883217in}{2.035278in}}%
\pgfpathlineto{\pgfqpoint{3.955478in}{2.029587in}}%
\pgfpathlineto{\pgfqpoint{4.027739in}{2.023896in}}%
\pgfpathlineto{\pgfqpoint{4.100000in}{2.018206in}}%
\pgfpathlineto{\pgfqpoint{4.172261in}{2.012515in}}%
\pgfpathlineto{\pgfqpoint{4.244522in}{2.006825in}}%
\pgfpathlineto{\pgfqpoint{4.316783in}{2.001134in}}%
\pgfpathlineto{\pgfqpoint{4.389044in}{1.995444in}}%
\pgfpathlineto{\pgfqpoint{4.461305in}{1.989753in}}%
\pgfpathlineto{\pgfqpoint{4.533566in}{1.984063in}}%
\pgfpathlineto{\pgfqpoint{4.605828in}{1.978372in}}%
\pgfpathlineto{\pgfqpoint{4.678089in}{1.972682in}}%
\pgfpathlineto{\pgfqpoint{4.750350in}{1.966991in}}%
\pgfpathlineto{\pgfqpoint{4.822611in}{1.961301in}}%
\pgfpathlineto{\pgfqpoint{4.894872in}{1.955610in}}%
\pgfpathlineto{\pgfqpoint{4.967133in}{1.949920in}}%
\pgfpathlineto{\pgfqpoint{5.039394in}{1.944229in}}%
\pgfpathlineto{\pgfqpoint{5.111655in}{1.938539in}}%
\pgfpathlineto{\pgfqpoint{5.183916in}{1.932848in}}%
\pgfpathlineto{\pgfqpoint{5.256177in}{1.927158in}}%
\pgfpathlineto{\pgfqpoint{5.328438in}{1.921467in}}%
\pgfpathlineto{\pgfqpoint{5.400699in}{1.915777in}}%
\pgfpathlineto{\pgfqpoint{5.472960in}{1.910086in}}%
\pgfpathlineto{\pgfqpoint{5.545221in}{1.904396in}}%
\pgfpathlineto{\pgfqpoint{5.617483in}{1.898705in}}%
\pgfpathlineto{\pgfqpoint{5.689744in}{1.893015in}}%
\pgfpathlineto{\pgfqpoint{5.762005in}{1.887324in}}%
\pgfpathlineto{\pgfqpoint{5.834266in}{1.881634in}}%
\pgfpathlineto{\pgfqpoint{5.906527in}{1.875943in}}%
\pgfpathlineto{\pgfqpoint{5.978788in}{1.870253in}}%
\pgfpathlineto{\pgfqpoint{6.051049in}{1.864562in}}%
\pgfpathlineto{\pgfqpoint{6.123310in}{1.858872in}}%
\pgfpathlineto{\pgfqpoint{6.195571in}{1.853181in}}%
\pgfpathlineto{\pgfqpoint{6.267832in}{1.847491in}}%
\pgfpathlineto{\pgfqpoint{6.340093in}{1.841800in}}%
\pgfpathlineto{\pgfqpoint{6.412354in}{1.836110in}}%
\pgfpathlineto{\pgfqpoint{6.484615in}{1.830419in}}%
\pgfpathlineto{\pgfqpoint{6.556876in}{1.824728in}}%
\pgfpathlineto{\pgfqpoint{6.629138in}{1.819038in}}%
\pgfpathlineto{\pgfqpoint{6.701399in}{1.813347in}}%
\pgfpathlineto{\pgfqpoint{6.773660in}{1.807657in}}%
\pgfpathlineto{\pgfqpoint{6.918182in}{1.796276in}}%
\pgfpathlineto{\pgfqpoint{6.918182in}{1.529475in}}%
\pgfusepath{stroke}%
\end{pgfscope}%
\begin{pgfscope}%
\pgfpathrectangle{\pgfqpoint{1.000000in}{0.330000in}}{\pgfqpoint{6.200000in}{2.310000in}}%
\pgfusepath{clip}%
\pgfsetrectcap%
\pgfsetroundjoin%
\pgfsetlinewidth{1.505625pt}%
\definecolor{currentstroke}{rgb}{0.890196,0.466667,0.760784}%
\pgfsetstrokecolor{currentstroke}%
\pgfsetdash{}{0pt}%
\pgfpathmoveto{\pgfqpoint{1.281818in}{1.529475in}}%
\pgfpathlineto{\pgfqpoint{1.281818in}{1.179060in}}%
\pgfpathlineto{\pgfqpoint{1.354079in}{1.173369in}}%
\pgfpathlineto{\pgfqpoint{1.426340in}{1.167679in}}%
\pgfpathlineto{\pgfqpoint{1.498601in}{1.161988in}}%
\pgfpathlineto{\pgfqpoint{1.570862in}{1.156298in}}%
\pgfpathlineto{\pgfqpoint{1.643124in}{1.150607in}}%
\pgfpathlineto{\pgfqpoint{1.715385in}{1.144917in}}%
\pgfpathlineto{\pgfqpoint{1.787646in}{1.139226in}}%
\pgfpathlineto{\pgfqpoint{1.859907in}{1.133536in}}%
\pgfpathlineto{\pgfqpoint{1.932168in}{1.127845in}}%
\pgfpathlineto{\pgfqpoint{2.004429in}{1.122154in}}%
\pgfpathlineto{\pgfqpoint{2.076690in}{1.116464in}}%
\pgfpathlineto{\pgfqpoint{2.148951in}{1.110773in}}%
\pgfpathlineto{\pgfqpoint{2.221212in}{1.105083in}}%
\pgfpathlineto{\pgfqpoint{2.293473in}{1.099392in}}%
\pgfpathlineto{\pgfqpoint{2.365734in}{1.093702in}}%
\pgfpathlineto{\pgfqpoint{2.437995in}{1.088011in}}%
\pgfpathlineto{\pgfqpoint{2.510256in}{1.082321in}}%
\pgfpathlineto{\pgfqpoint{2.582517in}{1.076630in}}%
\pgfpathlineto{\pgfqpoint{2.654779in}{1.070940in}}%
\pgfpathlineto{\pgfqpoint{2.727040in}{1.065249in}}%
\pgfpathlineto{\pgfqpoint{2.799301in}{1.059559in}}%
\pgfpathlineto{\pgfqpoint{2.871562in}{1.053868in}}%
\pgfpathlineto{\pgfqpoint{2.943823in}{1.048178in}}%
\pgfpathlineto{\pgfqpoint{3.016084in}{1.042487in}}%
\pgfpathlineto{\pgfqpoint{3.088345in}{1.036797in}}%
\pgfpathlineto{\pgfqpoint{3.160606in}{1.031106in}}%
\pgfpathlineto{\pgfqpoint{3.232867in}{1.025416in}}%
\pgfpathlineto{\pgfqpoint{3.305128in}{1.019725in}}%
\pgfpathlineto{\pgfqpoint{3.377389in}{1.014035in}}%
\pgfpathlineto{\pgfqpoint{3.449650in}{1.008344in}}%
\pgfpathlineto{\pgfqpoint{3.521911in}{1.002654in}}%
\pgfpathlineto{\pgfqpoint{3.594172in}{0.996963in}}%
\pgfpathlineto{\pgfqpoint{3.666434in}{0.991273in}}%
\pgfpathlineto{\pgfqpoint{3.738695in}{0.985582in}}%
\pgfpathlineto{\pgfqpoint{3.883217in}{2.061266in}}%
\pgfpathlineto{\pgfqpoint{3.955478in}{2.054327in}}%
\pgfpathlineto{\pgfqpoint{4.027739in}{2.047389in}}%
\pgfpathlineto{\pgfqpoint{4.100000in}{2.040451in}}%
\pgfpathlineto{\pgfqpoint{4.172261in}{2.033512in}}%
\pgfpathlineto{\pgfqpoint{4.244522in}{2.026574in}}%
\pgfpathlineto{\pgfqpoint{4.316783in}{2.019635in}}%
\pgfpathlineto{\pgfqpoint{4.389044in}{2.012697in}}%
\pgfpathlineto{\pgfqpoint{4.461305in}{2.005758in}}%
\pgfpathlineto{\pgfqpoint{4.533566in}{1.998820in}}%
\pgfpathlineto{\pgfqpoint{4.605828in}{1.991882in}}%
\pgfpathlineto{\pgfqpoint{4.678089in}{1.984943in}}%
\pgfpathlineto{\pgfqpoint{4.750350in}{1.978005in}}%
\pgfpathlineto{\pgfqpoint{4.822611in}{1.971066in}}%
\pgfpathlineto{\pgfqpoint{4.894872in}{1.964128in}}%
\pgfpathlineto{\pgfqpoint{4.967133in}{1.957189in}}%
\pgfpathlineto{\pgfqpoint{5.039394in}{1.950251in}}%
\pgfpathlineto{\pgfqpoint{5.111655in}{1.943313in}}%
\pgfpathlineto{\pgfqpoint{5.183916in}{1.936374in}}%
\pgfpathlineto{\pgfqpoint{5.256177in}{1.929436in}}%
\pgfpathlineto{\pgfqpoint{5.328438in}{1.922497in}}%
\pgfpathlineto{\pgfqpoint{5.400699in}{1.915559in}}%
\pgfpathlineto{\pgfqpoint{5.472960in}{1.908620in}}%
\pgfpathlineto{\pgfqpoint{5.545221in}{1.901682in}}%
\pgfpathlineto{\pgfqpoint{5.617483in}{1.894744in}}%
\pgfpathlineto{\pgfqpoint{5.689744in}{1.887805in}}%
\pgfpathlineto{\pgfqpoint{5.762005in}{1.880867in}}%
\pgfpathlineto{\pgfqpoint{5.834266in}{1.873928in}}%
\pgfpathlineto{\pgfqpoint{5.906527in}{1.866990in}}%
\pgfpathlineto{\pgfqpoint{5.978788in}{1.860051in}}%
\pgfpathlineto{\pgfqpoint{6.051049in}{1.853113in}}%
\pgfpathlineto{\pgfqpoint{6.123310in}{1.846174in}}%
\pgfpathlineto{\pgfqpoint{6.195571in}{1.839236in}}%
\pgfpathlineto{\pgfqpoint{6.267832in}{1.832298in}}%
\pgfpathlineto{\pgfqpoint{6.340093in}{1.825359in}}%
\pgfpathlineto{\pgfqpoint{6.412354in}{1.818421in}}%
\pgfpathlineto{\pgfqpoint{6.484615in}{1.811482in}}%
\pgfpathlineto{\pgfqpoint{6.556876in}{1.804544in}}%
\pgfpathlineto{\pgfqpoint{6.629138in}{1.797605in}}%
\pgfpathlineto{\pgfqpoint{6.701399in}{1.790667in}}%
\pgfpathlineto{\pgfqpoint{6.773660in}{1.783729in}}%
\pgfpathlineto{\pgfqpoint{6.918182in}{1.769852in}}%
\pgfpathlineto{\pgfqpoint{6.918182in}{1.529475in}}%
\pgfusepath{stroke}%
\end{pgfscope}%
\begin{pgfscope}%
\pgfpathrectangle{\pgfqpoint{1.000000in}{0.330000in}}{\pgfqpoint{6.200000in}{2.310000in}}%
\pgfusepath{clip}%
\pgfsetrectcap%
\pgfsetroundjoin%
\pgfsetlinewidth{1.505625pt}%
\definecolor{currentstroke}{rgb}{0.498039,0.498039,0.498039}%
\pgfsetstrokecolor{currentstroke}%
\pgfsetdash{}{0pt}%
\pgfpathmoveto{\pgfqpoint{1.281818in}{1.529475in}}%
\pgfpathlineto{\pgfqpoint{1.281818in}{1.179060in}}%
\pgfpathlineto{\pgfqpoint{1.354079in}{1.173369in}}%
\pgfpathlineto{\pgfqpoint{1.426340in}{1.167679in}}%
\pgfpathlineto{\pgfqpoint{1.498601in}{1.161988in}}%
\pgfpathlineto{\pgfqpoint{1.570862in}{1.156298in}}%
\pgfpathlineto{\pgfqpoint{1.643124in}{1.150607in}}%
\pgfpathlineto{\pgfqpoint{1.715385in}{1.144917in}}%
\pgfpathlineto{\pgfqpoint{1.787646in}{1.139226in}}%
\pgfpathlineto{\pgfqpoint{1.859907in}{1.133536in}}%
\pgfpathlineto{\pgfqpoint{1.932168in}{1.127845in}}%
\pgfpathlineto{\pgfqpoint{2.004429in}{1.122154in}}%
\pgfpathlineto{\pgfqpoint{2.076690in}{1.116464in}}%
\pgfpathlineto{\pgfqpoint{2.148951in}{1.110773in}}%
\pgfpathlineto{\pgfqpoint{2.221212in}{1.105083in}}%
\pgfpathlineto{\pgfqpoint{2.293473in}{1.099392in}}%
\pgfpathlineto{\pgfqpoint{2.365734in}{1.093702in}}%
\pgfpathlineto{\pgfqpoint{2.437995in}{1.088011in}}%
\pgfpathlineto{\pgfqpoint{2.510256in}{1.082321in}}%
\pgfpathlineto{\pgfqpoint{2.582517in}{1.076630in}}%
\pgfpathlineto{\pgfqpoint{2.654779in}{1.070940in}}%
\pgfpathlineto{\pgfqpoint{2.727040in}{1.065249in}}%
\pgfpathlineto{\pgfqpoint{2.799301in}{1.059559in}}%
\pgfpathlineto{\pgfqpoint{2.871562in}{1.053868in}}%
\pgfpathlineto{\pgfqpoint{2.943823in}{1.048178in}}%
\pgfpathlineto{\pgfqpoint{3.016084in}{1.042487in}}%
\pgfpathlineto{\pgfqpoint{3.088345in}{1.036797in}}%
\pgfpathlineto{\pgfqpoint{3.160606in}{1.031106in}}%
\pgfpathlineto{\pgfqpoint{3.232867in}{1.025416in}}%
\pgfpathlineto{\pgfqpoint{3.305128in}{1.019725in}}%
\pgfpathlineto{\pgfqpoint{3.377389in}{1.014035in}}%
\pgfpathlineto{\pgfqpoint{3.449650in}{1.008344in}}%
\pgfpathlineto{\pgfqpoint{3.521911in}{1.002654in}}%
\pgfpathlineto{\pgfqpoint{3.594172in}{0.996963in}}%
\pgfpathlineto{\pgfqpoint{3.666434in}{0.991273in}}%
\pgfpathlineto{\pgfqpoint{3.738695in}{0.985582in}}%
\pgfpathlineto{\pgfqpoint{3.883217in}{2.061266in}}%
\pgfpathlineto{\pgfqpoint{3.955478in}{2.054327in}}%
\pgfpathlineto{\pgfqpoint{4.027739in}{2.047389in}}%
\pgfpathlineto{\pgfqpoint{4.100000in}{2.040451in}}%
\pgfpathlineto{\pgfqpoint{4.172261in}{2.033512in}}%
\pgfpathlineto{\pgfqpoint{4.244522in}{2.026574in}}%
\pgfpathlineto{\pgfqpoint{4.316783in}{2.019635in}}%
\pgfpathlineto{\pgfqpoint{4.389044in}{2.012697in}}%
\pgfpathlineto{\pgfqpoint{4.461305in}{2.005758in}}%
\pgfpathlineto{\pgfqpoint{4.533566in}{1.998820in}}%
\pgfpathlineto{\pgfqpoint{4.605828in}{1.991882in}}%
\pgfpathlineto{\pgfqpoint{4.678089in}{1.984943in}}%
\pgfpathlineto{\pgfqpoint{4.750350in}{1.978005in}}%
\pgfpathlineto{\pgfqpoint{4.822611in}{1.971066in}}%
\pgfpathlineto{\pgfqpoint{4.894872in}{1.964128in}}%
\pgfpathlineto{\pgfqpoint{4.967133in}{1.957189in}}%
\pgfpathlineto{\pgfqpoint{5.039394in}{1.950251in}}%
\pgfpathlineto{\pgfqpoint{5.111655in}{1.943313in}}%
\pgfpathlineto{\pgfqpoint{5.183916in}{1.936374in}}%
\pgfpathlineto{\pgfqpoint{5.256177in}{1.929436in}}%
\pgfpathlineto{\pgfqpoint{5.328438in}{1.922497in}}%
\pgfpathlineto{\pgfqpoint{5.400699in}{1.915559in}}%
\pgfpathlineto{\pgfqpoint{5.472960in}{1.908620in}}%
\pgfpathlineto{\pgfqpoint{5.545221in}{1.901682in}}%
\pgfpathlineto{\pgfqpoint{5.617483in}{1.894744in}}%
\pgfpathlineto{\pgfqpoint{5.689744in}{1.887805in}}%
\pgfpathlineto{\pgfqpoint{5.762005in}{1.880867in}}%
\pgfpathlineto{\pgfqpoint{5.834266in}{1.873928in}}%
\pgfpathlineto{\pgfqpoint{5.906527in}{1.866990in}}%
\pgfpathlineto{\pgfqpoint{5.978788in}{1.860051in}}%
\pgfpathlineto{\pgfqpoint{6.051049in}{1.853113in}}%
\pgfpathlineto{\pgfqpoint{6.123310in}{1.846174in}}%
\pgfpathlineto{\pgfqpoint{6.195571in}{1.839236in}}%
\pgfpathlineto{\pgfqpoint{6.267832in}{1.832298in}}%
\pgfpathlineto{\pgfqpoint{6.340093in}{1.825359in}}%
\pgfpathlineto{\pgfqpoint{6.412354in}{1.818421in}}%
\pgfpathlineto{\pgfqpoint{6.484615in}{1.811482in}}%
\pgfpathlineto{\pgfqpoint{6.556876in}{1.804544in}}%
\pgfpathlineto{\pgfqpoint{6.629138in}{1.797605in}}%
\pgfpathlineto{\pgfqpoint{6.701399in}{1.790667in}}%
\pgfpathlineto{\pgfqpoint{6.773660in}{1.783729in}}%
\pgfpathlineto{\pgfqpoint{6.918182in}{1.769852in}}%
\pgfpathlineto{\pgfqpoint{6.918182in}{1.529475in}}%
\pgfusepath{stroke}%
\end{pgfscope}%
\begin{pgfscope}%
\pgfpathrectangle{\pgfqpoint{1.000000in}{0.330000in}}{\pgfqpoint{6.200000in}{2.310000in}}%
\pgfusepath{clip}%
\pgfsetrectcap%
\pgfsetroundjoin%
\pgfsetlinewidth{1.505625pt}%
\definecolor{currentstroke}{rgb}{0.737255,0.741176,0.133333}%
\pgfsetstrokecolor{currentstroke}%
\pgfsetdash{}{0pt}%
\pgfpathmoveto{\pgfqpoint{1.281818in}{1.529475in}}%
\pgfpathlineto{\pgfqpoint{1.281818in}{1.266664in}}%
\pgfpathlineto{\pgfqpoint{1.354079in}{1.262396in}}%
\pgfpathlineto{\pgfqpoint{1.426340in}{1.258128in}}%
\pgfpathlineto{\pgfqpoint{1.498601in}{1.253860in}}%
\pgfpathlineto{\pgfqpoint{1.570862in}{1.249592in}}%
\pgfpathlineto{\pgfqpoint{1.643124in}{1.245324in}}%
\pgfpathlineto{\pgfqpoint{1.715385in}{1.241056in}}%
\pgfpathlineto{\pgfqpoint{1.787646in}{1.236788in}}%
\pgfpathlineto{\pgfqpoint{1.859907in}{1.232520in}}%
\pgfpathlineto{\pgfqpoint{1.932168in}{1.228253in}}%
\pgfpathlineto{\pgfqpoint{2.004429in}{1.223985in}}%
\pgfpathlineto{\pgfqpoint{2.076690in}{1.219717in}}%
\pgfpathlineto{\pgfqpoint{2.148951in}{1.215449in}}%
\pgfpathlineto{\pgfqpoint{2.221212in}{1.211181in}}%
\pgfpathlineto{\pgfqpoint{2.293473in}{1.206913in}}%
\pgfpathlineto{\pgfqpoint{2.365734in}{1.202645in}}%
\pgfpathlineto{\pgfqpoint{2.437995in}{1.198377in}}%
\pgfpathlineto{\pgfqpoint{2.510256in}{1.194110in}}%
\pgfpathlineto{\pgfqpoint{2.582517in}{1.189842in}}%
\pgfpathlineto{\pgfqpoint{2.654779in}{1.185574in}}%
\pgfpathlineto{\pgfqpoint{2.727040in}{1.181306in}}%
\pgfpathlineto{\pgfqpoint{2.799301in}{1.177038in}}%
\pgfpathlineto{\pgfqpoint{2.871562in}{1.172770in}}%
\pgfpathlineto{\pgfqpoint{2.943823in}{1.168502in}}%
\pgfpathlineto{\pgfqpoint{3.016084in}{1.164234in}}%
\pgfpathlineto{\pgfqpoint{3.088345in}{1.159966in}}%
\pgfpathlineto{\pgfqpoint{3.160606in}{1.155699in}}%
\pgfpathlineto{\pgfqpoint{3.232867in}{1.151431in}}%
\pgfpathlineto{\pgfqpoint{3.305128in}{1.147163in}}%
\pgfpathlineto{\pgfqpoint{3.377389in}{1.142895in}}%
\pgfpathlineto{\pgfqpoint{3.449650in}{1.138627in}}%
\pgfpathlineto{\pgfqpoint{3.521911in}{1.134359in}}%
\pgfpathlineto{\pgfqpoint{3.594172in}{1.130091in}}%
\pgfpathlineto{\pgfqpoint{3.666434in}{1.125823in}}%
\pgfpathlineto{\pgfqpoint{3.738695in}{1.121555in}}%
\pgfpathlineto{\pgfqpoint{3.883217in}{1.908827in}}%
\pgfpathlineto{\pgfqpoint{3.955478in}{1.904559in}}%
\pgfpathlineto{\pgfqpoint{4.027739in}{1.900291in}}%
\pgfpathlineto{\pgfqpoint{4.100000in}{1.896023in}}%
\pgfpathlineto{\pgfqpoint{4.172261in}{1.891755in}}%
\pgfpathlineto{\pgfqpoint{4.244522in}{1.887488in}}%
\pgfpathlineto{\pgfqpoint{4.316783in}{1.883220in}}%
\pgfpathlineto{\pgfqpoint{4.389044in}{1.878952in}}%
\pgfpathlineto{\pgfqpoint{4.461305in}{1.874684in}}%
\pgfpathlineto{\pgfqpoint{4.533566in}{1.870416in}}%
\pgfpathlineto{\pgfqpoint{4.605828in}{1.866148in}}%
\pgfpathlineto{\pgfqpoint{4.678089in}{1.861880in}}%
\pgfpathlineto{\pgfqpoint{4.750350in}{1.857612in}}%
\pgfpathlineto{\pgfqpoint{4.822611in}{1.853344in}}%
\pgfpathlineto{\pgfqpoint{4.894872in}{1.849077in}}%
\pgfpathlineto{\pgfqpoint{4.967133in}{1.844809in}}%
\pgfpathlineto{\pgfqpoint{5.039394in}{1.840541in}}%
\pgfpathlineto{\pgfqpoint{5.111655in}{1.836273in}}%
\pgfpathlineto{\pgfqpoint{5.183916in}{1.832005in}}%
\pgfpathlineto{\pgfqpoint{5.256177in}{1.827737in}}%
\pgfpathlineto{\pgfqpoint{5.328438in}{1.823469in}}%
\pgfpathlineto{\pgfqpoint{5.400699in}{1.819201in}}%
\pgfpathlineto{\pgfqpoint{5.472960in}{1.814934in}}%
\pgfpathlineto{\pgfqpoint{5.545221in}{1.810666in}}%
\pgfpathlineto{\pgfqpoint{5.617483in}{1.806398in}}%
\pgfpathlineto{\pgfqpoint{5.689744in}{1.802130in}}%
\pgfpathlineto{\pgfqpoint{5.762005in}{1.797862in}}%
\pgfpathlineto{\pgfqpoint{5.834266in}{1.793594in}}%
\pgfpathlineto{\pgfqpoint{5.906527in}{1.789326in}}%
\pgfpathlineto{\pgfqpoint{5.978788in}{1.785058in}}%
\pgfpathlineto{\pgfqpoint{6.051049in}{1.780790in}}%
\pgfpathlineto{\pgfqpoint{6.123310in}{1.776523in}}%
\pgfpathlineto{\pgfqpoint{6.195571in}{1.772255in}}%
\pgfpathlineto{\pgfqpoint{6.267832in}{1.767987in}}%
\pgfpathlineto{\pgfqpoint{6.340093in}{1.763719in}}%
\pgfpathlineto{\pgfqpoint{6.412354in}{1.759451in}}%
\pgfpathlineto{\pgfqpoint{6.484615in}{1.755183in}}%
\pgfpathlineto{\pgfqpoint{6.556876in}{1.750915in}}%
\pgfpathlineto{\pgfqpoint{6.629138in}{1.746647in}}%
\pgfpathlineto{\pgfqpoint{6.701399in}{1.742379in}}%
\pgfpathlineto{\pgfqpoint{6.773660in}{1.738112in}}%
\pgfpathlineto{\pgfqpoint{6.918182in}{1.729576in}}%
\pgfpathlineto{\pgfqpoint{6.918182in}{1.529475in}}%
\pgfusepath{stroke}%
\end{pgfscope}%
\begin{pgfscope}%
\pgfsetroundcap%
\pgfsetroundjoin%
\pgfsetlinewidth{1.003750pt}%
\definecolor{currentstroke}{rgb}{0.000000,0.000000,0.000000}%
\pgfsetstrokecolor{currentstroke}%
\pgfsetdash{}{0pt}%
\pgfpathmoveto{\pgfqpoint{4.229520in}{0.435000in}}%
\pgfpathquadraticcurveto{\pgfqpoint{3.997993in}{0.435000in}}{\pgfqpoint{3.766465in}{0.435000in}}%
\pgfusepath{stroke}%
\end{pgfscope}%
\begin{pgfscope}%
\pgfsetbuttcap%
\pgfsetmiterjoin%
\definecolor{currentfill}{rgb}{0.800000,0.800000,0.800000}%
\pgfsetfillcolor{currentfill}%
\pgfsetlinewidth{1.003750pt}%
\definecolor{currentstroke}{rgb}{0.000000,0.000000,0.000000}%
\pgfsetstrokecolor{currentstroke}%
\pgfsetdash{}{0pt}%
\pgfpathmoveto{\pgfqpoint{4.287224in}{0.338549in}}%
\pgfpathcurveto{\pgfqpoint{4.321947in}{0.303827in}}{\pgfqpoint{5.127584in}{0.303827in}}{\pgfqpoint{5.162306in}{0.338549in}}%
\pgfpathcurveto{\pgfqpoint{5.197028in}{0.373272in}}{\pgfqpoint{5.197028in}{0.496728in}}{\pgfqpoint{5.162306in}{0.531451in}}%
\pgfpathcurveto{\pgfqpoint{5.127584in}{0.566173in}}{\pgfqpoint{4.321947in}{0.566173in}}{\pgfqpoint{4.287224in}{0.531451in}}%
\pgfpathcurveto{\pgfqpoint{4.252502in}{0.496728in}}{\pgfqpoint{4.252502in}{0.373272in}}{\pgfqpoint{4.287224in}{0.338549in}}%
\pgfpathclose%
\pgfusepath{stroke,fill}%
\end{pgfscope}%
\begin{pgfscope}%
\definecolor{textcolor}{rgb}{0.000000,0.000000,0.000000}%
\pgfsetstrokecolor{textcolor}%
\pgfsetfillcolor{textcolor}%
\pgftext[x=5.127584in,y=0.435000in,right,]{\color{textcolor}\rmfamily\fontsize{10.000000}{12.000000}\selectfont \(\displaystyle V_u =\) -1.5 kip}%
\end{pgfscope}%
\begin{pgfscope}%
\pgfsetbuttcap%
\pgfsetmiterjoin%
\definecolor{currentfill}{rgb}{0.800000,0.800000,0.800000}%
\pgfsetfillcolor{currentfill}%
\pgfsetlinewidth{1.003750pt}%
\definecolor{currentstroke}{rgb}{0.000000,0.000000,0.000000}%
\pgfsetstrokecolor{currentstroke}%
\pgfsetdash{}{0pt}%
\pgfpathmoveto{\pgfqpoint{0.965278in}{0.358599in}}%
\pgfpathcurveto{\pgfqpoint{1.000000in}{0.323877in}}{\pgfqpoint{4.339899in}{0.323877in}}{\pgfqpoint{4.374622in}{0.358599in}}%
\pgfpathcurveto{\pgfqpoint{4.409344in}{0.393321in}}{\pgfqpoint{4.409344in}{0.668784in}}{\pgfqpoint{4.374622in}{0.703506in}}%
\pgfpathcurveto{\pgfqpoint{4.339899in}{0.738228in}}{\pgfqpoint{1.000000in}{0.738228in}}{\pgfqpoint{0.965278in}{0.703506in}}%
\pgfpathcurveto{\pgfqpoint{0.930556in}{0.668784in}}{\pgfqpoint{0.930556in}{0.393321in}}{\pgfqpoint{0.965278in}{0.358599in}}%
\pgfpathclose%
\pgfusepath{stroke,fill}%
\end{pgfscope}%
\begin{pgfscope}%
\definecolor{textcolor}{rgb}{0.000000,0.000000,0.000000}%
\pgfsetstrokecolor{textcolor}%
\pgfsetfillcolor{textcolor}%
\pgftext[x=1.000000in, y=0.580049in, left, base]{\color{textcolor}\rmfamily\fontsize{10.000000}{12.000000}\selectfont Max combo: 1.2D + 1.6Lr0 + 1.0L0 + 1.6Lr1 + 1.0L1}%
\end{pgfscope}%
\begin{pgfscope}%
\definecolor{textcolor}{rgb}{0.000000,0.000000,0.000000}%
\pgfsetstrokecolor{textcolor}%
\pgfsetfillcolor{textcolor}%
\pgftext[x=1.000000in, y=0.428043in, left, base]{\color{textcolor}\rmfamily\fontsize{10.000000}{12.000000}\selectfont ASCE7-16 Sec. 2.3.1 (LC 3)}%
\end{pgfscope}%
\end{pgfpicture}%
\makeatother%
\endgroup%

\end{center}
\caption{Shear Demand Envelope}
\end{figure}
C\textsubscript{v1}, the web shear strength coefficient, is calculated per AISC/ANSI 360-16 Eq. G2-2 as follows, based on the ratio of the clear distance between flanges to web thickness:
\begin{flalign*}
\frac{h}{t_w} = \frac{8.8 {\color{darkBlue}{\mathbf{ \; in}}}}{0.2 {\color{darkBlue}{\mathbf{ \; in}}}} = \mathbf{46.6 } <= 2.24\sqrt{\frac{E}{F_y}} = 2.24\sqrt{\frac{29000 {\color{darkBlue}{\mathbf{ \; ksi}}}}{50 {\color{darkBlue}{\mathbf{ \; ksi}}}}} = \mathbf{53.9 } \rightarrow C_{v1} = \mathbf{1.0}
\end{flalign*}
\textphi\textsubscript{v}, the resistance factor for shear, is calculated per AISC/ANSI 360-16 {\S}G2.1.a as follows:
\begin{flalign*}
\frac{h}{t_w} = \frac{8.8 {\color{darkBlue}{\mathbf{ \; in}}}}{0.19 {\color{darkBlue}{\mathbf{ \; in}}}} = \mathbf{46.6 } <= 2.24\cdot \sqrt{\frac{E}{F_y}} = 2.24\cdot \sqrt{\frac{29000 {\color{darkBlue}{\mathbf{ \; ksi}}}}{50 {\color{darkBlue}{\mathbf{ \; ksi}}}}} = \mathbf{53.9} \rightarrow \phi_v = \mathbf{1.0}
\end{flalign*}
\textphi\textsubscript{v}V\textsubscript{n}, the design shear strength, is calculated per AISC/ANSI 360-16 Eq. G2-1 as follows:
\begin{flalign*}
\phi_v V_n = 0.6\cdot F_y \cdot A_w \cdot C_{v1}  = 0.6\cdot 50 {\color{darkBlue}{\mathbf{ \; ksi}}} \cdot 1.88 {\color{darkBlue}{\mathbf{ \; {\color{darkBlue}{\mathbf{ \; in}}}^{2}}}} \cdot 1.0  = \mathbf{56.3 {\color{darkBlue}{\mathbf{ \; kip}}}}
\end{flalign*}
\vspace{-26pt}
{\setlength{\mathindent}{0cm}
\begin{flalign*}
\mathbf{|V_u| = 1.5 {\color{darkBlue}{\mathbf{ \; kip}}}  \;  < \phi_v \cdot V_n = 56.3 {\color{darkBlue}{\mathbf{ \; kip}}}  \;  (DCR = 0.03 - OK)}
\end{flalign*}
%	----------------------------- DEFLECTION CHECK -------------------------------
\section{Deflection Check}
\begin{figure}[H]
\begin{center}
%% Creator: Matplotlib, PGF backend
%%
%% To include the figure in your LaTeX document, write
%%   \input{<filename>.pgf}
%%
%% Make sure the required packages are loaded in your preamble
%%   \usepackage{pgf}
%%
%% Figures using additional raster images can only be included by \input if
%% they are in the same directory as the main LaTeX file. For loading figures
%% from other directories you can use the `import` package
%%   \usepackage{import}
%%
%% and then include the figures with
%%   \import{<path to file>}{<filename>.pgf}
%%
%% Matplotlib used the following preamble
%%
\begingroup%
\makeatletter%
\begin{pgfpicture}%
\pgfpathrectangle{\pgfpointorigin}{\pgfqpoint{8.000000in}{3.000000in}}%
\pgfusepath{use as bounding box, clip}%
\begin{pgfscope}%
\pgfsetbuttcap%
\pgfsetmiterjoin%
\definecolor{currentfill}{rgb}{1.000000,1.000000,1.000000}%
\pgfsetfillcolor{currentfill}%
\pgfsetlinewidth{0.000000pt}%
\definecolor{currentstroke}{rgb}{1.000000,1.000000,1.000000}%
\pgfsetstrokecolor{currentstroke}%
\pgfsetdash{}{0pt}%
\pgfpathmoveto{\pgfqpoint{0.000000in}{0.000000in}}%
\pgfpathlineto{\pgfqpoint{8.000000in}{0.000000in}}%
\pgfpathlineto{\pgfqpoint{8.000000in}{3.000000in}}%
\pgfpathlineto{\pgfqpoint{0.000000in}{3.000000in}}%
\pgfpathclose%
\pgfusepath{fill}%
\end{pgfscope}%
\begin{pgfscope}%
\pgfsetbuttcap%
\pgfsetmiterjoin%
\definecolor{currentfill}{rgb}{1.000000,1.000000,1.000000}%
\pgfsetfillcolor{currentfill}%
\pgfsetlinewidth{0.000000pt}%
\definecolor{currentstroke}{rgb}{0.000000,0.000000,0.000000}%
\pgfsetstrokecolor{currentstroke}%
\pgfsetstrokeopacity{0.000000}%
\pgfsetdash{}{0pt}%
\pgfpathmoveto{\pgfqpoint{1.000000in}{0.330000in}}%
\pgfpathlineto{\pgfqpoint{7.200000in}{0.330000in}}%
\pgfpathlineto{\pgfqpoint{7.200000in}{2.640000in}}%
\pgfpathlineto{\pgfqpoint{1.000000in}{2.640000in}}%
\pgfpathclose%
\pgfusepath{fill}%
\end{pgfscope}%
\begin{pgfscope}%
\pgfpathrectangle{\pgfqpoint{1.000000in}{0.330000in}}{\pgfqpoint{6.200000in}{2.310000in}}%
\pgfusepath{clip}%
\pgfsetbuttcap%
\pgfsetroundjoin%
\pgfsetlinewidth{0.803000pt}%
\definecolor{currentstroke}{rgb}{0.000000,0.000000,0.000000}%
\pgfsetstrokecolor{currentstroke}%
\pgfsetdash{{0.800000pt}{1.320000pt}}{0.000000pt}%
\pgfpathmoveto{\pgfqpoint{1.281818in}{0.330000in}}%
\pgfpathlineto{\pgfqpoint{1.281818in}{2.640000in}}%
\pgfusepath{stroke}%
\end{pgfscope}%
\begin{pgfscope}%
\pgfsetbuttcap%
\pgfsetroundjoin%
\definecolor{currentfill}{rgb}{0.000000,0.000000,0.000000}%
\pgfsetfillcolor{currentfill}%
\pgfsetlinewidth{0.803000pt}%
\definecolor{currentstroke}{rgb}{0.000000,0.000000,0.000000}%
\pgfsetstrokecolor{currentstroke}%
\pgfsetdash{}{0pt}%
\pgfsys@defobject{currentmarker}{\pgfqpoint{0.000000in}{-0.048611in}}{\pgfqpoint{0.000000in}{0.000000in}}{%
\pgfpathmoveto{\pgfqpoint{0.000000in}{0.000000in}}%
\pgfpathlineto{\pgfqpoint{0.000000in}{-0.048611in}}%
\pgfusepath{stroke,fill}%
}%
\begin{pgfscope}%
\pgfsys@transformshift{1.281818in}{0.330000in}%
\pgfsys@useobject{currentmarker}{}%
\end{pgfscope}%
\end{pgfscope}%
\begin{pgfscope}%
\pgfsetbuttcap%
\pgfsetroundjoin%
\definecolor{currentfill}{rgb}{0.000000,0.000000,0.000000}%
\pgfsetfillcolor{currentfill}%
\pgfsetlinewidth{0.803000pt}%
\definecolor{currentstroke}{rgb}{0.000000,0.000000,0.000000}%
\pgfsetstrokecolor{currentstroke}%
\pgfsetdash{}{0pt}%
\pgfsys@defobject{currentmarker}{\pgfqpoint{0.000000in}{0.000000in}}{\pgfqpoint{0.000000in}{0.048611in}}{%
\pgfpathmoveto{\pgfqpoint{0.000000in}{0.000000in}}%
\pgfpathlineto{\pgfqpoint{0.000000in}{0.048611in}}%
\pgfusepath{stroke,fill}%
}%
\begin{pgfscope}%
\pgfsys@transformshift{1.281818in}{2.640000in}%
\pgfsys@useobject{currentmarker}{}%
\end{pgfscope}%
\end{pgfscope}%
\begin{pgfscope}%
\definecolor{textcolor}{rgb}{0.000000,0.000000,0.000000}%
\pgfsetstrokecolor{textcolor}%
\pgfsetfillcolor{textcolor}%
\pgftext[x=1.281818in,y=0.232778in,,top]{\color{textcolor}\rmfamily\fontsize{10.000000}{12.000000}\selectfont \(\displaystyle {0}\)}%
\end{pgfscope}%
\begin{pgfscope}%
\pgfpathrectangle{\pgfqpoint{1.000000in}{0.330000in}}{\pgfqpoint{6.200000in}{2.310000in}}%
\pgfusepath{clip}%
\pgfsetbuttcap%
\pgfsetroundjoin%
\pgfsetlinewidth{0.803000pt}%
\definecolor{currentstroke}{rgb}{0.000000,0.000000,0.000000}%
\pgfsetstrokecolor{currentstroke}%
\pgfsetdash{{0.800000pt}{1.320000pt}}{0.000000pt}%
\pgfpathmoveto{\pgfqpoint{2.148951in}{0.330000in}}%
\pgfpathlineto{\pgfqpoint{2.148951in}{2.640000in}}%
\pgfusepath{stroke}%
\end{pgfscope}%
\begin{pgfscope}%
\pgfsetbuttcap%
\pgfsetroundjoin%
\definecolor{currentfill}{rgb}{0.000000,0.000000,0.000000}%
\pgfsetfillcolor{currentfill}%
\pgfsetlinewidth{0.803000pt}%
\definecolor{currentstroke}{rgb}{0.000000,0.000000,0.000000}%
\pgfsetstrokecolor{currentstroke}%
\pgfsetdash{}{0pt}%
\pgfsys@defobject{currentmarker}{\pgfqpoint{0.000000in}{-0.048611in}}{\pgfqpoint{0.000000in}{0.000000in}}{%
\pgfpathmoveto{\pgfqpoint{0.000000in}{0.000000in}}%
\pgfpathlineto{\pgfqpoint{0.000000in}{-0.048611in}}%
\pgfusepath{stroke,fill}%
}%
\begin{pgfscope}%
\pgfsys@transformshift{2.148951in}{0.330000in}%
\pgfsys@useobject{currentmarker}{}%
\end{pgfscope}%
\end{pgfscope}%
\begin{pgfscope}%
\pgfsetbuttcap%
\pgfsetroundjoin%
\definecolor{currentfill}{rgb}{0.000000,0.000000,0.000000}%
\pgfsetfillcolor{currentfill}%
\pgfsetlinewidth{0.803000pt}%
\definecolor{currentstroke}{rgb}{0.000000,0.000000,0.000000}%
\pgfsetstrokecolor{currentstroke}%
\pgfsetdash{}{0pt}%
\pgfsys@defobject{currentmarker}{\pgfqpoint{0.000000in}{0.000000in}}{\pgfqpoint{0.000000in}{0.048611in}}{%
\pgfpathmoveto{\pgfqpoint{0.000000in}{0.000000in}}%
\pgfpathlineto{\pgfqpoint{0.000000in}{0.048611in}}%
\pgfusepath{stroke,fill}%
}%
\begin{pgfscope}%
\pgfsys@transformshift{2.148951in}{2.640000in}%
\pgfsys@useobject{currentmarker}{}%
\end{pgfscope}%
\end{pgfscope}%
\begin{pgfscope}%
\definecolor{textcolor}{rgb}{0.000000,0.000000,0.000000}%
\pgfsetstrokecolor{textcolor}%
\pgfsetfillcolor{textcolor}%
\pgftext[x=2.148951in,y=0.232778in,,top]{\color{textcolor}\rmfamily\fontsize{10.000000}{12.000000}\selectfont \(\displaystyle {2}\)}%
\end{pgfscope}%
\begin{pgfscope}%
\pgfpathrectangle{\pgfqpoint{1.000000in}{0.330000in}}{\pgfqpoint{6.200000in}{2.310000in}}%
\pgfusepath{clip}%
\pgfsetbuttcap%
\pgfsetroundjoin%
\pgfsetlinewidth{0.803000pt}%
\definecolor{currentstroke}{rgb}{0.000000,0.000000,0.000000}%
\pgfsetstrokecolor{currentstroke}%
\pgfsetdash{{0.800000pt}{1.320000pt}}{0.000000pt}%
\pgfpathmoveto{\pgfqpoint{3.016084in}{0.330000in}}%
\pgfpathlineto{\pgfqpoint{3.016084in}{2.640000in}}%
\pgfusepath{stroke}%
\end{pgfscope}%
\begin{pgfscope}%
\pgfsetbuttcap%
\pgfsetroundjoin%
\definecolor{currentfill}{rgb}{0.000000,0.000000,0.000000}%
\pgfsetfillcolor{currentfill}%
\pgfsetlinewidth{0.803000pt}%
\definecolor{currentstroke}{rgb}{0.000000,0.000000,0.000000}%
\pgfsetstrokecolor{currentstroke}%
\pgfsetdash{}{0pt}%
\pgfsys@defobject{currentmarker}{\pgfqpoint{0.000000in}{-0.048611in}}{\pgfqpoint{0.000000in}{0.000000in}}{%
\pgfpathmoveto{\pgfqpoint{0.000000in}{0.000000in}}%
\pgfpathlineto{\pgfqpoint{0.000000in}{-0.048611in}}%
\pgfusepath{stroke,fill}%
}%
\begin{pgfscope}%
\pgfsys@transformshift{3.016084in}{0.330000in}%
\pgfsys@useobject{currentmarker}{}%
\end{pgfscope}%
\end{pgfscope}%
\begin{pgfscope}%
\pgfsetbuttcap%
\pgfsetroundjoin%
\definecolor{currentfill}{rgb}{0.000000,0.000000,0.000000}%
\pgfsetfillcolor{currentfill}%
\pgfsetlinewidth{0.803000pt}%
\definecolor{currentstroke}{rgb}{0.000000,0.000000,0.000000}%
\pgfsetstrokecolor{currentstroke}%
\pgfsetdash{}{0pt}%
\pgfsys@defobject{currentmarker}{\pgfqpoint{0.000000in}{0.000000in}}{\pgfqpoint{0.000000in}{0.048611in}}{%
\pgfpathmoveto{\pgfqpoint{0.000000in}{0.000000in}}%
\pgfpathlineto{\pgfqpoint{0.000000in}{0.048611in}}%
\pgfusepath{stroke,fill}%
}%
\begin{pgfscope}%
\pgfsys@transformshift{3.016084in}{2.640000in}%
\pgfsys@useobject{currentmarker}{}%
\end{pgfscope}%
\end{pgfscope}%
\begin{pgfscope}%
\definecolor{textcolor}{rgb}{0.000000,0.000000,0.000000}%
\pgfsetstrokecolor{textcolor}%
\pgfsetfillcolor{textcolor}%
\pgftext[x=3.016084in,y=0.232778in,,top]{\color{textcolor}\rmfamily\fontsize{10.000000}{12.000000}\selectfont \(\displaystyle {4}\)}%
\end{pgfscope}%
\begin{pgfscope}%
\pgfpathrectangle{\pgfqpoint{1.000000in}{0.330000in}}{\pgfqpoint{6.200000in}{2.310000in}}%
\pgfusepath{clip}%
\pgfsetbuttcap%
\pgfsetroundjoin%
\pgfsetlinewidth{0.803000pt}%
\definecolor{currentstroke}{rgb}{0.000000,0.000000,0.000000}%
\pgfsetstrokecolor{currentstroke}%
\pgfsetdash{{0.800000pt}{1.320000pt}}{0.000000pt}%
\pgfpathmoveto{\pgfqpoint{3.883217in}{0.330000in}}%
\pgfpathlineto{\pgfqpoint{3.883217in}{2.640000in}}%
\pgfusepath{stroke}%
\end{pgfscope}%
\begin{pgfscope}%
\pgfsetbuttcap%
\pgfsetroundjoin%
\definecolor{currentfill}{rgb}{0.000000,0.000000,0.000000}%
\pgfsetfillcolor{currentfill}%
\pgfsetlinewidth{0.803000pt}%
\definecolor{currentstroke}{rgb}{0.000000,0.000000,0.000000}%
\pgfsetstrokecolor{currentstroke}%
\pgfsetdash{}{0pt}%
\pgfsys@defobject{currentmarker}{\pgfqpoint{0.000000in}{-0.048611in}}{\pgfqpoint{0.000000in}{0.000000in}}{%
\pgfpathmoveto{\pgfqpoint{0.000000in}{0.000000in}}%
\pgfpathlineto{\pgfqpoint{0.000000in}{-0.048611in}}%
\pgfusepath{stroke,fill}%
}%
\begin{pgfscope}%
\pgfsys@transformshift{3.883217in}{0.330000in}%
\pgfsys@useobject{currentmarker}{}%
\end{pgfscope}%
\end{pgfscope}%
\begin{pgfscope}%
\pgfsetbuttcap%
\pgfsetroundjoin%
\definecolor{currentfill}{rgb}{0.000000,0.000000,0.000000}%
\pgfsetfillcolor{currentfill}%
\pgfsetlinewidth{0.803000pt}%
\definecolor{currentstroke}{rgb}{0.000000,0.000000,0.000000}%
\pgfsetstrokecolor{currentstroke}%
\pgfsetdash{}{0pt}%
\pgfsys@defobject{currentmarker}{\pgfqpoint{0.000000in}{0.000000in}}{\pgfqpoint{0.000000in}{0.048611in}}{%
\pgfpathmoveto{\pgfqpoint{0.000000in}{0.000000in}}%
\pgfpathlineto{\pgfqpoint{0.000000in}{0.048611in}}%
\pgfusepath{stroke,fill}%
}%
\begin{pgfscope}%
\pgfsys@transformshift{3.883217in}{2.640000in}%
\pgfsys@useobject{currentmarker}{}%
\end{pgfscope}%
\end{pgfscope}%
\begin{pgfscope}%
\definecolor{textcolor}{rgb}{0.000000,0.000000,0.000000}%
\pgfsetstrokecolor{textcolor}%
\pgfsetfillcolor{textcolor}%
\pgftext[x=3.883217in,y=0.232778in,,top]{\color{textcolor}\rmfamily\fontsize{10.000000}{12.000000}\selectfont \(\displaystyle {6}\)}%
\end{pgfscope}%
\begin{pgfscope}%
\pgfpathrectangle{\pgfqpoint{1.000000in}{0.330000in}}{\pgfqpoint{6.200000in}{2.310000in}}%
\pgfusepath{clip}%
\pgfsetbuttcap%
\pgfsetroundjoin%
\pgfsetlinewidth{0.803000pt}%
\definecolor{currentstroke}{rgb}{0.000000,0.000000,0.000000}%
\pgfsetstrokecolor{currentstroke}%
\pgfsetdash{{0.800000pt}{1.320000pt}}{0.000000pt}%
\pgfpathmoveto{\pgfqpoint{4.750350in}{0.330000in}}%
\pgfpathlineto{\pgfqpoint{4.750350in}{2.640000in}}%
\pgfusepath{stroke}%
\end{pgfscope}%
\begin{pgfscope}%
\pgfsetbuttcap%
\pgfsetroundjoin%
\definecolor{currentfill}{rgb}{0.000000,0.000000,0.000000}%
\pgfsetfillcolor{currentfill}%
\pgfsetlinewidth{0.803000pt}%
\definecolor{currentstroke}{rgb}{0.000000,0.000000,0.000000}%
\pgfsetstrokecolor{currentstroke}%
\pgfsetdash{}{0pt}%
\pgfsys@defobject{currentmarker}{\pgfqpoint{0.000000in}{-0.048611in}}{\pgfqpoint{0.000000in}{0.000000in}}{%
\pgfpathmoveto{\pgfqpoint{0.000000in}{0.000000in}}%
\pgfpathlineto{\pgfqpoint{0.000000in}{-0.048611in}}%
\pgfusepath{stroke,fill}%
}%
\begin{pgfscope}%
\pgfsys@transformshift{4.750350in}{0.330000in}%
\pgfsys@useobject{currentmarker}{}%
\end{pgfscope}%
\end{pgfscope}%
\begin{pgfscope}%
\pgfsetbuttcap%
\pgfsetroundjoin%
\definecolor{currentfill}{rgb}{0.000000,0.000000,0.000000}%
\pgfsetfillcolor{currentfill}%
\pgfsetlinewidth{0.803000pt}%
\definecolor{currentstroke}{rgb}{0.000000,0.000000,0.000000}%
\pgfsetstrokecolor{currentstroke}%
\pgfsetdash{}{0pt}%
\pgfsys@defobject{currentmarker}{\pgfqpoint{0.000000in}{0.000000in}}{\pgfqpoint{0.000000in}{0.048611in}}{%
\pgfpathmoveto{\pgfqpoint{0.000000in}{0.000000in}}%
\pgfpathlineto{\pgfqpoint{0.000000in}{0.048611in}}%
\pgfusepath{stroke,fill}%
}%
\begin{pgfscope}%
\pgfsys@transformshift{4.750350in}{2.640000in}%
\pgfsys@useobject{currentmarker}{}%
\end{pgfscope}%
\end{pgfscope}%
\begin{pgfscope}%
\definecolor{textcolor}{rgb}{0.000000,0.000000,0.000000}%
\pgfsetstrokecolor{textcolor}%
\pgfsetfillcolor{textcolor}%
\pgftext[x=4.750350in,y=0.232778in,,top]{\color{textcolor}\rmfamily\fontsize{10.000000}{12.000000}\selectfont \(\displaystyle {8}\)}%
\end{pgfscope}%
\begin{pgfscope}%
\pgfpathrectangle{\pgfqpoint{1.000000in}{0.330000in}}{\pgfqpoint{6.200000in}{2.310000in}}%
\pgfusepath{clip}%
\pgfsetbuttcap%
\pgfsetroundjoin%
\pgfsetlinewidth{0.803000pt}%
\definecolor{currentstroke}{rgb}{0.000000,0.000000,0.000000}%
\pgfsetstrokecolor{currentstroke}%
\pgfsetdash{{0.800000pt}{1.320000pt}}{0.000000pt}%
\pgfpathmoveto{\pgfqpoint{5.617483in}{0.330000in}}%
\pgfpathlineto{\pgfqpoint{5.617483in}{2.640000in}}%
\pgfusepath{stroke}%
\end{pgfscope}%
\begin{pgfscope}%
\pgfsetbuttcap%
\pgfsetroundjoin%
\definecolor{currentfill}{rgb}{0.000000,0.000000,0.000000}%
\pgfsetfillcolor{currentfill}%
\pgfsetlinewidth{0.803000pt}%
\definecolor{currentstroke}{rgb}{0.000000,0.000000,0.000000}%
\pgfsetstrokecolor{currentstroke}%
\pgfsetdash{}{0pt}%
\pgfsys@defobject{currentmarker}{\pgfqpoint{0.000000in}{-0.048611in}}{\pgfqpoint{0.000000in}{0.000000in}}{%
\pgfpathmoveto{\pgfqpoint{0.000000in}{0.000000in}}%
\pgfpathlineto{\pgfqpoint{0.000000in}{-0.048611in}}%
\pgfusepath{stroke,fill}%
}%
\begin{pgfscope}%
\pgfsys@transformshift{5.617483in}{0.330000in}%
\pgfsys@useobject{currentmarker}{}%
\end{pgfscope}%
\end{pgfscope}%
\begin{pgfscope}%
\pgfsetbuttcap%
\pgfsetroundjoin%
\definecolor{currentfill}{rgb}{0.000000,0.000000,0.000000}%
\pgfsetfillcolor{currentfill}%
\pgfsetlinewidth{0.803000pt}%
\definecolor{currentstroke}{rgb}{0.000000,0.000000,0.000000}%
\pgfsetstrokecolor{currentstroke}%
\pgfsetdash{}{0pt}%
\pgfsys@defobject{currentmarker}{\pgfqpoint{0.000000in}{0.000000in}}{\pgfqpoint{0.000000in}{0.048611in}}{%
\pgfpathmoveto{\pgfqpoint{0.000000in}{0.000000in}}%
\pgfpathlineto{\pgfqpoint{0.000000in}{0.048611in}}%
\pgfusepath{stroke,fill}%
}%
\begin{pgfscope}%
\pgfsys@transformshift{5.617483in}{2.640000in}%
\pgfsys@useobject{currentmarker}{}%
\end{pgfscope}%
\end{pgfscope}%
\begin{pgfscope}%
\definecolor{textcolor}{rgb}{0.000000,0.000000,0.000000}%
\pgfsetstrokecolor{textcolor}%
\pgfsetfillcolor{textcolor}%
\pgftext[x=5.617483in,y=0.232778in,,top]{\color{textcolor}\rmfamily\fontsize{10.000000}{12.000000}\selectfont \(\displaystyle {10}\)}%
\end{pgfscope}%
\begin{pgfscope}%
\pgfpathrectangle{\pgfqpoint{1.000000in}{0.330000in}}{\pgfqpoint{6.200000in}{2.310000in}}%
\pgfusepath{clip}%
\pgfsetbuttcap%
\pgfsetroundjoin%
\pgfsetlinewidth{0.803000pt}%
\definecolor{currentstroke}{rgb}{0.000000,0.000000,0.000000}%
\pgfsetstrokecolor{currentstroke}%
\pgfsetdash{{0.800000pt}{1.320000pt}}{0.000000pt}%
\pgfpathmoveto{\pgfqpoint{6.484615in}{0.330000in}}%
\pgfpathlineto{\pgfqpoint{6.484615in}{2.640000in}}%
\pgfusepath{stroke}%
\end{pgfscope}%
\begin{pgfscope}%
\pgfsetbuttcap%
\pgfsetroundjoin%
\definecolor{currentfill}{rgb}{0.000000,0.000000,0.000000}%
\pgfsetfillcolor{currentfill}%
\pgfsetlinewidth{0.803000pt}%
\definecolor{currentstroke}{rgb}{0.000000,0.000000,0.000000}%
\pgfsetstrokecolor{currentstroke}%
\pgfsetdash{}{0pt}%
\pgfsys@defobject{currentmarker}{\pgfqpoint{0.000000in}{-0.048611in}}{\pgfqpoint{0.000000in}{0.000000in}}{%
\pgfpathmoveto{\pgfqpoint{0.000000in}{0.000000in}}%
\pgfpathlineto{\pgfqpoint{0.000000in}{-0.048611in}}%
\pgfusepath{stroke,fill}%
}%
\begin{pgfscope}%
\pgfsys@transformshift{6.484615in}{0.330000in}%
\pgfsys@useobject{currentmarker}{}%
\end{pgfscope}%
\end{pgfscope}%
\begin{pgfscope}%
\pgfsetbuttcap%
\pgfsetroundjoin%
\definecolor{currentfill}{rgb}{0.000000,0.000000,0.000000}%
\pgfsetfillcolor{currentfill}%
\pgfsetlinewidth{0.803000pt}%
\definecolor{currentstroke}{rgb}{0.000000,0.000000,0.000000}%
\pgfsetstrokecolor{currentstroke}%
\pgfsetdash{}{0pt}%
\pgfsys@defobject{currentmarker}{\pgfqpoint{0.000000in}{0.000000in}}{\pgfqpoint{0.000000in}{0.048611in}}{%
\pgfpathmoveto{\pgfqpoint{0.000000in}{0.000000in}}%
\pgfpathlineto{\pgfqpoint{0.000000in}{0.048611in}}%
\pgfusepath{stroke,fill}%
}%
\begin{pgfscope}%
\pgfsys@transformshift{6.484615in}{2.640000in}%
\pgfsys@useobject{currentmarker}{}%
\end{pgfscope}%
\end{pgfscope}%
\begin{pgfscope}%
\definecolor{textcolor}{rgb}{0.000000,0.000000,0.000000}%
\pgfsetstrokecolor{textcolor}%
\pgfsetfillcolor{textcolor}%
\pgftext[x=6.484615in,y=0.232778in,,top]{\color{textcolor}\rmfamily\fontsize{10.000000}{12.000000}\selectfont \(\displaystyle {12}\)}%
\end{pgfscope}%
\begin{pgfscope}%
\pgfpathrectangle{\pgfqpoint{1.000000in}{0.330000in}}{\pgfqpoint{6.200000in}{2.310000in}}%
\pgfusepath{clip}%
\pgfsetbuttcap%
\pgfsetroundjoin%
\pgfsetlinewidth{0.803000pt}%
\definecolor{currentstroke}{rgb}{0.000000,0.000000,0.000000}%
\pgfsetstrokecolor{currentstroke}%
\pgfsetdash{{0.800000pt}{1.320000pt}}{0.000000pt}%
\pgfpathmoveto{\pgfqpoint{1.000000in}{0.490923in}}%
\pgfpathlineto{\pgfqpoint{7.200000in}{0.490923in}}%
\pgfusepath{stroke}%
\end{pgfscope}%
\begin{pgfscope}%
\pgfsetbuttcap%
\pgfsetroundjoin%
\definecolor{currentfill}{rgb}{0.000000,0.000000,0.000000}%
\pgfsetfillcolor{currentfill}%
\pgfsetlinewidth{0.803000pt}%
\definecolor{currentstroke}{rgb}{0.000000,0.000000,0.000000}%
\pgfsetstrokecolor{currentstroke}%
\pgfsetdash{}{0pt}%
\pgfsys@defobject{currentmarker}{\pgfqpoint{-0.048611in}{0.000000in}}{\pgfqpoint{-0.000000in}{0.000000in}}{%
\pgfpathmoveto{\pgfqpoint{-0.000000in}{0.000000in}}%
\pgfpathlineto{\pgfqpoint{-0.048611in}{0.000000in}}%
\pgfusepath{stroke,fill}%
}%
\begin{pgfscope}%
\pgfsys@transformshift{1.000000in}{0.490923in}%
\pgfsys@useobject{currentmarker}{}%
\end{pgfscope}%
\end{pgfscope}%
\begin{pgfscope}%
\pgfsetbuttcap%
\pgfsetroundjoin%
\definecolor{currentfill}{rgb}{0.000000,0.000000,0.000000}%
\pgfsetfillcolor{currentfill}%
\pgfsetlinewidth{0.803000pt}%
\definecolor{currentstroke}{rgb}{0.000000,0.000000,0.000000}%
\pgfsetstrokecolor{currentstroke}%
\pgfsetdash{}{0pt}%
\pgfsys@defobject{currentmarker}{\pgfqpoint{0.000000in}{0.000000in}}{\pgfqpoint{0.048611in}{0.000000in}}{%
\pgfpathmoveto{\pgfqpoint{0.000000in}{0.000000in}}%
\pgfpathlineto{\pgfqpoint{0.048611in}{0.000000in}}%
\pgfusepath{stroke,fill}%
}%
\begin{pgfscope}%
\pgfsys@transformshift{7.200000in}{0.490923in}%
\pgfsys@useobject{currentmarker}{}%
\end{pgfscope}%
\end{pgfscope}%
\begin{pgfscope}%
\definecolor{textcolor}{rgb}{0.000000,0.000000,0.000000}%
\pgfsetstrokecolor{textcolor}%
\pgfsetfillcolor{textcolor}%
\pgftext[x=0.478394in, y=0.442698in, left, base]{\color{textcolor}\rmfamily\fontsize{10.000000}{12.000000}\selectfont \(\displaystyle {\ensuremath{-}0.150}\)}%
\end{pgfscope}%
\begin{pgfscope}%
\pgfpathrectangle{\pgfqpoint{1.000000in}{0.330000in}}{\pgfqpoint{6.200000in}{2.310000in}}%
\pgfusepath{clip}%
\pgfsetbuttcap%
\pgfsetroundjoin%
\pgfsetlinewidth{0.803000pt}%
\definecolor{currentstroke}{rgb}{0.000000,0.000000,0.000000}%
\pgfsetstrokecolor{currentstroke}%
\pgfsetdash{{0.800000pt}{1.320000pt}}{0.000000pt}%
\pgfpathmoveto{\pgfqpoint{1.000000in}{0.796128in}}%
\pgfpathlineto{\pgfqpoint{7.200000in}{0.796128in}}%
\pgfusepath{stroke}%
\end{pgfscope}%
\begin{pgfscope}%
\pgfsetbuttcap%
\pgfsetroundjoin%
\definecolor{currentfill}{rgb}{0.000000,0.000000,0.000000}%
\pgfsetfillcolor{currentfill}%
\pgfsetlinewidth{0.803000pt}%
\definecolor{currentstroke}{rgb}{0.000000,0.000000,0.000000}%
\pgfsetstrokecolor{currentstroke}%
\pgfsetdash{}{0pt}%
\pgfsys@defobject{currentmarker}{\pgfqpoint{-0.048611in}{0.000000in}}{\pgfqpoint{-0.000000in}{0.000000in}}{%
\pgfpathmoveto{\pgfqpoint{-0.000000in}{0.000000in}}%
\pgfpathlineto{\pgfqpoint{-0.048611in}{0.000000in}}%
\pgfusepath{stroke,fill}%
}%
\begin{pgfscope}%
\pgfsys@transformshift{1.000000in}{0.796128in}%
\pgfsys@useobject{currentmarker}{}%
\end{pgfscope}%
\end{pgfscope}%
\begin{pgfscope}%
\pgfsetbuttcap%
\pgfsetroundjoin%
\definecolor{currentfill}{rgb}{0.000000,0.000000,0.000000}%
\pgfsetfillcolor{currentfill}%
\pgfsetlinewidth{0.803000pt}%
\definecolor{currentstroke}{rgb}{0.000000,0.000000,0.000000}%
\pgfsetstrokecolor{currentstroke}%
\pgfsetdash{}{0pt}%
\pgfsys@defobject{currentmarker}{\pgfqpoint{0.000000in}{0.000000in}}{\pgfqpoint{0.048611in}{0.000000in}}{%
\pgfpathmoveto{\pgfqpoint{0.000000in}{0.000000in}}%
\pgfpathlineto{\pgfqpoint{0.048611in}{0.000000in}}%
\pgfusepath{stroke,fill}%
}%
\begin{pgfscope}%
\pgfsys@transformshift{7.200000in}{0.796128in}%
\pgfsys@useobject{currentmarker}{}%
\end{pgfscope}%
\end{pgfscope}%
\begin{pgfscope}%
\definecolor{textcolor}{rgb}{0.000000,0.000000,0.000000}%
\pgfsetstrokecolor{textcolor}%
\pgfsetfillcolor{textcolor}%
\pgftext[x=0.478394in, y=0.747903in, left, base]{\color{textcolor}\rmfamily\fontsize{10.000000}{12.000000}\selectfont \(\displaystyle {\ensuremath{-}0.125}\)}%
\end{pgfscope}%
\begin{pgfscope}%
\pgfpathrectangle{\pgfqpoint{1.000000in}{0.330000in}}{\pgfqpoint{6.200000in}{2.310000in}}%
\pgfusepath{clip}%
\pgfsetbuttcap%
\pgfsetroundjoin%
\pgfsetlinewidth{0.803000pt}%
\definecolor{currentstroke}{rgb}{0.000000,0.000000,0.000000}%
\pgfsetstrokecolor{currentstroke}%
\pgfsetdash{{0.800000pt}{1.320000pt}}{0.000000pt}%
\pgfpathmoveto{\pgfqpoint{1.000000in}{1.101333in}}%
\pgfpathlineto{\pgfqpoint{7.200000in}{1.101333in}}%
\pgfusepath{stroke}%
\end{pgfscope}%
\begin{pgfscope}%
\pgfsetbuttcap%
\pgfsetroundjoin%
\definecolor{currentfill}{rgb}{0.000000,0.000000,0.000000}%
\pgfsetfillcolor{currentfill}%
\pgfsetlinewidth{0.803000pt}%
\definecolor{currentstroke}{rgb}{0.000000,0.000000,0.000000}%
\pgfsetstrokecolor{currentstroke}%
\pgfsetdash{}{0pt}%
\pgfsys@defobject{currentmarker}{\pgfqpoint{-0.048611in}{0.000000in}}{\pgfqpoint{-0.000000in}{0.000000in}}{%
\pgfpathmoveto{\pgfqpoint{-0.000000in}{0.000000in}}%
\pgfpathlineto{\pgfqpoint{-0.048611in}{0.000000in}}%
\pgfusepath{stroke,fill}%
}%
\begin{pgfscope}%
\pgfsys@transformshift{1.000000in}{1.101333in}%
\pgfsys@useobject{currentmarker}{}%
\end{pgfscope}%
\end{pgfscope}%
\begin{pgfscope}%
\pgfsetbuttcap%
\pgfsetroundjoin%
\definecolor{currentfill}{rgb}{0.000000,0.000000,0.000000}%
\pgfsetfillcolor{currentfill}%
\pgfsetlinewidth{0.803000pt}%
\definecolor{currentstroke}{rgb}{0.000000,0.000000,0.000000}%
\pgfsetstrokecolor{currentstroke}%
\pgfsetdash{}{0pt}%
\pgfsys@defobject{currentmarker}{\pgfqpoint{0.000000in}{0.000000in}}{\pgfqpoint{0.048611in}{0.000000in}}{%
\pgfpathmoveto{\pgfqpoint{0.000000in}{0.000000in}}%
\pgfpathlineto{\pgfqpoint{0.048611in}{0.000000in}}%
\pgfusepath{stroke,fill}%
}%
\begin{pgfscope}%
\pgfsys@transformshift{7.200000in}{1.101333in}%
\pgfsys@useobject{currentmarker}{}%
\end{pgfscope}%
\end{pgfscope}%
\begin{pgfscope}%
\definecolor{textcolor}{rgb}{0.000000,0.000000,0.000000}%
\pgfsetstrokecolor{textcolor}%
\pgfsetfillcolor{textcolor}%
\pgftext[x=0.478394in, y=1.053107in, left, base]{\color{textcolor}\rmfamily\fontsize{10.000000}{12.000000}\selectfont \(\displaystyle {\ensuremath{-}0.100}\)}%
\end{pgfscope}%
\begin{pgfscope}%
\pgfpathrectangle{\pgfqpoint{1.000000in}{0.330000in}}{\pgfqpoint{6.200000in}{2.310000in}}%
\pgfusepath{clip}%
\pgfsetbuttcap%
\pgfsetroundjoin%
\pgfsetlinewidth{0.803000pt}%
\definecolor{currentstroke}{rgb}{0.000000,0.000000,0.000000}%
\pgfsetstrokecolor{currentstroke}%
\pgfsetdash{{0.800000pt}{1.320000pt}}{0.000000pt}%
\pgfpathmoveto{\pgfqpoint{1.000000in}{1.406537in}}%
\pgfpathlineto{\pgfqpoint{7.200000in}{1.406537in}}%
\pgfusepath{stroke}%
\end{pgfscope}%
\begin{pgfscope}%
\pgfsetbuttcap%
\pgfsetroundjoin%
\definecolor{currentfill}{rgb}{0.000000,0.000000,0.000000}%
\pgfsetfillcolor{currentfill}%
\pgfsetlinewidth{0.803000pt}%
\definecolor{currentstroke}{rgb}{0.000000,0.000000,0.000000}%
\pgfsetstrokecolor{currentstroke}%
\pgfsetdash{}{0pt}%
\pgfsys@defobject{currentmarker}{\pgfqpoint{-0.048611in}{0.000000in}}{\pgfqpoint{-0.000000in}{0.000000in}}{%
\pgfpathmoveto{\pgfqpoint{-0.000000in}{0.000000in}}%
\pgfpathlineto{\pgfqpoint{-0.048611in}{0.000000in}}%
\pgfusepath{stroke,fill}%
}%
\begin{pgfscope}%
\pgfsys@transformshift{1.000000in}{1.406537in}%
\pgfsys@useobject{currentmarker}{}%
\end{pgfscope}%
\end{pgfscope}%
\begin{pgfscope}%
\pgfsetbuttcap%
\pgfsetroundjoin%
\definecolor{currentfill}{rgb}{0.000000,0.000000,0.000000}%
\pgfsetfillcolor{currentfill}%
\pgfsetlinewidth{0.803000pt}%
\definecolor{currentstroke}{rgb}{0.000000,0.000000,0.000000}%
\pgfsetstrokecolor{currentstroke}%
\pgfsetdash{}{0pt}%
\pgfsys@defobject{currentmarker}{\pgfqpoint{0.000000in}{0.000000in}}{\pgfqpoint{0.048611in}{0.000000in}}{%
\pgfpathmoveto{\pgfqpoint{0.000000in}{0.000000in}}%
\pgfpathlineto{\pgfqpoint{0.048611in}{0.000000in}}%
\pgfusepath{stroke,fill}%
}%
\begin{pgfscope}%
\pgfsys@transformshift{7.200000in}{1.406537in}%
\pgfsys@useobject{currentmarker}{}%
\end{pgfscope}%
\end{pgfscope}%
\begin{pgfscope}%
\definecolor{textcolor}{rgb}{0.000000,0.000000,0.000000}%
\pgfsetstrokecolor{textcolor}%
\pgfsetfillcolor{textcolor}%
\pgftext[x=0.478394in, y=1.358312in, left, base]{\color{textcolor}\rmfamily\fontsize{10.000000}{12.000000}\selectfont \(\displaystyle {\ensuremath{-}0.075}\)}%
\end{pgfscope}%
\begin{pgfscope}%
\pgfpathrectangle{\pgfqpoint{1.000000in}{0.330000in}}{\pgfqpoint{6.200000in}{2.310000in}}%
\pgfusepath{clip}%
\pgfsetbuttcap%
\pgfsetroundjoin%
\pgfsetlinewidth{0.803000pt}%
\definecolor{currentstroke}{rgb}{0.000000,0.000000,0.000000}%
\pgfsetstrokecolor{currentstroke}%
\pgfsetdash{{0.800000pt}{1.320000pt}}{0.000000pt}%
\pgfpathmoveto{\pgfqpoint{1.000000in}{1.711742in}}%
\pgfpathlineto{\pgfqpoint{7.200000in}{1.711742in}}%
\pgfusepath{stroke}%
\end{pgfscope}%
\begin{pgfscope}%
\pgfsetbuttcap%
\pgfsetroundjoin%
\definecolor{currentfill}{rgb}{0.000000,0.000000,0.000000}%
\pgfsetfillcolor{currentfill}%
\pgfsetlinewidth{0.803000pt}%
\definecolor{currentstroke}{rgb}{0.000000,0.000000,0.000000}%
\pgfsetstrokecolor{currentstroke}%
\pgfsetdash{}{0pt}%
\pgfsys@defobject{currentmarker}{\pgfqpoint{-0.048611in}{0.000000in}}{\pgfqpoint{-0.000000in}{0.000000in}}{%
\pgfpathmoveto{\pgfqpoint{-0.000000in}{0.000000in}}%
\pgfpathlineto{\pgfqpoint{-0.048611in}{0.000000in}}%
\pgfusepath{stroke,fill}%
}%
\begin{pgfscope}%
\pgfsys@transformshift{1.000000in}{1.711742in}%
\pgfsys@useobject{currentmarker}{}%
\end{pgfscope}%
\end{pgfscope}%
\begin{pgfscope}%
\pgfsetbuttcap%
\pgfsetroundjoin%
\definecolor{currentfill}{rgb}{0.000000,0.000000,0.000000}%
\pgfsetfillcolor{currentfill}%
\pgfsetlinewidth{0.803000pt}%
\definecolor{currentstroke}{rgb}{0.000000,0.000000,0.000000}%
\pgfsetstrokecolor{currentstroke}%
\pgfsetdash{}{0pt}%
\pgfsys@defobject{currentmarker}{\pgfqpoint{0.000000in}{0.000000in}}{\pgfqpoint{0.048611in}{0.000000in}}{%
\pgfpathmoveto{\pgfqpoint{0.000000in}{0.000000in}}%
\pgfpathlineto{\pgfqpoint{0.048611in}{0.000000in}}%
\pgfusepath{stroke,fill}%
}%
\begin{pgfscope}%
\pgfsys@transformshift{7.200000in}{1.711742in}%
\pgfsys@useobject{currentmarker}{}%
\end{pgfscope}%
\end{pgfscope}%
\begin{pgfscope}%
\definecolor{textcolor}{rgb}{0.000000,0.000000,0.000000}%
\pgfsetstrokecolor{textcolor}%
\pgfsetfillcolor{textcolor}%
\pgftext[x=0.478394in, y=1.663516in, left, base]{\color{textcolor}\rmfamily\fontsize{10.000000}{12.000000}\selectfont \(\displaystyle {\ensuremath{-}0.050}\)}%
\end{pgfscope}%
\begin{pgfscope}%
\pgfpathrectangle{\pgfqpoint{1.000000in}{0.330000in}}{\pgfqpoint{6.200000in}{2.310000in}}%
\pgfusepath{clip}%
\pgfsetbuttcap%
\pgfsetroundjoin%
\pgfsetlinewidth{0.803000pt}%
\definecolor{currentstroke}{rgb}{0.000000,0.000000,0.000000}%
\pgfsetstrokecolor{currentstroke}%
\pgfsetdash{{0.800000pt}{1.320000pt}}{0.000000pt}%
\pgfpathmoveto{\pgfqpoint{1.000000in}{2.016946in}}%
\pgfpathlineto{\pgfqpoint{7.200000in}{2.016946in}}%
\pgfusepath{stroke}%
\end{pgfscope}%
\begin{pgfscope}%
\pgfsetbuttcap%
\pgfsetroundjoin%
\definecolor{currentfill}{rgb}{0.000000,0.000000,0.000000}%
\pgfsetfillcolor{currentfill}%
\pgfsetlinewidth{0.803000pt}%
\definecolor{currentstroke}{rgb}{0.000000,0.000000,0.000000}%
\pgfsetstrokecolor{currentstroke}%
\pgfsetdash{}{0pt}%
\pgfsys@defobject{currentmarker}{\pgfqpoint{-0.048611in}{0.000000in}}{\pgfqpoint{-0.000000in}{0.000000in}}{%
\pgfpathmoveto{\pgfqpoint{-0.000000in}{0.000000in}}%
\pgfpathlineto{\pgfqpoint{-0.048611in}{0.000000in}}%
\pgfusepath{stroke,fill}%
}%
\begin{pgfscope}%
\pgfsys@transformshift{1.000000in}{2.016946in}%
\pgfsys@useobject{currentmarker}{}%
\end{pgfscope}%
\end{pgfscope}%
\begin{pgfscope}%
\pgfsetbuttcap%
\pgfsetroundjoin%
\definecolor{currentfill}{rgb}{0.000000,0.000000,0.000000}%
\pgfsetfillcolor{currentfill}%
\pgfsetlinewidth{0.803000pt}%
\definecolor{currentstroke}{rgb}{0.000000,0.000000,0.000000}%
\pgfsetstrokecolor{currentstroke}%
\pgfsetdash{}{0pt}%
\pgfsys@defobject{currentmarker}{\pgfqpoint{0.000000in}{0.000000in}}{\pgfqpoint{0.048611in}{0.000000in}}{%
\pgfpathmoveto{\pgfqpoint{0.000000in}{0.000000in}}%
\pgfpathlineto{\pgfqpoint{0.048611in}{0.000000in}}%
\pgfusepath{stroke,fill}%
}%
\begin{pgfscope}%
\pgfsys@transformshift{7.200000in}{2.016946in}%
\pgfsys@useobject{currentmarker}{}%
\end{pgfscope}%
\end{pgfscope}%
\begin{pgfscope}%
\definecolor{textcolor}{rgb}{0.000000,0.000000,0.000000}%
\pgfsetstrokecolor{textcolor}%
\pgfsetfillcolor{textcolor}%
\pgftext[x=0.478394in, y=1.968721in, left, base]{\color{textcolor}\rmfamily\fontsize{10.000000}{12.000000}\selectfont \(\displaystyle {\ensuremath{-}0.025}\)}%
\end{pgfscope}%
\begin{pgfscope}%
\pgfpathrectangle{\pgfqpoint{1.000000in}{0.330000in}}{\pgfqpoint{6.200000in}{2.310000in}}%
\pgfusepath{clip}%
\pgfsetbuttcap%
\pgfsetroundjoin%
\pgfsetlinewidth{0.803000pt}%
\definecolor{currentstroke}{rgb}{0.000000,0.000000,0.000000}%
\pgfsetstrokecolor{currentstroke}%
\pgfsetdash{{0.800000pt}{1.320000pt}}{0.000000pt}%
\pgfpathmoveto{\pgfqpoint{1.000000in}{2.322151in}}%
\pgfpathlineto{\pgfqpoint{7.200000in}{2.322151in}}%
\pgfusepath{stroke}%
\end{pgfscope}%
\begin{pgfscope}%
\pgfsetbuttcap%
\pgfsetroundjoin%
\definecolor{currentfill}{rgb}{0.000000,0.000000,0.000000}%
\pgfsetfillcolor{currentfill}%
\pgfsetlinewidth{0.803000pt}%
\definecolor{currentstroke}{rgb}{0.000000,0.000000,0.000000}%
\pgfsetstrokecolor{currentstroke}%
\pgfsetdash{}{0pt}%
\pgfsys@defobject{currentmarker}{\pgfqpoint{-0.048611in}{0.000000in}}{\pgfqpoint{-0.000000in}{0.000000in}}{%
\pgfpathmoveto{\pgfqpoint{-0.000000in}{0.000000in}}%
\pgfpathlineto{\pgfqpoint{-0.048611in}{0.000000in}}%
\pgfusepath{stroke,fill}%
}%
\begin{pgfscope}%
\pgfsys@transformshift{1.000000in}{2.322151in}%
\pgfsys@useobject{currentmarker}{}%
\end{pgfscope}%
\end{pgfscope}%
\begin{pgfscope}%
\pgfsetbuttcap%
\pgfsetroundjoin%
\definecolor{currentfill}{rgb}{0.000000,0.000000,0.000000}%
\pgfsetfillcolor{currentfill}%
\pgfsetlinewidth{0.803000pt}%
\definecolor{currentstroke}{rgb}{0.000000,0.000000,0.000000}%
\pgfsetstrokecolor{currentstroke}%
\pgfsetdash{}{0pt}%
\pgfsys@defobject{currentmarker}{\pgfqpoint{0.000000in}{0.000000in}}{\pgfqpoint{0.048611in}{0.000000in}}{%
\pgfpathmoveto{\pgfqpoint{0.000000in}{0.000000in}}%
\pgfpathlineto{\pgfqpoint{0.048611in}{0.000000in}}%
\pgfusepath{stroke,fill}%
}%
\begin{pgfscope}%
\pgfsys@transformshift{7.200000in}{2.322151in}%
\pgfsys@useobject{currentmarker}{}%
\end{pgfscope}%
\end{pgfscope}%
\begin{pgfscope}%
\definecolor{textcolor}{rgb}{0.000000,0.000000,0.000000}%
\pgfsetstrokecolor{textcolor}%
\pgfsetfillcolor{textcolor}%
\pgftext[x=0.586419in, y=2.273926in, left, base]{\color{textcolor}\rmfamily\fontsize{10.000000}{12.000000}\selectfont \(\displaystyle {0.000}\)}%
\end{pgfscope}%
\begin{pgfscope}%
\pgfpathrectangle{\pgfqpoint{1.000000in}{0.330000in}}{\pgfqpoint{6.200000in}{2.310000in}}%
\pgfusepath{clip}%
\pgfsetbuttcap%
\pgfsetroundjoin%
\pgfsetlinewidth{0.803000pt}%
\definecolor{currentstroke}{rgb}{0.000000,0.000000,0.000000}%
\pgfsetstrokecolor{currentstroke}%
\pgfsetdash{{0.800000pt}{1.320000pt}}{0.000000pt}%
\pgfpathmoveto{\pgfqpoint{1.000000in}{2.627355in}}%
\pgfpathlineto{\pgfqpoint{7.200000in}{2.627355in}}%
\pgfusepath{stroke}%
\end{pgfscope}%
\begin{pgfscope}%
\pgfsetbuttcap%
\pgfsetroundjoin%
\definecolor{currentfill}{rgb}{0.000000,0.000000,0.000000}%
\pgfsetfillcolor{currentfill}%
\pgfsetlinewidth{0.803000pt}%
\definecolor{currentstroke}{rgb}{0.000000,0.000000,0.000000}%
\pgfsetstrokecolor{currentstroke}%
\pgfsetdash{}{0pt}%
\pgfsys@defobject{currentmarker}{\pgfqpoint{-0.048611in}{0.000000in}}{\pgfqpoint{-0.000000in}{0.000000in}}{%
\pgfpathmoveto{\pgfqpoint{-0.000000in}{0.000000in}}%
\pgfpathlineto{\pgfqpoint{-0.048611in}{0.000000in}}%
\pgfusepath{stroke,fill}%
}%
\begin{pgfscope}%
\pgfsys@transformshift{1.000000in}{2.627355in}%
\pgfsys@useobject{currentmarker}{}%
\end{pgfscope}%
\end{pgfscope}%
\begin{pgfscope}%
\pgfsetbuttcap%
\pgfsetroundjoin%
\definecolor{currentfill}{rgb}{0.000000,0.000000,0.000000}%
\pgfsetfillcolor{currentfill}%
\pgfsetlinewidth{0.803000pt}%
\definecolor{currentstroke}{rgb}{0.000000,0.000000,0.000000}%
\pgfsetstrokecolor{currentstroke}%
\pgfsetdash{}{0pt}%
\pgfsys@defobject{currentmarker}{\pgfqpoint{0.000000in}{0.000000in}}{\pgfqpoint{0.048611in}{0.000000in}}{%
\pgfpathmoveto{\pgfqpoint{0.000000in}{0.000000in}}%
\pgfpathlineto{\pgfqpoint{0.048611in}{0.000000in}}%
\pgfusepath{stroke,fill}%
}%
\begin{pgfscope}%
\pgfsys@transformshift{7.200000in}{2.627355in}%
\pgfsys@useobject{currentmarker}{}%
\end{pgfscope}%
\end{pgfscope}%
\begin{pgfscope}%
\definecolor{textcolor}{rgb}{0.000000,0.000000,0.000000}%
\pgfsetstrokecolor{textcolor}%
\pgfsetfillcolor{textcolor}%
\pgftext[x=0.586419in, y=2.579130in, left, base]{\color{textcolor}\rmfamily\fontsize{10.000000}{12.000000}\selectfont \(\displaystyle {0.025}\)}%
\end{pgfscope}%
\begin{pgfscope}%
\pgfpathrectangle{\pgfqpoint{1.000000in}{0.330000in}}{\pgfqpoint{6.200000in}{2.310000in}}%
\pgfusepath{clip}%
\pgfsetrectcap%
\pgfsetroundjoin%
\pgfsetlinewidth{1.505625pt}%
\definecolor{currentstroke}{rgb}{0.121569,0.466667,0.705882}%
\pgfsetstrokecolor{currentstroke}%
\pgfsetdash{}{0pt}%
\pgfpathmoveto{\pgfqpoint{1.281818in}{1.293992in}}%
\pgfpathlineto{\pgfqpoint{1.354079in}{1.328449in}}%
\pgfpathlineto{\pgfqpoint{1.426340in}{1.362889in}}%
\pgfpathlineto{\pgfqpoint{1.498601in}{1.397287in}}%
\pgfpathlineto{\pgfqpoint{1.570862in}{1.431617in}}%
\pgfpathlineto{\pgfqpoint{1.643124in}{1.465855in}}%
\pgfpathlineto{\pgfqpoint{1.715385in}{1.499972in}}%
\pgfpathlineto{\pgfqpoint{1.787646in}{1.533944in}}%
\pgfpathlineto{\pgfqpoint{1.859907in}{1.567743in}}%
\pgfpathlineto{\pgfqpoint{1.932168in}{1.601342in}}%
\pgfpathlineto{\pgfqpoint{2.004429in}{1.634713in}}%
\pgfpathlineto{\pgfqpoint{2.076690in}{1.667828in}}%
\pgfpathlineto{\pgfqpoint{2.148951in}{1.700658in}}%
\pgfpathlineto{\pgfqpoint{2.221212in}{1.733175in}}%
\pgfpathlineto{\pgfqpoint{2.293473in}{1.765349in}}%
\pgfpathlineto{\pgfqpoint{2.365734in}{1.797151in}}%
\pgfpathlineto{\pgfqpoint{2.437995in}{1.828550in}}%
\pgfpathlineto{\pgfqpoint{2.510256in}{1.859517in}}%
\pgfpathlineto{\pgfqpoint{2.582517in}{1.890019in}}%
\pgfpathlineto{\pgfqpoint{2.654779in}{1.920026in}}%
\pgfpathlineto{\pgfqpoint{2.727040in}{1.949507in}}%
\pgfpathlineto{\pgfqpoint{2.799301in}{1.978428in}}%
\pgfpathlineto{\pgfqpoint{2.871562in}{2.006758in}}%
\pgfpathlineto{\pgfqpoint{2.943823in}{2.034464in}}%
\pgfpathlineto{\pgfqpoint{3.016084in}{2.061512in}}%
\pgfpathlineto{\pgfqpoint{3.088345in}{2.087869in}}%
\pgfpathlineto{\pgfqpoint{3.160606in}{2.113501in}}%
\pgfpathlineto{\pgfqpoint{3.232867in}{2.138373in}}%
\pgfpathlineto{\pgfqpoint{3.305128in}{2.162450in}}%
\pgfpathlineto{\pgfqpoint{3.377389in}{2.185698in}}%
\pgfpathlineto{\pgfqpoint{3.449650in}{2.208080in}}%
\pgfpathlineto{\pgfqpoint{3.521911in}{2.229560in}}%
\pgfpathlineto{\pgfqpoint{3.594172in}{2.250103in}}%
\pgfpathlineto{\pgfqpoint{3.666434in}{2.269671in}}%
\pgfpathlineto{\pgfqpoint{3.738695in}{2.288226in}}%
\pgfpathlineto{\pgfqpoint{3.883217in}{2.322151in}}%
\pgfpathlineto{\pgfqpoint{3.955478in}{2.336083in}}%
\pgfpathlineto{\pgfqpoint{4.027739in}{2.348924in}}%
\pgfpathlineto{\pgfqpoint{4.100000in}{2.360708in}}%
\pgfpathlineto{\pgfqpoint{4.172261in}{2.371471in}}%
\pgfpathlineto{\pgfqpoint{4.244522in}{2.381246in}}%
\pgfpathlineto{\pgfqpoint{4.316783in}{2.390067in}}%
\pgfpathlineto{\pgfqpoint{4.389044in}{2.397966in}}%
\pgfpathlineto{\pgfqpoint{4.461305in}{2.404977in}}%
\pgfpathlineto{\pgfqpoint{4.533566in}{2.411132in}}%
\pgfpathlineto{\pgfqpoint{4.605828in}{2.416462in}}%
\pgfpathlineto{\pgfqpoint{4.678089in}{2.421000in}}%
\pgfpathlineto{\pgfqpoint{4.750350in}{2.424775in}}%
\pgfpathlineto{\pgfqpoint{4.822611in}{2.427821in}}%
\pgfpathlineto{\pgfqpoint{4.894872in}{2.430165in}}%
\pgfpathlineto{\pgfqpoint{4.967133in}{2.431839in}}%
\pgfpathlineto{\pgfqpoint{5.039394in}{2.432872in}}%
\pgfpathlineto{\pgfqpoint{5.111655in}{2.433293in}}%
\pgfpathlineto{\pgfqpoint{5.183916in}{2.433131in}}%
\pgfpathlineto{\pgfqpoint{5.256177in}{2.432415in}}%
\pgfpathlineto{\pgfqpoint{5.328438in}{2.431172in}}%
\pgfpathlineto{\pgfqpoint{5.400699in}{2.429429in}}%
\pgfpathlineto{\pgfqpoint{5.472960in}{2.427215in}}%
\pgfpathlineto{\pgfqpoint{5.545221in}{2.424555in}}%
\pgfpathlineto{\pgfqpoint{5.617483in}{2.421476in}}%
\pgfpathlineto{\pgfqpoint{5.689744in}{2.418005in}}%
\pgfpathlineto{\pgfqpoint{5.762005in}{2.414166in}}%
\pgfpathlineto{\pgfqpoint{5.834266in}{2.409986in}}%
\pgfpathlineto{\pgfqpoint{5.906527in}{2.405488in}}%
\pgfpathlineto{\pgfqpoint{5.978788in}{2.400697in}}%
\pgfpathlineto{\pgfqpoint{6.051049in}{2.395638in}}%
\pgfpathlineto{\pgfqpoint{6.123310in}{2.390333in}}%
\pgfpathlineto{\pgfqpoint{6.195571in}{2.384806in}}%
\pgfpathlineto{\pgfqpoint{6.267832in}{2.379080in}}%
\pgfpathlineto{\pgfqpoint{6.340093in}{2.373177in}}%
\pgfpathlineto{\pgfqpoint{6.412354in}{2.367120in}}%
\pgfpathlineto{\pgfqpoint{6.484615in}{2.360929in}}%
\pgfpathlineto{\pgfqpoint{6.556876in}{2.354627in}}%
\pgfpathlineto{\pgfqpoint{6.629138in}{2.348233in}}%
\pgfpathlineto{\pgfqpoint{6.701399in}{2.341769in}}%
\pgfpathlineto{\pgfqpoint{6.773660in}{2.335254in}}%
\pgfpathlineto{\pgfqpoint{6.918182in}{2.322151in}}%
\pgfusepath{stroke}%
\end{pgfscope}%
\begin{pgfscope}%
\pgfpathrectangle{\pgfqpoint{1.000000in}{0.330000in}}{\pgfqpoint{6.200000in}{2.310000in}}%
\pgfusepath{clip}%
\pgfsetrectcap%
\pgfsetroundjoin%
\pgfsetlinewidth{1.505625pt}%
\definecolor{currentstroke}{rgb}{1.000000,0.498039,0.054902}%
\pgfsetstrokecolor{currentstroke}%
\pgfsetdash{}{0pt}%
\pgfpathmoveto{\pgfqpoint{1.281818in}{0.633345in}}%
\pgfpathlineto{\pgfqpoint{1.354079in}{0.689948in}}%
\pgfpathlineto{\pgfqpoint{1.426340in}{0.746521in}}%
\pgfpathlineto{\pgfqpoint{1.498601in}{0.803021in}}%
\pgfpathlineto{\pgfqpoint{1.570862in}{0.859406in}}%
\pgfpathlineto{\pgfqpoint{1.643124in}{0.915634in}}%
\pgfpathlineto{\pgfqpoint{1.715385in}{0.971661in}}%
\pgfpathlineto{\pgfqpoint{1.787646in}{1.027445in}}%
\pgfpathlineto{\pgfqpoint{1.859907in}{1.082939in}}%
\pgfpathlineto{\pgfqpoint{1.932168in}{1.138100in}}%
\pgfpathlineto{\pgfqpoint{2.004429in}{1.192882in}}%
\pgfpathlineto{\pgfqpoint{2.076690in}{1.247239in}}%
\pgfpathlineto{\pgfqpoint{2.148951in}{1.301125in}}%
\pgfpathlineto{\pgfqpoint{2.221212in}{1.354492in}}%
\pgfpathlineto{\pgfqpoint{2.293473in}{1.407292in}}%
\pgfpathlineto{\pgfqpoint{2.365734in}{1.459478in}}%
\pgfpathlineto{\pgfqpoint{2.437995in}{1.511001in}}%
\pgfpathlineto{\pgfqpoint{2.510256in}{1.561811in}}%
\pgfpathlineto{\pgfqpoint{2.582517in}{1.611858in}}%
\pgfpathlineto{\pgfqpoint{2.654779in}{1.661093in}}%
\pgfpathlineto{\pgfqpoint{2.727040in}{1.709463in}}%
\pgfpathlineto{\pgfqpoint{2.799301in}{1.756918in}}%
\pgfpathlineto{\pgfqpoint{2.871562in}{1.803406in}}%
\pgfpathlineto{\pgfqpoint{2.943823in}{1.848873in}}%
\pgfpathlineto{\pgfqpoint{3.016084in}{1.893267in}}%
\pgfpathlineto{\pgfqpoint{3.088345in}{1.936534in}}%
\pgfpathlineto{\pgfqpoint{3.160606in}{1.978619in}}%
\pgfpathlineto{\pgfqpoint{3.232867in}{2.019468in}}%
\pgfpathlineto{\pgfqpoint{3.305128in}{2.059025in}}%
\pgfpathlineto{\pgfqpoint{3.377389in}{2.097235in}}%
\pgfpathlineto{\pgfqpoint{3.449650in}{2.134040in}}%
\pgfpathlineto{\pgfqpoint{3.521911in}{2.169385in}}%
\pgfpathlineto{\pgfqpoint{3.594172in}{2.203210in}}%
\pgfpathlineto{\pgfqpoint{3.666434in}{2.235458in}}%
\pgfpathlineto{\pgfqpoint{3.738695in}{2.266071in}}%
\pgfpathlineto{\pgfqpoint{3.883217in}{2.322151in}}%
\pgfpathlineto{\pgfqpoint{3.955478in}{2.345427in}}%
\pgfpathlineto{\pgfqpoint{4.027739in}{2.366937in}}%
\pgfpathlineto{\pgfqpoint{4.100000in}{2.386731in}}%
\pgfpathlineto{\pgfqpoint{4.172261in}{2.404862in}}%
\pgfpathlineto{\pgfqpoint{4.244522in}{2.421378in}}%
\pgfpathlineto{\pgfqpoint{4.316783in}{2.436330in}}%
\pgfpathlineto{\pgfqpoint{4.389044in}{2.449768in}}%
\pgfpathlineto{\pgfqpoint{4.461305in}{2.461741in}}%
\pgfpathlineto{\pgfqpoint{4.533566in}{2.472297in}}%
\pgfpathlineto{\pgfqpoint{4.605828in}{2.481485in}}%
\pgfpathlineto{\pgfqpoint{4.678089in}{2.489353in}}%
\pgfpathlineto{\pgfqpoint{4.750350in}{2.495949in}}%
\pgfpathlineto{\pgfqpoint{4.822611in}{2.501320in}}%
\pgfpathlineto{\pgfqpoint{4.894872in}{2.505513in}}%
\pgfpathlineto{\pgfqpoint{4.967133in}{2.508574in}}%
\pgfpathlineto{\pgfqpoint{5.039394in}{2.510548in}}%
\pgfpathlineto{\pgfqpoint{5.111655in}{2.511483in}}%
\pgfpathlineto{\pgfqpoint{5.183916in}{2.511423in}}%
\pgfpathlineto{\pgfqpoint{5.256177in}{2.510412in}}%
\pgfpathlineto{\pgfqpoint{5.328438in}{2.508496in}}%
\pgfpathlineto{\pgfqpoint{5.400699in}{2.505719in}}%
\pgfpathlineto{\pgfqpoint{5.472960in}{2.502123in}}%
\pgfpathlineto{\pgfqpoint{5.545221in}{2.497753in}}%
\pgfpathlineto{\pgfqpoint{5.617483in}{2.492650in}}%
\pgfpathlineto{\pgfqpoint{5.689744in}{2.486859in}}%
\pgfpathlineto{\pgfqpoint{5.762005in}{2.480420in}}%
\pgfpathlineto{\pgfqpoint{5.834266in}{2.473375in}}%
\pgfpathlineto{\pgfqpoint{5.906527in}{2.465766in}}%
\pgfpathlineto{\pgfqpoint{5.978788in}{2.457634in}}%
\pgfpathlineto{\pgfqpoint{6.051049in}{2.449018in}}%
\pgfpathlineto{\pgfqpoint{6.123310in}{2.439960in}}%
\pgfpathlineto{\pgfqpoint{6.195571in}{2.430498in}}%
\pgfpathlineto{\pgfqpoint{6.267832in}{2.420672in}}%
\pgfpathlineto{\pgfqpoint{6.340093in}{2.410522in}}%
\pgfpathlineto{\pgfqpoint{6.412354in}{2.400085in}}%
\pgfpathlineto{\pgfqpoint{6.484615in}{2.389399in}}%
\pgfpathlineto{\pgfqpoint{6.556876in}{2.378502in}}%
\pgfpathlineto{\pgfqpoint{6.629138in}{2.367433in}}%
\pgfpathlineto{\pgfqpoint{6.701399in}{2.356226in}}%
\pgfpathlineto{\pgfqpoint{6.773660in}{2.344920in}}%
\pgfpathlineto{\pgfqpoint{6.918182in}{2.322151in}}%
\pgfusepath{stroke}%
\end{pgfscope}%
\begin{pgfscope}%
\pgfpathrectangle{\pgfqpoint{1.000000in}{0.330000in}}{\pgfqpoint{6.200000in}{2.310000in}}%
\pgfusepath{clip}%
\pgfsetrectcap%
\pgfsetroundjoin%
\pgfsetlinewidth{1.505625pt}%
\definecolor{currentstroke}{rgb}{0.172549,0.627451,0.172549}%
\pgfsetstrokecolor{currentstroke}%
\pgfsetdash{}{0pt}%
\pgfpathmoveto{\pgfqpoint{1.281818in}{1.293992in}}%
\pgfpathlineto{\pgfqpoint{1.354079in}{1.328449in}}%
\pgfpathlineto{\pgfqpoint{1.426340in}{1.362889in}}%
\pgfpathlineto{\pgfqpoint{1.498601in}{1.397287in}}%
\pgfpathlineto{\pgfqpoint{1.570862in}{1.431617in}}%
\pgfpathlineto{\pgfqpoint{1.643124in}{1.465855in}}%
\pgfpathlineto{\pgfqpoint{1.715385in}{1.499972in}}%
\pgfpathlineto{\pgfqpoint{1.787646in}{1.533944in}}%
\pgfpathlineto{\pgfqpoint{1.859907in}{1.567743in}}%
\pgfpathlineto{\pgfqpoint{1.932168in}{1.601342in}}%
\pgfpathlineto{\pgfqpoint{2.004429in}{1.634713in}}%
\pgfpathlineto{\pgfqpoint{2.076690in}{1.667828in}}%
\pgfpathlineto{\pgfqpoint{2.148951in}{1.700658in}}%
\pgfpathlineto{\pgfqpoint{2.221212in}{1.733175in}}%
\pgfpathlineto{\pgfqpoint{2.293473in}{1.765349in}}%
\pgfpathlineto{\pgfqpoint{2.365734in}{1.797151in}}%
\pgfpathlineto{\pgfqpoint{2.437995in}{1.828550in}}%
\pgfpathlineto{\pgfqpoint{2.510256in}{1.859517in}}%
\pgfpathlineto{\pgfqpoint{2.582517in}{1.890019in}}%
\pgfpathlineto{\pgfqpoint{2.654779in}{1.920026in}}%
\pgfpathlineto{\pgfqpoint{2.727040in}{1.949507in}}%
\pgfpathlineto{\pgfqpoint{2.799301in}{1.978428in}}%
\pgfpathlineto{\pgfqpoint{2.871562in}{2.006758in}}%
\pgfpathlineto{\pgfqpoint{2.943823in}{2.034464in}}%
\pgfpathlineto{\pgfqpoint{3.016084in}{2.061512in}}%
\pgfpathlineto{\pgfqpoint{3.088345in}{2.087869in}}%
\pgfpathlineto{\pgfqpoint{3.160606in}{2.113501in}}%
\pgfpathlineto{\pgfqpoint{3.232867in}{2.138373in}}%
\pgfpathlineto{\pgfqpoint{3.305128in}{2.162450in}}%
\pgfpathlineto{\pgfqpoint{3.377389in}{2.185698in}}%
\pgfpathlineto{\pgfqpoint{3.449650in}{2.208080in}}%
\pgfpathlineto{\pgfqpoint{3.521911in}{2.229560in}}%
\pgfpathlineto{\pgfqpoint{3.594172in}{2.250103in}}%
\pgfpathlineto{\pgfqpoint{3.666434in}{2.269671in}}%
\pgfpathlineto{\pgfqpoint{3.738695in}{2.288226in}}%
\pgfpathlineto{\pgfqpoint{3.883217in}{2.322151in}}%
\pgfpathlineto{\pgfqpoint{3.955478in}{2.336083in}}%
\pgfpathlineto{\pgfqpoint{4.027739in}{2.348924in}}%
\pgfpathlineto{\pgfqpoint{4.100000in}{2.360708in}}%
\pgfpathlineto{\pgfqpoint{4.172261in}{2.371471in}}%
\pgfpathlineto{\pgfqpoint{4.244522in}{2.381246in}}%
\pgfpathlineto{\pgfqpoint{4.316783in}{2.390067in}}%
\pgfpathlineto{\pgfqpoint{4.389044in}{2.397966in}}%
\pgfpathlineto{\pgfqpoint{4.461305in}{2.404977in}}%
\pgfpathlineto{\pgfqpoint{4.533566in}{2.411132in}}%
\pgfpathlineto{\pgfqpoint{4.605828in}{2.416462in}}%
\pgfpathlineto{\pgfqpoint{4.678089in}{2.421000in}}%
\pgfpathlineto{\pgfqpoint{4.750350in}{2.424775in}}%
\pgfpathlineto{\pgfqpoint{4.822611in}{2.427821in}}%
\pgfpathlineto{\pgfqpoint{4.894872in}{2.430165in}}%
\pgfpathlineto{\pgfqpoint{4.967133in}{2.431839in}}%
\pgfpathlineto{\pgfqpoint{5.039394in}{2.432872in}}%
\pgfpathlineto{\pgfqpoint{5.111655in}{2.433293in}}%
\pgfpathlineto{\pgfqpoint{5.183916in}{2.433131in}}%
\pgfpathlineto{\pgfqpoint{5.256177in}{2.432415in}}%
\pgfpathlineto{\pgfqpoint{5.328438in}{2.431172in}}%
\pgfpathlineto{\pgfqpoint{5.400699in}{2.429429in}}%
\pgfpathlineto{\pgfqpoint{5.472960in}{2.427215in}}%
\pgfpathlineto{\pgfqpoint{5.545221in}{2.424555in}}%
\pgfpathlineto{\pgfqpoint{5.617483in}{2.421476in}}%
\pgfpathlineto{\pgfqpoint{5.689744in}{2.418005in}}%
\pgfpathlineto{\pgfqpoint{5.762005in}{2.414166in}}%
\pgfpathlineto{\pgfqpoint{5.834266in}{2.409986in}}%
\pgfpathlineto{\pgfqpoint{5.906527in}{2.405488in}}%
\pgfpathlineto{\pgfqpoint{5.978788in}{2.400697in}}%
\pgfpathlineto{\pgfqpoint{6.051049in}{2.395638in}}%
\pgfpathlineto{\pgfqpoint{6.123310in}{2.390333in}}%
\pgfpathlineto{\pgfqpoint{6.195571in}{2.384806in}}%
\pgfpathlineto{\pgfqpoint{6.267832in}{2.379080in}}%
\pgfpathlineto{\pgfqpoint{6.340093in}{2.373177in}}%
\pgfpathlineto{\pgfqpoint{6.412354in}{2.367120in}}%
\pgfpathlineto{\pgfqpoint{6.484615in}{2.360929in}}%
\pgfpathlineto{\pgfqpoint{6.556876in}{2.354627in}}%
\pgfpathlineto{\pgfqpoint{6.629138in}{2.348233in}}%
\pgfpathlineto{\pgfqpoint{6.701399in}{2.341769in}}%
\pgfpathlineto{\pgfqpoint{6.773660in}{2.335254in}}%
\pgfpathlineto{\pgfqpoint{6.918182in}{2.322151in}}%
\pgfusepath{stroke}%
\end{pgfscope}%
\begin{pgfscope}%
\pgfpathrectangle{\pgfqpoint{1.000000in}{0.330000in}}{\pgfqpoint{6.200000in}{2.310000in}}%
\pgfusepath{clip}%
\pgfsetrectcap%
\pgfsetroundjoin%
\pgfsetlinewidth{1.505625pt}%
\definecolor{currentstroke}{rgb}{0.839216,0.152941,0.156863}%
\pgfsetstrokecolor{currentstroke}%
\pgfsetdash{}{0pt}%
\pgfpathmoveto{\pgfqpoint{1.281818in}{0.458183in}}%
\pgfpathlineto{\pgfqpoint{1.354079in}{0.520781in}}%
\pgfpathlineto{\pgfqpoint{1.426340in}{0.583345in}}%
\pgfpathlineto{\pgfqpoint{1.498601in}{0.645827in}}%
\pgfpathlineto{\pgfqpoint{1.570862in}{0.708181in}}%
\pgfpathlineto{\pgfqpoint{1.643124in}{0.770360in}}%
\pgfpathlineto{\pgfqpoint{1.715385in}{0.832313in}}%
\pgfpathlineto{\pgfqpoint{1.787646in}{0.893993in}}%
\pgfpathlineto{\pgfqpoint{1.859907in}{0.955351in}}%
\pgfpathlineto{\pgfqpoint{1.932168in}{1.016335in}}%
\pgfpathlineto{\pgfqpoint{2.004429in}{1.076896in}}%
\pgfpathlineto{\pgfqpoint{2.076690in}{1.136982in}}%
\pgfpathlineto{\pgfqpoint{2.148951in}{1.196540in}}%
\pgfpathlineto{\pgfqpoint{2.221212in}{1.255519in}}%
\pgfpathlineto{\pgfqpoint{2.293473in}{1.313866in}}%
\pgfpathlineto{\pgfqpoint{2.365734in}{1.371525in}}%
\pgfpathlineto{\pgfqpoint{2.437995in}{1.428444in}}%
\pgfpathlineto{\pgfqpoint{2.510256in}{1.484567in}}%
\pgfpathlineto{\pgfqpoint{2.582517in}{1.539839in}}%
\pgfpathlineto{\pgfqpoint{2.654779in}{1.594203in}}%
\pgfpathlineto{\pgfqpoint{2.727040in}{1.647603in}}%
\pgfpathlineto{\pgfqpoint{2.799301in}{1.699980in}}%
\pgfpathlineto{\pgfqpoint{2.871562in}{1.751278in}}%
\pgfpathlineto{\pgfqpoint{2.943823in}{1.801438in}}%
\pgfpathlineto{\pgfqpoint{3.016084in}{1.850400in}}%
\pgfpathlineto{\pgfqpoint{3.088345in}{1.898105in}}%
\pgfpathlineto{\pgfqpoint{3.160606in}{1.944491in}}%
\pgfpathlineto{\pgfqpoint{3.232867in}{1.989499in}}%
\pgfpathlineto{\pgfqpoint{3.305128in}{2.033066in}}%
\pgfpathlineto{\pgfqpoint{3.377389in}{2.075131in}}%
\pgfpathlineto{\pgfqpoint{3.449650in}{2.115630in}}%
\pgfpathlineto{\pgfqpoint{3.521911in}{2.154500in}}%
\pgfpathlineto{\pgfqpoint{3.594172in}{2.191677in}}%
\pgfpathlineto{\pgfqpoint{3.666434in}{2.227096in}}%
\pgfpathlineto{\pgfqpoint{3.738695in}{2.260692in}}%
\pgfpathlineto{\pgfqpoint{3.883217in}{2.322151in}}%
\pgfpathlineto{\pgfqpoint{3.955478in}{2.347515in}}%
\pgfpathlineto{\pgfqpoint{4.027739in}{2.370919in}}%
\pgfpathlineto{\pgfqpoint{4.100000in}{2.392421in}}%
\pgfpathlineto{\pgfqpoint{4.172261in}{2.412081in}}%
\pgfpathlineto{\pgfqpoint{4.244522in}{2.429959in}}%
\pgfpathlineto{\pgfqpoint{4.316783in}{2.446112in}}%
\pgfpathlineto{\pgfqpoint{4.389044in}{2.460600in}}%
\pgfpathlineto{\pgfqpoint{4.461305in}{2.473477in}}%
\pgfpathlineto{\pgfqpoint{4.533566in}{2.484803in}}%
\pgfpathlineto{\pgfqpoint{4.605828in}{2.494631in}}%
\pgfpathlineto{\pgfqpoint{4.678089in}{2.503019in}}%
\pgfpathlineto{\pgfqpoint{4.750350in}{2.510019in}}%
\pgfpathlineto{\pgfqpoint{4.822611in}{2.515688in}}%
\pgfpathlineto{\pgfqpoint{4.894872in}{2.520077in}}%
\pgfpathlineto{\pgfqpoint{4.967133in}{2.523241in}}%
\pgfpathlineto{\pgfqpoint{5.039394in}{2.525231in}}%
\pgfpathlineto{\pgfqpoint{5.111655in}{2.526099in}}%
\pgfpathlineto{\pgfqpoint{5.183916in}{2.525896in}}%
\pgfpathlineto{\pgfqpoint{5.256177in}{2.524674in}}%
\pgfpathlineto{\pgfqpoint{5.328438in}{2.522481in}}%
\pgfpathlineto{\pgfqpoint{5.400699in}{2.519367in}}%
\pgfpathlineto{\pgfqpoint{5.472960in}{2.515381in}}%
\pgfpathlineto{\pgfqpoint{5.545221in}{2.510571in}}%
\pgfpathlineto{\pgfqpoint{5.617483in}{2.504984in}}%
\pgfpathlineto{\pgfqpoint{5.689744in}{2.498667in}}%
\pgfpathlineto{\pgfqpoint{5.762005in}{2.491667in}}%
\pgfpathlineto{\pgfqpoint{5.834266in}{2.484030in}}%
\pgfpathlineto{\pgfqpoint{5.906527in}{2.475799in}}%
\pgfpathlineto{\pgfqpoint{5.978788in}{2.467021in}}%
\pgfpathlineto{\pgfqpoint{6.051049in}{2.457738in}}%
\pgfpathlineto{\pgfqpoint{6.123310in}{2.447994in}}%
\pgfpathlineto{\pgfqpoint{6.195571in}{2.437832in}}%
\pgfpathlineto{\pgfqpoint{6.267832in}{2.427293in}}%
\pgfpathlineto{\pgfqpoint{6.340093in}{2.416420in}}%
\pgfpathlineto{\pgfqpoint{6.412354in}{2.405253in}}%
\pgfpathlineto{\pgfqpoint{6.484615in}{2.393831in}}%
\pgfpathlineto{\pgfqpoint{6.556876in}{2.382196in}}%
\pgfpathlineto{\pgfqpoint{6.629138in}{2.370385in}}%
\pgfpathlineto{\pgfqpoint{6.701399in}{2.358438in}}%
\pgfpathlineto{\pgfqpoint{6.773660in}{2.346392in}}%
\pgfpathlineto{\pgfqpoint{6.918182in}{2.322151in}}%
\pgfusepath{stroke}%
\end{pgfscope}%
\begin{pgfscope}%
\pgfpathrectangle{\pgfqpoint{1.000000in}{0.330000in}}{\pgfqpoint{6.200000in}{2.310000in}}%
\pgfusepath{clip}%
\pgfsetrectcap%
\pgfsetroundjoin%
\pgfsetlinewidth{1.505625pt}%
\definecolor{currentstroke}{rgb}{0.580392,0.403922,0.741176}%
\pgfsetstrokecolor{currentstroke}%
\pgfsetdash{}{0pt}%
\pgfpathmoveto{\pgfqpoint{1.281818in}{1.277589in}}%
\pgfpathlineto{\pgfqpoint{1.354079in}{1.312603in}}%
\pgfpathlineto{\pgfqpoint{1.426340in}{1.347600in}}%
\pgfpathlineto{\pgfqpoint{1.498601in}{1.382553in}}%
\pgfpathlineto{\pgfqpoint{1.570862in}{1.417438in}}%
\pgfpathlineto{\pgfqpoint{1.643124in}{1.452227in}}%
\pgfpathlineto{\pgfqpoint{1.715385in}{1.486895in}}%
\pgfpathlineto{\pgfqpoint{1.787646in}{1.521414in}}%
\pgfpathlineto{\pgfqpoint{1.859907in}{1.555757in}}%
\pgfpathlineto{\pgfqpoint{1.932168in}{1.589895in}}%
\pgfpathlineto{\pgfqpoint{2.004429in}{1.623802in}}%
\pgfpathlineto{\pgfqpoint{2.076690in}{1.657448in}}%
\pgfpathlineto{\pgfqpoint{2.148951in}{1.690804in}}%
\pgfpathlineto{\pgfqpoint{2.221212in}{1.723842in}}%
\pgfpathlineto{\pgfqpoint{2.293473in}{1.756530in}}%
\pgfpathlineto{\pgfqpoint{2.365734in}{1.788840in}}%
\pgfpathlineto{\pgfqpoint{2.437995in}{1.820740in}}%
\pgfpathlineto{\pgfqpoint{2.510256in}{1.852199in}}%
\pgfpathlineto{\pgfqpoint{2.582517in}{1.883187in}}%
\pgfpathlineto{\pgfqpoint{2.654779in}{1.913671in}}%
\pgfpathlineto{\pgfqpoint{2.727040in}{1.943619in}}%
\pgfpathlineto{\pgfqpoint{2.799301in}{1.972998in}}%
\pgfpathlineto{\pgfqpoint{2.871562in}{2.001777in}}%
\pgfpathlineto{\pgfqpoint{2.943823in}{2.029921in}}%
\pgfpathlineto{\pgfqpoint{3.016084in}{2.057397in}}%
\pgfpathlineto{\pgfqpoint{3.088345in}{2.084170in}}%
\pgfpathlineto{\pgfqpoint{3.160606in}{2.110206in}}%
\pgfpathlineto{\pgfqpoint{3.232867in}{2.135470in}}%
\pgfpathlineto{\pgfqpoint{3.305128in}{2.159927in}}%
\pgfpathlineto{\pgfqpoint{3.377389in}{2.183541in}}%
\pgfpathlineto{\pgfqpoint{3.449650in}{2.206275in}}%
\pgfpathlineto{\pgfqpoint{3.521911in}{2.228095in}}%
\pgfpathlineto{\pgfqpoint{3.594172in}{2.248961in}}%
\pgfpathlineto{\pgfqpoint{3.666434in}{2.268838in}}%
\pgfpathlineto{\pgfqpoint{3.738695in}{2.287688in}}%
\pgfpathlineto{\pgfqpoint{3.883217in}{2.322151in}}%
\pgfpathlineto{\pgfqpoint{3.955478in}{2.336311in}}%
\pgfpathlineto{\pgfqpoint{4.027739in}{2.349364in}}%
\pgfpathlineto{\pgfqpoint{4.100000in}{2.361344in}}%
\pgfpathlineto{\pgfqpoint{4.172261in}{2.372287in}}%
\pgfpathlineto{\pgfqpoint{4.244522in}{2.382227in}}%
\pgfpathlineto{\pgfqpoint{4.316783in}{2.391197in}}%
\pgfpathlineto{\pgfqpoint{4.389044in}{2.399232in}}%
\pgfpathlineto{\pgfqpoint{4.461305in}{2.406364in}}%
\pgfpathlineto{\pgfqpoint{4.533566in}{2.412626in}}%
\pgfpathlineto{\pgfqpoint{4.605828in}{2.418050in}}%
\pgfpathlineto{\pgfqpoint{4.678089in}{2.422669in}}%
\pgfpathlineto{\pgfqpoint{4.750350in}{2.426514in}}%
\pgfpathlineto{\pgfqpoint{4.822611in}{2.429616in}}%
\pgfpathlineto{\pgfqpoint{4.894872in}{2.432006in}}%
\pgfpathlineto{\pgfqpoint{4.967133in}{2.433713in}}%
\pgfpathlineto{\pgfqpoint{5.039394in}{2.434769in}}%
\pgfpathlineto{\pgfqpoint{5.111655in}{2.435203in}}%
\pgfpathlineto{\pgfqpoint{5.183916in}{2.435044in}}%
\pgfpathlineto{\pgfqpoint{5.256177in}{2.434320in}}%
\pgfpathlineto{\pgfqpoint{5.328438in}{2.433060in}}%
\pgfpathlineto{\pgfqpoint{5.400699in}{2.431292in}}%
\pgfpathlineto{\pgfqpoint{5.472960in}{2.429044in}}%
\pgfpathlineto{\pgfqpoint{5.545221in}{2.426343in}}%
\pgfpathlineto{\pgfqpoint{5.617483in}{2.423215in}}%
\pgfpathlineto{\pgfqpoint{5.689744in}{2.419687in}}%
\pgfpathlineto{\pgfqpoint{5.762005in}{2.415785in}}%
\pgfpathlineto{\pgfqpoint{5.834266in}{2.411534in}}%
\pgfpathlineto{\pgfqpoint{5.906527in}{2.406960in}}%
\pgfpathlineto{\pgfqpoint{5.978788in}{2.402088in}}%
\pgfpathlineto{\pgfqpoint{6.051049in}{2.396942in}}%
\pgfpathlineto{\pgfqpoint{6.123310in}{2.391545in}}%
\pgfpathlineto{\pgfqpoint{6.195571in}{2.385922in}}%
\pgfpathlineto{\pgfqpoint{6.267832in}{2.380096in}}%
\pgfpathlineto{\pgfqpoint{6.340093in}{2.374090in}}%
\pgfpathlineto{\pgfqpoint{6.412354in}{2.367925in}}%
\pgfpathlineto{\pgfqpoint{6.484615in}{2.361625in}}%
\pgfpathlineto{\pgfqpoint{6.556876in}{2.355210in}}%
\pgfpathlineto{\pgfqpoint{6.629138in}{2.348702in}}%
\pgfpathlineto{\pgfqpoint{6.701399in}{2.342122in}}%
\pgfpathlineto{\pgfqpoint{6.773660in}{2.335490in}}%
\pgfpathlineto{\pgfqpoint{6.918182in}{2.322151in}}%
\pgfusepath{stroke}%
\end{pgfscope}%
\begin{pgfscope}%
\pgfpathrectangle{\pgfqpoint{1.000000in}{0.330000in}}{\pgfqpoint{6.200000in}{2.310000in}}%
\pgfusepath{clip}%
\pgfsetrectcap%
\pgfsetroundjoin%
\pgfsetlinewidth{1.505625pt}%
\definecolor{currentstroke}{rgb}{0.549020,0.337255,0.294118}%
\pgfsetstrokecolor{currentstroke}%
\pgfsetdash{}{0pt}%
\pgfpathmoveto{\pgfqpoint{1.281818in}{1.186035in}}%
\pgfpathlineto{\pgfqpoint{1.354079in}{1.224110in}}%
\pgfpathlineto{\pgfqpoint{1.426340in}{1.262167in}}%
\pgfpathlineto{\pgfqpoint{1.498601in}{1.300176in}}%
\pgfpathlineto{\pgfqpoint{1.570862in}{1.338111in}}%
\pgfpathlineto{\pgfqpoint{1.643124in}{1.375943in}}%
\pgfpathlineto{\pgfqpoint{1.715385in}{1.413644in}}%
\pgfpathlineto{\pgfqpoint{1.787646in}{1.451182in}}%
\pgfpathlineto{\pgfqpoint{1.859907in}{1.488530in}}%
\pgfpathlineto{\pgfqpoint{1.932168in}{1.525657in}}%
\pgfpathlineto{\pgfqpoint{2.004429in}{1.562532in}}%
\pgfpathlineto{\pgfqpoint{2.076690in}{1.599124in}}%
\pgfpathlineto{\pgfqpoint{2.148951in}{1.635401in}}%
\pgfpathlineto{\pgfqpoint{2.221212in}{1.671332in}}%
\pgfpathlineto{\pgfqpoint{2.293473in}{1.706885in}}%
\pgfpathlineto{\pgfqpoint{2.365734in}{1.742026in}}%
\pgfpathlineto{\pgfqpoint{2.437995in}{1.776722in}}%
\pgfpathlineto{\pgfqpoint{2.510256in}{1.810940in}}%
\pgfpathlineto{\pgfqpoint{2.582517in}{1.844645in}}%
\pgfpathlineto{\pgfqpoint{2.654779in}{1.877803in}}%
\pgfpathlineto{\pgfqpoint{2.727040in}{1.910379in}}%
\pgfpathlineto{\pgfqpoint{2.799301in}{1.942337in}}%
\pgfpathlineto{\pgfqpoint{2.871562in}{1.973642in}}%
\pgfpathlineto{\pgfqpoint{2.943823in}{2.004257in}}%
\pgfpathlineto{\pgfqpoint{3.016084in}{2.034145in}}%
\pgfpathlineto{\pgfqpoint{3.088345in}{2.063269in}}%
\pgfpathlineto{\pgfqpoint{3.160606in}{2.091592in}}%
\pgfpathlineto{\pgfqpoint{3.232867in}{2.119076in}}%
\pgfpathlineto{\pgfqpoint{3.305128in}{2.145681in}}%
\pgfpathlineto{\pgfqpoint{3.377389in}{2.171370in}}%
\pgfpathlineto{\pgfqpoint{3.449650in}{2.196102in}}%
\pgfpathlineto{\pgfqpoint{3.521911in}{2.219838in}}%
\pgfpathlineto{\pgfqpoint{3.594172in}{2.242538in}}%
\pgfpathlineto{\pgfqpoint{3.666434in}{2.264160in}}%
\pgfpathlineto{\pgfqpoint{3.738695in}{2.284664in}}%
\pgfpathlineto{\pgfqpoint{3.883217in}{2.322151in}}%
\pgfpathlineto{\pgfqpoint{3.955478in}{2.337546in}}%
\pgfpathlineto{\pgfqpoint{4.027739in}{2.351735in}}%
\pgfpathlineto{\pgfqpoint{4.100000in}{2.364757in}}%
\pgfpathlineto{\pgfqpoint{4.172261in}{2.376650in}}%
\pgfpathlineto{\pgfqpoint{4.244522in}{2.387451in}}%
\pgfpathlineto{\pgfqpoint{4.316783in}{2.397198in}}%
\pgfpathlineto{\pgfqpoint{4.389044in}{2.405927in}}%
\pgfpathlineto{\pgfqpoint{4.461305in}{2.413674in}}%
\pgfpathlineto{\pgfqpoint{4.533566in}{2.420475in}}%
\pgfpathlineto{\pgfqpoint{4.605828in}{2.426365in}}%
\pgfpathlineto{\pgfqpoint{4.678089in}{2.431379in}}%
\pgfpathlineto{\pgfqpoint{4.750350in}{2.435551in}}%
\pgfpathlineto{\pgfqpoint{4.822611in}{2.438916in}}%
\pgfpathlineto{\pgfqpoint{4.894872in}{2.441507in}}%
\pgfpathlineto{\pgfqpoint{4.967133in}{2.443356in}}%
\pgfpathlineto{\pgfqpoint{5.039394in}{2.444498in}}%
\pgfpathlineto{\pgfqpoint{5.111655in}{2.444963in}}%
\pgfpathlineto{\pgfqpoint{5.183916in}{2.444784in}}%
\pgfpathlineto{\pgfqpoint{5.256177in}{2.443993in}}%
\pgfpathlineto{\pgfqpoint{5.328438in}{2.442619in}}%
\pgfpathlineto{\pgfqpoint{5.400699in}{2.440693in}}%
\pgfpathlineto{\pgfqpoint{5.472960in}{2.438246in}}%
\pgfpathlineto{\pgfqpoint{5.545221in}{2.435307in}}%
\pgfpathlineto{\pgfqpoint{5.617483in}{2.431906in}}%
\pgfpathlineto{\pgfqpoint{5.689744in}{2.428070in}}%
\pgfpathlineto{\pgfqpoint{5.762005in}{2.423828in}}%
\pgfpathlineto{\pgfqpoint{5.834266in}{2.419209in}}%
\pgfpathlineto{\pgfqpoint{5.906527in}{2.414238in}}%
\pgfpathlineto{\pgfqpoint{5.978788in}{2.408945in}}%
\pgfpathlineto{\pgfqpoint{6.051049in}{2.403354in}}%
\pgfpathlineto{\pgfqpoint{6.123310in}{2.397492in}}%
\pgfpathlineto{\pgfqpoint{6.195571in}{2.391385in}}%
\pgfpathlineto{\pgfqpoint{6.267832in}{2.385058in}}%
\pgfpathlineto{\pgfqpoint{6.340093in}{2.378535in}}%
\pgfpathlineto{\pgfqpoint{6.412354in}{2.371842in}}%
\pgfpathlineto{\pgfqpoint{6.484615in}{2.365001in}}%
\pgfpathlineto{\pgfqpoint{6.556876in}{2.358037in}}%
\pgfpathlineto{\pgfqpoint{6.629138in}{2.350972in}}%
\pgfpathlineto{\pgfqpoint{6.701399in}{2.343829in}}%
\pgfpathlineto{\pgfqpoint{6.773660in}{2.336630in}}%
\pgfpathlineto{\pgfqpoint{6.918182in}{2.322151in}}%
\pgfusepath{stroke}%
\end{pgfscope}%
\begin{pgfscope}%
\pgfpathrectangle{\pgfqpoint{1.000000in}{0.330000in}}{\pgfqpoint{6.200000in}{2.310000in}}%
\pgfusepath{clip}%
\pgfsetrectcap%
\pgfsetroundjoin%
\pgfsetlinewidth{1.505625pt}%
\definecolor{currentstroke}{rgb}{0.890196,0.466667,0.760784}%
\pgfsetstrokecolor{currentstroke}%
\pgfsetdash{}{0pt}%
\pgfpathmoveto{\pgfqpoint{1.281818in}{0.435000in}}%
\pgfpathlineto{\pgfqpoint{1.354079in}{0.498242in}}%
\pgfpathlineto{\pgfqpoint{1.426340in}{0.561450in}}%
\pgfpathlineto{\pgfqpoint{1.498601in}{0.624577in}}%
\pgfpathlineto{\pgfqpoint{1.570862in}{0.687575in}}%
\pgfpathlineto{\pgfqpoint{1.643124in}{0.750397in}}%
\pgfpathlineto{\pgfqpoint{1.715385in}{0.812994in}}%
\pgfpathlineto{\pgfqpoint{1.787646in}{0.875319in}}%
\pgfpathlineto{\pgfqpoint{1.859907in}{0.937320in}}%
\pgfpathlineto{\pgfqpoint{1.932168in}{0.998948in}}%
\pgfpathlineto{\pgfqpoint{2.004429in}{1.060153in}}%
\pgfpathlineto{\pgfqpoint{2.076690in}{1.120883in}}%
\pgfpathlineto{\pgfqpoint{2.148951in}{1.181085in}}%
\pgfpathlineto{\pgfqpoint{2.221212in}{1.240708in}}%
\pgfpathlineto{\pgfqpoint{2.293473in}{1.299698in}}%
\pgfpathlineto{\pgfqpoint{2.365734in}{1.358002in}}%
\pgfpathlineto{\pgfqpoint{2.437995in}{1.415565in}}%
\pgfpathlineto{\pgfqpoint{2.510256in}{1.472332in}}%
\pgfpathlineto{\pgfqpoint{2.582517in}{1.528248in}}%
\pgfpathlineto{\pgfqpoint{2.654779in}{1.583256in}}%
\pgfpathlineto{\pgfqpoint{2.727040in}{1.637299in}}%
\pgfpathlineto{\pgfqpoint{2.799301in}{1.690321in}}%
\pgfpathlineto{\pgfqpoint{2.871562in}{1.742263in}}%
\pgfpathlineto{\pgfqpoint{2.943823in}{1.793067in}}%
\pgfpathlineto{\pgfqpoint{3.016084in}{1.842672in}}%
\pgfpathlineto{\pgfqpoint{3.088345in}{1.891021in}}%
\pgfpathlineto{\pgfqpoint{3.160606in}{1.938052in}}%
\pgfpathlineto{\pgfqpoint{3.232867in}{1.983703in}}%
\pgfpathlineto{\pgfqpoint{3.305128in}{2.027915in}}%
\pgfpathlineto{\pgfqpoint{3.377389in}{2.070623in}}%
\pgfpathlineto{\pgfqpoint{3.449650in}{2.111766in}}%
\pgfpathlineto{\pgfqpoint{3.521911in}{2.151280in}}%
\pgfpathlineto{\pgfqpoint{3.594172in}{2.189101in}}%
\pgfpathlineto{\pgfqpoint{3.666434in}{2.225164in}}%
\pgfpathlineto{\pgfqpoint{3.738695in}{2.259404in}}%
\pgfpathlineto{\pgfqpoint{3.883217in}{2.322151in}}%
\pgfpathlineto{\pgfqpoint{3.955478in}{2.348237in}}%
\pgfpathlineto{\pgfqpoint{4.027739in}{2.372355in}}%
\pgfpathlineto{\pgfqpoint{4.100000in}{2.394558in}}%
\pgfpathlineto{\pgfqpoint{4.172261in}{2.414904in}}%
\pgfpathlineto{\pgfqpoint{4.244522in}{2.433448in}}%
\pgfpathlineto{\pgfqpoint{4.316783in}{2.450244in}}%
\pgfpathlineto{\pgfqpoint{4.389044in}{2.465348in}}%
\pgfpathlineto{\pgfqpoint{4.461305in}{2.478813in}}%
\pgfpathlineto{\pgfqpoint{4.533566in}{2.490693in}}%
\pgfpathlineto{\pgfqpoint{4.605828in}{2.501042in}}%
\pgfpathlineto{\pgfqpoint{4.678089in}{2.509912in}}%
\pgfpathlineto{\pgfqpoint{4.750350in}{2.517356in}}%
\pgfpathlineto{\pgfqpoint{4.822611in}{2.523427in}}%
\pgfpathlineto{\pgfqpoint{4.894872in}{2.528175in}}%
\pgfpathlineto{\pgfqpoint{4.967133in}{2.531653in}}%
\pgfpathlineto{\pgfqpoint{5.039394in}{2.533911in}}%
\pgfpathlineto{\pgfqpoint{5.111655in}{2.535000in}}%
\pgfpathlineto{\pgfqpoint{5.183916in}{2.534970in}}%
\pgfpathlineto{\pgfqpoint{5.256177in}{2.533871in}}%
\pgfpathlineto{\pgfqpoint{5.328438in}{2.531753in}}%
\pgfpathlineto{\pgfqpoint{5.400699in}{2.528664in}}%
\pgfpathlineto{\pgfqpoint{5.472960in}{2.524653in}}%
\pgfpathlineto{\pgfqpoint{5.545221in}{2.519768in}}%
\pgfpathlineto{\pgfqpoint{5.617483in}{2.514057in}}%
\pgfpathlineto{\pgfqpoint{5.689744in}{2.507568in}}%
\pgfpathlineto{\pgfqpoint{5.762005in}{2.500347in}}%
\pgfpathlineto{\pgfqpoint{5.834266in}{2.492441in}}%
\pgfpathlineto{\pgfqpoint{5.906527in}{2.483896in}}%
\pgfpathlineto{\pgfqpoint{5.978788in}{2.474758in}}%
\pgfpathlineto{\pgfqpoint{6.051049in}{2.465073in}}%
\pgfpathlineto{\pgfqpoint{6.123310in}{2.454886in}}%
\pgfpathlineto{\pgfqpoint{6.195571in}{2.444241in}}%
\pgfpathlineto{\pgfqpoint{6.267832in}{2.433182in}}%
\pgfpathlineto{\pgfqpoint{6.340093in}{2.421754in}}%
\pgfpathlineto{\pgfqpoint{6.412354in}{2.409999in}}%
\pgfpathlineto{\pgfqpoint{6.484615in}{2.397962in}}%
\pgfpathlineto{\pgfqpoint{6.556876in}{2.385683in}}%
\pgfpathlineto{\pgfqpoint{6.629138in}{2.373207in}}%
\pgfpathlineto{\pgfqpoint{6.701399in}{2.360574in}}%
\pgfpathlineto{\pgfqpoint{6.773660in}{2.347827in}}%
\pgfpathlineto{\pgfqpoint{6.918182in}{2.322151in}}%
\pgfusepath{stroke}%
\end{pgfscope}%
\begin{pgfscope}%
\pgfpathrectangle{\pgfqpoint{1.000000in}{0.330000in}}{\pgfqpoint{6.200000in}{2.310000in}}%
\pgfusepath{clip}%
\pgfsetrectcap%
\pgfsetroundjoin%
\pgfsetlinewidth{1.505625pt}%
\definecolor{currentstroke}{rgb}{0.498039,0.498039,0.498039}%
\pgfsetstrokecolor{currentstroke}%
\pgfsetdash{}{0pt}%
\pgfpathmoveto{\pgfqpoint{1.281818in}{1.910887in}}%
\pgfpathlineto{\pgfqpoint{1.354079in}{1.924670in}}%
\pgfpathlineto{\pgfqpoint{1.426340in}{1.938446in}}%
\pgfpathlineto{\pgfqpoint{1.498601in}{1.952205in}}%
\pgfpathlineto{\pgfqpoint{1.570862in}{1.965938in}}%
\pgfpathlineto{\pgfqpoint{1.643124in}{1.979632in}}%
\pgfpathlineto{\pgfqpoint{1.715385in}{1.993279in}}%
\pgfpathlineto{\pgfqpoint{1.787646in}{2.006868in}}%
\pgfpathlineto{\pgfqpoint{1.859907in}{2.020388in}}%
\pgfpathlineto{\pgfqpoint{1.932168in}{2.033827in}}%
\pgfpathlineto{\pgfqpoint{2.004429in}{2.047176in}}%
\pgfpathlineto{\pgfqpoint{2.076690in}{2.060422in}}%
\pgfpathlineto{\pgfqpoint{2.148951in}{2.073554in}}%
\pgfpathlineto{\pgfqpoint{2.221212in}{2.086560in}}%
\pgfpathlineto{\pgfqpoint{2.293473in}{2.099430in}}%
\pgfpathlineto{\pgfqpoint{2.365734in}{2.112151in}}%
\pgfpathlineto{\pgfqpoint{2.437995in}{2.124711in}}%
\pgfpathlineto{\pgfqpoint{2.510256in}{2.137097in}}%
\pgfpathlineto{\pgfqpoint{2.582517in}{2.149298in}}%
\pgfpathlineto{\pgfqpoint{2.654779in}{2.161301in}}%
\pgfpathlineto{\pgfqpoint{2.727040in}{2.173093in}}%
\pgfpathlineto{\pgfqpoint{2.799301in}{2.184662in}}%
\pgfpathlineto{\pgfqpoint{2.871562in}{2.195994in}}%
\pgfpathlineto{\pgfqpoint{2.943823in}{2.207076in}}%
\pgfpathlineto{\pgfqpoint{3.016084in}{2.217895in}}%
\pgfpathlineto{\pgfqpoint{3.088345in}{2.228438in}}%
\pgfpathlineto{\pgfqpoint{3.160606in}{2.238691in}}%
\pgfpathlineto{\pgfqpoint{3.232867in}{2.248640in}}%
\pgfpathlineto{\pgfqpoint{3.305128in}{2.258271in}}%
\pgfpathlineto{\pgfqpoint{3.377389in}{2.267570in}}%
\pgfpathlineto{\pgfqpoint{3.449650in}{2.276522in}}%
\pgfpathlineto{\pgfqpoint{3.521911in}{2.285115in}}%
\pgfpathlineto{\pgfqpoint{3.594172in}{2.293332in}}%
\pgfpathlineto{\pgfqpoint{3.666434in}{2.301159in}}%
\pgfpathlineto{\pgfqpoint{3.738695in}{2.308581in}}%
\pgfpathlineto{\pgfqpoint{3.883217in}{2.322151in}}%
\pgfpathlineto{\pgfqpoint{3.955478in}{2.327724in}}%
\pgfpathlineto{\pgfqpoint{4.027739in}{2.332860in}}%
\pgfpathlineto{\pgfqpoint{4.100000in}{2.337574in}}%
\pgfpathlineto{\pgfqpoint{4.172261in}{2.341879in}}%
\pgfpathlineto{\pgfqpoint{4.244522in}{2.345789in}}%
\pgfpathlineto{\pgfqpoint{4.316783in}{2.349317in}}%
\pgfpathlineto{\pgfqpoint{4.389044in}{2.352477in}}%
\pgfpathlineto{\pgfqpoint{4.461305in}{2.355281in}}%
\pgfpathlineto{\pgfqpoint{4.533566in}{2.357743in}}%
\pgfpathlineto{\pgfqpoint{4.605828in}{2.359875in}}%
\pgfpathlineto{\pgfqpoint{4.678089in}{2.361690in}}%
\pgfpathlineto{\pgfqpoint{4.750350in}{2.363201in}}%
\pgfpathlineto{\pgfqpoint{4.822611in}{2.364419in}}%
\pgfpathlineto{\pgfqpoint{4.894872in}{2.365357in}}%
\pgfpathlineto{\pgfqpoint{4.967133in}{2.366026in}}%
\pgfpathlineto{\pgfqpoint{5.039394in}{2.366439in}}%
\pgfpathlineto{\pgfqpoint{5.111655in}{2.366608in}}%
\pgfpathlineto{\pgfqpoint{5.183916in}{2.366543in}}%
\pgfpathlineto{\pgfqpoint{5.256177in}{2.366256in}}%
\pgfpathlineto{\pgfqpoint{5.328438in}{2.365759in}}%
\pgfpathlineto{\pgfqpoint{5.400699in}{2.365062in}}%
\pgfpathlineto{\pgfqpoint{5.472960in}{2.364176in}}%
\pgfpathlineto{\pgfqpoint{5.545221in}{2.363112in}}%
\pgfpathlineto{\pgfqpoint{5.617483in}{2.361881in}}%
\pgfpathlineto{\pgfqpoint{5.689744in}{2.360493in}}%
\pgfpathlineto{\pgfqpoint{5.762005in}{2.358957in}}%
\pgfpathlineto{\pgfqpoint{5.834266in}{2.357285in}}%
\pgfpathlineto{\pgfqpoint{5.906527in}{2.355486in}}%
\pgfpathlineto{\pgfqpoint{5.978788in}{2.353569in}}%
\pgfpathlineto{\pgfqpoint{6.051049in}{2.351546in}}%
\pgfpathlineto{\pgfqpoint{6.123310in}{2.349424in}}%
\pgfpathlineto{\pgfqpoint{6.195571in}{2.347213in}}%
\pgfpathlineto{\pgfqpoint{6.267832in}{2.344923in}}%
\pgfpathlineto{\pgfqpoint{6.340093in}{2.342561in}}%
\pgfpathlineto{\pgfqpoint{6.412354in}{2.340138in}}%
\pgfpathlineto{\pgfqpoint{6.484615in}{2.337662in}}%
\pgfpathlineto{\pgfqpoint{6.556876in}{2.335141in}}%
\pgfpathlineto{\pgfqpoint{6.629138in}{2.332584in}}%
\pgfpathlineto{\pgfqpoint{6.701399in}{2.329998in}}%
\pgfpathlineto{\pgfqpoint{6.773660in}{2.327392in}}%
\pgfpathlineto{\pgfqpoint{6.918182in}{2.322151in}}%
\pgfusepath{stroke}%
\end{pgfscope}%
\begin{pgfscope}%
\pgfpathrectangle{\pgfqpoint{1.000000in}{0.330000in}}{\pgfqpoint{6.200000in}{2.310000in}}%
\pgfusepath{clip}%
\pgfsetrectcap%
\pgfsetroundjoin%
\pgfsetlinewidth{1.505625pt}%
\definecolor{currentstroke}{rgb}{0.737255,0.741176,0.133333}%
\pgfsetstrokecolor{currentstroke}%
\pgfsetdash{}{0pt}%
\pgfpathmoveto{\pgfqpoint{1.281818in}{1.910887in}}%
\pgfpathlineto{\pgfqpoint{1.354079in}{1.924670in}}%
\pgfpathlineto{\pgfqpoint{1.426340in}{1.938446in}}%
\pgfpathlineto{\pgfqpoint{1.498601in}{1.952205in}}%
\pgfpathlineto{\pgfqpoint{1.570862in}{1.965938in}}%
\pgfpathlineto{\pgfqpoint{1.643124in}{1.979632in}}%
\pgfpathlineto{\pgfqpoint{1.715385in}{1.993279in}}%
\pgfpathlineto{\pgfqpoint{1.787646in}{2.006868in}}%
\pgfpathlineto{\pgfqpoint{1.859907in}{2.020388in}}%
\pgfpathlineto{\pgfqpoint{1.932168in}{2.033827in}}%
\pgfpathlineto{\pgfqpoint{2.004429in}{2.047176in}}%
\pgfpathlineto{\pgfqpoint{2.076690in}{2.060422in}}%
\pgfpathlineto{\pgfqpoint{2.148951in}{2.073554in}}%
\pgfpathlineto{\pgfqpoint{2.221212in}{2.086560in}}%
\pgfpathlineto{\pgfqpoint{2.293473in}{2.099430in}}%
\pgfpathlineto{\pgfqpoint{2.365734in}{2.112151in}}%
\pgfpathlineto{\pgfqpoint{2.437995in}{2.124711in}}%
\pgfpathlineto{\pgfqpoint{2.510256in}{2.137097in}}%
\pgfpathlineto{\pgfqpoint{2.582517in}{2.149298in}}%
\pgfpathlineto{\pgfqpoint{2.654779in}{2.161301in}}%
\pgfpathlineto{\pgfqpoint{2.727040in}{2.173093in}}%
\pgfpathlineto{\pgfqpoint{2.799301in}{2.184662in}}%
\pgfpathlineto{\pgfqpoint{2.871562in}{2.195994in}}%
\pgfpathlineto{\pgfqpoint{2.943823in}{2.207076in}}%
\pgfpathlineto{\pgfqpoint{3.016084in}{2.217895in}}%
\pgfpathlineto{\pgfqpoint{3.088345in}{2.228438in}}%
\pgfpathlineto{\pgfqpoint{3.160606in}{2.238691in}}%
\pgfpathlineto{\pgfqpoint{3.232867in}{2.248640in}}%
\pgfpathlineto{\pgfqpoint{3.305128in}{2.258271in}}%
\pgfpathlineto{\pgfqpoint{3.377389in}{2.267570in}}%
\pgfpathlineto{\pgfqpoint{3.449650in}{2.276522in}}%
\pgfpathlineto{\pgfqpoint{3.521911in}{2.285115in}}%
\pgfpathlineto{\pgfqpoint{3.594172in}{2.293332in}}%
\pgfpathlineto{\pgfqpoint{3.666434in}{2.301159in}}%
\pgfpathlineto{\pgfqpoint{3.738695in}{2.308581in}}%
\pgfpathlineto{\pgfqpoint{3.883217in}{2.322151in}}%
\pgfpathlineto{\pgfqpoint{3.955478in}{2.327724in}}%
\pgfpathlineto{\pgfqpoint{4.027739in}{2.332860in}}%
\pgfpathlineto{\pgfqpoint{4.100000in}{2.337574in}}%
\pgfpathlineto{\pgfqpoint{4.172261in}{2.341879in}}%
\pgfpathlineto{\pgfqpoint{4.244522in}{2.345789in}}%
\pgfpathlineto{\pgfqpoint{4.316783in}{2.349317in}}%
\pgfpathlineto{\pgfqpoint{4.389044in}{2.352477in}}%
\pgfpathlineto{\pgfqpoint{4.461305in}{2.355281in}}%
\pgfpathlineto{\pgfqpoint{4.533566in}{2.357743in}}%
\pgfpathlineto{\pgfqpoint{4.605828in}{2.359875in}}%
\pgfpathlineto{\pgfqpoint{4.678089in}{2.361690in}}%
\pgfpathlineto{\pgfqpoint{4.750350in}{2.363201in}}%
\pgfpathlineto{\pgfqpoint{4.822611in}{2.364419in}}%
\pgfpathlineto{\pgfqpoint{4.894872in}{2.365357in}}%
\pgfpathlineto{\pgfqpoint{4.967133in}{2.366026in}}%
\pgfpathlineto{\pgfqpoint{5.039394in}{2.366439in}}%
\pgfpathlineto{\pgfqpoint{5.111655in}{2.366608in}}%
\pgfpathlineto{\pgfqpoint{5.183916in}{2.366543in}}%
\pgfpathlineto{\pgfqpoint{5.256177in}{2.366256in}}%
\pgfpathlineto{\pgfqpoint{5.328438in}{2.365759in}}%
\pgfpathlineto{\pgfqpoint{5.400699in}{2.365062in}}%
\pgfpathlineto{\pgfqpoint{5.472960in}{2.364176in}}%
\pgfpathlineto{\pgfqpoint{5.545221in}{2.363112in}}%
\pgfpathlineto{\pgfqpoint{5.617483in}{2.361881in}}%
\pgfpathlineto{\pgfqpoint{5.689744in}{2.360493in}}%
\pgfpathlineto{\pgfqpoint{5.762005in}{2.358957in}}%
\pgfpathlineto{\pgfqpoint{5.834266in}{2.357285in}}%
\pgfpathlineto{\pgfqpoint{5.906527in}{2.355486in}}%
\pgfpathlineto{\pgfqpoint{5.978788in}{2.353569in}}%
\pgfpathlineto{\pgfqpoint{6.051049in}{2.351546in}}%
\pgfpathlineto{\pgfqpoint{6.123310in}{2.349424in}}%
\pgfpathlineto{\pgfqpoint{6.195571in}{2.347213in}}%
\pgfpathlineto{\pgfqpoint{6.267832in}{2.344923in}}%
\pgfpathlineto{\pgfqpoint{6.340093in}{2.342561in}}%
\pgfpathlineto{\pgfqpoint{6.412354in}{2.340138in}}%
\pgfpathlineto{\pgfqpoint{6.484615in}{2.337662in}}%
\pgfpathlineto{\pgfqpoint{6.556876in}{2.335141in}}%
\pgfpathlineto{\pgfqpoint{6.629138in}{2.332584in}}%
\pgfpathlineto{\pgfqpoint{6.701399in}{2.329998in}}%
\pgfpathlineto{\pgfqpoint{6.773660in}{2.327392in}}%
\pgfpathlineto{\pgfqpoint{6.918182in}{2.322151in}}%
\pgfusepath{stroke}%
\end{pgfscope}%
\begin{pgfscope}%
\pgfpathrectangle{\pgfqpoint{1.000000in}{0.330000in}}{\pgfqpoint{6.200000in}{2.310000in}}%
\pgfusepath{clip}%
\pgfsetrectcap%
\pgfsetroundjoin%
\pgfsetlinewidth{1.505625pt}%
\definecolor{currentstroke}{rgb}{0.090196,0.745098,0.811765}%
\pgfsetstrokecolor{currentstroke}%
\pgfsetdash{}{0pt}%
\pgfpathmoveto{\pgfqpoint{1.281818in}{2.322151in}}%
\pgfpathlineto{\pgfqpoint{1.354079in}{2.322151in}}%
\pgfpathlineto{\pgfqpoint{1.426340in}{2.322151in}}%
\pgfpathlineto{\pgfqpoint{1.498601in}{2.322151in}}%
\pgfpathlineto{\pgfqpoint{1.570862in}{2.322151in}}%
\pgfpathlineto{\pgfqpoint{1.643124in}{2.322151in}}%
\pgfpathlineto{\pgfqpoint{1.715385in}{2.322151in}}%
\pgfpathlineto{\pgfqpoint{1.787646in}{2.322151in}}%
\pgfpathlineto{\pgfqpoint{1.859907in}{2.322151in}}%
\pgfpathlineto{\pgfqpoint{1.932168in}{2.322151in}}%
\pgfpathlineto{\pgfqpoint{2.004429in}{2.322151in}}%
\pgfpathlineto{\pgfqpoint{2.076690in}{2.322151in}}%
\pgfpathlineto{\pgfqpoint{2.148951in}{2.322151in}}%
\pgfpathlineto{\pgfqpoint{2.221212in}{2.322151in}}%
\pgfpathlineto{\pgfqpoint{2.293473in}{2.322151in}}%
\pgfpathlineto{\pgfqpoint{2.365734in}{2.322151in}}%
\pgfpathlineto{\pgfqpoint{2.437995in}{2.322151in}}%
\pgfpathlineto{\pgfqpoint{2.510256in}{2.322151in}}%
\pgfpathlineto{\pgfqpoint{2.582517in}{2.322151in}}%
\pgfpathlineto{\pgfqpoint{2.654779in}{2.322151in}}%
\pgfpathlineto{\pgfqpoint{2.727040in}{2.322151in}}%
\pgfpathlineto{\pgfqpoint{2.799301in}{2.322151in}}%
\pgfpathlineto{\pgfqpoint{2.871562in}{2.322151in}}%
\pgfpathlineto{\pgfqpoint{2.943823in}{2.322151in}}%
\pgfpathlineto{\pgfqpoint{3.016084in}{2.322151in}}%
\pgfpathlineto{\pgfqpoint{3.088345in}{2.322151in}}%
\pgfpathlineto{\pgfqpoint{3.160606in}{2.322151in}}%
\pgfpathlineto{\pgfqpoint{3.232867in}{2.322151in}}%
\pgfpathlineto{\pgfqpoint{3.305128in}{2.322151in}}%
\pgfpathlineto{\pgfqpoint{3.377389in}{2.322151in}}%
\pgfpathlineto{\pgfqpoint{3.449650in}{2.322151in}}%
\pgfpathlineto{\pgfqpoint{3.521911in}{2.322151in}}%
\pgfpathlineto{\pgfqpoint{3.594172in}{2.322151in}}%
\pgfpathlineto{\pgfqpoint{3.666434in}{2.322151in}}%
\pgfpathlineto{\pgfqpoint{3.738695in}{2.322151in}}%
\pgfpathlineto{\pgfqpoint{3.883217in}{2.322151in}}%
\pgfpathlineto{\pgfqpoint{3.955478in}{2.322151in}}%
\pgfpathlineto{\pgfqpoint{4.027739in}{2.322151in}}%
\pgfpathlineto{\pgfqpoint{4.100000in}{2.322151in}}%
\pgfpathlineto{\pgfqpoint{4.172261in}{2.322151in}}%
\pgfpathlineto{\pgfqpoint{4.244522in}{2.322151in}}%
\pgfpathlineto{\pgfqpoint{4.316783in}{2.322151in}}%
\pgfpathlineto{\pgfqpoint{4.389044in}{2.322151in}}%
\pgfpathlineto{\pgfqpoint{4.461305in}{2.322151in}}%
\pgfpathlineto{\pgfqpoint{4.533566in}{2.322151in}}%
\pgfpathlineto{\pgfqpoint{4.605828in}{2.322151in}}%
\pgfpathlineto{\pgfqpoint{4.678089in}{2.322151in}}%
\pgfpathlineto{\pgfqpoint{4.750350in}{2.322151in}}%
\pgfpathlineto{\pgfqpoint{4.822611in}{2.322151in}}%
\pgfpathlineto{\pgfqpoint{4.894872in}{2.322151in}}%
\pgfpathlineto{\pgfqpoint{4.967133in}{2.322151in}}%
\pgfpathlineto{\pgfqpoint{5.039394in}{2.322151in}}%
\pgfpathlineto{\pgfqpoint{5.111655in}{2.322151in}}%
\pgfpathlineto{\pgfqpoint{5.183916in}{2.322151in}}%
\pgfpathlineto{\pgfqpoint{5.256177in}{2.322151in}}%
\pgfpathlineto{\pgfqpoint{5.328438in}{2.322151in}}%
\pgfpathlineto{\pgfqpoint{5.400699in}{2.322151in}}%
\pgfpathlineto{\pgfqpoint{5.472960in}{2.322151in}}%
\pgfpathlineto{\pgfqpoint{5.545221in}{2.322151in}}%
\pgfpathlineto{\pgfqpoint{5.617483in}{2.322151in}}%
\pgfpathlineto{\pgfqpoint{5.689744in}{2.322151in}}%
\pgfpathlineto{\pgfqpoint{5.762005in}{2.322151in}}%
\pgfpathlineto{\pgfqpoint{5.834266in}{2.322151in}}%
\pgfpathlineto{\pgfqpoint{5.906527in}{2.322151in}}%
\pgfpathlineto{\pgfqpoint{5.978788in}{2.322151in}}%
\pgfpathlineto{\pgfqpoint{6.051049in}{2.322151in}}%
\pgfpathlineto{\pgfqpoint{6.123310in}{2.322151in}}%
\pgfpathlineto{\pgfqpoint{6.195571in}{2.322151in}}%
\pgfpathlineto{\pgfqpoint{6.267832in}{2.322151in}}%
\pgfpathlineto{\pgfqpoint{6.340093in}{2.322151in}}%
\pgfpathlineto{\pgfqpoint{6.412354in}{2.322151in}}%
\pgfpathlineto{\pgfqpoint{6.484615in}{2.322151in}}%
\pgfpathlineto{\pgfqpoint{6.556876in}{2.322151in}}%
\pgfpathlineto{\pgfqpoint{6.629138in}{2.322151in}}%
\pgfpathlineto{\pgfqpoint{6.701399in}{2.322151in}}%
\pgfpathlineto{\pgfqpoint{6.773660in}{2.322151in}}%
\pgfpathlineto{\pgfqpoint{6.918182in}{2.322151in}}%
\pgfusepath{stroke}%
\end{pgfscope}%
\begin{pgfscope}%
\pgfpathrectangle{\pgfqpoint{1.000000in}{0.330000in}}{\pgfqpoint{6.200000in}{2.310000in}}%
\pgfusepath{clip}%
\pgfsetrectcap%
\pgfsetroundjoin%
\pgfsetlinewidth{1.505625pt}%
\definecolor{currentstroke}{rgb}{0.121569,0.466667,0.705882}%
\pgfsetstrokecolor{currentstroke}%
\pgfsetdash{}{0pt}%
\pgfpathmoveto{\pgfqpoint{1.281818in}{2.300281in}}%
\pgfpathlineto{\pgfqpoint{1.354079in}{2.301023in}}%
\pgfpathlineto{\pgfqpoint{1.426340in}{2.301765in}}%
\pgfpathlineto{\pgfqpoint{1.498601in}{2.302506in}}%
\pgfpathlineto{\pgfqpoint{1.570862in}{2.303245in}}%
\pgfpathlineto{\pgfqpoint{1.643124in}{2.303981in}}%
\pgfpathlineto{\pgfqpoint{1.715385in}{2.304714in}}%
\pgfpathlineto{\pgfqpoint{1.787646in}{2.305444in}}%
\pgfpathlineto{\pgfqpoint{1.859907in}{2.306169in}}%
\pgfpathlineto{\pgfqpoint{1.932168in}{2.306889in}}%
\pgfpathlineto{\pgfqpoint{2.004429in}{2.307604in}}%
\pgfpathlineto{\pgfqpoint{2.076690in}{2.308312in}}%
\pgfpathlineto{\pgfqpoint{2.148951in}{2.309013in}}%
\pgfpathlineto{\pgfqpoint{2.221212in}{2.309707in}}%
\pgfpathlineto{\pgfqpoint{2.293473in}{2.310392in}}%
\pgfpathlineto{\pgfqpoint{2.365734in}{2.311069in}}%
\pgfpathlineto{\pgfqpoint{2.437995in}{2.311737in}}%
\pgfpathlineto{\pgfqpoint{2.510256in}{2.312394in}}%
\pgfpathlineto{\pgfqpoint{2.582517in}{2.313041in}}%
\pgfpathlineto{\pgfqpoint{2.654779in}{2.313677in}}%
\pgfpathlineto{\pgfqpoint{2.727040in}{2.314300in}}%
\pgfpathlineto{\pgfqpoint{2.799301in}{2.314912in}}%
\pgfpathlineto{\pgfqpoint{2.871562in}{2.315510in}}%
\pgfpathlineto{\pgfqpoint{2.943823in}{2.316094in}}%
\pgfpathlineto{\pgfqpoint{3.016084in}{2.316664in}}%
\pgfpathlineto{\pgfqpoint{3.088345in}{2.317218in}}%
\pgfpathlineto{\pgfqpoint{3.160606in}{2.317758in}}%
\pgfpathlineto{\pgfqpoint{3.232867in}{2.318281in}}%
\pgfpathlineto{\pgfqpoint{3.305128in}{2.318787in}}%
\pgfpathlineto{\pgfqpoint{3.377389in}{2.319275in}}%
\pgfpathlineto{\pgfqpoint{3.449650in}{2.319745in}}%
\pgfpathlineto{\pgfqpoint{3.521911in}{2.320197in}}%
\pgfpathlineto{\pgfqpoint{3.594172in}{2.320629in}}%
\pgfpathlineto{\pgfqpoint{3.666434in}{2.321041in}}%
\pgfpathlineto{\pgfqpoint{3.738695in}{2.321432in}}%
\pgfpathlineto{\pgfqpoint{3.883217in}{2.322151in}}%
\pgfpathlineto{\pgfqpoint{3.955478in}{2.322455in}}%
\pgfpathlineto{\pgfqpoint{4.027739in}{2.322738in}}%
\pgfpathlineto{\pgfqpoint{4.100000in}{2.322998in}}%
\pgfpathlineto{\pgfqpoint{4.172261in}{2.323238in}}%
\pgfpathlineto{\pgfqpoint{4.244522in}{2.323458in}}%
\pgfpathlineto{\pgfqpoint{4.316783in}{2.323658in}}%
\pgfpathlineto{\pgfqpoint{4.389044in}{2.323838in}}%
\pgfpathlineto{\pgfqpoint{4.461305in}{2.324000in}}%
\pgfpathlineto{\pgfqpoint{4.533566in}{2.324143in}}%
\pgfpathlineto{\pgfqpoint{4.605828in}{2.324268in}}%
\pgfpathlineto{\pgfqpoint{4.678089in}{2.324377in}}%
\pgfpathlineto{\pgfqpoint{4.750350in}{2.324469in}}%
\pgfpathlineto{\pgfqpoint{4.822611in}{2.324545in}}%
\pgfpathlineto{\pgfqpoint{4.894872in}{2.324605in}}%
\pgfpathlineto{\pgfqpoint{4.967133in}{2.324650in}}%
\pgfpathlineto{\pgfqpoint{5.039394in}{2.324681in}}%
\pgfpathlineto{\pgfqpoint{5.111655in}{2.324697in}}%
\pgfpathlineto{\pgfqpoint{5.183916in}{2.324701in}}%
\pgfpathlineto{\pgfqpoint{5.256177in}{2.324691in}}%
\pgfpathlineto{\pgfqpoint{5.328438in}{2.324669in}}%
\pgfpathlineto{\pgfqpoint{5.400699in}{2.324635in}}%
\pgfpathlineto{\pgfqpoint{5.472960in}{2.324590in}}%
\pgfpathlineto{\pgfqpoint{5.545221in}{2.324535in}}%
\pgfpathlineto{\pgfqpoint{5.617483in}{2.324469in}}%
\pgfpathlineto{\pgfqpoint{5.689744in}{2.324393in}}%
\pgfpathlineto{\pgfqpoint{5.762005in}{2.324309in}}%
\pgfpathlineto{\pgfqpoint{5.834266in}{2.324215in}}%
\pgfpathlineto{\pgfqpoint{5.906527in}{2.324114in}}%
\pgfpathlineto{\pgfqpoint{5.978788in}{2.324005in}}%
\pgfpathlineto{\pgfqpoint{6.051049in}{2.323889in}}%
\pgfpathlineto{\pgfqpoint{6.123310in}{2.323767in}}%
\pgfpathlineto{\pgfqpoint{6.195571in}{2.323639in}}%
\pgfpathlineto{\pgfqpoint{6.267832in}{2.323505in}}%
\pgfpathlineto{\pgfqpoint{6.340093in}{2.323367in}}%
\pgfpathlineto{\pgfqpoint{6.412354in}{2.323224in}}%
\pgfpathlineto{\pgfqpoint{6.484615in}{2.323078in}}%
\pgfpathlineto{\pgfqpoint{6.556876in}{2.322928in}}%
\pgfpathlineto{\pgfqpoint{6.629138in}{2.322776in}}%
\pgfpathlineto{\pgfqpoint{6.701399in}{2.322622in}}%
\pgfpathlineto{\pgfqpoint{6.773660in}{2.322466in}}%
\pgfpathlineto{\pgfqpoint{6.918182in}{2.322151in}}%
\pgfusepath{stroke}%
\end{pgfscope}%
\begin{pgfscope}%
\pgfpathrectangle{\pgfqpoint{1.000000in}{0.330000in}}{\pgfqpoint{6.200000in}{2.310000in}}%
\pgfusepath{clip}%
\pgfsetrectcap%
\pgfsetroundjoin%
\pgfsetlinewidth{1.505625pt}%
\definecolor{currentstroke}{rgb}{1.000000,0.498039,0.054902}%
\pgfsetstrokecolor{currentstroke}%
\pgfsetdash{}{0pt}%
\pgfpathmoveto{\pgfqpoint{1.281818in}{1.317174in}}%
\pgfpathlineto{\pgfqpoint{1.354079in}{1.350988in}}%
\pgfpathlineto{\pgfqpoint{1.426340in}{1.384784in}}%
\pgfpathlineto{\pgfqpoint{1.498601in}{1.418538in}}%
\pgfpathlineto{\pgfqpoint{1.570862in}{1.452224in}}%
\pgfpathlineto{\pgfqpoint{1.643124in}{1.485817in}}%
\pgfpathlineto{\pgfqpoint{1.715385in}{1.519291in}}%
\pgfpathlineto{\pgfqpoint{1.787646in}{1.552619in}}%
\pgfpathlineto{\pgfqpoint{1.859907in}{1.585774in}}%
\pgfpathlineto{\pgfqpoint{1.932168in}{1.618729in}}%
\pgfpathlineto{\pgfqpoint{2.004429in}{1.651456in}}%
\pgfpathlineto{\pgfqpoint{2.076690in}{1.683927in}}%
\pgfpathlineto{\pgfqpoint{2.148951in}{1.716113in}}%
\pgfpathlineto{\pgfqpoint{2.221212in}{1.747986in}}%
\pgfpathlineto{\pgfqpoint{2.293473in}{1.779516in}}%
\pgfpathlineto{\pgfqpoint{2.365734in}{1.810674in}}%
\pgfpathlineto{\pgfqpoint{2.437995in}{1.841429in}}%
\pgfpathlineto{\pgfqpoint{2.510256in}{1.871752in}}%
\pgfpathlineto{\pgfqpoint{2.582517in}{1.901610in}}%
\pgfpathlineto{\pgfqpoint{2.654779in}{1.930974in}}%
\pgfpathlineto{\pgfqpoint{2.727040in}{1.959810in}}%
\pgfpathlineto{\pgfqpoint{2.799301in}{1.988087in}}%
\pgfpathlineto{\pgfqpoint{2.871562in}{2.015773in}}%
\pgfpathlineto{\pgfqpoint{2.943823in}{2.042835in}}%
\pgfpathlineto{\pgfqpoint{3.016084in}{2.069240in}}%
\pgfpathlineto{\pgfqpoint{3.088345in}{2.094953in}}%
\pgfpathlineto{\pgfqpoint{3.160606in}{2.119940in}}%
\pgfpathlineto{\pgfqpoint{3.232867in}{2.144168in}}%
\pgfpathlineto{\pgfqpoint{3.305128in}{2.167602in}}%
\pgfpathlineto{\pgfqpoint{3.377389in}{2.190205in}}%
\pgfpathlineto{\pgfqpoint{3.449650in}{2.211943in}}%
\pgfpathlineto{\pgfqpoint{3.521911in}{2.232780in}}%
\pgfpathlineto{\pgfqpoint{3.594172in}{2.252679in}}%
\pgfpathlineto{\pgfqpoint{3.666434in}{2.271603in}}%
\pgfpathlineto{\pgfqpoint{3.738695in}{2.289514in}}%
\pgfpathlineto{\pgfqpoint{3.883217in}{2.322151in}}%
\pgfpathlineto{\pgfqpoint{3.955478in}{2.335361in}}%
\pgfpathlineto{\pgfqpoint{4.027739in}{2.347488in}}%
\pgfpathlineto{\pgfqpoint{4.100000in}{2.358571in}}%
\pgfpathlineto{\pgfqpoint{4.172261in}{2.368648in}}%
\pgfpathlineto{\pgfqpoint{4.244522in}{2.377757in}}%
\pgfpathlineto{\pgfqpoint{4.316783in}{2.385935in}}%
\pgfpathlineto{\pgfqpoint{4.389044in}{2.393218in}}%
\pgfpathlineto{\pgfqpoint{4.461305in}{2.399641in}}%
\pgfpathlineto{\pgfqpoint{4.533566in}{2.405241in}}%
\pgfpathlineto{\pgfqpoint{4.605828in}{2.410051in}}%
\pgfpathlineto{\pgfqpoint{4.678089in}{2.414106in}}%
\pgfpathlineto{\pgfqpoint{4.750350in}{2.417439in}}%
\pgfpathlineto{\pgfqpoint{4.822611in}{2.420082in}}%
\pgfpathlineto{\pgfqpoint{4.894872in}{2.422067in}}%
\pgfpathlineto{\pgfqpoint{4.967133in}{2.423427in}}%
\pgfpathlineto{\pgfqpoint{5.039394in}{2.424192in}}%
\pgfpathlineto{\pgfqpoint{5.111655in}{2.424392in}}%
\pgfpathlineto{\pgfqpoint{5.183916in}{2.424057in}}%
\pgfpathlineto{\pgfqpoint{5.256177in}{2.423217in}}%
\pgfpathlineto{\pgfqpoint{5.328438in}{2.421899in}}%
\pgfpathlineto{\pgfqpoint{5.400699in}{2.420132in}}%
\pgfpathlineto{\pgfqpoint{5.472960in}{2.417942in}}%
\pgfpathlineto{\pgfqpoint{5.545221in}{2.415357in}}%
\pgfpathlineto{\pgfqpoint{5.617483in}{2.412403in}}%
\pgfpathlineto{\pgfqpoint{5.689744in}{2.409105in}}%
\pgfpathlineto{\pgfqpoint{5.762005in}{2.405487in}}%
\pgfpathlineto{\pgfqpoint{5.834266in}{2.401575in}}%
\pgfpathlineto{\pgfqpoint{5.906527in}{2.397392in}}%
\pgfpathlineto{\pgfqpoint{5.978788in}{2.392960in}}%
\pgfpathlineto{\pgfqpoint{6.051049in}{2.388303in}}%
\pgfpathlineto{\pgfqpoint{6.123310in}{2.383441in}}%
\pgfpathlineto{\pgfqpoint{6.195571in}{2.378398in}}%
\pgfpathlineto{\pgfqpoint{6.267832in}{2.373192in}}%
\pgfpathlineto{\pgfqpoint{6.340093in}{2.367844in}}%
\pgfpathlineto{\pgfqpoint{6.412354in}{2.362373in}}%
\pgfpathlineto{\pgfqpoint{6.484615in}{2.356799in}}%
\pgfpathlineto{\pgfqpoint{6.556876in}{2.351139in}}%
\pgfpathlineto{\pgfqpoint{6.629138in}{2.345411in}}%
\pgfpathlineto{\pgfqpoint{6.701399in}{2.339632in}}%
\pgfpathlineto{\pgfqpoint{6.773660in}{2.333819in}}%
\pgfpathlineto{\pgfqpoint{6.918182in}{2.322151in}}%
\pgfusepath{stroke}%
\end{pgfscope}%
\begin{pgfscope}%
\pgfpathrectangle{\pgfqpoint{1.000000in}{0.330000in}}{\pgfqpoint{6.200000in}{2.310000in}}%
\pgfusepath{clip}%
\pgfsetrectcap%
\pgfsetroundjoin%
\pgfsetlinewidth{1.505625pt}%
\definecolor{currentstroke}{rgb}{0.172549,0.627451,0.172549}%
\pgfsetstrokecolor{currentstroke}%
\pgfsetdash{}{0pt}%
\pgfpathmoveto{\pgfqpoint{1.281818in}{1.311379in}}%
\pgfpathlineto{\pgfqpoint{1.354079in}{1.345353in}}%
\pgfpathlineto{\pgfqpoint{1.426340in}{1.379310in}}%
\pgfpathlineto{\pgfqpoint{1.498601in}{1.413225in}}%
\pgfpathlineto{\pgfqpoint{1.570862in}{1.447073in}}%
\pgfpathlineto{\pgfqpoint{1.643124in}{1.480827in}}%
\pgfpathlineto{\pgfqpoint{1.715385in}{1.514461in}}%
\pgfpathlineto{\pgfqpoint{1.787646in}{1.547950in}}%
\pgfpathlineto{\pgfqpoint{1.859907in}{1.581266in}}%
\pgfpathlineto{\pgfqpoint{1.932168in}{1.614382in}}%
\pgfpathlineto{\pgfqpoint{2.004429in}{1.647270in}}%
\pgfpathlineto{\pgfqpoint{2.076690in}{1.679902in}}%
\pgfpathlineto{\pgfqpoint{2.148951in}{1.712249in}}%
\pgfpathlineto{\pgfqpoint{2.221212in}{1.744283in}}%
\pgfpathlineto{\pgfqpoint{2.293473in}{1.775974in}}%
\pgfpathlineto{\pgfqpoint{2.365734in}{1.807293in}}%
\pgfpathlineto{\pgfqpoint{2.437995in}{1.838210in}}%
\pgfpathlineto{\pgfqpoint{2.510256in}{1.868693in}}%
\pgfpathlineto{\pgfqpoint{2.582517in}{1.898712in}}%
\pgfpathlineto{\pgfqpoint{2.654779in}{1.928237in}}%
\pgfpathlineto{\pgfqpoint{2.727040in}{1.957234in}}%
\pgfpathlineto{\pgfqpoint{2.799301in}{1.985673in}}%
\pgfpathlineto{\pgfqpoint{2.871562in}{2.013520in}}%
\pgfpathlineto{\pgfqpoint{2.943823in}{2.040742in}}%
\pgfpathlineto{\pgfqpoint{3.016084in}{2.067308in}}%
\pgfpathlineto{\pgfqpoint{3.088345in}{2.093182in}}%
\pgfpathlineto{\pgfqpoint{3.160606in}{2.118330in}}%
\pgfpathlineto{\pgfqpoint{3.232867in}{2.142719in}}%
\pgfpathlineto{\pgfqpoint{3.305128in}{2.166314in}}%
\pgfpathlineto{\pgfqpoint{3.377389in}{2.189078in}}%
\pgfpathlineto{\pgfqpoint{3.449650in}{2.210978in}}%
\pgfpathlineto{\pgfqpoint{3.521911in}{2.231975in}}%
\pgfpathlineto{\pgfqpoint{3.594172in}{2.252035in}}%
\pgfpathlineto{\pgfqpoint{3.666434in}{2.271120in}}%
\pgfpathlineto{\pgfqpoint{3.738695in}{2.289192in}}%
\pgfpathlineto{\pgfqpoint{3.883217in}{2.322151in}}%
\pgfpathlineto{\pgfqpoint{3.955478in}{2.335541in}}%
\pgfpathlineto{\pgfqpoint{4.027739in}{2.347847in}}%
\pgfpathlineto{\pgfqpoint{4.100000in}{2.359105in}}%
\pgfpathlineto{\pgfqpoint{4.172261in}{2.369354in}}%
\pgfpathlineto{\pgfqpoint{4.244522in}{2.378629in}}%
\pgfpathlineto{\pgfqpoint{4.316783in}{2.386968in}}%
\pgfpathlineto{\pgfqpoint{4.389044in}{2.394405in}}%
\pgfpathlineto{\pgfqpoint{4.461305in}{2.400975in}}%
\pgfpathlineto{\pgfqpoint{4.533566in}{2.406714in}}%
\pgfpathlineto{\pgfqpoint{4.605828in}{2.411654in}}%
\pgfpathlineto{\pgfqpoint{4.678089in}{2.415829in}}%
\pgfpathlineto{\pgfqpoint{4.750350in}{2.419273in}}%
\pgfpathlineto{\pgfqpoint{4.822611in}{2.422016in}}%
\pgfpathlineto{\pgfqpoint{4.894872in}{2.424092in}}%
\pgfpathlineto{\pgfqpoint{4.967133in}{2.425530in}}%
\pgfpathlineto{\pgfqpoint{5.039394in}{2.426362in}}%
\pgfpathlineto{\pgfqpoint{5.111655in}{2.426617in}}%
\pgfpathlineto{\pgfqpoint{5.183916in}{2.426326in}}%
\pgfpathlineto{\pgfqpoint{5.256177in}{2.425516in}}%
\pgfpathlineto{\pgfqpoint{5.328438in}{2.424217in}}%
\pgfpathlineto{\pgfqpoint{5.400699in}{2.422456in}}%
\pgfpathlineto{\pgfqpoint{5.472960in}{2.420260in}}%
\pgfpathlineto{\pgfqpoint{5.545221in}{2.417657in}}%
\pgfpathlineto{\pgfqpoint{5.617483in}{2.414671in}}%
\pgfpathlineto{\pgfqpoint{5.689744in}{2.411330in}}%
\pgfpathlineto{\pgfqpoint{5.762005in}{2.407657in}}%
\pgfpathlineto{\pgfqpoint{5.834266in}{2.403678in}}%
\pgfpathlineto{\pgfqpoint{5.906527in}{2.399416in}}%
\pgfpathlineto{\pgfqpoint{5.978788in}{2.394894in}}%
\pgfpathlineto{\pgfqpoint{6.051049in}{2.390136in}}%
\pgfpathlineto{\pgfqpoint{6.123310in}{2.385164in}}%
\pgfpathlineto{\pgfqpoint{6.195571in}{2.380000in}}%
\pgfpathlineto{\pgfqpoint{6.267832in}{2.374664in}}%
\pgfpathlineto{\pgfqpoint{6.340093in}{2.369177in}}%
\pgfpathlineto{\pgfqpoint{6.412354in}{2.363560in}}%
\pgfpathlineto{\pgfqpoint{6.484615in}{2.357832in}}%
\pgfpathlineto{\pgfqpoint{6.556876in}{2.352011in}}%
\pgfpathlineto{\pgfqpoint{6.629138in}{2.346117in}}%
\pgfpathlineto{\pgfqpoint{6.701399in}{2.340167in}}%
\pgfpathlineto{\pgfqpoint{6.773660in}{2.334178in}}%
\pgfpathlineto{\pgfqpoint{6.918182in}{2.322151in}}%
\pgfusepath{stroke}%
\end{pgfscope}%
\begin{pgfscope}%
\pgfpathrectangle{\pgfqpoint{1.000000in}{0.330000in}}{\pgfqpoint{6.200000in}{2.310000in}}%
\pgfusepath{clip}%
\pgfsetrectcap%
\pgfsetroundjoin%
\pgfsetlinewidth{1.505625pt}%
\definecolor{currentstroke}{rgb}{0.839216,0.152941,0.156863}%
\pgfsetstrokecolor{currentstroke}%
\pgfsetdash{}{0pt}%
\pgfpathmoveto{\pgfqpoint{1.281818in}{1.150050in}}%
\pgfpathlineto{\pgfqpoint{1.354079in}{1.189331in}}%
\pgfpathlineto{\pgfqpoint{1.426340in}{1.228593in}}%
\pgfpathlineto{\pgfqpoint{1.498601in}{1.267806in}}%
\pgfpathlineto{\pgfqpoint{1.570862in}{1.306943in}}%
\pgfpathlineto{\pgfqpoint{1.643124in}{1.345973in}}%
\pgfpathlineto{\pgfqpoint{1.715385in}{1.384867in}}%
\pgfpathlineto{\pgfqpoint{1.787646in}{1.423595in}}%
\pgfpathlineto{\pgfqpoint{1.859907in}{1.462126in}}%
\pgfpathlineto{\pgfqpoint{1.932168in}{1.500428in}}%
\pgfpathlineto{\pgfqpoint{2.004429in}{1.538471in}}%
\pgfpathlineto{\pgfqpoint{2.076690in}{1.576222in}}%
\pgfpathlineto{\pgfqpoint{2.148951in}{1.613649in}}%
\pgfpathlineto{\pgfqpoint{2.221212in}{1.650718in}}%
\pgfpathlineto{\pgfqpoint{2.293473in}{1.687397in}}%
\pgfpathlineto{\pgfqpoint{2.365734in}{1.723651in}}%
\pgfpathlineto{\pgfqpoint{2.437995in}{1.759446in}}%
\pgfpathlineto{\pgfqpoint{2.510256in}{1.794748in}}%
\pgfpathlineto{\pgfqpoint{2.582517in}{1.829520in}}%
\pgfpathlineto{\pgfqpoint{2.654779in}{1.863729in}}%
\pgfpathlineto{\pgfqpoint{2.727040in}{1.897336in}}%
\pgfpathlineto{\pgfqpoint{2.799301in}{1.930307in}}%
\pgfpathlineto{\pgfqpoint{2.871562in}{1.962603in}}%
\pgfpathlineto{\pgfqpoint{2.943823in}{1.994188in}}%
\pgfpathlineto{\pgfqpoint{3.016084in}{2.025023in}}%
\pgfpathlineto{\pgfqpoint{3.088345in}{2.055070in}}%
\pgfpathlineto{\pgfqpoint{3.160606in}{2.084290in}}%
\pgfpathlineto{\pgfqpoint{3.232867in}{2.112644in}}%
\pgfpathlineto{\pgfqpoint{3.305128in}{2.140092in}}%
\pgfpathlineto{\pgfqpoint{3.377389in}{2.166594in}}%
\pgfpathlineto{\pgfqpoint{3.449650in}{2.192110in}}%
\pgfpathlineto{\pgfqpoint{3.521911in}{2.216598in}}%
\pgfpathlineto{\pgfqpoint{3.594172in}{2.240016in}}%
\pgfpathlineto{\pgfqpoint{3.666434in}{2.262323in}}%
\pgfpathlineto{\pgfqpoint{3.738695in}{2.283477in}}%
\pgfpathlineto{\pgfqpoint{3.883217in}{2.322151in}}%
\pgfpathlineto{\pgfqpoint{3.955478in}{2.338033in}}%
\pgfpathlineto{\pgfqpoint{4.027739in}{2.352672in}}%
\pgfpathlineto{\pgfqpoint{4.100000in}{2.366106in}}%
\pgfpathlineto{\pgfqpoint{4.172261in}{2.378376in}}%
\pgfpathlineto{\pgfqpoint{4.244522in}{2.389520in}}%
\pgfpathlineto{\pgfqpoint{4.316783in}{2.399575in}}%
\pgfpathlineto{\pgfqpoint{4.389044in}{2.408581in}}%
\pgfpathlineto{\pgfqpoint{4.461305in}{2.416573in}}%
\pgfpathlineto{\pgfqpoint{4.533566in}{2.423589in}}%
\pgfpathlineto{\pgfqpoint{4.605828in}{2.429666in}}%
\pgfpathlineto{\pgfqpoint{4.678089in}{2.434838in}}%
\pgfpathlineto{\pgfqpoint{4.750350in}{2.439143in}}%
\pgfpathlineto{\pgfqpoint{4.822611in}{2.442614in}}%
\pgfpathlineto{\pgfqpoint{4.894872in}{2.445287in}}%
\pgfpathlineto{\pgfqpoint{4.967133in}{2.447196in}}%
\pgfpathlineto{\pgfqpoint{5.039394in}{2.448373in}}%
\pgfpathlineto{\pgfqpoint{5.111655in}{2.448853in}}%
\pgfpathlineto{\pgfqpoint{5.183916in}{2.448669in}}%
\pgfpathlineto{\pgfqpoint{5.256177in}{2.447852in}}%
\pgfpathlineto{\pgfqpoint{5.328438in}{2.446434in}}%
\pgfpathlineto{\pgfqpoint{5.400699in}{2.444448in}}%
\pgfpathlineto{\pgfqpoint{5.472960in}{2.441923in}}%
\pgfpathlineto{\pgfqpoint{5.545221in}{2.438891in}}%
\pgfpathlineto{\pgfqpoint{5.617483in}{2.435382in}}%
\pgfpathlineto{\pgfqpoint{5.689744in}{2.431425in}}%
\pgfpathlineto{\pgfqpoint{5.762005in}{2.427049in}}%
\pgfpathlineto{\pgfqpoint{5.834266in}{2.422283in}}%
\pgfpathlineto{\pgfqpoint{5.906527in}{2.417155in}}%
\pgfpathlineto{\pgfqpoint{5.978788in}{2.411694in}}%
\pgfpathlineto{\pgfqpoint{6.051049in}{2.405926in}}%
\pgfpathlineto{\pgfqpoint{6.123310in}{2.399878in}}%
\pgfpathlineto{\pgfqpoint{6.195571in}{2.393578in}}%
\pgfpathlineto{\pgfqpoint{6.267832in}{2.387050in}}%
\pgfpathlineto{\pgfqpoint{6.340093in}{2.380321in}}%
\pgfpathlineto{\pgfqpoint{6.412354in}{2.373416in}}%
\pgfpathlineto{\pgfqpoint{6.484615in}{2.366358in}}%
\pgfpathlineto{\pgfqpoint{6.556876in}{2.359173in}}%
\pgfpathlineto{\pgfqpoint{6.629138in}{2.351885in}}%
\pgfpathlineto{\pgfqpoint{6.701399in}{2.344515in}}%
\pgfpathlineto{\pgfqpoint{6.773660in}{2.337089in}}%
\pgfpathlineto{\pgfqpoint{6.918182in}{2.322151in}}%
\pgfusepath{stroke}%
\end{pgfscope}%
\begin{pgfscope}%
\pgfpathrectangle{\pgfqpoint{1.000000in}{0.330000in}}{\pgfqpoint{6.200000in}{2.310000in}}%
\pgfusepath{clip}%
\pgfsetrectcap%
\pgfsetroundjoin%
\pgfsetlinewidth{1.505625pt}%
\definecolor{currentstroke}{rgb}{0.580392,0.403922,0.741176}%
\pgfsetstrokecolor{currentstroke}%
\pgfsetdash{}{0pt}%
\pgfpathmoveto{\pgfqpoint{1.281818in}{0.650732in}}%
\pgfpathlineto{\pgfqpoint{1.354079in}{0.706852in}}%
\pgfpathlineto{\pgfqpoint{1.426340in}{0.762942in}}%
\pgfpathlineto{\pgfqpoint{1.498601in}{0.818959in}}%
\pgfpathlineto{\pgfqpoint{1.570862in}{0.874861in}}%
\pgfpathlineto{\pgfqpoint{1.643124in}{0.930606in}}%
\pgfpathlineto{\pgfqpoint{1.715385in}{0.986151in}}%
\pgfpathlineto{\pgfqpoint{1.787646in}{1.041451in}}%
\pgfpathlineto{\pgfqpoint{1.859907in}{1.096462in}}%
\pgfpathlineto{\pgfqpoint{1.932168in}{1.151141in}}%
\pgfpathlineto{\pgfqpoint{2.004429in}{1.205440in}}%
\pgfpathlineto{\pgfqpoint{2.076690in}{1.259314in}}%
\pgfpathlineto{\pgfqpoint{2.148951in}{1.312716in}}%
\pgfpathlineto{\pgfqpoint{2.221212in}{1.365600in}}%
\pgfpathlineto{\pgfqpoint{2.293473in}{1.417918in}}%
\pgfpathlineto{\pgfqpoint{2.365734in}{1.469621in}}%
\pgfpathlineto{\pgfqpoint{2.437995in}{1.520660in}}%
\pgfpathlineto{\pgfqpoint{2.510256in}{1.570987in}}%
\pgfpathlineto{\pgfqpoint{2.582517in}{1.620552in}}%
\pgfpathlineto{\pgfqpoint{2.654779in}{1.669303in}}%
\pgfpathlineto{\pgfqpoint{2.727040in}{1.717191in}}%
\pgfpathlineto{\pgfqpoint{2.799301in}{1.764163in}}%
\pgfpathlineto{\pgfqpoint{2.871562in}{1.810167in}}%
\pgfpathlineto{\pgfqpoint{2.943823in}{1.855152in}}%
\pgfpathlineto{\pgfqpoint{3.016084in}{1.899063in}}%
\pgfpathlineto{\pgfqpoint{3.088345in}{1.941846in}}%
\pgfpathlineto{\pgfqpoint{3.160606in}{1.983449in}}%
\pgfpathlineto{\pgfqpoint{3.232867in}{2.023815in}}%
\pgfpathlineto{\pgfqpoint{3.305128in}{2.062889in}}%
\pgfpathlineto{\pgfqpoint{3.377389in}{2.100616in}}%
\pgfpathlineto{\pgfqpoint{3.449650in}{2.136938in}}%
\pgfpathlineto{\pgfqpoint{3.521911in}{2.171799in}}%
\pgfpathlineto{\pgfqpoint{3.594172in}{2.205142in}}%
\pgfpathlineto{\pgfqpoint{3.666434in}{2.236907in}}%
\pgfpathlineto{\pgfqpoint{3.738695in}{2.267037in}}%
\pgfpathlineto{\pgfqpoint{3.883217in}{2.322151in}}%
\pgfpathlineto{\pgfqpoint{3.955478in}{2.344886in}}%
\pgfpathlineto{\pgfqpoint{4.027739in}{2.365860in}}%
\pgfpathlineto{\pgfqpoint{4.100000in}{2.385128in}}%
\pgfpathlineto{\pgfqpoint{4.172261in}{2.402744in}}%
\pgfpathlineto{\pgfqpoint{4.244522in}{2.418761in}}%
\pgfpathlineto{\pgfqpoint{4.316783in}{2.433231in}}%
\pgfpathlineto{\pgfqpoint{4.389044in}{2.446207in}}%
\pgfpathlineto{\pgfqpoint{4.461305in}{2.457739in}}%
\pgfpathlineto{\pgfqpoint{4.533566in}{2.467879in}}%
\pgfpathlineto{\pgfqpoint{4.605828in}{2.476677in}}%
\pgfpathlineto{\pgfqpoint{4.678089in}{2.484183in}}%
\pgfpathlineto{\pgfqpoint{4.750350in}{2.490447in}}%
\pgfpathlineto{\pgfqpoint{4.822611in}{2.495516in}}%
\pgfpathlineto{\pgfqpoint{4.894872in}{2.499439in}}%
\pgfpathlineto{\pgfqpoint{4.967133in}{2.502264in}}%
\pgfpathlineto{\pgfqpoint{5.039394in}{2.504038in}}%
\pgfpathlineto{\pgfqpoint{5.111655in}{2.504807in}}%
\pgfpathlineto{\pgfqpoint{5.183916in}{2.504617in}}%
\pgfpathlineto{\pgfqpoint{5.256177in}{2.503514in}}%
\pgfpathlineto{\pgfqpoint{5.328438in}{2.501542in}}%
\pgfpathlineto{\pgfqpoint{5.400699in}{2.498746in}}%
\pgfpathlineto{\pgfqpoint{5.472960in}{2.495169in}}%
\pgfpathlineto{\pgfqpoint{5.545221in}{2.490855in}}%
\pgfpathlineto{\pgfqpoint{5.617483in}{2.485845in}}%
\pgfpathlineto{\pgfqpoint{5.689744in}{2.480184in}}%
\pgfpathlineto{\pgfqpoint{5.762005in}{2.473910in}}%
\pgfpathlineto{\pgfqpoint{5.834266in}{2.467067in}}%
\pgfpathlineto{\pgfqpoint{5.906527in}{2.459694in}}%
\pgfpathlineto{\pgfqpoint{5.978788in}{2.451831in}}%
\pgfpathlineto{\pgfqpoint{6.051049in}{2.443517in}}%
\pgfpathlineto{\pgfqpoint{6.123310in}{2.434791in}}%
\pgfpathlineto{\pgfqpoint{6.195571in}{2.425692in}}%
\pgfpathlineto{\pgfqpoint{6.267832in}{2.416256in}}%
\pgfpathlineto{\pgfqpoint{6.340093in}{2.406522in}}%
\pgfpathlineto{\pgfqpoint{6.412354in}{2.396525in}}%
\pgfpathlineto{\pgfqpoint{6.484615in}{2.386301in}}%
\pgfpathlineto{\pgfqpoint{6.556876in}{2.375887in}}%
\pgfpathlineto{\pgfqpoint{6.629138in}{2.365316in}}%
\pgfpathlineto{\pgfqpoint{6.701399in}{2.354624in}}%
\pgfpathlineto{\pgfqpoint{6.773660in}{2.343843in}}%
\pgfpathlineto{\pgfqpoint{6.918182in}{2.322151in}}%
\pgfusepath{stroke}%
\end{pgfscope}%
\begin{pgfscope}%
\pgfpathrectangle{\pgfqpoint{1.000000in}{0.330000in}}{\pgfqpoint{6.200000in}{2.310000in}}%
\pgfusepath{clip}%
\pgfsetrectcap%
\pgfsetroundjoin%
\pgfsetlinewidth{1.505625pt}%
\definecolor{currentstroke}{rgb}{0.549020,0.337255,0.294118}%
\pgfsetstrokecolor{currentstroke}%
\pgfsetdash{}{0pt}%
\pgfpathmoveto{\pgfqpoint{1.281818in}{1.293992in}}%
\pgfpathlineto{\pgfqpoint{1.354079in}{1.328449in}}%
\pgfpathlineto{\pgfqpoint{1.426340in}{1.362889in}}%
\pgfpathlineto{\pgfqpoint{1.498601in}{1.397287in}}%
\pgfpathlineto{\pgfqpoint{1.570862in}{1.431617in}}%
\pgfpathlineto{\pgfqpoint{1.643124in}{1.465855in}}%
\pgfpathlineto{\pgfqpoint{1.715385in}{1.499972in}}%
\pgfpathlineto{\pgfqpoint{1.787646in}{1.533944in}}%
\pgfpathlineto{\pgfqpoint{1.859907in}{1.567743in}}%
\pgfpathlineto{\pgfqpoint{1.932168in}{1.601342in}}%
\pgfpathlineto{\pgfqpoint{2.004429in}{1.634713in}}%
\pgfpathlineto{\pgfqpoint{2.076690in}{1.667828in}}%
\pgfpathlineto{\pgfqpoint{2.148951in}{1.700658in}}%
\pgfpathlineto{\pgfqpoint{2.221212in}{1.733175in}}%
\pgfpathlineto{\pgfqpoint{2.293473in}{1.765349in}}%
\pgfpathlineto{\pgfqpoint{2.365734in}{1.797151in}}%
\pgfpathlineto{\pgfqpoint{2.437995in}{1.828550in}}%
\pgfpathlineto{\pgfqpoint{2.510256in}{1.859517in}}%
\pgfpathlineto{\pgfqpoint{2.582517in}{1.890019in}}%
\pgfpathlineto{\pgfqpoint{2.654779in}{1.920026in}}%
\pgfpathlineto{\pgfqpoint{2.727040in}{1.949507in}}%
\pgfpathlineto{\pgfqpoint{2.799301in}{1.978428in}}%
\pgfpathlineto{\pgfqpoint{2.871562in}{2.006758in}}%
\pgfpathlineto{\pgfqpoint{2.943823in}{2.034464in}}%
\pgfpathlineto{\pgfqpoint{3.016084in}{2.061512in}}%
\pgfpathlineto{\pgfqpoint{3.088345in}{2.087869in}}%
\pgfpathlineto{\pgfqpoint{3.160606in}{2.113501in}}%
\pgfpathlineto{\pgfqpoint{3.232867in}{2.138373in}}%
\pgfpathlineto{\pgfqpoint{3.305128in}{2.162450in}}%
\pgfpathlineto{\pgfqpoint{3.377389in}{2.185698in}}%
\pgfpathlineto{\pgfqpoint{3.449650in}{2.208080in}}%
\pgfpathlineto{\pgfqpoint{3.521911in}{2.229560in}}%
\pgfpathlineto{\pgfqpoint{3.594172in}{2.250103in}}%
\pgfpathlineto{\pgfqpoint{3.666434in}{2.269671in}}%
\pgfpathlineto{\pgfqpoint{3.738695in}{2.288226in}}%
\pgfpathlineto{\pgfqpoint{3.883217in}{2.322151in}}%
\pgfpathlineto{\pgfqpoint{3.955478in}{2.336083in}}%
\pgfpathlineto{\pgfqpoint{4.027739in}{2.348924in}}%
\pgfpathlineto{\pgfqpoint{4.100000in}{2.360708in}}%
\pgfpathlineto{\pgfqpoint{4.172261in}{2.371471in}}%
\pgfpathlineto{\pgfqpoint{4.244522in}{2.381246in}}%
\pgfpathlineto{\pgfqpoint{4.316783in}{2.390067in}}%
\pgfpathlineto{\pgfqpoint{4.389044in}{2.397966in}}%
\pgfpathlineto{\pgfqpoint{4.461305in}{2.404977in}}%
\pgfpathlineto{\pgfqpoint{4.533566in}{2.411132in}}%
\pgfpathlineto{\pgfqpoint{4.605828in}{2.416462in}}%
\pgfpathlineto{\pgfqpoint{4.678089in}{2.421000in}}%
\pgfpathlineto{\pgfqpoint{4.750350in}{2.424775in}}%
\pgfpathlineto{\pgfqpoint{4.822611in}{2.427821in}}%
\pgfpathlineto{\pgfqpoint{4.894872in}{2.430165in}}%
\pgfpathlineto{\pgfqpoint{4.967133in}{2.431839in}}%
\pgfpathlineto{\pgfqpoint{5.039394in}{2.432872in}}%
\pgfpathlineto{\pgfqpoint{5.111655in}{2.433293in}}%
\pgfpathlineto{\pgfqpoint{5.183916in}{2.433131in}}%
\pgfpathlineto{\pgfqpoint{5.256177in}{2.432415in}}%
\pgfpathlineto{\pgfqpoint{5.328438in}{2.431172in}}%
\pgfpathlineto{\pgfqpoint{5.400699in}{2.429429in}}%
\pgfpathlineto{\pgfqpoint{5.472960in}{2.427215in}}%
\pgfpathlineto{\pgfqpoint{5.545221in}{2.424555in}}%
\pgfpathlineto{\pgfqpoint{5.617483in}{2.421476in}}%
\pgfpathlineto{\pgfqpoint{5.689744in}{2.418005in}}%
\pgfpathlineto{\pgfqpoint{5.762005in}{2.414166in}}%
\pgfpathlineto{\pgfqpoint{5.834266in}{2.409986in}}%
\pgfpathlineto{\pgfqpoint{5.906527in}{2.405488in}}%
\pgfpathlineto{\pgfqpoint{5.978788in}{2.400697in}}%
\pgfpathlineto{\pgfqpoint{6.051049in}{2.395638in}}%
\pgfpathlineto{\pgfqpoint{6.123310in}{2.390333in}}%
\pgfpathlineto{\pgfqpoint{6.195571in}{2.384806in}}%
\pgfpathlineto{\pgfqpoint{6.267832in}{2.379080in}}%
\pgfpathlineto{\pgfqpoint{6.340093in}{2.373177in}}%
\pgfpathlineto{\pgfqpoint{6.412354in}{2.367120in}}%
\pgfpathlineto{\pgfqpoint{6.484615in}{2.360929in}}%
\pgfpathlineto{\pgfqpoint{6.556876in}{2.354627in}}%
\pgfpathlineto{\pgfqpoint{6.629138in}{2.348233in}}%
\pgfpathlineto{\pgfqpoint{6.701399in}{2.341769in}}%
\pgfpathlineto{\pgfqpoint{6.773660in}{2.335254in}}%
\pgfpathlineto{\pgfqpoint{6.918182in}{2.322151in}}%
\pgfusepath{stroke}%
\end{pgfscope}%
\begin{pgfscope}%
\pgfpathrectangle{\pgfqpoint{1.000000in}{0.330000in}}{\pgfqpoint{6.200000in}{2.310000in}}%
\pgfusepath{clip}%
\pgfsetrectcap%
\pgfsetroundjoin%
\pgfsetlinewidth{1.505625pt}%
\definecolor{currentstroke}{rgb}{0.890196,0.466667,0.760784}%
\pgfsetstrokecolor{currentstroke}%
\pgfsetdash{}{0pt}%
\pgfpathmoveto{\pgfqpoint{1.281818in}{1.705255in}}%
\pgfpathlineto{\pgfqpoint{1.354079in}{1.725930in}}%
\pgfpathlineto{\pgfqpoint{1.426340in}{1.746594in}}%
\pgfpathlineto{\pgfqpoint{1.498601in}{1.767233in}}%
\pgfpathlineto{\pgfqpoint{1.570862in}{1.787831in}}%
\pgfpathlineto{\pgfqpoint{1.643124in}{1.808373in}}%
\pgfpathlineto{\pgfqpoint{1.715385in}{1.828844in}}%
\pgfpathlineto{\pgfqpoint{1.787646in}{1.849227in}}%
\pgfpathlineto{\pgfqpoint{1.859907in}{1.869506in}}%
\pgfpathlineto{\pgfqpoint{1.932168in}{1.889665in}}%
\pgfpathlineto{\pgfqpoint{2.004429in}{1.909688in}}%
\pgfpathlineto{\pgfqpoint{2.076690in}{1.929557in}}%
\pgfpathlineto{\pgfqpoint{2.148951in}{1.949255in}}%
\pgfpathlineto{\pgfqpoint{2.221212in}{1.968765in}}%
\pgfpathlineto{\pgfqpoint{2.293473in}{1.988070in}}%
\pgfpathlineto{\pgfqpoint{2.365734in}{2.007151in}}%
\pgfpathlineto{\pgfqpoint{2.437995in}{2.025990in}}%
\pgfpathlineto{\pgfqpoint{2.510256in}{2.044570in}}%
\pgfpathlineto{\pgfqpoint{2.582517in}{2.062872in}}%
\pgfpathlineto{\pgfqpoint{2.654779in}{2.080876in}}%
\pgfpathlineto{\pgfqpoint{2.727040in}{2.098564in}}%
\pgfpathlineto{\pgfqpoint{2.799301in}{2.115917in}}%
\pgfpathlineto{\pgfqpoint{2.871562in}{2.132915in}}%
\pgfpathlineto{\pgfqpoint{2.943823in}{2.149539in}}%
\pgfpathlineto{\pgfqpoint{3.016084in}{2.165768in}}%
\pgfpathlineto{\pgfqpoint{3.088345in}{2.181582in}}%
\pgfpathlineto{\pgfqpoint{3.160606in}{2.196961in}}%
\pgfpathlineto{\pgfqpoint{3.232867in}{2.211884in}}%
\pgfpathlineto{\pgfqpoint{3.305128in}{2.226330in}}%
\pgfpathlineto{\pgfqpoint{3.377389in}{2.240279in}}%
\pgfpathlineto{\pgfqpoint{3.449650in}{2.253708in}}%
\pgfpathlineto{\pgfqpoint{3.521911in}{2.266597in}}%
\pgfpathlineto{\pgfqpoint{3.594172in}{2.278922in}}%
\pgfpathlineto{\pgfqpoint{3.666434in}{2.290663in}}%
\pgfpathlineto{\pgfqpoint{3.738695in}{2.301796in}}%
\pgfpathlineto{\pgfqpoint{3.883217in}{2.322151in}}%
\pgfpathlineto{\pgfqpoint{3.955478in}{2.330510in}}%
\pgfpathlineto{\pgfqpoint{4.027739in}{2.338215in}}%
\pgfpathlineto{\pgfqpoint{4.100000in}{2.345285in}}%
\pgfpathlineto{\pgfqpoint{4.172261in}{2.351743in}}%
\pgfpathlineto{\pgfqpoint{4.244522in}{2.357608in}}%
\pgfpathlineto{\pgfqpoint{4.316783in}{2.362901in}}%
\pgfpathlineto{\pgfqpoint{4.389044in}{2.367640in}}%
\pgfpathlineto{\pgfqpoint{4.461305in}{2.371847in}}%
\pgfpathlineto{\pgfqpoint{4.533566in}{2.375539in}}%
\pgfpathlineto{\pgfqpoint{4.605828in}{2.378738in}}%
\pgfpathlineto{\pgfqpoint{4.678089in}{2.381460in}}%
\pgfpathlineto{\pgfqpoint{4.750350in}{2.383726in}}%
\pgfpathlineto{\pgfqpoint{4.822611in}{2.385553in}}%
\pgfpathlineto{\pgfqpoint{4.894872in}{2.386959in}}%
\pgfpathlineto{\pgfqpoint{4.967133in}{2.387964in}}%
\pgfpathlineto{\pgfqpoint{5.039394in}{2.388584in}}%
\pgfpathlineto{\pgfqpoint{5.111655in}{2.388836in}}%
\pgfpathlineto{\pgfqpoint{5.183916in}{2.388739in}}%
\pgfpathlineto{\pgfqpoint{5.256177in}{2.388309in}}%
\pgfpathlineto{\pgfqpoint{5.328438in}{2.387563in}}%
\pgfpathlineto{\pgfqpoint{5.400699in}{2.386518in}}%
\pgfpathlineto{\pgfqpoint{5.472960in}{2.385189in}}%
\pgfpathlineto{\pgfqpoint{5.545221in}{2.383593in}}%
\pgfpathlineto{\pgfqpoint{5.617483in}{2.381746in}}%
\pgfpathlineto{\pgfqpoint{5.689744in}{2.379663in}}%
\pgfpathlineto{\pgfqpoint{5.762005in}{2.377360in}}%
\pgfpathlineto{\pgfqpoint{5.834266in}{2.374852in}}%
\pgfpathlineto{\pgfqpoint{5.906527in}{2.372153in}}%
\pgfpathlineto{\pgfqpoint{5.978788in}{2.369279in}}%
\pgfpathlineto{\pgfqpoint{6.051049in}{2.366243in}}%
\pgfpathlineto{\pgfqpoint{6.123310in}{2.363060in}}%
\pgfpathlineto{\pgfqpoint{6.195571in}{2.359744in}}%
\pgfpathlineto{\pgfqpoint{6.267832in}{2.356308in}}%
\pgfpathlineto{\pgfqpoint{6.340093in}{2.352767in}}%
\pgfpathlineto{\pgfqpoint{6.412354in}{2.349132in}}%
\pgfpathlineto{\pgfqpoint{6.484615in}{2.345418in}}%
\pgfpathlineto{\pgfqpoint{6.556876in}{2.341636in}}%
\pgfpathlineto{\pgfqpoint{6.629138in}{2.337800in}}%
\pgfpathlineto{\pgfqpoint{6.701399in}{2.333922in}}%
\pgfpathlineto{\pgfqpoint{6.773660in}{2.330013in}}%
\pgfpathlineto{\pgfqpoint{6.918182in}{2.322151in}}%
\pgfusepath{stroke}%
\end{pgfscope}%
\begin{pgfscope}%
\pgfpathrectangle{\pgfqpoint{1.000000in}{0.330000in}}{\pgfqpoint{6.200000in}{2.310000in}}%
\pgfusepath{clip}%
\pgfsetrectcap%
\pgfsetroundjoin%
\pgfsetlinewidth{1.505625pt}%
\definecolor{currentstroke}{rgb}{0.498039,0.498039,0.498039}%
\pgfsetstrokecolor{currentstroke}%
\pgfsetdash{}{0pt}%
\pgfpathmoveto{\pgfqpoint{1.281818in}{1.169632in}}%
\pgfpathlineto{\pgfqpoint{1.354079in}{1.208265in}}%
\pgfpathlineto{\pgfqpoint{1.426340in}{1.246877in}}%
\pgfpathlineto{\pgfqpoint{1.498601in}{1.285443in}}%
\pgfpathlineto{\pgfqpoint{1.570862in}{1.323932in}}%
\pgfpathlineto{\pgfqpoint{1.643124in}{1.362316in}}%
\pgfpathlineto{\pgfqpoint{1.715385in}{1.400566in}}%
\pgfpathlineto{\pgfqpoint{1.787646in}{1.438652in}}%
\pgfpathlineto{\pgfqpoint{1.859907in}{1.476544in}}%
\pgfpathlineto{\pgfqpoint{1.932168in}{1.514210in}}%
\pgfpathlineto{\pgfqpoint{2.004429in}{1.551621in}}%
\pgfpathlineto{\pgfqpoint{2.076690in}{1.588744in}}%
\pgfpathlineto{\pgfqpoint{2.148951in}{1.625548in}}%
\pgfpathlineto{\pgfqpoint{2.221212in}{1.661999in}}%
\pgfpathlineto{\pgfqpoint{2.293473in}{1.698066in}}%
\pgfpathlineto{\pgfqpoint{2.365734in}{1.733715in}}%
\pgfpathlineto{\pgfqpoint{2.437995in}{1.768912in}}%
\pgfpathlineto{\pgfqpoint{2.510256in}{1.803622in}}%
\pgfpathlineto{\pgfqpoint{2.582517in}{1.837813in}}%
\pgfpathlineto{\pgfqpoint{2.654779in}{1.871447in}}%
\pgfpathlineto{\pgfqpoint{2.727040in}{1.904491in}}%
\pgfpathlineto{\pgfqpoint{2.799301in}{1.936908in}}%
\pgfpathlineto{\pgfqpoint{2.871562in}{1.968661in}}%
\pgfpathlineto{\pgfqpoint{2.943823in}{1.999714in}}%
\pgfpathlineto{\pgfqpoint{3.016084in}{2.030029in}}%
\pgfpathlineto{\pgfqpoint{3.088345in}{2.059570in}}%
\pgfpathlineto{\pgfqpoint{3.160606in}{2.088297in}}%
\pgfpathlineto{\pgfqpoint{3.232867in}{2.116173in}}%
\pgfpathlineto{\pgfqpoint{3.305128in}{2.143158in}}%
\pgfpathlineto{\pgfqpoint{3.377389in}{2.169213in}}%
\pgfpathlineto{\pgfqpoint{3.449650in}{2.194298in}}%
\pgfpathlineto{\pgfqpoint{3.521911in}{2.218373in}}%
\pgfpathlineto{\pgfqpoint{3.594172in}{2.241396in}}%
\pgfpathlineto{\pgfqpoint{3.666434in}{2.263328in}}%
\pgfpathlineto{\pgfqpoint{3.738695in}{2.284126in}}%
\pgfpathlineto{\pgfqpoint{3.883217in}{2.322151in}}%
\pgfpathlineto{\pgfqpoint{3.955478in}{2.337774in}}%
\pgfpathlineto{\pgfqpoint{4.027739in}{2.352175in}}%
\pgfpathlineto{\pgfqpoint{4.100000in}{2.365393in}}%
\pgfpathlineto{\pgfqpoint{4.172261in}{2.377466in}}%
\pgfpathlineto{\pgfqpoint{4.244522in}{2.388432in}}%
\pgfpathlineto{\pgfqpoint{4.316783in}{2.398328in}}%
\pgfpathlineto{\pgfqpoint{4.389044in}{2.407192in}}%
\pgfpathlineto{\pgfqpoint{4.461305in}{2.415060in}}%
\pgfpathlineto{\pgfqpoint{4.533566in}{2.421969in}}%
\pgfpathlineto{\pgfqpoint{4.605828in}{2.427953in}}%
\pgfpathlineto{\pgfqpoint{4.678089in}{2.433048in}}%
\pgfpathlineto{\pgfqpoint{4.750350in}{2.437289in}}%
\pgfpathlineto{\pgfqpoint{4.822611in}{2.440711in}}%
\pgfpathlineto{\pgfqpoint{4.894872in}{2.443347in}}%
\pgfpathlineto{\pgfqpoint{4.967133in}{2.445231in}}%
\pgfpathlineto{\pgfqpoint{5.039394in}{2.446395in}}%
\pgfpathlineto{\pgfqpoint{5.111655in}{2.446873in}}%
\pgfpathlineto{\pgfqpoint{5.183916in}{2.446697in}}%
\pgfpathlineto{\pgfqpoint{5.256177in}{2.445898in}}%
\pgfpathlineto{\pgfqpoint{5.328438in}{2.444507in}}%
\pgfpathlineto{\pgfqpoint{5.400699in}{2.442557in}}%
\pgfpathlineto{\pgfqpoint{5.472960in}{2.440076in}}%
\pgfpathlineto{\pgfqpoint{5.545221in}{2.437095in}}%
\pgfpathlineto{\pgfqpoint{5.617483in}{2.433644in}}%
\pgfpathlineto{\pgfqpoint{5.689744in}{2.429751in}}%
\pgfpathlineto{\pgfqpoint{5.762005in}{2.425446in}}%
\pgfpathlineto{\pgfqpoint{5.834266in}{2.420757in}}%
\pgfpathlineto{\pgfqpoint{5.906527in}{2.415711in}}%
\pgfpathlineto{\pgfqpoint{5.978788in}{2.410335in}}%
\pgfpathlineto{\pgfqpoint{6.051049in}{2.404658in}}%
\pgfpathlineto{\pgfqpoint{6.123310in}{2.398704in}}%
\pgfpathlineto{\pgfqpoint{6.195571in}{2.392501in}}%
\pgfpathlineto{\pgfqpoint{6.267832in}{2.386074in}}%
\pgfpathlineto{\pgfqpoint{6.340093in}{2.379447in}}%
\pgfpathlineto{\pgfqpoint{6.412354in}{2.372647in}}%
\pgfpathlineto{\pgfqpoint{6.484615in}{2.365696in}}%
\pgfpathlineto{\pgfqpoint{6.556876in}{2.358620in}}%
\pgfpathlineto{\pgfqpoint{6.629138in}{2.351441in}}%
\pgfpathlineto{\pgfqpoint{6.701399in}{2.344182in}}%
\pgfpathlineto{\pgfqpoint{6.773660in}{2.336866in}}%
\pgfpathlineto{\pgfqpoint{6.918182in}{2.322151in}}%
\pgfusepath{stroke}%
\end{pgfscope}%
\begin{pgfscope}%
\pgfpathrectangle{\pgfqpoint{1.000000in}{0.330000in}}{\pgfqpoint{6.200000in}{2.310000in}}%
\pgfusepath{clip}%
\pgfsetrectcap%
\pgfsetroundjoin%
\pgfsetlinewidth{1.505625pt}%
\definecolor{currentstroke}{rgb}{0.737255,0.741176,0.133333}%
\pgfsetstrokecolor{currentstroke}%
\pgfsetdash{}{0pt}%
\pgfpathmoveto{\pgfqpoint{1.281818in}{1.169632in}}%
\pgfpathlineto{\pgfqpoint{1.354079in}{1.208265in}}%
\pgfpathlineto{\pgfqpoint{1.426340in}{1.246877in}}%
\pgfpathlineto{\pgfqpoint{1.498601in}{1.285443in}}%
\pgfpathlineto{\pgfqpoint{1.570862in}{1.323932in}}%
\pgfpathlineto{\pgfqpoint{1.643124in}{1.362316in}}%
\pgfpathlineto{\pgfqpoint{1.715385in}{1.400566in}}%
\pgfpathlineto{\pgfqpoint{1.787646in}{1.438652in}}%
\pgfpathlineto{\pgfqpoint{1.859907in}{1.476544in}}%
\pgfpathlineto{\pgfqpoint{1.932168in}{1.514210in}}%
\pgfpathlineto{\pgfqpoint{2.004429in}{1.551621in}}%
\pgfpathlineto{\pgfqpoint{2.076690in}{1.588744in}}%
\pgfpathlineto{\pgfqpoint{2.148951in}{1.625548in}}%
\pgfpathlineto{\pgfqpoint{2.221212in}{1.661999in}}%
\pgfpathlineto{\pgfqpoint{2.293473in}{1.698066in}}%
\pgfpathlineto{\pgfqpoint{2.365734in}{1.733715in}}%
\pgfpathlineto{\pgfqpoint{2.437995in}{1.768912in}}%
\pgfpathlineto{\pgfqpoint{2.510256in}{1.803622in}}%
\pgfpathlineto{\pgfqpoint{2.582517in}{1.837813in}}%
\pgfpathlineto{\pgfqpoint{2.654779in}{1.871447in}}%
\pgfpathlineto{\pgfqpoint{2.727040in}{1.904491in}}%
\pgfpathlineto{\pgfqpoint{2.799301in}{1.936908in}}%
\pgfpathlineto{\pgfqpoint{2.871562in}{1.968661in}}%
\pgfpathlineto{\pgfqpoint{2.943823in}{1.999714in}}%
\pgfpathlineto{\pgfqpoint{3.016084in}{2.030029in}}%
\pgfpathlineto{\pgfqpoint{3.088345in}{2.059570in}}%
\pgfpathlineto{\pgfqpoint{3.160606in}{2.088297in}}%
\pgfpathlineto{\pgfqpoint{3.232867in}{2.116173in}}%
\pgfpathlineto{\pgfqpoint{3.305128in}{2.143158in}}%
\pgfpathlineto{\pgfqpoint{3.377389in}{2.169213in}}%
\pgfpathlineto{\pgfqpoint{3.449650in}{2.194298in}}%
\pgfpathlineto{\pgfqpoint{3.521911in}{2.218373in}}%
\pgfpathlineto{\pgfqpoint{3.594172in}{2.241396in}}%
\pgfpathlineto{\pgfqpoint{3.666434in}{2.263328in}}%
\pgfpathlineto{\pgfqpoint{3.738695in}{2.284126in}}%
\pgfpathlineto{\pgfqpoint{3.883217in}{2.322151in}}%
\pgfpathlineto{\pgfqpoint{3.955478in}{2.337774in}}%
\pgfpathlineto{\pgfqpoint{4.027739in}{2.352175in}}%
\pgfpathlineto{\pgfqpoint{4.100000in}{2.365393in}}%
\pgfpathlineto{\pgfqpoint{4.172261in}{2.377466in}}%
\pgfpathlineto{\pgfqpoint{4.244522in}{2.388432in}}%
\pgfpathlineto{\pgfqpoint{4.316783in}{2.398328in}}%
\pgfpathlineto{\pgfqpoint{4.389044in}{2.407192in}}%
\pgfpathlineto{\pgfqpoint{4.461305in}{2.415060in}}%
\pgfpathlineto{\pgfqpoint{4.533566in}{2.421969in}}%
\pgfpathlineto{\pgfqpoint{4.605828in}{2.427953in}}%
\pgfpathlineto{\pgfqpoint{4.678089in}{2.433048in}}%
\pgfpathlineto{\pgfqpoint{4.750350in}{2.437289in}}%
\pgfpathlineto{\pgfqpoint{4.822611in}{2.440711in}}%
\pgfpathlineto{\pgfqpoint{4.894872in}{2.443347in}}%
\pgfpathlineto{\pgfqpoint{4.967133in}{2.445231in}}%
\pgfpathlineto{\pgfqpoint{5.039394in}{2.446395in}}%
\pgfpathlineto{\pgfqpoint{5.111655in}{2.446873in}}%
\pgfpathlineto{\pgfqpoint{5.183916in}{2.446697in}}%
\pgfpathlineto{\pgfqpoint{5.256177in}{2.445898in}}%
\pgfpathlineto{\pgfqpoint{5.328438in}{2.444507in}}%
\pgfpathlineto{\pgfqpoint{5.400699in}{2.442557in}}%
\pgfpathlineto{\pgfqpoint{5.472960in}{2.440076in}}%
\pgfpathlineto{\pgfqpoint{5.545221in}{2.437095in}}%
\pgfpathlineto{\pgfqpoint{5.617483in}{2.433644in}}%
\pgfpathlineto{\pgfqpoint{5.689744in}{2.429751in}}%
\pgfpathlineto{\pgfqpoint{5.762005in}{2.425446in}}%
\pgfpathlineto{\pgfqpoint{5.834266in}{2.420757in}}%
\pgfpathlineto{\pgfqpoint{5.906527in}{2.415711in}}%
\pgfpathlineto{\pgfqpoint{5.978788in}{2.410335in}}%
\pgfpathlineto{\pgfqpoint{6.051049in}{2.404658in}}%
\pgfpathlineto{\pgfqpoint{6.123310in}{2.398704in}}%
\pgfpathlineto{\pgfqpoint{6.195571in}{2.392501in}}%
\pgfpathlineto{\pgfqpoint{6.267832in}{2.386074in}}%
\pgfpathlineto{\pgfqpoint{6.340093in}{2.379447in}}%
\pgfpathlineto{\pgfqpoint{6.412354in}{2.372647in}}%
\pgfpathlineto{\pgfqpoint{6.484615in}{2.365696in}}%
\pgfpathlineto{\pgfqpoint{6.556876in}{2.358620in}}%
\pgfpathlineto{\pgfqpoint{6.629138in}{2.351441in}}%
\pgfpathlineto{\pgfqpoint{6.701399in}{2.344182in}}%
\pgfpathlineto{\pgfqpoint{6.773660in}{2.336866in}}%
\pgfpathlineto{\pgfqpoint{6.918182in}{2.322151in}}%
\pgfusepath{stroke}%
\end{pgfscope}%
\begin{pgfscope}%
\pgfpathrectangle{\pgfqpoint{1.000000in}{0.330000in}}{\pgfqpoint{6.200000in}{2.310000in}}%
\pgfusepath{clip}%
\pgfsetrectcap%
\pgfsetroundjoin%
\pgfsetlinewidth{1.505625pt}%
\definecolor{currentstroke}{rgb}{0.090196,0.745098,0.811765}%
\pgfsetstrokecolor{currentstroke}%
\pgfsetdash{}{0pt}%
\pgfpathmoveto{\pgfqpoint{1.281818in}{1.272122in}}%
\pgfpathlineto{\pgfqpoint{1.354079in}{1.307321in}}%
\pgfpathlineto{\pgfqpoint{1.426340in}{1.342504in}}%
\pgfpathlineto{\pgfqpoint{1.498601in}{1.377642in}}%
\pgfpathlineto{\pgfqpoint{1.570862in}{1.412711in}}%
\pgfpathlineto{\pgfqpoint{1.643124in}{1.447685in}}%
\pgfpathlineto{\pgfqpoint{1.715385in}{1.482536in}}%
\pgfpathlineto{\pgfqpoint{1.787646in}{1.517237in}}%
\pgfpathlineto{\pgfqpoint{1.859907in}{1.551761in}}%
\pgfpathlineto{\pgfqpoint{1.932168in}{1.586080in}}%
\pgfpathlineto{\pgfqpoint{2.004429in}{1.620165in}}%
\pgfpathlineto{\pgfqpoint{2.076690in}{1.653988in}}%
\pgfpathlineto{\pgfqpoint{2.148951in}{1.687520in}}%
\pgfpathlineto{\pgfqpoint{2.221212in}{1.720731in}}%
\pgfpathlineto{\pgfqpoint{2.293473in}{1.753591in}}%
\pgfpathlineto{\pgfqpoint{2.365734in}{1.786069in}}%
\pgfpathlineto{\pgfqpoint{2.437995in}{1.818136in}}%
\pgfpathlineto{\pgfqpoint{2.510256in}{1.849760in}}%
\pgfpathlineto{\pgfqpoint{2.582517in}{1.880909in}}%
\pgfpathlineto{\pgfqpoint{2.654779in}{1.911552in}}%
\pgfpathlineto{\pgfqpoint{2.727040in}{1.941656in}}%
\pgfpathlineto{\pgfqpoint{2.799301in}{1.971189in}}%
\pgfpathlineto{\pgfqpoint{2.871562in}{2.000117in}}%
\pgfpathlineto{\pgfqpoint{2.943823in}{2.028407in}}%
\pgfpathlineto{\pgfqpoint{3.016084in}{2.056025in}}%
\pgfpathlineto{\pgfqpoint{3.088345in}{2.082937in}}%
\pgfpathlineto{\pgfqpoint{3.160606in}{2.109107in}}%
\pgfpathlineto{\pgfqpoint{3.232867in}{2.134502in}}%
\pgfpathlineto{\pgfqpoint{3.305128in}{2.159086in}}%
\pgfpathlineto{\pgfqpoint{3.377389in}{2.182822in}}%
\pgfpathlineto{\pgfqpoint{3.449650in}{2.205674in}}%
\pgfpathlineto{\pgfqpoint{3.521911in}{2.227606in}}%
\pgfpathlineto{\pgfqpoint{3.594172in}{2.248581in}}%
\pgfpathlineto{\pgfqpoint{3.666434in}{2.268561in}}%
\pgfpathlineto{\pgfqpoint{3.738695in}{2.287508in}}%
\pgfpathlineto{\pgfqpoint{3.883217in}{2.322151in}}%
\pgfpathlineto{\pgfqpoint{3.955478in}{2.336387in}}%
\pgfpathlineto{\pgfqpoint{4.027739in}{2.349510in}}%
\pgfpathlineto{\pgfqpoint{4.100000in}{2.361556in}}%
\pgfpathlineto{\pgfqpoint{4.172261in}{2.372559in}}%
\pgfpathlineto{\pgfqpoint{4.244522in}{2.382553in}}%
\pgfpathlineto{\pgfqpoint{4.316783in}{2.391574in}}%
\pgfpathlineto{\pgfqpoint{4.389044in}{2.399653in}}%
\pgfpathlineto{\pgfqpoint{4.461305in}{2.406826in}}%
\pgfpathlineto{\pgfqpoint{4.533566in}{2.413124in}}%
\pgfpathlineto{\pgfqpoint{4.605828in}{2.418580in}}%
\pgfpathlineto{\pgfqpoint{4.678089in}{2.423226in}}%
\pgfpathlineto{\pgfqpoint{4.750350in}{2.427093in}}%
\pgfpathlineto{\pgfqpoint{4.822611in}{2.430214in}}%
\pgfpathlineto{\pgfqpoint{4.894872in}{2.432619in}}%
\pgfpathlineto{\pgfqpoint{4.967133in}{2.434338in}}%
\pgfpathlineto{\pgfqpoint{5.039394in}{2.435402in}}%
\pgfpathlineto{\pgfqpoint{5.111655in}{2.435840in}}%
\pgfpathlineto{\pgfqpoint{5.183916in}{2.435681in}}%
\pgfpathlineto{\pgfqpoint{5.256177in}{2.434955in}}%
\pgfpathlineto{\pgfqpoint{5.328438in}{2.433690in}}%
\pgfpathlineto{\pgfqpoint{5.400699in}{2.431914in}}%
\pgfpathlineto{\pgfqpoint{5.472960in}{2.429654in}}%
\pgfpathlineto{\pgfqpoint{5.545221in}{2.426939in}}%
\pgfpathlineto{\pgfqpoint{5.617483in}{2.423794in}}%
\pgfpathlineto{\pgfqpoint{5.689744in}{2.420247in}}%
\pgfpathlineto{\pgfqpoint{5.762005in}{2.416324in}}%
\pgfpathlineto{\pgfqpoint{5.834266in}{2.412050in}}%
\pgfpathlineto{\pgfqpoint{5.906527in}{2.407451in}}%
\pgfpathlineto{\pgfqpoint{5.978788in}{2.402551in}}%
\pgfpathlineto{\pgfqpoint{6.051049in}{2.397376in}}%
\pgfpathlineto{\pgfqpoint{6.123310in}{2.391949in}}%
\pgfpathlineto{\pgfqpoint{6.195571in}{2.386294in}}%
\pgfpathlineto{\pgfqpoint{6.267832in}{2.380435in}}%
\pgfpathlineto{\pgfqpoint{6.340093in}{2.374394in}}%
\pgfpathlineto{\pgfqpoint{6.412354in}{2.368193in}}%
\pgfpathlineto{\pgfqpoint{6.484615in}{2.361856in}}%
\pgfpathlineto{\pgfqpoint{6.556876in}{2.355404in}}%
\pgfpathlineto{\pgfqpoint{6.629138in}{2.348858in}}%
\pgfpathlineto{\pgfqpoint{6.701399in}{2.342240in}}%
\pgfpathlineto{\pgfqpoint{6.773660in}{2.335569in}}%
\pgfpathlineto{\pgfqpoint{6.918182in}{2.322151in}}%
\pgfusepath{stroke}%
\end{pgfscope}%
\begin{pgfscope}%
\pgfpathrectangle{\pgfqpoint{1.000000in}{0.330000in}}{\pgfqpoint{6.200000in}{2.310000in}}%
\pgfusepath{clip}%
\pgfsetrectcap%
\pgfsetroundjoin%
\pgfsetlinewidth{1.505625pt}%
\definecolor{currentstroke}{rgb}{0.121569,0.466667,0.705882}%
\pgfsetstrokecolor{currentstroke}%
\pgfsetdash{}{0pt}%
\pgfpathmoveto{\pgfqpoint{1.281818in}{2.300281in}}%
\pgfpathlineto{\pgfqpoint{1.354079in}{2.301023in}}%
\pgfpathlineto{\pgfqpoint{1.426340in}{2.301765in}}%
\pgfpathlineto{\pgfqpoint{1.498601in}{2.302506in}}%
\pgfpathlineto{\pgfqpoint{1.570862in}{2.303245in}}%
\pgfpathlineto{\pgfqpoint{1.643124in}{2.303981in}}%
\pgfpathlineto{\pgfqpoint{1.715385in}{2.304714in}}%
\pgfpathlineto{\pgfqpoint{1.787646in}{2.305444in}}%
\pgfpathlineto{\pgfqpoint{1.859907in}{2.306169in}}%
\pgfpathlineto{\pgfqpoint{1.932168in}{2.306889in}}%
\pgfpathlineto{\pgfqpoint{2.004429in}{2.307604in}}%
\pgfpathlineto{\pgfqpoint{2.076690in}{2.308312in}}%
\pgfpathlineto{\pgfqpoint{2.148951in}{2.309013in}}%
\pgfpathlineto{\pgfqpoint{2.221212in}{2.309707in}}%
\pgfpathlineto{\pgfqpoint{2.293473in}{2.310392in}}%
\pgfpathlineto{\pgfqpoint{2.365734in}{2.311069in}}%
\pgfpathlineto{\pgfqpoint{2.437995in}{2.311737in}}%
\pgfpathlineto{\pgfqpoint{2.510256in}{2.312394in}}%
\pgfpathlineto{\pgfqpoint{2.582517in}{2.313041in}}%
\pgfpathlineto{\pgfqpoint{2.654779in}{2.313677in}}%
\pgfpathlineto{\pgfqpoint{2.727040in}{2.314300in}}%
\pgfpathlineto{\pgfqpoint{2.799301in}{2.314912in}}%
\pgfpathlineto{\pgfqpoint{2.871562in}{2.315510in}}%
\pgfpathlineto{\pgfqpoint{2.943823in}{2.316094in}}%
\pgfpathlineto{\pgfqpoint{3.016084in}{2.316664in}}%
\pgfpathlineto{\pgfqpoint{3.088345in}{2.317218in}}%
\pgfpathlineto{\pgfqpoint{3.160606in}{2.317758in}}%
\pgfpathlineto{\pgfqpoint{3.232867in}{2.318281in}}%
\pgfpathlineto{\pgfqpoint{3.305128in}{2.318787in}}%
\pgfpathlineto{\pgfqpoint{3.377389in}{2.319275in}}%
\pgfpathlineto{\pgfqpoint{3.449650in}{2.319745in}}%
\pgfpathlineto{\pgfqpoint{3.521911in}{2.320197in}}%
\pgfpathlineto{\pgfqpoint{3.594172in}{2.320629in}}%
\pgfpathlineto{\pgfqpoint{3.666434in}{2.321041in}}%
\pgfpathlineto{\pgfqpoint{3.738695in}{2.321432in}}%
\pgfpathlineto{\pgfqpoint{3.883217in}{2.322151in}}%
\pgfpathlineto{\pgfqpoint{3.955478in}{2.322455in}}%
\pgfpathlineto{\pgfqpoint{4.027739in}{2.322738in}}%
\pgfpathlineto{\pgfqpoint{4.100000in}{2.322998in}}%
\pgfpathlineto{\pgfqpoint{4.172261in}{2.323238in}}%
\pgfpathlineto{\pgfqpoint{4.244522in}{2.323458in}}%
\pgfpathlineto{\pgfqpoint{4.316783in}{2.323658in}}%
\pgfpathlineto{\pgfqpoint{4.389044in}{2.323838in}}%
\pgfpathlineto{\pgfqpoint{4.461305in}{2.324000in}}%
\pgfpathlineto{\pgfqpoint{4.533566in}{2.324143in}}%
\pgfpathlineto{\pgfqpoint{4.605828in}{2.324268in}}%
\pgfpathlineto{\pgfqpoint{4.678089in}{2.324377in}}%
\pgfpathlineto{\pgfqpoint{4.750350in}{2.324469in}}%
\pgfpathlineto{\pgfqpoint{4.822611in}{2.324545in}}%
\pgfpathlineto{\pgfqpoint{4.894872in}{2.324605in}}%
\pgfpathlineto{\pgfqpoint{4.967133in}{2.324650in}}%
\pgfpathlineto{\pgfqpoint{5.039394in}{2.324681in}}%
\pgfpathlineto{\pgfqpoint{5.111655in}{2.324697in}}%
\pgfpathlineto{\pgfqpoint{5.183916in}{2.324701in}}%
\pgfpathlineto{\pgfqpoint{5.256177in}{2.324691in}}%
\pgfpathlineto{\pgfqpoint{5.328438in}{2.324669in}}%
\pgfpathlineto{\pgfqpoint{5.400699in}{2.324635in}}%
\pgfpathlineto{\pgfqpoint{5.472960in}{2.324590in}}%
\pgfpathlineto{\pgfqpoint{5.545221in}{2.324535in}}%
\pgfpathlineto{\pgfqpoint{5.617483in}{2.324469in}}%
\pgfpathlineto{\pgfqpoint{5.689744in}{2.324393in}}%
\pgfpathlineto{\pgfqpoint{5.762005in}{2.324309in}}%
\pgfpathlineto{\pgfqpoint{5.834266in}{2.324215in}}%
\pgfpathlineto{\pgfqpoint{5.906527in}{2.324114in}}%
\pgfpathlineto{\pgfqpoint{5.978788in}{2.324005in}}%
\pgfpathlineto{\pgfqpoint{6.051049in}{2.323889in}}%
\pgfpathlineto{\pgfqpoint{6.123310in}{2.323767in}}%
\pgfpathlineto{\pgfqpoint{6.195571in}{2.323639in}}%
\pgfpathlineto{\pgfqpoint{6.267832in}{2.323505in}}%
\pgfpathlineto{\pgfqpoint{6.340093in}{2.323367in}}%
\pgfpathlineto{\pgfqpoint{6.412354in}{2.323224in}}%
\pgfpathlineto{\pgfqpoint{6.484615in}{2.323078in}}%
\pgfpathlineto{\pgfqpoint{6.556876in}{2.322928in}}%
\pgfpathlineto{\pgfqpoint{6.629138in}{2.322776in}}%
\pgfpathlineto{\pgfqpoint{6.701399in}{2.322622in}}%
\pgfpathlineto{\pgfqpoint{6.773660in}{2.322466in}}%
\pgfpathlineto{\pgfqpoint{6.918182in}{2.322151in}}%
\pgfusepath{stroke}%
\end{pgfscope}%
\begin{pgfscope}%
\pgfpathrectangle{\pgfqpoint{1.000000in}{0.330000in}}{\pgfqpoint{6.200000in}{2.310000in}}%
\pgfusepath{clip}%
\pgfsetrectcap%
\pgfsetroundjoin%
\pgfsetlinewidth{1.505625pt}%
\definecolor{currentstroke}{rgb}{1.000000,0.498039,0.054902}%
\pgfsetstrokecolor{currentstroke}%
\pgfsetdash{}{0pt}%
\pgfpathmoveto{\pgfqpoint{1.281818in}{1.277589in}}%
\pgfpathlineto{\pgfqpoint{1.354079in}{1.312603in}}%
\pgfpathlineto{\pgfqpoint{1.426340in}{1.347600in}}%
\pgfpathlineto{\pgfqpoint{1.498601in}{1.382553in}}%
\pgfpathlineto{\pgfqpoint{1.570862in}{1.417438in}}%
\pgfpathlineto{\pgfqpoint{1.643124in}{1.452227in}}%
\pgfpathlineto{\pgfqpoint{1.715385in}{1.486895in}}%
\pgfpathlineto{\pgfqpoint{1.787646in}{1.521414in}}%
\pgfpathlineto{\pgfqpoint{1.859907in}{1.555757in}}%
\pgfpathlineto{\pgfqpoint{1.932168in}{1.589895in}}%
\pgfpathlineto{\pgfqpoint{2.004429in}{1.623802in}}%
\pgfpathlineto{\pgfqpoint{2.076690in}{1.657448in}}%
\pgfpathlineto{\pgfqpoint{2.148951in}{1.690804in}}%
\pgfpathlineto{\pgfqpoint{2.221212in}{1.723842in}}%
\pgfpathlineto{\pgfqpoint{2.293473in}{1.756530in}}%
\pgfpathlineto{\pgfqpoint{2.365734in}{1.788840in}}%
\pgfpathlineto{\pgfqpoint{2.437995in}{1.820740in}}%
\pgfpathlineto{\pgfqpoint{2.510256in}{1.852199in}}%
\pgfpathlineto{\pgfqpoint{2.582517in}{1.883187in}}%
\pgfpathlineto{\pgfqpoint{2.654779in}{1.913671in}}%
\pgfpathlineto{\pgfqpoint{2.727040in}{1.943619in}}%
\pgfpathlineto{\pgfqpoint{2.799301in}{1.972998in}}%
\pgfpathlineto{\pgfqpoint{2.871562in}{2.001777in}}%
\pgfpathlineto{\pgfqpoint{2.943823in}{2.029921in}}%
\pgfpathlineto{\pgfqpoint{3.016084in}{2.057397in}}%
\pgfpathlineto{\pgfqpoint{3.088345in}{2.084170in}}%
\pgfpathlineto{\pgfqpoint{3.160606in}{2.110206in}}%
\pgfpathlineto{\pgfqpoint{3.232867in}{2.135470in}}%
\pgfpathlineto{\pgfqpoint{3.305128in}{2.159927in}}%
\pgfpathlineto{\pgfqpoint{3.377389in}{2.183541in}}%
\pgfpathlineto{\pgfqpoint{3.449650in}{2.206275in}}%
\pgfpathlineto{\pgfqpoint{3.521911in}{2.228095in}}%
\pgfpathlineto{\pgfqpoint{3.594172in}{2.248961in}}%
\pgfpathlineto{\pgfqpoint{3.666434in}{2.268838in}}%
\pgfpathlineto{\pgfqpoint{3.738695in}{2.287688in}}%
\pgfpathlineto{\pgfqpoint{3.883217in}{2.322151in}}%
\pgfpathlineto{\pgfqpoint{3.955478in}{2.336311in}}%
\pgfpathlineto{\pgfqpoint{4.027739in}{2.349364in}}%
\pgfpathlineto{\pgfqpoint{4.100000in}{2.361344in}}%
\pgfpathlineto{\pgfqpoint{4.172261in}{2.372287in}}%
\pgfpathlineto{\pgfqpoint{4.244522in}{2.382227in}}%
\pgfpathlineto{\pgfqpoint{4.316783in}{2.391197in}}%
\pgfpathlineto{\pgfqpoint{4.389044in}{2.399232in}}%
\pgfpathlineto{\pgfqpoint{4.461305in}{2.406364in}}%
\pgfpathlineto{\pgfqpoint{4.533566in}{2.412626in}}%
\pgfpathlineto{\pgfqpoint{4.605828in}{2.418050in}}%
\pgfpathlineto{\pgfqpoint{4.678089in}{2.422669in}}%
\pgfpathlineto{\pgfqpoint{4.750350in}{2.426514in}}%
\pgfpathlineto{\pgfqpoint{4.822611in}{2.429616in}}%
\pgfpathlineto{\pgfqpoint{4.894872in}{2.432006in}}%
\pgfpathlineto{\pgfqpoint{4.967133in}{2.433713in}}%
\pgfpathlineto{\pgfqpoint{5.039394in}{2.434769in}}%
\pgfpathlineto{\pgfqpoint{5.111655in}{2.435203in}}%
\pgfpathlineto{\pgfqpoint{5.183916in}{2.435044in}}%
\pgfpathlineto{\pgfqpoint{5.256177in}{2.434320in}}%
\pgfpathlineto{\pgfqpoint{5.328438in}{2.433060in}}%
\pgfpathlineto{\pgfqpoint{5.400699in}{2.431292in}}%
\pgfpathlineto{\pgfqpoint{5.472960in}{2.429044in}}%
\pgfpathlineto{\pgfqpoint{5.545221in}{2.426343in}}%
\pgfpathlineto{\pgfqpoint{5.617483in}{2.423215in}}%
\pgfpathlineto{\pgfqpoint{5.689744in}{2.419687in}}%
\pgfpathlineto{\pgfqpoint{5.762005in}{2.415785in}}%
\pgfpathlineto{\pgfqpoint{5.834266in}{2.411534in}}%
\pgfpathlineto{\pgfqpoint{5.906527in}{2.406960in}}%
\pgfpathlineto{\pgfqpoint{5.978788in}{2.402088in}}%
\pgfpathlineto{\pgfqpoint{6.051049in}{2.396942in}}%
\pgfpathlineto{\pgfqpoint{6.123310in}{2.391545in}}%
\pgfpathlineto{\pgfqpoint{6.195571in}{2.385922in}}%
\pgfpathlineto{\pgfqpoint{6.267832in}{2.380096in}}%
\pgfpathlineto{\pgfqpoint{6.340093in}{2.374090in}}%
\pgfpathlineto{\pgfqpoint{6.412354in}{2.367925in}}%
\pgfpathlineto{\pgfqpoint{6.484615in}{2.361625in}}%
\pgfpathlineto{\pgfqpoint{6.556876in}{2.355210in}}%
\pgfpathlineto{\pgfqpoint{6.629138in}{2.348702in}}%
\pgfpathlineto{\pgfqpoint{6.701399in}{2.342122in}}%
\pgfpathlineto{\pgfqpoint{6.773660in}{2.335490in}}%
\pgfpathlineto{\pgfqpoint{6.918182in}{2.322151in}}%
\pgfusepath{stroke}%
\end{pgfscope}%
\begin{pgfscope}%
\pgfpathrectangle{\pgfqpoint{1.000000in}{0.330000in}}{\pgfqpoint{6.200000in}{2.310000in}}%
\pgfusepath{clip}%
\pgfsetrectcap%
\pgfsetroundjoin%
\pgfsetlinewidth{1.505625pt}%
\definecolor{currentstroke}{rgb}{0.172549,0.627451,0.172549}%
\pgfsetstrokecolor{currentstroke}%
\pgfsetdash{}{0pt}%
\pgfpathmoveto{\pgfqpoint{1.281818in}{1.186035in}}%
\pgfpathlineto{\pgfqpoint{1.354079in}{1.224110in}}%
\pgfpathlineto{\pgfqpoint{1.426340in}{1.262167in}}%
\pgfpathlineto{\pgfqpoint{1.498601in}{1.300176in}}%
\pgfpathlineto{\pgfqpoint{1.570862in}{1.338111in}}%
\pgfpathlineto{\pgfqpoint{1.643124in}{1.375943in}}%
\pgfpathlineto{\pgfqpoint{1.715385in}{1.413644in}}%
\pgfpathlineto{\pgfqpoint{1.787646in}{1.451182in}}%
\pgfpathlineto{\pgfqpoint{1.859907in}{1.488530in}}%
\pgfpathlineto{\pgfqpoint{1.932168in}{1.525657in}}%
\pgfpathlineto{\pgfqpoint{2.004429in}{1.562532in}}%
\pgfpathlineto{\pgfqpoint{2.076690in}{1.599124in}}%
\pgfpathlineto{\pgfqpoint{2.148951in}{1.635401in}}%
\pgfpathlineto{\pgfqpoint{2.221212in}{1.671332in}}%
\pgfpathlineto{\pgfqpoint{2.293473in}{1.706885in}}%
\pgfpathlineto{\pgfqpoint{2.365734in}{1.742026in}}%
\pgfpathlineto{\pgfqpoint{2.437995in}{1.776722in}}%
\pgfpathlineto{\pgfqpoint{2.510256in}{1.810940in}}%
\pgfpathlineto{\pgfqpoint{2.582517in}{1.844645in}}%
\pgfpathlineto{\pgfqpoint{2.654779in}{1.877803in}}%
\pgfpathlineto{\pgfqpoint{2.727040in}{1.910379in}}%
\pgfpathlineto{\pgfqpoint{2.799301in}{1.942337in}}%
\pgfpathlineto{\pgfqpoint{2.871562in}{1.973642in}}%
\pgfpathlineto{\pgfqpoint{2.943823in}{2.004257in}}%
\pgfpathlineto{\pgfqpoint{3.016084in}{2.034145in}}%
\pgfpathlineto{\pgfqpoint{3.088345in}{2.063269in}}%
\pgfpathlineto{\pgfqpoint{3.160606in}{2.091592in}}%
\pgfpathlineto{\pgfqpoint{3.232867in}{2.119076in}}%
\pgfpathlineto{\pgfqpoint{3.305128in}{2.145681in}}%
\pgfpathlineto{\pgfqpoint{3.377389in}{2.171370in}}%
\pgfpathlineto{\pgfqpoint{3.449650in}{2.196102in}}%
\pgfpathlineto{\pgfqpoint{3.521911in}{2.219838in}}%
\pgfpathlineto{\pgfqpoint{3.594172in}{2.242538in}}%
\pgfpathlineto{\pgfqpoint{3.666434in}{2.264160in}}%
\pgfpathlineto{\pgfqpoint{3.738695in}{2.284664in}}%
\pgfpathlineto{\pgfqpoint{3.883217in}{2.322151in}}%
\pgfpathlineto{\pgfqpoint{3.955478in}{2.337546in}}%
\pgfpathlineto{\pgfqpoint{4.027739in}{2.351735in}}%
\pgfpathlineto{\pgfqpoint{4.100000in}{2.364757in}}%
\pgfpathlineto{\pgfqpoint{4.172261in}{2.376650in}}%
\pgfpathlineto{\pgfqpoint{4.244522in}{2.387451in}}%
\pgfpathlineto{\pgfqpoint{4.316783in}{2.397198in}}%
\pgfpathlineto{\pgfqpoint{4.389044in}{2.405927in}}%
\pgfpathlineto{\pgfqpoint{4.461305in}{2.413674in}}%
\pgfpathlineto{\pgfqpoint{4.533566in}{2.420475in}}%
\pgfpathlineto{\pgfqpoint{4.605828in}{2.426365in}}%
\pgfpathlineto{\pgfqpoint{4.678089in}{2.431379in}}%
\pgfpathlineto{\pgfqpoint{4.750350in}{2.435551in}}%
\pgfpathlineto{\pgfqpoint{4.822611in}{2.438916in}}%
\pgfpathlineto{\pgfqpoint{4.894872in}{2.441507in}}%
\pgfpathlineto{\pgfqpoint{4.967133in}{2.443356in}}%
\pgfpathlineto{\pgfqpoint{5.039394in}{2.444498in}}%
\pgfpathlineto{\pgfqpoint{5.111655in}{2.444963in}}%
\pgfpathlineto{\pgfqpoint{5.183916in}{2.444784in}}%
\pgfpathlineto{\pgfqpoint{5.256177in}{2.443993in}}%
\pgfpathlineto{\pgfqpoint{5.328438in}{2.442619in}}%
\pgfpathlineto{\pgfqpoint{5.400699in}{2.440693in}}%
\pgfpathlineto{\pgfqpoint{5.472960in}{2.438246in}}%
\pgfpathlineto{\pgfqpoint{5.545221in}{2.435307in}}%
\pgfpathlineto{\pgfqpoint{5.617483in}{2.431906in}}%
\pgfpathlineto{\pgfqpoint{5.689744in}{2.428070in}}%
\pgfpathlineto{\pgfqpoint{5.762005in}{2.423828in}}%
\pgfpathlineto{\pgfqpoint{5.834266in}{2.419209in}}%
\pgfpathlineto{\pgfqpoint{5.906527in}{2.414238in}}%
\pgfpathlineto{\pgfqpoint{5.978788in}{2.408945in}}%
\pgfpathlineto{\pgfqpoint{6.051049in}{2.403354in}}%
\pgfpathlineto{\pgfqpoint{6.123310in}{2.397492in}}%
\pgfpathlineto{\pgfqpoint{6.195571in}{2.391385in}}%
\pgfpathlineto{\pgfqpoint{6.267832in}{2.385058in}}%
\pgfpathlineto{\pgfqpoint{6.340093in}{2.378535in}}%
\pgfpathlineto{\pgfqpoint{6.412354in}{2.371842in}}%
\pgfpathlineto{\pgfqpoint{6.484615in}{2.365001in}}%
\pgfpathlineto{\pgfqpoint{6.556876in}{2.358037in}}%
\pgfpathlineto{\pgfqpoint{6.629138in}{2.350972in}}%
\pgfpathlineto{\pgfqpoint{6.701399in}{2.343829in}}%
\pgfpathlineto{\pgfqpoint{6.773660in}{2.336630in}}%
\pgfpathlineto{\pgfqpoint{6.918182in}{2.322151in}}%
\pgfusepath{stroke}%
\end{pgfscope}%
\begin{pgfscope}%
\pgfpathrectangle{\pgfqpoint{1.000000in}{0.330000in}}{\pgfqpoint{6.200000in}{2.310000in}}%
\pgfusepath{clip}%
\pgfsetrectcap%
\pgfsetroundjoin%
\pgfsetlinewidth{1.505625pt}%
\definecolor{currentstroke}{rgb}{0.839216,0.152941,0.156863}%
\pgfsetstrokecolor{currentstroke}%
\pgfsetdash{}{0pt}%
\pgfpathmoveto{\pgfqpoint{1.281818in}{1.169632in}}%
\pgfpathlineto{\pgfqpoint{1.354079in}{1.208265in}}%
\pgfpathlineto{\pgfqpoint{1.426340in}{1.246877in}}%
\pgfpathlineto{\pgfqpoint{1.498601in}{1.285443in}}%
\pgfpathlineto{\pgfqpoint{1.570862in}{1.323932in}}%
\pgfpathlineto{\pgfqpoint{1.643124in}{1.362316in}}%
\pgfpathlineto{\pgfqpoint{1.715385in}{1.400566in}}%
\pgfpathlineto{\pgfqpoint{1.787646in}{1.438652in}}%
\pgfpathlineto{\pgfqpoint{1.859907in}{1.476544in}}%
\pgfpathlineto{\pgfqpoint{1.932168in}{1.514210in}}%
\pgfpathlineto{\pgfqpoint{2.004429in}{1.551621in}}%
\pgfpathlineto{\pgfqpoint{2.076690in}{1.588744in}}%
\pgfpathlineto{\pgfqpoint{2.148951in}{1.625548in}}%
\pgfpathlineto{\pgfqpoint{2.221212in}{1.661999in}}%
\pgfpathlineto{\pgfqpoint{2.293473in}{1.698066in}}%
\pgfpathlineto{\pgfqpoint{2.365734in}{1.733715in}}%
\pgfpathlineto{\pgfqpoint{2.437995in}{1.768912in}}%
\pgfpathlineto{\pgfqpoint{2.510256in}{1.803622in}}%
\pgfpathlineto{\pgfqpoint{2.582517in}{1.837813in}}%
\pgfpathlineto{\pgfqpoint{2.654779in}{1.871447in}}%
\pgfpathlineto{\pgfqpoint{2.727040in}{1.904491in}}%
\pgfpathlineto{\pgfqpoint{2.799301in}{1.936908in}}%
\pgfpathlineto{\pgfqpoint{2.871562in}{1.968661in}}%
\pgfpathlineto{\pgfqpoint{2.943823in}{1.999714in}}%
\pgfpathlineto{\pgfqpoint{3.016084in}{2.030029in}}%
\pgfpathlineto{\pgfqpoint{3.088345in}{2.059570in}}%
\pgfpathlineto{\pgfqpoint{3.160606in}{2.088297in}}%
\pgfpathlineto{\pgfqpoint{3.232867in}{2.116173in}}%
\pgfpathlineto{\pgfqpoint{3.305128in}{2.143158in}}%
\pgfpathlineto{\pgfqpoint{3.377389in}{2.169213in}}%
\pgfpathlineto{\pgfqpoint{3.449650in}{2.194298in}}%
\pgfpathlineto{\pgfqpoint{3.521911in}{2.218373in}}%
\pgfpathlineto{\pgfqpoint{3.594172in}{2.241396in}}%
\pgfpathlineto{\pgfqpoint{3.666434in}{2.263328in}}%
\pgfpathlineto{\pgfqpoint{3.738695in}{2.284126in}}%
\pgfpathlineto{\pgfqpoint{3.883217in}{2.322151in}}%
\pgfpathlineto{\pgfqpoint{3.955478in}{2.337774in}}%
\pgfpathlineto{\pgfqpoint{4.027739in}{2.352175in}}%
\pgfpathlineto{\pgfqpoint{4.100000in}{2.365393in}}%
\pgfpathlineto{\pgfqpoint{4.172261in}{2.377466in}}%
\pgfpathlineto{\pgfqpoint{4.244522in}{2.388432in}}%
\pgfpathlineto{\pgfqpoint{4.316783in}{2.398328in}}%
\pgfpathlineto{\pgfqpoint{4.389044in}{2.407192in}}%
\pgfpathlineto{\pgfqpoint{4.461305in}{2.415060in}}%
\pgfpathlineto{\pgfqpoint{4.533566in}{2.421969in}}%
\pgfpathlineto{\pgfqpoint{4.605828in}{2.427953in}}%
\pgfpathlineto{\pgfqpoint{4.678089in}{2.433048in}}%
\pgfpathlineto{\pgfqpoint{4.750350in}{2.437289in}}%
\pgfpathlineto{\pgfqpoint{4.822611in}{2.440711in}}%
\pgfpathlineto{\pgfqpoint{4.894872in}{2.443347in}}%
\pgfpathlineto{\pgfqpoint{4.967133in}{2.445231in}}%
\pgfpathlineto{\pgfqpoint{5.039394in}{2.446395in}}%
\pgfpathlineto{\pgfqpoint{5.111655in}{2.446873in}}%
\pgfpathlineto{\pgfqpoint{5.183916in}{2.446697in}}%
\pgfpathlineto{\pgfqpoint{5.256177in}{2.445898in}}%
\pgfpathlineto{\pgfqpoint{5.328438in}{2.444507in}}%
\pgfpathlineto{\pgfqpoint{5.400699in}{2.442557in}}%
\pgfpathlineto{\pgfqpoint{5.472960in}{2.440076in}}%
\pgfpathlineto{\pgfqpoint{5.545221in}{2.437095in}}%
\pgfpathlineto{\pgfqpoint{5.617483in}{2.433644in}}%
\pgfpathlineto{\pgfqpoint{5.689744in}{2.429751in}}%
\pgfpathlineto{\pgfqpoint{5.762005in}{2.425446in}}%
\pgfpathlineto{\pgfqpoint{5.834266in}{2.420757in}}%
\pgfpathlineto{\pgfqpoint{5.906527in}{2.415711in}}%
\pgfpathlineto{\pgfqpoint{5.978788in}{2.410335in}}%
\pgfpathlineto{\pgfqpoint{6.051049in}{2.404658in}}%
\pgfpathlineto{\pgfqpoint{6.123310in}{2.398704in}}%
\pgfpathlineto{\pgfqpoint{6.195571in}{2.392501in}}%
\pgfpathlineto{\pgfqpoint{6.267832in}{2.386074in}}%
\pgfpathlineto{\pgfqpoint{6.340093in}{2.379447in}}%
\pgfpathlineto{\pgfqpoint{6.412354in}{2.372647in}}%
\pgfpathlineto{\pgfqpoint{6.484615in}{2.365696in}}%
\pgfpathlineto{\pgfqpoint{6.556876in}{2.358620in}}%
\pgfpathlineto{\pgfqpoint{6.629138in}{2.351441in}}%
\pgfpathlineto{\pgfqpoint{6.701399in}{2.344182in}}%
\pgfpathlineto{\pgfqpoint{6.773660in}{2.336866in}}%
\pgfpathlineto{\pgfqpoint{6.918182in}{2.322151in}}%
\pgfusepath{stroke}%
\end{pgfscope}%
\begin{pgfscope}%
\pgfpathrectangle{\pgfqpoint{1.000000in}{0.330000in}}{\pgfqpoint{6.200000in}{2.310000in}}%
\pgfusepath{clip}%
\pgfsetrectcap%
\pgfsetroundjoin%
\pgfsetlinewidth{1.505625pt}%
\definecolor{currentstroke}{rgb}{0.580392,0.403922,0.741176}%
\pgfsetstrokecolor{currentstroke}%
\pgfsetdash{}{0pt}%
\pgfpathmoveto{\pgfqpoint{1.281818in}{1.272122in}}%
\pgfpathlineto{\pgfqpoint{1.354079in}{1.307321in}}%
\pgfpathlineto{\pgfqpoint{1.426340in}{1.342504in}}%
\pgfpathlineto{\pgfqpoint{1.498601in}{1.377642in}}%
\pgfpathlineto{\pgfqpoint{1.570862in}{1.412711in}}%
\pgfpathlineto{\pgfqpoint{1.643124in}{1.447685in}}%
\pgfpathlineto{\pgfqpoint{1.715385in}{1.482536in}}%
\pgfpathlineto{\pgfqpoint{1.787646in}{1.517237in}}%
\pgfpathlineto{\pgfqpoint{1.859907in}{1.551761in}}%
\pgfpathlineto{\pgfqpoint{1.932168in}{1.586080in}}%
\pgfpathlineto{\pgfqpoint{2.004429in}{1.620165in}}%
\pgfpathlineto{\pgfqpoint{2.076690in}{1.653988in}}%
\pgfpathlineto{\pgfqpoint{2.148951in}{1.687520in}}%
\pgfpathlineto{\pgfqpoint{2.221212in}{1.720731in}}%
\pgfpathlineto{\pgfqpoint{2.293473in}{1.753591in}}%
\pgfpathlineto{\pgfqpoint{2.365734in}{1.786069in}}%
\pgfpathlineto{\pgfqpoint{2.437995in}{1.818136in}}%
\pgfpathlineto{\pgfqpoint{2.510256in}{1.849760in}}%
\pgfpathlineto{\pgfqpoint{2.582517in}{1.880909in}}%
\pgfpathlineto{\pgfqpoint{2.654779in}{1.911552in}}%
\pgfpathlineto{\pgfqpoint{2.727040in}{1.941656in}}%
\pgfpathlineto{\pgfqpoint{2.799301in}{1.971189in}}%
\pgfpathlineto{\pgfqpoint{2.871562in}{2.000117in}}%
\pgfpathlineto{\pgfqpoint{2.943823in}{2.028407in}}%
\pgfpathlineto{\pgfqpoint{3.016084in}{2.056025in}}%
\pgfpathlineto{\pgfqpoint{3.088345in}{2.082937in}}%
\pgfpathlineto{\pgfqpoint{3.160606in}{2.109107in}}%
\pgfpathlineto{\pgfqpoint{3.232867in}{2.134502in}}%
\pgfpathlineto{\pgfqpoint{3.305128in}{2.159086in}}%
\pgfpathlineto{\pgfqpoint{3.377389in}{2.182822in}}%
\pgfpathlineto{\pgfqpoint{3.449650in}{2.205674in}}%
\pgfpathlineto{\pgfqpoint{3.521911in}{2.227606in}}%
\pgfpathlineto{\pgfqpoint{3.594172in}{2.248581in}}%
\pgfpathlineto{\pgfqpoint{3.666434in}{2.268561in}}%
\pgfpathlineto{\pgfqpoint{3.738695in}{2.287508in}}%
\pgfpathlineto{\pgfqpoint{3.883217in}{2.322151in}}%
\pgfpathlineto{\pgfqpoint{3.955478in}{2.336387in}}%
\pgfpathlineto{\pgfqpoint{4.027739in}{2.349510in}}%
\pgfpathlineto{\pgfqpoint{4.100000in}{2.361556in}}%
\pgfpathlineto{\pgfqpoint{4.172261in}{2.372559in}}%
\pgfpathlineto{\pgfqpoint{4.244522in}{2.382553in}}%
\pgfpathlineto{\pgfqpoint{4.316783in}{2.391574in}}%
\pgfpathlineto{\pgfqpoint{4.389044in}{2.399653in}}%
\pgfpathlineto{\pgfqpoint{4.461305in}{2.406826in}}%
\pgfpathlineto{\pgfqpoint{4.533566in}{2.413124in}}%
\pgfpathlineto{\pgfqpoint{4.605828in}{2.418580in}}%
\pgfpathlineto{\pgfqpoint{4.678089in}{2.423226in}}%
\pgfpathlineto{\pgfqpoint{4.750350in}{2.427093in}}%
\pgfpathlineto{\pgfqpoint{4.822611in}{2.430214in}}%
\pgfpathlineto{\pgfqpoint{4.894872in}{2.432619in}}%
\pgfpathlineto{\pgfqpoint{4.967133in}{2.434338in}}%
\pgfpathlineto{\pgfqpoint{5.039394in}{2.435402in}}%
\pgfpathlineto{\pgfqpoint{5.111655in}{2.435840in}}%
\pgfpathlineto{\pgfqpoint{5.183916in}{2.435681in}}%
\pgfpathlineto{\pgfqpoint{5.256177in}{2.434955in}}%
\pgfpathlineto{\pgfqpoint{5.328438in}{2.433690in}}%
\pgfpathlineto{\pgfqpoint{5.400699in}{2.431914in}}%
\pgfpathlineto{\pgfqpoint{5.472960in}{2.429654in}}%
\pgfpathlineto{\pgfqpoint{5.545221in}{2.426939in}}%
\pgfpathlineto{\pgfqpoint{5.617483in}{2.423794in}}%
\pgfpathlineto{\pgfqpoint{5.689744in}{2.420247in}}%
\pgfpathlineto{\pgfqpoint{5.762005in}{2.416324in}}%
\pgfpathlineto{\pgfqpoint{5.834266in}{2.412050in}}%
\pgfpathlineto{\pgfqpoint{5.906527in}{2.407451in}}%
\pgfpathlineto{\pgfqpoint{5.978788in}{2.402551in}}%
\pgfpathlineto{\pgfqpoint{6.051049in}{2.397376in}}%
\pgfpathlineto{\pgfqpoint{6.123310in}{2.391949in}}%
\pgfpathlineto{\pgfqpoint{6.195571in}{2.386294in}}%
\pgfpathlineto{\pgfqpoint{6.267832in}{2.380435in}}%
\pgfpathlineto{\pgfqpoint{6.340093in}{2.374394in}}%
\pgfpathlineto{\pgfqpoint{6.412354in}{2.368193in}}%
\pgfpathlineto{\pgfqpoint{6.484615in}{2.361856in}}%
\pgfpathlineto{\pgfqpoint{6.556876in}{2.355404in}}%
\pgfpathlineto{\pgfqpoint{6.629138in}{2.348858in}}%
\pgfpathlineto{\pgfqpoint{6.701399in}{2.342240in}}%
\pgfpathlineto{\pgfqpoint{6.773660in}{2.335569in}}%
\pgfpathlineto{\pgfqpoint{6.918182in}{2.322151in}}%
\pgfusepath{stroke}%
\end{pgfscope}%
\begin{pgfscope}%
\pgfpathrectangle{\pgfqpoint{1.000000in}{0.330000in}}{\pgfqpoint{6.200000in}{2.310000in}}%
\pgfusepath{clip}%
\pgfsetrectcap%
\pgfsetroundjoin%
\pgfsetlinewidth{1.505625pt}%
\definecolor{currentstroke}{rgb}{0.549020,0.337255,0.294118}%
\pgfsetstrokecolor{currentstroke}%
\pgfsetdash{}{0pt}%
\pgfpathmoveto{\pgfqpoint{1.281818in}{1.150050in}}%
\pgfpathlineto{\pgfqpoint{1.354079in}{1.189331in}}%
\pgfpathlineto{\pgfqpoint{1.426340in}{1.228593in}}%
\pgfpathlineto{\pgfqpoint{1.498601in}{1.267806in}}%
\pgfpathlineto{\pgfqpoint{1.570862in}{1.306943in}}%
\pgfpathlineto{\pgfqpoint{1.643124in}{1.345973in}}%
\pgfpathlineto{\pgfqpoint{1.715385in}{1.384867in}}%
\pgfpathlineto{\pgfqpoint{1.787646in}{1.423595in}}%
\pgfpathlineto{\pgfqpoint{1.859907in}{1.462126in}}%
\pgfpathlineto{\pgfqpoint{1.932168in}{1.500428in}}%
\pgfpathlineto{\pgfqpoint{2.004429in}{1.538471in}}%
\pgfpathlineto{\pgfqpoint{2.076690in}{1.576222in}}%
\pgfpathlineto{\pgfqpoint{2.148951in}{1.613649in}}%
\pgfpathlineto{\pgfqpoint{2.221212in}{1.650718in}}%
\pgfpathlineto{\pgfqpoint{2.293473in}{1.687397in}}%
\pgfpathlineto{\pgfqpoint{2.365734in}{1.723651in}}%
\pgfpathlineto{\pgfqpoint{2.437995in}{1.759446in}}%
\pgfpathlineto{\pgfqpoint{2.510256in}{1.794748in}}%
\pgfpathlineto{\pgfqpoint{2.582517in}{1.829520in}}%
\pgfpathlineto{\pgfqpoint{2.654779in}{1.863729in}}%
\pgfpathlineto{\pgfqpoint{2.727040in}{1.897336in}}%
\pgfpathlineto{\pgfqpoint{2.799301in}{1.930307in}}%
\pgfpathlineto{\pgfqpoint{2.871562in}{1.962603in}}%
\pgfpathlineto{\pgfqpoint{2.943823in}{1.994188in}}%
\pgfpathlineto{\pgfqpoint{3.016084in}{2.025023in}}%
\pgfpathlineto{\pgfqpoint{3.088345in}{2.055070in}}%
\pgfpathlineto{\pgfqpoint{3.160606in}{2.084290in}}%
\pgfpathlineto{\pgfqpoint{3.232867in}{2.112644in}}%
\pgfpathlineto{\pgfqpoint{3.305128in}{2.140092in}}%
\pgfpathlineto{\pgfqpoint{3.377389in}{2.166594in}}%
\pgfpathlineto{\pgfqpoint{3.449650in}{2.192110in}}%
\pgfpathlineto{\pgfqpoint{3.521911in}{2.216598in}}%
\pgfpathlineto{\pgfqpoint{3.594172in}{2.240016in}}%
\pgfpathlineto{\pgfqpoint{3.666434in}{2.262323in}}%
\pgfpathlineto{\pgfqpoint{3.738695in}{2.283477in}}%
\pgfpathlineto{\pgfqpoint{3.883217in}{2.322151in}}%
\pgfpathlineto{\pgfqpoint{3.955478in}{2.338033in}}%
\pgfpathlineto{\pgfqpoint{4.027739in}{2.352672in}}%
\pgfpathlineto{\pgfqpoint{4.100000in}{2.366106in}}%
\pgfpathlineto{\pgfqpoint{4.172261in}{2.378376in}}%
\pgfpathlineto{\pgfqpoint{4.244522in}{2.389520in}}%
\pgfpathlineto{\pgfqpoint{4.316783in}{2.399575in}}%
\pgfpathlineto{\pgfqpoint{4.389044in}{2.408581in}}%
\pgfpathlineto{\pgfqpoint{4.461305in}{2.416573in}}%
\pgfpathlineto{\pgfqpoint{4.533566in}{2.423589in}}%
\pgfpathlineto{\pgfqpoint{4.605828in}{2.429666in}}%
\pgfpathlineto{\pgfqpoint{4.678089in}{2.434838in}}%
\pgfpathlineto{\pgfqpoint{4.750350in}{2.439143in}}%
\pgfpathlineto{\pgfqpoint{4.822611in}{2.442614in}}%
\pgfpathlineto{\pgfqpoint{4.894872in}{2.445287in}}%
\pgfpathlineto{\pgfqpoint{4.967133in}{2.447196in}}%
\pgfpathlineto{\pgfqpoint{5.039394in}{2.448373in}}%
\pgfpathlineto{\pgfqpoint{5.111655in}{2.448853in}}%
\pgfpathlineto{\pgfqpoint{5.183916in}{2.448669in}}%
\pgfpathlineto{\pgfqpoint{5.256177in}{2.447852in}}%
\pgfpathlineto{\pgfqpoint{5.328438in}{2.446434in}}%
\pgfpathlineto{\pgfqpoint{5.400699in}{2.444448in}}%
\pgfpathlineto{\pgfqpoint{5.472960in}{2.441923in}}%
\pgfpathlineto{\pgfqpoint{5.545221in}{2.438891in}}%
\pgfpathlineto{\pgfqpoint{5.617483in}{2.435382in}}%
\pgfpathlineto{\pgfqpoint{5.689744in}{2.431425in}}%
\pgfpathlineto{\pgfqpoint{5.762005in}{2.427049in}}%
\pgfpathlineto{\pgfqpoint{5.834266in}{2.422283in}}%
\pgfpathlineto{\pgfqpoint{5.906527in}{2.417155in}}%
\pgfpathlineto{\pgfqpoint{5.978788in}{2.411694in}}%
\pgfpathlineto{\pgfqpoint{6.051049in}{2.405926in}}%
\pgfpathlineto{\pgfqpoint{6.123310in}{2.399878in}}%
\pgfpathlineto{\pgfqpoint{6.195571in}{2.393578in}}%
\pgfpathlineto{\pgfqpoint{6.267832in}{2.387050in}}%
\pgfpathlineto{\pgfqpoint{6.340093in}{2.380321in}}%
\pgfpathlineto{\pgfqpoint{6.412354in}{2.373416in}}%
\pgfpathlineto{\pgfqpoint{6.484615in}{2.366358in}}%
\pgfpathlineto{\pgfqpoint{6.556876in}{2.359173in}}%
\pgfpathlineto{\pgfqpoint{6.629138in}{2.351885in}}%
\pgfpathlineto{\pgfqpoint{6.701399in}{2.344515in}}%
\pgfpathlineto{\pgfqpoint{6.773660in}{2.337089in}}%
\pgfpathlineto{\pgfqpoint{6.918182in}{2.322151in}}%
\pgfusepath{stroke}%
\end{pgfscope}%
\begin{pgfscope}%
\pgfpathrectangle{\pgfqpoint{1.000000in}{0.330000in}}{\pgfqpoint{6.200000in}{2.310000in}}%
\pgfusepath{clip}%
\pgfsetrectcap%
\pgfsetroundjoin%
\pgfsetlinewidth{1.505625pt}%
\definecolor{currentstroke}{rgb}{0.890196,0.466667,0.760784}%
\pgfsetstrokecolor{currentstroke}%
\pgfsetdash{}{0pt}%
\pgfpathmoveto{\pgfqpoint{1.281818in}{1.169632in}}%
\pgfpathlineto{\pgfqpoint{1.354079in}{1.208265in}}%
\pgfpathlineto{\pgfqpoint{1.426340in}{1.246877in}}%
\pgfpathlineto{\pgfqpoint{1.498601in}{1.285443in}}%
\pgfpathlineto{\pgfqpoint{1.570862in}{1.323932in}}%
\pgfpathlineto{\pgfqpoint{1.643124in}{1.362316in}}%
\pgfpathlineto{\pgfqpoint{1.715385in}{1.400566in}}%
\pgfpathlineto{\pgfqpoint{1.787646in}{1.438652in}}%
\pgfpathlineto{\pgfqpoint{1.859907in}{1.476544in}}%
\pgfpathlineto{\pgfqpoint{1.932168in}{1.514210in}}%
\pgfpathlineto{\pgfqpoint{2.004429in}{1.551621in}}%
\pgfpathlineto{\pgfqpoint{2.076690in}{1.588744in}}%
\pgfpathlineto{\pgfqpoint{2.148951in}{1.625548in}}%
\pgfpathlineto{\pgfqpoint{2.221212in}{1.661999in}}%
\pgfpathlineto{\pgfqpoint{2.293473in}{1.698066in}}%
\pgfpathlineto{\pgfqpoint{2.365734in}{1.733715in}}%
\pgfpathlineto{\pgfqpoint{2.437995in}{1.768912in}}%
\pgfpathlineto{\pgfqpoint{2.510256in}{1.803622in}}%
\pgfpathlineto{\pgfqpoint{2.582517in}{1.837813in}}%
\pgfpathlineto{\pgfqpoint{2.654779in}{1.871447in}}%
\pgfpathlineto{\pgfqpoint{2.727040in}{1.904491in}}%
\pgfpathlineto{\pgfqpoint{2.799301in}{1.936908in}}%
\pgfpathlineto{\pgfqpoint{2.871562in}{1.968661in}}%
\pgfpathlineto{\pgfqpoint{2.943823in}{1.999714in}}%
\pgfpathlineto{\pgfqpoint{3.016084in}{2.030029in}}%
\pgfpathlineto{\pgfqpoint{3.088345in}{2.059570in}}%
\pgfpathlineto{\pgfqpoint{3.160606in}{2.088297in}}%
\pgfpathlineto{\pgfqpoint{3.232867in}{2.116173in}}%
\pgfpathlineto{\pgfqpoint{3.305128in}{2.143158in}}%
\pgfpathlineto{\pgfqpoint{3.377389in}{2.169213in}}%
\pgfpathlineto{\pgfqpoint{3.449650in}{2.194298in}}%
\pgfpathlineto{\pgfqpoint{3.521911in}{2.218373in}}%
\pgfpathlineto{\pgfqpoint{3.594172in}{2.241396in}}%
\pgfpathlineto{\pgfqpoint{3.666434in}{2.263328in}}%
\pgfpathlineto{\pgfqpoint{3.738695in}{2.284126in}}%
\pgfpathlineto{\pgfqpoint{3.883217in}{2.322151in}}%
\pgfpathlineto{\pgfqpoint{3.955478in}{2.337774in}}%
\pgfpathlineto{\pgfqpoint{4.027739in}{2.352175in}}%
\pgfpathlineto{\pgfqpoint{4.100000in}{2.365393in}}%
\pgfpathlineto{\pgfqpoint{4.172261in}{2.377466in}}%
\pgfpathlineto{\pgfqpoint{4.244522in}{2.388432in}}%
\pgfpathlineto{\pgfqpoint{4.316783in}{2.398328in}}%
\pgfpathlineto{\pgfqpoint{4.389044in}{2.407192in}}%
\pgfpathlineto{\pgfqpoint{4.461305in}{2.415060in}}%
\pgfpathlineto{\pgfqpoint{4.533566in}{2.421969in}}%
\pgfpathlineto{\pgfqpoint{4.605828in}{2.427953in}}%
\pgfpathlineto{\pgfqpoint{4.678089in}{2.433048in}}%
\pgfpathlineto{\pgfqpoint{4.750350in}{2.437289in}}%
\pgfpathlineto{\pgfqpoint{4.822611in}{2.440711in}}%
\pgfpathlineto{\pgfqpoint{4.894872in}{2.443347in}}%
\pgfpathlineto{\pgfqpoint{4.967133in}{2.445231in}}%
\pgfpathlineto{\pgfqpoint{5.039394in}{2.446395in}}%
\pgfpathlineto{\pgfqpoint{5.111655in}{2.446873in}}%
\pgfpathlineto{\pgfqpoint{5.183916in}{2.446697in}}%
\pgfpathlineto{\pgfqpoint{5.256177in}{2.445898in}}%
\pgfpathlineto{\pgfqpoint{5.328438in}{2.444507in}}%
\pgfpathlineto{\pgfqpoint{5.400699in}{2.442557in}}%
\pgfpathlineto{\pgfqpoint{5.472960in}{2.440076in}}%
\pgfpathlineto{\pgfqpoint{5.545221in}{2.437095in}}%
\pgfpathlineto{\pgfqpoint{5.617483in}{2.433644in}}%
\pgfpathlineto{\pgfqpoint{5.689744in}{2.429751in}}%
\pgfpathlineto{\pgfqpoint{5.762005in}{2.425446in}}%
\pgfpathlineto{\pgfqpoint{5.834266in}{2.420757in}}%
\pgfpathlineto{\pgfqpoint{5.906527in}{2.415711in}}%
\pgfpathlineto{\pgfqpoint{5.978788in}{2.410335in}}%
\pgfpathlineto{\pgfqpoint{6.051049in}{2.404658in}}%
\pgfpathlineto{\pgfqpoint{6.123310in}{2.398704in}}%
\pgfpathlineto{\pgfqpoint{6.195571in}{2.392501in}}%
\pgfpathlineto{\pgfqpoint{6.267832in}{2.386074in}}%
\pgfpathlineto{\pgfqpoint{6.340093in}{2.379447in}}%
\pgfpathlineto{\pgfqpoint{6.412354in}{2.372647in}}%
\pgfpathlineto{\pgfqpoint{6.484615in}{2.365696in}}%
\pgfpathlineto{\pgfqpoint{6.556876in}{2.358620in}}%
\pgfpathlineto{\pgfqpoint{6.629138in}{2.351441in}}%
\pgfpathlineto{\pgfqpoint{6.701399in}{2.344182in}}%
\pgfpathlineto{\pgfqpoint{6.773660in}{2.336866in}}%
\pgfpathlineto{\pgfqpoint{6.918182in}{2.322151in}}%
\pgfusepath{stroke}%
\end{pgfscope}%
\begin{pgfscope}%
\pgfsetroundcap%
\pgfsetroundjoin%
\pgfsetlinewidth{1.003750pt}%
\definecolor{currentstroke}{rgb}{0.000000,0.000000,0.000000}%
\pgfsetstrokecolor{currentstroke}%
\pgfsetdash{}{0pt}%
\pgfpathmoveto{\pgfqpoint{1.748785in}{0.435000in}}%
\pgfpathquadraticcurveto{\pgfqpoint{1.529196in}{0.435000in}}{\pgfqpoint{1.309607in}{0.435000in}}%
\pgfusepath{stroke}%
\end{pgfscope}%
\begin{pgfscope}%
\pgfsetbuttcap%
\pgfsetmiterjoin%
\definecolor{currentfill}{rgb}{0.800000,0.800000,0.800000}%
\pgfsetfillcolor{currentfill}%
\pgfsetlinewidth{1.003750pt}%
\definecolor{currentstroke}{rgb}{0.000000,0.000000,0.000000}%
\pgfsetstrokecolor{currentstroke}%
\pgfsetdash{}{0pt}%
\pgfpathmoveto{\pgfqpoint{1.806509in}{0.338549in}}%
\pgfpathcurveto{\pgfqpoint{1.841231in}{0.303827in}}{\pgfqpoint{2.670707in}{0.303827in}}{\pgfqpoint{2.705429in}{0.338549in}}%
\pgfpathcurveto{\pgfqpoint{2.740152in}{0.373272in}}{\pgfqpoint{2.740152in}{0.496728in}}{\pgfqpoint{2.705429in}{0.531451in}}%
\pgfpathcurveto{\pgfqpoint{2.670707in}{0.566173in}}{\pgfqpoint{1.841231in}{0.566173in}}{\pgfqpoint{1.806509in}{0.531451in}}%
\pgfpathcurveto{\pgfqpoint{1.771787in}{0.496728in}}{\pgfqpoint{1.771787in}{0.373272in}}{\pgfqpoint{1.806509in}{0.338549in}}%
\pgfpathclose%
\pgfusepath{stroke,fill}%
\end{pgfscope}%
\begin{pgfscope}%
\definecolor{textcolor}{rgb}{0.000000,0.000000,0.000000}%
\pgfsetstrokecolor{textcolor}%
\pgfsetfillcolor{textcolor}%
\pgftext[x=2.670707in,y=0.435000in,right,]{\color{textcolor}\rmfamily\fontsize{10.000000}{12.000000}\selectfont \(\displaystyle \Delta =\) -0.2 inch}%
\end{pgfscope}%
\begin{pgfscope}%
\pgfsetbuttcap%
\pgfsetmiterjoin%
\definecolor{currentfill}{rgb}{0.800000,0.800000,0.800000}%
\pgfsetfillcolor{currentfill}%
\pgfsetlinewidth{1.003750pt}%
\definecolor{currentstroke}{rgb}{0.000000,0.000000,0.000000}%
\pgfsetstrokecolor{currentstroke}%
\pgfsetdash{}{0pt}%
\pgfpathmoveto{\pgfqpoint{0.965278in}{0.358599in}}%
\pgfpathcurveto{\pgfqpoint{1.000000in}{0.323877in}}{\pgfqpoint{2.720682in}{0.323877in}}{\pgfqpoint{2.755404in}{0.358599in}}%
\pgfpathcurveto{\pgfqpoint{2.790127in}{0.393321in}}{\pgfqpoint{2.790127in}{0.668784in}}{\pgfqpoint{2.755404in}{0.703506in}}%
\pgfpathcurveto{\pgfqpoint{2.720682in}{0.738228in}}{\pgfqpoint{1.000000in}{0.738228in}}{\pgfqpoint{0.965278in}{0.703506in}}%
\pgfpathcurveto{\pgfqpoint{0.930556in}{0.668784in}}{\pgfqpoint{0.930556in}{0.393321in}}{\pgfqpoint{0.965278in}{0.358599in}}%
\pgfpathclose%
\pgfusepath{stroke,fill}%
\end{pgfscope}%
\begin{pgfscope}%
\definecolor{textcolor}{rgb}{0.000000,0.000000,0.000000}%
\pgfsetstrokecolor{textcolor}%
\pgfsetfillcolor{textcolor}%
\pgftext[x=1.000000in, y=0.580049in, left, base]{\color{textcolor}\rmfamily\fontsize{10.000000}{12.000000}\selectfont Max combo: 1.0D + 1.0Lr0}%
\end{pgfscope}%
\begin{pgfscope}%
\definecolor{textcolor}{rgb}{0.000000,0.000000,0.000000}%
\pgfsetstrokecolor{textcolor}%
\pgfsetfillcolor{textcolor}%
\pgftext[x=1.000000in, y=0.428043in, left, base]{\color{textcolor}\rmfamily\fontsize{10.000000}{12.000000}\selectfont ASCE7-16 Sec. 2.4.1 (LC 3)}%
\end{pgfscope}%
\begin{pgfscope}%
\pgfsetroundcap%
\pgfsetroundjoin%
\pgfsetlinewidth{1.003750pt}%
\definecolor{currentstroke}{rgb}{0.000000,0.000000,0.000000}%
\pgfsetstrokecolor{currentstroke}%
\pgfsetdash{}{0pt}%
\pgfpathmoveto{\pgfqpoint{1.748785in}{2.300281in}}%
\pgfpathquadraticcurveto{\pgfqpoint{1.529196in}{2.300281in}}{\pgfqpoint{1.309607in}{2.300281in}}%
\pgfusepath{stroke}%
\end{pgfscope}%
\begin{pgfscope}%
\pgfsetbuttcap%
\pgfsetmiterjoin%
\definecolor{currentfill}{rgb}{0.800000,0.800000,0.800000}%
\pgfsetfillcolor{currentfill}%
\pgfsetlinewidth{1.003750pt}%
\definecolor{currentstroke}{rgb}{0.000000,0.000000,0.000000}%
\pgfsetstrokecolor{currentstroke}%
\pgfsetdash{}{0pt}%
\pgfpathmoveto{\pgfqpoint{1.806509in}{2.203830in}}%
\pgfpathcurveto{\pgfqpoint{1.841231in}{2.169108in}}{\pgfqpoint{2.670707in}{2.169108in}}{\pgfqpoint{2.705429in}{2.203830in}}%
\pgfpathcurveto{\pgfqpoint{2.740152in}{2.238552in}}{\pgfqpoint{2.740152in}{2.362009in}}{\pgfqpoint{2.705429in}{2.396731in}}%
\pgfpathcurveto{\pgfqpoint{2.670707in}{2.431453in}}{\pgfqpoint{1.841231in}{2.431453in}}{\pgfqpoint{1.806509in}{2.396731in}}%
\pgfpathcurveto{\pgfqpoint{1.771787in}{2.362009in}}{\pgfqpoint{1.771787in}{2.238552in}}{\pgfqpoint{1.806509in}{2.203830in}}%
\pgfpathclose%
\pgfusepath{stroke,fill}%
\end{pgfscope}%
\begin{pgfscope}%
\definecolor{textcolor}{rgb}{0.000000,0.000000,0.000000}%
\pgfsetstrokecolor{textcolor}%
\pgfsetfillcolor{textcolor}%
\pgftext[x=2.670707in,y=2.300281in,right,]{\color{textcolor}\rmfamily\fontsize{10.000000}{12.000000}\selectfont \(\displaystyle \Delta =\) -0.0 inch}%
\end{pgfscope}%
\begin{pgfscope}%
\pgfsetbuttcap%
\pgfsetmiterjoin%
\definecolor{currentfill}{rgb}{0.800000,0.800000,0.800000}%
\pgfsetfillcolor{currentfill}%
\pgfsetlinewidth{1.003750pt}%
\definecolor{currentstroke}{rgb}{0.000000,0.000000,0.000000}%
\pgfsetstrokecolor{currentstroke}%
\pgfsetdash{}{0pt}%
\pgfpathmoveto{\pgfqpoint{0.965278in}{0.375574in}}%
\pgfpathcurveto{\pgfqpoint{1.000000in}{0.340852in}}{\pgfqpoint{2.390820in}{0.340852in}}{\pgfqpoint{2.425542in}{0.375574in}}%
\pgfpathcurveto{\pgfqpoint{2.460265in}{0.410297in}}{\pgfqpoint{2.460265in}{0.676500in}}{\pgfqpoint{2.425542in}{0.711222in}}%
\pgfpathcurveto{\pgfqpoint{2.390820in}{0.745944in}}{\pgfqpoint{1.000000in}{0.745944in}}{\pgfqpoint{0.965278in}{0.711222in}}%
\pgfpathcurveto{\pgfqpoint{0.930556in}{0.676500in}}{\pgfqpoint{0.930556in}{0.410297in}}{\pgfqpoint{0.965278in}{0.375574in}}%
\pgfpathclose%
\pgfusepath{stroke,fill}%
\end{pgfscope}%
\begin{pgfscope}%
\definecolor{textcolor}{rgb}{0.000000,0.000000,0.000000}%
\pgfsetstrokecolor{textcolor}%
\pgfsetfillcolor{textcolor}%
\pgftext[x=1.000000in, y=0.580049in, left, base]{\color{textcolor}\rmfamily\fontsize{10.000000}{12.000000}\selectfont Max combo: 1.0L0}%
\end{pgfscope}%
\begin{pgfscope}%
\definecolor{textcolor}{rgb}{0.000000,0.000000,0.000000}%
\pgfsetstrokecolor{textcolor}%
\pgfsetfillcolor{textcolor}%
\pgftext[x=1.000000in, y=0.437303in, left, base]{\color{textcolor}\rmfamily\fontsize{10.000000}{12.000000}\selectfont L only deflection check}%
\end{pgfscope}%
\end{pgfpicture}%
\makeatother%
\endgroup%

\end{center}
\caption{Deflection Envelope}
\end{figure}
Tl Deflection Check: 
$\Delta_{max} = -0.15 {\color{darkBlue}{\mathbf{ \; in}}} = \cfrac{L}{465} < \cfrac{L}{120.0}  \;  \mathbf{(OK)}$\\
\bigbreak
Ll Deflection Check: 
$\Delta_{max} = -0.0 {\color{darkBlue}{\mathbf{ \; in}}} = \cfrac{L}{40191} < \cfrac{L}{180.0}  \;  \mathbf{(OK)}$\\
\bigbreak
\vspace{-30pt}
%	---------------------------------- REACTIONS ---------------------------------
\section{Reactions}
The following is a summary of service-level reactions at each support:
\begin{table}[ht]
\caption{Reactions at Supports}
\centering
\begin{tabular}{l l l l l l l l }
\hline
Loc. & Type & D & E & L0 & L1 & Lr0 & Lr1\\
\hline
6 {\color{darkBlue}{\textbf{ft}}} & Shear & 2.5 {\color{darkBlue}{\textbf{kip}}} & 0.0 {\color{darkBlue}{\textbf{kip}}} & 0.3 {\color{darkBlue}{\textbf{kip}}} & 0.0 {\color{darkBlue}{\textbf{kip}}} & 1.5 {\color{darkBlue}{\textbf{kip}}} & 0.1 {\color{darkBlue}{\textbf{kip}}}\\ 
13 {\color{darkBlue}{\textbf{ft}}} & Shear & 3.3 {\color{darkBlue}{\textbf{kip}}} & 0.1 {\color{darkBlue}{\textbf{kip}}} & -0.0 {\color{darkBlue}{\textbf{kip}}} & 2.7 {\color{darkBlue}{\textbf{kip}}} & -0.3 {\color{darkBlue}{\textbf{kip}}} & 0.7 {\color{darkBlue}{\textbf{kip}}}\\ 
\hline
\end{tabular}
\end{table}
\end{document}