\documentclass[12pt, fleqn]{article}
\usepackage{pgfplots}
\usepackage{bm}
\usepackage{marginnote}
\usepackage{wallpaper}
\usepackage{lastpage}
\usepackage[left=1.3cm,right=2.0cm,top=1.8cm,bottom=5.0cm,marginparwidth=3.4cm]{geometry}
\usepackage{amsmath}
\usepackage{amssymb}
\usepackage{xcolor}
\usepackage{enumitem}
\usepackage{float}
\usepackage{textgreek}
\usepackage{textcomp}
\usepackage{fancyhdr}
\usepackage{graphicx}
\usepackage{pstricks}
\usepackage{subfigure}
\usepackage{caption}
\captionsetup{justification=centering,labelfont=bf, belowskip=12pt,aboveskip=12pt}
\usepackage{textcomp}
\setlength{\headheight}{70pt}
\setlength{\textfloatsep}{12pt}
\setlength{\intextsep}{0pt}
\pagestyle{fancy}\fancyhf{}
\renewcommand{\headrulewidth}{0pt}
\definecolor{darkBlue}{cmyk}{.80, .32, 0, 0}
\setlength{\parindent}{0cm}
\newcommand{\tab}{\hspace*{2em}}
\newcommand\BackgroundStructure{
\setlength{\unitlength}{1mm}
\setlength\fboxsep{0mm}
\setlength\fboxrule{0.5mm}
\put(10, 20pr){\fcolorbox{black}{gray!5}{\framebox(155,247){}}}
\put(165, 20){\fcolorbox{black}{gray!10}{\framebox(37,247){}}}
\put(10, 262){\fcolorbox{black}{white!10}{\framebox(192, 25){}}}
\put(175, 263){\includegraphics{}}}
\setlength{\abovedisplayskip}{0pt}
\setlength{\belowdisplayskip}{0pt}
%	----------------------------------- HEADER -----------------------------------
\fancyhead[L]{\begin{tabular}{l l | l l}
\textbf{Member:} & {Test2} & \textbf{Firm:} & {ABC Company} \\
\textbf{Project:} & {123 Maple Street, San Francisco CA} & \textbf{Engineer:} & {Jesse} \\
\textbf{Level:} & {2} & \textbf{Checker:} & {Joey}  \\
\textbf{Date:} & {2021-05-23} & \textbf{Page:} & \thepage\\
\end{tabular}}
%	---------------------------- APPLIED LOADS SECTION ---------------------------
\begin{document}
\begin{center}
\textbf{\LARGE W14x132 Design Report}
\end{center}
\section{Applied Loading}
\vspace{-30pt}
\begin{figure}[H]
\begin{center}
%% Creator: Matplotlib, PGF backend
%%
%% To include the figure in your LaTeX document, write
%%   \input{<filename>.pgf}
%%
%% Make sure the required packages are loaded in your preamble
%%   \usepackage{pgf}
%%
%% Figures using additional raster images can only be included by \input if
%% they are in the same directory as the main LaTeX file. For loading figures
%% from other directories you can use the `import` package
%%   \usepackage{import}
%%
%% and then include the figures with
%%   \import{<path to file>}{<filename>.pgf}
%%
%% Matplotlib used the following preamble
%%
\begingroup%
\makeatletter%
\begin{pgfpicture}%
\pgfpathrectangle{\pgfpointorigin}{\pgfqpoint{8.000000in}{3.000000in}}%
\pgfusepath{use as bounding box, clip}%
\begin{pgfscope}%
\pgfpathrectangle{\pgfqpoint{1.000000in}{0.825864in}}{\pgfqpoint{6.200000in}{1.814136in}}%
\pgfusepath{clip}%
\pgfsetbuttcap%
\pgfsetmiterjoin%
\definecolor{currentfill}{rgb}{0.121569,0.466667,0.705882}%
\pgfsetfillcolor{currentfill}%
\pgfsetfillopacity{0.400000}%
\pgfsetlinewidth{1.003750pt}%
\definecolor{currentstroke}{rgb}{0.121569,0.466667,0.705882}%
\pgfsetstrokecolor{currentstroke}%
\pgfsetstrokeopacity{0.400000}%
\pgfsetdash{}{0pt}%
\pgfpathmoveto{\pgfqpoint{1.281818in}{0.908325in}}%
\pgfpathlineto{\pgfqpoint{1.281818in}{1.054208in}}%
\pgfpathlineto{\pgfqpoint{6.918182in}{1.054208in}}%
\pgfpathlineto{\pgfqpoint{6.918182in}{0.908325in}}%
\pgfpathclose%
\pgfusepath{stroke,fill}%
\end{pgfscope}%
\begin{pgfscope}%
\pgfpathrectangle{\pgfqpoint{1.000000in}{0.825864in}}{\pgfqpoint{6.200000in}{1.814136in}}%
\pgfusepath{clip}%
\pgfsetbuttcap%
\pgfsetmiterjoin%
\definecolor{currentfill}{rgb}{1.000000,0.498039,0.054902}%
\pgfsetfillcolor{currentfill}%
\pgfsetfillopacity{0.400000}%
\pgfsetlinewidth{1.003750pt}%
\definecolor{currentstroke}{rgb}{1.000000,0.498039,0.054902}%
\pgfsetstrokecolor{currentstroke}%
\pgfsetstrokeopacity{0.400000}%
\pgfsetdash{}{0pt}%
\pgfpathmoveto{\pgfqpoint{1.281818in}{1.054208in}}%
\pgfpathlineto{\pgfqpoint{1.281818in}{1.307919in}}%
\pgfpathlineto{\pgfqpoint{6.918182in}{1.307919in}}%
\pgfpathlineto{\pgfqpoint{6.918182in}{1.054208in}}%
\pgfpathclose%
\pgfusepath{stroke,fill}%
\end{pgfscope}%
\begin{pgfscope}%
\pgfpathrectangle{\pgfqpoint{1.000000in}{0.825864in}}{\pgfqpoint{6.200000in}{1.814136in}}%
\pgfusepath{clip}%
\pgfsetbuttcap%
\pgfsetmiterjoin%
\definecolor{currentfill}{rgb}{0.172549,0.627451,0.172549}%
\pgfsetfillcolor{currentfill}%
\pgfsetfillopacity{0.400000}%
\pgfsetlinewidth{1.003750pt}%
\definecolor{currentstroke}{rgb}{0.172549,0.627451,0.172549}%
\pgfsetstrokecolor{currentstroke}%
\pgfsetstrokeopacity{0.400000}%
\pgfsetdash{}{0pt}%
\pgfpathmoveto{\pgfqpoint{1.281818in}{1.307919in}}%
\pgfpathlineto{\pgfqpoint{1.281818in}{2.557539in}}%
\pgfpathlineto{\pgfqpoint{6.918182in}{2.557539in}}%
\pgfpathlineto{\pgfqpoint{6.918182in}{1.307919in}}%
\pgfpathclose%
\pgfusepath{stroke,fill}%
\end{pgfscope}%
\begin{pgfscope}%
\pgfsetbuttcap%
\pgfsetmiterjoin%
\definecolor{currentfill}{rgb}{0.800000,0.800000,0.800000}%
\pgfsetfillcolor{currentfill}%
\pgfsetlinewidth{1.003750pt}%
\definecolor{currentstroke}{rgb}{0.000000,0.000000,0.000000}%
\pgfsetstrokecolor{currentstroke}%
\pgfsetdash{}{0pt}%
\pgfpathmoveto{\pgfqpoint{3.983970in}{0.884816in}}%
\pgfpathcurveto{\pgfqpoint{4.018692in}{0.850094in}}{\pgfqpoint{4.181308in}{0.850094in}}{\pgfqpoint{4.216030in}{0.884816in}}%
\pgfpathcurveto{\pgfqpoint{4.250752in}{0.919538in}}{\pgfqpoint{4.250752in}{1.042995in}}{\pgfqpoint{4.216030in}{1.077717in}}%
\pgfpathcurveto{\pgfqpoint{4.181308in}{1.112439in}}{\pgfqpoint{4.018692in}{1.112439in}}{\pgfqpoint{3.983970in}{1.077717in}}%
\pgfpathcurveto{\pgfqpoint{3.949248in}{1.042995in}}{\pgfqpoint{3.949248in}{0.919538in}}{\pgfqpoint{3.983970in}{0.884816in}}%
\pgfpathclose%
\pgfusepath{stroke,fill}%
\end{pgfscope}%
\begin{pgfscope}%
\definecolor{textcolor}{rgb}{0.000000,0.000000,0.000000}%
\pgfsetstrokecolor{textcolor}%
\pgfsetfillcolor{textcolor}%
\pgftext[x=4.100000in,y=0.981266in,,]{\color{textcolor}\rmfamily\fontsize{10.000000}{12.000000}\selectfont w\textsubscript{1}}%
\end{pgfscope}%
\begin{pgfscope}%
\pgfsetbuttcap%
\pgfsetmiterjoin%
\definecolor{currentfill}{rgb}{0.800000,0.800000,0.800000}%
\pgfsetfillcolor{currentfill}%
\pgfsetlinewidth{1.003750pt}%
\definecolor{currentstroke}{rgb}{0.000000,0.000000,0.000000}%
\pgfsetstrokecolor{currentstroke}%
\pgfsetdash{}{0pt}%
\pgfpathmoveto{\pgfqpoint{3.983970in}{1.084613in}}%
\pgfpathcurveto{\pgfqpoint{4.018692in}{1.049891in}}{\pgfqpoint{4.181308in}{1.049891in}}{\pgfqpoint{4.216030in}{1.084613in}}%
\pgfpathcurveto{\pgfqpoint{4.250752in}{1.119335in}}{\pgfqpoint{4.250752in}{1.242792in}}{\pgfqpoint{4.216030in}{1.277514in}}%
\pgfpathcurveto{\pgfqpoint{4.181308in}{1.312236in}}{\pgfqpoint{4.018692in}{1.312236in}}{\pgfqpoint{3.983970in}{1.277514in}}%
\pgfpathcurveto{\pgfqpoint{3.949248in}{1.242792in}}{\pgfqpoint{3.949248in}{1.119335in}}{\pgfqpoint{3.983970in}{1.084613in}}%
\pgfpathclose%
\pgfusepath{stroke,fill}%
\end{pgfscope}%
\begin{pgfscope}%
\definecolor{textcolor}{rgb}{0.000000,0.000000,0.000000}%
\pgfsetstrokecolor{textcolor}%
\pgfsetfillcolor{textcolor}%
\pgftext[x=4.100000in,y=1.181064in,,]{\color{textcolor}\rmfamily\fontsize{10.000000}{12.000000}\selectfont w\textsubscript{2}}%
\end{pgfscope}%
\begin{pgfscope}%
\pgfsetbuttcap%
\pgfsetmiterjoin%
\definecolor{currentfill}{rgb}{0.800000,0.800000,0.800000}%
\pgfsetfillcolor{currentfill}%
\pgfsetlinewidth{1.003750pt}%
\definecolor{currentstroke}{rgb}{0.000000,0.000000,0.000000}%
\pgfsetstrokecolor{currentstroke}%
\pgfsetdash{}{0pt}%
\pgfpathmoveto{\pgfqpoint{3.983970in}{1.836279in}}%
\pgfpathcurveto{\pgfqpoint{4.018692in}{1.801556in}}{\pgfqpoint{4.181308in}{1.801556in}}{\pgfqpoint{4.216030in}{1.836279in}}%
\pgfpathcurveto{\pgfqpoint{4.250752in}{1.871001in}}{\pgfqpoint{4.250752in}{1.994458in}}{\pgfqpoint{4.216030in}{2.029180in}}%
\pgfpathcurveto{\pgfqpoint{4.181308in}{2.063902in}}{\pgfqpoint{4.018692in}{2.063902in}}{\pgfqpoint{3.983970in}{2.029180in}}%
\pgfpathcurveto{\pgfqpoint{3.949248in}{1.994458in}}{\pgfqpoint{3.949248in}{1.871001in}}{\pgfqpoint{3.983970in}{1.836279in}}%
\pgfpathclose%
\pgfusepath{stroke,fill}%
\end{pgfscope}%
\begin{pgfscope}%
\definecolor{textcolor}{rgb}{0.000000,0.000000,0.000000}%
\pgfsetstrokecolor{textcolor}%
\pgfsetfillcolor{textcolor}%
\pgftext[x=4.100000in,y=1.932729in,,]{\color{textcolor}\rmfamily\fontsize{10.000000}{12.000000}\selectfont w\textsubscript{3}}%
\end{pgfscope}%
\begin{pgfscope}%
\pgfsetroundcap%
\pgfsetroundjoin%
\pgfsetlinewidth{1.003750pt}%
\definecolor{currentstroke}{rgb}{0.000000,0.000000,0.000000}%
\pgfsetstrokecolor{currentstroke}%
\pgfsetdash{}{0pt}%
\pgfpathmoveto{\pgfqpoint{3.951675in}{2.158833in}}%
\pgfpathquadraticcurveto{\pgfqpoint{3.951675in}{1.547451in}}{\pgfqpoint{3.951675in}{0.944962in}}%
\pgfusepath{stroke}%
\end{pgfscope}%
\begin{pgfscope}%
\pgfsetroundcap%
\pgfsetroundjoin%
\pgfsetlinewidth{1.003750pt}%
\definecolor{currentstroke}{rgb}{0.000000,0.000000,0.000000}%
\pgfsetstrokecolor{currentstroke}%
\pgfsetdash{}{0pt}%
\pgfpathmoveto{\pgfqpoint{4.035008in}{1.011629in}}%
\pgfpathlineto{\pgfqpoint{3.951675in}{0.944962in}}%
\pgfpathlineto{\pgfqpoint{3.868341in}{1.011629in}}%
\pgfusepath{stroke}%
\end{pgfscope}%
\begin{pgfscope}%
\pgfsetbuttcap%
\pgfsetmiterjoin%
\definecolor{currentfill}{rgb}{0.800000,0.800000,0.800000}%
\pgfsetfillcolor{currentfill}%
\pgfsetlinewidth{1.003750pt}%
\definecolor{currentstroke}{rgb}{0.000000,0.000000,0.000000}%
\pgfsetstrokecolor{currentstroke}%
\pgfsetdash{}{0pt}%
\pgfpathmoveto{\pgfqpoint{3.820714in}{2.223140in}}%
\pgfpathcurveto{\pgfqpoint{3.862381in}{2.181473in}}{\pgfqpoint{4.040969in}{2.181473in}}{\pgfqpoint{4.082635in}{2.223140in}}%
\pgfpathcurveto{\pgfqpoint{4.124302in}{2.264806in}}{\pgfqpoint{4.124302in}{2.412954in}}{\pgfqpoint{4.082635in}{2.454621in}}%
\pgfpathcurveto{\pgfqpoint{4.040969in}{2.496287in}}{\pgfqpoint{3.862381in}{2.496287in}}{\pgfqpoint{3.820714in}{2.454621in}}%
\pgfpathcurveto{\pgfqpoint{3.779047in}{2.412954in}}{\pgfqpoint{3.779047in}{2.264806in}}{\pgfqpoint{3.820714in}{2.223140in}}%
\pgfpathclose%
\pgfusepath{stroke,fill}%
\end{pgfscope}%
\begin{pgfscope}%
\definecolor{textcolor}{rgb}{0.000000,0.000000,0.000000}%
\pgfsetstrokecolor{textcolor}%
\pgfsetfillcolor{textcolor}%
\pgftext[x=3.951675in,y=2.297213in,,base]{\color{textcolor}\rmfamily\fontsize{12.000000}{14.400000}\selectfont \(\displaystyle P_x\)}%
\end{pgfscope}%
\begin{pgfscope}%
\pgfsetroundcap%
\pgfsetroundjoin%
\pgfsetlinewidth{1.003750pt}%
\definecolor{currentstroke}{rgb}{0.000000,0.000000,0.000000}%
\pgfsetstrokecolor{currentstroke}%
\pgfsetdash{}{0pt}%
\pgfpathmoveto{\pgfqpoint{5.583254in}{2.158833in}}%
\pgfpathquadraticcurveto{\pgfqpoint{5.583254in}{1.547451in}}{\pgfqpoint{5.583254in}{0.944962in}}%
\pgfusepath{stroke}%
\end{pgfscope}%
\begin{pgfscope}%
\pgfsetroundcap%
\pgfsetroundjoin%
\pgfsetlinewidth{1.003750pt}%
\definecolor{currentstroke}{rgb}{0.000000,0.000000,0.000000}%
\pgfsetstrokecolor{currentstroke}%
\pgfsetdash{}{0pt}%
\pgfpathmoveto{\pgfqpoint{5.666587in}{1.011629in}}%
\pgfpathlineto{\pgfqpoint{5.583254in}{0.944962in}}%
\pgfpathlineto{\pgfqpoint{5.499920in}{1.011629in}}%
\pgfusepath{stroke}%
\end{pgfscope}%
\begin{pgfscope}%
\pgfsetbuttcap%
\pgfsetmiterjoin%
\definecolor{currentfill}{rgb}{0.800000,0.800000,0.800000}%
\pgfsetfillcolor{currentfill}%
\pgfsetlinewidth{1.003750pt}%
\definecolor{currentstroke}{rgb}{0.000000,0.000000,0.000000}%
\pgfsetstrokecolor{currentstroke}%
\pgfsetdash{}{0pt}%
\pgfpathmoveto{\pgfqpoint{5.452293in}{2.223140in}}%
\pgfpathcurveto{\pgfqpoint{5.493960in}{2.181473in}}{\pgfqpoint{5.672548in}{2.181473in}}{\pgfqpoint{5.714214in}{2.223140in}}%
\pgfpathcurveto{\pgfqpoint{5.755881in}{2.264806in}}{\pgfqpoint{5.755881in}{2.412954in}}{\pgfqpoint{5.714214in}{2.454621in}}%
\pgfpathcurveto{\pgfqpoint{5.672548in}{2.496287in}}{\pgfqpoint{5.493960in}{2.496287in}}{\pgfqpoint{5.452293in}{2.454621in}}%
\pgfpathcurveto{\pgfqpoint{5.410626in}{2.412954in}}{\pgfqpoint{5.410626in}{2.264806in}}{\pgfqpoint{5.452293in}{2.223140in}}%
\pgfpathclose%
\pgfusepath{stroke,fill}%
\end{pgfscope}%
\begin{pgfscope}%
\definecolor{textcolor}{rgb}{0.000000,0.000000,0.000000}%
\pgfsetstrokecolor{textcolor}%
\pgfsetfillcolor{textcolor}%
\pgftext[x=5.583254in,y=2.297213in,,base]{\color{textcolor}\rmfamily\fontsize{12.000000}{14.400000}\selectfont \(\displaystyle P_x\)}%
\end{pgfscope}%
\begin{pgfscope}%
\pgfpathrectangle{\pgfqpoint{1.000000in}{0.330000in}}{\pgfqpoint{6.200000in}{0.604712in}}%
\pgfusepath{clip}%
\pgfsetbuttcap%
\pgfsetroundjoin%
\definecolor{currentfill}{rgb}{1.000000,0.000000,0.000000}%
\pgfsetfillcolor{currentfill}%
\pgfsetlinewidth{1.003750pt}%
\definecolor{currentstroke}{rgb}{1.000000,0.000000,0.000000}%
\pgfsetstrokecolor{currentstroke}%
\pgfsetdash{}{0pt}%
\pgfsys@defobject{currentmarker}{\pgfqpoint{-0.098209in}{-0.098209in}}{\pgfqpoint{0.098209in}{0.098209in}}{%
\pgfpathmoveto{\pgfqpoint{0.000000in}{0.098209in}}%
\pgfpathlineto{\pgfqpoint{-0.098209in}{-0.098209in}}%
\pgfpathlineto{\pgfqpoint{0.098209in}{-0.098209in}}%
\pgfpathclose%
\pgfusepath{stroke,fill}%
}%
\begin{pgfscope}%
\pgfsys@transformshift{1.281818in}{0.357487in}%
\pgfsys@useobject{currentmarker}{}%
\end{pgfscope}%
\begin{pgfscope}%
\pgfsys@transformshift{6.918182in}{0.357487in}%
\pgfsys@useobject{currentmarker}{}%
\end{pgfscope}%
\end{pgfscope}%
\begin{pgfscope}%
\pgfpathrectangle{\pgfqpoint{1.000000in}{0.330000in}}{\pgfqpoint{6.200000in}{0.604712in}}%
\pgfusepath{clip}%
\pgfsetbuttcap%
\pgfsetroundjoin%
\definecolor{currentfill}{rgb}{0.000000,0.000000,1.000000}%
\pgfsetfillcolor{currentfill}%
\pgfsetlinewidth{1.003750pt}%
\definecolor{currentstroke}{rgb}{0.000000,0.000000,1.000000}%
\pgfsetstrokecolor{currentstroke}%
\pgfsetdash{}{0pt}%
\pgfsys@defobject{currentmarker}{\pgfqpoint{-0.098209in}{-0.098209in}}{\pgfqpoint{0.098209in}{0.098209in}}{%
\pgfpathmoveto{\pgfqpoint{-0.098209in}{-0.098209in}}%
\pgfpathlineto{\pgfqpoint{0.098209in}{-0.098209in}}%
\pgfpathlineto{\pgfqpoint{0.098209in}{0.098209in}}%
\pgfpathlineto{\pgfqpoint{-0.098209in}{0.098209in}}%
\pgfpathclose%
\pgfusepath{stroke,fill}%
}%
\end{pgfscope}%
\begin{pgfscope}%
\pgfpathrectangle{\pgfqpoint{1.000000in}{0.330000in}}{\pgfqpoint{6.200000in}{0.604712in}}%
\pgfusepath{clip}%
\pgfsetrectcap%
\pgfsetroundjoin%
\pgfsetlinewidth{1.003750pt}%
\definecolor{currentstroke}{rgb}{0.000000,0.000000,0.000000}%
\pgfsetstrokecolor{currentstroke}%
\pgfsetdash{}{0pt}%
\pgfpathmoveto{\pgfqpoint{1.281818in}{0.907225in}}%
\pgfpathlineto{\pgfqpoint{6.918182in}{0.907225in}}%
\pgfusepath{stroke}%
\end{pgfscope}%
\begin{pgfscope}%
\pgfpathrectangle{\pgfqpoint{1.000000in}{0.330000in}}{\pgfqpoint{6.200000in}{0.604712in}}%
\pgfusepath{clip}%
\pgfsetrectcap%
\pgfsetroundjoin%
\pgfsetlinewidth{1.003750pt}%
\definecolor{currentstroke}{rgb}{0.000000,0.000000,0.000000}%
\pgfsetstrokecolor{currentstroke}%
\pgfsetdash{}{0pt}%
\pgfpathmoveto{\pgfqpoint{1.281818in}{0.869178in}}%
\pgfpathlineto{\pgfqpoint{6.918182in}{0.869178in}}%
\pgfusepath{stroke}%
\end{pgfscope}%
\begin{pgfscope}%
\pgfpathrectangle{\pgfqpoint{1.000000in}{0.330000in}}{\pgfqpoint{6.200000in}{0.604712in}}%
\pgfusepath{clip}%
\pgfsetrectcap%
\pgfsetroundjoin%
\pgfsetlinewidth{1.003750pt}%
\definecolor{currentstroke}{rgb}{0.000000,0.000000,0.000000}%
\pgfsetstrokecolor{currentstroke}%
\pgfsetdash{}{0pt}%
\pgfpathmoveto{\pgfqpoint{1.281818in}{0.450663in}}%
\pgfpathlineto{\pgfqpoint{6.918182in}{0.450663in}}%
\pgfusepath{stroke}%
\end{pgfscope}%
\begin{pgfscope}%
\pgfpathrectangle{\pgfqpoint{1.000000in}{0.330000in}}{\pgfqpoint{6.200000in}{0.604712in}}%
\pgfusepath{clip}%
\pgfsetrectcap%
\pgfsetroundjoin%
\pgfsetlinewidth{1.003750pt}%
\definecolor{currentstroke}{rgb}{0.000000,0.000000,0.000000}%
\pgfsetstrokecolor{currentstroke}%
\pgfsetdash{}{0pt}%
\pgfpathmoveto{\pgfqpoint{1.281818in}{0.488710in}}%
\pgfpathlineto{\pgfqpoint{6.918182in}{0.488710in}}%
\pgfusepath{stroke}%
\end{pgfscope}%
\begin{pgfscope}%
\pgfpathrectangle{\pgfqpoint{1.000000in}{0.330000in}}{\pgfqpoint{6.200000in}{0.604712in}}%
\pgfusepath{clip}%
\pgfsetrectcap%
\pgfsetroundjoin%
\pgfsetlinewidth{1.003750pt}%
\definecolor{currentstroke}{rgb}{0.000000,0.000000,0.000000}%
\pgfsetstrokecolor{currentstroke}%
\pgfsetdash{}{0pt}%
\pgfpathmoveto{\pgfqpoint{1.281818in}{0.450663in}}%
\pgfpathlineto{\pgfqpoint{1.281818in}{0.907225in}}%
\pgfusepath{stroke}%
\end{pgfscope}%
\begin{pgfscope}%
\pgfpathrectangle{\pgfqpoint{1.000000in}{0.330000in}}{\pgfqpoint{6.200000in}{0.604712in}}%
\pgfusepath{clip}%
\pgfsetrectcap%
\pgfsetroundjoin%
\pgfsetlinewidth{1.003750pt}%
\definecolor{currentstroke}{rgb}{0.000000,0.000000,0.000000}%
\pgfsetstrokecolor{currentstroke}%
\pgfsetdash{}{0pt}%
\pgfpathmoveto{\pgfqpoint{6.918182in}{0.450663in}}%
\pgfpathlineto{\pgfqpoint{6.918182in}{0.907225in}}%
\pgfusepath{stroke}%
\end{pgfscope}%
\begin{pgfscope}%
\pgfsetbuttcap%
\pgfsetmiterjoin%
\definecolor{currentfill}{rgb}{0.800000,0.800000,0.800000}%
\pgfsetfillcolor{currentfill}%
\pgfsetlinewidth{1.003750pt}%
\definecolor{currentstroke}{rgb}{0.000000,0.000000,0.000000}%
\pgfsetstrokecolor{currentstroke}%
\pgfsetdash{}{0pt}%
\pgfpathmoveto{\pgfqpoint{3.783641in}{0.582493in}}%
\pgfpathcurveto{\pgfqpoint{3.818363in}{0.547771in}}{\pgfqpoint{4.381637in}{0.547771in}}{\pgfqpoint{4.416359in}{0.582493in}}%
\pgfpathcurveto{\pgfqpoint{4.451081in}{0.617216in}}{\pgfqpoint{4.451081in}{0.740672in}}{\pgfqpoint{4.416359in}{0.775395in}}%
\pgfpathcurveto{\pgfqpoint{4.381637in}{0.810117in}}{\pgfqpoint{3.818363in}{0.810117in}}{\pgfqpoint{3.783641in}{0.775395in}}%
\pgfpathcurveto{\pgfqpoint{3.748919in}{0.740672in}}{\pgfqpoint{3.748919in}{0.617216in}}{\pgfqpoint{3.783641in}{0.582493in}}%
\pgfpathclose%
\pgfusepath{stroke,fill}%
\end{pgfscope}%
\begin{pgfscope}%
\definecolor{textcolor}{rgb}{0.000000,0.000000,0.000000}%
\pgfsetstrokecolor{textcolor}%
\pgfsetfillcolor{textcolor}%
\pgftext[x=4.100000in,y=0.678944in,,]{\color{textcolor}\rmfamily\fontsize{10.000000}{12.000000}\selectfont W14x132}%
\end{pgfscope}%
\begin{pgfscope}%
\pgfsetbuttcap%
\pgfsetmiterjoin%
\definecolor{currentfill}{rgb}{0.800000,0.800000,0.800000}%
\pgfsetfillcolor{currentfill}%
\pgfsetlinewidth{1.003750pt}%
\definecolor{currentstroke}{rgb}{0.000000,0.000000,0.000000}%
\pgfsetstrokecolor{currentstroke}%
\pgfsetdash{}{0pt}%
\pgfpathmoveto{\pgfqpoint{3.615817in}{0.198919in}}%
\pgfpathcurveto{\pgfqpoint{3.650539in}{0.164197in}}{\pgfqpoint{4.549461in}{0.164197in}}{\pgfqpoint{4.584183in}{0.198919in}}%
\pgfpathcurveto{\pgfqpoint{4.618905in}{0.233641in}}{\pgfqpoint{4.618905in}{0.357098in}}{\pgfqpoint{4.584183in}{0.391820in}}%
\pgfpathcurveto{\pgfqpoint{4.549461in}{0.426542in}}{\pgfqpoint{3.650539in}{0.426542in}}{\pgfqpoint{3.615817in}{0.391820in}}%
\pgfpathcurveto{\pgfqpoint{3.581095in}{0.357098in}}{\pgfqpoint{3.581095in}{0.233641in}}{\pgfqpoint{3.615817in}{0.198919in}}%
\pgfpathclose%
\pgfusepath{stroke,fill}%
\end{pgfscope}%
\begin{pgfscope}%
\definecolor{textcolor}{rgb}{0.000000,0.000000,0.000000}%
\pgfsetstrokecolor{textcolor}%
\pgfsetfillcolor{textcolor}%
\pgftext[x=4.100000in,y=0.295370in,,]{\color{textcolor}\rmfamily\fontsize{10.000000}{12.000000}\selectfont Span 0 = 19 ft}%
\end{pgfscope}%
\end{pgfpicture}%
\makeatother%
\endgroup%

\end{center}
\vspace{-18pt}
\caption{Applied Loads}
\end{figure}
The following distributed loads are applied to the beam. The program can handle all possible mass and force units in both metric and imperial systems simultaneously. Loads are plotted to scale according to their relative magnitudes. A "positive" load is defined as a load acting in the direction of gravity.
\begin{table}[ht]
\caption{Applied Distributed Loads}
\centering
\begin{tabular}{l l l l l l l}
\hline
Load & Start Loc. & Start Mag. & End Loc. & End Mag. & Type & Description\\
\hline
w\textsubscript{1} & 0 {\color{darkBlue}{\textbf{ft}}} & 15.41 {\color{darkBlue}{\textbf{ft}}} * psf & 19 {\color{darkBlue}{\textbf{ft}}} & 15.41 {\color{darkBlue}{\textbf{ft}}} * psf & D & typ floor\\
w\textsubscript{2} & 0 {\color{darkBlue}{\textbf{ft}}} & 26.8 {\color{darkBlue}{\textbf{ft}}} * psf & 19 {\color{darkBlue}{\textbf{ft}}} & 26.8 {\color{darkBlue}{\textbf{ft}}} * psf & L & typ floor\\
w\textsubscript{3} & 0 {\color{darkBlue}{\textbf{ft}}} & 132.0 {\color{darkBlue}{\textbf{plf}}} & 19 {\color{darkBlue}{\textbf{ft}}} & 132.0 {\color{darkBlue}{\textbf{plf}}} & D & Self weight\\
\hline
\end{tabular}
\end{table}
\begin{table}[ht]
\caption{Applied Point Loads}
\centering
\begin{tabular}{l l l l l l}
\hline
Load & Loc. & Shear & Type & Description \\
\hline
P\textsubscript{1} & 9 {\color{darkBlue}{\textbf{ft}}} & 17.8 {\color{darkBlue}{\textbf{kip}}} & D & 2B-4 rxn\\
P\textsubscript{2} & 9 {\color{darkBlue}{\textbf{ft}}} & 1.7 {\color{darkBlue}{\textbf{kip}}} & Lr & 2B-4 rxn\\
P\textsubscript{3} & 9 {\color{darkBlue}{\textbf{ft}}} & 21.5 {\color{darkBlue}{\textbf{kip}}} & L & 2B-4 rxn\\
P\textsubscript{4} & 9 {\color{darkBlue}{\textbf{ft}}} & -4.6 {\color{darkBlue}{\textbf{kip}}} & E & 2B-4 rxn\\
P\textsubscript{5} & 14.5 {\color{darkBlue}{\textbf{ft}}} & 13.5 {\color{darkBlue}{\textbf{kip}}} & E & shear wall\\
\hline
\end{tabular}
\end{table}
%	-------------------------------- LOAD COMBOS	--------------------------------
\section{Load Combinations}
The following load combinations are used for the design. Duplicate load combinations are not listed and only loads that are used on the beam are included in the load combinations (i.e. If soil load is not included as a load type in any of the applied loads, then "H" loads will not be included in the listed load combinations). S\textsubscript{DS} is input as 1.0 and \textOmega\textsubscript{0} is input as 2.5 for use in seismic load combinations. Any load designated as a pattern load is applied to spans in all possible permutations to create the most extreme loading condition. Numbers after a load indicate the span over which the pattern load is applied (i.e. L0 indicates that live load is applied only on the first span).
\begin{table}[H]
\caption{Strength (LRFD) Load Combinations}
\centering
\begin{tabular}{l l l}
\hline
Load Combo & Loads and Factors & Reference\\
\hline
LC 1 & 0.9D & ASCE7-16 \S2.3.1 (LC 5)\\
LC 2 & 1.2D + 0.5L\textsubscriptr0 + 1.0L0 & ASCE7-16 \S2.3.1 (LC 4)\\
LC 3 & 1.4D + -2.5E + 0.5L0 & ASCE7-16 \S2.3.6 (LC 6)\\
LC 4 & 1.2D + 1.0L0 & ASCE7-16 \S2.3.1 (LC 3)\\
LC 5 & 1.4D & ASCE7-16 \S2.3.1 (LC 1)\\
LC 6 & 0.7D + 2.5E & ASCE7-16 \S2.3.6 (LC 7)\\
LC 7 & 1.2D & ASCE7-16 \S2.3.1 (LC 3)\\
LC 8 & 1.2D + 1.6L\textsubscriptr0 + 1.0L0 & ASCE7-16 \S2.3.1 (LC 3)\\
LC 9 & 1.2D + 0.5L\textsubscriptr0 + 1.6L0 & ASCE7-16 \S2.3.1 (LC 2)\\
LC 10 & 1.2D + 1.6L\textsubscriptr0 & ASCE7-16 \S2.3.1 (LC 3)\\
LC 11 & 1.4D + 2.5E + 0.5L0 & ASCE7-16 \S2.3.6 (LC 6)\\
LC 12 & 1.2D + 1.6L0 & ASCE7-16 \S2.3.1 (LC 2)\\
LC 13 & 0.7D + -2.5E & ASCE7-16 \S2.3.6 (LC 7)\\
\hline
\end{tabular}
\end{table}
\begin{table}[H]
\caption{Deflection (ASD) Load Combinations}
\centering
\begin{tabular}{l l l}
\hline
Load Combo & Loads and Factors & Reference\\
\hline
LC 1 & 1.0D + 0.75L\textsubscriptr0 + 0.75L0 & ASCE7-16 \S2.4.1 (LC 4)\\
LC 2 & 1.0D + 1.0L\textsubscriptr0 & ASCE7-16 \S2.4.1 (LC 3)\\
LC 3 & 1.0L0 & L only deflection check\\
LC 4 & 1.0D + 1.0L0 & ASCE7-16 \S2.4.1 (LC 2)\\
LC 5 & 1.1D + 1.31E + 0.75L0 & ASCE7-16 \S2.4.5 (LC 8)\\
LC 6 & 0.6D & ASCE7-16 \S2.4.1 (LC 7)\\
LC 7 & 1.0D + 0.75L0 & ASCE7-16 \S2.4.1 (LC 4)\\
LC 8 & 1.14D + 1.75E & ASCE7-16 \S2.4.5 (LC 8)\\
LC 9 & 1.0D & ASCE7-16 \S2.4.1 (LC 1)\\
LC 10 & 1.14D + -1.75E & ASCE7-16 \S2.4.5 (LC 8)\\
LC 11 & 0.4D + 1.75E & ASCE7-16 \S2.4.5 (LC 9)\\
LC 12 & 0.4D + -1.75E & ASCE7-16 \S2.4.5 (LC 9)\\
LC 13 & 1.1D + -1.31E + 0.75L0 & ASCE7-16 \S2.4.5 (LC 8)\\
\hline
\end{tabular}
\end{table}
%	---------------------- SECTIONAL & MATERIAL PROPERTIES -----------------------
\section{Sectional and Material Properties}
The following are sectional and material properties used for analysis \textbf{(W14x132, Grade A992)}:
\begin{table}[ht]
\caption{Sectional and Material Properties}
\vspace{-10pt}
\centering
\begin{tabular}{lll}
\centering
\begin{tabular}[t]{ll}
\cline{1-2}
Property & Value \\
\cline{1-2}
A\textsubscript{w} & 9.5 {\color{darkBlue}{\textbf{{\color{darkBlue}{\textbf{in}}}\textsuperscript{2}}}} \\
C\textsubscript{w} & 25500 {\color{darkBlue}{\textbf{{\color{darkBlue}{\textbf{in}}}\textsuperscript{6}}}} \\
F\textsubscript{u} & 65 {\color{darkBlue}{\textbf{ksi}}} \\
F\textsubscript{y} & 50 {\color{darkBlue}{\textbf{ksi}}} \\
I\textsubscript{x} & 1530 {\color{darkBlue}{\textbf{{\color{darkBlue}{\textbf{in}}}\textsuperscript{4}}}} \\
I\textsubscript{y} & 548 {\color{darkBlue}{\textbf{{\color{darkBlue}{\textbf{in}}}\textsuperscript{4}}}} \\
\cline{1-2}
\end{tabular}
&
\begin{tabular}[t]{ll}
\cline{1-2}
Property & Value \\
\cline{1-2}
S\textsubscript{x} & 209 {\color{darkBlue}{\textbf{{\color{darkBlue}{\textbf{in}}}\textsuperscript{3}}}} \\
S\textsubscript{y} & 74.5 {\color{darkBlue}{\textbf{{\color{darkBlue}{\textbf{in}}}\textsuperscript{3}}}} \\
Z\textsubscript{x} & 234 {\color{darkBlue}{\textbf{{\color{darkBlue}{\textbf{in}}}\textsuperscript{3}}}} \\
Z\textsubscript{y} & 113 {\color{darkBlue}{\textbf{{\color{darkBlue}{\textbf{in}}}\textsuperscript{3}}}} \\
r\textsubscript{x} & 6.3 {\color{darkBlue}{\textbf{in}}} \\
r\textsubscript{y} & 3.8 {\color{darkBlue}{\textbf{in}}} \\
\cline{1-2}
\end{tabular}
&
\begin{tabular}[t]{ll}
\cline{1-2}
Property & Value \\
\cline{1-2}
b\textsubscript{f} & 14.7 {\color{darkBlue}{\textbf{in}}} \\
t\textsubscript{f} & 1.0 {\color{darkBlue}{\textbf{in}}} \\
t\textsubscript{w} & 0.6 {\color{darkBlue}{\textbf{in}}} \\
h\textsubscript{0} & 13.7 {\color{darkBlue}{\textbf{in}}} \\
U.W. & 490 {\color{darkBlue}{\textbf{pcf}}} \\
\cline{1-2}
\end{tabular}
\end{tabular}
\end{table}
%	-------------------------------- BENDING CHECK -------------------------------
\section{Bending Check}
\begin{figure}[H]
\begin{center}
%% Creator: Matplotlib, PGF backend
%%
%% To include the figure in your LaTeX document, write
%%   \input{<filename>.pgf}
%%
%% Make sure the required packages are loaded in your preamble
%%   \usepackage{pgf}
%%
%% Figures using additional raster images can only be included by \input if
%% they are in the same directory as the main LaTeX file. For loading figures
%% from other directories you can use the `import` package
%%   \usepackage{import}
%%
%% and then include the figures with
%%   \import{<path to file>}{<filename>.pgf}
%%
%% Matplotlib used the following preamble
%%
\begingroup%
\makeatletter%
\begin{pgfpicture}%
\pgfpathrectangle{\pgfpointorigin}{\pgfqpoint{8.000000in}{3.000000in}}%
\pgfusepath{use as bounding box, clip}%
\begin{pgfscope}%
\pgfsetbuttcap%
\pgfsetmiterjoin%
\definecolor{currentfill}{rgb}{1.000000,1.000000,1.000000}%
\pgfsetfillcolor{currentfill}%
\pgfsetlinewidth{0.000000pt}%
\definecolor{currentstroke}{rgb}{1.000000,1.000000,1.000000}%
\pgfsetstrokecolor{currentstroke}%
\pgfsetdash{}{0pt}%
\pgfpathmoveto{\pgfqpoint{0.000000in}{0.000000in}}%
\pgfpathlineto{\pgfqpoint{8.000000in}{0.000000in}}%
\pgfpathlineto{\pgfqpoint{8.000000in}{3.000000in}}%
\pgfpathlineto{\pgfqpoint{0.000000in}{3.000000in}}%
\pgfpathclose%
\pgfusepath{fill}%
\end{pgfscope}%
\begin{pgfscope}%
\pgfsetbuttcap%
\pgfsetmiterjoin%
\definecolor{currentfill}{rgb}{1.000000,1.000000,1.000000}%
\pgfsetfillcolor{currentfill}%
\pgfsetlinewidth{0.000000pt}%
\definecolor{currentstroke}{rgb}{0.000000,0.000000,0.000000}%
\pgfsetstrokecolor{currentstroke}%
\pgfsetstrokeopacity{0.000000}%
\pgfsetdash{}{0pt}%
\pgfpathmoveto{\pgfqpoint{1.000000in}{0.330000in}}%
\pgfpathlineto{\pgfqpoint{7.200000in}{0.330000in}}%
\pgfpathlineto{\pgfqpoint{7.200000in}{2.640000in}}%
\pgfpathlineto{\pgfqpoint{1.000000in}{2.640000in}}%
\pgfpathclose%
\pgfusepath{fill}%
\end{pgfscope}%
\begin{pgfscope}%
\pgfpathrectangle{\pgfqpoint{1.000000in}{0.330000in}}{\pgfqpoint{6.200000in}{2.310000in}}%
\pgfusepath{clip}%
\pgfsetbuttcap%
\pgfsetroundjoin%
\pgfsetlinewidth{0.803000pt}%
\definecolor{currentstroke}{rgb}{0.000000,0.000000,0.000000}%
\pgfsetstrokecolor{currentstroke}%
\pgfsetdash{{0.800000pt}{1.320000pt}}{0.000000pt}%
\pgfpathmoveto{\pgfqpoint{1.281818in}{0.330000in}}%
\pgfpathlineto{\pgfqpoint{1.281818in}{2.640000in}}%
\pgfusepath{stroke}%
\end{pgfscope}%
\begin{pgfscope}%
\pgfsetbuttcap%
\pgfsetroundjoin%
\definecolor{currentfill}{rgb}{0.000000,0.000000,0.000000}%
\pgfsetfillcolor{currentfill}%
\pgfsetlinewidth{0.803000pt}%
\definecolor{currentstroke}{rgb}{0.000000,0.000000,0.000000}%
\pgfsetstrokecolor{currentstroke}%
\pgfsetdash{}{0pt}%
\pgfsys@defobject{currentmarker}{\pgfqpoint{0.000000in}{-0.048611in}}{\pgfqpoint{0.000000in}{0.000000in}}{%
\pgfpathmoveto{\pgfqpoint{0.000000in}{0.000000in}}%
\pgfpathlineto{\pgfqpoint{0.000000in}{-0.048611in}}%
\pgfusepath{stroke,fill}%
}%
\begin{pgfscope}%
\pgfsys@transformshift{1.281818in}{0.330000in}%
\pgfsys@useobject{currentmarker}{}%
\end{pgfscope}%
\end{pgfscope}%
\begin{pgfscope}%
\pgfsetbuttcap%
\pgfsetroundjoin%
\definecolor{currentfill}{rgb}{0.000000,0.000000,0.000000}%
\pgfsetfillcolor{currentfill}%
\pgfsetlinewidth{0.803000pt}%
\definecolor{currentstroke}{rgb}{0.000000,0.000000,0.000000}%
\pgfsetstrokecolor{currentstroke}%
\pgfsetdash{}{0pt}%
\pgfsys@defobject{currentmarker}{\pgfqpoint{0.000000in}{0.000000in}}{\pgfqpoint{0.000000in}{0.048611in}}{%
\pgfpathmoveto{\pgfqpoint{0.000000in}{0.000000in}}%
\pgfpathlineto{\pgfqpoint{0.000000in}{0.048611in}}%
\pgfusepath{stroke,fill}%
}%
\begin{pgfscope}%
\pgfsys@transformshift{1.281818in}{2.640000in}%
\pgfsys@useobject{currentmarker}{}%
\end{pgfscope}%
\end{pgfscope}%
\begin{pgfscope}%
\definecolor{textcolor}{rgb}{0.000000,0.000000,0.000000}%
\pgfsetstrokecolor{textcolor}%
\pgfsetfillcolor{textcolor}%
\pgftext[x=1.281818in,y=0.232778in,,top]{\color{textcolor}\rmfamily\fontsize{10.000000}{12.000000}\selectfont \(\displaystyle {0.0}\)}%
\end{pgfscope}%
\begin{pgfscope}%
\pgfpathrectangle{\pgfqpoint{1.000000in}{0.330000in}}{\pgfqpoint{6.200000in}{2.310000in}}%
\pgfusepath{clip}%
\pgfsetbuttcap%
\pgfsetroundjoin%
\pgfsetlinewidth{0.803000pt}%
\definecolor{currentstroke}{rgb}{0.000000,0.000000,0.000000}%
\pgfsetstrokecolor{currentstroke}%
\pgfsetdash{{0.800000pt}{1.320000pt}}{0.000000pt}%
\pgfpathmoveto{\pgfqpoint{2.023445in}{0.330000in}}%
\pgfpathlineto{\pgfqpoint{2.023445in}{2.640000in}}%
\pgfusepath{stroke}%
\end{pgfscope}%
\begin{pgfscope}%
\pgfsetbuttcap%
\pgfsetroundjoin%
\definecolor{currentfill}{rgb}{0.000000,0.000000,0.000000}%
\pgfsetfillcolor{currentfill}%
\pgfsetlinewidth{0.803000pt}%
\definecolor{currentstroke}{rgb}{0.000000,0.000000,0.000000}%
\pgfsetstrokecolor{currentstroke}%
\pgfsetdash{}{0pt}%
\pgfsys@defobject{currentmarker}{\pgfqpoint{0.000000in}{-0.048611in}}{\pgfqpoint{0.000000in}{0.000000in}}{%
\pgfpathmoveto{\pgfqpoint{0.000000in}{0.000000in}}%
\pgfpathlineto{\pgfqpoint{0.000000in}{-0.048611in}}%
\pgfusepath{stroke,fill}%
}%
\begin{pgfscope}%
\pgfsys@transformshift{2.023445in}{0.330000in}%
\pgfsys@useobject{currentmarker}{}%
\end{pgfscope}%
\end{pgfscope}%
\begin{pgfscope}%
\pgfsetbuttcap%
\pgfsetroundjoin%
\definecolor{currentfill}{rgb}{0.000000,0.000000,0.000000}%
\pgfsetfillcolor{currentfill}%
\pgfsetlinewidth{0.803000pt}%
\definecolor{currentstroke}{rgb}{0.000000,0.000000,0.000000}%
\pgfsetstrokecolor{currentstroke}%
\pgfsetdash{}{0pt}%
\pgfsys@defobject{currentmarker}{\pgfqpoint{0.000000in}{0.000000in}}{\pgfqpoint{0.000000in}{0.048611in}}{%
\pgfpathmoveto{\pgfqpoint{0.000000in}{0.000000in}}%
\pgfpathlineto{\pgfqpoint{0.000000in}{0.048611in}}%
\pgfusepath{stroke,fill}%
}%
\begin{pgfscope}%
\pgfsys@transformshift{2.023445in}{2.640000in}%
\pgfsys@useobject{currentmarker}{}%
\end{pgfscope}%
\end{pgfscope}%
\begin{pgfscope}%
\definecolor{textcolor}{rgb}{0.000000,0.000000,0.000000}%
\pgfsetstrokecolor{textcolor}%
\pgfsetfillcolor{textcolor}%
\pgftext[x=2.023445in,y=0.232778in,,top]{\color{textcolor}\rmfamily\fontsize{10.000000}{12.000000}\selectfont \(\displaystyle {2.5}\)}%
\end{pgfscope}%
\begin{pgfscope}%
\pgfpathrectangle{\pgfqpoint{1.000000in}{0.330000in}}{\pgfqpoint{6.200000in}{2.310000in}}%
\pgfusepath{clip}%
\pgfsetbuttcap%
\pgfsetroundjoin%
\pgfsetlinewidth{0.803000pt}%
\definecolor{currentstroke}{rgb}{0.000000,0.000000,0.000000}%
\pgfsetstrokecolor{currentstroke}%
\pgfsetdash{{0.800000pt}{1.320000pt}}{0.000000pt}%
\pgfpathmoveto{\pgfqpoint{2.765072in}{0.330000in}}%
\pgfpathlineto{\pgfqpoint{2.765072in}{2.640000in}}%
\pgfusepath{stroke}%
\end{pgfscope}%
\begin{pgfscope}%
\pgfsetbuttcap%
\pgfsetroundjoin%
\definecolor{currentfill}{rgb}{0.000000,0.000000,0.000000}%
\pgfsetfillcolor{currentfill}%
\pgfsetlinewidth{0.803000pt}%
\definecolor{currentstroke}{rgb}{0.000000,0.000000,0.000000}%
\pgfsetstrokecolor{currentstroke}%
\pgfsetdash{}{0pt}%
\pgfsys@defobject{currentmarker}{\pgfqpoint{0.000000in}{-0.048611in}}{\pgfqpoint{0.000000in}{0.000000in}}{%
\pgfpathmoveto{\pgfqpoint{0.000000in}{0.000000in}}%
\pgfpathlineto{\pgfqpoint{0.000000in}{-0.048611in}}%
\pgfusepath{stroke,fill}%
}%
\begin{pgfscope}%
\pgfsys@transformshift{2.765072in}{0.330000in}%
\pgfsys@useobject{currentmarker}{}%
\end{pgfscope}%
\end{pgfscope}%
\begin{pgfscope}%
\pgfsetbuttcap%
\pgfsetroundjoin%
\definecolor{currentfill}{rgb}{0.000000,0.000000,0.000000}%
\pgfsetfillcolor{currentfill}%
\pgfsetlinewidth{0.803000pt}%
\definecolor{currentstroke}{rgb}{0.000000,0.000000,0.000000}%
\pgfsetstrokecolor{currentstroke}%
\pgfsetdash{}{0pt}%
\pgfsys@defobject{currentmarker}{\pgfqpoint{0.000000in}{0.000000in}}{\pgfqpoint{0.000000in}{0.048611in}}{%
\pgfpathmoveto{\pgfqpoint{0.000000in}{0.000000in}}%
\pgfpathlineto{\pgfqpoint{0.000000in}{0.048611in}}%
\pgfusepath{stroke,fill}%
}%
\begin{pgfscope}%
\pgfsys@transformshift{2.765072in}{2.640000in}%
\pgfsys@useobject{currentmarker}{}%
\end{pgfscope}%
\end{pgfscope}%
\begin{pgfscope}%
\definecolor{textcolor}{rgb}{0.000000,0.000000,0.000000}%
\pgfsetstrokecolor{textcolor}%
\pgfsetfillcolor{textcolor}%
\pgftext[x=2.765072in,y=0.232778in,,top]{\color{textcolor}\rmfamily\fontsize{10.000000}{12.000000}\selectfont \(\displaystyle {5.0}\)}%
\end{pgfscope}%
\begin{pgfscope}%
\pgfpathrectangle{\pgfqpoint{1.000000in}{0.330000in}}{\pgfqpoint{6.200000in}{2.310000in}}%
\pgfusepath{clip}%
\pgfsetbuttcap%
\pgfsetroundjoin%
\pgfsetlinewidth{0.803000pt}%
\definecolor{currentstroke}{rgb}{0.000000,0.000000,0.000000}%
\pgfsetstrokecolor{currentstroke}%
\pgfsetdash{{0.800000pt}{1.320000pt}}{0.000000pt}%
\pgfpathmoveto{\pgfqpoint{3.506699in}{0.330000in}}%
\pgfpathlineto{\pgfqpoint{3.506699in}{2.640000in}}%
\pgfusepath{stroke}%
\end{pgfscope}%
\begin{pgfscope}%
\pgfsetbuttcap%
\pgfsetroundjoin%
\definecolor{currentfill}{rgb}{0.000000,0.000000,0.000000}%
\pgfsetfillcolor{currentfill}%
\pgfsetlinewidth{0.803000pt}%
\definecolor{currentstroke}{rgb}{0.000000,0.000000,0.000000}%
\pgfsetstrokecolor{currentstroke}%
\pgfsetdash{}{0pt}%
\pgfsys@defobject{currentmarker}{\pgfqpoint{0.000000in}{-0.048611in}}{\pgfqpoint{0.000000in}{0.000000in}}{%
\pgfpathmoveto{\pgfqpoint{0.000000in}{0.000000in}}%
\pgfpathlineto{\pgfqpoint{0.000000in}{-0.048611in}}%
\pgfusepath{stroke,fill}%
}%
\begin{pgfscope}%
\pgfsys@transformshift{3.506699in}{0.330000in}%
\pgfsys@useobject{currentmarker}{}%
\end{pgfscope}%
\end{pgfscope}%
\begin{pgfscope}%
\pgfsetbuttcap%
\pgfsetroundjoin%
\definecolor{currentfill}{rgb}{0.000000,0.000000,0.000000}%
\pgfsetfillcolor{currentfill}%
\pgfsetlinewidth{0.803000pt}%
\definecolor{currentstroke}{rgb}{0.000000,0.000000,0.000000}%
\pgfsetstrokecolor{currentstroke}%
\pgfsetdash{}{0pt}%
\pgfsys@defobject{currentmarker}{\pgfqpoint{0.000000in}{0.000000in}}{\pgfqpoint{0.000000in}{0.048611in}}{%
\pgfpathmoveto{\pgfqpoint{0.000000in}{0.000000in}}%
\pgfpathlineto{\pgfqpoint{0.000000in}{0.048611in}}%
\pgfusepath{stroke,fill}%
}%
\begin{pgfscope}%
\pgfsys@transformshift{3.506699in}{2.640000in}%
\pgfsys@useobject{currentmarker}{}%
\end{pgfscope}%
\end{pgfscope}%
\begin{pgfscope}%
\definecolor{textcolor}{rgb}{0.000000,0.000000,0.000000}%
\pgfsetstrokecolor{textcolor}%
\pgfsetfillcolor{textcolor}%
\pgftext[x=3.506699in,y=0.232778in,,top]{\color{textcolor}\rmfamily\fontsize{10.000000}{12.000000}\selectfont \(\displaystyle {7.5}\)}%
\end{pgfscope}%
\begin{pgfscope}%
\pgfpathrectangle{\pgfqpoint{1.000000in}{0.330000in}}{\pgfqpoint{6.200000in}{2.310000in}}%
\pgfusepath{clip}%
\pgfsetbuttcap%
\pgfsetroundjoin%
\pgfsetlinewidth{0.803000pt}%
\definecolor{currentstroke}{rgb}{0.000000,0.000000,0.000000}%
\pgfsetstrokecolor{currentstroke}%
\pgfsetdash{{0.800000pt}{1.320000pt}}{0.000000pt}%
\pgfpathmoveto{\pgfqpoint{4.248325in}{0.330000in}}%
\pgfpathlineto{\pgfqpoint{4.248325in}{2.640000in}}%
\pgfusepath{stroke}%
\end{pgfscope}%
\begin{pgfscope}%
\pgfsetbuttcap%
\pgfsetroundjoin%
\definecolor{currentfill}{rgb}{0.000000,0.000000,0.000000}%
\pgfsetfillcolor{currentfill}%
\pgfsetlinewidth{0.803000pt}%
\definecolor{currentstroke}{rgb}{0.000000,0.000000,0.000000}%
\pgfsetstrokecolor{currentstroke}%
\pgfsetdash{}{0pt}%
\pgfsys@defobject{currentmarker}{\pgfqpoint{0.000000in}{-0.048611in}}{\pgfqpoint{0.000000in}{0.000000in}}{%
\pgfpathmoveto{\pgfqpoint{0.000000in}{0.000000in}}%
\pgfpathlineto{\pgfqpoint{0.000000in}{-0.048611in}}%
\pgfusepath{stroke,fill}%
}%
\begin{pgfscope}%
\pgfsys@transformshift{4.248325in}{0.330000in}%
\pgfsys@useobject{currentmarker}{}%
\end{pgfscope}%
\end{pgfscope}%
\begin{pgfscope}%
\pgfsetbuttcap%
\pgfsetroundjoin%
\definecolor{currentfill}{rgb}{0.000000,0.000000,0.000000}%
\pgfsetfillcolor{currentfill}%
\pgfsetlinewidth{0.803000pt}%
\definecolor{currentstroke}{rgb}{0.000000,0.000000,0.000000}%
\pgfsetstrokecolor{currentstroke}%
\pgfsetdash{}{0pt}%
\pgfsys@defobject{currentmarker}{\pgfqpoint{0.000000in}{0.000000in}}{\pgfqpoint{0.000000in}{0.048611in}}{%
\pgfpathmoveto{\pgfqpoint{0.000000in}{0.000000in}}%
\pgfpathlineto{\pgfqpoint{0.000000in}{0.048611in}}%
\pgfusepath{stroke,fill}%
}%
\begin{pgfscope}%
\pgfsys@transformshift{4.248325in}{2.640000in}%
\pgfsys@useobject{currentmarker}{}%
\end{pgfscope}%
\end{pgfscope}%
\begin{pgfscope}%
\definecolor{textcolor}{rgb}{0.000000,0.000000,0.000000}%
\pgfsetstrokecolor{textcolor}%
\pgfsetfillcolor{textcolor}%
\pgftext[x=4.248325in,y=0.232778in,,top]{\color{textcolor}\rmfamily\fontsize{10.000000}{12.000000}\selectfont \(\displaystyle {10.0}\)}%
\end{pgfscope}%
\begin{pgfscope}%
\pgfpathrectangle{\pgfqpoint{1.000000in}{0.330000in}}{\pgfqpoint{6.200000in}{2.310000in}}%
\pgfusepath{clip}%
\pgfsetbuttcap%
\pgfsetroundjoin%
\pgfsetlinewidth{0.803000pt}%
\definecolor{currentstroke}{rgb}{0.000000,0.000000,0.000000}%
\pgfsetstrokecolor{currentstroke}%
\pgfsetdash{{0.800000pt}{1.320000pt}}{0.000000pt}%
\pgfpathmoveto{\pgfqpoint{4.989952in}{0.330000in}}%
\pgfpathlineto{\pgfqpoint{4.989952in}{2.640000in}}%
\pgfusepath{stroke}%
\end{pgfscope}%
\begin{pgfscope}%
\pgfsetbuttcap%
\pgfsetroundjoin%
\definecolor{currentfill}{rgb}{0.000000,0.000000,0.000000}%
\pgfsetfillcolor{currentfill}%
\pgfsetlinewidth{0.803000pt}%
\definecolor{currentstroke}{rgb}{0.000000,0.000000,0.000000}%
\pgfsetstrokecolor{currentstroke}%
\pgfsetdash{}{0pt}%
\pgfsys@defobject{currentmarker}{\pgfqpoint{0.000000in}{-0.048611in}}{\pgfqpoint{0.000000in}{0.000000in}}{%
\pgfpathmoveto{\pgfqpoint{0.000000in}{0.000000in}}%
\pgfpathlineto{\pgfqpoint{0.000000in}{-0.048611in}}%
\pgfusepath{stroke,fill}%
}%
\begin{pgfscope}%
\pgfsys@transformshift{4.989952in}{0.330000in}%
\pgfsys@useobject{currentmarker}{}%
\end{pgfscope}%
\end{pgfscope}%
\begin{pgfscope}%
\pgfsetbuttcap%
\pgfsetroundjoin%
\definecolor{currentfill}{rgb}{0.000000,0.000000,0.000000}%
\pgfsetfillcolor{currentfill}%
\pgfsetlinewidth{0.803000pt}%
\definecolor{currentstroke}{rgb}{0.000000,0.000000,0.000000}%
\pgfsetstrokecolor{currentstroke}%
\pgfsetdash{}{0pt}%
\pgfsys@defobject{currentmarker}{\pgfqpoint{0.000000in}{0.000000in}}{\pgfqpoint{0.000000in}{0.048611in}}{%
\pgfpathmoveto{\pgfqpoint{0.000000in}{0.000000in}}%
\pgfpathlineto{\pgfqpoint{0.000000in}{0.048611in}}%
\pgfusepath{stroke,fill}%
}%
\begin{pgfscope}%
\pgfsys@transformshift{4.989952in}{2.640000in}%
\pgfsys@useobject{currentmarker}{}%
\end{pgfscope}%
\end{pgfscope}%
\begin{pgfscope}%
\definecolor{textcolor}{rgb}{0.000000,0.000000,0.000000}%
\pgfsetstrokecolor{textcolor}%
\pgfsetfillcolor{textcolor}%
\pgftext[x=4.989952in,y=0.232778in,,top]{\color{textcolor}\rmfamily\fontsize{10.000000}{12.000000}\selectfont \(\displaystyle {12.5}\)}%
\end{pgfscope}%
\begin{pgfscope}%
\pgfpathrectangle{\pgfqpoint{1.000000in}{0.330000in}}{\pgfqpoint{6.200000in}{2.310000in}}%
\pgfusepath{clip}%
\pgfsetbuttcap%
\pgfsetroundjoin%
\pgfsetlinewidth{0.803000pt}%
\definecolor{currentstroke}{rgb}{0.000000,0.000000,0.000000}%
\pgfsetstrokecolor{currentstroke}%
\pgfsetdash{{0.800000pt}{1.320000pt}}{0.000000pt}%
\pgfpathmoveto{\pgfqpoint{5.731579in}{0.330000in}}%
\pgfpathlineto{\pgfqpoint{5.731579in}{2.640000in}}%
\pgfusepath{stroke}%
\end{pgfscope}%
\begin{pgfscope}%
\pgfsetbuttcap%
\pgfsetroundjoin%
\definecolor{currentfill}{rgb}{0.000000,0.000000,0.000000}%
\pgfsetfillcolor{currentfill}%
\pgfsetlinewidth{0.803000pt}%
\definecolor{currentstroke}{rgb}{0.000000,0.000000,0.000000}%
\pgfsetstrokecolor{currentstroke}%
\pgfsetdash{}{0pt}%
\pgfsys@defobject{currentmarker}{\pgfqpoint{0.000000in}{-0.048611in}}{\pgfqpoint{0.000000in}{0.000000in}}{%
\pgfpathmoveto{\pgfqpoint{0.000000in}{0.000000in}}%
\pgfpathlineto{\pgfqpoint{0.000000in}{-0.048611in}}%
\pgfusepath{stroke,fill}%
}%
\begin{pgfscope}%
\pgfsys@transformshift{5.731579in}{0.330000in}%
\pgfsys@useobject{currentmarker}{}%
\end{pgfscope}%
\end{pgfscope}%
\begin{pgfscope}%
\pgfsetbuttcap%
\pgfsetroundjoin%
\definecolor{currentfill}{rgb}{0.000000,0.000000,0.000000}%
\pgfsetfillcolor{currentfill}%
\pgfsetlinewidth{0.803000pt}%
\definecolor{currentstroke}{rgb}{0.000000,0.000000,0.000000}%
\pgfsetstrokecolor{currentstroke}%
\pgfsetdash{}{0pt}%
\pgfsys@defobject{currentmarker}{\pgfqpoint{0.000000in}{0.000000in}}{\pgfqpoint{0.000000in}{0.048611in}}{%
\pgfpathmoveto{\pgfqpoint{0.000000in}{0.000000in}}%
\pgfpathlineto{\pgfqpoint{0.000000in}{0.048611in}}%
\pgfusepath{stroke,fill}%
}%
\begin{pgfscope}%
\pgfsys@transformshift{5.731579in}{2.640000in}%
\pgfsys@useobject{currentmarker}{}%
\end{pgfscope}%
\end{pgfscope}%
\begin{pgfscope}%
\definecolor{textcolor}{rgb}{0.000000,0.000000,0.000000}%
\pgfsetstrokecolor{textcolor}%
\pgfsetfillcolor{textcolor}%
\pgftext[x=5.731579in,y=0.232778in,,top]{\color{textcolor}\rmfamily\fontsize{10.000000}{12.000000}\selectfont \(\displaystyle {15.0}\)}%
\end{pgfscope}%
\begin{pgfscope}%
\pgfpathrectangle{\pgfqpoint{1.000000in}{0.330000in}}{\pgfqpoint{6.200000in}{2.310000in}}%
\pgfusepath{clip}%
\pgfsetbuttcap%
\pgfsetroundjoin%
\pgfsetlinewidth{0.803000pt}%
\definecolor{currentstroke}{rgb}{0.000000,0.000000,0.000000}%
\pgfsetstrokecolor{currentstroke}%
\pgfsetdash{{0.800000pt}{1.320000pt}}{0.000000pt}%
\pgfpathmoveto{\pgfqpoint{6.473206in}{0.330000in}}%
\pgfpathlineto{\pgfqpoint{6.473206in}{2.640000in}}%
\pgfusepath{stroke}%
\end{pgfscope}%
\begin{pgfscope}%
\pgfsetbuttcap%
\pgfsetroundjoin%
\definecolor{currentfill}{rgb}{0.000000,0.000000,0.000000}%
\pgfsetfillcolor{currentfill}%
\pgfsetlinewidth{0.803000pt}%
\definecolor{currentstroke}{rgb}{0.000000,0.000000,0.000000}%
\pgfsetstrokecolor{currentstroke}%
\pgfsetdash{}{0pt}%
\pgfsys@defobject{currentmarker}{\pgfqpoint{0.000000in}{-0.048611in}}{\pgfqpoint{0.000000in}{0.000000in}}{%
\pgfpathmoveto{\pgfqpoint{0.000000in}{0.000000in}}%
\pgfpathlineto{\pgfqpoint{0.000000in}{-0.048611in}}%
\pgfusepath{stroke,fill}%
}%
\begin{pgfscope}%
\pgfsys@transformshift{6.473206in}{0.330000in}%
\pgfsys@useobject{currentmarker}{}%
\end{pgfscope}%
\end{pgfscope}%
\begin{pgfscope}%
\pgfsetbuttcap%
\pgfsetroundjoin%
\definecolor{currentfill}{rgb}{0.000000,0.000000,0.000000}%
\pgfsetfillcolor{currentfill}%
\pgfsetlinewidth{0.803000pt}%
\definecolor{currentstroke}{rgb}{0.000000,0.000000,0.000000}%
\pgfsetstrokecolor{currentstroke}%
\pgfsetdash{}{0pt}%
\pgfsys@defobject{currentmarker}{\pgfqpoint{0.000000in}{0.000000in}}{\pgfqpoint{0.000000in}{0.048611in}}{%
\pgfpathmoveto{\pgfqpoint{0.000000in}{0.000000in}}%
\pgfpathlineto{\pgfqpoint{0.000000in}{0.048611in}}%
\pgfusepath{stroke,fill}%
}%
\begin{pgfscope}%
\pgfsys@transformshift{6.473206in}{2.640000in}%
\pgfsys@useobject{currentmarker}{}%
\end{pgfscope}%
\end{pgfscope}%
\begin{pgfscope}%
\definecolor{textcolor}{rgb}{0.000000,0.000000,0.000000}%
\pgfsetstrokecolor{textcolor}%
\pgfsetfillcolor{textcolor}%
\pgftext[x=6.473206in,y=0.232778in,,top]{\color{textcolor}\rmfamily\fontsize{10.000000}{12.000000}\selectfont \(\displaystyle {17.5}\)}%
\end{pgfscope}%
\begin{pgfscope}%
\pgfpathrectangle{\pgfqpoint{1.000000in}{0.330000in}}{\pgfqpoint{6.200000in}{2.310000in}}%
\pgfusepath{clip}%
\pgfsetbuttcap%
\pgfsetroundjoin%
\pgfsetlinewidth{0.803000pt}%
\definecolor{currentstroke}{rgb}{0.000000,0.000000,0.000000}%
\pgfsetstrokecolor{currentstroke}%
\pgfsetdash{{0.800000pt}{1.320000pt}}{0.000000pt}%
\pgfpathmoveto{\pgfqpoint{1.000000in}{0.505918in}}%
\pgfpathlineto{\pgfqpoint{7.200000in}{0.505918in}}%
\pgfusepath{stroke}%
\end{pgfscope}%
\begin{pgfscope}%
\pgfsetbuttcap%
\pgfsetroundjoin%
\definecolor{currentfill}{rgb}{0.000000,0.000000,0.000000}%
\pgfsetfillcolor{currentfill}%
\pgfsetlinewidth{0.803000pt}%
\definecolor{currentstroke}{rgb}{0.000000,0.000000,0.000000}%
\pgfsetstrokecolor{currentstroke}%
\pgfsetdash{}{0pt}%
\pgfsys@defobject{currentmarker}{\pgfqpoint{-0.048611in}{0.000000in}}{\pgfqpoint{-0.000000in}{0.000000in}}{%
\pgfpathmoveto{\pgfqpoint{-0.000000in}{0.000000in}}%
\pgfpathlineto{\pgfqpoint{-0.048611in}{0.000000in}}%
\pgfusepath{stroke,fill}%
}%
\begin{pgfscope}%
\pgfsys@transformshift{1.000000in}{0.505918in}%
\pgfsys@useobject{currentmarker}{}%
\end{pgfscope}%
\end{pgfscope}%
\begin{pgfscope}%
\pgfsetbuttcap%
\pgfsetroundjoin%
\definecolor{currentfill}{rgb}{0.000000,0.000000,0.000000}%
\pgfsetfillcolor{currentfill}%
\pgfsetlinewidth{0.803000pt}%
\definecolor{currentstroke}{rgb}{0.000000,0.000000,0.000000}%
\pgfsetstrokecolor{currentstroke}%
\pgfsetdash{}{0pt}%
\pgfsys@defobject{currentmarker}{\pgfqpoint{0.000000in}{0.000000in}}{\pgfqpoint{0.048611in}{0.000000in}}{%
\pgfpathmoveto{\pgfqpoint{0.000000in}{0.000000in}}%
\pgfpathlineto{\pgfqpoint{0.048611in}{0.000000in}}%
\pgfusepath{stroke,fill}%
}%
\begin{pgfscope}%
\pgfsys@transformshift{7.200000in}{0.505918in}%
\pgfsys@useobject{currentmarker}{}%
\end{pgfscope}%
\end{pgfscope}%
\begin{pgfscope}%
\definecolor{textcolor}{rgb}{0.000000,0.000000,0.000000}%
\pgfsetstrokecolor{textcolor}%
\pgfsetfillcolor{textcolor}%
\pgftext[x=0.655863in, y=0.457693in, left, base]{\color{textcolor}\rmfamily\fontsize{10.000000}{12.000000}\selectfont \(\displaystyle {\ensuremath{-}50}\)}%
\end{pgfscope}%
\begin{pgfscope}%
\pgfpathrectangle{\pgfqpoint{1.000000in}{0.330000in}}{\pgfqpoint{6.200000in}{2.310000in}}%
\pgfusepath{clip}%
\pgfsetbuttcap%
\pgfsetroundjoin%
\pgfsetlinewidth{0.803000pt}%
\definecolor{currentstroke}{rgb}{0.000000,0.000000,0.000000}%
\pgfsetstrokecolor{currentstroke}%
\pgfsetdash{{0.800000pt}{1.320000pt}}{0.000000pt}%
\pgfpathmoveto{\pgfqpoint{1.000000in}{0.815190in}}%
\pgfpathlineto{\pgfqpoint{7.200000in}{0.815190in}}%
\pgfusepath{stroke}%
\end{pgfscope}%
\begin{pgfscope}%
\pgfsetbuttcap%
\pgfsetroundjoin%
\definecolor{currentfill}{rgb}{0.000000,0.000000,0.000000}%
\pgfsetfillcolor{currentfill}%
\pgfsetlinewidth{0.803000pt}%
\definecolor{currentstroke}{rgb}{0.000000,0.000000,0.000000}%
\pgfsetstrokecolor{currentstroke}%
\pgfsetdash{}{0pt}%
\pgfsys@defobject{currentmarker}{\pgfqpoint{-0.048611in}{0.000000in}}{\pgfqpoint{-0.000000in}{0.000000in}}{%
\pgfpathmoveto{\pgfqpoint{-0.000000in}{0.000000in}}%
\pgfpathlineto{\pgfqpoint{-0.048611in}{0.000000in}}%
\pgfusepath{stroke,fill}%
}%
\begin{pgfscope}%
\pgfsys@transformshift{1.000000in}{0.815190in}%
\pgfsys@useobject{currentmarker}{}%
\end{pgfscope}%
\end{pgfscope}%
\begin{pgfscope}%
\pgfsetbuttcap%
\pgfsetroundjoin%
\definecolor{currentfill}{rgb}{0.000000,0.000000,0.000000}%
\pgfsetfillcolor{currentfill}%
\pgfsetlinewidth{0.803000pt}%
\definecolor{currentstroke}{rgb}{0.000000,0.000000,0.000000}%
\pgfsetstrokecolor{currentstroke}%
\pgfsetdash{}{0pt}%
\pgfsys@defobject{currentmarker}{\pgfqpoint{0.000000in}{0.000000in}}{\pgfqpoint{0.048611in}{0.000000in}}{%
\pgfpathmoveto{\pgfqpoint{0.000000in}{0.000000in}}%
\pgfpathlineto{\pgfqpoint{0.048611in}{0.000000in}}%
\pgfusepath{stroke,fill}%
}%
\begin{pgfscope}%
\pgfsys@transformshift{7.200000in}{0.815190in}%
\pgfsys@useobject{currentmarker}{}%
\end{pgfscope}%
\end{pgfscope}%
\begin{pgfscope}%
\definecolor{textcolor}{rgb}{0.000000,0.000000,0.000000}%
\pgfsetstrokecolor{textcolor}%
\pgfsetfillcolor{textcolor}%
\pgftext[x=0.833333in, y=0.766964in, left, base]{\color{textcolor}\rmfamily\fontsize{10.000000}{12.000000}\selectfont \(\displaystyle {0}\)}%
\end{pgfscope}%
\begin{pgfscope}%
\pgfpathrectangle{\pgfqpoint{1.000000in}{0.330000in}}{\pgfqpoint{6.200000in}{2.310000in}}%
\pgfusepath{clip}%
\pgfsetbuttcap%
\pgfsetroundjoin%
\pgfsetlinewidth{0.803000pt}%
\definecolor{currentstroke}{rgb}{0.000000,0.000000,0.000000}%
\pgfsetstrokecolor{currentstroke}%
\pgfsetdash{{0.800000pt}{1.320000pt}}{0.000000pt}%
\pgfpathmoveto{\pgfqpoint{1.000000in}{1.124461in}}%
\pgfpathlineto{\pgfqpoint{7.200000in}{1.124461in}}%
\pgfusepath{stroke}%
\end{pgfscope}%
\begin{pgfscope}%
\pgfsetbuttcap%
\pgfsetroundjoin%
\definecolor{currentfill}{rgb}{0.000000,0.000000,0.000000}%
\pgfsetfillcolor{currentfill}%
\pgfsetlinewidth{0.803000pt}%
\definecolor{currentstroke}{rgb}{0.000000,0.000000,0.000000}%
\pgfsetstrokecolor{currentstroke}%
\pgfsetdash{}{0pt}%
\pgfsys@defobject{currentmarker}{\pgfqpoint{-0.048611in}{0.000000in}}{\pgfqpoint{-0.000000in}{0.000000in}}{%
\pgfpathmoveto{\pgfqpoint{-0.000000in}{0.000000in}}%
\pgfpathlineto{\pgfqpoint{-0.048611in}{0.000000in}}%
\pgfusepath{stroke,fill}%
}%
\begin{pgfscope}%
\pgfsys@transformshift{1.000000in}{1.124461in}%
\pgfsys@useobject{currentmarker}{}%
\end{pgfscope}%
\end{pgfscope}%
\begin{pgfscope}%
\pgfsetbuttcap%
\pgfsetroundjoin%
\definecolor{currentfill}{rgb}{0.000000,0.000000,0.000000}%
\pgfsetfillcolor{currentfill}%
\pgfsetlinewidth{0.803000pt}%
\definecolor{currentstroke}{rgb}{0.000000,0.000000,0.000000}%
\pgfsetstrokecolor{currentstroke}%
\pgfsetdash{}{0pt}%
\pgfsys@defobject{currentmarker}{\pgfqpoint{0.000000in}{0.000000in}}{\pgfqpoint{0.048611in}{0.000000in}}{%
\pgfpathmoveto{\pgfqpoint{0.000000in}{0.000000in}}%
\pgfpathlineto{\pgfqpoint{0.048611in}{0.000000in}}%
\pgfusepath{stroke,fill}%
}%
\begin{pgfscope}%
\pgfsys@transformshift{7.200000in}{1.124461in}%
\pgfsys@useobject{currentmarker}{}%
\end{pgfscope}%
\end{pgfscope}%
\begin{pgfscope}%
\definecolor{textcolor}{rgb}{0.000000,0.000000,0.000000}%
\pgfsetstrokecolor{textcolor}%
\pgfsetfillcolor{textcolor}%
\pgftext[x=0.763888in, y=1.076236in, left, base]{\color{textcolor}\rmfamily\fontsize{10.000000}{12.000000}\selectfont \(\displaystyle {50}\)}%
\end{pgfscope}%
\begin{pgfscope}%
\pgfpathrectangle{\pgfqpoint{1.000000in}{0.330000in}}{\pgfqpoint{6.200000in}{2.310000in}}%
\pgfusepath{clip}%
\pgfsetbuttcap%
\pgfsetroundjoin%
\pgfsetlinewidth{0.803000pt}%
\definecolor{currentstroke}{rgb}{0.000000,0.000000,0.000000}%
\pgfsetstrokecolor{currentstroke}%
\pgfsetdash{{0.800000pt}{1.320000pt}}{0.000000pt}%
\pgfpathmoveto{\pgfqpoint{1.000000in}{1.433733in}}%
\pgfpathlineto{\pgfqpoint{7.200000in}{1.433733in}}%
\pgfusepath{stroke}%
\end{pgfscope}%
\begin{pgfscope}%
\pgfsetbuttcap%
\pgfsetroundjoin%
\definecolor{currentfill}{rgb}{0.000000,0.000000,0.000000}%
\pgfsetfillcolor{currentfill}%
\pgfsetlinewidth{0.803000pt}%
\definecolor{currentstroke}{rgb}{0.000000,0.000000,0.000000}%
\pgfsetstrokecolor{currentstroke}%
\pgfsetdash{}{0pt}%
\pgfsys@defobject{currentmarker}{\pgfqpoint{-0.048611in}{0.000000in}}{\pgfqpoint{-0.000000in}{0.000000in}}{%
\pgfpathmoveto{\pgfqpoint{-0.000000in}{0.000000in}}%
\pgfpathlineto{\pgfqpoint{-0.048611in}{0.000000in}}%
\pgfusepath{stroke,fill}%
}%
\begin{pgfscope}%
\pgfsys@transformshift{1.000000in}{1.433733in}%
\pgfsys@useobject{currentmarker}{}%
\end{pgfscope}%
\end{pgfscope}%
\begin{pgfscope}%
\pgfsetbuttcap%
\pgfsetroundjoin%
\definecolor{currentfill}{rgb}{0.000000,0.000000,0.000000}%
\pgfsetfillcolor{currentfill}%
\pgfsetlinewidth{0.803000pt}%
\definecolor{currentstroke}{rgb}{0.000000,0.000000,0.000000}%
\pgfsetstrokecolor{currentstroke}%
\pgfsetdash{}{0pt}%
\pgfsys@defobject{currentmarker}{\pgfqpoint{0.000000in}{0.000000in}}{\pgfqpoint{0.048611in}{0.000000in}}{%
\pgfpathmoveto{\pgfqpoint{0.000000in}{0.000000in}}%
\pgfpathlineto{\pgfqpoint{0.048611in}{0.000000in}}%
\pgfusepath{stroke,fill}%
}%
\begin{pgfscope}%
\pgfsys@transformshift{7.200000in}{1.433733in}%
\pgfsys@useobject{currentmarker}{}%
\end{pgfscope}%
\end{pgfscope}%
\begin{pgfscope}%
\definecolor{textcolor}{rgb}{0.000000,0.000000,0.000000}%
\pgfsetstrokecolor{textcolor}%
\pgfsetfillcolor{textcolor}%
\pgftext[x=0.694444in, y=1.385507in, left, base]{\color{textcolor}\rmfamily\fontsize{10.000000}{12.000000}\selectfont \(\displaystyle {100}\)}%
\end{pgfscope}%
\begin{pgfscope}%
\pgfpathrectangle{\pgfqpoint{1.000000in}{0.330000in}}{\pgfqpoint{6.200000in}{2.310000in}}%
\pgfusepath{clip}%
\pgfsetbuttcap%
\pgfsetroundjoin%
\pgfsetlinewidth{0.803000pt}%
\definecolor{currentstroke}{rgb}{0.000000,0.000000,0.000000}%
\pgfsetstrokecolor{currentstroke}%
\pgfsetdash{{0.800000pt}{1.320000pt}}{0.000000pt}%
\pgfpathmoveto{\pgfqpoint{1.000000in}{1.743004in}}%
\pgfpathlineto{\pgfqpoint{7.200000in}{1.743004in}}%
\pgfusepath{stroke}%
\end{pgfscope}%
\begin{pgfscope}%
\pgfsetbuttcap%
\pgfsetroundjoin%
\definecolor{currentfill}{rgb}{0.000000,0.000000,0.000000}%
\pgfsetfillcolor{currentfill}%
\pgfsetlinewidth{0.803000pt}%
\definecolor{currentstroke}{rgb}{0.000000,0.000000,0.000000}%
\pgfsetstrokecolor{currentstroke}%
\pgfsetdash{}{0pt}%
\pgfsys@defobject{currentmarker}{\pgfqpoint{-0.048611in}{0.000000in}}{\pgfqpoint{-0.000000in}{0.000000in}}{%
\pgfpathmoveto{\pgfqpoint{-0.000000in}{0.000000in}}%
\pgfpathlineto{\pgfqpoint{-0.048611in}{0.000000in}}%
\pgfusepath{stroke,fill}%
}%
\begin{pgfscope}%
\pgfsys@transformshift{1.000000in}{1.743004in}%
\pgfsys@useobject{currentmarker}{}%
\end{pgfscope}%
\end{pgfscope}%
\begin{pgfscope}%
\pgfsetbuttcap%
\pgfsetroundjoin%
\definecolor{currentfill}{rgb}{0.000000,0.000000,0.000000}%
\pgfsetfillcolor{currentfill}%
\pgfsetlinewidth{0.803000pt}%
\definecolor{currentstroke}{rgb}{0.000000,0.000000,0.000000}%
\pgfsetstrokecolor{currentstroke}%
\pgfsetdash{}{0pt}%
\pgfsys@defobject{currentmarker}{\pgfqpoint{0.000000in}{0.000000in}}{\pgfqpoint{0.048611in}{0.000000in}}{%
\pgfpathmoveto{\pgfqpoint{0.000000in}{0.000000in}}%
\pgfpathlineto{\pgfqpoint{0.048611in}{0.000000in}}%
\pgfusepath{stroke,fill}%
}%
\begin{pgfscope}%
\pgfsys@transformshift{7.200000in}{1.743004in}%
\pgfsys@useobject{currentmarker}{}%
\end{pgfscope}%
\end{pgfscope}%
\begin{pgfscope}%
\definecolor{textcolor}{rgb}{0.000000,0.000000,0.000000}%
\pgfsetstrokecolor{textcolor}%
\pgfsetfillcolor{textcolor}%
\pgftext[x=0.694444in, y=1.694779in, left, base]{\color{textcolor}\rmfamily\fontsize{10.000000}{12.000000}\selectfont \(\displaystyle {150}\)}%
\end{pgfscope}%
\begin{pgfscope}%
\pgfpathrectangle{\pgfqpoint{1.000000in}{0.330000in}}{\pgfqpoint{6.200000in}{2.310000in}}%
\pgfusepath{clip}%
\pgfsetbuttcap%
\pgfsetroundjoin%
\pgfsetlinewidth{0.803000pt}%
\definecolor{currentstroke}{rgb}{0.000000,0.000000,0.000000}%
\pgfsetstrokecolor{currentstroke}%
\pgfsetdash{{0.800000pt}{1.320000pt}}{0.000000pt}%
\pgfpathmoveto{\pgfqpoint{1.000000in}{2.052276in}}%
\pgfpathlineto{\pgfqpoint{7.200000in}{2.052276in}}%
\pgfusepath{stroke}%
\end{pgfscope}%
\begin{pgfscope}%
\pgfsetbuttcap%
\pgfsetroundjoin%
\definecolor{currentfill}{rgb}{0.000000,0.000000,0.000000}%
\pgfsetfillcolor{currentfill}%
\pgfsetlinewidth{0.803000pt}%
\definecolor{currentstroke}{rgb}{0.000000,0.000000,0.000000}%
\pgfsetstrokecolor{currentstroke}%
\pgfsetdash{}{0pt}%
\pgfsys@defobject{currentmarker}{\pgfqpoint{-0.048611in}{0.000000in}}{\pgfqpoint{-0.000000in}{0.000000in}}{%
\pgfpathmoveto{\pgfqpoint{-0.000000in}{0.000000in}}%
\pgfpathlineto{\pgfqpoint{-0.048611in}{0.000000in}}%
\pgfusepath{stroke,fill}%
}%
\begin{pgfscope}%
\pgfsys@transformshift{1.000000in}{2.052276in}%
\pgfsys@useobject{currentmarker}{}%
\end{pgfscope}%
\end{pgfscope}%
\begin{pgfscope}%
\pgfsetbuttcap%
\pgfsetroundjoin%
\definecolor{currentfill}{rgb}{0.000000,0.000000,0.000000}%
\pgfsetfillcolor{currentfill}%
\pgfsetlinewidth{0.803000pt}%
\definecolor{currentstroke}{rgb}{0.000000,0.000000,0.000000}%
\pgfsetstrokecolor{currentstroke}%
\pgfsetdash{}{0pt}%
\pgfsys@defobject{currentmarker}{\pgfqpoint{0.000000in}{0.000000in}}{\pgfqpoint{0.048611in}{0.000000in}}{%
\pgfpathmoveto{\pgfqpoint{0.000000in}{0.000000in}}%
\pgfpathlineto{\pgfqpoint{0.048611in}{0.000000in}}%
\pgfusepath{stroke,fill}%
}%
\begin{pgfscope}%
\pgfsys@transformshift{7.200000in}{2.052276in}%
\pgfsys@useobject{currentmarker}{}%
\end{pgfscope}%
\end{pgfscope}%
\begin{pgfscope}%
\definecolor{textcolor}{rgb}{0.000000,0.000000,0.000000}%
\pgfsetstrokecolor{textcolor}%
\pgfsetfillcolor{textcolor}%
\pgftext[x=0.694444in, y=2.004050in, left, base]{\color{textcolor}\rmfamily\fontsize{10.000000}{12.000000}\selectfont \(\displaystyle {200}\)}%
\end{pgfscope}%
\begin{pgfscope}%
\pgfpathrectangle{\pgfqpoint{1.000000in}{0.330000in}}{\pgfqpoint{6.200000in}{2.310000in}}%
\pgfusepath{clip}%
\pgfsetbuttcap%
\pgfsetroundjoin%
\pgfsetlinewidth{0.803000pt}%
\definecolor{currentstroke}{rgb}{0.000000,0.000000,0.000000}%
\pgfsetstrokecolor{currentstroke}%
\pgfsetdash{{0.800000pt}{1.320000pt}}{0.000000pt}%
\pgfpathmoveto{\pgfqpoint{1.000000in}{2.361547in}}%
\pgfpathlineto{\pgfqpoint{7.200000in}{2.361547in}}%
\pgfusepath{stroke}%
\end{pgfscope}%
\begin{pgfscope}%
\pgfsetbuttcap%
\pgfsetroundjoin%
\definecolor{currentfill}{rgb}{0.000000,0.000000,0.000000}%
\pgfsetfillcolor{currentfill}%
\pgfsetlinewidth{0.803000pt}%
\definecolor{currentstroke}{rgb}{0.000000,0.000000,0.000000}%
\pgfsetstrokecolor{currentstroke}%
\pgfsetdash{}{0pt}%
\pgfsys@defobject{currentmarker}{\pgfqpoint{-0.048611in}{0.000000in}}{\pgfqpoint{-0.000000in}{0.000000in}}{%
\pgfpathmoveto{\pgfqpoint{-0.000000in}{0.000000in}}%
\pgfpathlineto{\pgfqpoint{-0.048611in}{0.000000in}}%
\pgfusepath{stroke,fill}%
}%
\begin{pgfscope}%
\pgfsys@transformshift{1.000000in}{2.361547in}%
\pgfsys@useobject{currentmarker}{}%
\end{pgfscope}%
\end{pgfscope}%
\begin{pgfscope}%
\pgfsetbuttcap%
\pgfsetroundjoin%
\definecolor{currentfill}{rgb}{0.000000,0.000000,0.000000}%
\pgfsetfillcolor{currentfill}%
\pgfsetlinewidth{0.803000pt}%
\definecolor{currentstroke}{rgb}{0.000000,0.000000,0.000000}%
\pgfsetstrokecolor{currentstroke}%
\pgfsetdash{}{0pt}%
\pgfsys@defobject{currentmarker}{\pgfqpoint{0.000000in}{0.000000in}}{\pgfqpoint{0.048611in}{0.000000in}}{%
\pgfpathmoveto{\pgfqpoint{0.000000in}{0.000000in}}%
\pgfpathlineto{\pgfqpoint{0.048611in}{0.000000in}}%
\pgfusepath{stroke,fill}%
}%
\begin{pgfscope}%
\pgfsys@transformshift{7.200000in}{2.361547in}%
\pgfsys@useobject{currentmarker}{}%
\end{pgfscope}%
\end{pgfscope}%
\begin{pgfscope}%
\definecolor{textcolor}{rgb}{0.000000,0.000000,0.000000}%
\pgfsetstrokecolor{textcolor}%
\pgfsetfillcolor{textcolor}%
\pgftext[x=0.694444in, y=2.313322in, left, base]{\color{textcolor}\rmfamily\fontsize{10.000000}{12.000000}\selectfont \(\displaystyle {250}\)}%
\end{pgfscope}%
\begin{pgfscope}%
\pgfpathrectangle{\pgfqpoint{1.000000in}{0.330000in}}{\pgfqpoint{6.200000in}{2.310000in}}%
\pgfusepath{clip}%
\pgfsetrectcap%
\pgfsetroundjoin%
\pgfsetlinewidth{1.505625pt}%
\definecolor{currentstroke}{rgb}{0.121569,0.466667,0.705882}%
\pgfsetstrokecolor{currentstroke}%
\pgfsetdash{}{0pt}%
\pgfpathmoveto{\pgfqpoint{1.281818in}{0.815190in}}%
\pgfpathlineto{\pgfqpoint{1.281818in}{0.815190in}}%
\pgfpathlineto{\pgfqpoint{1.776236in}{0.913964in}}%
\pgfpathlineto{\pgfqpoint{2.270654in}{1.010459in}}%
\pgfpathlineto{\pgfqpoint{2.765072in}{1.104674in}}%
\pgfpathlineto{\pgfqpoint{3.259490in}{1.196610in}}%
\pgfpathlineto{\pgfqpoint{3.753907in}{1.286267in}}%
\pgfpathlineto{\pgfqpoint{3.951675in}{1.321493in}}%
\pgfpathlineto{\pgfqpoint{4.446093in}{1.242809in}}%
\pgfpathlineto{\pgfqpoint{4.940510in}{1.161844in}}%
\pgfpathlineto{\pgfqpoint{5.434928in}{1.078599in}}%
\pgfpathlineto{\pgfqpoint{5.929346in}{0.993077in}}%
\pgfpathlineto{\pgfqpoint{6.423764in}{0.905273in}}%
\pgfpathlineto{\pgfqpoint{6.918182in}{0.815190in}}%
\pgfpathlineto{\pgfqpoint{6.918182in}{0.815190in}}%
\pgfusepath{stroke}%
\end{pgfscope}%
\begin{pgfscope}%
\pgfpathrectangle{\pgfqpoint{1.000000in}{0.330000in}}{\pgfqpoint{6.200000in}{2.310000in}}%
\pgfusepath{clip}%
\pgfsetrectcap%
\pgfsetroundjoin%
\pgfsetlinewidth{1.505625pt}%
\definecolor{currentstroke}{rgb}{1.000000,0.498039,0.054902}%
\pgfsetstrokecolor{currentstroke}%
\pgfsetdash{}{0pt}%
\pgfpathmoveto{\pgfqpoint{1.281818in}{0.815190in}}%
\pgfpathlineto{\pgfqpoint{1.281818in}{0.815190in}}%
\pgfpathlineto{\pgfqpoint{1.702073in}{1.032469in}}%
\pgfpathlineto{\pgfqpoint{2.122329in}{1.247220in}}%
\pgfpathlineto{\pgfqpoint{2.542584in}{1.459442in}}%
\pgfpathlineto{\pgfqpoint{2.962839in}{1.669136in}}%
\pgfpathlineto{\pgfqpoint{3.383094in}{1.876301in}}%
\pgfpathlineto{\pgfqpoint{3.803349in}{2.080938in}}%
\pgfpathlineto{\pgfqpoint{3.951675in}{2.152562in}}%
\pgfpathlineto{\pgfqpoint{4.371930in}{1.970763in}}%
\pgfpathlineto{\pgfqpoint{4.792185in}{1.786433in}}%
\pgfpathlineto{\pgfqpoint{5.212440in}{1.599574in}}%
\pgfpathlineto{\pgfqpoint{5.632695in}{1.410189in}}%
\pgfpathlineto{\pgfqpoint{6.052951in}{1.218273in}}%
\pgfpathlineto{\pgfqpoint{6.473206in}{1.023828in}}%
\pgfpathlineto{\pgfqpoint{6.893461in}{0.826855in}}%
\pgfpathlineto{\pgfqpoint{6.918182in}{0.815190in}}%
\pgfpathlineto{\pgfqpoint{6.918182in}{0.815190in}}%
\pgfusepath{stroke}%
\end{pgfscope}%
\begin{pgfscope}%
\pgfpathrectangle{\pgfqpoint{1.000000in}{0.330000in}}{\pgfqpoint{6.200000in}{2.310000in}}%
\pgfusepath{clip}%
\pgfsetrectcap%
\pgfsetroundjoin%
\pgfsetlinewidth{1.505625pt}%
\definecolor{currentstroke}{rgb}{0.172549,0.627451,0.172549}%
\pgfsetstrokecolor{currentstroke}%
\pgfsetdash{}{0pt}%
\pgfpathmoveto{\pgfqpoint{1.281818in}{0.815190in}}%
\pgfpathlineto{\pgfqpoint{1.281818in}{0.815190in}}%
\pgfpathlineto{\pgfqpoint{1.677352in}{0.970024in}}%
\pgfpathlineto{\pgfqpoint{2.072887in}{1.122443in}}%
\pgfpathlineto{\pgfqpoint{2.468421in}{1.272444in}}%
\pgfpathlineto{\pgfqpoint{2.863955in}{1.420029in}}%
\pgfpathlineto{\pgfqpoint{3.259490in}{1.565197in}}%
\pgfpathlineto{\pgfqpoint{3.655024in}{1.707949in}}%
\pgfpathlineto{\pgfqpoint{3.951675in}{1.813429in}}%
\pgfpathlineto{\pgfqpoint{4.371930in}{1.547196in}}%
\pgfpathlineto{\pgfqpoint{4.792185in}{1.278231in}}%
\pgfpathlineto{\pgfqpoint{5.212440in}{1.006538in}}%
\pgfpathlineto{\pgfqpoint{5.583254in}{0.764546in}}%
\pgfpathlineto{\pgfqpoint{5.954067in}{0.781375in}}%
\pgfpathlineto{\pgfqpoint{6.324880in}{0.796080in}}%
\pgfpathlineto{\pgfqpoint{6.695694in}{0.808661in}}%
\pgfpathlineto{\pgfqpoint{6.918182in}{0.815190in}}%
\pgfpathlineto{\pgfqpoint{6.918182in}{0.815190in}}%
\pgfusepath{stroke}%
\end{pgfscope}%
\begin{pgfscope}%
\pgfpathrectangle{\pgfqpoint{1.000000in}{0.330000in}}{\pgfqpoint{6.200000in}{2.310000in}}%
\pgfusepath{clip}%
\pgfsetrectcap%
\pgfsetroundjoin%
\pgfsetlinewidth{1.505625pt}%
\definecolor{currentstroke}{rgb}{0.839216,0.152941,0.156863}%
\pgfsetstrokecolor{currentstroke}%
\pgfsetdash{}{0pt}%
\pgfpathmoveto{\pgfqpoint{1.281818in}{0.815190in}}%
\pgfpathlineto{\pgfqpoint{1.281818in}{0.815190in}}%
\pgfpathlineto{\pgfqpoint{1.702073in}{1.028549in}}%
\pgfpathlineto{\pgfqpoint{2.122329in}{1.239380in}}%
\pgfpathlineto{\pgfqpoint{2.542584in}{1.447682in}}%
\pgfpathlineto{\pgfqpoint{2.962839in}{1.653456in}}%
\pgfpathlineto{\pgfqpoint{3.383094in}{1.856701in}}%
\pgfpathlineto{\pgfqpoint{3.803349in}{2.057417in}}%
\pgfpathlineto{\pgfqpoint{3.951675in}{2.127657in}}%
\pgfpathlineto{\pgfqpoint{4.371930in}{1.949387in}}%
\pgfpathlineto{\pgfqpoint{4.792185in}{1.768585in}}%
\pgfpathlineto{\pgfqpoint{5.212440in}{1.585254in}}%
\pgfpathlineto{\pgfqpoint{5.632695in}{1.399397in}}%
\pgfpathlineto{\pgfqpoint{6.052951in}{1.211009in}}%
\pgfpathlineto{\pgfqpoint{6.473206in}{1.020093in}}%
\pgfpathlineto{\pgfqpoint{6.893461in}{0.826647in}}%
\pgfpathlineto{\pgfqpoint{6.918182in}{0.815190in}}%
\pgfpathlineto{\pgfqpoint{6.918182in}{0.815190in}}%
\pgfusepath{stroke}%
\end{pgfscope}%
\begin{pgfscope}%
\pgfpathrectangle{\pgfqpoint{1.000000in}{0.330000in}}{\pgfqpoint{6.200000in}{2.310000in}}%
\pgfusepath{clip}%
\pgfsetrectcap%
\pgfsetroundjoin%
\pgfsetlinewidth{1.505625pt}%
\definecolor{currentstroke}{rgb}{0.580392,0.403922,0.741176}%
\pgfsetstrokecolor{currentstroke}%
\pgfsetdash{}{0pt}%
\pgfpathmoveto{\pgfqpoint{1.281818in}{0.815190in}}%
\pgfpathlineto{\pgfqpoint{1.281818in}{0.815190in}}%
\pgfpathlineto{\pgfqpoint{1.677352in}{0.938392in}}%
\pgfpathlineto{\pgfqpoint{2.072887in}{1.059326in}}%
\pgfpathlineto{\pgfqpoint{2.468421in}{1.177990in}}%
\pgfpathlineto{\pgfqpoint{2.863955in}{1.294385in}}%
\pgfpathlineto{\pgfqpoint{3.259490in}{1.408511in}}%
\pgfpathlineto{\pgfqpoint{3.655024in}{1.520367in}}%
\pgfpathlineto{\pgfqpoint{3.951675in}{1.602773in}}%
\pgfpathlineto{\pgfqpoint{4.347209in}{1.505139in}}%
\pgfpathlineto{\pgfqpoint{4.742743in}{1.405233in}}%
\pgfpathlineto{\pgfqpoint{5.138278in}{1.303058in}}%
\pgfpathlineto{\pgfqpoint{5.533812in}{1.198614in}}%
\pgfpathlineto{\pgfqpoint{5.929346in}{1.091903in}}%
\pgfpathlineto{\pgfqpoint{6.324880in}{0.982919in}}%
\pgfpathlineto{\pgfqpoint{6.720415in}{0.871667in}}%
\pgfpathlineto{\pgfqpoint{6.918182in}{0.815190in}}%
\pgfpathlineto{\pgfqpoint{6.918182in}{0.815190in}}%
\pgfusepath{stroke}%
\end{pgfscope}%
\begin{pgfscope}%
\pgfpathrectangle{\pgfqpoint{1.000000in}{0.330000in}}{\pgfqpoint{6.200000in}{2.310000in}}%
\pgfusepath{clip}%
\pgfsetrectcap%
\pgfsetroundjoin%
\pgfsetlinewidth{1.505625pt}%
\definecolor{currentstroke}{rgb}{0.549020,0.337255,0.294118}%
\pgfsetstrokecolor{currentstroke}%
\pgfsetdash{}{0pt}%
\pgfpathmoveto{\pgfqpoint{1.281818in}{0.815190in}}%
\pgfpathlineto{\pgfqpoint{1.281818in}{0.815190in}}%
\pgfpathlineto{\pgfqpoint{1.850399in}{0.926394in}}%
\pgfpathlineto{\pgfqpoint{2.418979in}{1.035253in}}%
\pgfpathlineto{\pgfqpoint{2.987560in}{1.141768in}}%
\pgfpathlineto{\pgfqpoint{3.556140in}{1.245938in}}%
\pgfpathlineto{\pgfqpoint{3.976396in}{1.320933in}}%
\pgfpathlineto{\pgfqpoint{4.544976in}{1.409645in}}%
\pgfpathlineto{\pgfqpoint{5.113557in}{1.496012in}}%
\pgfpathlineto{\pgfqpoint{5.583254in}{1.565591in}}%
\pgfpathlineto{\pgfqpoint{6.176555in}{1.233675in}}%
\pgfpathlineto{\pgfqpoint{6.769856in}{0.899206in}}%
\pgfpathlineto{\pgfqpoint{6.918182in}{0.815190in}}%
\pgfpathlineto{\pgfqpoint{6.918182in}{0.815190in}}%
\pgfusepath{stroke}%
\end{pgfscope}%
\begin{pgfscope}%
\pgfpathrectangle{\pgfqpoint{1.000000in}{0.330000in}}{\pgfqpoint{6.200000in}{2.310000in}}%
\pgfusepath{clip}%
\pgfsetrectcap%
\pgfsetroundjoin%
\pgfsetlinewidth{1.505625pt}%
\definecolor{currentstroke}{rgb}{0.890196,0.466667,0.760784}%
\pgfsetstrokecolor{currentstroke}%
\pgfsetdash{}{0pt}%
\pgfpathmoveto{\pgfqpoint{1.281818in}{0.815190in}}%
\pgfpathlineto{\pgfqpoint{1.281818in}{0.815190in}}%
\pgfpathlineto{\pgfqpoint{1.702073in}{0.927328in}}%
\pgfpathlineto{\pgfqpoint{2.122329in}{1.037270in}}%
\pgfpathlineto{\pgfqpoint{2.542584in}{1.145016in}}%
\pgfpathlineto{\pgfqpoint{2.962839in}{1.250566in}}%
\pgfpathlineto{\pgfqpoint{3.383094in}{1.353921in}}%
\pgfpathlineto{\pgfqpoint{3.803349in}{1.455079in}}%
\pgfpathlineto{\pgfqpoint{3.951675in}{1.490261in}}%
\pgfpathlineto{\pgfqpoint{4.371930in}{1.401280in}}%
\pgfpathlineto{\pgfqpoint{4.792185in}{1.310100in}}%
\pgfpathlineto{\pgfqpoint{5.212440in}{1.216725in}}%
\pgfpathlineto{\pgfqpoint{5.632695in}{1.121156in}}%
\pgfpathlineto{\pgfqpoint{6.052951in}{1.023389in}}%
\pgfpathlineto{\pgfqpoint{6.473206in}{0.923426in}}%
\pgfpathlineto{\pgfqpoint{6.893461in}{0.821267in}}%
\pgfpathlineto{\pgfqpoint{6.918182in}{0.815190in}}%
\pgfpathlineto{\pgfqpoint{6.918182in}{0.815190in}}%
\pgfusepath{stroke}%
\end{pgfscope}%
\begin{pgfscope}%
\pgfpathrectangle{\pgfqpoint{1.000000in}{0.330000in}}{\pgfqpoint{6.200000in}{2.310000in}}%
\pgfusepath{clip}%
\pgfsetrectcap%
\pgfsetroundjoin%
\pgfsetlinewidth{1.505625pt}%
\definecolor{currentstroke}{rgb}{0.498039,0.498039,0.498039}%
\pgfsetstrokecolor{currentstroke}%
\pgfsetdash{}{0pt}%
\pgfpathmoveto{\pgfqpoint{1.281818in}{0.815190in}}%
\pgfpathlineto{\pgfqpoint{1.281818in}{0.815190in}}%
\pgfpathlineto{\pgfqpoint{1.702073in}{1.041093in}}%
\pgfpathlineto{\pgfqpoint{2.122329in}{1.264469in}}%
\pgfpathlineto{\pgfqpoint{2.542584in}{1.485315in}}%
\pgfpathlineto{\pgfqpoint{2.962839in}{1.703634in}}%
\pgfpathlineto{\pgfqpoint{3.383094in}{1.919423in}}%
\pgfpathlineto{\pgfqpoint{3.803349in}{2.132684in}}%
\pgfpathlineto{\pgfqpoint{3.951675in}{2.207352in}}%
\pgfpathlineto{\pgfqpoint{4.371930in}{2.017791in}}%
\pgfpathlineto{\pgfqpoint{4.792185in}{1.825699in}}%
\pgfpathlineto{\pgfqpoint{5.212440in}{1.631078in}}%
\pgfpathlineto{\pgfqpoint{5.632695in}{1.433931in}}%
\pgfpathlineto{\pgfqpoint{6.052951in}{1.234253in}}%
\pgfpathlineto{\pgfqpoint{6.473206in}{1.032047in}}%
\pgfpathlineto{\pgfqpoint{6.893461in}{0.827312in}}%
\pgfpathlineto{\pgfqpoint{6.918182in}{0.815190in}}%
\pgfpathlineto{\pgfqpoint{6.918182in}{0.815190in}}%
\pgfusepath{stroke}%
\end{pgfscope}%
\begin{pgfscope}%
\pgfpathrectangle{\pgfqpoint{1.000000in}{0.330000in}}{\pgfqpoint{6.200000in}{2.310000in}}%
\pgfusepath{clip}%
\pgfsetrectcap%
\pgfsetroundjoin%
\pgfsetlinewidth{1.505625pt}%
\definecolor{currentstroke}{rgb}{0.737255,0.741176,0.133333}%
\pgfsetstrokecolor{currentstroke}%
\pgfsetdash{}{0pt}%
\pgfpathmoveto{\pgfqpoint{1.281818in}{0.815190in}}%
\pgfpathlineto{\pgfqpoint{1.281818in}{0.815190in}}%
\pgfpathlineto{\pgfqpoint{1.702073in}{1.093202in}}%
\pgfpathlineto{\pgfqpoint{2.122329in}{1.368486in}}%
\pgfpathlineto{\pgfqpoint{2.542584in}{1.641042in}}%
\pgfpathlineto{\pgfqpoint{2.962839in}{1.910870in}}%
\pgfpathlineto{\pgfqpoint{3.383094in}{2.177969in}}%
\pgfpathlineto{\pgfqpoint{3.803349in}{2.442341in}}%
\pgfpathlineto{\pgfqpoint{3.951675in}{2.535000in}}%
\pgfpathlineto{\pgfqpoint{4.347209in}{2.313548in}}%
\pgfpathlineto{\pgfqpoint{4.742743in}{2.089677in}}%
\pgfpathlineto{\pgfqpoint{5.138278in}{1.863389in}}%
\pgfpathlineto{\pgfqpoint{5.533812in}{1.634684in}}%
\pgfpathlineto{\pgfqpoint{5.954067in}{1.389041in}}%
\pgfpathlineto{\pgfqpoint{6.374322in}{1.140666in}}%
\pgfpathlineto{\pgfqpoint{6.794577in}{0.889563in}}%
\pgfpathlineto{\pgfqpoint{6.918182in}{0.815190in}}%
\pgfpathlineto{\pgfqpoint{6.918182in}{0.815190in}}%
\pgfusepath{stroke}%
\end{pgfscope}%
\begin{pgfscope}%
\pgfpathrectangle{\pgfqpoint{1.000000in}{0.330000in}}{\pgfqpoint{6.200000in}{2.310000in}}%
\pgfusepath{clip}%
\pgfsetrectcap%
\pgfsetroundjoin%
\pgfsetlinewidth{1.505625pt}%
\definecolor{currentstroke}{rgb}{0.090196,0.745098,0.811765}%
\pgfsetstrokecolor{currentstroke}%
\pgfsetdash{}{0pt}%
\pgfpathmoveto{\pgfqpoint{1.281818in}{0.815190in}}%
\pgfpathlineto{\pgfqpoint{1.281818in}{0.815190in}}%
\pgfpathlineto{\pgfqpoint{1.702073in}{0.939872in}}%
\pgfpathlineto{\pgfqpoint{2.122329in}{1.062359in}}%
\pgfpathlineto{\pgfqpoint{2.542584in}{1.182650in}}%
\pgfpathlineto{\pgfqpoint{2.962839in}{1.300744in}}%
\pgfpathlineto{\pgfqpoint{3.383094in}{1.416643in}}%
\pgfpathlineto{\pgfqpoint{3.803349in}{1.530346in}}%
\pgfpathlineto{\pgfqpoint{3.951675in}{1.569955in}}%
\pgfpathlineto{\pgfqpoint{4.371930in}{1.469684in}}%
\pgfpathlineto{\pgfqpoint{4.792185in}{1.367215in}}%
\pgfpathlineto{\pgfqpoint{5.212440in}{1.262549in}}%
\pgfpathlineto{\pgfqpoint{5.632695in}{1.155691in}}%
\pgfpathlineto{\pgfqpoint{6.052951in}{1.046633in}}%
\pgfpathlineto{\pgfqpoint{6.473206in}{0.935380in}}%
\pgfpathlineto{\pgfqpoint{6.893461in}{0.821931in}}%
\pgfpathlineto{\pgfqpoint{6.918182in}{0.815190in}}%
\pgfpathlineto{\pgfqpoint{6.918182in}{0.815190in}}%
\pgfusepath{stroke}%
\end{pgfscope}%
\begin{pgfscope}%
\pgfpathrectangle{\pgfqpoint{1.000000in}{0.330000in}}{\pgfqpoint{6.200000in}{2.310000in}}%
\pgfusepath{clip}%
\pgfsetrectcap%
\pgfsetroundjoin%
\pgfsetlinewidth{1.505625pt}%
\definecolor{currentstroke}{rgb}{0.121569,0.466667,0.705882}%
\pgfsetstrokecolor{currentstroke}%
\pgfsetdash{}{0pt}%
\pgfpathmoveto{\pgfqpoint{1.281818in}{0.815190in}}%
\pgfpathlineto{\pgfqpoint{1.281818in}{0.815190in}}%
\pgfpathlineto{\pgfqpoint{1.677352in}{1.002037in}}%
\pgfpathlineto{\pgfqpoint{2.072887in}{1.186467in}}%
\pgfpathlineto{\pgfqpoint{2.468421in}{1.368481in}}%
\pgfpathlineto{\pgfqpoint{2.863955in}{1.548078in}}%
\pgfpathlineto{\pgfqpoint{3.259490in}{1.725259in}}%
\pgfpathlineto{\pgfqpoint{3.655024in}{1.900023in}}%
\pgfpathlineto{\pgfqpoint{3.951675in}{2.029513in}}%
\pgfpathlineto{\pgfqpoint{4.322488in}{2.002585in}}%
\pgfpathlineto{\pgfqpoint{4.693301in}{1.973531in}}%
\pgfpathlineto{\pgfqpoint{5.064115in}{1.942352in}}%
\pgfpathlineto{\pgfqpoint{5.434928in}{1.909050in}}%
\pgfpathlineto{\pgfqpoint{5.583254in}{1.895137in}}%
\pgfpathlineto{\pgfqpoint{6.003509in}{1.558122in}}%
\pgfpathlineto{\pgfqpoint{6.423764in}{1.218380in}}%
\pgfpathlineto{\pgfqpoint{6.844019in}{0.875909in}}%
\pgfpathlineto{\pgfqpoint{6.918182in}{0.815190in}}%
\pgfpathlineto{\pgfqpoint{6.918182in}{0.815190in}}%
\pgfusepath{stroke}%
\end{pgfscope}%
\begin{pgfscope}%
\pgfpathrectangle{\pgfqpoint{1.000000in}{0.330000in}}{\pgfqpoint{6.200000in}{2.310000in}}%
\pgfusepath{clip}%
\pgfsetrectcap%
\pgfsetroundjoin%
\pgfsetlinewidth{1.505625pt}%
\definecolor{currentstroke}{rgb}{1.000000,0.498039,0.054902}%
\pgfsetstrokecolor{currentstroke}%
\pgfsetdash{}{0pt}%
\pgfpathmoveto{\pgfqpoint{1.281818in}{0.815190in}}%
\pgfpathlineto{\pgfqpoint{1.281818in}{0.815190in}}%
\pgfpathlineto{\pgfqpoint{1.702073in}{1.089282in}}%
\pgfpathlineto{\pgfqpoint{2.122329in}{1.360646in}}%
\pgfpathlineto{\pgfqpoint{2.542584in}{1.629282in}}%
\pgfpathlineto{\pgfqpoint{2.962839in}{1.895189in}}%
\pgfpathlineto{\pgfqpoint{3.383094in}{2.158369in}}%
\pgfpathlineto{\pgfqpoint{3.803349in}{2.418820in}}%
\pgfpathlineto{\pgfqpoint{3.951675in}{2.510096in}}%
\pgfpathlineto{\pgfqpoint{4.347209in}{2.291965in}}%
\pgfpathlineto{\pgfqpoint{4.742743in}{2.071414in}}%
\pgfpathlineto{\pgfqpoint{5.138278in}{1.848446in}}%
\pgfpathlineto{\pgfqpoint{5.533812in}{1.623062in}}%
\pgfpathlineto{\pgfqpoint{5.954067in}{1.380947in}}%
\pgfpathlineto{\pgfqpoint{6.374322in}{1.136100in}}%
\pgfpathlineto{\pgfqpoint{6.794577in}{0.888525in}}%
\pgfpathlineto{\pgfqpoint{6.918182in}{0.815190in}}%
\pgfpathlineto{\pgfqpoint{6.918182in}{0.815190in}}%
\pgfusepath{stroke}%
\end{pgfscope}%
\begin{pgfscope}%
\pgfpathrectangle{\pgfqpoint{1.000000in}{0.330000in}}{\pgfqpoint{6.200000in}{2.310000in}}%
\pgfusepath{clip}%
\pgfsetrectcap%
\pgfsetroundjoin%
\pgfsetlinewidth{1.505625pt}%
\definecolor{currentstroke}{rgb}{0.172549,0.627451,0.172549}%
\pgfsetstrokecolor{currentstroke}%
\pgfsetdash{}{0pt}%
\pgfpathmoveto{\pgfqpoint{1.281818in}{0.815190in}}%
\pgfpathlineto{\pgfqpoint{1.281818in}{0.815190in}}%
\pgfpathlineto{\pgfqpoint{1.825678in}{0.877591in}}%
\pgfpathlineto{\pgfqpoint{2.369537in}{0.937846in}}%
\pgfpathlineto{\pgfqpoint{2.913397in}{0.995957in}}%
\pgfpathlineto{\pgfqpoint{3.457257in}{1.051922in}}%
\pgfpathlineto{\pgfqpoint{3.951675in}{1.100939in}}%
\pgfpathlineto{\pgfqpoint{4.520255in}{0.871062in}}%
\pgfpathlineto{\pgfqpoint{5.088836in}{0.638838in}}%
\pgfpathlineto{\pgfqpoint{5.583254in}{0.435000in}}%
\pgfpathlineto{\pgfqpoint{6.151834in}{0.598513in}}%
\pgfpathlineto{\pgfqpoint{6.720415in}{0.759681in}}%
\pgfpathlineto{\pgfqpoint{6.918182in}{0.815190in}}%
\pgfpathlineto{\pgfqpoint{6.918182in}{0.815190in}}%
\pgfusepath{stroke}%
\end{pgfscope}%
\begin{pgfscope}%
\pgfsetroundcap%
\pgfsetroundjoin%
\pgfsetlinewidth{1.003750pt}%
\definecolor{currentstroke}{rgb}{0.000000,0.000000,0.000000}%
\pgfsetstrokecolor{currentstroke}%
\pgfsetdash{}{0pt}%
\pgfpathmoveto{\pgfqpoint{4.315446in}{2.535000in}}%
\pgfpathquadraticcurveto{\pgfqpoint{4.147437in}{2.535000in}}{\pgfqpoint{3.979428in}{2.535000in}}%
\pgfusepath{stroke}%
\end{pgfscope}%
\begin{pgfscope}%
\pgfsetbuttcap%
\pgfsetmiterjoin%
\definecolor{currentfill}{rgb}{0.800000,0.800000,0.800000}%
\pgfsetfillcolor{currentfill}%
\pgfsetlinewidth{1.003750pt}%
\definecolor{currentstroke}{rgb}{0.000000,0.000000,0.000000}%
\pgfsetstrokecolor{currentstroke}%
\pgfsetdash{}{0pt}%
\pgfpathmoveto{\pgfqpoint{4.373178in}{2.438549in}}%
\pgfpathcurveto{\pgfqpoint{4.407900in}{2.403827in}}{\pgfqpoint{5.340564in}{2.403827in}}{\pgfqpoint{5.375286in}{2.438549in}}%
\pgfpathcurveto{\pgfqpoint{5.410008in}{2.473272in}}{\pgfqpoint{5.410008in}{2.596728in}}{\pgfqpoint{5.375286in}{2.631451in}}%
\pgfpathcurveto{\pgfqpoint{5.340564in}{2.666173in}}{\pgfqpoint{4.407900in}{2.666173in}}{\pgfqpoint{4.373178in}{2.631451in}}%
\pgfpathcurveto{\pgfqpoint{4.338456in}{2.596728in}}{\pgfqpoint{4.338456in}{2.473272in}}{\pgfqpoint{4.373178in}{2.438549in}}%
\pgfpathclose%
\pgfusepath{stroke,fill}%
\end{pgfscope}%
\begin{pgfscope}%
\definecolor{textcolor}{rgb}{0.000000,0.000000,0.000000}%
\pgfsetstrokecolor{textcolor}%
\pgfsetfillcolor{textcolor}%
\pgftext[x=5.340564in,y=2.535000in,right,]{\color{textcolor}\rmfamily\fontsize{10.000000}{12.000000}\selectfont \(\displaystyle M_u =\) 278.0 kft}%
\end{pgfscope}%
\begin{pgfscope}%
\pgfsetbuttcap%
\pgfsetmiterjoin%
\definecolor{currentfill}{rgb}{0.800000,0.800000,0.800000}%
\pgfsetfillcolor{currentfill}%
\pgfsetlinewidth{1.003750pt}%
\definecolor{currentstroke}{rgb}{0.000000,0.000000,0.000000}%
\pgfsetstrokecolor{currentstroke}%
\pgfsetdash{}{0pt}%
\pgfpathmoveto{\pgfqpoint{0.965278in}{0.358599in}}%
\pgfpathcurveto{\pgfqpoint{1.000000in}{0.323877in}}{\pgfqpoint{3.216826in}{0.323877in}}{\pgfqpoint{3.251548in}{0.358599in}}%
\pgfpathcurveto{\pgfqpoint{3.286270in}{0.393321in}}{\pgfqpoint{3.286270in}{0.668784in}}{\pgfqpoint{3.251548in}{0.703506in}}%
\pgfpathcurveto{\pgfqpoint{3.216826in}{0.738228in}}{\pgfqpoint{1.000000in}{0.738228in}}{\pgfqpoint{0.965278in}{0.703506in}}%
\pgfpathcurveto{\pgfqpoint{0.930556in}{0.668784in}}{\pgfqpoint{0.930556in}{0.393321in}}{\pgfqpoint{0.965278in}{0.358599in}}%
\pgfpathclose%
\pgfusepath{stroke,fill}%
\end{pgfscope}%
\begin{pgfscope}%
\definecolor{textcolor}{rgb}{0.000000,0.000000,0.000000}%
\pgfsetstrokecolor{textcolor}%
\pgfsetfillcolor{textcolor}%
\pgftext[x=1.000000in, y=0.580049in, left, base]{\color{textcolor}\rmfamily\fontsize{10.000000}{12.000000}\selectfont Max combo: 1.2D + 0.5Lr0 + 1.6L0}%
\end{pgfscope}%
\begin{pgfscope}%
\definecolor{textcolor}{rgb}{0.000000,0.000000,0.000000}%
\pgfsetstrokecolor{textcolor}%
\pgfsetfillcolor{textcolor}%
\pgftext[x=1.000000in, y=0.428043in, left, base]{\color{textcolor}\rmfamily\fontsize{10.000000}{12.000000}\selectfont ASCE7-16 Sec. 2.3.1 (LC 2)}%
\end{pgfscope}%
\end{pgfpicture}%
\makeatother%
\endgroup%

\end{center}
\caption{Moment Demand Envelope}
\end{figure}
L\textsubscript{p}, the limiting laterally unbraced length for the limit state of yielding, is calculated per AISC/ANSI 360-16 Eq. F2-5 as follows:
\begin{flalign*}
L_p = 1.76\cdot r_y \cdot \sqrt{\frac{E}{F_y}}  = 1.76\cdot 3.76 {\color{darkBlue}{\mathbf{ \; in}}} \cdot \sqrt{\frac{29000 {\color{darkBlue}{\mathbf{ \; ksi}}}}{50 {\color{darkBlue}{\mathbf{ \; ksi}}}}}  = \mathbf{13.3 {\color{darkBlue}{\mathbf{ \; ft}}}}
\end{flalign*}
r\textsubscript{ts}, a coefficient used in the calculation of L\textsubscript{r} and C\textsubscript{b}, is calculated per AISC/ANSI 360-16 Eq. F2-7 as follows:
\begin{flalign*}
r_{{ts}} = \sqrt{\frac{\sqrt{I_y \cdot C_w}}{S_x}}  = \sqrt{\frac{\sqrt{548 {\color{darkBlue}{\mathbf{ \; {\color{darkBlue}{\mathbf{ \; in}}}^{4}}}} \cdot 25500 {\color{darkBlue}{\mathbf{ \; {\color{darkBlue}{\mathbf{ \; in}}}^{6}}}}}}{209 {\color{darkBlue}{\mathbf{ \; {\color{darkBlue}{\mathbf{ \; in}}}^{3}}}}}}  = \mathbf{4.2 {\color{darkBlue}{\mathbf{ \; in}}}}
\end{flalign*}
L\textsubscript{r}, the limiting unbraced length for the limit state of inelastic lateral-torsional buckling, is calculated per AISC/ANSI 360-16 Eq. F2-6 as follows:
\begin{flalign*}
L_r &= 1.95\cdot r_{ts} \cdot \frac{E}{0.7\cdot F_y} \sqrt{\frac{J \cdot c}{S_x \cdot h_0}+\sqrt{{\left(\frac{J \cdot c}{S_x \cdot h_0}\right)}^2+6.76{\left(\frac{0.7\cdot F_y}{E}\right)}^2}} \\ &= 1.95\cdot 4.2 {\color{darkBlue}{\mathbf{ \; in}}} \cdot \frac{29000 {\color{darkBlue}{\mathbf{ \; ksi}}}}{0.7\cdot 50 {\color{darkBlue}{\mathbf{ \; ksi}}}} \sqrt{\frac{12.3 {\color{darkBlue}{\mathbf{ \; {\color{darkBlue}{\mathbf{ \; in}}}^{4}}}} \cdot 1}{209 {\color{darkBlue}{\mathbf{ \; {\color{darkBlue}{\mathbf{ \; in}}}^{3}}}} \cdot 13.7 {\color{darkBlue}{\mathbf{ \; in}}}}+\sqrt{{\left(\frac{12.3 {\color{darkBlue}{\mathbf{ \; {\color{darkBlue}{\mathbf{ \; in}}}^{4}}}} \cdot 1}{209 {\color{darkBlue}{\mathbf{ \; {\color{darkBlue}{\mathbf{ \; in}}}^{3}}}} \cdot 13.7 {\color{darkBlue}{\mathbf{ \; in}}}}\right)}^2+6.76{\left(\frac{0.7\cdot 50 {\color{darkBlue}{\mathbf{ \; ksi}}}}{29000 {\color{darkBlue}{\mathbf{ \; ksi}}}}\right)}^2}} \\ &= \mathbf{55.9 {\color{darkBlue}{\mathbf{ \; ft}}}}
\end{flalign*}
\textlambda\textsubscript{web}, the web width-to-thickness ratio, is calculated per {AISC/ANSI 360-16 Table B4.1b} as follows:
\begin{flalign*}
\lambda_{{web}} = \frac{h}{t_w}  = \frac{11.44 {\color{darkBlue}{\mathbf{ \; in}}}}{0.645 {\color{darkBlue}{\mathbf{ \; in}}}}  = \mathbf{17.7 }
\end{flalign*}
\textlambda\textsubscript{P-web}, the limiting width-to-thickness ratio for compact/noncompact web, is calculated per {AISC/ANSI 360-16 Table B4.1b} as follows:
\begin{flalign*}
\lambda_{P-web} = 3.76\cdot \sqrt{\frac{E}{F_y}} = 3.76\cdot \sqrt{\frac{29000 {\color{darkBlue}{\mathbf{ \; ksi}}}}{50 {\color{darkBlue}{\mathbf{ \; ksi}}}}} = \mathbf{90.6}
\end{flalign*}
\textlambda\textsubscript{R-web}, the limiting width-to-thickness ratio for noncompact/slender web, is calculated per {AISC/ANSI 360-16 Table B4.1b} as follows:
\begin{flalign*}
\lambda_{R-web} = 5.7\cdot \sqrt{\frac{E}{F_y}} = 5.7\cdot \sqrt{\frac{29000 {\color{darkBlue}{\mathbf{ \; ksi}}}}{50 {\color{darkBlue}{\mathbf{ \; ksi}}}}} = \mathbf{137.3}
\end{flalign*}
\textlambda\textsubscript{web} $<$ \textlambda\textsubscript{P-web} \textrightarrow \; \textbf{Compact Web}
\\\\
\textlambda\textsubscript{flange}, the flange width-to-thickness ratio, is calculated per {AISC/ANSI 360-16 Table B4.1b} as follows:
\begin{flalign*}
\lambda_{{flange}} = \frac{b}{t}  = \frac{7.35 {\color{darkBlue}{\mathbf{ \; in}}}}{1.03 {\color{darkBlue}{\mathbf{ \; in}}}}  = \mathbf{7.1 }
\end{flalign*}
\textlambda\textsubscript{P-flange}, the limiting width-to-thickness ratio for compact/noncompact flange, is calculated per {AISC/ANSI 360-16 Table B4.1b} as follows:
\begin{flalign*}
\lambda_{P-flange} = 0.38\cdot \sqrt{\frac{E}{F_y}} = 0.38\cdot \sqrt{\frac{29000 {\color{darkBlue}{\mathbf{ \; ksi}}}}{50 {\color{darkBlue}{\mathbf{ \; ksi}}}}} = \mathbf{9.2}
\end{flalign*}
\textlambda\textsubscript{R-flange}, the limiting width-to-thickness ratio for noncompact/slender flange, is calculated per {AISC/ANSI 360-16 Table B4.1b} as follows:
\begin{flalign*}
\lambda_{R-flange} = 1\cdot \sqrt{\frac{E}{F_y}} = 1\cdot \sqrt{\frac{29000 {\color{darkBlue}{\mathbf{ \; ksi}}}}{50 {\color{darkBlue}{\mathbf{ \; ksi}}}}} = \mathbf{24.1}
\end{flalign*}
\textlambda\textsubscript{flange} $<$ \textlambda\textsubscript{P-flange} \textrightarrow \; \textbf{Compact Flange}
\\\\
Since \(\mathbf{{L_p} < {L_b} <= {L_r}}\) and the beam's flanges are \textbf{compact}, controlling limit state for flexure is \textbf{LTB (not to exceed capacity based on yielding)}.
\\\\
M\textsubscript{p}, the plastic bending moment, is calculated per AISC/ANSI 360-16 Eq. F2-1 as follows:
\begin{flalign*}
M_p = F_y \cdot Z_x  = 50 {\color{darkBlue}{\mathbf{ \; ksi}}} \cdot 234 {\color{darkBlue}{\mathbf{ \; {\color{darkBlue}{\mathbf{ \; in}}}^{3}}}}  = \mathbf{975.0 {\color{darkBlue}{\mathbf{ \; kft}}}}
\end{flalign*}
C\textsubscript{b}, the lateral-torsional buckling modification factor in the critical unbraced span for the critical load combination, is calculated per AISC/ANSI 360- 16 Sec. F1 as follows:
\\
\begin{flalign*}
C_b &= \frac{12.5\cdot M_{max}}{2.5\cdot M_{max}+3\cdot M_A+4\cdot M_B+3\cdot M_C} \\ &= \frac{12.5\cdot 278.0 {\color{darkBlue}{\mathbf{ \; kft}}}}{2.5\cdot 278.0 {\color{darkBlue}{\mathbf{ \; kft}}}+3\cdot 149.0 {\color{darkBlue}{\mathbf{ \; kft}}}+4\cdot 264.7 {\color{darkBlue}{\mathbf{ \; kft}}}+3\cdot 137.1 {\color{darkBlue}{\mathbf{ \; kft}}}} \\ &= \mathbf{1.3 }
\end{flalign*}
\\
For brevity, the C\textsubscript{b} calculation is not shown for each span. The following figure illustrates the value of C\textsubscript{b} for each span.
\begin{figure}[H]
\begin{center}
%% Creator: Matplotlib, PGF backend
%%
%% To include the figure in your LaTeX document, write
%%   \input{<filename>.pgf}
%%
%% Make sure the required packages are loaded in your preamble
%%   \usepackage{pgf}
%%
%% Figures using additional raster images can only be included by \input if
%% they are in the same directory as the main LaTeX file. For loading figures
%% from other directories you can use the `import` package
%%   \usepackage{import}
%%
%% and then include the figures with
%%   \import{<path to file>}{<filename>.pgf}
%%
%% Matplotlib used the following preamble
%%
\begingroup%
\makeatletter%
\begin{pgfpicture}%
\pgfpathrectangle{\pgfpointorigin}{\pgfqpoint{8.000000in}{1.000000in}}%
\pgfusepath{use as bounding box, clip}%
\begin{pgfscope}%
\pgfpathrectangle{\pgfqpoint{1.000000in}{0.110000in}}{\pgfqpoint{6.200000in}{0.770000in}}%
\pgfusepath{clip}%
\pgfsetrectcap%
\pgfsetroundjoin%
\pgfsetlinewidth{0.752812pt}%
\definecolor{currentstroke}{rgb}{0.000000,0.000000,0.000000}%
\pgfsetstrokecolor{currentstroke}%
\pgfsetdash{}{0pt}%
\pgfpathmoveto{\pgfqpoint{1.281818in}{0.880000in}}%
\pgfpathlineto{\pgfqpoint{6.918182in}{0.880000in}}%
\pgfusepath{stroke}%
\end{pgfscope}%
\begin{pgfscope}%
\pgfpathrectangle{\pgfqpoint{1.000000in}{0.110000in}}{\pgfqpoint{6.200000in}{0.770000in}}%
\pgfusepath{clip}%
\pgfsetrectcap%
\pgfsetroundjoin%
\pgfsetlinewidth{0.752812pt}%
\definecolor{currentstroke}{rgb}{0.000000,0.000000,0.000000}%
\pgfsetstrokecolor{currentstroke}%
\pgfsetdash{}{0pt}%
\pgfpathmoveto{\pgfqpoint{1.281818in}{0.826048in}}%
\pgfpathlineto{\pgfqpoint{6.918182in}{0.826048in}}%
\pgfusepath{stroke}%
\end{pgfscope}%
\begin{pgfscope}%
\pgfpathrectangle{\pgfqpoint{1.000000in}{0.110000in}}{\pgfqpoint{6.200000in}{0.770000in}}%
\pgfusepath{clip}%
\pgfsetrectcap%
\pgfsetroundjoin%
\pgfsetlinewidth{0.752812pt}%
\definecolor{currentstroke}{rgb}{0.000000,0.000000,0.000000}%
\pgfsetstrokecolor{currentstroke}%
\pgfsetdash{}{0pt}%
\pgfpathmoveto{\pgfqpoint{1.281818in}{0.110000in}}%
\pgfpathlineto{\pgfqpoint{6.918182in}{0.110000in}}%
\pgfusepath{stroke}%
\end{pgfscope}%
\begin{pgfscope}%
\pgfpathrectangle{\pgfqpoint{1.000000in}{0.110000in}}{\pgfqpoint{6.200000in}{0.770000in}}%
\pgfusepath{clip}%
\pgfsetrectcap%
\pgfsetroundjoin%
\pgfsetlinewidth{0.752812pt}%
\definecolor{currentstroke}{rgb}{0.000000,0.000000,0.000000}%
\pgfsetstrokecolor{currentstroke}%
\pgfsetdash{}{0pt}%
\pgfpathmoveto{\pgfqpoint{1.281818in}{0.163952in}}%
\pgfpathlineto{\pgfqpoint{6.918182in}{0.163952in}}%
\pgfusepath{stroke}%
\end{pgfscope}%
\begin{pgfscope}%
\pgfpathrectangle{\pgfqpoint{1.000000in}{0.110000in}}{\pgfqpoint{6.200000in}{0.770000in}}%
\pgfusepath{clip}%
\pgfsetbuttcap%
\pgfsetroundjoin%
\pgfsetlinewidth{1.505625pt}%
\definecolor{currentstroke}{rgb}{1.000000,0.000000,0.000000}%
\pgfsetstrokecolor{currentstroke}%
\pgfsetdash{{1.500000pt}{2.475000pt}}{0.000000pt}%
\pgfpathmoveto{\pgfqpoint{1.281818in}{0.110000in}}%
\pgfpathlineto{\pgfqpoint{1.281818in}{0.880000in}}%
\pgfusepath{stroke}%
\end{pgfscope}%
\begin{pgfscope}%
\pgfpathrectangle{\pgfqpoint{1.000000in}{0.110000in}}{\pgfqpoint{6.200000in}{0.770000in}}%
\pgfusepath{clip}%
\pgfsetbuttcap%
\pgfsetroundjoin%
\pgfsetlinewidth{1.505625pt}%
\definecolor{currentstroke}{rgb}{1.000000,0.000000,0.000000}%
\pgfsetstrokecolor{currentstroke}%
\pgfsetdash{{1.500000pt}{2.475000pt}}{0.000000pt}%
\pgfpathmoveto{\pgfqpoint{6.918182in}{0.110000in}}%
\pgfpathlineto{\pgfqpoint{6.918182in}{0.880000in}}%
\pgfusepath{stroke}%
\end{pgfscope}%
\begin{pgfscope}%
\pgfsetbuttcap%
\pgfsetmiterjoin%
\definecolor{currentfill}{rgb}{0.800000,0.800000,0.800000}%
\pgfsetfillcolor{currentfill}%
\pgfsetlinewidth{1.003750pt}%
\definecolor{currentstroke}{rgb}{0.000000,0.000000,0.000000}%
\pgfsetstrokecolor{currentstroke}%
\pgfsetdash{}{0pt}%
\pgfpathmoveto{\pgfqpoint{3.757214in}{0.397240in}}%
\pgfpathcurveto{\pgfqpoint{3.791936in}{0.362518in}}{\pgfqpoint{4.408064in}{0.362518in}}{\pgfqpoint{4.442786in}{0.397240in}}%
\pgfpathcurveto{\pgfqpoint{4.477508in}{0.431962in}}{\pgfqpoint{4.477508in}{0.555419in}}{\pgfqpoint{4.442786in}{0.590141in}}%
\pgfpathcurveto{\pgfqpoint{4.408064in}{0.624863in}}{\pgfqpoint{3.791936in}{0.624863in}}{\pgfqpoint{3.757214in}{0.590141in}}%
\pgfpathcurveto{\pgfqpoint{3.722492in}{0.555419in}}{\pgfqpoint{3.722492in}{0.431962in}}{\pgfqpoint{3.757214in}{0.397240in}}%
\pgfpathclose%
\pgfusepath{stroke,fill}%
\end{pgfscope}%
\begin{pgfscope}%
\definecolor{textcolor}{rgb}{0.000000,0.000000,0.000000}%
\pgfsetstrokecolor{textcolor}%
\pgfsetfillcolor{textcolor}%
\pgftext[x=4.100000in,y=0.493690in,,]{\color{textcolor}\rmfamily\fontsize{10.000000}{12.000000}\selectfont C\textsubscript{b} = 1.33}%
\end{pgfscope}%
\end{pgfpicture}%
\makeatother%
\endgroup%

\end{center}
\caption{C\textsubscript{b} Along Member}
\end{figure}
F\textsubscript{cr}, the buckling stress for the critical section in the critical unbraced span, is calculated per AISC/ANSI 360- 16 Eq. F2-4 as follows:
\begin{flalign*}
F_{cr} & = \cfrac{C_b \cdot \pi^2 \cdot {{E}}} {\left(\cfrac{L_b}{r_{ts}}\right)^2} \cdot \sqrt{1 + 0.078 \cdot \cfrac{{J} \cdot {c}}{{S_x} \cdot h_0} \cdot \left(\cfrac{L_b}{r_{ts}}\right)^2} \\ & = \cfrac{1.33  \cdot \pi^2 \cdot {{29000 {\color{darkBlue}{\mathbf{ \; ksi}}}}}} {\left(\cfrac{19 {\color{darkBlue}{\mathbf{ \; ft}}}}{4.2 {\color{darkBlue}{\mathbf{ \; in}}}}\right)^2} \cdot \sqrt{1 + 0.078 \cdot \cfrac{{12.3 {\color{darkBlue}{\mathbf{ \; {\color{darkBlue}{\mathbf{ \; in}}}^{4}}}}} \cdot {1}}{{209 {\color{darkBlue}{\mathbf{ \; {\color{darkBlue}{\mathbf{ \; in}}}^{3}}}}} \cdot 13.7 {\color{darkBlue}{\mathbf{ \; in}}}} \cdot \left(\cfrac{19 {\color{darkBlue}{\mathbf{ \; ft}}}}{4.2 {\color{darkBlue}{\mathbf{ \; in}}}}\right)^2} = \mathbf{184.2 {\color{darkBlue}{\mathbf{ \; ksi}}}}
\end{flalign*}
\\
\textphi\textsubscript{b}, the resistance factor for bending, is determined per AISC/ANSI 360-16 {\S}F1a as \textbf{0.9}.
\\\\
\textphi\textsubscript{b}M\textsubscript{n}, the design flexural strength, is calculated per AISC/ANSI 360-16 Eq. F2-1 as follows:
\begin{flalign*}
{\phi_b}{M_{n}} & = {\phi_b} \cdot {C_b} \cdot {{M_p} - 0.7 \cdot {F_{y}} \cdot {S_{x}} \cdot \frac{{L_b} - {L_p}}{{L_r} - {L_p}}} < {\phi_b} \cdot {M_p} \\ & = {0.9} \cdot {1.33 } \cdot {{975.0 {\color{darkBlue}{\mathbf{ \; kft}}}} - 0.7 \cdot {50 {\color{darkBlue}{\mathbf{ \; ksi}}}} \cdot {209 {\color{darkBlue}{\mathbf{ \; {\color{darkBlue}{\mathbf{ \; in}}}^{3}}}}} \cdot \frac{{{19 {\color{darkBlue}{\mathbf{ \; ft}}}}} - {13.3 {\color{darkBlue}{\mathbf{ \; ft}}}}}{{55.9 {\color{darkBlue}{\mathbf{ \; ft}}}} - {13.3 {\color{darkBlue}{\mathbf{ \; ft}}}}}} < {0.9} \cdot {975.0 {\color{darkBlue}{\mathbf{ \; kft}}}} = \mathbf{877.5 {\color{darkBlue}{\mathbf{ \; kft}}}}
\end{flalign*}
\vspace{-20pt}
{\setlength{\mathindent}{0cm}
\begin{flalign*}
\mathbf{|M_u| = 278.0 {\color{darkBlue}{\mathbf{ \; kft}}}  \;  < \phi_b \cdot M_n = 877.5 {\color{darkBlue}{\mathbf{ \; kft}}}  \;  (DCR = 0.32 - OK)}
\end{flalign*}
\textbf{(yielding controls)}
%	-------------------------------- SHEAR CHECK ---------------------------------
\section{Shear Check}
\begin{figure}[H]
\begin{center}
%% Creator: Matplotlib, PGF backend
%%
%% To include the figure in your LaTeX document, write
%%   \input{<filename>.pgf}
%%
%% Make sure the required packages are loaded in your preamble
%%   \usepackage{pgf}
%%
%% Figures using additional raster images can only be included by \input if
%% they are in the same directory as the main LaTeX file. For loading figures
%% from other directories you can use the `import` package
%%   \usepackage{import}
%%
%% and then include the figures with
%%   \import{<path to file>}{<filename>.pgf}
%%
%% Matplotlib used the following preamble
%%
\begingroup%
\makeatletter%
\begin{pgfpicture}%
\pgfpathrectangle{\pgfpointorigin}{\pgfqpoint{8.000000in}{3.000000in}}%
\pgfusepath{use as bounding box, clip}%
\begin{pgfscope}%
\pgfsetbuttcap%
\pgfsetmiterjoin%
\definecolor{currentfill}{rgb}{1.000000,1.000000,1.000000}%
\pgfsetfillcolor{currentfill}%
\pgfsetlinewidth{0.000000pt}%
\definecolor{currentstroke}{rgb}{1.000000,1.000000,1.000000}%
\pgfsetstrokecolor{currentstroke}%
\pgfsetdash{}{0pt}%
\pgfpathmoveto{\pgfqpoint{0.000000in}{0.000000in}}%
\pgfpathlineto{\pgfqpoint{8.000000in}{0.000000in}}%
\pgfpathlineto{\pgfqpoint{8.000000in}{3.000000in}}%
\pgfpathlineto{\pgfqpoint{0.000000in}{3.000000in}}%
\pgfpathclose%
\pgfusepath{fill}%
\end{pgfscope}%
\begin{pgfscope}%
\pgfsetbuttcap%
\pgfsetmiterjoin%
\definecolor{currentfill}{rgb}{1.000000,1.000000,1.000000}%
\pgfsetfillcolor{currentfill}%
\pgfsetlinewidth{0.000000pt}%
\definecolor{currentstroke}{rgb}{0.000000,0.000000,0.000000}%
\pgfsetstrokecolor{currentstroke}%
\pgfsetstrokeopacity{0.000000}%
\pgfsetdash{}{0pt}%
\pgfpathmoveto{\pgfqpoint{1.000000in}{0.330000in}}%
\pgfpathlineto{\pgfqpoint{7.200000in}{0.330000in}}%
\pgfpathlineto{\pgfqpoint{7.200000in}{2.640000in}}%
\pgfpathlineto{\pgfqpoint{1.000000in}{2.640000in}}%
\pgfpathclose%
\pgfusepath{fill}%
\end{pgfscope}%
\begin{pgfscope}%
\pgfpathrectangle{\pgfqpoint{1.000000in}{0.330000in}}{\pgfqpoint{6.200000in}{2.310000in}}%
\pgfusepath{clip}%
\pgfsetbuttcap%
\pgfsetroundjoin%
\pgfsetlinewidth{0.803000pt}%
\definecolor{currentstroke}{rgb}{0.000000,0.000000,0.000000}%
\pgfsetstrokecolor{currentstroke}%
\pgfsetdash{{0.800000pt}{1.320000pt}}{0.000000pt}%
\pgfpathmoveto{\pgfqpoint{1.281818in}{0.330000in}}%
\pgfpathlineto{\pgfqpoint{1.281818in}{2.640000in}}%
\pgfusepath{stroke}%
\end{pgfscope}%
\begin{pgfscope}%
\pgfsetbuttcap%
\pgfsetroundjoin%
\definecolor{currentfill}{rgb}{0.000000,0.000000,0.000000}%
\pgfsetfillcolor{currentfill}%
\pgfsetlinewidth{0.803000pt}%
\definecolor{currentstroke}{rgb}{0.000000,0.000000,0.000000}%
\pgfsetstrokecolor{currentstroke}%
\pgfsetdash{}{0pt}%
\pgfsys@defobject{currentmarker}{\pgfqpoint{0.000000in}{-0.048611in}}{\pgfqpoint{0.000000in}{0.000000in}}{%
\pgfpathmoveto{\pgfqpoint{0.000000in}{0.000000in}}%
\pgfpathlineto{\pgfqpoint{0.000000in}{-0.048611in}}%
\pgfusepath{stroke,fill}%
}%
\begin{pgfscope}%
\pgfsys@transformshift{1.281818in}{0.330000in}%
\pgfsys@useobject{currentmarker}{}%
\end{pgfscope}%
\end{pgfscope}%
\begin{pgfscope}%
\pgfsetbuttcap%
\pgfsetroundjoin%
\definecolor{currentfill}{rgb}{0.000000,0.000000,0.000000}%
\pgfsetfillcolor{currentfill}%
\pgfsetlinewidth{0.803000pt}%
\definecolor{currentstroke}{rgb}{0.000000,0.000000,0.000000}%
\pgfsetstrokecolor{currentstroke}%
\pgfsetdash{}{0pt}%
\pgfsys@defobject{currentmarker}{\pgfqpoint{0.000000in}{0.000000in}}{\pgfqpoint{0.000000in}{0.048611in}}{%
\pgfpathmoveto{\pgfqpoint{0.000000in}{0.000000in}}%
\pgfpathlineto{\pgfqpoint{0.000000in}{0.048611in}}%
\pgfusepath{stroke,fill}%
}%
\begin{pgfscope}%
\pgfsys@transformshift{1.281818in}{2.640000in}%
\pgfsys@useobject{currentmarker}{}%
\end{pgfscope}%
\end{pgfscope}%
\begin{pgfscope}%
\definecolor{textcolor}{rgb}{0.000000,0.000000,0.000000}%
\pgfsetstrokecolor{textcolor}%
\pgfsetfillcolor{textcolor}%
\pgftext[x=1.281818in,y=0.232778in,,top]{\color{textcolor}\rmfamily\fontsize{10.000000}{12.000000}\selectfont \(\displaystyle {0.0}\)}%
\end{pgfscope}%
\begin{pgfscope}%
\pgfpathrectangle{\pgfqpoint{1.000000in}{0.330000in}}{\pgfqpoint{6.200000in}{2.310000in}}%
\pgfusepath{clip}%
\pgfsetbuttcap%
\pgfsetroundjoin%
\pgfsetlinewidth{0.803000pt}%
\definecolor{currentstroke}{rgb}{0.000000,0.000000,0.000000}%
\pgfsetstrokecolor{currentstroke}%
\pgfsetdash{{0.800000pt}{1.320000pt}}{0.000000pt}%
\pgfpathmoveto{\pgfqpoint{2.023445in}{0.330000in}}%
\pgfpathlineto{\pgfqpoint{2.023445in}{2.640000in}}%
\pgfusepath{stroke}%
\end{pgfscope}%
\begin{pgfscope}%
\pgfsetbuttcap%
\pgfsetroundjoin%
\definecolor{currentfill}{rgb}{0.000000,0.000000,0.000000}%
\pgfsetfillcolor{currentfill}%
\pgfsetlinewidth{0.803000pt}%
\definecolor{currentstroke}{rgb}{0.000000,0.000000,0.000000}%
\pgfsetstrokecolor{currentstroke}%
\pgfsetdash{}{0pt}%
\pgfsys@defobject{currentmarker}{\pgfqpoint{0.000000in}{-0.048611in}}{\pgfqpoint{0.000000in}{0.000000in}}{%
\pgfpathmoveto{\pgfqpoint{0.000000in}{0.000000in}}%
\pgfpathlineto{\pgfqpoint{0.000000in}{-0.048611in}}%
\pgfusepath{stroke,fill}%
}%
\begin{pgfscope}%
\pgfsys@transformshift{2.023445in}{0.330000in}%
\pgfsys@useobject{currentmarker}{}%
\end{pgfscope}%
\end{pgfscope}%
\begin{pgfscope}%
\pgfsetbuttcap%
\pgfsetroundjoin%
\definecolor{currentfill}{rgb}{0.000000,0.000000,0.000000}%
\pgfsetfillcolor{currentfill}%
\pgfsetlinewidth{0.803000pt}%
\definecolor{currentstroke}{rgb}{0.000000,0.000000,0.000000}%
\pgfsetstrokecolor{currentstroke}%
\pgfsetdash{}{0pt}%
\pgfsys@defobject{currentmarker}{\pgfqpoint{0.000000in}{0.000000in}}{\pgfqpoint{0.000000in}{0.048611in}}{%
\pgfpathmoveto{\pgfqpoint{0.000000in}{0.000000in}}%
\pgfpathlineto{\pgfqpoint{0.000000in}{0.048611in}}%
\pgfusepath{stroke,fill}%
}%
\begin{pgfscope}%
\pgfsys@transformshift{2.023445in}{2.640000in}%
\pgfsys@useobject{currentmarker}{}%
\end{pgfscope}%
\end{pgfscope}%
\begin{pgfscope}%
\definecolor{textcolor}{rgb}{0.000000,0.000000,0.000000}%
\pgfsetstrokecolor{textcolor}%
\pgfsetfillcolor{textcolor}%
\pgftext[x=2.023445in,y=0.232778in,,top]{\color{textcolor}\rmfamily\fontsize{10.000000}{12.000000}\selectfont \(\displaystyle {2.5}\)}%
\end{pgfscope}%
\begin{pgfscope}%
\pgfpathrectangle{\pgfqpoint{1.000000in}{0.330000in}}{\pgfqpoint{6.200000in}{2.310000in}}%
\pgfusepath{clip}%
\pgfsetbuttcap%
\pgfsetroundjoin%
\pgfsetlinewidth{0.803000pt}%
\definecolor{currentstroke}{rgb}{0.000000,0.000000,0.000000}%
\pgfsetstrokecolor{currentstroke}%
\pgfsetdash{{0.800000pt}{1.320000pt}}{0.000000pt}%
\pgfpathmoveto{\pgfqpoint{2.765072in}{0.330000in}}%
\pgfpathlineto{\pgfqpoint{2.765072in}{2.640000in}}%
\pgfusepath{stroke}%
\end{pgfscope}%
\begin{pgfscope}%
\pgfsetbuttcap%
\pgfsetroundjoin%
\definecolor{currentfill}{rgb}{0.000000,0.000000,0.000000}%
\pgfsetfillcolor{currentfill}%
\pgfsetlinewidth{0.803000pt}%
\definecolor{currentstroke}{rgb}{0.000000,0.000000,0.000000}%
\pgfsetstrokecolor{currentstroke}%
\pgfsetdash{}{0pt}%
\pgfsys@defobject{currentmarker}{\pgfqpoint{0.000000in}{-0.048611in}}{\pgfqpoint{0.000000in}{0.000000in}}{%
\pgfpathmoveto{\pgfqpoint{0.000000in}{0.000000in}}%
\pgfpathlineto{\pgfqpoint{0.000000in}{-0.048611in}}%
\pgfusepath{stroke,fill}%
}%
\begin{pgfscope}%
\pgfsys@transformshift{2.765072in}{0.330000in}%
\pgfsys@useobject{currentmarker}{}%
\end{pgfscope}%
\end{pgfscope}%
\begin{pgfscope}%
\pgfsetbuttcap%
\pgfsetroundjoin%
\definecolor{currentfill}{rgb}{0.000000,0.000000,0.000000}%
\pgfsetfillcolor{currentfill}%
\pgfsetlinewidth{0.803000pt}%
\definecolor{currentstroke}{rgb}{0.000000,0.000000,0.000000}%
\pgfsetstrokecolor{currentstroke}%
\pgfsetdash{}{0pt}%
\pgfsys@defobject{currentmarker}{\pgfqpoint{0.000000in}{0.000000in}}{\pgfqpoint{0.000000in}{0.048611in}}{%
\pgfpathmoveto{\pgfqpoint{0.000000in}{0.000000in}}%
\pgfpathlineto{\pgfqpoint{0.000000in}{0.048611in}}%
\pgfusepath{stroke,fill}%
}%
\begin{pgfscope}%
\pgfsys@transformshift{2.765072in}{2.640000in}%
\pgfsys@useobject{currentmarker}{}%
\end{pgfscope}%
\end{pgfscope}%
\begin{pgfscope}%
\definecolor{textcolor}{rgb}{0.000000,0.000000,0.000000}%
\pgfsetstrokecolor{textcolor}%
\pgfsetfillcolor{textcolor}%
\pgftext[x=2.765072in,y=0.232778in,,top]{\color{textcolor}\rmfamily\fontsize{10.000000}{12.000000}\selectfont \(\displaystyle {5.0}\)}%
\end{pgfscope}%
\begin{pgfscope}%
\pgfpathrectangle{\pgfqpoint{1.000000in}{0.330000in}}{\pgfqpoint{6.200000in}{2.310000in}}%
\pgfusepath{clip}%
\pgfsetbuttcap%
\pgfsetroundjoin%
\pgfsetlinewidth{0.803000pt}%
\definecolor{currentstroke}{rgb}{0.000000,0.000000,0.000000}%
\pgfsetstrokecolor{currentstroke}%
\pgfsetdash{{0.800000pt}{1.320000pt}}{0.000000pt}%
\pgfpathmoveto{\pgfqpoint{3.506699in}{0.330000in}}%
\pgfpathlineto{\pgfqpoint{3.506699in}{2.640000in}}%
\pgfusepath{stroke}%
\end{pgfscope}%
\begin{pgfscope}%
\pgfsetbuttcap%
\pgfsetroundjoin%
\definecolor{currentfill}{rgb}{0.000000,0.000000,0.000000}%
\pgfsetfillcolor{currentfill}%
\pgfsetlinewidth{0.803000pt}%
\definecolor{currentstroke}{rgb}{0.000000,0.000000,0.000000}%
\pgfsetstrokecolor{currentstroke}%
\pgfsetdash{}{0pt}%
\pgfsys@defobject{currentmarker}{\pgfqpoint{0.000000in}{-0.048611in}}{\pgfqpoint{0.000000in}{0.000000in}}{%
\pgfpathmoveto{\pgfqpoint{0.000000in}{0.000000in}}%
\pgfpathlineto{\pgfqpoint{0.000000in}{-0.048611in}}%
\pgfusepath{stroke,fill}%
}%
\begin{pgfscope}%
\pgfsys@transformshift{3.506699in}{0.330000in}%
\pgfsys@useobject{currentmarker}{}%
\end{pgfscope}%
\end{pgfscope}%
\begin{pgfscope}%
\pgfsetbuttcap%
\pgfsetroundjoin%
\definecolor{currentfill}{rgb}{0.000000,0.000000,0.000000}%
\pgfsetfillcolor{currentfill}%
\pgfsetlinewidth{0.803000pt}%
\definecolor{currentstroke}{rgb}{0.000000,0.000000,0.000000}%
\pgfsetstrokecolor{currentstroke}%
\pgfsetdash{}{0pt}%
\pgfsys@defobject{currentmarker}{\pgfqpoint{0.000000in}{0.000000in}}{\pgfqpoint{0.000000in}{0.048611in}}{%
\pgfpathmoveto{\pgfqpoint{0.000000in}{0.000000in}}%
\pgfpathlineto{\pgfqpoint{0.000000in}{0.048611in}}%
\pgfusepath{stroke,fill}%
}%
\begin{pgfscope}%
\pgfsys@transformshift{3.506699in}{2.640000in}%
\pgfsys@useobject{currentmarker}{}%
\end{pgfscope}%
\end{pgfscope}%
\begin{pgfscope}%
\definecolor{textcolor}{rgb}{0.000000,0.000000,0.000000}%
\pgfsetstrokecolor{textcolor}%
\pgfsetfillcolor{textcolor}%
\pgftext[x=3.506699in,y=0.232778in,,top]{\color{textcolor}\rmfamily\fontsize{10.000000}{12.000000}\selectfont \(\displaystyle {7.5}\)}%
\end{pgfscope}%
\begin{pgfscope}%
\pgfpathrectangle{\pgfqpoint{1.000000in}{0.330000in}}{\pgfqpoint{6.200000in}{2.310000in}}%
\pgfusepath{clip}%
\pgfsetbuttcap%
\pgfsetroundjoin%
\pgfsetlinewidth{0.803000pt}%
\definecolor{currentstroke}{rgb}{0.000000,0.000000,0.000000}%
\pgfsetstrokecolor{currentstroke}%
\pgfsetdash{{0.800000pt}{1.320000pt}}{0.000000pt}%
\pgfpathmoveto{\pgfqpoint{4.248325in}{0.330000in}}%
\pgfpathlineto{\pgfqpoint{4.248325in}{2.640000in}}%
\pgfusepath{stroke}%
\end{pgfscope}%
\begin{pgfscope}%
\pgfsetbuttcap%
\pgfsetroundjoin%
\definecolor{currentfill}{rgb}{0.000000,0.000000,0.000000}%
\pgfsetfillcolor{currentfill}%
\pgfsetlinewidth{0.803000pt}%
\definecolor{currentstroke}{rgb}{0.000000,0.000000,0.000000}%
\pgfsetstrokecolor{currentstroke}%
\pgfsetdash{}{0pt}%
\pgfsys@defobject{currentmarker}{\pgfqpoint{0.000000in}{-0.048611in}}{\pgfqpoint{0.000000in}{0.000000in}}{%
\pgfpathmoveto{\pgfqpoint{0.000000in}{0.000000in}}%
\pgfpathlineto{\pgfqpoint{0.000000in}{-0.048611in}}%
\pgfusepath{stroke,fill}%
}%
\begin{pgfscope}%
\pgfsys@transformshift{4.248325in}{0.330000in}%
\pgfsys@useobject{currentmarker}{}%
\end{pgfscope}%
\end{pgfscope}%
\begin{pgfscope}%
\pgfsetbuttcap%
\pgfsetroundjoin%
\definecolor{currentfill}{rgb}{0.000000,0.000000,0.000000}%
\pgfsetfillcolor{currentfill}%
\pgfsetlinewidth{0.803000pt}%
\definecolor{currentstroke}{rgb}{0.000000,0.000000,0.000000}%
\pgfsetstrokecolor{currentstroke}%
\pgfsetdash{}{0pt}%
\pgfsys@defobject{currentmarker}{\pgfqpoint{0.000000in}{0.000000in}}{\pgfqpoint{0.000000in}{0.048611in}}{%
\pgfpathmoveto{\pgfqpoint{0.000000in}{0.000000in}}%
\pgfpathlineto{\pgfqpoint{0.000000in}{0.048611in}}%
\pgfusepath{stroke,fill}%
}%
\begin{pgfscope}%
\pgfsys@transformshift{4.248325in}{2.640000in}%
\pgfsys@useobject{currentmarker}{}%
\end{pgfscope}%
\end{pgfscope}%
\begin{pgfscope}%
\definecolor{textcolor}{rgb}{0.000000,0.000000,0.000000}%
\pgfsetstrokecolor{textcolor}%
\pgfsetfillcolor{textcolor}%
\pgftext[x=4.248325in,y=0.232778in,,top]{\color{textcolor}\rmfamily\fontsize{10.000000}{12.000000}\selectfont \(\displaystyle {10.0}\)}%
\end{pgfscope}%
\begin{pgfscope}%
\pgfpathrectangle{\pgfqpoint{1.000000in}{0.330000in}}{\pgfqpoint{6.200000in}{2.310000in}}%
\pgfusepath{clip}%
\pgfsetbuttcap%
\pgfsetroundjoin%
\pgfsetlinewidth{0.803000pt}%
\definecolor{currentstroke}{rgb}{0.000000,0.000000,0.000000}%
\pgfsetstrokecolor{currentstroke}%
\pgfsetdash{{0.800000pt}{1.320000pt}}{0.000000pt}%
\pgfpathmoveto{\pgfqpoint{4.989952in}{0.330000in}}%
\pgfpathlineto{\pgfqpoint{4.989952in}{2.640000in}}%
\pgfusepath{stroke}%
\end{pgfscope}%
\begin{pgfscope}%
\pgfsetbuttcap%
\pgfsetroundjoin%
\definecolor{currentfill}{rgb}{0.000000,0.000000,0.000000}%
\pgfsetfillcolor{currentfill}%
\pgfsetlinewidth{0.803000pt}%
\definecolor{currentstroke}{rgb}{0.000000,0.000000,0.000000}%
\pgfsetstrokecolor{currentstroke}%
\pgfsetdash{}{0pt}%
\pgfsys@defobject{currentmarker}{\pgfqpoint{0.000000in}{-0.048611in}}{\pgfqpoint{0.000000in}{0.000000in}}{%
\pgfpathmoveto{\pgfqpoint{0.000000in}{0.000000in}}%
\pgfpathlineto{\pgfqpoint{0.000000in}{-0.048611in}}%
\pgfusepath{stroke,fill}%
}%
\begin{pgfscope}%
\pgfsys@transformshift{4.989952in}{0.330000in}%
\pgfsys@useobject{currentmarker}{}%
\end{pgfscope}%
\end{pgfscope}%
\begin{pgfscope}%
\pgfsetbuttcap%
\pgfsetroundjoin%
\definecolor{currentfill}{rgb}{0.000000,0.000000,0.000000}%
\pgfsetfillcolor{currentfill}%
\pgfsetlinewidth{0.803000pt}%
\definecolor{currentstroke}{rgb}{0.000000,0.000000,0.000000}%
\pgfsetstrokecolor{currentstroke}%
\pgfsetdash{}{0pt}%
\pgfsys@defobject{currentmarker}{\pgfqpoint{0.000000in}{0.000000in}}{\pgfqpoint{0.000000in}{0.048611in}}{%
\pgfpathmoveto{\pgfqpoint{0.000000in}{0.000000in}}%
\pgfpathlineto{\pgfqpoint{0.000000in}{0.048611in}}%
\pgfusepath{stroke,fill}%
}%
\begin{pgfscope}%
\pgfsys@transformshift{4.989952in}{2.640000in}%
\pgfsys@useobject{currentmarker}{}%
\end{pgfscope}%
\end{pgfscope}%
\begin{pgfscope}%
\definecolor{textcolor}{rgb}{0.000000,0.000000,0.000000}%
\pgfsetstrokecolor{textcolor}%
\pgfsetfillcolor{textcolor}%
\pgftext[x=4.989952in,y=0.232778in,,top]{\color{textcolor}\rmfamily\fontsize{10.000000}{12.000000}\selectfont \(\displaystyle {12.5}\)}%
\end{pgfscope}%
\begin{pgfscope}%
\pgfpathrectangle{\pgfqpoint{1.000000in}{0.330000in}}{\pgfqpoint{6.200000in}{2.310000in}}%
\pgfusepath{clip}%
\pgfsetbuttcap%
\pgfsetroundjoin%
\pgfsetlinewidth{0.803000pt}%
\definecolor{currentstroke}{rgb}{0.000000,0.000000,0.000000}%
\pgfsetstrokecolor{currentstroke}%
\pgfsetdash{{0.800000pt}{1.320000pt}}{0.000000pt}%
\pgfpathmoveto{\pgfqpoint{5.731579in}{0.330000in}}%
\pgfpathlineto{\pgfqpoint{5.731579in}{2.640000in}}%
\pgfusepath{stroke}%
\end{pgfscope}%
\begin{pgfscope}%
\pgfsetbuttcap%
\pgfsetroundjoin%
\definecolor{currentfill}{rgb}{0.000000,0.000000,0.000000}%
\pgfsetfillcolor{currentfill}%
\pgfsetlinewidth{0.803000pt}%
\definecolor{currentstroke}{rgb}{0.000000,0.000000,0.000000}%
\pgfsetstrokecolor{currentstroke}%
\pgfsetdash{}{0pt}%
\pgfsys@defobject{currentmarker}{\pgfqpoint{0.000000in}{-0.048611in}}{\pgfqpoint{0.000000in}{0.000000in}}{%
\pgfpathmoveto{\pgfqpoint{0.000000in}{0.000000in}}%
\pgfpathlineto{\pgfqpoint{0.000000in}{-0.048611in}}%
\pgfusepath{stroke,fill}%
}%
\begin{pgfscope}%
\pgfsys@transformshift{5.731579in}{0.330000in}%
\pgfsys@useobject{currentmarker}{}%
\end{pgfscope}%
\end{pgfscope}%
\begin{pgfscope}%
\pgfsetbuttcap%
\pgfsetroundjoin%
\definecolor{currentfill}{rgb}{0.000000,0.000000,0.000000}%
\pgfsetfillcolor{currentfill}%
\pgfsetlinewidth{0.803000pt}%
\definecolor{currentstroke}{rgb}{0.000000,0.000000,0.000000}%
\pgfsetstrokecolor{currentstroke}%
\pgfsetdash{}{0pt}%
\pgfsys@defobject{currentmarker}{\pgfqpoint{0.000000in}{0.000000in}}{\pgfqpoint{0.000000in}{0.048611in}}{%
\pgfpathmoveto{\pgfqpoint{0.000000in}{0.000000in}}%
\pgfpathlineto{\pgfqpoint{0.000000in}{0.048611in}}%
\pgfusepath{stroke,fill}%
}%
\begin{pgfscope}%
\pgfsys@transformshift{5.731579in}{2.640000in}%
\pgfsys@useobject{currentmarker}{}%
\end{pgfscope}%
\end{pgfscope}%
\begin{pgfscope}%
\definecolor{textcolor}{rgb}{0.000000,0.000000,0.000000}%
\pgfsetstrokecolor{textcolor}%
\pgfsetfillcolor{textcolor}%
\pgftext[x=5.731579in,y=0.232778in,,top]{\color{textcolor}\rmfamily\fontsize{10.000000}{12.000000}\selectfont \(\displaystyle {15.0}\)}%
\end{pgfscope}%
\begin{pgfscope}%
\pgfpathrectangle{\pgfqpoint{1.000000in}{0.330000in}}{\pgfqpoint{6.200000in}{2.310000in}}%
\pgfusepath{clip}%
\pgfsetbuttcap%
\pgfsetroundjoin%
\pgfsetlinewidth{0.803000pt}%
\definecolor{currentstroke}{rgb}{0.000000,0.000000,0.000000}%
\pgfsetstrokecolor{currentstroke}%
\pgfsetdash{{0.800000pt}{1.320000pt}}{0.000000pt}%
\pgfpathmoveto{\pgfqpoint{6.473206in}{0.330000in}}%
\pgfpathlineto{\pgfqpoint{6.473206in}{2.640000in}}%
\pgfusepath{stroke}%
\end{pgfscope}%
\begin{pgfscope}%
\pgfsetbuttcap%
\pgfsetroundjoin%
\definecolor{currentfill}{rgb}{0.000000,0.000000,0.000000}%
\pgfsetfillcolor{currentfill}%
\pgfsetlinewidth{0.803000pt}%
\definecolor{currentstroke}{rgb}{0.000000,0.000000,0.000000}%
\pgfsetstrokecolor{currentstroke}%
\pgfsetdash{}{0pt}%
\pgfsys@defobject{currentmarker}{\pgfqpoint{0.000000in}{-0.048611in}}{\pgfqpoint{0.000000in}{0.000000in}}{%
\pgfpathmoveto{\pgfqpoint{0.000000in}{0.000000in}}%
\pgfpathlineto{\pgfqpoint{0.000000in}{-0.048611in}}%
\pgfusepath{stroke,fill}%
}%
\begin{pgfscope}%
\pgfsys@transformshift{6.473206in}{0.330000in}%
\pgfsys@useobject{currentmarker}{}%
\end{pgfscope}%
\end{pgfscope}%
\begin{pgfscope}%
\pgfsetbuttcap%
\pgfsetroundjoin%
\definecolor{currentfill}{rgb}{0.000000,0.000000,0.000000}%
\pgfsetfillcolor{currentfill}%
\pgfsetlinewidth{0.803000pt}%
\definecolor{currentstroke}{rgb}{0.000000,0.000000,0.000000}%
\pgfsetstrokecolor{currentstroke}%
\pgfsetdash{}{0pt}%
\pgfsys@defobject{currentmarker}{\pgfqpoint{0.000000in}{0.000000in}}{\pgfqpoint{0.000000in}{0.048611in}}{%
\pgfpathmoveto{\pgfqpoint{0.000000in}{0.000000in}}%
\pgfpathlineto{\pgfqpoint{0.000000in}{0.048611in}}%
\pgfusepath{stroke,fill}%
}%
\begin{pgfscope}%
\pgfsys@transformshift{6.473206in}{2.640000in}%
\pgfsys@useobject{currentmarker}{}%
\end{pgfscope}%
\end{pgfscope}%
\begin{pgfscope}%
\definecolor{textcolor}{rgb}{0.000000,0.000000,0.000000}%
\pgfsetstrokecolor{textcolor}%
\pgfsetfillcolor{textcolor}%
\pgftext[x=6.473206in,y=0.232778in,,top]{\color{textcolor}\rmfamily\fontsize{10.000000}{12.000000}\selectfont \(\displaystyle {17.5}\)}%
\end{pgfscope}%
\begin{pgfscope}%
\pgfpathrectangle{\pgfqpoint{1.000000in}{0.330000in}}{\pgfqpoint{6.200000in}{2.310000in}}%
\pgfusepath{clip}%
\pgfsetbuttcap%
\pgfsetroundjoin%
\pgfsetlinewidth{0.803000pt}%
\definecolor{currentstroke}{rgb}{0.000000,0.000000,0.000000}%
\pgfsetstrokecolor{currentstroke}%
\pgfsetdash{{0.800000pt}{1.320000pt}}{0.000000pt}%
\pgfpathmoveto{\pgfqpoint{1.000000in}{0.414244in}}%
\pgfpathlineto{\pgfqpoint{7.200000in}{0.414244in}}%
\pgfusepath{stroke}%
\end{pgfscope}%
\begin{pgfscope}%
\pgfsetbuttcap%
\pgfsetroundjoin%
\definecolor{currentfill}{rgb}{0.000000,0.000000,0.000000}%
\pgfsetfillcolor{currentfill}%
\pgfsetlinewidth{0.803000pt}%
\definecolor{currentstroke}{rgb}{0.000000,0.000000,0.000000}%
\pgfsetstrokecolor{currentstroke}%
\pgfsetdash{}{0pt}%
\pgfsys@defobject{currentmarker}{\pgfqpoint{-0.048611in}{0.000000in}}{\pgfqpoint{-0.000000in}{0.000000in}}{%
\pgfpathmoveto{\pgfqpoint{-0.000000in}{0.000000in}}%
\pgfpathlineto{\pgfqpoint{-0.048611in}{0.000000in}}%
\pgfusepath{stroke,fill}%
}%
\begin{pgfscope}%
\pgfsys@transformshift{1.000000in}{0.414244in}%
\pgfsys@useobject{currentmarker}{}%
\end{pgfscope}%
\end{pgfscope}%
\begin{pgfscope}%
\pgfsetbuttcap%
\pgfsetroundjoin%
\definecolor{currentfill}{rgb}{0.000000,0.000000,0.000000}%
\pgfsetfillcolor{currentfill}%
\pgfsetlinewidth{0.803000pt}%
\definecolor{currentstroke}{rgb}{0.000000,0.000000,0.000000}%
\pgfsetstrokecolor{currentstroke}%
\pgfsetdash{}{0pt}%
\pgfsys@defobject{currentmarker}{\pgfqpoint{0.000000in}{0.000000in}}{\pgfqpoint{0.048611in}{0.000000in}}{%
\pgfpathmoveto{\pgfqpoint{0.000000in}{0.000000in}}%
\pgfpathlineto{\pgfqpoint{0.048611in}{0.000000in}}%
\pgfusepath{stroke,fill}%
}%
\begin{pgfscope}%
\pgfsys@transformshift{7.200000in}{0.414244in}%
\pgfsys@useobject{currentmarker}{}%
\end{pgfscope}%
\end{pgfscope}%
\begin{pgfscope}%
\definecolor{textcolor}{rgb}{0.000000,0.000000,0.000000}%
\pgfsetstrokecolor{textcolor}%
\pgfsetfillcolor{textcolor}%
\pgftext[x=0.655863in, y=0.366019in, left, base]{\color{textcolor}\rmfamily\fontsize{10.000000}{12.000000}\selectfont \(\displaystyle {\ensuremath{-}40}\)}%
\end{pgfscope}%
\begin{pgfscope}%
\pgfpathrectangle{\pgfqpoint{1.000000in}{0.330000in}}{\pgfqpoint{6.200000in}{2.310000in}}%
\pgfusepath{clip}%
\pgfsetbuttcap%
\pgfsetroundjoin%
\pgfsetlinewidth{0.803000pt}%
\definecolor{currentstroke}{rgb}{0.000000,0.000000,0.000000}%
\pgfsetstrokecolor{currentstroke}%
\pgfsetdash{{0.800000pt}{1.320000pt}}{0.000000pt}%
\pgfpathmoveto{\pgfqpoint{1.000000in}{0.709288in}}%
\pgfpathlineto{\pgfqpoint{7.200000in}{0.709288in}}%
\pgfusepath{stroke}%
\end{pgfscope}%
\begin{pgfscope}%
\pgfsetbuttcap%
\pgfsetroundjoin%
\definecolor{currentfill}{rgb}{0.000000,0.000000,0.000000}%
\pgfsetfillcolor{currentfill}%
\pgfsetlinewidth{0.803000pt}%
\definecolor{currentstroke}{rgb}{0.000000,0.000000,0.000000}%
\pgfsetstrokecolor{currentstroke}%
\pgfsetdash{}{0pt}%
\pgfsys@defobject{currentmarker}{\pgfqpoint{-0.048611in}{0.000000in}}{\pgfqpoint{-0.000000in}{0.000000in}}{%
\pgfpathmoveto{\pgfqpoint{-0.000000in}{0.000000in}}%
\pgfpathlineto{\pgfqpoint{-0.048611in}{0.000000in}}%
\pgfusepath{stroke,fill}%
}%
\begin{pgfscope}%
\pgfsys@transformshift{1.000000in}{0.709288in}%
\pgfsys@useobject{currentmarker}{}%
\end{pgfscope}%
\end{pgfscope}%
\begin{pgfscope}%
\pgfsetbuttcap%
\pgfsetroundjoin%
\definecolor{currentfill}{rgb}{0.000000,0.000000,0.000000}%
\pgfsetfillcolor{currentfill}%
\pgfsetlinewidth{0.803000pt}%
\definecolor{currentstroke}{rgb}{0.000000,0.000000,0.000000}%
\pgfsetstrokecolor{currentstroke}%
\pgfsetdash{}{0pt}%
\pgfsys@defobject{currentmarker}{\pgfqpoint{0.000000in}{0.000000in}}{\pgfqpoint{0.048611in}{0.000000in}}{%
\pgfpathmoveto{\pgfqpoint{0.000000in}{0.000000in}}%
\pgfpathlineto{\pgfqpoint{0.048611in}{0.000000in}}%
\pgfusepath{stroke,fill}%
}%
\begin{pgfscope}%
\pgfsys@transformshift{7.200000in}{0.709288in}%
\pgfsys@useobject{currentmarker}{}%
\end{pgfscope}%
\end{pgfscope}%
\begin{pgfscope}%
\definecolor{textcolor}{rgb}{0.000000,0.000000,0.000000}%
\pgfsetstrokecolor{textcolor}%
\pgfsetfillcolor{textcolor}%
\pgftext[x=0.655863in, y=0.661062in, left, base]{\color{textcolor}\rmfamily\fontsize{10.000000}{12.000000}\selectfont \(\displaystyle {\ensuremath{-}30}\)}%
\end{pgfscope}%
\begin{pgfscope}%
\pgfpathrectangle{\pgfqpoint{1.000000in}{0.330000in}}{\pgfqpoint{6.200000in}{2.310000in}}%
\pgfusepath{clip}%
\pgfsetbuttcap%
\pgfsetroundjoin%
\pgfsetlinewidth{0.803000pt}%
\definecolor{currentstroke}{rgb}{0.000000,0.000000,0.000000}%
\pgfsetstrokecolor{currentstroke}%
\pgfsetdash{{0.800000pt}{1.320000pt}}{0.000000pt}%
\pgfpathmoveto{\pgfqpoint{1.000000in}{1.004331in}}%
\pgfpathlineto{\pgfqpoint{7.200000in}{1.004331in}}%
\pgfusepath{stroke}%
\end{pgfscope}%
\begin{pgfscope}%
\pgfsetbuttcap%
\pgfsetroundjoin%
\definecolor{currentfill}{rgb}{0.000000,0.000000,0.000000}%
\pgfsetfillcolor{currentfill}%
\pgfsetlinewidth{0.803000pt}%
\definecolor{currentstroke}{rgb}{0.000000,0.000000,0.000000}%
\pgfsetstrokecolor{currentstroke}%
\pgfsetdash{}{0pt}%
\pgfsys@defobject{currentmarker}{\pgfqpoint{-0.048611in}{0.000000in}}{\pgfqpoint{-0.000000in}{0.000000in}}{%
\pgfpathmoveto{\pgfqpoint{-0.000000in}{0.000000in}}%
\pgfpathlineto{\pgfqpoint{-0.048611in}{0.000000in}}%
\pgfusepath{stroke,fill}%
}%
\begin{pgfscope}%
\pgfsys@transformshift{1.000000in}{1.004331in}%
\pgfsys@useobject{currentmarker}{}%
\end{pgfscope}%
\end{pgfscope}%
\begin{pgfscope}%
\pgfsetbuttcap%
\pgfsetroundjoin%
\definecolor{currentfill}{rgb}{0.000000,0.000000,0.000000}%
\pgfsetfillcolor{currentfill}%
\pgfsetlinewidth{0.803000pt}%
\definecolor{currentstroke}{rgb}{0.000000,0.000000,0.000000}%
\pgfsetstrokecolor{currentstroke}%
\pgfsetdash{}{0pt}%
\pgfsys@defobject{currentmarker}{\pgfqpoint{0.000000in}{0.000000in}}{\pgfqpoint{0.048611in}{0.000000in}}{%
\pgfpathmoveto{\pgfqpoint{0.000000in}{0.000000in}}%
\pgfpathlineto{\pgfqpoint{0.048611in}{0.000000in}}%
\pgfusepath{stroke,fill}%
}%
\begin{pgfscope}%
\pgfsys@transformshift{7.200000in}{1.004331in}%
\pgfsys@useobject{currentmarker}{}%
\end{pgfscope}%
\end{pgfscope}%
\begin{pgfscope}%
\definecolor{textcolor}{rgb}{0.000000,0.000000,0.000000}%
\pgfsetstrokecolor{textcolor}%
\pgfsetfillcolor{textcolor}%
\pgftext[x=0.655863in, y=0.956106in, left, base]{\color{textcolor}\rmfamily\fontsize{10.000000}{12.000000}\selectfont \(\displaystyle {\ensuremath{-}20}\)}%
\end{pgfscope}%
\begin{pgfscope}%
\pgfpathrectangle{\pgfqpoint{1.000000in}{0.330000in}}{\pgfqpoint{6.200000in}{2.310000in}}%
\pgfusepath{clip}%
\pgfsetbuttcap%
\pgfsetroundjoin%
\pgfsetlinewidth{0.803000pt}%
\definecolor{currentstroke}{rgb}{0.000000,0.000000,0.000000}%
\pgfsetstrokecolor{currentstroke}%
\pgfsetdash{{0.800000pt}{1.320000pt}}{0.000000pt}%
\pgfpathmoveto{\pgfqpoint{1.000000in}{1.299374in}}%
\pgfpathlineto{\pgfqpoint{7.200000in}{1.299374in}}%
\pgfusepath{stroke}%
\end{pgfscope}%
\begin{pgfscope}%
\pgfsetbuttcap%
\pgfsetroundjoin%
\definecolor{currentfill}{rgb}{0.000000,0.000000,0.000000}%
\pgfsetfillcolor{currentfill}%
\pgfsetlinewidth{0.803000pt}%
\definecolor{currentstroke}{rgb}{0.000000,0.000000,0.000000}%
\pgfsetstrokecolor{currentstroke}%
\pgfsetdash{}{0pt}%
\pgfsys@defobject{currentmarker}{\pgfqpoint{-0.048611in}{0.000000in}}{\pgfqpoint{-0.000000in}{0.000000in}}{%
\pgfpathmoveto{\pgfqpoint{-0.000000in}{0.000000in}}%
\pgfpathlineto{\pgfqpoint{-0.048611in}{0.000000in}}%
\pgfusepath{stroke,fill}%
}%
\begin{pgfscope}%
\pgfsys@transformshift{1.000000in}{1.299374in}%
\pgfsys@useobject{currentmarker}{}%
\end{pgfscope}%
\end{pgfscope}%
\begin{pgfscope}%
\pgfsetbuttcap%
\pgfsetroundjoin%
\definecolor{currentfill}{rgb}{0.000000,0.000000,0.000000}%
\pgfsetfillcolor{currentfill}%
\pgfsetlinewidth{0.803000pt}%
\definecolor{currentstroke}{rgb}{0.000000,0.000000,0.000000}%
\pgfsetstrokecolor{currentstroke}%
\pgfsetdash{}{0pt}%
\pgfsys@defobject{currentmarker}{\pgfqpoint{0.000000in}{0.000000in}}{\pgfqpoint{0.048611in}{0.000000in}}{%
\pgfpathmoveto{\pgfqpoint{0.000000in}{0.000000in}}%
\pgfpathlineto{\pgfqpoint{0.048611in}{0.000000in}}%
\pgfusepath{stroke,fill}%
}%
\begin{pgfscope}%
\pgfsys@transformshift{7.200000in}{1.299374in}%
\pgfsys@useobject{currentmarker}{}%
\end{pgfscope}%
\end{pgfscope}%
\begin{pgfscope}%
\definecolor{textcolor}{rgb}{0.000000,0.000000,0.000000}%
\pgfsetstrokecolor{textcolor}%
\pgfsetfillcolor{textcolor}%
\pgftext[x=0.655863in, y=1.251149in, left, base]{\color{textcolor}\rmfamily\fontsize{10.000000}{12.000000}\selectfont \(\displaystyle {\ensuremath{-}10}\)}%
\end{pgfscope}%
\begin{pgfscope}%
\pgfpathrectangle{\pgfqpoint{1.000000in}{0.330000in}}{\pgfqpoint{6.200000in}{2.310000in}}%
\pgfusepath{clip}%
\pgfsetbuttcap%
\pgfsetroundjoin%
\pgfsetlinewidth{0.803000pt}%
\definecolor{currentstroke}{rgb}{0.000000,0.000000,0.000000}%
\pgfsetstrokecolor{currentstroke}%
\pgfsetdash{{0.800000pt}{1.320000pt}}{0.000000pt}%
\pgfpathmoveto{\pgfqpoint{1.000000in}{1.594418in}}%
\pgfpathlineto{\pgfqpoint{7.200000in}{1.594418in}}%
\pgfusepath{stroke}%
\end{pgfscope}%
\begin{pgfscope}%
\pgfsetbuttcap%
\pgfsetroundjoin%
\definecolor{currentfill}{rgb}{0.000000,0.000000,0.000000}%
\pgfsetfillcolor{currentfill}%
\pgfsetlinewidth{0.803000pt}%
\definecolor{currentstroke}{rgb}{0.000000,0.000000,0.000000}%
\pgfsetstrokecolor{currentstroke}%
\pgfsetdash{}{0pt}%
\pgfsys@defobject{currentmarker}{\pgfqpoint{-0.048611in}{0.000000in}}{\pgfqpoint{-0.000000in}{0.000000in}}{%
\pgfpathmoveto{\pgfqpoint{-0.000000in}{0.000000in}}%
\pgfpathlineto{\pgfqpoint{-0.048611in}{0.000000in}}%
\pgfusepath{stroke,fill}%
}%
\begin{pgfscope}%
\pgfsys@transformshift{1.000000in}{1.594418in}%
\pgfsys@useobject{currentmarker}{}%
\end{pgfscope}%
\end{pgfscope}%
\begin{pgfscope}%
\pgfsetbuttcap%
\pgfsetroundjoin%
\definecolor{currentfill}{rgb}{0.000000,0.000000,0.000000}%
\pgfsetfillcolor{currentfill}%
\pgfsetlinewidth{0.803000pt}%
\definecolor{currentstroke}{rgb}{0.000000,0.000000,0.000000}%
\pgfsetstrokecolor{currentstroke}%
\pgfsetdash{}{0pt}%
\pgfsys@defobject{currentmarker}{\pgfqpoint{0.000000in}{0.000000in}}{\pgfqpoint{0.048611in}{0.000000in}}{%
\pgfpathmoveto{\pgfqpoint{0.000000in}{0.000000in}}%
\pgfpathlineto{\pgfqpoint{0.048611in}{0.000000in}}%
\pgfusepath{stroke,fill}%
}%
\begin{pgfscope}%
\pgfsys@transformshift{7.200000in}{1.594418in}%
\pgfsys@useobject{currentmarker}{}%
\end{pgfscope}%
\end{pgfscope}%
\begin{pgfscope}%
\definecolor{textcolor}{rgb}{0.000000,0.000000,0.000000}%
\pgfsetstrokecolor{textcolor}%
\pgfsetfillcolor{textcolor}%
\pgftext[x=0.833333in, y=1.546193in, left, base]{\color{textcolor}\rmfamily\fontsize{10.000000}{12.000000}\selectfont \(\displaystyle {0}\)}%
\end{pgfscope}%
\begin{pgfscope}%
\pgfpathrectangle{\pgfqpoint{1.000000in}{0.330000in}}{\pgfqpoint{6.200000in}{2.310000in}}%
\pgfusepath{clip}%
\pgfsetbuttcap%
\pgfsetroundjoin%
\pgfsetlinewidth{0.803000pt}%
\definecolor{currentstroke}{rgb}{0.000000,0.000000,0.000000}%
\pgfsetstrokecolor{currentstroke}%
\pgfsetdash{{0.800000pt}{1.320000pt}}{0.000000pt}%
\pgfpathmoveto{\pgfqpoint{1.000000in}{1.889461in}}%
\pgfpathlineto{\pgfqpoint{7.200000in}{1.889461in}}%
\pgfusepath{stroke}%
\end{pgfscope}%
\begin{pgfscope}%
\pgfsetbuttcap%
\pgfsetroundjoin%
\definecolor{currentfill}{rgb}{0.000000,0.000000,0.000000}%
\pgfsetfillcolor{currentfill}%
\pgfsetlinewidth{0.803000pt}%
\definecolor{currentstroke}{rgb}{0.000000,0.000000,0.000000}%
\pgfsetstrokecolor{currentstroke}%
\pgfsetdash{}{0pt}%
\pgfsys@defobject{currentmarker}{\pgfqpoint{-0.048611in}{0.000000in}}{\pgfqpoint{-0.000000in}{0.000000in}}{%
\pgfpathmoveto{\pgfqpoint{-0.000000in}{0.000000in}}%
\pgfpathlineto{\pgfqpoint{-0.048611in}{0.000000in}}%
\pgfusepath{stroke,fill}%
}%
\begin{pgfscope}%
\pgfsys@transformshift{1.000000in}{1.889461in}%
\pgfsys@useobject{currentmarker}{}%
\end{pgfscope}%
\end{pgfscope}%
\begin{pgfscope}%
\pgfsetbuttcap%
\pgfsetroundjoin%
\definecolor{currentfill}{rgb}{0.000000,0.000000,0.000000}%
\pgfsetfillcolor{currentfill}%
\pgfsetlinewidth{0.803000pt}%
\definecolor{currentstroke}{rgb}{0.000000,0.000000,0.000000}%
\pgfsetstrokecolor{currentstroke}%
\pgfsetdash{}{0pt}%
\pgfsys@defobject{currentmarker}{\pgfqpoint{0.000000in}{0.000000in}}{\pgfqpoint{0.048611in}{0.000000in}}{%
\pgfpathmoveto{\pgfqpoint{0.000000in}{0.000000in}}%
\pgfpathlineto{\pgfqpoint{0.048611in}{0.000000in}}%
\pgfusepath{stroke,fill}%
}%
\begin{pgfscope}%
\pgfsys@transformshift{7.200000in}{1.889461in}%
\pgfsys@useobject{currentmarker}{}%
\end{pgfscope}%
\end{pgfscope}%
\begin{pgfscope}%
\definecolor{textcolor}{rgb}{0.000000,0.000000,0.000000}%
\pgfsetstrokecolor{textcolor}%
\pgfsetfillcolor{textcolor}%
\pgftext[x=0.763888in, y=1.841236in, left, base]{\color{textcolor}\rmfamily\fontsize{10.000000}{12.000000}\selectfont \(\displaystyle {10}\)}%
\end{pgfscope}%
\begin{pgfscope}%
\pgfpathrectangle{\pgfqpoint{1.000000in}{0.330000in}}{\pgfqpoint{6.200000in}{2.310000in}}%
\pgfusepath{clip}%
\pgfsetbuttcap%
\pgfsetroundjoin%
\pgfsetlinewidth{0.803000pt}%
\definecolor{currentstroke}{rgb}{0.000000,0.000000,0.000000}%
\pgfsetstrokecolor{currentstroke}%
\pgfsetdash{{0.800000pt}{1.320000pt}}{0.000000pt}%
\pgfpathmoveto{\pgfqpoint{1.000000in}{2.184505in}}%
\pgfpathlineto{\pgfqpoint{7.200000in}{2.184505in}}%
\pgfusepath{stroke}%
\end{pgfscope}%
\begin{pgfscope}%
\pgfsetbuttcap%
\pgfsetroundjoin%
\definecolor{currentfill}{rgb}{0.000000,0.000000,0.000000}%
\pgfsetfillcolor{currentfill}%
\pgfsetlinewidth{0.803000pt}%
\definecolor{currentstroke}{rgb}{0.000000,0.000000,0.000000}%
\pgfsetstrokecolor{currentstroke}%
\pgfsetdash{}{0pt}%
\pgfsys@defobject{currentmarker}{\pgfqpoint{-0.048611in}{0.000000in}}{\pgfqpoint{-0.000000in}{0.000000in}}{%
\pgfpathmoveto{\pgfqpoint{-0.000000in}{0.000000in}}%
\pgfpathlineto{\pgfqpoint{-0.048611in}{0.000000in}}%
\pgfusepath{stroke,fill}%
}%
\begin{pgfscope}%
\pgfsys@transformshift{1.000000in}{2.184505in}%
\pgfsys@useobject{currentmarker}{}%
\end{pgfscope}%
\end{pgfscope}%
\begin{pgfscope}%
\pgfsetbuttcap%
\pgfsetroundjoin%
\definecolor{currentfill}{rgb}{0.000000,0.000000,0.000000}%
\pgfsetfillcolor{currentfill}%
\pgfsetlinewidth{0.803000pt}%
\definecolor{currentstroke}{rgb}{0.000000,0.000000,0.000000}%
\pgfsetstrokecolor{currentstroke}%
\pgfsetdash{}{0pt}%
\pgfsys@defobject{currentmarker}{\pgfqpoint{0.000000in}{0.000000in}}{\pgfqpoint{0.048611in}{0.000000in}}{%
\pgfpathmoveto{\pgfqpoint{0.000000in}{0.000000in}}%
\pgfpathlineto{\pgfqpoint{0.048611in}{0.000000in}}%
\pgfusepath{stroke,fill}%
}%
\begin{pgfscope}%
\pgfsys@transformshift{7.200000in}{2.184505in}%
\pgfsys@useobject{currentmarker}{}%
\end{pgfscope}%
\end{pgfscope}%
\begin{pgfscope}%
\definecolor{textcolor}{rgb}{0.000000,0.000000,0.000000}%
\pgfsetstrokecolor{textcolor}%
\pgfsetfillcolor{textcolor}%
\pgftext[x=0.763888in, y=2.136280in, left, base]{\color{textcolor}\rmfamily\fontsize{10.000000}{12.000000}\selectfont \(\displaystyle {20}\)}%
\end{pgfscope}%
\begin{pgfscope}%
\pgfpathrectangle{\pgfqpoint{1.000000in}{0.330000in}}{\pgfqpoint{6.200000in}{2.310000in}}%
\pgfusepath{clip}%
\pgfsetbuttcap%
\pgfsetroundjoin%
\pgfsetlinewidth{0.803000pt}%
\definecolor{currentstroke}{rgb}{0.000000,0.000000,0.000000}%
\pgfsetstrokecolor{currentstroke}%
\pgfsetdash{{0.800000pt}{1.320000pt}}{0.000000pt}%
\pgfpathmoveto{\pgfqpoint{1.000000in}{2.479548in}}%
\pgfpathlineto{\pgfqpoint{7.200000in}{2.479548in}}%
\pgfusepath{stroke}%
\end{pgfscope}%
\begin{pgfscope}%
\pgfsetbuttcap%
\pgfsetroundjoin%
\definecolor{currentfill}{rgb}{0.000000,0.000000,0.000000}%
\pgfsetfillcolor{currentfill}%
\pgfsetlinewidth{0.803000pt}%
\definecolor{currentstroke}{rgb}{0.000000,0.000000,0.000000}%
\pgfsetstrokecolor{currentstroke}%
\pgfsetdash{}{0pt}%
\pgfsys@defobject{currentmarker}{\pgfqpoint{-0.048611in}{0.000000in}}{\pgfqpoint{-0.000000in}{0.000000in}}{%
\pgfpathmoveto{\pgfqpoint{-0.000000in}{0.000000in}}%
\pgfpathlineto{\pgfqpoint{-0.048611in}{0.000000in}}%
\pgfusepath{stroke,fill}%
}%
\begin{pgfscope}%
\pgfsys@transformshift{1.000000in}{2.479548in}%
\pgfsys@useobject{currentmarker}{}%
\end{pgfscope}%
\end{pgfscope}%
\begin{pgfscope}%
\pgfsetbuttcap%
\pgfsetroundjoin%
\definecolor{currentfill}{rgb}{0.000000,0.000000,0.000000}%
\pgfsetfillcolor{currentfill}%
\pgfsetlinewidth{0.803000pt}%
\definecolor{currentstroke}{rgb}{0.000000,0.000000,0.000000}%
\pgfsetstrokecolor{currentstroke}%
\pgfsetdash{}{0pt}%
\pgfsys@defobject{currentmarker}{\pgfqpoint{0.000000in}{0.000000in}}{\pgfqpoint{0.048611in}{0.000000in}}{%
\pgfpathmoveto{\pgfqpoint{0.000000in}{0.000000in}}%
\pgfpathlineto{\pgfqpoint{0.048611in}{0.000000in}}%
\pgfusepath{stroke,fill}%
}%
\begin{pgfscope}%
\pgfsys@transformshift{7.200000in}{2.479548in}%
\pgfsys@useobject{currentmarker}{}%
\end{pgfscope}%
\end{pgfscope}%
\begin{pgfscope}%
\definecolor{textcolor}{rgb}{0.000000,0.000000,0.000000}%
\pgfsetstrokecolor{textcolor}%
\pgfsetfillcolor{textcolor}%
\pgftext[x=0.763888in, y=2.431323in, left, base]{\color{textcolor}\rmfamily\fontsize{10.000000}{12.000000}\selectfont \(\displaystyle {30}\)}%
\end{pgfscope}%
\begin{pgfscope}%
\pgfpathrectangle{\pgfqpoint{1.000000in}{0.330000in}}{\pgfqpoint{6.200000in}{2.310000in}}%
\pgfusepath{clip}%
\pgfsetrectcap%
\pgfsetroundjoin%
\pgfsetlinewidth{1.505625pt}%
\definecolor{currentstroke}{rgb}{0.121569,0.466667,0.705882}%
\pgfsetstrokecolor{currentstroke}%
\pgfsetdash{}{0pt}%
\pgfpathmoveto{\pgfqpoint{1.281818in}{1.594418in}}%
\pgfpathlineto{\pgfqpoint{1.281818in}{1.880316in}}%
\pgfpathlineto{\pgfqpoint{3.902233in}{1.845740in}}%
\pgfpathlineto{\pgfqpoint{3.951675in}{1.372428in}}%
\pgfpathlineto{\pgfqpoint{6.918182in}{1.333285in}}%
\pgfpathlineto{\pgfqpoint{6.918182in}{1.594418in}}%
\pgfpathlineto{\pgfqpoint{6.918182in}{1.594418in}}%
\pgfusepath{stroke}%
\end{pgfscope}%
\begin{pgfscope}%
\pgfpathrectangle{\pgfqpoint{1.000000in}{0.330000in}}{\pgfqpoint{6.200000in}{2.310000in}}%
\pgfusepath{clip}%
\pgfsetrectcap%
\pgfsetroundjoin%
\pgfsetlinewidth{1.505625pt}%
\definecolor{currentstroke}{rgb}{1.000000,0.498039,0.054902}%
\pgfsetstrokecolor{currentstroke}%
\pgfsetdash{}{0pt}%
\pgfpathmoveto{\pgfqpoint{1.281818in}{1.594418in}}%
\pgfpathlineto{\pgfqpoint{1.281818in}{2.330181in}}%
\pgfpathlineto{\pgfqpoint{3.902233in}{2.277094in}}%
\pgfpathlineto{\pgfqpoint{3.951675in}{0.986457in}}%
\pgfpathlineto{\pgfqpoint{6.918182in}{0.926359in}}%
\pgfpathlineto{\pgfqpoint{6.918182in}{1.594418in}}%
\pgfpathlineto{\pgfqpoint{6.918182in}{1.594418in}}%
\pgfusepath{stroke}%
\end{pgfscope}%
\begin{pgfscope}%
\pgfpathrectangle{\pgfqpoint{1.000000in}{0.330000in}}{\pgfqpoint{6.200000in}{2.310000in}}%
\pgfusepath{clip}%
\pgfsetrectcap%
\pgfsetroundjoin%
\pgfsetlinewidth{1.505625pt}%
\definecolor{currentstroke}{rgb}{0.172549,0.627451,0.172549}%
\pgfsetstrokecolor{currentstroke}%
\pgfsetdash{}{0pt}%
\pgfpathmoveto{\pgfqpoint{1.281818in}{1.594418in}}%
\pgfpathlineto{\pgfqpoint{1.281818in}{2.152570in}}%
\pgfpathlineto{\pgfqpoint{3.902233in}{2.095292in}}%
\pgfpathlineto{\pgfqpoint{3.951675in}{0.702491in}}%
\pgfpathlineto{\pgfqpoint{5.533812in}{0.667908in}}%
\pgfpathlineto{\pgfqpoint{5.583254in}{1.662599in}}%
\pgfpathlineto{\pgfqpoint{6.918182in}{1.633420in}}%
\pgfpathlineto{\pgfqpoint{6.918182in}{1.594418in}}%
\pgfpathlineto{\pgfqpoint{6.918182in}{1.594418in}}%
\pgfusepath{stroke}%
\end{pgfscope}%
\begin{pgfscope}%
\pgfpathrectangle{\pgfqpoint{1.000000in}{0.330000in}}{\pgfqpoint{6.200000in}{2.310000in}}%
\pgfusepath{clip}%
\pgfsetrectcap%
\pgfsetroundjoin%
\pgfsetlinewidth{1.505625pt}%
\definecolor{currentstroke}{rgb}{0.839216,0.152941,0.156863}%
\pgfsetstrokecolor{currentstroke}%
\pgfsetdash{}{0pt}%
\pgfpathmoveto{\pgfqpoint{1.281818in}{1.594418in}}%
\pgfpathlineto{\pgfqpoint{1.281818in}{2.316981in}}%
\pgfpathlineto{\pgfqpoint{3.902233in}{2.263895in}}%
\pgfpathlineto{\pgfqpoint{3.951675in}{0.998337in}}%
\pgfpathlineto{\pgfqpoint{6.918182in}{0.938239in}}%
\pgfpathlineto{\pgfqpoint{6.918182in}{1.594418in}}%
\pgfpathlineto{\pgfqpoint{6.918182in}{1.594418in}}%
\pgfusepath{stroke}%
\end{pgfscope}%
\begin{pgfscope}%
\pgfpathrectangle{\pgfqpoint{1.000000in}{0.330000in}}{\pgfqpoint{6.200000in}{2.310000in}}%
\pgfusepath{clip}%
\pgfsetrectcap%
\pgfsetroundjoin%
\pgfsetlinewidth{1.505625pt}%
\definecolor{currentstroke}{rgb}{0.580392,0.403922,0.741176}%
\pgfsetstrokecolor{currentstroke}%
\pgfsetdash{}{0pt}%
\pgfpathmoveto{\pgfqpoint{1.281818in}{1.594418in}}%
\pgfpathlineto{\pgfqpoint{1.281818in}{2.039149in}}%
\pgfpathlineto{\pgfqpoint{3.902233in}{1.985363in}}%
\pgfpathlineto{\pgfqpoint{3.951675in}{1.249100in}}%
\pgfpathlineto{\pgfqpoint{6.918182in}{1.188211in}}%
\pgfpathlineto{\pgfqpoint{6.918182in}{1.594418in}}%
\pgfpathlineto{\pgfqpoint{6.918182in}{1.594418in}}%
\pgfusepath{stroke}%
\end{pgfscope}%
\begin{pgfscope}%
\pgfpathrectangle{\pgfqpoint{1.000000in}{0.330000in}}{\pgfqpoint{6.200000in}{2.310000in}}%
\pgfusepath{clip}%
\pgfsetrectcap%
\pgfsetroundjoin%
\pgfsetlinewidth{1.505625pt}%
\definecolor{currentstroke}{rgb}{0.549020,0.337255,0.294118}%
\pgfsetstrokecolor{currentstroke}%
\pgfsetdash{}{0pt}%
\pgfpathmoveto{\pgfqpoint{1.281818in}{1.594418in}}%
\pgfpathlineto{\pgfqpoint{1.281818in}{1.874045in}}%
\pgfpathlineto{\pgfqpoint{3.902233in}{1.847152in}}%
\pgfpathlineto{\pgfqpoint{3.951675in}{1.818321in}}%
\pgfpathlineto{\pgfqpoint{5.533812in}{1.802084in}}%
\pgfpathlineto{\pgfqpoint{5.583254in}{0.805804in}}%
\pgfpathlineto{\pgfqpoint{6.918182in}{0.792104in}}%
\pgfpathlineto{\pgfqpoint{6.918182in}{1.594418in}}%
\pgfpathlineto{\pgfqpoint{6.918182in}{1.594418in}}%
\pgfusepath{stroke}%
\end{pgfscope}%
\begin{pgfscope}%
\pgfpathrectangle{\pgfqpoint{1.000000in}{0.330000in}}{\pgfqpoint{6.200000in}{2.310000in}}%
\pgfusepath{clip}%
\pgfsetrectcap%
\pgfsetroundjoin%
\pgfsetlinewidth{1.505625pt}%
\definecolor{currentstroke}{rgb}{0.890196,0.466667,0.760784}%
\pgfsetstrokecolor{currentstroke}%
\pgfsetdash{}{0pt}%
\pgfpathmoveto{\pgfqpoint{1.281818in}{1.594418in}}%
\pgfpathlineto{\pgfqpoint{1.281818in}{1.975616in}}%
\pgfpathlineto{\pgfqpoint{3.902233in}{1.929514in}}%
\pgfpathlineto{\pgfqpoint{3.951675in}{1.298431in}}%
\pgfpathlineto{\pgfqpoint{6.918182in}{1.246240in}}%
\pgfpathlineto{\pgfqpoint{6.918182in}{1.594418in}}%
\pgfpathlineto{\pgfqpoint{6.918182in}{1.594418in}}%
\pgfusepath{stroke}%
\end{pgfscope}%
\begin{pgfscope}%
\pgfpathrectangle{\pgfqpoint{1.000000in}{0.330000in}}{\pgfqpoint{6.200000in}{2.310000in}}%
\pgfusepath{clip}%
\pgfsetrectcap%
\pgfsetroundjoin%
\pgfsetlinewidth{1.505625pt}%
\definecolor{currentstroke}{rgb}{0.498039,0.498039,0.498039}%
\pgfsetstrokecolor{currentstroke}%
\pgfsetdash{}{0pt}%
\pgfpathmoveto{\pgfqpoint{1.281818in}{1.594418in}}%
\pgfpathlineto{\pgfqpoint{1.281818in}{2.359219in}}%
\pgfpathlineto{\pgfqpoint{3.902233in}{2.306133in}}%
\pgfpathlineto{\pgfqpoint{3.951675in}{0.960323in}}%
\pgfpathlineto{\pgfqpoint{6.918182in}{0.900225in}}%
\pgfpathlineto{\pgfqpoint{6.918182in}{1.594418in}}%
\pgfpathlineto{\pgfqpoint{6.918182in}{1.594418in}}%
\pgfusepath{stroke}%
\end{pgfscope}%
\begin{pgfscope}%
\pgfpathrectangle{\pgfqpoint{1.000000in}{0.330000in}}{\pgfqpoint{6.200000in}{2.310000in}}%
\pgfusepath{clip}%
\pgfsetrectcap%
\pgfsetroundjoin%
\pgfsetlinewidth{1.505625pt}%
\definecolor{currentstroke}{rgb}{0.737255,0.741176,0.133333}%
\pgfsetstrokecolor{currentstroke}%
\pgfsetdash{}{0pt}%
\pgfpathmoveto{\pgfqpoint{1.281818in}{1.594418in}}%
\pgfpathlineto{\pgfqpoint{1.281818in}{2.535000in}}%
\pgfpathlineto{\pgfqpoint{3.902233in}{2.477723in}}%
\pgfpathlineto{\pgfqpoint{3.951675in}{0.806401in}}%
\pgfpathlineto{\pgfqpoint{6.918182in}{0.741559in}}%
\pgfpathlineto{\pgfqpoint{6.918182in}{1.594418in}}%
\pgfpathlineto{\pgfqpoint{6.918182in}{1.594418in}}%
\pgfusepath{stroke}%
\end{pgfscope}%
\begin{pgfscope}%
\pgfpathrectangle{\pgfqpoint{1.000000in}{0.330000in}}{\pgfqpoint{6.200000in}{2.310000in}}%
\pgfusepath{clip}%
\pgfsetrectcap%
\pgfsetroundjoin%
\pgfsetlinewidth{1.505625pt}%
\definecolor{currentstroke}{rgb}{0.090196,0.745098,0.811765}%
\pgfsetstrokecolor{currentstroke}%
\pgfsetdash{}{0pt}%
\pgfpathmoveto{\pgfqpoint{1.281818in}{1.594418in}}%
\pgfpathlineto{\pgfqpoint{1.281818in}{2.017854in}}%
\pgfpathlineto{\pgfqpoint{3.902233in}{1.971752in}}%
\pgfpathlineto{\pgfqpoint{3.951675in}{1.260417in}}%
\pgfpathlineto{\pgfqpoint{6.918182in}{1.208226in}}%
\pgfpathlineto{\pgfqpoint{6.918182in}{1.594418in}}%
\pgfpathlineto{\pgfqpoint{6.918182in}{1.594418in}}%
\pgfusepath{stroke}%
\end{pgfscope}%
\begin{pgfscope}%
\pgfpathrectangle{\pgfqpoint{1.000000in}{0.330000in}}{\pgfqpoint{6.200000in}{2.310000in}}%
\pgfusepath{clip}%
\pgfsetrectcap%
\pgfsetroundjoin%
\pgfsetlinewidth{1.505625pt}%
\definecolor{currentstroke}{rgb}{0.121569,0.466667,0.705882}%
\pgfsetstrokecolor{currentstroke}%
\pgfsetdash{}{0pt}%
\pgfpathmoveto{\pgfqpoint{1.281818in}{1.594418in}}%
\pgfpathlineto{\pgfqpoint{1.281818in}{2.267093in}}%
\pgfpathlineto{\pgfqpoint{3.902233in}{2.209815in}}%
\pgfpathlineto{\pgfqpoint{3.951675in}{1.495615in}}%
\pgfpathlineto{\pgfqpoint{5.533812in}{1.461032in}}%
\pgfpathlineto{\pgfqpoint{5.583254in}{0.464179in}}%
\pgfpathlineto{\pgfqpoint{6.918182in}{0.435000in}}%
\pgfpathlineto{\pgfqpoint{6.918182in}{1.594418in}}%
\pgfpathlineto{\pgfqpoint{6.918182in}{1.594418in}}%
\pgfusepath{stroke}%
\end{pgfscope}%
\begin{pgfscope}%
\pgfpathrectangle{\pgfqpoint{1.000000in}{0.330000in}}{\pgfqpoint{6.200000in}{2.310000in}}%
\pgfusepath{clip}%
\pgfsetrectcap%
\pgfsetroundjoin%
\pgfsetlinewidth{1.505625pt}%
\definecolor{currentstroke}{rgb}{1.000000,0.498039,0.054902}%
\pgfsetstrokecolor{currentstroke}%
\pgfsetdash{}{0pt}%
\pgfpathmoveto{\pgfqpoint{1.281818in}{1.594418in}}%
\pgfpathlineto{\pgfqpoint{1.281818in}{2.521801in}}%
\pgfpathlineto{\pgfqpoint{3.902233in}{2.464523in}}%
\pgfpathlineto{\pgfqpoint{3.951675in}{0.818280in}}%
\pgfpathlineto{\pgfqpoint{6.918182in}{0.753438in}}%
\pgfpathlineto{\pgfqpoint{6.918182in}{1.594418in}}%
\pgfpathlineto{\pgfqpoint{6.918182in}{1.594418in}}%
\pgfusepath{stroke}%
\end{pgfscope}%
\begin{pgfscope}%
\pgfpathrectangle{\pgfqpoint{1.000000in}{0.330000in}}{\pgfqpoint{6.200000in}{2.310000in}}%
\pgfusepath{clip}%
\pgfsetrectcap%
\pgfsetroundjoin%
\pgfsetlinewidth{1.505625pt}%
\definecolor{currentstroke}{rgb}{0.172549,0.627451,0.172549}%
\pgfsetstrokecolor{currentstroke}%
\pgfsetdash{}{0pt}%
\pgfpathmoveto{\pgfqpoint{1.281818in}{1.594418in}}%
\pgfpathlineto{\pgfqpoint{1.281818in}{1.759522in}}%
\pgfpathlineto{\pgfqpoint{3.902233in}{1.732629in}}%
\pgfpathlineto{\pgfqpoint{3.951675in}{1.025197in}}%
\pgfpathlineto{\pgfqpoint{5.533812in}{1.008960in}}%
\pgfpathlineto{\pgfqpoint{5.583254in}{2.004224in}}%
\pgfpathlineto{\pgfqpoint{6.918182in}{1.990524in}}%
\pgfpathlineto{\pgfqpoint{6.918182in}{1.594418in}}%
\pgfpathlineto{\pgfqpoint{6.918182in}{1.594418in}}%
\pgfusepath{stroke}%
\end{pgfscope}%
\begin{pgfscope}%
\pgfsetroundcap%
\pgfsetroundjoin%
\pgfsetlinewidth{1.003750pt}%
\definecolor{currentstroke}{rgb}{0.000000,0.000000,0.000000}%
\pgfsetstrokecolor{currentstroke}%
\pgfsetdash{}{0pt}%
\pgfpathmoveto{\pgfqpoint{6.496841in}{0.435000in}}%
\pgfpathquadraticcurveto{\pgfqpoint{6.693611in}{0.435000in}}{\pgfqpoint{6.890382in}{0.435000in}}%
\pgfusepath{stroke}%
\end{pgfscope}%
\begin{pgfscope}%
\pgfsetbuttcap%
\pgfsetmiterjoin%
\definecolor{currentfill}{rgb}{0.800000,0.800000,0.800000}%
\pgfsetfillcolor{currentfill}%
\pgfsetlinewidth{1.003750pt}%
\definecolor{currentstroke}{rgb}{0.000000,0.000000,0.000000}%
\pgfsetstrokecolor{currentstroke}%
\pgfsetdash{}{0pt}%
\pgfpathmoveto{\pgfqpoint{5.494571in}{0.338549in}}%
\pgfpathcurveto{\pgfqpoint{5.529293in}{0.303827in}}{\pgfqpoint{6.404374in}{0.303827in}}{\pgfqpoint{6.439097in}{0.338549in}}%
\pgfpathcurveto{\pgfqpoint{6.473819in}{0.373272in}}{\pgfqpoint{6.473819in}{0.496728in}}{\pgfqpoint{6.439097in}{0.531451in}}%
\pgfpathcurveto{\pgfqpoint{6.404374in}{0.566173in}}{\pgfqpoint{5.529293in}{0.566173in}}{\pgfqpoint{5.494571in}{0.531451in}}%
\pgfpathcurveto{\pgfqpoint{5.459848in}{0.496728in}}{\pgfqpoint{5.459848in}{0.373272in}}{\pgfqpoint{5.494571in}{0.338549in}}%
\pgfpathclose%
\pgfusepath{stroke,fill}%
\end{pgfscope}%
\begin{pgfscope}%
\definecolor{textcolor}{rgb}{0.000000,0.000000,0.000000}%
\pgfsetstrokecolor{textcolor}%
\pgfsetfillcolor{textcolor}%
\pgftext[x=5.529293in,y=0.435000in,left,]{\color{textcolor}\rmfamily\fontsize{10.000000}{12.000000}\selectfont \(\displaystyle V_u =\) -39.3 kip}%
\end{pgfscope}%
\begin{pgfscope}%
\pgfsetbuttcap%
\pgfsetmiterjoin%
\definecolor{currentfill}{rgb}{0.800000,0.800000,0.800000}%
\pgfsetfillcolor{currentfill}%
\pgfsetlinewidth{1.003750pt}%
\definecolor{currentstroke}{rgb}{0.000000,0.000000,0.000000}%
\pgfsetstrokecolor{currentstroke}%
\pgfsetdash{}{0pt}%
\pgfpathmoveto{\pgfqpoint{0.965278in}{0.358599in}}%
\pgfpathcurveto{\pgfqpoint{1.000000in}{0.323877in}}{\pgfqpoint{3.100699in}{0.323877in}}{\pgfqpoint{3.135421in}{0.358599in}}%
\pgfpathcurveto{\pgfqpoint{3.170144in}{0.393321in}}{\pgfqpoint{3.170144in}{0.668784in}}{\pgfqpoint{3.135421in}{0.703506in}}%
\pgfpathcurveto{\pgfqpoint{3.100699in}{0.738228in}}{\pgfqpoint{1.000000in}{0.738228in}}{\pgfqpoint{0.965278in}{0.703506in}}%
\pgfpathcurveto{\pgfqpoint{0.930556in}{0.668784in}}{\pgfqpoint{0.930556in}{0.393321in}}{\pgfqpoint{0.965278in}{0.358599in}}%
\pgfpathclose%
\pgfusepath{stroke,fill}%
\end{pgfscope}%
\begin{pgfscope}%
\definecolor{textcolor}{rgb}{0.000000,0.000000,0.000000}%
\pgfsetstrokecolor{textcolor}%
\pgfsetfillcolor{textcolor}%
\pgftext[x=1.000000in, y=0.580049in, left, base]{\color{textcolor}\rmfamily\fontsize{10.000000}{12.000000}\selectfont Max combo: 1.4D + 2.5E + 0.5L0}%
\end{pgfscope}%
\begin{pgfscope}%
\definecolor{textcolor}{rgb}{0.000000,0.000000,0.000000}%
\pgfsetstrokecolor{textcolor}%
\pgfsetfillcolor{textcolor}%
\pgftext[x=1.000000in, y=0.428043in, left, base]{\color{textcolor}\rmfamily\fontsize{10.000000}{12.000000}\selectfont ASCE7-16 Sec. 2.3.6 (LC 6)}%
\end{pgfscope}%
\end{pgfpicture}%
\makeatother%
\endgroup%

\end{center}
\caption{Shear Demand Envelope}
\end{figure}
C\textsubscript{v1}, the web shear strength coefficient, is calculated per AISC/ANSI 360-16 Eq. G2-2 as follows, based on the ratio of the clear distance between flanges to web thickness:
\begin{flalign*}
\frac{h}{t_w} = \frac{11.4 {\color{darkBlue}{\mathbf{ \; in}}}}{0.6 {\color{darkBlue}{\mathbf{ \; in}}}} = \mathbf{17.7 } <= 2.24\sqrt{\frac{E}{F_y}} = 2.24\sqrt{\frac{29000 {\color{darkBlue}{\mathbf{ \; ksi}}}}{50 {\color{darkBlue}{\mathbf{ \; ksi}}}}} = \mathbf{53.9 } \rightarrow C_{v1} = \mathbf{1.0}
\end{flalign*}
\textphi\textsubscript{v}, the resistance factor for shear, is calculated per AISC/ANSI 360-16 {\S}G2.1.a as follows:
\begin{flalign*}
\frac{h}{t_w} = \frac{11.4 {\color{darkBlue}{\mathbf{ \; in}}}}{0.645 {\color{darkBlue}{\mathbf{ \; in}}}} = \mathbf{17.7 } <= 2.24\cdot \sqrt{\frac{E}{F_y}} = 2.24\cdot \sqrt{\frac{29000 {\color{darkBlue}{\mathbf{ \; ksi}}}}{50 {\color{darkBlue}{\mathbf{ \; ksi}}}}} = \mathbf{53.9} \rightarrow \phi_v = \mathbf{1.0}
\end{flalign*}
\textphi\textsubscript{v}V\textsubscript{n}, the design shear strength, is calculated per AISC/ANSI 360-16 Eq. G2-1 as follows:
\begin{flalign*}
\phi_v V_n = 0.6\cdot F_y \cdot A_w \cdot C_{v1}  = 0.6\cdot 50 {\color{darkBlue}{\mathbf{ \; ksi}}} \cdot 9.48 {\color{darkBlue}{\mathbf{ \; {\color{darkBlue}{\mathbf{ \; in}}}^{2}}}} \cdot 1.0  = \mathbf{284.4 {\color{darkBlue}{\mathbf{ \; kip}}}}
\end{flalign*}
\vspace{-26pt}
{\setlength{\mathindent}{0cm}
\begin{flalign*}
\mathbf{|V_u| = 39.3 {\color{darkBlue}{\mathbf{ \; kip}}}  \;  < \phi_v \cdot V_n = 284.4 {\color{darkBlue}{\mathbf{ \; kip}}}  \;  (DCR = 0.14 - OK)}
\end{flalign*}
%	----------------------------- DEFLECTION CHECK -------------------------------
\section{Deflection Check}
\begin{figure}[H]
\begin{center}
%% Creator: Matplotlib, PGF backend
%%
%% To include the figure in your LaTeX document, write
%%   \input{<filename>.pgf}
%%
%% Make sure the required packages are loaded in your preamble
%%   \usepackage{pgf}
%%
%% Figures using additional raster images can only be included by \input if
%% they are in the same directory as the main LaTeX file. For loading figures
%% from other directories you can use the `import` package
%%   \usepackage{import}
%%
%% and then include the figures with
%%   \import{<path to file>}{<filename>.pgf}
%%
%% Matplotlib used the following preamble
%%
\begingroup%
\makeatletter%
\begin{pgfpicture}%
\pgfpathrectangle{\pgfpointorigin}{\pgfqpoint{8.000000in}{3.000000in}}%
\pgfusepath{use as bounding box, clip}%
\begin{pgfscope}%
\pgfsetbuttcap%
\pgfsetmiterjoin%
\definecolor{currentfill}{rgb}{1.000000,1.000000,1.000000}%
\pgfsetfillcolor{currentfill}%
\pgfsetlinewidth{0.000000pt}%
\definecolor{currentstroke}{rgb}{1.000000,1.000000,1.000000}%
\pgfsetstrokecolor{currentstroke}%
\pgfsetdash{}{0pt}%
\pgfpathmoveto{\pgfqpoint{0.000000in}{0.000000in}}%
\pgfpathlineto{\pgfqpoint{8.000000in}{0.000000in}}%
\pgfpathlineto{\pgfqpoint{8.000000in}{3.000000in}}%
\pgfpathlineto{\pgfqpoint{0.000000in}{3.000000in}}%
\pgfpathclose%
\pgfusepath{fill}%
\end{pgfscope}%
\begin{pgfscope}%
\pgfsetbuttcap%
\pgfsetmiterjoin%
\definecolor{currentfill}{rgb}{1.000000,1.000000,1.000000}%
\pgfsetfillcolor{currentfill}%
\pgfsetlinewidth{0.000000pt}%
\definecolor{currentstroke}{rgb}{0.000000,0.000000,0.000000}%
\pgfsetstrokecolor{currentstroke}%
\pgfsetstrokeopacity{0.000000}%
\pgfsetdash{}{0pt}%
\pgfpathmoveto{\pgfqpoint{1.000000in}{0.330000in}}%
\pgfpathlineto{\pgfqpoint{7.200000in}{0.330000in}}%
\pgfpathlineto{\pgfqpoint{7.200000in}{2.640000in}}%
\pgfpathlineto{\pgfqpoint{1.000000in}{2.640000in}}%
\pgfpathclose%
\pgfusepath{fill}%
\end{pgfscope}%
\begin{pgfscope}%
\pgfpathrectangle{\pgfqpoint{1.000000in}{0.330000in}}{\pgfqpoint{6.200000in}{2.310000in}}%
\pgfusepath{clip}%
\pgfsetbuttcap%
\pgfsetroundjoin%
\pgfsetlinewidth{0.803000pt}%
\definecolor{currentstroke}{rgb}{0.000000,0.000000,0.000000}%
\pgfsetstrokecolor{currentstroke}%
\pgfsetdash{{0.800000pt}{1.320000pt}}{0.000000pt}%
\pgfpathmoveto{\pgfqpoint{1.281818in}{0.330000in}}%
\pgfpathlineto{\pgfqpoint{1.281818in}{2.640000in}}%
\pgfusepath{stroke}%
\end{pgfscope}%
\begin{pgfscope}%
\pgfsetbuttcap%
\pgfsetroundjoin%
\definecolor{currentfill}{rgb}{0.000000,0.000000,0.000000}%
\pgfsetfillcolor{currentfill}%
\pgfsetlinewidth{0.803000pt}%
\definecolor{currentstroke}{rgb}{0.000000,0.000000,0.000000}%
\pgfsetstrokecolor{currentstroke}%
\pgfsetdash{}{0pt}%
\pgfsys@defobject{currentmarker}{\pgfqpoint{0.000000in}{-0.048611in}}{\pgfqpoint{0.000000in}{0.000000in}}{%
\pgfpathmoveto{\pgfqpoint{0.000000in}{0.000000in}}%
\pgfpathlineto{\pgfqpoint{0.000000in}{-0.048611in}}%
\pgfusepath{stroke,fill}%
}%
\begin{pgfscope}%
\pgfsys@transformshift{1.281818in}{0.330000in}%
\pgfsys@useobject{currentmarker}{}%
\end{pgfscope}%
\end{pgfscope}%
\begin{pgfscope}%
\pgfsetbuttcap%
\pgfsetroundjoin%
\definecolor{currentfill}{rgb}{0.000000,0.000000,0.000000}%
\pgfsetfillcolor{currentfill}%
\pgfsetlinewidth{0.803000pt}%
\definecolor{currentstroke}{rgb}{0.000000,0.000000,0.000000}%
\pgfsetstrokecolor{currentstroke}%
\pgfsetdash{}{0pt}%
\pgfsys@defobject{currentmarker}{\pgfqpoint{0.000000in}{0.000000in}}{\pgfqpoint{0.000000in}{0.048611in}}{%
\pgfpathmoveto{\pgfqpoint{0.000000in}{0.000000in}}%
\pgfpathlineto{\pgfqpoint{0.000000in}{0.048611in}}%
\pgfusepath{stroke,fill}%
}%
\begin{pgfscope}%
\pgfsys@transformshift{1.281818in}{2.640000in}%
\pgfsys@useobject{currentmarker}{}%
\end{pgfscope}%
\end{pgfscope}%
\begin{pgfscope}%
\definecolor{textcolor}{rgb}{0.000000,0.000000,0.000000}%
\pgfsetstrokecolor{textcolor}%
\pgfsetfillcolor{textcolor}%
\pgftext[x=1.281818in,y=0.232778in,,top]{\color{textcolor}\rmfamily\fontsize{10.000000}{12.000000}\selectfont \(\displaystyle {0.0}\)}%
\end{pgfscope}%
\begin{pgfscope}%
\pgfpathrectangle{\pgfqpoint{1.000000in}{0.330000in}}{\pgfqpoint{6.200000in}{2.310000in}}%
\pgfusepath{clip}%
\pgfsetbuttcap%
\pgfsetroundjoin%
\pgfsetlinewidth{0.803000pt}%
\definecolor{currentstroke}{rgb}{0.000000,0.000000,0.000000}%
\pgfsetstrokecolor{currentstroke}%
\pgfsetdash{{0.800000pt}{1.320000pt}}{0.000000pt}%
\pgfpathmoveto{\pgfqpoint{2.023445in}{0.330000in}}%
\pgfpathlineto{\pgfqpoint{2.023445in}{2.640000in}}%
\pgfusepath{stroke}%
\end{pgfscope}%
\begin{pgfscope}%
\pgfsetbuttcap%
\pgfsetroundjoin%
\definecolor{currentfill}{rgb}{0.000000,0.000000,0.000000}%
\pgfsetfillcolor{currentfill}%
\pgfsetlinewidth{0.803000pt}%
\definecolor{currentstroke}{rgb}{0.000000,0.000000,0.000000}%
\pgfsetstrokecolor{currentstroke}%
\pgfsetdash{}{0pt}%
\pgfsys@defobject{currentmarker}{\pgfqpoint{0.000000in}{-0.048611in}}{\pgfqpoint{0.000000in}{0.000000in}}{%
\pgfpathmoveto{\pgfqpoint{0.000000in}{0.000000in}}%
\pgfpathlineto{\pgfqpoint{0.000000in}{-0.048611in}}%
\pgfusepath{stroke,fill}%
}%
\begin{pgfscope}%
\pgfsys@transformshift{2.023445in}{0.330000in}%
\pgfsys@useobject{currentmarker}{}%
\end{pgfscope}%
\end{pgfscope}%
\begin{pgfscope}%
\pgfsetbuttcap%
\pgfsetroundjoin%
\definecolor{currentfill}{rgb}{0.000000,0.000000,0.000000}%
\pgfsetfillcolor{currentfill}%
\pgfsetlinewidth{0.803000pt}%
\definecolor{currentstroke}{rgb}{0.000000,0.000000,0.000000}%
\pgfsetstrokecolor{currentstroke}%
\pgfsetdash{}{0pt}%
\pgfsys@defobject{currentmarker}{\pgfqpoint{0.000000in}{0.000000in}}{\pgfqpoint{0.000000in}{0.048611in}}{%
\pgfpathmoveto{\pgfqpoint{0.000000in}{0.000000in}}%
\pgfpathlineto{\pgfqpoint{0.000000in}{0.048611in}}%
\pgfusepath{stroke,fill}%
}%
\begin{pgfscope}%
\pgfsys@transformshift{2.023445in}{2.640000in}%
\pgfsys@useobject{currentmarker}{}%
\end{pgfscope}%
\end{pgfscope}%
\begin{pgfscope}%
\definecolor{textcolor}{rgb}{0.000000,0.000000,0.000000}%
\pgfsetstrokecolor{textcolor}%
\pgfsetfillcolor{textcolor}%
\pgftext[x=2.023445in,y=0.232778in,,top]{\color{textcolor}\rmfamily\fontsize{10.000000}{12.000000}\selectfont \(\displaystyle {2.5}\)}%
\end{pgfscope}%
\begin{pgfscope}%
\pgfpathrectangle{\pgfqpoint{1.000000in}{0.330000in}}{\pgfqpoint{6.200000in}{2.310000in}}%
\pgfusepath{clip}%
\pgfsetbuttcap%
\pgfsetroundjoin%
\pgfsetlinewidth{0.803000pt}%
\definecolor{currentstroke}{rgb}{0.000000,0.000000,0.000000}%
\pgfsetstrokecolor{currentstroke}%
\pgfsetdash{{0.800000pt}{1.320000pt}}{0.000000pt}%
\pgfpathmoveto{\pgfqpoint{2.765072in}{0.330000in}}%
\pgfpathlineto{\pgfqpoint{2.765072in}{2.640000in}}%
\pgfusepath{stroke}%
\end{pgfscope}%
\begin{pgfscope}%
\pgfsetbuttcap%
\pgfsetroundjoin%
\definecolor{currentfill}{rgb}{0.000000,0.000000,0.000000}%
\pgfsetfillcolor{currentfill}%
\pgfsetlinewidth{0.803000pt}%
\definecolor{currentstroke}{rgb}{0.000000,0.000000,0.000000}%
\pgfsetstrokecolor{currentstroke}%
\pgfsetdash{}{0pt}%
\pgfsys@defobject{currentmarker}{\pgfqpoint{0.000000in}{-0.048611in}}{\pgfqpoint{0.000000in}{0.000000in}}{%
\pgfpathmoveto{\pgfqpoint{0.000000in}{0.000000in}}%
\pgfpathlineto{\pgfqpoint{0.000000in}{-0.048611in}}%
\pgfusepath{stroke,fill}%
}%
\begin{pgfscope}%
\pgfsys@transformshift{2.765072in}{0.330000in}%
\pgfsys@useobject{currentmarker}{}%
\end{pgfscope}%
\end{pgfscope}%
\begin{pgfscope}%
\pgfsetbuttcap%
\pgfsetroundjoin%
\definecolor{currentfill}{rgb}{0.000000,0.000000,0.000000}%
\pgfsetfillcolor{currentfill}%
\pgfsetlinewidth{0.803000pt}%
\definecolor{currentstroke}{rgb}{0.000000,0.000000,0.000000}%
\pgfsetstrokecolor{currentstroke}%
\pgfsetdash{}{0pt}%
\pgfsys@defobject{currentmarker}{\pgfqpoint{0.000000in}{0.000000in}}{\pgfqpoint{0.000000in}{0.048611in}}{%
\pgfpathmoveto{\pgfqpoint{0.000000in}{0.000000in}}%
\pgfpathlineto{\pgfqpoint{0.000000in}{0.048611in}}%
\pgfusepath{stroke,fill}%
}%
\begin{pgfscope}%
\pgfsys@transformshift{2.765072in}{2.640000in}%
\pgfsys@useobject{currentmarker}{}%
\end{pgfscope}%
\end{pgfscope}%
\begin{pgfscope}%
\definecolor{textcolor}{rgb}{0.000000,0.000000,0.000000}%
\pgfsetstrokecolor{textcolor}%
\pgfsetfillcolor{textcolor}%
\pgftext[x=2.765072in,y=0.232778in,,top]{\color{textcolor}\rmfamily\fontsize{10.000000}{12.000000}\selectfont \(\displaystyle {5.0}\)}%
\end{pgfscope}%
\begin{pgfscope}%
\pgfpathrectangle{\pgfqpoint{1.000000in}{0.330000in}}{\pgfqpoint{6.200000in}{2.310000in}}%
\pgfusepath{clip}%
\pgfsetbuttcap%
\pgfsetroundjoin%
\pgfsetlinewidth{0.803000pt}%
\definecolor{currentstroke}{rgb}{0.000000,0.000000,0.000000}%
\pgfsetstrokecolor{currentstroke}%
\pgfsetdash{{0.800000pt}{1.320000pt}}{0.000000pt}%
\pgfpathmoveto{\pgfqpoint{3.506699in}{0.330000in}}%
\pgfpathlineto{\pgfqpoint{3.506699in}{2.640000in}}%
\pgfusepath{stroke}%
\end{pgfscope}%
\begin{pgfscope}%
\pgfsetbuttcap%
\pgfsetroundjoin%
\definecolor{currentfill}{rgb}{0.000000,0.000000,0.000000}%
\pgfsetfillcolor{currentfill}%
\pgfsetlinewidth{0.803000pt}%
\definecolor{currentstroke}{rgb}{0.000000,0.000000,0.000000}%
\pgfsetstrokecolor{currentstroke}%
\pgfsetdash{}{0pt}%
\pgfsys@defobject{currentmarker}{\pgfqpoint{0.000000in}{-0.048611in}}{\pgfqpoint{0.000000in}{0.000000in}}{%
\pgfpathmoveto{\pgfqpoint{0.000000in}{0.000000in}}%
\pgfpathlineto{\pgfqpoint{0.000000in}{-0.048611in}}%
\pgfusepath{stroke,fill}%
}%
\begin{pgfscope}%
\pgfsys@transformshift{3.506699in}{0.330000in}%
\pgfsys@useobject{currentmarker}{}%
\end{pgfscope}%
\end{pgfscope}%
\begin{pgfscope}%
\pgfsetbuttcap%
\pgfsetroundjoin%
\definecolor{currentfill}{rgb}{0.000000,0.000000,0.000000}%
\pgfsetfillcolor{currentfill}%
\pgfsetlinewidth{0.803000pt}%
\definecolor{currentstroke}{rgb}{0.000000,0.000000,0.000000}%
\pgfsetstrokecolor{currentstroke}%
\pgfsetdash{}{0pt}%
\pgfsys@defobject{currentmarker}{\pgfqpoint{0.000000in}{0.000000in}}{\pgfqpoint{0.000000in}{0.048611in}}{%
\pgfpathmoveto{\pgfqpoint{0.000000in}{0.000000in}}%
\pgfpathlineto{\pgfqpoint{0.000000in}{0.048611in}}%
\pgfusepath{stroke,fill}%
}%
\begin{pgfscope}%
\pgfsys@transformshift{3.506699in}{2.640000in}%
\pgfsys@useobject{currentmarker}{}%
\end{pgfscope}%
\end{pgfscope}%
\begin{pgfscope}%
\definecolor{textcolor}{rgb}{0.000000,0.000000,0.000000}%
\pgfsetstrokecolor{textcolor}%
\pgfsetfillcolor{textcolor}%
\pgftext[x=3.506699in,y=0.232778in,,top]{\color{textcolor}\rmfamily\fontsize{10.000000}{12.000000}\selectfont \(\displaystyle {7.5}\)}%
\end{pgfscope}%
\begin{pgfscope}%
\pgfpathrectangle{\pgfqpoint{1.000000in}{0.330000in}}{\pgfqpoint{6.200000in}{2.310000in}}%
\pgfusepath{clip}%
\pgfsetbuttcap%
\pgfsetroundjoin%
\pgfsetlinewidth{0.803000pt}%
\definecolor{currentstroke}{rgb}{0.000000,0.000000,0.000000}%
\pgfsetstrokecolor{currentstroke}%
\pgfsetdash{{0.800000pt}{1.320000pt}}{0.000000pt}%
\pgfpathmoveto{\pgfqpoint{4.248325in}{0.330000in}}%
\pgfpathlineto{\pgfqpoint{4.248325in}{2.640000in}}%
\pgfusepath{stroke}%
\end{pgfscope}%
\begin{pgfscope}%
\pgfsetbuttcap%
\pgfsetroundjoin%
\definecolor{currentfill}{rgb}{0.000000,0.000000,0.000000}%
\pgfsetfillcolor{currentfill}%
\pgfsetlinewidth{0.803000pt}%
\definecolor{currentstroke}{rgb}{0.000000,0.000000,0.000000}%
\pgfsetstrokecolor{currentstroke}%
\pgfsetdash{}{0pt}%
\pgfsys@defobject{currentmarker}{\pgfqpoint{0.000000in}{-0.048611in}}{\pgfqpoint{0.000000in}{0.000000in}}{%
\pgfpathmoveto{\pgfqpoint{0.000000in}{0.000000in}}%
\pgfpathlineto{\pgfqpoint{0.000000in}{-0.048611in}}%
\pgfusepath{stroke,fill}%
}%
\begin{pgfscope}%
\pgfsys@transformshift{4.248325in}{0.330000in}%
\pgfsys@useobject{currentmarker}{}%
\end{pgfscope}%
\end{pgfscope}%
\begin{pgfscope}%
\pgfsetbuttcap%
\pgfsetroundjoin%
\definecolor{currentfill}{rgb}{0.000000,0.000000,0.000000}%
\pgfsetfillcolor{currentfill}%
\pgfsetlinewidth{0.803000pt}%
\definecolor{currentstroke}{rgb}{0.000000,0.000000,0.000000}%
\pgfsetstrokecolor{currentstroke}%
\pgfsetdash{}{0pt}%
\pgfsys@defobject{currentmarker}{\pgfqpoint{0.000000in}{0.000000in}}{\pgfqpoint{0.000000in}{0.048611in}}{%
\pgfpathmoveto{\pgfqpoint{0.000000in}{0.000000in}}%
\pgfpathlineto{\pgfqpoint{0.000000in}{0.048611in}}%
\pgfusepath{stroke,fill}%
}%
\begin{pgfscope}%
\pgfsys@transformshift{4.248325in}{2.640000in}%
\pgfsys@useobject{currentmarker}{}%
\end{pgfscope}%
\end{pgfscope}%
\begin{pgfscope}%
\definecolor{textcolor}{rgb}{0.000000,0.000000,0.000000}%
\pgfsetstrokecolor{textcolor}%
\pgfsetfillcolor{textcolor}%
\pgftext[x=4.248325in,y=0.232778in,,top]{\color{textcolor}\rmfamily\fontsize{10.000000}{12.000000}\selectfont \(\displaystyle {10.0}\)}%
\end{pgfscope}%
\begin{pgfscope}%
\pgfpathrectangle{\pgfqpoint{1.000000in}{0.330000in}}{\pgfqpoint{6.200000in}{2.310000in}}%
\pgfusepath{clip}%
\pgfsetbuttcap%
\pgfsetroundjoin%
\pgfsetlinewidth{0.803000pt}%
\definecolor{currentstroke}{rgb}{0.000000,0.000000,0.000000}%
\pgfsetstrokecolor{currentstroke}%
\pgfsetdash{{0.800000pt}{1.320000pt}}{0.000000pt}%
\pgfpathmoveto{\pgfqpoint{4.989952in}{0.330000in}}%
\pgfpathlineto{\pgfqpoint{4.989952in}{2.640000in}}%
\pgfusepath{stroke}%
\end{pgfscope}%
\begin{pgfscope}%
\pgfsetbuttcap%
\pgfsetroundjoin%
\definecolor{currentfill}{rgb}{0.000000,0.000000,0.000000}%
\pgfsetfillcolor{currentfill}%
\pgfsetlinewidth{0.803000pt}%
\definecolor{currentstroke}{rgb}{0.000000,0.000000,0.000000}%
\pgfsetstrokecolor{currentstroke}%
\pgfsetdash{}{0pt}%
\pgfsys@defobject{currentmarker}{\pgfqpoint{0.000000in}{-0.048611in}}{\pgfqpoint{0.000000in}{0.000000in}}{%
\pgfpathmoveto{\pgfqpoint{0.000000in}{0.000000in}}%
\pgfpathlineto{\pgfqpoint{0.000000in}{-0.048611in}}%
\pgfusepath{stroke,fill}%
}%
\begin{pgfscope}%
\pgfsys@transformshift{4.989952in}{0.330000in}%
\pgfsys@useobject{currentmarker}{}%
\end{pgfscope}%
\end{pgfscope}%
\begin{pgfscope}%
\pgfsetbuttcap%
\pgfsetroundjoin%
\definecolor{currentfill}{rgb}{0.000000,0.000000,0.000000}%
\pgfsetfillcolor{currentfill}%
\pgfsetlinewidth{0.803000pt}%
\definecolor{currentstroke}{rgb}{0.000000,0.000000,0.000000}%
\pgfsetstrokecolor{currentstroke}%
\pgfsetdash{}{0pt}%
\pgfsys@defobject{currentmarker}{\pgfqpoint{0.000000in}{0.000000in}}{\pgfqpoint{0.000000in}{0.048611in}}{%
\pgfpathmoveto{\pgfqpoint{0.000000in}{0.000000in}}%
\pgfpathlineto{\pgfqpoint{0.000000in}{0.048611in}}%
\pgfusepath{stroke,fill}%
}%
\begin{pgfscope}%
\pgfsys@transformshift{4.989952in}{2.640000in}%
\pgfsys@useobject{currentmarker}{}%
\end{pgfscope}%
\end{pgfscope}%
\begin{pgfscope}%
\definecolor{textcolor}{rgb}{0.000000,0.000000,0.000000}%
\pgfsetstrokecolor{textcolor}%
\pgfsetfillcolor{textcolor}%
\pgftext[x=4.989952in,y=0.232778in,,top]{\color{textcolor}\rmfamily\fontsize{10.000000}{12.000000}\selectfont \(\displaystyle {12.5}\)}%
\end{pgfscope}%
\begin{pgfscope}%
\pgfpathrectangle{\pgfqpoint{1.000000in}{0.330000in}}{\pgfqpoint{6.200000in}{2.310000in}}%
\pgfusepath{clip}%
\pgfsetbuttcap%
\pgfsetroundjoin%
\pgfsetlinewidth{0.803000pt}%
\definecolor{currentstroke}{rgb}{0.000000,0.000000,0.000000}%
\pgfsetstrokecolor{currentstroke}%
\pgfsetdash{{0.800000pt}{1.320000pt}}{0.000000pt}%
\pgfpathmoveto{\pgfqpoint{5.731579in}{0.330000in}}%
\pgfpathlineto{\pgfqpoint{5.731579in}{2.640000in}}%
\pgfusepath{stroke}%
\end{pgfscope}%
\begin{pgfscope}%
\pgfsetbuttcap%
\pgfsetroundjoin%
\definecolor{currentfill}{rgb}{0.000000,0.000000,0.000000}%
\pgfsetfillcolor{currentfill}%
\pgfsetlinewidth{0.803000pt}%
\definecolor{currentstroke}{rgb}{0.000000,0.000000,0.000000}%
\pgfsetstrokecolor{currentstroke}%
\pgfsetdash{}{0pt}%
\pgfsys@defobject{currentmarker}{\pgfqpoint{0.000000in}{-0.048611in}}{\pgfqpoint{0.000000in}{0.000000in}}{%
\pgfpathmoveto{\pgfqpoint{0.000000in}{0.000000in}}%
\pgfpathlineto{\pgfqpoint{0.000000in}{-0.048611in}}%
\pgfusepath{stroke,fill}%
}%
\begin{pgfscope}%
\pgfsys@transformshift{5.731579in}{0.330000in}%
\pgfsys@useobject{currentmarker}{}%
\end{pgfscope}%
\end{pgfscope}%
\begin{pgfscope}%
\pgfsetbuttcap%
\pgfsetroundjoin%
\definecolor{currentfill}{rgb}{0.000000,0.000000,0.000000}%
\pgfsetfillcolor{currentfill}%
\pgfsetlinewidth{0.803000pt}%
\definecolor{currentstroke}{rgb}{0.000000,0.000000,0.000000}%
\pgfsetstrokecolor{currentstroke}%
\pgfsetdash{}{0pt}%
\pgfsys@defobject{currentmarker}{\pgfqpoint{0.000000in}{0.000000in}}{\pgfqpoint{0.000000in}{0.048611in}}{%
\pgfpathmoveto{\pgfqpoint{0.000000in}{0.000000in}}%
\pgfpathlineto{\pgfqpoint{0.000000in}{0.048611in}}%
\pgfusepath{stroke,fill}%
}%
\begin{pgfscope}%
\pgfsys@transformshift{5.731579in}{2.640000in}%
\pgfsys@useobject{currentmarker}{}%
\end{pgfscope}%
\end{pgfscope}%
\begin{pgfscope}%
\definecolor{textcolor}{rgb}{0.000000,0.000000,0.000000}%
\pgfsetstrokecolor{textcolor}%
\pgfsetfillcolor{textcolor}%
\pgftext[x=5.731579in,y=0.232778in,,top]{\color{textcolor}\rmfamily\fontsize{10.000000}{12.000000}\selectfont \(\displaystyle {15.0}\)}%
\end{pgfscope}%
\begin{pgfscope}%
\pgfpathrectangle{\pgfqpoint{1.000000in}{0.330000in}}{\pgfqpoint{6.200000in}{2.310000in}}%
\pgfusepath{clip}%
\pgfsetbuttcap%
\pgfsetroundjoin%
\pgfsetlinewidth{0.803000pt}%
\definecolor{currentstroke}{rgb}{0.000000,0.000000,0.000000}%
\pgfsetstrokecolor{currentstroke}%
\pgfsetdash{{0.800000pt}{1.320000pt}}{0.000000pt}%
\pgfpathmoveto{\pgfqpoint{6.473206in}{0.330000in}}%
\pgfpathlineto{\pgfqpoint{6.473206in}{2.640000in}}%
\pgfusepath{stroke}%
\end{pgfscope}%
\begin{pgfscope}%
\pgfsetbuttcap%
\pgfsetroundjoin%
\definecolor{currentfill}{rgb}{0.000000,0.000000,0.000000}%
\pgfsetfillcolor{currentfill}%
\pgfsetlinewidth{0.803000pt}%
\definecolor{currentstroke}{rgb}{0.000000,0.000000,0.000000}%
\pgfsetstrokecolor{currentstroke}%
\pgfsetdash{}{0pt}%
\pgfsys@defobject{currentmarker}{\pgfqpoint{0.000000in}{-0.048611in}}{\pgfqpoint{0.000000in}{0.000000in}}{%
\pgfpathmoveto{\pgfqpoint{0.000000in}{0.000000in}}%
\pgfpathlineto{\pgfqpoint{0.000000in}{-0.048611in}}%
\pgfusepath{stroke,fill}%
}%
\begin{pgfscope}%
\pgfsys@transformshift{6.473206in}{0.330000in}%
\pgfsys@useobject{currentmarker}{}%
\end{pgfscope}%
\end{pgfscope}%
\begin{pgfscope}%
\pgfsetbuttcap%
\pgfsetroundjoin%
\definecolor{currentfill}{rgb}{0.000000,0.000000,0.000000}%
\pgfsetfillcolor{currentfill}%
\pgfsetlinewidth{0.803000pt}%
\definecolor{currentstroke}{rgb}{0.000000,0.000000,0.000000}%
\pgfsetstrokecolor{currentstroke}%
\pgfsetdash{}{0pt}%
\pgfsys@defobject{currentmarker}{\pgfqpoint{0.000000in}{0.000000in}}{\pgfqpoint{0.000000in}{0.048611in}}{%
\pgfpathmoveto{\pgfqpoint{0.000000in}{0.000000in}}%
\pgfpathlineto{\pgfqpoint{0.000000in}{0.048611in}}%
\pgfusepath{stroke,fill}%
}%
\begin{pgfscope}%
\pgfsys@transformshift{6.473206in}{2.640000in}%
\pgfsys@useobject{currentmarker}{}%
\end{pgfscope}%
\end{pgfscope}%
\begin{pgfscope}%
\definecolor{textcolor}{rgb}{0.000000,0.000000,0.000000}%
\pgfsetstrokecolor{textcolor}%
\pgfsetfillcolor{textcolor}%
\pgftext[x=6.473206in,y=0.232778in,,top]{\color{textcolor}\rmfamily\fontsize{10.000000}{12.000000}\selectfont \(\displaystyle {17.5}\)}%
\end{pgfscope}%
\begin{pgfscope}%
\pgfpathrectangle{\pgfqpoint{1.000000in}{0.330000in}}{\pgfqpoint{6.200000in}{2.310000in}}%
\pgfusepath{clip}%
\pgfsetbuttcap%
\pgfsetroundjoin%
\pgfsetlinewidth{0.803000pt}%
\definecolor{currentstroke}{rgb}{0.000000,0.000000,0.000000}%
\pgfsetstrokecolor{currentstroke}%
\pgfsetdash{{0.800000pt}{1.320000pt}}{0.000000pt}%
\pgfpathmoveto{\pgfqpoint{1.000000in}{0.528302in}}%
\pgfpathlineto{\pgfqpoint{7.200000in}{0.528302in}}%
\pgfusepath{stroke}%
\end{pgfscope}%
\begin{pgfscope}%
\pgfsetbuttcap%
\pgfsetroundjoin%
\definecolor{currentfill}{rgb}{0.000000,0.000000,0.000000}%
\pgfsetfillcolor{currentfill}%
\pgfsetlinewidth{0.803000pt}%
\definecolor{currentstroke}{rgb}{0.000000,0.000000,0.000000}%
\pgfsetstrokecolor{currentstroke}%
\pgfsetdash{}{0pt}%
\pgfsys@defobject{currentmarker}{\pgfqpoint{-0.048611in}{0.000000in}}{\pgfqpoint{-0.000000in}{0.000000in}}{%
\pgfpathmoveto{\pgfqpoint{-0.000000in}{0.000000in}}%
\pgfpathlineto{\pgfqpoint{-0.048611in}{0.000000in}}%
\pgfusepath{stroke,fill}%
}%
\begin{pgfscope}%
\pgfsys@transformshift{1.000000in}{0.528302in}%
\pgfsys@useobject{currentmarker}{}%
\end{pgfscope}%
\end{pgfscope}%
\begin{pgfscope}%
\pgfsetbuttcap%
\pgfsetroundjoin%
\definecolor{currentfill}{rgb}{0.000000,0.000000,0.000000}%
\pgfsetfillcolor{currentfill}%
\pgfsetlinewidth{0.803000pt}%
\definecolor{currentstroke}{rgb}{0.000000,0.000000,0.000000}%
\pgfsetstrokecolor{currentstroke}%
\pgfsetdash{}{0pt}%
\pgfsys@defobject{currentmarker}{\pgfqpoint{0.000000in}{0.000000in}}{\pgfqpoint{0.048611in}{0.000000in}}{%
\pgfpathmoveto{\pgfqpoint{0.000000in}{0.000000in}}%
\pgfpathlineto{\pgfqpoint{0.048611in}{0.000000in}}%
\pgfusepath{stroke,fill}%
}%
\begin{pgfscope}%
\pgfsys@transformshift{7.200000in}{0.528302in}%
\pgfsys@useobject{currentmarker}{}%
\end{pgfscope}%
\end{pgfscope}%
\begin{pgfscope}%
\definecolor{textcolor}{rgb}{0.000000,0.000000,0.000000}%
\pgfsetstrokecolor{textcolor}%
\pgfsetfillcolor{textcolor}%
\pgftext[x=0.547838in, y=0.480077in, left, base]{\color{textcolor}\rmfamily\fontsize{10.000000}{12.000000}\selectfont \(\displaystyle {\ensuremath{-}0.25}\)}%
\end{pgfscope}%
\begin{pgfscope}%
\pgfpathrectangle{\pgfqpoint{1.000000in}{0.330000in}}{\pgfqpoint{6.200000in}{2.310000in}}%
\pgfusepath{clip}%
\pgfsetbuttcap%
\pgfsetroundjoin%
\pgfsetlinewidth{0.803000pt}%
\definecolor{currentstroke}{rgb}{0.000000,0.000000,0.000000}%
\pgfsetstrokecolor{currentstroke}%
\pgfsetdash{{0.800000pt}{1.320000pt}}{0.000000pt}%
\pgfpathmoveto{\pgfqpoint{1.000000in}{0.904311in}}%
\pgfpathlineto{\pgfqpoint{7.200000in}{0.904311in}}%
\pgfusepath{stroke}%
\end{pgfscope}%
\begin{pgfscope}%
\pgfsetbuttcap%
\pgfsetroundjoin%
\definecolor{currentfill}{rgb}{0.000000,0.000000,0.000000}%
\pgfsetfillcolor{currentfill}%
\pgfsetlinewidth{0.803000pt}%
\definecolor{currentstroke}{rgb}{0.000000,0.000000,0.000000}%
\pgfsetstrokecolor{currentstroke}%
\pgfsetdash{}{0pt}%
\pgfsys@defobject{currentmarker}{\pgfqpoint{-0.048611in}{0.000000in}}{\pgfqpoint{-0.000000in}{0.000000in}}{%
\pgfpathmoveto{\pgfqpoint{-0.000000in}{0.000000in}}%
\pgfpathlineto{\pgfqpoint{-0.048611in}{0.000000in}}%
\pgfusepath{stroke,fill}%
}%
\begin{pgfscope}%
\pgfsys@transformshift{1.000000in}{0.904311in}%
\pgfsys@useobject{currentmarker}{}%
\end{pgfscope}%
\end{pgfscope}%
\begin{pgfscope}%
\pgfsetbuttcap%
\pgfsetroundjoin%
\definecolor{currentfill}{rgb}{0.000000,0.000000,0.000000}%
\pgfsetfillcolor{currentfill}%
\pgfsetlinewidth{0.803000pt}%
\definecolor{currentstroke}{rgb}{0.000000,0.000000,0.000000}%
\pgfsetstrokecolor{currentstroke}%
\pgfsetdash{}{0pt}%
\pgfsys@defobject{currentmarker}{\pgfqpoint{0.000000in}{0.000000in}}{\pgfqpoint{0.048611in}{0.000000in}}{%
\pgfpathmoveto{\pgfqpoint{0.000000in}{0.000000in}}%
\pgfpathlineto{\pgfqpoint{0.048611in}{0.000000in}}%
\pgfusepath{stroke,fill}%
}%
\begin{pgfscope}%
\pgfsys@transformshift{7.200000in}{0.904311in}%
\pgfsys@useobject{currentmarker}{}%
\end{pgfscope}%
\end{pgfscope}%
\begin{pgfscope}%
\definecolor{textcolor}{rgb}{0.000000,0.000000,0.000000}%
\pgfsetstrokecolor{textcolor}%
\pgfsetfillcolor{textcolor}%
\pgftext[x=0.547838in, y=0.856086in, left, base]{\color{textcolor}\rmfamily\fontsize{10.000000}{12.000000}\selectfont \(\displaystyle {\ensuremath{-}0.20}\)}%
\end{pgfscope}%
\begin{pgfscope}%
\pgfpathrectangle{\pgfqpoint{1.000000in}{0.330000in}}{\pgfqpoint{6.200000in}{2.310000in}}%
\pgfusepath{clip}%
\pgfsetbuttcap%
\pgfsetroundjoin%
\pgfsetlinewidth{0.803000pt}%
\definecolor{currentstroke}{rgb}{0.000000,0.000000,0.000000}%
\pgfsetstrokecolor{currentstroke}%
\pgfsetdash{{0.800000pt}{1.320000pt}}{0.000000pt}%
\pgfpathmoveto{\pgfqpoint{1.000000in}{1.280321in}}%
\pgfpathlineto{\pgfqpoint{7.200000in}{1.280321in}}%
\pgfusepath{stroke}%
\end{pgfscope}%
\begin{pgfscope}%
\pgfsetbuttcap%
\pgfsetroundjoin%
\definecolor{currentfill}{rgb}{0.000000,0.000000,0.000000}%
\pgfsetfillcolor{currentfill}%
\pgfsetlinewidth{0.803000pt}%
\definecolor{currentstroke}{rgb}{0.000000,0.000000,0.000000}%
\pgfsetstrokecolor{currentstroke}%
\pgfsetdash{}{0pt}%
\pgfsys@defobject{currentmarker}{\pgfqpoint{-0.048611in}{0.000000in}}{\pgfqpoint{-0.000000in}{0.000000in}}{%
\pgfpathmoveto{\pgfqpoint{-0.000000in}{0.000000in}}%
\pgfpathlineto{\pgfqpoint{-0.048611in}{0.000000in}}%
\pgfusepath{stroke,fill}%
}%
\begin{pgfscope}%
\pgfsys@transformshift{1.000000in}{1.280321in}%
\pgfsys@useobject{currentmarker}{}%
\end{pgfscope}%
\end{pgfscope}%
\begin{pgfscope}%
\pgfsetbuttcap%
\pgfsetroundjoin%
\definecolor{currentfill}{rgb}{0.000000,0.000000,0.000000}%
\pgfsetfillcolor{currentfill}%
\pgfsetlinewidth{0.803000pt}%
\definecolor{currentstroke}{rgb}{0.000000,0.000000,0.000000}%
\pgfsetstrokecolor{currentstroke}%
\pgfsetdash{}{0pt}%
\pgfsys@defobject{currentmarker}{\pgfqpoint{0.000000in}{0.000000in}}{\pgfqpoint{0.048611in}{0.000000in}}{%
\pgfpathmoveto{\pgfqpoint{0.000000in}{0.000000in}}%
\pgfpathlineto{\pgfqpoint{0.048611in}{0.000000in}}%
\pgfusepath{stroke,fill}%
}%
\begin{pgfscope}%
\pgfsys@transformshift{7.200000in}{1.280321in}%
\pgfsys@useobject{currentmarker}{}%
\end{pgfscope}%
\end{pgfscope}%
\begin{pgfscope}%
\definecolor{textcolor}{rgb}{0.000000,0.000000,0.000000}%
\pgfsetstrokecolor{textcolor}%
\pgfsetfillcolor{textcolor}%
\pgftext[x=0.547838in, y=1.232095in, left, base]{\color{textcolor}\rmfamily\fontsize{10.000000}{12.000000}\selectfont \(\displaystyle {\ensuremath{-}0.15}\)}%
\end{pgfscope}%
\begin{pgfscope}%
\pgfpathrectangle{\pgfqpoint{1.000000in}{0.330000in}}{\pgfqpoint{6.200000in}{2.310000in}}%
\pgfusepath{clip}%
\pgfsetbuttcap%
\pgfsetroundjoin%
\pgfsetlinewidth{0.803000pt}%
\definecolor{currentstroke}{rgb}{0.000000,0.000000,0.000000}%
\pgfsetstrokecolor{currentstroke}%
\pgfsetdash{{0.800000pt}{1.320000pt}}{0.000000pt}%
\pgfpathmoveto{\pgfqpoint{1.000000in}{1.656330in}}%
\pgfpathlineto{\pgfqpoint{7.200000in}{1.656330in}}%
\pgfusepath{stroke}%
\end{pgfscope}%
\begin{pgfscope}%
\pgfsetbuttcap%
\pgfsetroundjoin%
\definecolor{currentfill}{rgb}{0.000000,0.000000,0.000000}%
\pgfsetfillcolor{currentfill}%
\pgfsetlinewidth{0.803000pt}%
\definecolor{currentstroke}{rgb}{0.000000,0.000000,0.000000}%
\pgfsetstrokecolor{currentstroke}%
\pgfsetdash{}{0pt}%
\pgfsys@defobject{currentmarker}{\pgfqpoint{-0.048611in}{0.000000in}}{\pgfqpoint{-0.000000in}{0.000000in}}{%
\pgfpathmoveto{\pgfqpoint{-0.000000in}{0.000000in}}%
\pgfpathlineto{\pgfqpoint{-0.048611in}{0.000000in}}%
\pgfusepath{stroke,fill}%
}%
\begin{pgfscope}%
\pgfsys@transformshift{1.000000in}{1.656330in}%
\pgfsys@useobject{currentmarker}{}%
\end{pgfscope}%
\end{pgfscope}%
\begin{pgfscope}%
\pgfsetbuttcap%
\pgfsetroundjoin%
\definecolor{currentfill}{rgb}{0.000000,0.000000,0.000000}%
\pgfsetfillcolor{currentfill}%
\pgfsetlinewidth{0.803000pt}%
\definecolor{currentstroke}{rgb}{0.000000,0.000000,0.000000}%
\pgfsetstrokecolor{currentstroke}%
\pgfsetdash{}{0pt}%
\pgfsys@defobject{currentmarker}{\pgfqpoint{0.000000in}{0.000000in}}{\pgfqpoint{0.048611in}{0.000000in}}{%
\pgfpathmoveto{\pgfqpoint{0.000000in}{0.000000in}}%
\pgfpathlineto{\pgfqpoint{0.048611in}{0.000000in}}%
\pgfusepath{stroke,fill}%
}%
\begin{pgfscope}%
\pgfsys@transformshift{7.200000in}{1.656330in}%
\pgfsys@useobject{currentmarker}{}%
\end{pgfscope}%
\end{pgfscope}%
\begin{pgfscope}%
\definecolor{textcolor}{rgb}{0.000000,0.000000,0.000000}%
\pgfsetstrokecolor{textcolor}%
\pgfsetfillcolor{textcolor}%
\pgftext[x=0.547838in, y=1.608104in, left, base]{\color{textcolor}\rmfamily\fontsize{10.000000}{12.000000}\selectfont \(\displaystyle {\ensuremath{-}0.10}\)}%
\end{pgfscope}%
\begin{pgfscope}%
\pgfpathrectangle{\pgfqpoint{1.000000in}{0.330000in}}{\pgfqpoint{6.200000in}{2.310000in}}%
\pgfusepath{clip}%
\pgfsetbuttcap%
\pgfsetroundjoin%
\pgfsetlinewidth{0.803000pt}%
\definecolor{currentstroke}{rgb}{0.000000,0.000000,0.000000}%
\pgfsetstrokecolor{currentstroke}%
\pgfsetdash{{0.800000pt}{1.320000pt}}{0.000000pt}%
\pgfpathmoveto{\pgfqpoint{1.000000in}{2.032339in}}%
\pgfpathlineto{\pgfqpoint{7.200000in}{2.032339in}}%
\pgfusepath{stroke}%
\end{pgfscope}%
\begin{pgfscope}%
\pgfsetbuttcap%
\pgfsetroundjoin%
\definecolor{currentfill}{rgb}{0.000000,0.000000,0.000000}%
\pgfsetfillcolor{currentfill}%
\pgfsetlinewidth{0.803000pt}%
\definecolor{currentstroke}{rgb}{0.000000,0.000000,0.000000}%
\pgfsetstrokecolor{currentstroke}%
\pgfsetdash{}{0pt}%
\pgfsys@defobject{currentmarker}{\pgfqpoint{-0.048611in}{0.000000in}}{\pgfqpoint{-0.000000in}{0.000000in}}{%
\pgfpathmoveto{\pgfqpoint{-0.000000in}{0.000000in}}%
\pgfpathlineto{\pgfqpoint{-0.048611in}{0.000000in}}%
\pgfusepath{stroke,fill}%
}%
\begin{pgfscope}%
\pgfsys@transformshift{1.000000in}{2.032339in}%
\pgfsys@useobject{currentmarker}{}%
\end{pgfscope}%
\end{pgfscope}%
\begin{pgfscope}%
\pgfsetbuttcap%
\pgfsetroundjoin%
\definecolor{currentfill}{rgb}{0.000000,0.000000,0.000000}%
\pgfsetfillcolor{currentfill}%
\pgfsetlinewidth{0.803000pt}%
\definecolor{currentstroke}{rgb}{0.000000,0.000000,0.000000}%
\pgfsetstrokecolor{currentstroke}%
\pgfsetdash{}{0pt}%
\pgfsys@defobject{currentmarker}{\pgfqpoint{0.000000in}{0.000000in}}{\pgfqpoint{0.048611in}{0.000000in}}{%
\pgfpathmoveto{\pgfqpoint{0.000000in}{0.000000in}}%
\pgfpathlineto{\pgfqpoint{0.048611in}{0.000000in}}%
\pgfusepath{stroke,fill}%
}%
\begin{pgfscope}%
\pgfsys@transformshift{7.200000in}{2.032339in}%
\pgfsys@useobject{currentmarker}{}%
\end{pgfscope}%
\end{pgfscope}%
\begin{pgfscope}%
\definecolor{textcolor}{rgb}{0.000000,0.000000,0.000000}%
\pgfsetstrokecolor{textcolor}%
\pgfsetfillcolor{textcolor}%
\pgftext[x=0.547838in, y=1.984114in, left, base]{\color{textcolor}\rmfamily\fontsize{10.000000}{12.000000}\selectfont \(\displaystyle {\ensuremath{-}0.05}\)}%
\end{pgfscope}%
\begin{pgfscope}%
\pgfpathrectangle{\pgfqpoint{1.000000in}{0.330000in}}{\pgfqpoint{6.200000in}{2.310000in}}%
\pgfusepath{clip}%
\pgfsetbuttcap%
\pgfsetroundjoin%
\pgfsetlinewidth{0.803000pt}%
\definecolor{currentstroke}{rgb}{0.000000,0.000000,0.000000}%
\pgfsetstrokecolor{currentstroke}%
\pgfsetdash{{0.800000pt}{1.320000pt}}{0.000000pt}%
\pgfpathmoveto{\pgfqpoint{1.000000in}{2.408348in}}%
\pgfpathlineto{\pgfqpoint{7.200000in}{2.408348in}}%
\pgfusepath{stroke}%
\end{pgfscope}%
\begin{pgfscope}%
\pgfsetbuttcap%
\pgfsetroundjoin%
\definecolor{currentfill}{rgb}{0.000000,0.000000,0.000000}%
\pgfsetfillcolor{currentfill}%
\pgfsetlinewidth{0.803000pt}%
\definecolor{currentstroke}{rgb}{0.000000,0.000000,0.000000}%
\pgfsetstrokecolor{currentstroke}%
\pgfsetdash{}{0pt}%
\pgfsys@defobject{currentmarker}{\pgfqpoint{-0.048611in}{0.000000in}}{\pgfqpoint{-0.000000in}{0.000000in}}{%
\pgfpathmoveto{\pgfqpoint{-0.000000in}{0.000000in}}%
\pgfpathlineto{\pgfqpoint{-0.048611in}{0.000000in}}%
\pgfusepath{stroke,fill}%
}%
\begin{pgfscope}%
\pgfsys@transformshift{1.000000in}{2.408348in}%
\pgfsys@useobject{currentmarker}{}%
\end{pgfscope}%
\end{pgfscope}%
\begin{pgfscope}%
\pgfsetbuttcap%
\pgfsetroundjoin%
\definecolor{currentfill}{rgb}{0.000000,0.000000,0.000000}%
\pgfsetfillcolor{currentfill}%
\pgfsetlinewidth{0.803000pt}%
\definecolor{currentstroke}{rgb}{0.000000,0.000000,0.000000}%
\pgfsetstrokecolor{currentstroke}%
\pgfsetdash{}{0pt}%
\pgfsys@defobject{currentmarker}{\pgfqpoint{0.000000in}{0.000000in}}{\pgfqpoint{0.048611in}{0.000000in}}{%
\pgfpathmoveto{\pgfqpoint{0.000000in}{0.000000in}}%
\pgfpathlineto{\pgfqpoint{0.048611in}{0.000000in}}%
\pgfusepath{stroke,fill}%
}%
\begin{pgfscope}%
\pgfsys@transformshift{7.200000in}{2.408348in}%
\pgfsys@useobject{currentmarker}{}%
\end{pgfscope}%
\end{pgfscope}%
\begin{pgfscope}%
\definecolor{textcolor}{rgb}{0.000000,0.000000,0.000000}%
\pgfsetstrokecolor{textcolor}%
\pgfsetfillcolor{textcolor}%
\pgftext[x=0.655863in, y=2.360123in, left, base]{\color{textcolor}\rmfamily\fontsize{10.000000}{12.000000}\selectfont \(\displaystyle {0.00}\)}%
\end{pgfscope}%
\begin{pgfscope}%
\pgfpathrectangle{\pgfqpoint{1.000000in}{0.330000in}}{\pgfqpoint{6.200000in}{2.310000in}}%
\pgfusepath{clip}%
\pgfsetrectcap%
\pgfsetroundjoin%
\pgfsetlinewidth{1.505625pt}%
\definecolor{currentstroke}{rgb}{0.121569,0.466667,0.705882}%
\pgfsetstrokecolor{currentstroke}%
\pgfsetdash{}{0pt}%
\pgfpathmoveto{\pgfqpoint{1.281818in}{2.408348in}}%
\pgfpathlineto{\pgfqpoint{1.603190in}{2.120273in}}%
\pgfpathlineto{\pgfqpoint{1.776236in}{1.967839in}}%
\pgfpathlineto{\pgfqpoint{1.924561in}{1.839816in}}%
\pgfpathlineto{\pgfqpoint{2.072887in}{1.714992in}}%
\pgfpathlineto{\pgfqpoint{2.196491in}{1.613935in}}%
\pgfpathlineto{\pgfqpoint{2.320096in}{1.516017in}}%
\pgfpathlineto{\pgfqpoint{2.418979in}{1.440220in}}%
\pgfpathlineto{\pgfqpoint{2.517863in}{1.366908in}}%
\pgfpathlineto{\pgfqpoint{2.616746in}{1.296293in}}%
\pgfpathlineto{\pgfqpoint{2.715630in}{1.228584in}}%
\pgfpathlineto{\pgfqpoint{2.814514in}{1.163991in}}%
\pgfpathlineto{\pgfqpoint{2.913397in}{1.102724in}}%
\pgfpathlineto{\pgfqpoint{2.987560in}{1.059082in}}%
\pgfpathlineto{\pgfqpoint{3.061722in}{1.017515in}}%
\pgfpathlineto{\pgfqpoint{3.135885in}{0.978111in}}%
\pgfpathlineto{\pgfqpoint{3.210048in}{0.940958in}}%
\pgfpathlineto{\pgfqpoint{3.284211in}{0.906142in}}%
\pgfpathlineto{\pgfqpoint{3.358373in}{0.873751in}}%
\pgfpathlineto{\pgfqpoint{3.432536in}{0.843872in}}%
\pgfpathlineto{\pgfqpoint{3.506699in}{0.816591in}}%
\pgfpathlineto{\pgfqpoint{3.580861in}{0.791994in}}%
\pgfpathlineto{\pgfqpoint{3.655024in}{0.770168in}}%
\pgfpathlineto{\pgfqpoint{3.729187in}{0.751198in}}%
\pgfpathlineto{\pgfqpoint{3.803349in}{0.735172in}}%
\pgfpathlineto{\pgfqpoint{3.877512in}{0.722173in}}%
\pgfpathlineto{\pgfqpoint{3.951675in}{0.712288in}}%
\pgfpathlineto{\pgfqpoint{4.025837in}{0.713054in}}%
\pgfpathlineto{\pgfqpoint{4.100000in}{0.716942in}}%
\pgfpathlineto{\pgfqpoint{4.174163in}{0.723877in}}%
\pgfpathlineto{\pgfqpoint{4.248325in}{0.733782in}}%
\pgfpathlineto{\pgfqpoint{4.322488in}{0.746581in}}%
\pgfpathlineto{\pgfqpoint{4.396651in}{0.762197in}}%
\pgfpathlineto{\pgfqpoint{4.470813in}{0.780553in}}%
\pgfpathlineto{\pgfqpoint{4.544976in}{0.801572in}}%
\pgfpathlineto{\pgfqpoint{4.619139in}{0.825176in}}%
\pgfpathlineto{\pgfqpoint{4.693301in}{0.851288in}}%
\pgfpathlineto{\pgfqpoint{4.767464in}{0.879830in}}%
\pgfpathlineto{\pgfqpoint{4.841627in}{0.910725in}}%
\pgfpathlineto{\pgfqpoint{4.915789in}{0.943893in}}%
\pgfpathlineto{\pgfqpoint{4.989952in}{0.979258in}}%
\pgfpathlineto{\pgfqpoint{5.064115in}{1.016739in}}%
\pgfpathlineto{\pgfqpoint{5.138278in}{1.056260in}}%
\pgfpathlineto{\pgfqpoint{5.212440in}{1.097740in}}%
\pgfpathlineto{\pgfqpoint{5.311324in}{1.155958in}}%
\pgfpathlineto{\pgfqpoint{5.410207in}{1.217332in}}%
\pgfpathlineto{\pgfqpoint{5.509091in}{1.281672in}}%
\pgfpathlineto{\pgfqpoint{5.607974in}{1.348790in}}%
\pgfpathlineto{\pgfqpoint{5.706858in}{1.418496in}}%
\pgfpathlineto{\pgfqpoint{5.805742in}{1.490599in}}%
\pgfpathlineto{\pgfqpoint{5.929346in}{1.583808in}}%
\pgfpathlineto{\pgfqpoint{6.052951in}{1.680090in}}%
\pgfpathlineto{\pgfqpoint{6.176555in}{1.779069in}}%
\pgfpathlineto{\pgfqpoint{6.324880in}{1.900873in}}%
\pgfpathlineto{\pgfqpoint{6.497927in}{2.046325in}}%
\pgfpathlineto{\pgfqpoint{6.695694in}{2.215691in}}%
\pgfpathlineto{\pgfqpoint{6.918182in}{2.408348in}}%
\pgfpathlineto{\pgfqpoint{6.918182in}{2.408348in}}%
\pgfusepath{stroke}%
\end{pgfscope}%
\begin{pgfscope}%
\pgfpathrectangle{\pgfqpoint{1.000000in}{0.330000in}}{\pgfqpoint{6.200000in}{2.310000in}}%
\pgfusepath{clip}%
\pgfsetrectcap%
\pgfsetroundjoin%
\pgfsetlinewidth{1.505625pt}%
\definecolor{currentstroke}{rgb}{1.000000,0.498039,0.054902}%
\pgfsetstrokecolor{currentstroke}%
\pgfsetdash{}{0pt}%
\pgfpathmoveto{\pgfqpoint{1.281818in}{2.408348in}}%
\pgfpathlineto{\pgfqpoint{1.652632in}{2.218300in}}%
\pgfpathlineto{\pgfqpoint{1.850399in}{2.119307in}}%
\pgfpathlineto{\pgfqpoint{2.023445in}{2.035058in}}%
\pgfpathlineto{\pgfqpoint{2.171770in}{1.965129in}}%
\pgfpathlineto{\pgfqpoint{2.320096in}{1.897745in}}%
\pgfpathlineto{\pgfqpoint{2.443700in}{1.843831in}}%
\pgfpathlineto{\pgfqpoint{2.567305in}{1.792204in}}%
\pgfpathlineto{\pgfqpoint{2.690909in}{1.743098in}}%
\pgfpathlineto{\pgfqpoint{2.814514in}{1.696745in}}%
\pgfpathlineto{\pgfqpoint{2.913397in}{1.661802in}}%
\pgfpathlineto{\pgfqpoint{3.012281in}{1.628888in}}%
\pgfpathlineto{\pgfqpoint{3.111164in}{1.598120in}}%
\pgfpathlineto{\pgfqpoint{3.210048in}{1.569615in}}%
\pgfpathlineto{\pgfqpoint{3.308931in}{1.543490in}}%
\pgfpathlineto{\pgfqpoint{3.407815in}{1.519861in}}%
\pgfpathlineto{\pgfqpoint{3.506699in}{1.498842in}}%
\pgfpathlineto{\pgfqpoint{3.605582in}{1.480548in}}%
\pgfpathlineto{\pgfqpoint{3.704466in}{1.465095in}}%
\pgfpathlineto{\pgfqpoint{3.778628in}{1.455437in}}%
\pgfpathlineto{\pgfqpoint{3.852791in}{1.447488in}}%
\pgfpathlineto{\pgfqpoint{3.902233in}{1.443162in}}%
\pgfpathlineto{\pgfqpoint{3.951675in}{1.439631in}}%
\pgfpathlineto{\pgfqpoint{4.025837in}{1.439983in}}%
\pgfpathlineto{\pgfqpoint{4.100000in}{1.442113in}}%
\pgfpathlineto{\pgfqpoint{4.174163in}{1.445979in}}%
\pgfpathlineto{\pgfqpoint{4.273046in}{1.453760in}}%
\pgfpathlineto{\pgfqpoint{4.371930in}{1.464451in}}%
\pgfpathlineto{\pgfqpoint{4.470813in}{1.477950in}}%
\pgfpathlineto{\pgfqpoint{4.569697in}{1.494156in}}%
\pgfpathlineto{\pgfqpoint{4.668581in}{1.512966in}}%
\pgfpathlineto{\pgfqpoint{4.767464in}{1.534276in}}%
\pgfpathlineto{\pgfqpoint{4.866348in}{1.557984in}}%
\pgfpathlineto{\pgfqpoint{4.965231in}{1.583986in}}%
\pgfpathlineto{\pgfqpoint{5.064115in}{1.612176in}}%
\pgfpathlineto{\pgfqpoint{5.162998in}{1.642450in}}%
\pgfpathlineto{\pgfqpoint{5.261882in}{1.674702in}}%
\pgfpathlineto{\pgfqpoint{5.385486in}{1.717636in}}%
\pgfpathlineto{\pgfqpoint{5.509091in}{1.763287in}}%
\pgfpathlineto{\pgfqpoint{5.632695in}{1.811443in}}%
\pgfpathlineto{\pgfqpoint{5.756300in}{1.861895in}}%
\pgfpathlineto{\pgfqpoint{5.904625in}{1.925168in}}%
\pgfpathlineto{\pgfqpoint{6.052951in}{1.991069in}}%
\pgfpathlineto{\pgfqpoint{6.225997in}{2.070780in}}%
\pgfpathlineto{\pgfqpoint{6.399043in}{2.152967in}}%
\pgfpathlineto{\pgfqpoint{6.621531in}{2.261306in}}%
\pgfpathlineto{\pgfqpoint{6.918182in}{2.408348in}}%
\pgfpathlineto{\pgfqpoint{6.918182in}{2.408348in}}%
\pgfusepath{stroke}%
\end{pgfscope}%
\begin{pgfscope}%
\pgfpathrectangle{\pgfqpoint{1.000000in}{0.330000in}}{\pgfqpoint{6.200000in}{2.310000in}}%
\pgfusepath{clip}%
\pgfsetrectcap%
\pgfsetroundjoin%
\pgfsetlinewidth{1.505625pt}%
\definecolor{currentstroke}{rgb}{0.172549,0.627451,0.172549}%
\pgfsetstrokecolor{currentstroke}%
\pgfsetdash{}{0pt}%
\pgfpathmoveto{\pgfqpoint{1.281818in}{2.408348in}}%
\pgfpathlineto{\pgfqpoint{1.652632in}{2.214169in}}%
\pgfpathlineto{\pgfqpoint{1.850399in}{2.112903in}}%
\pgfpathlineto{\pgfqpoint{2.023445in}{2.026632in}}%
\pgfpathlineto{\pgfqpoint{2.171770in}{1.954954in}}%
\pgfpathlineto{\pgfqpoint{2.320096in}{1.885815in}}%
\pgfpathlineto{\pgfqpoint{2.443700in}{1.830441in}}%
\pgfpathlineto{\pgfqpoint{2.567305in}{1.777365in}}%
\pgfpathlineto{\pgfqpoint{2.690909in}{1.726829in}}%
\pgfpathlineto{\pgfqpoint{2.814514in}{1.679076in}}%
\pgfpathlineto{\pgfqpoint{2.913397in}{1.643041in}}%
\pgfpathlineto{\pgfqpoint{3.012281in}{1.609066in}}%
\pgfpathlineto{\pgfqpoint{3.111164in}{1.577274in}}%
\pgfpathlineto{\pgfqpoint{3.210048in}{1.547791in}}%
\pgfpathlineto{\pgfqpoint{3.308931in}{1.520739in}}%
\pgfpathlineto{\pgfqpoint{3.407815in}{1.496242in}}%
\pgfpathlineto{\pgfqpoint{3.506699in}{1.474423in}}%
\pgfpathlineto{\pgfqpoint{3.605582in}{1.455407in}}%
\pgfpathlineto{\pgfqpoint{3.704466in}{1.439316in}}%
\pgfpathlineto{\pgfqpoint{3.778628in}{1.429241in}}%
\pgfpathlineto{\pgfqpoint{3.852791in}{1.420933in}}%
\pgfpathlineto{\pgfqpoint{3.902233in}{1.416401in}}%
\pgfpathlineto{\pgfqpoint{3.951675in}{1.412693in}}%
\pgfpathlineto{\pgfqpoint{4.025837in}{1.413260in}}%
\pgfpathlineto{\pgfqpoint{4.100000in}{1.415668in}}%
\pgfpathlineto{\pgfqpoint{4.174163in}{1.419871in}}%
\pgfpathlineto{\pgfqpoint{4.248325in}{1.425821in}}%
\pgfpathlineto{\pgfqpoint{4.322488in}{1.433473in}}%
\pgfpathlineto{\pgfqpoint{4.421372in}{1.446242in}}%
\pgfpathlineto{\pgfqpoint{4.520255in}{1.461841in}}%
\pgfpathlineto{\pgfqpoint{4.619139in}{1.480159in}}%
\pgfpathlineto{\pgfqpoint{4.718022in}{1.501087in}}%
\pgfpathlineto{\pgfqpoint{4.816906in}{1.524513in}}%
\pgfpathlineto{\pgfqpoint{4.915789in}{1.550326in}}%
\pgfpathlineto{\pgfqpoint{5.014673in}{1.578414in}}%
\pgfpathlineto{\pgfqpoint{5.113557in}{1.608667in}}%
\pgfpathlineto{\pgfqpoint{5.212440in}{1.640972in}}%
\pgfpathlineto{\pgfqpoint{5.311324in}{1.675219in}}%
\pgfpathlineto{\pgfqpoint{5.434928in}{1.720587in}}%
\pgfpathlineto{\pgfqpoint{5.533812in}{1.758794in}}%
\pgfpathlineto{\pgfqpoint{5.682137in}{1.819028in}}%
\pgfpathlineto{\pgfqpoint{5.805742in}{1.871663in}}%
\pgfpathlineto{\pgfqpoint{5.954067in}{1.937426in}}%
\pgfpathlineto{\pgfqpoint{6.102392in}{2.005669in}}%
\pgfpathlineto{\pgfqpoint{6.275439in}{2.087912in}}%
\pgfpathlineto{\pgfqpoint{6.473206in}{2.184632in}}%
\pgfpathlineto{\pgfqpoint{6.720415in}{2.308300in}}%
\pgfpathlineto{\pgfqpoint{6.918182in}{2.408348in}}%
\pgfpathlineto{\pgfqpoint{6.918182in}{2.408348in}}%
\pgfusepath{stroke}%
\end{pgfscope}%
\begin{pgfscope}%
\pgfpathrectangle{\pgfqpoint{1.000000in}{0.330000in}}{\pgfqpoint{6.200000in}{2.310000in}}%
\pgfusepath{clip}%
\pgfsetrectcap%
\pgfsetroundjoin%
\pgfsetlinewidth{1.505625pt}%
\definecolor{currentstroke}{rgb}{0.839216,0.152941,0.156863}%
\pgfsetstrokecolor{currentstroke}%
\pgfsetdash{}{0pt}%
\pgfpathmoveto{\pgfqpoint{1.281818in}{2.408348in}}%
\pgfpathlineto{\pgfqpoint{1.603190in}{2.087979in}}%
\pgfpathlineto{\pgfqpoint{1.776236in}{1.918447in}}%
\pgfpathlineto{\pgfqpoint{1.924561in}{1.776060in}}%
\pgfpathlineto{\pgfqpoint{2.048166in}{1.660082in}}%
\pgfpathlineto{\pgfqpoint{2.171770in}{1.547035in}}%
\pgfpathlineto{\pgfqpoint{2.295375in}{1.437380in}}%
\pgfpathlineto{\pgfqpoint{2.394258in}{1.352410in}}%
\pgfpathlineto{\pgfqpoint{2.493142in}{1.270143in}}%
\pgfpathlineto{\pgfqpoint{2.592026in}{1.190813in}}%
\pgfpathlineto{\pgfqpoint{2.690909in}{1.114655in}}%
\pgfpathlineto{\pgfqpoint{2.789793in}{1.041902in}}%
\pgfpathlineto{\pgfqpoint{2.863955in}{0.989713in}}%
\pgfpathlineto{\pgfqpoint{2.938118in}{0.939668in}}%
\pgfpathlineto{\pgfqpoint{3.012281in}{0.891866in}}%
\pgfpathlineto{\pgfqpoint{3.086443in}{0.846404in}}%
\pgfpathlineto{\pgfqpoint{3.160606in}{0.803380in}}%
\pgfpathlineto{\pgfqpoint{3.234769in}{0.762891in}}%
\pgfpathlineto{\pgfqpoint{3.308931in}{0.725034in}}%
\pgfpathlineto{\pgfqpoint{3.383094in}{0.689906in}}%
\pgfpathlineto{\pgfqpoint{3.457257in}{0.657604in}}%
\pgfpathlineto{\pgfqpoint{3.531419in}{0.628224in}}%
\pgfpathlineto{\pgfqpoint{3.605582in}{0.601863in}}%
\pgfpathlineto{\pgfqpoint{3.679745in}{0.578616in}}%
\pgfpathlineto{\pgfqpoint{3.753907in}{0.558579in}}%
\pgfpathlineto{\pgfqpoint{3.828070in}{0.541849in}}%
\pgfpathlineto{\pgfqpoint{3.877512in}{0.532579in}}%
\pgfpathlineto{\pgfqpoint{3.902233in}{0.528520in}}%
\pgfpathlineto{\pgfqpoint{3.951675in}{0.521571in}}%
\pgfpathlineto{\pgfqpoint{4.001116in}{0.521762in}}%
\pgfpathlineto{\pgfqpoint{4.050558in}{0.523509in}}%
\pgfpathlineto{\pgfqpoint{4.124721in}{0.528996in}}%
\pgfpathlineto{\pgfqpoint{4.198884in}{0.537844in}}%
\pgfpathlineto{\pgfqpoint{4.273046in}{0.549969in}}%
\pgfpathlineto{\pgfqpoint{4.347209in}{0.565285in}}%
\pgfpathlineto{\pgfqpoint{4.421372in}{0.583706in}}%
\pgfpathlineto{\pgfqpoint{4.495534in}{0.605146in}}%
\pgfpathlineto{\pgfqpoint{4.569697in}{0.629520in}}%
\pgfpathlineto{\pgfqpoint{4.643860in}{0.656740in}}%
\pgfpathlineto{\pgfqpoint{4.718022in}{0.686721in}}%
\pgfpathlineto{\pgfqpoint{4.792185in}{0.719375in}}%
\pgfpathlineto{\pgfqpoint{4.866348in}{0.754615in}}%
\pgfpathlineto{\pgfqpoint{4.940510in}{0.792355in}}%
\pgfpathlineto{\pgfqpoint{5.014673in}{0.832507in}}%
\pgfpathlineto{\pgfqpoint{5.088836in}{0.874984in}}%
\pgfpathlineto{\pgfqpoint{5.162998in}{0.919697in}}%
\pgfpathlineto{\pgfqpoint{5.237161in}{0.966559in}}%
\pgfpathlineto{\pgfqpoint{5.311324in}{1.015482in}}%
\pgfpathlineto{\pgfqpoint{5.410207in}{1.083765in}}%
\pgfpathlineto{\pgfqpoint{5.509091in}{1.155345in}}%
\pgfpathlineto{\pgfqpoint{5.607974in}{1.230010in}}%
\pgfpathlineto{\pgfqpoint{5.706858in}{1.307549in}}%
\pgfpathlineto{\pgfqpoint{5.805742in}{1.387752in}}%
\pgfpathlineto{\pgfqpoint{5.904625in}{1.470405in}}%
\pgfpathlineto{\pgfqpoint{6.028230in}{1.576841in}}%
\pgfpathlineto{\pgfqpoint{6.151834in}{1.686356in}}%
\pgfpathlineto{\pgfqpoint{6.300159in}{1.821245in}}%
\pgfpathlineto{\pgfqpoint{6.448485in}{1.959236in}}%
\pgfpathlineto{\pgfqpoint{6.621531in}{2.123171in}}%
\pgfpathlineto{\pgfqpoint{6.868740in}{2.360648in}}%
\pgfpathlineto{\pgfqpoint{6.918182in}{2.408348in}}%
\pgfpathlineto{\pgfqpoint{6.918182in}{2.408348in}}%
\pgfusepath{stroke}%
\end{pgfscope}%
\begin{pgfscope}%
\pgfpathrectangle{\pgfqpoint{1.000000in}{0.330000in}}{\pgfqpoint{6.200000in}{2.310000in}}%
\pgfusepath{clip}%
\pgfsetrectcap%
\pgfsetroundjoin%
\pgfsetlinewidth{1.505625pt}%
\definecolor{currentstroke}{rgb}{0.580392,0.403922,0.741176}%
\pgfsetstrokecolor{currentstroke}%
\pgfsetdash{}{0pt}%
\pgfpathmoveto{\pgfqpoint{1.281818in}{2.408348in}}%
\pgfpathlineto{\pgfqpoint{1.603190in}{2.081532in}}%
\pgfpathlineto{\pgfqpoint{1.776236in}{1.908436in}}%
\pgfpathlineto{\pgfqpoint{1.924561in}{1.762897in}}%
\pgfpathlineto{\pgfqpoint{2.048166in}{1.644205in}}%
\pgfpathlineto{\pgfqpoint{2.171770in}{1.528345in}}%
\pgfpathlineto{\pgfqpoint{2.295375in}{1.415764in}}%
\pgfpathlineto{\pgfqpoint{2.394258in}{1.328359in}}%
\pgfpathlineto{\pgfqpoint{2.493142in}{1.243566in}}%
\pgfpathlineto{\pgfqpoint{2.592026in}{1.161609in}}%
\pgfpathlineto{\pgfqpoint{2.690909in}{1.082716in}}%
\pgfpathlineto{\pgfqpoint{2.789793in}{1.007111in}}%
\pgfpathlineto{\pgfqpoint{2.888676in}{0.935019in}}%
\pgfpathlineto{\pgfqpoint{2.962839in}{0.883389in}}%
\pgfpathlineto{\pgfqpoint{3.037002in}{0.833956in}}%
\pgfpathlineto{\pgfqpoint{3.111164in}{0.786813in}}%
\pgfpathlineto{\pgfqpoint{3.185327in}{0.742053in}}%
\pgfpathlineto{\pgfqpoint{3.259490in}{0.699771in}}%
\pgfpathlineto{\pgfqpoint{3.333652in}{0.660059in}}%
\pgfpathlineto{\pgfqpoint{3.407815in}{0.623012in}}%
\pgfpathlineto{\pgfqpoint{3.481978in}{0.588721in}}%
\pgfpathlineto{\pgfqpoint{3.556140in}{0.557279in}}%
\pgfpathlineto{\pgfqpoint{3.630303in}{0.528780in}}%
\pgfpathlineto{\pgfqpoint{3.704466in}{0.503314in}}%
\pgfpathlineto{\pgfqpoint{3.778628in}{0.480975in}}%
\pgfpathlineto{\pgfqpoint{3.852791in}{0.461855in}}%
\pgfpathlineto{\pgfqpoint{3.902233in}{0.450941in}}%
\pgfpathlineto{\pgfqpoint{3.951675in}{0.441525in}}%
\pgfpathlineto{\pgfqpoint{4.001116in}{0.437843in}}%
\pgfpathlineto{\pgfqpoint{4.050558in}{0.435672in}}%
\pgfpathlineto{\pgfqpoint{4.124721in}{0.435222in}}%
\pgfpathlineto{\pgfqpoint{4.198884in}{0.438098in}}%
\pgfpathlineto{\pgfqpoint{4.273046in}{0.444255in}}%
\pgfpathlineto{\pgfqpoint{4.347209in}{0.453648in}}%
\pgfpathlineto{\pgfqpoint{4.421372in}{0.466230in}}%
\pgfpathlineto{\pgfqpoint{4.495534in}{0.481956in}}%
\pgfpathlineto{\pgfqpoint{4.569697in}{0.500779in}}%
\pgfpathlineto{\pgfqpoint{4.643860in}{0.522653in}}%
\pgfpathlineto{\pgfqpoint{4.718022in}{0.547530in}}%
\pgfpathlineto{\pgfqpoint{4.792185in}{0.575364in}}%
\pgfpathlineto{\pgfqpoint{4.866348in}{0.606108in}}%
\pgfpathlineto{\pgfqpoint{4.940510in}{0.639715in}}%
\pgfpathlineto{\pgfqpoint{5.014673in}{0.676136in}}%
\pgfpathlineto{\pgfqpoint{5.088836in}{0.715324in}}%
\pgfpathlineto{\pgfqpoint{5.162998in}{0.757232in}}%
\pgfpathlineto{\pgfqpoint{5.237161in}{0.801810in}}%
\pgfpathlineto{\pgfqpoint{5.311324in}{0.849011in}}%
\pgfpathlineto{\pgfqpoint{5.385486in}{0.898786in}}%
\pgfpathlineto{\pgfqpoint{5.459649in}{0.951086in}}%
\pgfpathlineto{\pgfqpoint{5.533812in}{1.005862in}}%
\pgfpathlineto{\pgfqpoint{5.583254in}{1.043731in}}%
\pgfpathlineto{\pgfqpoint{5.657416in}{1.106275in}}%
\pgfpathlineto{\pgfqpoint{5.756300in}{1.193164in}}%
\pgfpathlineto{\pgfqpoint{5.855183in}{1.283767in}}%
\pgfpathlineto{\pgfqpoint{5.954067in}{1.377772in}}%
\pgfpathlineto{\pgfqpoint{6.052951in}{1.474868in}}%
\pgfpathlineto{\pgfqpoint{6.151834in}{1.574742in}}%
\pgfpathlineto{\pgfqpoint{6.275439in}{1.703014in}}%
\pgfpathlineto{\pgfqpoint{6.399043in}{1.834525in}}%
\pgfpathlineto{\pgfqpoint{6.547368in}{1.995748in}}%
\pgfpathlineto{\pgfqpoint{6.720415in}{2.187199in}}%
\pgfpathlineto{\pgfqpoint{6.918182in}{2.408348in}}%
\pgfpathlineto{\pgfqpoint{6.918182in}{2.408348in}}%
\pgfusepath{stroke}%
\end{pgfscope}%
\begin{pgfscope}%
\pgfpathrectangle{\pgfqpoint{1.000000in}{0.330000in}}{\pgfqpoint{6.200000in}{2.310000in}}%
\pgfusepath{clip}%
\pgfsetrectcap%
\pgfsetroundjoin%
\pgfsetlinewidth{1.505625pt}%
\definecolor{currentstroke}{rgb}{0.549020,0.337255,0.294118}%
\pgfsetstrokecolor{currentstroke}%
\pgfsetdash{}{0pt}%
\pgfpathmoveto{\pgfqpoint{1.281818in}{2.408348in}}%
\pgfpathlineto{\pgfqpoint{1.726794in}{2.282733in}}%
\pgfpathlineto{\pgfqpoint{1.974003in}{2.215347in}}%
\pgfpathlineto{\pgfqpoint{2.171770in}{2.163596in}}%
\pgfpathlineto{\pgfqpoint{2.344817in}{2.120351in}}%
\pgfpathlineto{\pgfqpoint{2.517863in}{2.079381in}}%
\pgfpathlineto{\pgfqpoint{2.666188in}{2.046346in}}%
\pgfpathlineto{\pgfqpoint{2.814514in}{2.015466in}}%
\pgfpathlineto{\pgfqpoint{2.962839in}{1.986963in}}%
\pgfpathlineto{\pgfqpoint{3.111164in}{1.961056in}}%
\pgfpathlineto{\pgfqpoint{3.234769in}{1.941606in}}%
\pgfpathlineto{\pgfqpoint{3.358373in}{1.924234in}}%
\pgfpathlineto{\pgfqpoint{3.481978in}{1.909065in}}%
\pgfpathlineto{\pgfqpoint{3.605582in}{1.896222in}}%
\pgfpathlineto{\pgfqpoint{3.729187in}{1.885827in}}%
\pgfpathlineto{\pgfqpoint{3.852791in}{1.878003in}}%
\pgfpathlineto{\pgfqpoint{3.951675in}{1.873675in}}%
\pgfpathlineto{\pgfqpoint{4.050558in}{1.874138in}}%
\pgfpathlineto{\pgfqpoint{4.174163in}{1.877142in}}%
\pgfpathlineto{\pgfqpoint{4.297767in}{1.882741in}}%
\pgfpathlineto{\pgfqpoint{4.421372in}{1.890827in}}%
\pgfpathlineto{\pgfqpoint{4.544976in}{1.901291in}}%
\pgfpathlineto{\pgfqpoint{4.668581in}{1.914024in}}%
\pgfpathlineto{\pgfqpoint{4.792185in}{1.928916in}}%
\pgfpathlineto{\pgfqpoint{4.915789in}{1.945855in}}%
\pgfpathlineto{\pgfqpoint{5.039394in}{1.964728in}}%
\pgfpathlineto{\pgfqpoint{5.187719in}{1.989773in}}%
\pgfpathlineto{\pgfqpoint{5.336045in}{2.017245in}}%
\pgfpathlineto{\pgfqpoint{5.484370in}{2.046948in}}%
\pgfpathlineto{\pgfqpoint{5.657416in}{2.084153in}}%
\pgfpathlineto{\pgfqpoint{5.830463in}{2.123803in}}%
\pgfpathlineto{\pgfqpoint{6.028230in}{2.171696in}}%
\pgfpathlineto{\pgfqpoint{6.225997in}{2.221874in}}%
\pgfpathlineto{\pgfqpoint{6.473206in}{2.287060in}}%
\pgfpathlineto{\pgfqpoint{6.794577in}{2.374393in}}%
\pgfpathlineto{\pgfqpoint{6.918182in}{2.408348in}}%
\pgfpathlineto{\pgfqpoint{6.918182in}{2.408348in}}%
\pgfusepath{stroke}%
\end{pgfscope}%
\begin{pgfscope}%
\pgfpathrectangle{\pgfqpoint{1.000000in}{0.330000in}}{\pgfqpoint{6.200000in}{2.310000in}}%
\pgfusepath{clip}%
\pgfsetrectcap%
\pgfsetroundjoin%
\pgfsetlinewidth{1.505625pt}%
\definecolor{currentstroke}{rgb}{0.890196,0.466667,0.760784}%
\pgfsetstrokecolor{currentstroke}%
\pgfsetdash{}{0pt}%
\pgfpathmoveto{\pgfqpoint{1.281818in}{2.408348in}}%
\pgfpathlineto{\pgfqpoint{1.603190in}{2.130111in}}%
\pgfpathlineto{\pgfqpoint{1.776236in}{1.982886in}}%
\pgfpathlineto{\pgfqpoint{1.924561in}{1.859242in}}%
\pgfpathlineto{\pgfqpoint{2.072887in}{1.738689in}}%
\pgfpathlineto{\pgfqpoint{2.196491in}{1.641094in}}%
\pgfpathlineto{\pgfqpoint{2.320096in}{1.546531in}}%
\pgfpathlineto{\pgfqpoint{2.418979in}{1.473333in}}%
\pgfpathlineto{\pgfqpoint{2.517863in}{1.402537in}}%
\pgfpathlineto{\pgfqpoint{2.616746in}{1.334346in}}%
\pgfpathlineto{\pgfqpoint{2.715630in}{1.268963in}}%
\pgfpathlineto{\pgfqpoint{2.814514in}{1.206591in}}%
\pgfpathlineto{\pgfqpoint{2.913397in}{1.147432in}}%
\pgfpathlineto{\pgfqpoint{2.987560in}{1.105292in}}%
\pgfpathlineto{\pgfqpoint{3.061722in}{1.065157in}}%
\pgfpathlineto{\pgfqpoint{3.135885in}{1.027111in}}%
\pgfpathlineto{\pgfqpoint{3.210048in}{0.991239in}}%
\pgfpathlineto{\pgfqpoint{3.284211in}{0.957625in}}%
\pgfpathlineto{\pgfqpoint{3.358373in}{0.926353in}}%
\pgfpathlineto{\pgfqpoint{3.432536in}{0.897506in}}%
\pgfpathlineto{\pgfqpoint{3.506699in}{0.871168in}}%
\pgfpathlineto{\pgfqpoint{3.580861in}{0.847423in}}%
\pgfpathlineto{\pgfqpoint{3.655024in}{0.826353in}}%
\pgfpathlineto{\pgfqpoint{3.729187in}{0.808042in}}%
\pgfpathlineto{\pgfqpoint{3.803349in}{0.792571in}}%
\pgfpathlineto{\pgfqpoint{3.877512in}{0.780025in}}%
\pgfpathlineto{\pgfqpoint{3.951675in}{0.770485in}}%
\pgfpathlineto{\pgfqpoint{4.025837in}{0.771215in}}%
\pgfpathlineto{\pgfqpoint{4.100000in}{0.774959in}}%
\pgfpathlineto{\pgfqpoint{4.174163in}{0.781646in}}%
\pgfpathlineto{\pgfqpoint{4.248325in}{0.791200in}}%
\pgfpathlineto{\pgfqpoint{4.322488in}{0.803549in}}%
\pgfpathlineto{\pgfqpoint{4.396651in}{0.818618in}}%
\pgfpathlineto{\pgfqpoint{4.470813in}{0.836333in}}%
\pgfpathlineto{\pgfqpoint{4.544976in}{0.856619in}}%
\pgfpathlineto{\pgfqpoint{4.619139in}{0.879403in}}%
\pgfpathlineto{\pgfqpoint{4.693301in}{0.904609in}}%
\pgfpathlineto{\pgfqpoint{4.767464in}{0.932162in}}%
\pgfpathlineto{\pgfqpoint{4.841627in}{0.961987in}}%
\pgfpathlineto{\pgfqpoint{4.915789in}{0.994009in}}%
\pgfpathlineto{\pgfqpoint{4.989952in}{1.028152in}}%
\pgfpathlineto{\pgfqpoint{5.064115in}{1.064340in}}%
\pgfpathlineto{\pgfqpoint{5.138278in}{1.102498in}}%
\pgfpathlineto{\pgfqpoint{5.237161in}{1.156307in}}%
\pgfpathlineto{\pgfqpoint{5.336045in}{1.213302in}}%
\pgfpathlineto{\pgfqpoint{5.434928in}{1.273299in}}%
\pgfpathlineto{\pgfqpoint{5.533812in}{1.336118in}}%
\pgfpathlineto{\pgfqpoint{5.657416in}{1.418331in}}%
\pgfpathlineto{\pgfqpoint{5.756300in}{1.486828in}}%
\pgfpathlineto{\pgfqpoint{5.879904in}{1.575558in}}%
\pgfpathlineto{\pgfqpoint{6.003509in}{1.667402in}}%
\pgfpathlineto{\pgfqpoint{6.127113in}{1.761999in}}%
\pgfpathlineto{\pgfqpoint{6.275439in}{1.878638in}}%
\pgfpathlineto{\pgfqpoint{6.423764in}{1.998084in}}%
\pgfpathlineto{\pgfqpoint{6.596810in}{2.140141in}}%
\pgfpathlineto{\pgfqpoint{6.844019in}{2.346201in}}%
\pgfpathlineto{\pgfqpoint{6.918182in}{2.408348in}}%
\pgfpathlineto{\pgfqpoint{6.918182in}{2.408348in}}%
\pgfusepath{stroke}%
\end{pgfscope}%
\begin{pgfscope}%
\pgfpathrectangle{\pgfqpoint{1.000000in}{0.330000in}}{\pgfqpoint{6.200000in}{2.310000in}}%
\pgfusepath{clip}%
\pgfsetrectcap%
\pgfsetroundjoin%
\pgfsetlinewidth{1.505625pt}%
\definecolor{currentstroke}{rgb}{0.498039,0.498039,0.498039}%
\pgfsetstrokecolor{currentstroke}%
\pgfsetdash{}{0pt}%
\pgfpathmoveto{\pgfqpoint{1.281818in}{2.408348in}}%
\pgfpathlineto{\pgfqpoint{1.627911in}{2.175225in}}%
\pgfpathlineto{\pgfqpoint{1.825678in}{2.044522in}}%
\pgfpathlineto{\pgfqpoint{1.998724in}{1.932808in}}%
\pgfpathlineto{\pgfqpoint{2.147049in}{1.839637in}}%
\pgfpathlineto{\pgfqpoint{2.295375in}{1.749361in}}%
\pgfpathlineto{\pgfqpoint{2.418979in}{1.676688in}}%
\pgfpathlineto{\pgfqpoint{2.542584in}{1.606635in}}%
\pgfpathlineto{\pgfqpoint{2.666188in}{1.539476in}}%
\pgfpathlineto{\pgfqpoint{2.765072in}{1.488016in}}%
\pgfpathlineto{\pgfqpoint{2.863955in}{1.438722in}}%
\pgfpathlineto{\pgfqpoint{2.962839in}{1.391733in}}%
\pgfpathlineto{\pgfqpoint{3.061722in}{1.347185in}}%
\pgfpathlineto{\pgfqpoint{3.160606in}{1.305218in}}%
\pgfpathlineto{\pgfqpoint{3.259490in}{1.265966in}}%
\pgfpathlineto{\pgfqpoint{3.358373in}{1.229567in}}%
\pgfpathlineto{\pgfqpoint{3.457257in}{1.196156in}}%
\pgfpathlineto{\pgfqpoint{3.531419in}{1.173139in}}%
\pgfpathlineto{\pgfqpoint{3.605582in}{1.151936in}}%
\pgfpathlineto{\pgfqpoint{3.679745in}{1.132602in}}%
\pgfpathlineto{\pgfqpoint{3.753907in}{1.115194in}}%
\pgfpathlineto{\pgfqpoint{3.828070in}{1.099768in}}%
\pgfpathlineto{\pgfqpoint{3.902233in}{1.086379in}}%
\pgfpathlineto{\pgfqpoint{3.951675in}{1.078614in}}%
\pgfpathlineto{\pgfqpoint{4.025837in}{1.071337in}}%
\pgfpathlineto{\pgfqpoint{4.100000in}{1.066190in}}%
\pgfpathlineto{\pgfqpoint{4.174163in}{1.063173in}}%
\pgfpathlineto{\pgfqpoint{4.248325in}{1.062283in}}%
\pgfpathlineto{\pgfqpoint{4.322488in}{1.063521in}}%
\pgfpathlineto{\pgfqpoint{4.396651in}{1.066883in}}%
\pgfpathlineto{\pgfqpoint{4.470813in}{1.072370in}}%
\pgfpathlineto{\pgfqpoint{4.544976in}{1.079979in}}%
\pgfpathlineto{\pgfqpoint{4.619139in}{1.089708in}}%
\pgfpathlineto{\pgfqpoint{4.693301in}{1.101554in}}%
\pgfpathlineto{\pgfqpoint{4.767464in}{1.115517in}}%
\pgfpathlineto{\pgfqpoint{4.841627in}{1.131592in}}%
\pgfpathlineto{\pgfqpoint{4.915789in}{1.149777in}}%
\pgfpathlineto{\pgfqpoint{4.989952in}{1.170070in}}%
\pgfpathlineto{\pgfqpoint{5.064115in}{1.192466in}}%
\pgfpathlineto{\pgfqpoint{5.138278in}{1.216964in}}%
\pgfpathlineto{\pgfqpoint{5.212440in}{1.243559in}}%
\pgfpathlineto{\pgfqpoint{5.286603in}{1.272247in}}%
\pgfpathlineto{\pgfqpoint{5.360766in}{1.303025in}}%
\pgfpathlineto{\pgfqpoint{5.434928in}{1.335888in}}%
\pgfpathlineto{\pgfqpoint{5.509091in}{1.370832in}}%
\pgfpathlineto{\pgfqpoint{5.583254in}{1.407854in}}%
\pgfpathlineto{\pgfqpoint{5.657416in}{1.451948in}}%
\pgfpathlineto{\pgfqpoint{5.756300in}{1.513768in}}%
\pgfpathlineto{\pgfqpoint{5.855183in}{1.578803in}}%
\pgfpathlineto{\pgfqpoint{5.954067in}{1.646783in}}%
\pgfpathlineto{\pgfqpoint{6.052951in}{1.717440in}}%
\pgfpathlineto{\pgfqpoint{6.151834in}{1.790501in}}%
\pgfpathlineto{\pgfqpoint{6.275439in}{1.884799in}}%
\pgfpathlineto{\pgfqpoint{6.399043in}{1.981903in}}%
\pgfpathlineto{\pgfqpoint{6.547368in}{2.101381in}}%
\pgfpathlineto{\pgfqpoint{6.720415in}{2.243682in}}%
\pgfpathlineto{\pgfqpoint{6.918182in}{2.408348in}}%
\pgfpathlineto{\pgfqpoint{6.918182in}{2.408348in}}%
\pgfusepath{stroke}%
\end{pgfscope}%
\begin{pgfscope}%
\pgfpathrectangle{\pgfqpoint{1.000000in}{0.330000in}}{\pgfqpoint{6.200000in}{2.310000in}}%
\pgfusepath{clip}%
\pgfsetrectcap%
\pgfsetroundjoin%
\pgfsetlinewidth{1.505625pt}%
\definecolor{currentstroke}{rgb}{0.737255,0.741176,0.133333}%
\pgfsetstrokecolor{currentstroke}%
\pgfsetdash{}{0pt}%
\pgfpathmoveto{\pgfqpoint{1.281818in}{2.408348in}}%
\pgfpathlineto{\pgfqpoint{1.652632in}{2.233414in}}%
\pgfpathlineto{\pgfqpoint{1.875120in}{2.131088in}}%
\pgfpathlineto{\pgfqpoint{2.048166in}{2.053903in}}%
\pgfpathlineto{\pgfqpoint{2.196491in}{1.989922in}}%
\pgfpathlineto{\pgfqpoint{2.344817in}{1.928352in}}%
\pgfpathlineto{\pgfqpoint{2.468421in}{1.879157in}}%
\pgfpathlineto{\pgfqpoint{2.592026in}{1.832114in}}%
\pgfpathlineto{\pgfqpoint{2.715630in}{1.787438in}}%
\pgfpathlineto{\pgfqpoint{2.839234in}{1.745343in}}%
\pgfpathlineto{\pgfqpoint{2.962839in}{1.706040in}}%
\pgfpathlineto{\pgfqpoint{3.061722in}{1.676751in}}%
\pgfpathlineto{\pgfqpoint{3.160606in}{1.649492in}}%
\pgfpathlineto{\pgfqpoint{3.259490in}{1.624370in}}%
\pgfpathlineto{\pgfqpoint{3.358373in}{1.601491in}}%
\pgfpathlineto{\pgfqpoint{3.457257in}{1.580962in}}%
\pgfpathlineto{\pgfqpoint{3.556140in}{1.562888in}}%
\pgfpathlineto{\pgfqpoint{3.655024in}{1.547373in}}%
\pgfpathlineto{\pgfqpoint{3.753907in}{1.534522in}}%
\pgfpathlineto{\pgfqpoint{3.852791in}{1.524439in}}%
\pgfpathlineto{\pgfqpoint{3.951675in}{1.517226in}}%
\pgfpathlineto{\pgfqpoint{4.050558in}{1.517997in}}%
\pgfpathlineto{\pgfqpoint{4.149442in}{1.521651in}}%
\pgfpathlineto{\pgfqpoint{4.248325in}{1.528095in}}%
\pgfpathlineto{\pgfqpoint{4.347209in}{1.537239in}}%
\pgfpathlineto{\pgfqpoint{4.446093in}{1.548989in}}%
\pgfpathlineto{\pgfqpoint{4.544976in}{1.563253in}}%
\pgfpathlineto{\pgfqpoint{4.643860in}{1.579937in}}%
\pgfpathlineto{\pgfqpoint{4.742743in}{1.598947in}}%
\pgfpathlineto{\pgfqpoint{4.841627in}{1.620188in}}%
\pgfpathlineto{\pgfqpoint{4.940510in}{1.643565in}}%
\pgfpathlineto{\pgfqpoint{5.039394in}{1.668982in}}%
\pgfpathlineto{\pgfqpoint{5.138278in}{1.696343in}}%
\pgfpathlineto{\pgfqpoint{5.261882in}{1.733131in}}%
\pgfpathlineto{\pgfqpoint{5.385486in}{1.772615in}}%
\pgfpathlineto{\pgfqpoint{5.509091in}{1.814603in}}%
\pgfpathlineto{\pgfqpoint{5.632695in}{1.858903in}}%
\pgfpathlineto{\pgfqpoint{5.756300in}{1.905320in}}%
\pgfpathlineto{\pgfqpoint{5.904625in}{1.963541in}}%
\pgfpathlineto{\pgfqpoint{6.052951in}{2.024190in}}%
\pgfpathlineto{\pgfqpoint{6.225997in}{2.097558in}}%
\pgfpathlineto{\pgfqpoint{6.423764in}{2.184177in}}%
\pgfpathlineto{\pgfqpoint{6.646252in}{2.284165in}}%
\pgfpathlineto{\pgfqpoint{6.918182in}{2.408348in}}%
\pgfpathlineto{\pgfqpoint{6.918182in}{2.408348in}}%
\pgfusepath{stroke}%
\end{pgfscope}%
\begin{pgfscope}%
\pgfpathrectangle{\pgfqpoint{1.000000in}{0.330000in}}{\pgfqpoint{6.200000in}{2.310000in}}%
\pgfusepath{clip}%
\pgfsetrectcap%
\pgfsetroundjoin%
\pgfsetlinewidth{1.505625pt}%
\definecolor{currentstroke}{rgb}{0.090196,0.745098,0.811765}%
\pgfsetstrokecolor{currentstroke}%
\pgfsetdash{}{0pt}%
\pgfpathmoveto{\pgfqpoint{1.281818in}{2.408348in}}%
\pgfpathlineto{\pgfqpoint{1.652632in}{2.259099in}}%
\pgfpathlineto{\pgfqpoint{1.850399in}{2.181727in}}%
\pgfpathlineto{\pgfqpoint{2.023445in}{2.116248in}}%
\pgfpathlineto{\pgfqpoint{2.171770in}{2.062262in}}%
\pgfpathlineto{\pgfqpoint{2.320096in}{2.010654in}}%
\pgfpathlineto{\pgfqpoint{2.443700in}{1.969738in}}%
\pgfpathlineto{\pgfqpoint{2.567305in}{1.930955in}}%
\pgfpathlineto{\pgfqpoint{2.690909in}{1.894523in}}%
\pgfpathlineto{\pgfqpoint{2.814514in}{1.860656in}}%
\pgfpathlineto{\pgfqpoint{2.938118in}{1.829567in}}%
\pgfpathlineto{\pgfqpoint{3.037002in}{1.806839in}}%
\pgfpathlineto{\pgfqpoint{3.135885in}{1.786134in}}%
\pgfpathlineto{\pgfqpoint{3.234769in}{1.767558in}}%
\pgfpathlineto{\pgfqpoint{3.333652in}{1.751219in}}%
\pgfpathlineto{\pgfqpoint{3.432536in}{1.737222in}}%
\pgfpathlineto{\pgfqpoint{3.531419in}{1.725673in}}%
\pgfpathlineto{\pgfqpoint{3.630303in}{1.716676in}}%
\pgfpathlineto{\pgfqpoint{3.729187in}{1.710337in}}%
\pgfpathlineto{\pgfqpoint{3.828070in}{1.706759in}}%
\pgfpathlineto{\pgfqpoint{3.902233in}{1.705949in}}%
\pgfpathlineto{\pgfqpoint{3.951675in}{1.706325in}}%
\pgfpathlineto{\pgfqpoint{4.050558in}{1.717311in}}%
\pgfpathlineto{\pgfqpoint{4.149442in}{1.731084in}}%
\pgfpathlineto{\pgfqpoint{4.248325in}{1.747437in}}%
\pgfpathlineto{\pgfqpoint{4.347209in}{1.766162in}}%
\pgfpathlineto{\pgfqpoint{4.446093in}{1.787053in}}%
\pgfpathlineto{\pgfqpoint{4.544976in}{1.809901in}}%
\pgfpathlineto{\pgfqpoint{4.668581in}{1.840896in}}%
\pgfpathlineto{\pgfqpoint{4.792185in}{1.874213in}}%
\pgfpathlineto{\pgfqpoint{4.940510in}{1.916683in}}%
\pgfpathlineto{\pgfqpoint{5.113557in}{1.968761in}}%
\pgfpathlineto{\pgfqpoint{5.385486in}{2.053476in}}%
\pgfpathlineto{\pgfqpoint{5.583254in}{2.115102in}}%
\pgfpathlineto{\pgfqpoint{5.805742in}{2.167477in}}%
\pgfpathlineto{\pgfqpoint{6.052951in}{2.223377in}}%
\pgfpathlineto{\pgfqpoint{6.349601in}{2.288089in}}%
\pgfpathlineto{\pgfqpoint{6.769856in}{2.377184in}}%
\pgfpathlineto{\pgfqpoint{6.918182in}{2.408348in}}%
\pgfpathlineto{\pgfqpoint{6.918182in}{2.408348in}}%
\pgfusepath{stroke}%
\end{pgfscope}%
\begin{pgfscope}%
\pgfpathrectangle{\pgfqpoint{1.000000in}{0.330000in}}{\pgfqpoint{6.200000in}{2.310000in}}%
\pgfusepath{clip}%
\pgfsetrectcap%
\pgfsetroundjoin%
\pgfsetlinewidth{1.505625pt}%
\definecolor{currentstroke}{rgb}{0.121569,0.466667,0.705882}%
\pgfsetstrokecolor{currentstroke}%
\pgfsetdash{}{0pt}%
\pgfpathmoveto{\pgfqpoint{1.281818in}{2.408348in}}%
\pgfpathlineto{\pgfqpoint{1.751515in}{2.256583in}}%
\pgfpathlineto{\pgfqpoint{1.998724in}{2.178968in}}%
\pgfpathlineto{\pgfqpoint{2.221212in}{2.111467in}}%
\pgfpathlineto{\pgfqpoint{2.418979in}{2.053910in}}%
\pgfpathlineto{\pgfqpoint{2.592026in}{2.005822in}}%
\pgfpathlineto{\pgfqpoint{2.765072in}{1.960188in}}%
\pgfpathlineto{\pgfqpoint{2.913397in}{1.923263in}}%
\pgfpathlineto{\pgfqpoint{3.061722in}{1.888567in}}%
\pgfpathlineto{\pgfqpoint{3.210048in}{1.856295in}}%
\pgfpathlineto{\pgfqpoint{3.358373in}{1.826641in}}%
\pgfpathlineto{\pgfqpoint{3.481978in}{1.804068in}}%
\pgfpathlineto{\pgfqpoint{3.605582in}{1.783558in}}%
\pgfpathlineto{\pgfqpoint{3.729187in}{1.765222in}}%
\pgfpathlineto{\pgfqpoint{3.852791in}{1.749169in}}%
\pgfpathlineto{\pgfqpoint{3.976396in}{1.735443in}}%
\pgfpathlineto{\pgfqpoint{4.100000in}{1.723961in}}%
\pgfpathlineto{\pgfqpoint{4.223604in}{1.715113in}}%
\pgfpathlineto{\pgfqpoint{4.322488in}{1.710016in}}%
\pgfpathlineto{\pgfqpoint{4.421372in}{1.706753in}}%
\pgfpathlineto{\pgfqpoint{4.520255in}{1.705387in}}%
\pgfpathlineto{\pgfqpoint{4.619139in}{1.705982in}}%
\pgfpathlineto{\pgfqpoint{4.718022in}{1.708605in}}%
\pgfpathlineto{\pgfqpoint{4.816906in}{1.713318in}}%
\pgfpathlineto{\pgfqpoint{4.915789in}{1.720186in}}%
\pgfpathlineto{\pgfqpoint{5.014673in}{1.729273in}}%
\pgfpathlineto{\pgfqpoint{5.113557in}{1.740641in}}%
\pgfpathlineto{\pgfqpoint{5.212440in}{1.754356in}}%
\pgfpathlineto{\pgfqpoint{5.311324in}{1.770479in}}%
\pgfpathlineto{\pgfqpoint{5.410207in}{1.789074in}}%
\pgfpathlineto{\pgfqpoint{5.509091in}{1.810204in}}%
\pgfpathlineto{\pgfqpoint{5.583254in}{1.827752in}}%
\pgfpathlineto{\pgfqpoint{5.682137in}{1.860117in}}%
\pgfpathlineto{\pgfqpoint{5.781021in}{1.894932in}}%
\pgfpathlineto{\pgfqpoint{5.879904in}{1.932002in}}%
\pgfpathlineto{\pgfqpoint{5.978788in}{1.971133in}}%
\pgfpathlineto{\pgfqpoint{6.102392in}{2.022648in}}%
\pgfpathlineto{\pgfqpoint{6.225997in}{2.076698in}}%
\pgfpathlineto{\pgfqpoint{6.374322in}{2.144368in}}%
\pgfpathlineto{\pgfqpoint{6.522648in}{2.214479in}}%
\pgfpathlineto{\pgfqpoint{6.720415in}{2.310616in}}%
\pgfpathlineto{\pgfqpoint{6.918182in}{2.408348in}}%
\pgfpathlineto{\pgfqpoint{6.918182in}{2.408348in}}%
\pgfusepath{stroke}%
\end{pgfscope}%
\begin{pgfscope}%
\pgfpathrectangle{\pgfqpoint{1.000000in}{0.330000in}}{\pgfqpoint{6.200000in}{2.310000in}}%
\pgfusepath{clip}%
\pgfsetrectcap%
\pgfsetroundjoin%
\pgfsetlinewidth{1.505625pt}%
\definecolor{currentstroke}{rgb}{1.000000,0.498039,0.054902}%
\pgfsetstrokecolor{currentstroke}%
\pgfsetdash{}{0pt}%
\pgfpathmoveto{\pgfqpoint{1.281818in}{2.408348in}}%
\pgfpathlineto{\pgfqpoint{1.850399in}{2.378599in}}%
\pgfpathlineto{\pgfqpoint{2.171770in}{2.364123in}}%
\pgfpathlineto{\pgfqpoint{2.443700in}{2.354181in}}%
\pgfpathlineto{\pgfqpoint{2.690909in}{2.347532in}}%
\pgfpathlineto{\pgfqpoint{2.913397in}{2.343885in}}%
\pgfpathlineto{\pgfqpoint{3.111164in}{2.342776in}}%
\pgfpathlineto{\pgfqpoint{3.308931in}{2.343910in}}%
\pgfpathlineto{\pgfqpoint{3.481978in}{2.346917in}}%
\pgfpathlineto{\pgfqpoint{3.655024in}{2.351960in}}%
\pgfpathlineto{\pgfqpoint{3.828070in}{2.359182in}}%
\pgfpathlineto{\pgfqpoint{3.951675in}{2.365755in}}%
\pgfpathlineto{\pgfqpoint{4.124721in}{2.384419in}}%
\pgfpathlineto{\pgfqpoint{4.347209in}{2.410783in}}%
\pgfpathlineto{\pgfqpoint{4.940510in}{2.482623in}}%
\pgfpathlineto{\pgfqpoint{5.113557in}{2.500837in}}%
\pgfpathlineto{\pgfqpoint{5.261882in}{2.514410in}}%
\pgfpathlineto{\pgfqpoint{5.410207in}{2.525598in}}%
\pgfpathlineto{\pgfqpoint{5.533812in}{2.532755in}}%
\pgfpathlineto{\pgfqpoint{5.583254in}{2.535000in}}%
\pgfpathlineto{\pgfqpoint{5.706858in}{2.530629in}}%
\pgfpathlineto{\pgfqpoint{5.830463in}{2.524106in}}%
\pgfpathlineto{\pgfqpoint{5.978788in}{2.513752in}}%
\pgfpathlineto{\pgfqpoint{6.127113in}{2.501005in}}%
\pgfpathlineto{\pgfqpoint{6.300159in}{2.483618in}}%
\pgfpathlineto{\pgfqpoint{6.497927in}{2.461174in}}%
\pgfpathlineto{\pgfqpoint{6.745136in}{2.430577in}}%
\pgfpathlineto{\pgfqpoint{6.918182in}{2.408348in}}%
\pgfpathlineto{\pgfqpoint{6.918182in}{2.408348in}}%
\pgfusepath{stroke}%
\end{pgfscope}%
\begin{pgfscope}%
\pgfpathrectangle{\pgfqpoint{1.000000in}{0.330000in}}{\pgfqpoint{6.200000in}{2.310000in}}%
\pgfusepath{clip}%
\pgfsetrectcap%
\pgfsetroundjoin%
\pgfsetlinewidth{1.505625pt}%
\definecolor{currentstroke}{rgb}{0.172549,0.627451,0.172549}%
\pgfsetstrokecolor{currentstroke}%
\pgfsetdash{}{0pt}%
\pgfpathmoveto{\pgfqpoint{1.281818in}{2.408348in}}%
\pgfpathlineto{\pgfqpoint{1.603190in}{2.146803in}}%
\pgfpathlineto{\pgfqpoint{1.776236in}{2.008586in}}%
\pgfpathlineto{\pgfqpoint{1.924561in}{1.892679in}}%
\pgfpathlineto{\pgfqpoint{2.048166in}{1.798436in}}%
\pgfpathlineto{\pgfqpoint{2.171770in}{1.706758in}}%
\pgfpathlineto{\pgfqpoint{2.295375in}{1.618048in}}%
\pgfpathlineto{\pgfqpoint{2.394258in}{1.549488in}}%
\pgfpathlineto{\pgfqpoint{2.493142in}{1.483290in}}%
\pgfpathlineto{\pgfqpoint{2.592026in}{1.419658in}}%
\pgfpathlineto{\pgfqpoint{2.690909in}{1.358796in}}%
\pgfpathlineto{\pgfqpoint{2.789793in}{1.300908in}}%
\pgfpathlineto{\pgfqpoint{2.888676in}{1.246195in}}%
\pgfpathlineto{\pgfqpoint{2.962839in}{1.207366in}}%
\pgfpathlineto{\pgfqpoint{3.037002in}{1.170521in}}%
\pgfpathlineto{\pgfqpoint{3.111164in}{1.135746in}}%
\pgfpathlineto{\pgfqpoint{3.185327in}{1.103125in}}%
\pgfpathlineto{\pgfqpoint{3.259490in}{1.072741in}}%
\pgfpathlineto{\pgfqpoint{3.333652in}{1.044679in}}%
\pgfpathlineto{\pgfqpoint{3.407815in}{1.019024in}}%
\pgfpathlineto{\pgfqpoint{3.481978in}{0.995857in}}%
\pgfpathlineto{\pgfqpoint{3.556140in}{0.975264in}}%
\pgfpathlineto{\pgfqpoint{3.630303in}{0.957326in}}%
\pgfpathlineto{\pgfqpoint{3.704466in}{0.942128in}}%
\pgfpathlineto{\pgfqpoint{3.778628in}{0.929751in}}%
\pgfpathlineto{\pgfqpoint{3.852791in}{0.920279in}}%
\pgfpathlineto{\pgfqpoint{3.902233in}{0.915618in}}%
\pgfpathlineto{\pgfqpoint{3.951675in}{0.912309in}}%
\pgfpathlineto{\pgfqpoint{4.025837in}{0.918788in}}%
\pgfpathlineto{\pgfqpoint{4.100000in}{0.928254in}}%
\pgfpathlineto{\pgfqpoint{4.174163in}{0.940597in}}%
\pgfpathlineto{\pgfqpoint{4.248325in}{0.955706in}}%
\pgfpathlineto{\pgfqpoint{4.322488in}{0.973472in}}%
\pgfpathlineto{\pgfqpoint{4.396651in}{0.993785in}}%
\pgfpathlineto{\pgfqpoint{4.470813in}{1.016533in}}%
\pgfpathlineto{\pgfqpoint{4.544976in}{1.041605in}}%
\pgfpathlineto{\pgfqpoint{4.619139in}{1.068892in}}%
\pgfpathlineto{\pgfqpoint{4.693301in}{1.098281in}}%
\pgfpathlineto{\pgfqpoint{4.767464in}{1.129661in}}%
\pgfpathlineto{\pgfqpoint{4.866348in}{1.174405in}}%
\pgfpathlineto{\pgfqpoint{4.965231in}{1.222225in}}%
\pgfpathlineto{\pgfqpoint{5.064115in}{1.272855in}}%
\pgfpathlineto{\pgfqpoint{5.162998in}{1.326029in}}%
\pgfpathlineto{\pgfqpoint{5.261882in}{1.381478in}}%
\pgfpathlineto{\pgfqpoint{5.385486in}{1.453583in}}%
\pgfpathlineto{\pgfqpoint{5.509091in}{1.528301in}}%
\pgfpathlineto{\pgfqpoint{5.607974in}{1.588348in}}%
\pgfpathlineto{\pgfqpoint{5.756300in}{1.674857in}}%
\pgfpathlineto{\pgfqpoint{5.929346in}{1.778594in}}%
\pgfpathlineto{\pgfqpoint{6.102392in}{1.884950in}}%
\pgfpathlineto{\pgfqpoint{6.300159in}{2.009150in}}%
\pgfpathlineto{\pgfqpoint{6.522648in}{2.151464in}}%
\pgfpathlineto{\pgfqpoint{6.844019in}{2.359995in}}%
\pgfpathlineto{\pgfqpoint{6.918182in}{2.408348in}}%
\pgfpathlineto{\pgfqpoint{6.918182in}{2.408348in}}%
\pgfusepath{stroke}%
\end{pgfscope}%
\begin{pgfscope}%
\pgfsetroundcap%
\pgfsetroundjoin%
\pgfsetlinewidth{1.003750pt}%
\definecolor{currentstroke}{rgb}{0.000000,0.000000,0.000000}%
\pgfsetstrokecolor{currentstroke}%
\pgfsetdash{}{0pt}%
\pgfpathmoveto{\pgfqpoint{4.566967in}{0.435000in}}%
\pgfpathquadraticcurveto{\pgfqpoint{4.347378in}{0.435000in}}{\pgfqpoint{4.127789in}{0.435000in}}%
\pgfusepath{stroke}%
\end{pgfscope}%
\begin{pgfscope}%
\pgfsetbuttcap%
\pgfsetmiterjoin%
\definecolor{currentfill}{rgb}{0.800000,0.800000,0.800000}%
\pgfsetfillcolor{currentfill}%
\pgfsetlinewidth{1.003750pt}%
\definecolor{currentstroke}{rgb}{0.000000,0.000000,0.000000}%
\pgfsetstrokecolor{currentstroke}%
\pgfsetdash{}{0pt}%
\pgfpathmoveto{\pgfqpoint{4.624691in}{0.338549in}}%
\pgfpathcurveto{\pgfqpoint{4.659413in}{0.303827in}}{\pgfqpoint{5.488889in}{0.303827in}}{\pgfqpoint{5.523611in}{0.338549in}}%
\pgfpathcurveto{\pgfqpoint{5.558333in}{0.373272in}}{\pgfqpoint{5.558333in}{0.496728in}}{\pgfqpoint{5.523611in}{0.531451in}}%
\pgfpathcurveto{\pgfqpoint{5.488889in}{0.566173in}}{\pgfqpoint{4.659413in}{0.566173in}}{\pgfqpoint{4.624691in}{0.531451in}}%
\pgfpathcurveto{\pgfqpoint{4.589969in}{0.496728in}}{\pgfqpoint{4.589969in}{0.373272in}}{\pgfqpoint{4.624691in}{0.338549in}}%
\pgfpathclose%
\pgfusepath{stroke,fill}%
\end{pgfscope}%
\begin{pgfscope}%
\definecolor{textcolor}{rgb}{0.000000,0.000000,0.000000}%
\pgfsetstrokecolor{textcolor}%
\pgfsetfillcolor{textcolor}%
\pgftext[x=5.488889in,y=0.435000in,right,]{\color{textcolor}\rmfamily\fontsize{10.000000}{12.000000}\selectfont \(\displaystyle \Delta =\) -0.3 inch}%
\end{pgfscope}%
\begin{pgfscope}%
\pgfsetbuttcap%
\pgfsetmiterjoin%
\definecolor{currentfill}{rgb}{0.800000,0.800000,0.800000}%
\pgfsetfillcolor{currentfill}%
\pgfsetlinewidth{1.003750pt}%
\definecolor{currentstroke}{rgb}{0.000000,0.000000,0.000000}%
\pgfsetstrokecolor{currentstroke}%
\pgfsetdash{}{0pt}%
\pgfpathmoveto{\pgfqpoint{0.965278in}{0.358599in}}%
\pgfpathcurveto{\pgfqpoint{1.000000in}{0.323877in}}{\pgfqpoint{3.239588in}{0.323877in}}{\pgfqpoint{3.274311in}{0.358599in}}%
\pgfpathcurveto{\pgfqpoint{3.309033in}{0.393321in}}{\pgfqpoint{3.309033in}{0.668784in}}{\pgfqpoint{3.274311in}{0.703506in}}%
\pgfpathcurveto{\pgfqpoint{3.239588in}{0.738228in}}{\pgfqpoint{1.000000in}{0.738228in}}{\pgfqpoint{0.965278in}{0.703506in}}%
\pgfpathcurveto{\pgfqpoint{0.930556in}{0.668784in}}{\pgfqpoint{0.930556in}{0.393321in}}{\pgfqpoint{0.965278in}{0.358599in}}%
\pgfpathclose%
\pgfusepath{stroke,fill}%
\end{pgfscope}%
\begin{pgfscope}%
\definecolor{textcolor}{rgb}{0.000000,0.000000,0.000000}%
\pgfsetstrokecolor{textcolor}%
\pgfsetfillcolor{textcolor}%
\pgftext[x=1.000000in, y=0.580049in, left, base]{\color{textcolor}\rmfamily\fontsize{10.000000}{12.000000}\selectfont Max combo: 1.1D + 1.31E + 0.75L0}%
\end{pgfscope}%
\begin{pgfscope}%
\definecolor{textcolor}{rgb}{0.000000,0.000000,0.000000}%
\pgfsetstrokecolor{textcolor}%
\pgfsetfillcolor{textcolor}%
\pgftext[x=1.000000in, y=0.428043in, left, base]{\color{textcolor}\rmfamily\fontsize{10.000000}{12.000000}\selectfont ASCE7-16 Sec. 2.4.5 (LC 8)}%
\end{pgfscope}%
\begin{pgfscope}%
\pgfsetroundcap%
\pgfsetroundjoin%
\pgfsetlinewidth{1.003750pt}%
\definecolor{currentstroke}{rgb}{0.000000,0.000000,0.000000}%
\pgfsetstrokecolor{currentstroke}%
\pgfsetdash{}{0pt}%
\pgfpathmoveto{\pgfqpoint{4.443362in}{1.412674in}}%
\pgfpathquadraticcurveto{\pgfqpoint{4.223773in}{1.412674in}}{\pgfqpoint{4.004184in}{1.412674in}}%
\pgfusepath{stroke}%
\end{pgfscope}%
\begin{pgfscope}%
\pgfsetbuttcap%
\pgfsetmiterjoin%
\definecolor{currentfill}{rgb}{0.800000,0.800000,0.800000}%
\pgfsetfillcolor{currentfill}%
\pgfsetlinewidth{1.003750pt}%
\definecolor{currentstroke}{rgb}{0.000000,0.000000,0.000000}%
\pgfsetstrokecolor{currentstroke}%
\pgfsetdash{}{0pt}%
\pgfpathmoveto{\pgfqpoint{4.501086in}{1.316224in}}%
\pgfpathcurveto{\pgfqpoint{4.535809in}{1.281502in}}{\pgfqpoint{5.365284in}{1.281502in}}{\pgfqpoint{5.400007in}{1.316224in}}%
\pgfpathcurveto{\pgfqpoint{5.434729in}{1.350946in}}{\pgfqpoint{5.434729in}{1.474403in}}{\pgfqpoint{5.400007in}{1.509125in}}%
\pgfpathcurveto{\pgfqpoint{5.365284in}{1.543847in}}{\pgfqpoint{4.535809in}{1.543847in}}{\pgfqpoint{4.501086in}{1.509125in}}%
\pgfpathcurveto{\pgfqpoint{4.466364in}{1.474403in}}{\pgfqpoint{4.466364in}{1.350946in}}{\pgfqpoint{4.501086in}{1.316224in}}%
\pgfpathclose%
\pgfusepath{stroke,fill}%
\end{pgfscope}%
\begin{pgfscope}%
\definecolor{textcolor}{rgb}{0.000000,0.000000,0.000000}%
\pgfsetstrokecolor{textcolor}%
\pgfsetfillcolor{textcolor}%
\pgftext[x=5.365284in,y=1.412674in,right,]{\color{textcolor}\rmfamily\fontsize{10.000000}{12.000000}\selectfont \(\displaystyle \Delta =\) -0.1 inch}%
\end{pgfscope}%
\begin{pgfscope}%
\pgfsetbuttcap%
\pgfsetmiterjoin%
\definecolor{currentfill}{rgb}{0.800000,0.800000,0.800000}%
\pgfsetfillcolor{currentfill}%
\pgfsetlinewidth{1.003750pt}%
\definecolor{currentstroke}{rgb}{0.000000,0.000000,0.000000}%
\pgfsetstrokecolor{currentstroke}%
\pgfsetdash{}{0pt}%
\pgfpathmoveto{\pgfqpoint{0.965278in}{0.375574in}}%
\pgfpathcurveto{\pgfqpoint{1.000000in}{0.340852in}}{\pgfqpoint{2.390820in}{0.340852in}}{\pgfqpoint{2.425542in}{0.375574in}}%
\pgfpathcurveto{\pgfqpoint{2.460265in}{0.410297in}}{\pgfqpoint{2.460265in}{0.676500in}}{\pgfqpoint{2.425542in}{0.711222in}}%
\pgfpathcurveto{\pgfqpoint{2.390820in}{0.745944in}}{\pgfqpoint{1.000000in}{0.745944in}}{\pgfqpoint{0.965278in}{0.711222in}}%
\pgfpathcurveto{\pgfqpoint{0.930556in}{0.676500in}}{\pgfqpoint{0.930556in}{0.410297in}}{\pgfqpoint{0.965278in}{0.375574in}}%
\pgfpathclose%
\pgfusepath{stroke,fill}%
\end{pgfscope}%
\begin{pgfscope}%
\definecolor{textcolor}{rgb}{0.000000,0.000000,0.000000}%
\pgfsetstrokecolor{textcolor}%
\pgfsetfillcolor{textcolor}%
\pgftext[x=1.000000in, y=0.580049in, left, base]{\color{textcolor}\rmfamily\fontsize{10.000000}{12.000000}\selectfont Max combo: 1.0L0}%
\end{pgfscope}%
\begin{pgfscope}%
\definecolor{textcolor}{rgb}{0.000000,0.000000,0.000000}%
\pgfsetstrokecolor{textcolor}%
\pgfsetfillcolor{textcolor}%
\pgftext[x=1.000000in, y=0.437303in, left, base]{\color{textcolor}\rmfamily\fontsize{10.000000}{12.000000}\selectfont L only deflection check}%
\end{pgfscope}%
\end{pgfpicture}%
\makeatother%
\endgroup%

\end{center}
\caption{Deflection Envelope}
\end{figure}
Tl Deflection Check: 
$\Delta_{max} = -0.26 {\color{darkBlue}{\mathbf{ \; in}}} = \cfrac{L}{868} < \cfrac{L}{1.0}  \;  \mathbf{(OK)}$\\
\bigbreak
Ll Deflection Check: 
$\Delta_{max} = -0.13 {\color{darkBlue}{\mathbf{ \; in}}} = \cfrac{L}{1722} < \cfrac{L}{1.0}  \;  \mathbf{(OK)}$\\
\bigbreak
\vspace{-30pt}
%	---------------------------------- REACTIONS ---------------------------------
\section{Reactions}
The following is a summary of service-level reactions at each support:
\begin{table}[ht]
\caption{Reactions at Supports}
\centering
\begin{tabular}{l l l l l l }
\hline
Loc. & Type & D & E & L0 & Lr0\\
\hline
0 {\color{darkBlue}{\textbf{ft}}} & Shear & 10.8 {\color{darkBlue}{\textbf{kip}}} & 0.8 {\color{darkBlue}{\textbf{kip}}} & 11.6 {\color{darkBlue}{\textbf{kip}}} & 0.9 {\color{darkBlue}{\textbf{kip}}}\\ 
19 {\color{darkBlue}{\textbf{ft}}} & Shear & 9.8 {\color{darkBlue}{\textbf{kip}}} & 8.1 {\color{darkBlue}{\textbf{kip}}} & 10.4 {\color{darkBlue}{\textbf{kip}}} & 0.8 {\color{darkBlue}{\textbf{kip}}}\\ 
\hline
\end{tabular}
\end{table}
\end{document}